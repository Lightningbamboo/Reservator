\PassOptionsToPackage{unicode=true}{hyperref} % options for packages loaded elsewhere
\PassOptionsToPackage{hyphens}{url}
%
\documentclass[]{article}
\usepackage{lmodern}
\usepackage{amssymb,amsmath}
\usepackage{ifxetex,ifluatex}
\usepackage{fixltx2e} % provides \textsubscript
\ifnum 0\ifxetex 1\fi\ifluatex 1\fi=0 % if pdftex
  \usepackage[T1]{fontenc}
  \usepackage[utf8]{inputenc}
  \usepackage{textcomp} % provides euro and other symbols
\else % if luatex or xelatex
  \usepackage{unicode-math}
  \defaultfontfeatures{Ligatures=TeX,Scale=MatchLowercase}
\fi
% use upquote if available, for straight quotes in verbatim environments
\IfFileExists{upquote.sty}{\usepackage{upquote}}{}
% use microtype if available
\IfFileExists{microtype.sty}{%
\usepackage[]{microtype}
\UseMicrotypeSet[protrusion]{basicmath} % disable protrusion for tt fonts
}{}
\IfFileExists{parskip.sty}{%
\usepackage{parskip}
}{% else
\setlength{\parindent}{0pt}
\setlength{\parskip}{6pt plus 2pt minus 1pt}
}
\usepackage{hyperref}
\hypersetup{
            pdfborder={0 0 0},
            breaklinks=true}
\urlstyle{same}  % don't use monospace font for urls
\setlength{\emergencystretch}{3em}  % prevent overfull lines
\providecommand{\tightlist}{%
  \setlength{\itemsep}{0pt}\setlength{\parskip}{0pt}}
\setcounter{secnumdepth}{0}
% Redefines (sub)paragraphs to behave more like sections
\ifx\paragraph\undefined\else
\let\oldparagraph\paragraph
\renewcommand{\paragraph}[1]{\oldparagraph{#1}\mbox{}}
\fi
\ifx\subparagraph\undefined\else
\let\oldsubparagraph\subparagraph
\renewcommand{\subparagraph}[1]{\oldsubparagraph{#1}\mbox{}}
\fi

% set default figure placement to htbp
\makeatletter
\def\fps@figure{htbp}
\makeatother


\date{}

\begin{document}

\hypertarget{header-n0}{%
\section{列子}\label{header-n0}}

\begin{center}\rule{0.5\linewidth}{\linethickness}\end{center}

\tableofcontents

\begin{center}\rule{0.5\linewidth}{\linethickness}\end{center}

\hypertarget{header-n6}{%
\subsection{天瑞}\label{header-n6}}

子列子居郑圃,四十年人无识者。国君卿大夫示之,犹众庶也。国不足,将嫁于卫。弟子曰:``先生往无反期,弟子敢有所谒;先生将何以教?先生不闻壶丘子林之言乎?''子列子笑曰:``壶子何言哉?虽然,夫子尝语伯昏瞀人,吾侧闻之,试以告女。其言曰:有生不生,有化不化。不生者能生生,不化者能化化。生者不能不生,化者不能不化,故常生常化。常生常化者,无时不生,无时不化。阴阳尔,四时尔,不生者疑独,不化者往复。往复其际不可终,疑独其道不可穷。《黄帝书》曰:`谷神不死,是谓玄牝。玄牝之门,是谓天地之根。绵绵若存,用之不勤。'故竹物者不生,化物者不化。自生自化,自形自色,自智自力,自消自息。谓之生化、形色、智力、消息者,非也。''

子列子曰:``昔者圣人因阴阳以统天地。夫有形者生于无形,则天地安从生?故曰:有太易,有太初,有太始,有太素。太易者,未见气也:太初者,气之始也;太始者,形之始也;太素者,质之始也。气形质具而未相离,故曰浑沦。浑沦者,言万物相浑沦而未相离也。视之不见,听之不闻,循之不得,故曰易也。易无形埒,易变而为一,一变而为七,七变而为九。九变者,穷也,乃复变而为一。一者,形变之始也。清轻者上为天,浊重者下为地,冲和气者为人;故天地含精,万物化生。''

子列子曰:``天地无全功,圣人无全能,万物无全用。故天职生覆,地职形载,圣职教化,物职所宜。然则天有所短,地有所长,圣有所否,物有所通。何则?生覆者不能形载,形载者。不能教化,教化者不能违所宜,宜定者不出所位。故天地之道,非阴则阳;圣人之教,非仁则义;万物之宜,非柔则刚:此皆随所宜而不能出所位者也。故有生者,有生生者;有形者,有形形者;有声者,有声声者;有色者,有色色者;有味者,有味味者。生之所生者死矣,而生生者未尝终;形之所形者实矣,而形形者未尝有;声之所声者闻矣,而声声者未尝发;色之所色者彰矣,而色色者未尝显;味之所味者尝矣,而味味者未尝呈:皆无为之职也。能阴能阳,能柔能刚,能短能长,能圆能方,能生能死,能暑能凉,能浮能沉,能宫能商,能出能没,能玄能黄,能甘能苦,能膻能香。无知也,无能也;而无不知也,而无不能也。''

子列子适卫,食于道,从者见百岁髑髅,攓蓬而指,顾谓弟子百丰曰:``唯予与彼知而未尝生未尝死也。此过养乎?此过欢乎?种有几:若\textless{}圭黾\textgreater{}为鹑,得水为畿,得水土之际,则为\textless{}圭黾\textgreater{}蠙之衣。生于陵屯,则为陵舄。陵舄得郁栖,则为乌足。乌足之根为蛴螬,其叶为蝴蝶。蝴蝶胥也,化而为虫,生灶下,其状若脱,其名曰\textless{}鸟句\textgreater{}掇,\textless{}鸟句\textgreater{}掇千日化而为鸟,其名曰乾余骨。乾余骨之沫为斯弥。斯弥为食醯颐辂。食醯颐辂生乎食醯黄軦,食醯黄軦生乎九猷。九猷生乎瞀芮,瞀芮生乎腐蠸,羊肝化为地皋,马血之为转邻也,人血之为野火也。鹞之为鹯,鹯之为布谷,布谷久复为鹞也。燕之为蛤也,田鼠之为鹑也,朽瓜之为鱼也,老韭之为苋也。老羭之为猨也,鱼卵之为虫。亶爰之兽,自孕而生,曰类。河泽之鸟视而生曰。纯雌其名大要,纯雄其名稚蜂。思士不妻而感,思女不夫而孕。后稷生乎巨迹,伊尹生乎空桑。厥昭生乎湿,醯鸡生乎酒。羊奚比乎不荀,久竹生青宁,青宁生程,程生马,马生人。人久入于机。万物皆出于机,皆入于机。''

《黄帝书》曰:``形动不生形而生影,声动不生声而生响,无动不生无而生有。''形,必终者也;天地终乎?与我偕终。终进乎?不知也。道终乎本无始,进乎本不久。有生则复于不生,有形则复于无形。不生者,非本不生者;无形者,非本无形者也。生者,理之必终者也。终者不得不终,亦如生者之不得不生。而欲恒其生,画其终,惑于数也。精神者,天之分;骨骸者,地之分。属天清而散,属地浊而聚。精神离形,各归其真,故谓之鬼。鬼,归也,归其真宅。黄帝曰:``精神入其门,骨骸反其根,我尚我存?''

人自生至终,大化有四:婴孩也,少壮也,老耄也,死亡也。其在婴孩,气专志一,和之至也;物不伤焉,德莫加焉。其在少壮,则血气飘溢,欲虑充起,物所攻焉,德故衰焉。其在老耄,则欲虑柔焉,体将休焉,物莫先焉;虽未及婴孩之全,方于少壮,间矣。其在死亡也,则之于息焉,反其极矣。

孔子游于太山,见荣启期行乎郕之野,鹿裘带索,鼓琴而歌。孔子问曰:``先生所以乐,何也?''对曰:``吾乐甚多。天生万物,唯人为贵。而吾得为人,是一乐也。男女之别,男尊女卑,故以男为贵,吾既得为男矣,是二乐也。人生有不见日月,不免襁褓者,吾既已行年九十矣,是三乐也。贫者士之常也,死者人之终也,处常得终,当何忧哉?''孔子曰:``善乎?能自宽者也。''

林类年且百岁,底春被裘,拾遗穗于故畦,并歌并进。孔子适卫,望之于野。顾谓弟子曰:``彼叟可与言者,试往讯之!''子贡请行。逆之垅端,面之而叹曰:``先生曾不悔乎,而行歌拾穗?''林类行不留。歌不辍。子贡叩之,不已,乃仰而应曰:``吾何悔邪?''子贡曰:``先生少不勤行,长不竞时,老无妻子,死期将至,亦有何乐而拾穗行歌乎?''林类笑曰:``吾之所以为乐,人皆有之,而反以为忧。少不勤行,长不竞时,故能寿若此。老无妻子,死期将至,故能乐若此。''子贡曰:``寿者人之情,死者人之恶。子以死为乐,何也?''林类曰:``死之与生,一往一反。故死于是者,安知不生于彼?故吾知其不相若矣,吾又安知营营而求生非惑乎?亦又安知吾今之死不愈昔之生乎?''子贡闻之,不喻其意,还以告夫子。夫子曰:``吾知其可与言,果然;然彼得之而不尽者也。''

子贡倦于学,告仲尼曰:``愿有所息。''仲尼曰:``生无所息。''子贡曰:``然则赐息无所乎?''仲尼曰:``有焉耳,望其圹,皋如也,宰如也,坟如也,鬲如也,则知所息矣。''子贡曰:``大哉死乎!君子息焉,小人伏焉。仲尼曰:``赐!汝知之矣。人胥知生之乐,未知生之苦;知老之惫,未知老之佚;知死之恶,未知死之息也。晏子曰:`善哉,古之有死也!仁者息焉,不仁者伏焉。'死也者,德之徼也。古者谓死人为归人。夫言死人为归人,则生人为行人矣。行而不知归,失家者了。一人失家,一世非之;天下失家,莫知非焉。有人去乡土、离六亲、废家业、游于四方而不归者,何人哉?世必谓之为狂荡之人矣。又有人钟贤世,矜巧能,修名誉,夸张于世而不知已者,亦何人哉?世必以为智谋之士。此二者,胥失者也。而世与一不与一,唯圣人知所与,知所去。''

或谓子列子曰:``子奚贵虚?''列子曰:``虚者无贵也。''子列子曰:``非其名也,莫如静,莫如虚。静也虚也,得其居矣;取也与也,失其民矣。事之破为而后有舞仁义者,弗能复也。''

粥熊曰:``运转亡已,天地密移,畴觉之哉?故物损于彼者盈于此,成于此者亏于彼。损盈成亏,随世随死。往来相接,间不可省,畴觉之哉?凡一气不顿进,一形不顿亏;亦不觉其成,亦不觉其亏。亦如人自世至老,貌色智态,亡日不异;皮肤爪发,随世随落,非婴孩时有停而不易也。间不可觉,俟至后知。''

杞国有人忧天地崩坠,身亡所寄,废寝食者;又有忧彼之所忧者,因往晓之,曰:``天,积气耳,亡处亡气。若屈伸呼吸,终日在天中行止,奈何忧崩坠乎?''其人曰:``天果积气,日月星宿,不当坠耶?''晓之者曰:``日月星宿,亦积气中之有光耀者;只使坠,亦不能有气中伤。''其人曰:``奈地坏何?''晓者曰:``地积块耳,充塞四虚,亡处亡块。若躇步跐蹈,终日在地上行止,奈何忧其坏?''其人舍然大喜,晓之者亦舍然大喜。长庐子闻而笑曰:``虹蜺也,云雾也,风雨也,四时也,此积气之成乎天者也。山岳也,河海也,金石也,火木也,此积形之成乎地者也。知积气也,知积块也,奚谓不坏?夫天地,空中之一细物,有中之最巨者。难终难穷,此固然矣;难测难识,此固然矣。忧其坏者,诚为大远;言其不坏者,亦为未是。天地不得不坏,则会归于坏。遇其坏时,奚为不忧哉?''子列子闻而笑曰:``言天地坏者亦谬,言天地不坏者亦谬。坏与不坏,吾所不能知也。虽然,彼一也,此一也。故生不知死,死不知生;来不知去,去不知来。坏与不坏,吾何容心哉?''

舜问乎烝曰:``道可得而有乎?''曰:``汝身非汝有也,汝何得有夫道?''舜曰:``吾身非吾有,孰有之哉?''曰:``是天地之委形也。生非汝有,是天地之委和也。性命非汝有,是天地之委顺也。孙子非汝有,是天地之委蜕也。故行不知所往,处不知所持,食不知所以。天地强阳,气也;又胡可得而有邪?''

齐之国氏大富,宋之向氏大贫;自宋之齐,请其术。国氏告之曰:``吾善为盗。始吾为盗也,一年而给,二年而足,三年大穰。自此以往,施及州闾。''向氏大喜,喻其为盗之言,而不喻其为盗之道,遂逾垣凿室,手目所及,亡不探也。未及时,以赃获罪,没其先居之财。向氏以国氏之谬己也,往而怨之。国氏曰:``若为盗若何?''向氏言其状。国氏曰:``嘻!若失为盗之道至此乎?今将告若矣。吾闻天有时,地有利。吾盗天地之时利,云雨之滂润,山泽之产育,以生吾禾,殖吾稼,筑吾垣,建吾舍,陆盗禽兽,水盗鱼鳖,亡非盗也。夫禾稼、土木、禽兽、鱼鳖,皆天之所生,岂吾之所有?然吾盗天而亡殃。夫金玉珍宝,谷帛财货,人之所聚,岂天之所与?若盗之而获罪,孰怨哉?''向氏大惑,以为国氏之重罔己也,过东郭先生问焉。东郭先生曰:``若一身庸非盗乎?盗阴阳之和以成若生,载若形;况外物而非盗哉?诚然,天地万物不相离也;仞而有之,皆惑也。国氏之盗,公道也,故亡殃;若之盗,私心也,故得罪。有公私者,亦盗也;亡公私者,亦盗也。公公私私,天地之德。知天地之德者,孰为盗邪?孰为不盗邪?''

\hypertarget{header-n23}{%
\subsection{黄帝}\label{header-n23}}

黄帝即位十有五年,喜天五戴己,养正命,娱耳目,供鼻口,焦然肌色\textless{}皮干\textgreater{}黣,昏然五情爽惑。又十有五年,忧天下之不治,竭聪明,进智力,营百姓,焦然肌色皯黣,昏然五情爽惑。黄帝乃喟然赞曰:``朕之过淫矣。养一己其患如此,治万物其患如此。''于是放万机,舍宫寝,去直待,彻钟县。减厨膳,退而间居大庭之馆,斋心服形,三月不亲政事。昼寝而梦,游于华胥氏之国。华胥氏之国在弇州之西,台州之北,不知斯齐国几千万里;盖非舟车足力之所及,神游而已。其国无帅长,自然而已。其民无嗜欲,自然而已。不知乐生,不知恶死,故无夭殇;不知亲己,不知疏物,故无爱憎;不知背逆,不知向顺,故无利害:者无的爱惜,都无所畏忌。入水不溺,入火不热。斫挞无伤痛,指擿无痒。乘空如履实,寝虚若处床。云雾不硋其视,雷霆不乱其听,美恶不滑其心,山谷不踬其步,神行而已。黄帝既寤,怡然自得,召天老、力牧、太山稽,告之,曰:``朕闲居三月,斋心服形,思有以养身治物之道,弗获其术。疲而睡,所梦若此。今知至道不可以情求矣。朕知之矣!朕得之矣!而不能以告若矣。''又二十有八年,天下大治,几若华胥氏之国,而帝登假,百姓号之,二百余年不辍。

列姑射山在海河洲中,山上有神人焉,吸风饮露,不食五谷;心如渊泉,形如处女;不偎不爱,仙圣为之臣;不畏不怒,愿悫为之使;不施不惠,而物自足;不聚不敛,而已无愆。阴阳常调,日月常明,四时常若,风雨常均,字育常时,年谷常丰;而土无札伤,人无夭恶,物无疵厉,鬼无灵响焉。

列子师老商氏,友伯高子,进二子之道,乘风而归。尹生闻之,从列子居,数月不省舍。因间请蕲其术者,十反而十不告。尹生怼而请辞,列子又不命。尹生退,数月,意不已,又往从之。列子曰:``汝何去来之频?''尹生曰:``曩章戴有请于子,子不我告,固有憾于子。今复脱然,是以又来。''列子曰:``嚷吾以汝为达,今汝之鄙至此乎。姬!将告汝所学于夫子者矣。自吾之事夫子友若人也,三年之后,心不敢念是非,口不敢言利害,始得夫子一眄而已。五年之后,心庚念是非,口庚言利害,夫子始一解颜而笑。七年之后,从心之所念,念庚无是非;从口之所言,庚无利害,夫子始一引吾并席而坐。九年之后,横心之所念,横口之所言,亦不知我之是非利害欤,亦不知彼之是非利害欤;亦不知夫子之为我师,若人之为我友:内外进矣。而后眼如耳,耳如鼻,鼻如口,无不同也。心凝形释骨肉都融;不觉形之所倚,足之所履,随风东西,犹木叶干壳。竟不知风乘我邪?我乘风乎?今女居先生之门,曾未浃时,而怼憾者再三。女之片体将气所不受,汝之一节将地所不载。履虚乘风,其可几乎?''尹生甚怍,屏息良久,不敢复言。

列子问关尹曰:``至人潜行不空,蹈火不热,行乎万物之上而不栗。请问何以至于此?''关尹曰:``是纯气之守也,非智巧果敢之列。姬!鱼语女。凡有貌像声色者,皆物也。物与物何以相远也?夫奚足以至乎先?是色而已。则物之造乎不形,而止乎无所化。夫得是而穷之者,得而正焉?彼将处乎不深之度,而藏乎无端之纪,游乎万物之所终始。壹其性,养其气,含其德,以通乎物之所造。夫若是者,其天守全,其神无郤,物奚自入焉?夫醉者之坠于车也,虽疾不死。骨节与人同,而犯害与人异,其神全也。乘亦弗知也,坠亦弗知也。死生惊惧不入乎其胸,是故忤物而不慑。彼得全于酒而犹若是,而况得全于天乎?圣人藏于天,故物莫之能伤也。''

列御寇为伯昏无人射,引之盈贯,措杯水其肘上,发之,镝矢复沓,方矢复寓。当是时也,犹象人也。伯昏无人曰:``是射之射,非不射之射也。当与汝登高山,履危石,临百仞之渊,背逡巡,足二分垂在外。揖御寇而进之。御寇伏地,汗流至踵。伯昏无人曰:``夫至人者,上窥青天,下潜黄泉,挥斥八极。神气不变。今汝怵然有恂目之志,尔于中也殆矣夫!''

范氏有子曰子华,善养私名,举国服之;有宠于晋君,不仕而居三卿之右。目所偏视,晋国爵之;口所偏肥,晋国黜之。游其庭者侔于朝。子华使其侠客以智鄙相攻,疆弱相凌。虽伤破于前,不用介意。终日夜以此为戏乐,国殆成俗。禾生、子伯、范氏之上客。出行,经坰外,宿于田更商丘开之舍。中夜,禾生、子伯二人相与言子华之名势,能使存者亡,亡者存;富者贫,贫者富。商丘开先窘于饥寒,潜于牖北听之。因假粮荷畚之子华之门。子华之门徒皆世族也,缟衣乘轩,缓步阔视。顾见商丘开年老力弱,面目黎黑,衣冠不检,莫不眲之。既而狎侮欺诒,扌党挨扌冘,亡所不为。商丘开常无愠容,而诸客之技单,惫于戏笑。遂与商丘开俱乘高台,于众中漫言曰:``有能自投下者赏百金。''众皆竞应。商丘开以为信然,遂先投下,形若飞鸟,扬于地,\textless{}骨几\textgreater{}骨于为。范氏之党以为偶然,未讵怪也。因复指河曲之淫隈曰:``彼中有宝珠,泳可得也。''商丘开复从而泳之,既出,果得珠焉。众昉同疑。子华昉令豫肉食衣帛之次。俄而范氏之藏大火。子华曰:``若能入火取绵者,从所得多少赏若。''商丘开往无难色,入火往还,埃不漫,身不焦。范氏之党以为有道,乃共谢之曰:``吾不知子之有道而诞子,吾不知子之神人而辱子。子其愚我也,子其聋我也,子其盲我也,敢问其道。''商丘开曰:`吾亡道。虽吾之心,亦不知所以。虽然,有一于此,试与子言之。嚷子二客之宿吾舍也,闻誉范氏之势,能使存者亡,亡者存;富者贫,贫者富。吾诚之无二心,故不远而来。及来,以子党之言皆实也,唯恐诚之之不至,行之之不及,不知形体之所措,利害之所存也。心一而已。物亡迕者,如斯而已。今昉知子党之诞我,我内藏猜虑,外矜观听,追幸昔日之不焦溺也,怛然内热。惕然震悸矣。水火岂复可近哉?''自此之后,范氏门徒路遇乞儿马医,弗敢辱也,必下车而揖之,宰我闻之,以告仲尼。仲尼曰:'汝弗知乎?夫至信之人,可以感物也。动天地,感鬼神,横六合,而无逆者,岂但履危险,入水火而已哉?商丘开信伪物犹不逆,况彼我皆诚哉?小子识之!''

周宣王文牧正有役人梁鸯者,能养野禽兽,委食于园庭之内,虽虎狼雕鹗之类,无不柔驯者。雄雌在前,孳尾成群,异类杂居,不相搏噬也。王虑其术终于其身,令毛丘园传之。梁鸯曰:``鸯,贱役也,何术以告尔?惧王之谓隐于尔也,且一言我养虎之法。凡顺之则喜,逆之则怒,此有血气者之性也。然喜怒岂妄发哉?皆逆之所犯也。夫食虎者,不敢以生物与之,为其杀之之怒也;不敢以全物与之,为其碎之之怒也。时其饥饱,达其怒心。虎之与人异类,而媚养己者,顺也;故其杀之,逆也。然则吾岂敢逆之使怒哉?亦不顺之使喜也。夫喜之复也必怒,怒之复也常喜,皆不中也。今吾心无逆顺者也,则鸟兽之视吾,犹其侪也。故游吾园者,不思高林旷泽;寝吾庭者,不愿深山幽谷,理使然也。''

颜回问乎仲尼曰:``吾尝济乎觞深之渊矣,津人操舟若神。吾问焉,曰:`操舟可学邪?'曰:`可;能游者可教也,善游者数能。乃若夫没人,则未尝见舟而谡操之者也。'吾问焉,而不告。敢问何谓也?''仲尼曰:`讠医!吾与若玩其文也久矣,而未达其实,而固且道与。能游者可救也,轻水也;善游者文数能也,忘水也。乃若夫没人之未尝见舟也而谡操之也,彼视渊若陵,视舟之覆犹其车郤也。覆郤万物方陈乎前而不得入其舍。恶往而不暇?以瓦抠者巧,以钩抠者惮,以黄金钩抠者惮。巧一也,而有所矜,则重外也。凡重外者拙内。''

孔子观于吕梁,悬水三十仞,流沫三十里,鼋鼍鱼鳖之所不能游也。见一丈夫游之,以为有苦而欲死者也,使弟子并流而承之。数百步而出,被发行歌,而游于棠行。孔子从而问之,曰:``吕梁悬水三十仞,流沫三十里,鼋鼍鱼鳖所不能游,向吾见子道之,以为有苦而欲死者,使弟子并流将承子。子出而被发行歌,吾以子为鬼也。察子,则人也。请问蹈水有道乎?''曰:``亡,吾无道。吾始乎故,长乎性,成乎命,与齐俱入,与汨偕出。从水之道而不为私焉,此吾所以道之也。''孔子曰:``何谓始乎故,长乎性,成乎命也?''曰:``吾生于陵安于陵,故也;长于水而安于水,性也;不知吾所以然而然,命也。''

仲尼适楚,出于林中,见佝偻者承蜩,犹掇之也。仲尼曰:``子巧乎!有道邪?''曰:``我有道也。五六月,累垸二而不坠,则失者锱铢;累三而不坠,则失者十一;累五而不坠,犹掇之也。吾处也,若橛株驹,吾执臂若槁木之枝。虽天地之大,万物之多,而唯蜩翼之知。吾不反不侧,不以万物易蜩之翼,何为而不得?''孔子顾谓弟子曰:``用志不分,乃凝于神。其佝偻丈人之谓乎!''丈人曰:``汝逢衣徒也,亦何知问是乎?修汝所以,而后载言其上。''

海上之人有好沤鸟者,每旦之海上,从沤鸟游,沤鸟之至者百住而不止。其父曰:``吾闻沤鸟皆从汝游,汝取来,吾玩之。''明日之海上,沤鸟舞而不下也。故曰:至言去言,至为无为;齐智之所知,则浅矣。

赵襄子率徒十万,狩于中山,藉仍燔林,扇赫百里,有一人从石壁中出,随烟烬上下,众谓鬼物。火过,徐行而出,若无所经涉者,襄子怪而留之,徐而察之:形色七窍,人也;气息音声,人也。问奚道而处石?奚道而入火?其人曰:``奚物而谓石?奚物而谓火?''襄子曰:``而向之所出者,石也;而向之所涉者,火也。''其人曰:``不知也。''魏文侯闻之,问子夏曰:``彼何人哉?''
子夏曰:``以商所闻夫子之言,和者大同于物,物无得伤阂者,游金石,蹈水火,皆可也。''文侯曰:``吾子奚不为之?''子夏曰:``刳心去智,商未之能。虽然,试语之有暇矣。''文侯曰:``夫子奚不为之?''子夏曰:``夫子能之而能不为者也。''文侯大说。

有神巫自齐来处于郑,命曰季咸,知人死生、存亡、祸福、寿夭,期以岁、月、旬、日如神。郑人见之,皆避而走。列子见之而心醉,而归以告壶丘子,曰:``始吾以夫子之道为至矣,则又有至焉者矣。''壶子曰:``吾与汝无其文,未既其实,而固得道与?众雌而无雄,而又奚卵焉?而以道与世抗,必信矣,夫故使人得而相汝。尝试与来,以予示之。''明日,列子与之见壶子。出而谓列子曰:``嘻!子之先生死矣,弗活矣,不可以旬数矣。吾见怪焉,见湿灰焉。''列子入,涕泣沾襟,以告壶子。壶子曰:``向吾示之以地文,罪乎不誫不止,是殆见吾杜德几也。尝又与来!''明日,又与之见壶子,出而谓列子曰:``幸矣,子之先生遇我也,有瘳矣。灰然有生矣,吾见杜权矣。''列子入告壶子。壶子曰:``向吾示之以天壤,名实不入,而机发于踵,此为杜权。是殆见吾善者几也。尝又与来!''明日,又与之见壶子,出而谓列子曰:``子之先生坐不斋,吾无得而相焉。试斋,将且复相之。''列子入告壶子。壶子曰:``向吾示之以太冲莫朕,是殆见吾衡气几也。鲵旋之潘为渊,止水之潘为渊,流水之潘为渊,滥水之潘为渊,沃水之潘为渊,氿水之潘为渊,雍水之潘为渊,汧水之潘为渊,肥水之潘为渊,是为九渊焉。尝又与来!''明日,又与之见壶子。立未定,自失而走。壶子曰:``追之!''列子追之而不及,反以报壶子,曰:``已灭矣,已失矣,吾不及也。''壶子曰:''向吾示之以未始出吾宗。吾与之虚而猗移,不知其谁何,因以为茅靡,因以为波流,故逃也。''然后列子自以为未始学而归,三年不出,为其妻爨,食豕如食人,于事无亲,雕彖复朴,块然独以其形立;忄分然而封戎,壹以是终。

子列子之齐,中道而反,遇伯昏瞀人。伯昏瞀人曰:``奚方而反?''曰:``吾惊焉。''``恶乎惊?''``吾食于十浆,而五浆先馈。''伯昏瞀人曰:``右是,则汝何为惊已?''曰:``夫内诚不解,形谍成光,以外镇人心,使人轻乎贵老,而敕其所患。夫浆人特为食羹之货,多余之赢;其为利也薄,其为权也轻,而犹若是。而况万乘之主,身劳于国,而智尽于事;彼将任我以事,而效我以功,吾是以惊。''伯昏瞀人曰:``善哉观乎!汝处己,人将保汝矣。''无几何而往,则户外之屦满矣。伯昏瞀人北面而立,敦杖蹙之乎颐,立有间,不言而出。宾者以告列子。列子提履徒跣而走,暨乎门,问曰:``先生既来,曾不废药乎?''曰:``已矣。吾固告汝曰:,人将保汝,果保汝矣。非汝能使人保汝,而汝不能使人无汝保也,而焉用之感也?感豫出异。且必有感也,摇而本身,又无谓也。与汝游者,莫汝告也。彼所小言,尽人毒也。莫觉莫悟,何相孰也。''

杨朱南之沛,老聃西游于秦。邀于郊。至梁而遇老子。老子中道仰天而叹曰:``始以汝为可教,今不可教也。''杨朱不答。至舍,进涫漱巾栉,脱履户外,膝行而前,曰:``向者夫子仰天而叹曰:`始以汝为可教,今不可教。'弟子欲请夫子辞,行不闲,是以不敢。今夫子闲矣,请问其过。''老子曰:``而睢睢而盱盱,而谁与居?大白若辱,盛德若不足。''杨朱蹴然变容曰:``敬闻命矣!''其往也,舍迎将家,公执席,妻执巾栉,舍者避席,炀者避灶。其反也,舍者与之争席矣。

杨朱过宋,东之于逆旅。逆旅人有妾二人,其一人美,其一人恶;恶乾贵而美者贱。杨子问其故。逆旅小子对曰:``其美者自美,吾不知其美也;其恶者自恶,吾不知其恶也。''杨子曰:``弟子记之!行贤而去自贤之行,安往而不爱哉!''

天下有常胜之道,有不常胜之道。常胜之道曰柔,常不胜之道曰强。二者亦知。而人未之知。故上古之言:强,先不己若者;柔,先出于己者。先不己若者,至于若己,则殆矣。先出于己者,亡所殆矣。以此胜一身若徒,以此任天下若徒,谓不胜而自胜,不任而自任也。粥子曰:``欲刚,必以柔守之;欲强,必以弱保之。积于柔必刚,积于弱必强。观其所积,以知祸福之乡。强胜不若己,至于若己者刚;柔胜出于己者,其力不可量。''老聃曰:``兵强则灭。木强则折。柔弱者生之徒,坚强者死之徒。''

状不必童而智童;智不必童而状童。圣人取童智而遗童状,众人近童状而疏童智。状与我童者,近而爱之;状与我异者,疏而畏之。有七尺之骸,手足之异,戴发含齿,倚而趣者,谓之人;而人未必无兽心。虽有兽心,以状而见亲矣。傅翼翼戴角,分牙布爪,仰飞伏走,谓之禽兽;而禽兽未必无人心。虽有人心,以状而见疏矣。庖牺氏、女娲氏、神农氏、夏后氏,蛇身人面,牛首虎鼻:此有非人之状,而有大圣之德。夏桀、殷纣、鲁桓、楚穆,状貌七窍,皆同于人,而有禽兽之心。而众人守一状以求至智,未可几也。黄帝与炎帝战于阪泉之野,帅熊、罴、狼、豹、貙、虎为前驱,雕、鹖、鹰、鸢为旗帜,此以力使禽兽者也。尧使夔典乐,击石拊石,百兽率舞;箫韶九成,凤皇来仪,此以声致禽兽者也。然则禽兽之心,奚为异人?形音与人异,而不知接之之道焉。圣人无所不知,无所不通,故得引而使之焉。禽兽之智有自然与人童者,其齐欲摄生,亦不假智于人也。牝牡相偶,母子相亲,避平依险,违寒就温;居则有群,行则有列;小者居内,壮者居外;饮则相携,食则鸣群。太古之时,则与人同处,与人并行。帝王之时,始惊骇散乱矣。逮于末世,隐伏逃窜,以避患害。今东方介氏之国,其国人数数解六畜之语者,盖偏知之所得,太古神圣之人,备知万物情态,悉解异类音声。会而聚之,训而受之,同于人民。故先会鬼神魑魅,次达八方人民,末聚禽兽虫蛾。言血气之类心智不殊远也。神圣知其如此,故其所教训者无所遗逸焉。

宋有狙公者,爱狙;养之成群,能解狙之意;狙亦得公之心。损其家口,充狙之欲。俄而匮焉,将限其食。恐众狙之不驯于己也,先诳之曰:``与若芧,朝三而暮四,足乎?''众狙皆起而怒。俄而曰:``与若芧,朝三而暮四,足乎?''众狙皆伏而喜。物之以能鄙相笼,皆犹此也。圣人以智笼群愚,亦犹狙公之以智笼众狙也。名实不亏,使其喜怒哉。

纪渻子为周宣王养斗鸡,十日而问:``鸡可斗已乎?''曰:``未也,方虚骄而恃气。''十日又问。曰:``未也,犹应影响。''十日又问。曰:``未也,犹疾视而盛气。:十日又问。曰:``几矣。鸡虽有鸣者,已无变矣。望之似木鸡矣,其德全矣。异鸡无敢应者,反走耳。''

惠盎见宋康王。康王蹀足謦欬,疾言曰:``寡人之所说者,勇有力也,不说为仁义者也。客将何以教寡人?''惠盎对曰:``臣有道于此,使人虽勇,刺之不入;虽有力,击之弗中。大王独无意邪?''宋王曰:``善;此寡人之所欲闻也。''惠盎曰:``夫刺之不入,击之不中,此犹辱也。臣有道于此,使人虽有勇,弗敢刺;虽有力,弗敢击。夫弗敢,非无其志也。臣有道于此,使人本无其志也。夫无其志也,未有爱利之心也。臣有道于此,使天下丈夫女子莫不驩然皆欲爱利之。此其贤于勇有力也,四累之上也。大王独无意邪?''宋王曰:``此寡人之所欲得也。''惠盎对曰:``孔墨是已。孔丘墨翟无地而为君,无官而为长;天下丈夫女子莫不延颈举踵而愿安利之。今大王,万乘之主也;诚有其志,则四竟之内,皆得其利矣。其贤于孔墨也远矣。''宋王无以应。惠盎趋而出。宋王谓左右曰:``辩矣,客之以说服寡人也!''

\hypertarget{header-n47}{%
\subsection{周穆王}\label{header-n47}}

周穆王时,西极之国有化人来,入水火,贯金石;反山川,移城邑;乘虚不坠,触实不硋。千变万化,不可穷极。既已变物之形,又且易人之虑。穆王敬之若神,事之若君。推路寝以居之,引三牲以进之,选女乐以娱之。化人以为王之宫室卑陋而不可处,王之厨馔腥蝼而不可飨,王之嫔御膻恶而不可亲。穆王乃为之改筑。土木之功。赭垩之色,无遗巧焉。五府为虚,而台始成。其高千仞,临终南之上,号曰中天之台。简郑卫之处子娥媌靡曼者,施芳泽,正蛾眉,设笄珥,衣阿锡。曳齐纨。粉白黛黑,佩玉环。杂芷若以满之,奏《承云》、《六莹》、《九韶》、《晨露》以乐之。日月献玉衣,旦旦荐玉食。化人犹不舍然,不得已而临之。居亡几何,谒王同游。王执化人之祛,腾而上者,中天乃止。暨及化人之宫。化人之宫构以金银,络以珠玉;出云雨之上,而不知下之据,望之若屯云焉。耳目所观听,鼻口所纳尝,皆非人间之有。王实以为清都、紫微、钧天、广乐,帝之所居。王俯而视之,其宫榭若累块积苏焉。王自以居数十年不思其国也。化人复谒王同游,所及之处,仰不见日月,俯不见河海。光影所照,王目眩不能得视;音响所来,王耳乱不能得听。百骸六藏,悸而不凝。意迷精丧,请化人求还。化人移之,王若殒虚焉。既寤,所坐犹向者之处,侍御犹向者之人。视其前,则酒未清,肴未昲。王问所从来。左右曰:``王默存耳。''由此穆王自失者三月而复。更问化人。化人曰:``吾与王神游也,形奚动哉?且曩之所居,奚异王之宫?曩之所游,奚异王之圃?王闲恒有,疑暂亡。变化之极,徐疾之间,可尽模哉?''王大悦。不恤国事,不乐臣妾,肆意远游。命驾八骏之乘,右服骅骝而左绿耳,右骖赤骥而左白\{减木\},主车则造父为御,离离右;次车之乘,右服渠黄而左逾轮,左骖盗骊而右山子,柏夭主车,参百为御,奔戎为右。驰驱千里,至于巨蒐氏之国。巨蒐氏乃献白鹄之血以饮王,具牛马之湩以洗王之足,及二乘之人。已饮而行,遂宿于昆仑之阿,赤水之阳。别日升昆仑之丘,以观黄帝之吕,而封之以诒后世。遂宾于西王母,觞于瑶池之上。西王母为王谣,王和之,其辞哀焉。乃观日之所入。一日行万里。王乃叹曰:``於乎!予一人不盈于德而谐于乐,后世其追数吾过乎!''穆王几神人哉!能穷当身之乐,犹百年乃徂,世以为登假焉。

老成子学幻于尹文先生,三年不告。老成子请其过而求退。尹文先生揖而进之于室,屏左右而与之言曰:``昔老聃之徂西也,顾而告予曰:有生之气,有形之状,尽幻也。造化之所始,阴阳之所变者,谓之生,谓之死。穷数达变,因形移易者,谓之化,谓之幻。造物者其巧妙,其功深,固难穷难终。因形者其巧显,其功浅,故随起随灭。知幻化之不异生死也,始可与学幻矣。吾与汝亦幻也,奚须学哉?''老成了归,用尹文先生之言深思三月,遂能存亡自在,憣校四时;冬起雷,夏造冰。飞者走,走者飞。终身不箸其术,故世莫传焉。子列子曰:``善为化者,其道密庸,其功同人。五帝之德,三王之功,未必尽智勇之力,或由化而成。孰测之哉?''

觉有八徵,梦有六侯。奚谓八徵?一曰故,二曰为,三曰得,四曰丧,五曰哀,六曰乐,七曰生,八曰死。此者八徵,形所接也。奚谓六侯?一曰正梦,二曰愕梦,三曰思梦,四曰寤梦,五曰喜梦,六曰惧梦。此六者,神所交也。不识感变之所起者,事至则惑其所由然,识感变之所起者,事至则知其所由然。知其所由然,则无所怛。一体之盈虚消息,皆通于天地,应于物类。故阴气壮,则梦涉大水而恐惧;阳气壮,则梦涉大火而燔内;阴阳俱壮,则梦生杀。甚饱则梦与,甚饥则梦取。是以以浮虚为疾者,则梦扬;以沈实为疾者,则梦溺。藉带而寝则梦蛇;飞鸟衔发则梦飞。将阴梦火,将疾梦食。饮酒者忧,歌舞者哭。子列子曰:''神遇为梦,形接为事。故昼想夜梦,神形所遇。故神凝者想梦自消。信觉不语,信梦不达,物化之往来者也。古之真人,其觉自忘,其寝不梦,几虚语哉?''

西极之南隅有国焉,不知境界之所接,名古莽之国。阴阳之气所不交,故寒暑亡辨;日月之光所不照,故昼夜亡辨。其民不食不衣而多眠。五旬一觉,以梦中所为者实,觉之所见者妄。四海之齐谓中央之国,跨河南北,越岱东西,万有余里。其阴阳之审度,故一寒一暑;昏明之分察,故一昼一夜。其民有智有愚。万物滋殖,才艺多方。有君臣相临,礼法相持。其所云为,不可称计。一觉一寐,以为觉之所为者实,梦之所见者妄。东极之北隅有国曰阜落之国。其土气常燠,日月余光之照。其土不生嘉苗。其民食草根木实,不知火食。性刚悍,强弱相藉,贵胜而不尚义;多驰步,少休息,常觉而不眠。

周之尹氏大治产,其下趣役者侵晨昏而弗息。有老役夫筋力竭矣,而使之弥勤。昼则呻呼而即事,夜则昏惫而熟寐。精神荒散,昔昔梦为国君。居人民之上,总一国之事。游燕宫观,恣意所欲,其乐无比。觉则复役。人有慰喻其勤者,役夫曰:``人生百年,昼夜各分。吾昼为仆虏,苦则苦矣;夜为人君,其乐无比。何所怨哉?''尹氏心营世事,虑钟家业,心形俱疲,夜亦昏惫而寐。昔昔梦为人仆,趋走作役,无不为也;数骂杖挞,无不至也。眠中啽呓呻呼,彻旦息焉。尹氏病之,以访其友。友曰:``若位足荣身,资财有余,胜人远矣。夜梦为仆,苦逸之复,数之常也。若欲觉梦兼之,岂可得邪?''尹氏闻其友言,宽其役夫之程,减己思虑之事,疾并少间。

郑人有薪于野者,遇骇鹿,御而击之,毙之。恐人见之也,遽而藏诸隍中,覆之以蕉,不胜其喜。俄而遗其所藏之处,遂以为梦焉。顺途而咏其事。傍人有闻者,用其言而取之。既归,告其室人曰:``向薪者梦得鹿而不知其处;吾今得之,彼直真梦者矣。''室人曰:``若将是梦见薪者之得鹿邪?讵有薪者邪?今真得鹿,是若之梦真邪?''夫曰:``吾据得鹿,何用知彼梦我梦邪?''薪者之归,不厌失鹿,其夜真梦藏之之处,又梦得之之主。爽旦,案所梦而寻得之。遂讼而争之,归之士师。士师曰:``若初真得鹿,妄谓之梦;真梦得鹿,妄谓之实。彼真取若鹿,而与若争鹿。室人又谓梦仞人鹿,无人得鹿。今据有此鹿,请二分之。''以闻郑君。郑君曰:``嘻!士师将复梦分人鹿乎?''访之国相。国相曰:``梦与不梦,臣所不能辨也。欲辨觉梦,唯黄帝孔丘。今亡黄帝孔丘,熟辨之哉?且恂士师之言可也。''

宋阳里华子中年病忘,朝取而夕忘,夕与而朝忘;在途则忘行,在室而忘坐;今不识先,后不识今。阖室毒之。谒史而卜之,弗占;谒巫而祷之,弗禁;谒医而攻之,弗已。鲁有儒生自媒能治之,华子之妻子以居产之半请其方。儒生曰:``此固非封兆之所占,非祈请之所祷,非药石之所攻。吾试化其心,变其虑,庶几其瘳乎!''于是试露之,而求衣;饥之,而求食;幽之,而求明。儒生欣然告其子曰:``疾可已也。然吾之方密,传世不以告人。试屏左右,独与居室七曰。''从之。莫知其所施为也,而积年之疾一朝都除。华子既悟,乃大怒,黜妻罚子,操戈逐儒生。宋人执而问其以。华子曰:``曩吾忘也,荡荡然不觉天地之有无。今顿识既往,数十年来存亡、得失、哀乐、好恶,扰扰万绪起矣。吾恐将来之存亡、得失、哀乐、好恶之乱吾心如此也,须臾之忘;可复得乎?''子贡闻而怪之,以告孔子。孔子曰:``此非汝所及乎!''顾谓颜回纪之。

秦人逄氏有子,少而惠,及壮而有迷罔之疾。闻歌以为哭,视白以为黑,飨香以为朽,尝甘以为苦,行非以为是:意之所之,天地、四方,水火、寒暑,无不倒错者焉。杨氏告其父曰:``鲁之君子多术艺,将能已乎?汝奚不访焉?''其父之鲁,过陈,遇老聃,因告其子之证。老聃曰:``汝庸知汝子之迷乎?今天下之人皆惑于是非,昏于利害。同疾者多,固莫有觉者。且一身之迷不足倾一家,一家之迷不足倾一乡,一乡之迷不足倾一国,一国之迷不足倾天下。天下尽迷,孰倾之哉?向使天下之人其心尽如汝子,汝则反迷矣。哀乐、声色、臭味、是非,孰能正之?且吾之此言未必非迷,而况鲁之君子,迷之邮者,焉能解人之迷哉?荣汝之粮,不若遄归也。''

燕人生于燕,长于楚,及老而还本国。过晋国,同行者诳之;指城曰:``此燕国之城。''其人愀然变容。指社曰:``此若里之社。''乃谓然而叹。指舍曰:``此若先人之庐。''乃涓然而泣。指垅曰:``此若先人之冢。''其人哭不自禁。同行者哑然大笑,曰:``予昔给若,此晋国耳。''其人大惭。及至燕,真见燕国之城社,真见先人之庐冢,悲心更微。

\hypertarget{header-n59}{%
\subsection{仲尼}\label{header-n59}}

仲尼闲居,子贡入待,而有忧色。子贡不敢问,出告颜回。颜回援琴而歌。孔子闻之,果召回入,问曰:``若奚独乐?''回曰:``夫子奚独忧?''孔子曰:``先言尔志。''曰:``吾昔闻之夫子曰:`乐天知命故不忧',回所以乐也。''孔子愀然有间曰:``有是言哉?汝之意失矣。此吾昔日之言尔,请以今言为正也。汝徒知乐天知命之无忧,未知乐天知命有忧之大也。今告若其实。修一身,任穷达,知去来之非我,亡变乱于心虑,尔之所谓乐天知命之无忧也。曩吾修《诗》《书》,正礼乐,将以治天下,遣来世;非但修一身,治鲁国而已。而鲁之君臣日失其序,仁义益衰,情性益薄。此道不行一国与当年,其如天下与来世矣?吾始知《诗》《书》礼乐无救于治乱,而未知所以革之之方:此乐天知命者之所忧。虽然,吾得之矣。夫乐而知者,非古人之谓所乐知也。无乐无知,是真乐真知;故无所不乐,无所不知,无所不忧,无所不为。《诗》《书》礼乐,何弃之有?革之何为?''颜回北面拜手曰:``回亦得之矣。''出告子贡。子贡茫然自失,归家淫思七日,不寝不食,以至骨立。颜回重往喻之,乃反丘门,弦歌诵书,终身不辍。

陈大夫聘鲁,私见叔孙氏。叔孙氏曰:``吾国有圣人。''曰:``非孔丘邪?''曰:``是也。''``何以知其圣乎?''叔孙氏曰:``吾常闻之颜回,曰:`孔丘能废心而用形。'''陈大夫曰:``吾国亦有圣人,子弗知乎?''曰:``圣人孰谓?''曰:``老聃之弟子有亢仓之者,得聃之道,能以耳视而目听。''鲁侯闻之大惊,使上卿厚礼而致之。亢仓子应聘而至。鲁侯卑辞请问之。亢仓子曰:``传之者妄。我能视听不用耳目,不能易耳目之用。''鲁侯曰:``此增异矣。其道奈何?寡人终愿闻之。''亢仓子曰:``我体合于心,心合于气,气合于神,神合于无。其有介然之有,唯然之音,虽远在八荒之外,近在眉睫之内,来干我者,我必知之。乃不知是我七孔四支之所觉,心腹六脏之知,其自知而已矣。''鲁侯大悦。他日以告仲尼,仲尼笑而不答。

商太宰见孔子曰:``丘圣者欤?''孔子曰:``圣则丘何敢,然则丘博学多识者也。''商太宰曰:``三王圣者欤?''孔子曰:``三王善任智勇者,圣则丘弗知。''曰:``五帝圣者欤?''孔子曰:``五帝善任仁义者,圣则丘弗知。''曰:``三皇圣者欤?''孔子曰:``三皇善任因时者,圣则丘弗知。''商太宰大骇,曰:``然则孰者为圣?''孔子动容有间,曰:``西方之人,有圣者焉,不治而不乱,不言而自信,不化而自行,荡荡乎民无能名焉。丘疑其为圣。弗知真为圣欤?真不圣欤?''商太宰嘿然心计曰:``孔丘欺我哉!''

子夏问孔子曰:``颜回之为人奚若?''子曰:``回之仁贤于丘也。''曰:``子贡之为人奚若?''子曰:``赐之辨贤于丘也。''曰:``子路之为人奚若?''子曰:``由之勇贤于丘也。''曰:``子张之为人奚若?''子曰:``师之庄贤于丘也。''子夏避席而问曰:``然则四子者何为事夫子?''曰:``居!吾语汝。夫回能仁而不能反,赐能辨而不能讷,由能勇而不能怯,师能庄而不能同。兼四子之有以易吾,吾弗许也。此其所以事吾而不贰也。''

子列子既师壶丘子林,友伯昏瞀人,乃居南郭。从之处者,日数而不及。虽然,子列子亦微焉,朝朝相与辨,无不闻。而与南郭子连墙二十年,不上谒请;相遇于道,目若不相见者。门之徒役以为子列子与南郭子有敌不疑。有自楚来者,问子列子曰:``先生与南郭子奚敌?''子列子曰:``南郭子貌充心虚,耳无闻,目无见,口无言,心无知,形无惕。往将奚为?虽然,试与汝偕往。''阅弟子四十人同行。见南郭子,果若欺魄焉,而不可与接。顾视子列子,形神不相偶,而不可与群。南郭子俄而指子列子之弟子末行者与言,衎衎然若专直而在雄者。子列子之徒骇之。反舍,咸有疑色。子列子曰:``得意者无言,进知者亦无言。用无言为言亦言,无知为知亦知。无言与不言,无知与不知,亦言亦知。亦无所不言,亦无所不知;亦无所言,亦无所知。如斯而已。汝奚妄骇哉?''

子列子学也,三年之后,心不敢念是非,口不敢言利害,始得老商一眄而已。五年之后,心更念是非,口更言利害,老商始一解颜而笑。七年之后,从心之所念,更无是非;从口之所言,更无利害。夫子始一引吾并席而坐。九年之后,横心之所念,横口之所言,亦不知我之是非利害欤,亦不知彼之是非利害欤,外内进矣。而后眼如耳,耳如鼻,鼻如口,口无不同。心凝形释,骨肉者融;不觉形之所倚,足之所履,心之所念,言之所藏。如斯而已。则理无所隐矣。

初,子列子好游。壶丘子曰:``御寇好游,游何所好?''列子曰:``游之乐所玩无故。人之游也,观其所见;我之游也,观之所变。游乎游乎!未有能辨其游者。''壶丘子曰:``御寇之游固与人同欤,而曰固与人异欤?凡所见,亦恒见其变。玩彼物之无故,不知我亦无故。务外游,不知务内观。外游者,求备于物;内观者,取足于身。取足于身,游之至也;求备于物,游之不至也。''于是列子终身不出,自以为不知游。壶丘子曰:``游其至乎!至游者,不知所适;至观者,不知所眂,物物皆游矣,物物皆观矣,是我之所谓游,是我之所谓观也。故曰:游其至矣乎!游其至矣乎!''

龙叔谓文挚曰:``子之术微矣。吾有疾,子能已乎?''文挚曰:``唯命所听。然先言子所病之正。''龙叔曰:``吾乡誉不以为荣,国毁不以为辱;得而不喜,失而弗忧;视生如死;视富如贫;视人如豕;视吾如人。处吾之家,如逆旅之舍;观吾之乡,如戎蛮之国。凡此众疾,爵赏不能劝,刑罚不能威,盛衰、利害不能易,哀乐不能移。固不可事国君,交亲友,御妻子,制仆隶。此奚疾哉?奚方能已之乎?''文挚乃命龙叔背明而立,文挚自后向明而望之。既而曰:``嘻!吾见子之心矣,方寸之地虚矣。几圣人也!子心六孔流通,一孔不达。今以圣智为疾者,或由此乎!非吾浅术所能已也。''

无所由而常生者,道也。由生而生,故虽终而不亡,常也。由生而亡,不幸也。有所由而常死者,亦道也。由死而死,故虽未终而自亡者,亦常也。由死而生,幸也。故无用而生谓之道,用道得终谓之常;有所用而死者亦谓之道,用道而得死者亦谓之常。季梁之死,杨朱望其门而歌。随梧之死,杨朱抚其尸而哭。隶人之生,隶人之死,众人且歌,众人且哭。目将眇者,先睹秋毫;耳将聋者,先闻蚋飞;口将爽者,先辨淄渑;鼻将窒者,先觉焦朽;体将僵者,先亟奔佚;心将迷者,先识是非:故物不至者则不反。

郑之圃泽多贤,东里多才。圃泽之役有伯丰子者,行过东里,遇邓析。观析顾其徒而笑曰:``为若舞,彼来者奚若?''其徒曰:``所愿知也。''邓析谓伯丰子曰:``汝知养养之义乎?受人养而不能自养者,犬豕之类也;养物而物为我用者,人之力也。使汝之徒食而饱,衣而息,执政之功也。长幼群聚而为牢藉庖厨之物,奚异犬豕之类乎?''伯丰子不应。伯丰子之从者越次而进曰:``大夫不闻齐鲁之多机乎?有善治土木者,有善治金革者,有善治声乐者,有善治书数者,有善治军旅者,有善治宗庙者,群才备也。而无相位者,无能相使者。而位之者无知,使之者无能,而知之与能为之使焉。执政者,乃吾之所使;子奚矜焉?''邓析无以应,目其徒而退。

公仪伯以力闻诸侯,堂谿公言之于周宣王,王备礼以聘之。公仪伯至;观形,懦夫也。宣王心惑而疑曰:``女之力何如?''公仪伯曰:``臣之力能折春螽之股,堪秋蝉之翼。''王作色曰:``吾之力者能裂犀兕之革,曳九牛之尾,犹憾其弱。女折春螽之股,堪秋蝉之翼,而力闻天下,何也?''公仪伯长息退席,曰:``善哉王之问也!臣敢以实对。臣之师有商丘子者,力无敌于天下,而六亲不知,以未尝用其力故也。臣以死事之。乃告臣曰:`人欲见其所不见,视人所不窥;欲得其所不得,修人所不为。故学眎者先见舆薪,学听者先闻掸钟。夫有易于内者无难于外。于外无难,故名不出其一家。'今臣之名闻于诸侯,是臣违师之教,显臣之能者也。然则臣之名不以负其力者也,以能用其力者也;不犹愈于负其力者乎?''

中山公子牟者,魏国之贤公子也。好与贤人游,不恤国事;而悦赵人公孙龙。乐正子舆之徒笑之。公子牟曰:``子何笑牟之悦公孙龙也?''子舆曰:``公孙龙之为人也,行无师,学无友,佞给而不中,漫衍而无家,好怪而妄言。欲惑人之心,屈人之口,与韩檀等肄之。''公子牟变容曰:``何子状公孙龙之过欤?请闻其实。''子舆曰:``吾笑龙之诒孔穿,言`善射者,能令后镞中前括,发发相及,矢矢相属;前矢造准而无绝落,后矢之括犹衔弦,视之若一焉。'孔穿骇之。龙曰:`此未其妙者。逢蒙之弟子曰鸿超,怒其妻而怖之。引乌号之弓,綦卫之箭,射其目。矢来注眸子而眶不睫,矢隧地而尘不扬。'是岂智者之言与?``公子牟曰:''智者之言固非愚者之所晓。後镞中前括,钧後于前。矢注眸子而眶不睫,尽矢之势也。子何疑焉?``乐正子舆曰:`子,龙之徒,焉得不饰其阙?吾又言其尤者。'龙诳魏王曰:`有意不心。有指不至。有物不尽。有影不移。发引千钧。白马非马。孤犊未尝有母。'`其负类反伦,不可胜言也。''公子牟曰:'子不谕至言而以为尤也,尤其在子矣。夫无意则心同。无指则皆至。尽物者常有。影不移者,说在改也。发引千钧,势至等也。白马非马,形名离也。孤犊未尝有母,非孤犊也。''乐正子舆曰:``子以公孙龙之鸣皆条也。设令发于余窍,子亦将承之。''公子牟默然良久,告退,曰:``请待余曰,更谒子论。''

尧治天下五十年,不知天下治欤,不治欤?不知亿兆之愿戴己欤?不愿戴己欤?顾问左右,左右不知。问外朝,外朝不知。问在野,在野不知。尧乃微服游于康衢,闻儿童谣曰:``立我蒸民,莫匪尔极。不识不知,顺帝不则。''尧喜问曰:``谁教尔为此言?''童儿曰:``我闻之大夫。''问大夫,大夫曰:``古诗也。''尧还宫,召舜,因禅以天下。舜不辞而受之。

关尹喜曰:``在己无居,形物其著,其动若水,其静若镜,其应若响。故其道若物者也。物自违道,道不违物。善若道者,亦不用耳,亦不用目,亦不用力,亦不用心。欲若道而用视听形智以求之,弗当矣。瞻之在前,忽焉在后;用之弥满,六虚废之莫知其所。亦非有心者所能得远,亦非无心者所能得近。唯默而得之而性成之者得之。知而忘情,能而不为,真知真能也。发无知,何能情?发不能,何能为?聚块也,积尘也,虽无为而非理也。''

\hypertarget{header-n76}{%
\subsection{汤问}\label{header-n76}}

殷汤问于夏革曰:``古初有物乎?''夏革曰:``古初无物,今恶得物?后之人将谓今之无物,可乎?''殷汤曰:``然则物无先后乎?''夏革曰:``物之终始,初无极已。始或为终,终或为始,恶知其纪?然自物之外,自事之先,朕所不知也。''殷汤曰:``然则上下八方有极尽乎?''革曰:``不知也。''汤固问。革曰:``无则无极,有则有尽;朕何以知之?然无极之外复无无极,无尽之中复无无尽。无极复无无极,无尽复无无尽。朕以是知其无极无尽也,而不知其有极有尽也。''汤又问曰:``四海之外奚有?''革曰:``犹齐州也。''汤曰:``汝奚以实之?''革曰:``朕东行至营,人民犹是也。问营之东,复犹营也。西行至豳,人民犹是也。问豳之西,复犹豳也。朕以是知四海、四荒、四极之不异是也。故大小相含,无穷极也。含万物者,亦如含天地。含万物也故不穷,含天地也故无极。朕亦焉知天地之表不有大天地者乎?亦吾所不知也。然则天地亦物也。物有不足,故昔者女娲氏炼五色石以补其阙;断鳌之足以立四极。其后共工氏与颛顼争为帝,怒而触不周之山,折天柱,绝地维;故天倾西北,日月星辰就焉;地不满东南,故百川水潦归焉。''

汤又问:``物有巨细乎?有修短乎?有同异乎?''革曰:``渤海之东不知几亿万里,有大壑焉,实惟无底之谷,其下无底,名曰归墟。八纮九野之水,天汉之流,莫不注之,而无增无减焉。其中有五山焉:一曰岱舆,二曰员峤,三曰方壶,四曰瀛洲,五曰蓬莱。其山高下周旋三万里,其顶平处九千里。山之中间相去七万里,以为邻居焉。其上台观皆金玉,其上禽兽皆纯缟。珠玕之树皆丛生,华实皆有滋味,食之皆不老不死。所居之人皆仙圣之种;一日一夕飞相往来者,不可数焉。而五山之根无所连著,常随潮波上下往还,不得暂峙焉。仙圣毒之,诉之于帝。帝恐流于西极,失群仙圣之居,乃命禺强使巨鳌十五举首而戴之。迭为三番,六万岁一交焉。五山始峙而不动。而龙伯之国有大人,举足不盈数步而暨五山之所,一钓而连六鳌,合负而趣,归其国,灼其骨以数焉。员峤二山流于北极,沈于大海,仙圣之播迁者巨亿计。帝凭怒,侵减龙伯之国使厄。侵小龙伯之民使短。至伏羲神农时,其国人犹数十丈。从中州以东四十万里得憔侥国。,人长一尺五寸。东北极有人名曰诤人,长九寸。荆之南有冥灵者,以五百岁为春,五百岁为秋。上古有大椿者,以八千岁为春,八千岁为秋。朽壤之上有菌芝者,生于朝,死于晦。春夏之月有蠓蚋者,因雨而生,见阳而死。终北之北有溟海者,天池也,有鱼焉。其广数千里,其长称焉,其名为鲲。有鸟焉,其名为鹏,翼若垂天之云,其体称焉。世岂知有此物哉?大禹行而见之,伯益知而名之,夷坚闻而志之。江浦之间生麽虫,其名曰焦螟,群飞而集于蚊睫,弗相触也。栖宿去来,蚊弗觉也。离朱子羽方昼拭眦扬眉而望之,弗见其形;虒俞师旷方夜擿耳俯首而听之,弗闻其声。唯黄帝与容成子居空峒之上,同斋三月,心死形废;徐以神视,块然见之,若嵩山之阿;徐以气听,砰然闻之,若雷霆之声。吴楚之国有大木焉,其名为櫾,碧树而冬生,实丹而味酸。食其皮汁,已愤厥之疾。齐州珍之,渡淮而北而化为枳焉。鸲鹆不逾济,貉逾汶则死矣。地气然也。虽然,形气异也,性钧已,无相易已。生皆全已,分皆足已。吾何以识其巨细?何以识其修短?何以识其同异哉?''

太形、王屋二山,方七百里,高万仞。本在冀州之南,河阳之北。北山愚公者,年且九十,面山而居。惩山北之塞,出入之迂也,聚室而谋,曰:``吾与汝毕力平险,指通豫南,达于汉阴,可乎?''杂然相许。其妻献疑曰:``以君之力,曾不能损魁父之丘,如太形王屋何?且焉置土石?''杂曰:``投诸渤海之尾,隐土之北。''遂率子孙荷担者三夫,叩石垦壤,箕畚运于渤海之尾。邻人京城氏之孀妻有遣男,始龀,跳往助之。寒暑易节,始一反焉。河曲智叟笑山之,曰:``甚矣汝之不惠!以残年馀力,曾不能悔山之一毛,其如土石何?''北山愚公长息曰:``汝心不固,固不可彻,曾不若孀妻弱子。虽我之死,有子存焉。子又生孙,孙又生子;子又有子,子又有孙:子子孙孙,无穷匮也,而山不加增,何苦而不平?''河曲智叟亡以应。操蛇之神闻之,惧其不已也,告之于帝。帝感其诚,命夸蛾氏二子负二山,一厝朔东,一厝雍南。自此冀之南、汉之阴,无陇断焉。

夸父不量力,欲追日影,逐之于隅谷之际。渴欲得饮,赴饮河渭。河谓不足,将走北饮大泽。未至,道渴而死。弃其杖,尸膏肉所浸,生邓林。邓林弥广数千里焉。

大禹曰:``六合之间,四海之内,照之以日月,经之以星辰,纪之以四时,要之以太岁。神灵所生,其物异形;或夭或寿,唯圣人能通其道。''夏革曰:``然则亦有不待神灵而生,不待阴阳而形,不待日月而明,不待杀戮而夭,不待将迎而寿,不待五谷而食,不待缯纩而衣,不待舟车而行。其道自然,非圣人之所通也。''

禹之治水土也,迷而失途,谬之一国。滨北海之北,不知距齐州几千万里,其国名曰终北,不知际畔之所齐限。无风雨霜露,不生鸟兽、虫鱼、草木之类。四方悉平,周以乔陟。当国之中有山,山名壶领,状若\textless{}詹瓦\textgreater{}甀。顶有口,状若员环,名曰滋穴。有水涌出,名曰神氵粪,臭过兰椒,味过醪醴。一源分为四埒,注于山下。经营一国,亡不悉遍。土气和,亡札厉。人性婉而从物,不竞不争。柔心而弱骨,不骄不忌;长幼侪居,不君不臣;男女杂游,不媒不聘;缘水而居,不耕不稼。土气温适,不织不衣;百年而死,不夭不病。其民孳阜亡数,有喜乐,亡衰老哀苦。其俗好声,相携而迭谣,终日不辍者。饥惓则饮神氵粪,力志和平。过则醉,经旬乃醒。沐浴神氵粪,肤色脂泽,香气经旬乃歇。周穆王北游过其国,三年忘归。既反周室,慕其国,忄敞然自失。不进酒肉,不召嫔御者,数月乃复。管仲勉齐桓公因游辽口,俱之其国。几克举,隰朋谏曰:``君舍齐国之广,人民之众,山川之观,殖物之阜,礼义之盛,章服之美;妖靡盈庭,忠良满朝。肆咤则徒卒百万,视捴则诸侯从命,亦奚羡于彼而弃齐国之社稷,从戎夷之国乎?此仲父之耄,奈何从之?''桓公乃止,以隰朋之言告管仲。仲曰:``此固非朋之所及也。臣恐彼国之不可知之也。齐国之富奚恋?隰朋之言奚顾?''

南国之人祝发而裸;北国之人曷巾而裘;中国之人冠冕而裳。九土所资,或农或商,或田或渔,如冬裘夏葛,水舟陆车,默而得之,性而成之。越之东有辄沐之国,其长子生,则鲜而食之,谓之宜弟。其大父死,负其大母而弃之,曰:``鬼妻不可以同居处。''楚之南有炎人之国,其亲戚死,剔其肉而弃之,然后埋其骨,乃成为孝子。秦之西有仪渠之国者,其亲戚死。聚柴积而焚之。燻则烟上,谓之登遐,然后成为孝子。此上以为政,下以为俗。而未足为异也。

孔子东游,见两小儿辩斗。问其故,一儿曰:``我以日始出时去人近,而日中时远也。''一儿以日初出远,而日中时近也。一儿曰:``日初出大如车盖,及日中则如盘盂,此不为远者小而近者大乎?''一儿曰:``日初出沧沧凉凉,及其日中如探汤,此不为近者热而远者凉乎?''孔子不能决也。两小儿笑曰:``孰为汝多知乎?''

均,天下之至理也,连于形物亦然。均发均县轻重而发绝,发不均也。均也,其绝也,莫绝。人以为不然,自有知其然者也。詹何以独茧丝为纶,芒针为钩,荆筱为竿,剖粒为饵,引盈车之鱼于百仞之渊、汨流之中,纶不绝,钩不伸,竿不挠。楚王闻而异之,召问其故。詹何曰:``臣闻先大夫之言。蒲且子之弋也,弱弓纤缴,乘风振之,连双仓于青云之际。用心专,动手均也。臣因其事,放而学钓,五年始尽其道。当臣之临河持竿,心无杂虑,唯鱼之念;投纶沉钩,手无轻重,物莫能乱。鱼见臣之钩饵,犹沉埃聚沫,吞之不疑。所以能以弱制强,以轻致重也。大王治国诚能若此,则天下可运于一握,将亦奚事哉?''楚王曰:``善。''

鲁公扈赵齐婴二人有疾,同请扁鹊求治。扁鹊治之。既同愈。谓公扈齐婴曰:``汝曩之所疾,自外而干府藏者,固药石之所已。今有偕生之疾,与体偕长,今为汝攻之,何如?''二人曰:``愿先闻其验。''扁鹊谓公扈曰:``汝志强而气弱,故足于谋而寡于断。齐婴志弱而气强,故少于虑而伤于专。若换汝之心,则均于善矣。''扁鹊遂饮二人毒酒,迷死三日,剖胸探心,易而置之;投以神药,既悟如初。二人辞归。于是公扈反齐婴之室,而有其妻子,妻子弗识。齐婴亦反公扈之室室,有其妻子,妻子亦弗识。二室因相与讼,求辨于扁鹊。扁鹊辨其所由,讼乃已。

匏巴鼓琴而鸟舞鱼跃,郑师文闻之,弃家从师襄游。柱指钧弦,三年不成章。师襄曰:``子可以归矣。''师文舍其琴,叹曰:``文非弦之不能钩,非章之不能成。文所存者不在弦,所志者不在声。内不得于心,外不应于器,故不敢发手而动弦。且小假之,以观其所。''无几何,复见师襄。师襄曰:``子之琴何如?''师文曰:``得之矣。请尝试之。''于是当春而叩商弦以召南吕,凉风忽至,草木成实。及秋而叩角弦,以激夹钟,温风徐回,草木发荣。当夏而叩羽弦以召黄钟,霜雪交下,川池暴沍。及冬而叩徵弦以激蕤宾,阳光炽烈,坚冰立散。将终,命宫而总四弦,则景风翔,庆云浮,甘露降,澧泉涌。师襄乃抚心高蹈曰:``微矣,子之弹也!虽师旷之清角,邹衍之吹律,亡以加之。被将挟琴执管而从子之后耳。''

薛谭学讴于秦青,未穷青之技,自谓尽之;遂辞归。秦青弗止。饯于郊衢,抚节悲歌,声振林木,响遏行云。薛谭乃谢求反,终身不敢言归。秦青顾谓其友曰:``昔韩娥东之齐,匮粮,过雍门,鬻歌假食。既去而余音绕梁欐,三日不绝,左右以其人弗去。过逆旅,逆旅人辱之。韩娥因曼声哀哭,一里老幼悲悉,垂涕相对,三日不食。遽百追之。娥还,复为曼声长歌,一里老幼善跃抃舞,弗能自禁,忘向之悲也。乃厚赂发之。故雍门之人至今善歌哭,放娥之遗声。''

伯牙善鼓琴,钟子期善听。伯牙鼓琴,志在登高山。钟子期曰:``善哉!峨峨兮若泰山!''志在流水,钟子期曰:``善哉洋洋兮若江河!''伯牙所念,钟子期必得之。伯牙游于泰山之阴,卒逢暴雨,止于岩下;心悲,用援琴而鼓之。初为霖雨之操,更造崩山之音。曲每奏,钟子期辄穷其趣。伯牙乃舍琴而叹曰:``善哉,善哉!子之听夫志想象犹吾心也。吾于何逃声哉?''

周穆王西巡狩,越昆仑,不至弇山。反还,未及中国,道有献工人名偃师,穆王荐之,问曰:``若有何能?''偃师曰:``臣唯命所试。然臣已有所造,愿王先观之。''穆王曰:``日以俱来,吾与若俱观之。''翌日,偃师谒见王。王荐之曰:``若与偕来者何人邪?''对曰:``臣之所造能倡者。''穆王惊视之,趋步俯仰,信人也。巧夫顉其颐,则歌合律;捧其手,则舞应节。千变万化,惟意所适。王以为实人也,与盛姬内御并观之。技将终,倡者瞬其目而招王之左右待妾。王大怒,立欲诛偃师。偃师大慑,立剖散倡者以示王,皆傅会革、木、胶、漆、白、黑、丹、青之所为。王谛料之,内则肝、胆、心、肺、脾、肾、肠、胃,外则筋骨、支节、皮毛、齿发,皆假物也,而无不毕具者。合会复如初见。王试废其心,则口不能言;废其肝,则目不能视;废其肾,则足不能步。穆王始悦而叹曰:``人之巧乃可与造化者同功乎?''诏贰车载之以归。夫班输之云梯,墨翟之飞鸢,自谓能之极也。弟子东门贾禽滑厘闻偃师之巧,以告二子,二子终身不敢语艺,而时执规矩。

甘蝇,古之善射者,彀弓而兽伏鸟下。弟子名飞卫,学射于甘蝇,而巧过其师。纪昌者,又学射于飞卫。飞卫曰:``尔先学不瞬,而后可言射矣。''纪昌归,偃卧其妻之机下,以目承牵挺。二年之后,虽锥末倒眦,而不瞬也。以告飞卫。飞卫曰:``未也,必学视而后可。视小如大,视微如著,而后告我。''昌以牦悬虱于牖。南面而望之。旬日之间,浸大也;三年之后,如车轮焉。以睹余物,皆丘山也。乃以燕角之弧、朔蓬之簳射之,贯虱之心,而悬不绝。以告飞卫。飞卫高蹈拊膺曰:``汝得之矣!''纪昌既尽卫之术,计天下之敌己者,一人而已;乃谋杀飞卫。相遇于野,二人交射;中路端锋相触,而坠于地,而尘不扬。飞卫之矢先穷。纪昌遗一矢;既发,飞卫以棘刺之端扌干之,而无差焉。于是二子泣而投弓,相拜于途,请为父子。克臂以誓,不得告术于人。

造父之师曰泰豆氏。造父之始从习御也,执礼甚稗,泰豆三年不告。造父执礼愈谨,乃告之曰:``古诗言:`良弓之子,必先为箕,良冶之子,必先为裘。'汝先观吾趣。趣如事,然后六辔可持,六马可御。''造父曰:``唯命所从。''泰豆乃立木为途,仅可容足;计步而置。履之而行。趣走往还,无跌失也。造父学子,三日尽其巧。泰豆叹曰:``子何其敏也?得之捷乎!凡所御者,亦如此也。嚷汝之行,得之于足,应之于心。推于御也,齐辑乎辔衔之际,而急缓乎唇吻之和,正度乎胸臆之中,而执节乎掌握之间。内得于中心,而外合于马志,是故能进退履绳而旋曲中规矩,取道致远而气力有余,诚得其术也。得之于衔,应之于辔;得之于辔,应之于手;得之于手,应之于心。则不以目视,不以策驱;心闲体正,六辔不乱,而二十四蹄所投无差;回旋进退,莫不中节。然后舆轮之外可使无余辙,马蹄之外可使无余地;未尝觉山谷之险,原隰之夷,视之一也。吾术穷矣。汝其识之!''

魏黑卵以暱嫌杀丘邴章。丘邴章之子来丹谋报父之仇。丹气甚猛,形甚露,计粒而食,顺风而趋。虽怒,不能称兵以报之。耻假力于人,誓手剑以屠黑卵。黑卵悍志绝众,九抗百夫,节骨皮肉,非人类也。延颈承刀,披胸受矢,铓锷摧屈,而体无痕挞。负其材力,视来丹犹雏鷇也。来丹之友申他曰:``子怨黑卵至矣,黑卵之易子过矣,将奚谋焉?''来丹垂涕曰:``愿子为我谋。''申他曰:`吾闻卫孔周其祖得殷帝之宝剑,一童子服之,却三军之众,奚不请焉?''来丹遂适卫,见孔周,执仆御之礼,请先纳妻子,后言所欲。孔周曰:``吾有三剑,唯子所译;皆不能杀人,且先言其状。一曰含光,视之不可见,运之不知有。其所触也,泯然无际,经物而物不觉。二曰承影,将旦昧爽之交,日夕昏明之际,北面而察之,淡淡焉若有物存,莫识其状。其所触也,窃窃然有声,经物而物不疾也。三曰宵练,方昼则见影而不见光,方夜见光而不见形。其触物也,騞然而过,随过随合,觉疾而不血刃焉。此三宝者,传之十三世矣,而无施于事。匣而藏之,未尝启封,''来丹曰:``虽然,吾必请其下者。''孔周乃归其妻子,与斋七日。晏阴之间,跪而授其下剑,来丹再拜受之以归。来丹遂执剑从黑卵。时黑卵之醉偃于牖下,自颈至腰三斩之。黑卵不觉。来丹以黑卵之死,趣而退。遇黑卵之子于门,击之三下,如投虚。黑卵之子方笑曰:``汝何蚩而三招予?''来丹知剑之不能杀人也,叹而归。黑卵既醒,怒其妻曰:``醉而露我,使人嗌疾而腰急。''其子曰:``畴昔来丹之来。遇我于门,三招我,亦使我体疾而支强,彼其厌我哉!''

周穆王大征西戎,西戎献锟铻之剑,火浣之布。其剑长尺有咫,练钢赤刃,用之切玉如切泥焉。火浣之布,浣之必投于火;布则火色,垢则布色;出火而振之,皓然疑乎雪。皇子以为无此物,传之者妄。萧叔曰:``皇子果于自信,果于诬理哉!''

\hypertarget{header-n97}{%
\subsection{力命}\label{header-n97}}

力谓命曰:``若之功奚若我哉?''命曰:``汝奚功于物,而物欲比朕?''力曰:``寿夭、穷达、贵贱、贫富,我力之所能也。''命曰:``彭祖之智不出尧舜之上,而寿八百;颜渊之才不出众人之下,而寿四八。仲尼之德。不出诸侯之下,而困于陈,蔡;殷纣之行,不出三仁之上,而居君位。季札无爵于吴,田恒专有齐国。夷齐饿于首阳,季氏富于展禽。若是汝力之所能,柰何寿彼而夭此,穷圣而达逆,贱贤而贵愚,贫善而富恶邪?''力曰:``若如若言,我固无功于物,而物若此邪,此则若之所制邪?''命曰:``既谓之命,柰何有制之者邪?朕直而推之,曲而任之。自寿自夭,自穷自达,自贵自贱,自富自贫,朕岂能识之哉?朕岂能识之哉?''

北宫子谓西门子曰:``朕与子并世也,而人子达;并族也,而人子敬;并貌也,而人子爱;并言也,而人子庸;并行也,而人子诚;并仕也,而人子贵;并农也,而人子富;并商也,而人子利。朕衣则裋褐,食则粢粝,居则蓬室,出则徒行。子衣则文锦,食则粱肉,居则连欐,出则结驷。在家熙然有弃朕之心,在朝谔然有敖朕之色。请谒不相及,遨游不同行,固有年矣。子自以德过朕邪?''西门了曰:``予无以知其实。汝造事而穷,予造事而达,此厚薄之验欤?而皆谓与予并,汝之颜厚矣。''北宫子无以应,自失而归。中途遇东郭先生。先生曰:``汝奚往而反,偊偊而步,有深愧之色邪?''北宫子言其状。东郭先生曰:``吾将舍汝之愧,与汝更之西门氏而问之。''曰:``汝奚辱北宫子之深乎?固且言之。''西门子曰:``北宫子言世族、年貌、言行与予并,而贱贵、贫富与予异。予语之曰:`予无以知其实。汝造事而穷,予造事而达,此将厚薄之验欤?而皆谓与予并,汝之颜厚矣。'''东郭先生曰:``汝之言厚薄不过言才德之差,吾之言厚薄异于是矣。夫北宫子厚于德,薄于命;汝厚于命,薄于德。汝之达,非智得也;北宫子之穷,非愚失也。皆天也,非人也。而汝以命厚自矜,北公子以德厚自愧,皆不识夫固然之理矣。''西门子曰:``先生止矣!予不敢复言。''北宫子既归,衣其裋褐,有狐貉之温;进其茙菽,有稻粱之味;庇其蓬室,若广厦之荫;乘其筚辂,若文轩之饰。终身\textless{}辶卣\textgreater{}然,不知荣辱之在彼也,在我也。东郭先生闻之曰:``北宫子之寐久矣,一言而能寤,易悟也哉!''

管夷吾、鲍叔牙二人相友甚戚,同处于齐。管夷吾事公子纠,鲍叔牙事公子小白。齐公族多宠,嫡庶并行。国人惧乱。管仲与召忽奉公子纠奔鲁,鲍叔奉公子小白奔莒。既而公孙无知作乱,齐无君,二公子争入。管夷君与小白战于莒道,射中小白带钩。小白既立,胁鲁杀子纠,召忽死之,管夷吾被囚。鲍叔牙谓桓公曰:``管夷吾能,可以治国。''桓公曰:`我仇也,愿杀之。``鲍叔牙曰:''吾闻贤君无私怨,且人能为其主,亦必能为人君。如欲霸王,非夷吾其弗可。君必舍之!''遂召管仲。鲁归之,齐鲍叔牙郊迎,释其囚。桓公礼之,而位于高国之上,鲍叔牙以身下之,任以国政。号曰仲父。桓公遂霸。管仲尝叹曰:``吾少穷困时,尝与鲍叔贾,分财多自与;鲍叔不以我为贪,知我贫也。吾尝为鲍叔谋事而大穷困,鲍叔不以我为愚,知时有利不利也。吾尝三仕,三见逐于君,鲍叔不以我为肖,知我不遭时也。吾尝三战三北,鲍叔不以我为怯,知我有老母也。公子纠败,召忽死之,吾幽囚受辱;鲍叔不以我为无耻,知我不羞小节而耻名不显于天下也。生我者父母,知我者鲍叔也!''此世称管鲍善交者,小白善用能者。然实无善交,实无用能也。实无善交实无用能者,非更有善交、更有善用能也。召忽非能死,不得不死;鲍叔非能举贤,不是不举;小白非能用仇,不得不用。及管夷吾有病,小白问之,曰:``仲父之病疾矣,可不讳。云,至于大病,则寡人恶乎属国而可?''夷吾曰:``公谁欲欤?''小白曰:``鲍叔牙可。''曰:``不可。其为人也,洁廉善土也,其于不己若者不比之人,一闻人之过,终身不忘。使之理国,上且钩乎君,下且逆乎民。其得罪于君也,将弗久矣。''小白曰:``然则孰可?''对曰:``勿已,则隰朋可。其为人也,上忘而下不叛,愧其不若黄帝,而哀不己若者。以德分人,谓之圣人;以财分人,谓之贤人。以贤临人,未有得人者了;以贤下人者,未有不得人者也。其于国有不闻也,其于家有不见也。勿已,则隰朋可。''然则管夷吾非薄鲍叔也,不得不薄;非厚隰朋也,不得不厚。厚之于始,或薄之于终;薄之于终,或厚之于始。厚薄之去来,弗由我也。

邓析操两可之说,设无穷之辞,当子产执政,作《竹刑》。郑国用之,数难子产之治。子产屈之。子产执而戮之,俄而诛之。然则子产非能用《竹刑》,不得不用;邓析非能屈子产,不得不屈;子产非能诛邓析,不得不诛也。

可以生而生,天福也;可以死而死,天福也。可以生而不生,天罚也;可以死而不死,天罚也。可以生,可以死,得生得死有矣;不可以生,不可以死,或死或生,有矣。然而生生死死,非物非我,皆命也,智之所无柰何。故曰,窈然无际,天道自会,漠然无分,天道自运。天地不能犯,圣智不能干,鬼魅不能欺。自然者,默之成之,平之宁之,将之迎之。

杨朱之友曰季梁。季梁得疾,七日大渐。其子环而泣之,请医。季梁谓杨朱曰:``吾子不肖如此之甚,汝奚不为我歌以晓之?''杨朱歌曰:``天其弗识,人胡能觉?匪祐自天,弗孽由人。我乎汝乎!其弗知乎!医乎巫乎!其知之乎?''其子弗晓,终谒三医。一曰矫氏,二曰俞氏,三曰卢氏,诊其所疾。矫氏谓季梁曰:``汝寒温不节,虚实失度,病由饥饱色欲。精虑烦散,非天非鬼。虽渐,可攻也。''季梁曰:``众医也,亟屏之!''俞氏曰:``女始则胎气不足,乳湩有余。病非一朝一夕之故,其所由来渐矣,弗可已也。''季梁曰:``良医也,且食之!''卢氏曰:``汝疾不由天,亦不由人,亦不由鬼。禀生受形,既有制之者矣,亦有知之者矣,药石其如汝何?''季梁曰:``神医也,重贶遣之!''俄而季梁之疾自瘳。

生非贵之所能存,身非爱之所能厚;生亦非贱之所能夭,身亦非轻之所能薄。故贵之或不生,贱之或不死;爱之或不厚,轻之或不薄。此似反也,非反也;此自生自死,自厚自薄。或贵之而生,或贱之而死;或爱之而厚,或轻之而薄。此似顺也,非顺也;此亦自生自死,自厚自薄。鬻熊语文王曰:``自长非所增,自短非所损。算之所亡若何?''老聃语关尹曰:``天之所恶,孰知其故?''言迎天意,揣利害,不如其已。

杨布问曰:``有人于此,年兄弟也,言兄弟也,才兄弟也,貌兄弟也;而寿夭父子也,贵贱父子也,名誉父子也,爱憎父子也。吾惑之。''杨子曰:``古之人有言,吾尝识之,将以告若。不知所以然而然,命也。今昏昏昧昧,纷纷若若,随所为,随所不为。日去日来,孰能知其故?皆命也。夫信命者,亡寿夭;信理者,亡是非;信心者,亡逆顺;信性者,亡安危。则谓之都亡所信,都亡所不信。真矣悫矣,奚去奚就?奚哀奚乐?奚为奚不为?《黄帝之书》云:`至人居若死,动若械。'亦不知所以居,亦不知所以不居;亦不知所以动,亦不知所以不动。亦不以众人之观易其情貌,亦不谓众人之不观不易其情貌。独往独来,独出独入,孰能碍之?''

墨杘、单至、啴咺、憋懯四人相与游于世,胥如志也;穷年不相知情,自以智之深也。巧佞、愚直、婩斫、便辟四人相与游于世,胥如志也;穷年而不相语术;自以巧之微也。狡犽、情露、瀽极、凌谇四人相与游于世,胥如志也;穷年不相晓悟,自以为才之得也。眠娗、諈诿、勇敢、怯疑四人相与游于世,胥如志也;穷年不相谪发,自以行无戾也。多偶、自专、乘权、支立四人相与游于世,胥如志也;穷年不相顾眄,自以时之适也。此众态也。其貌不一,而咸之于道,命所归也。

佹佹成者,俏成也,初非成也。佹佹败者,俏败者也,初非败也。故迷生于俏,俏之际昧然。于俏而不昧然,则不骇外祸,不喜内福;随时动,随时止,智不能知也。信命者,于彼我无二心。于彼我而有二心者,不若掩目塞耳,背阪面隍,亦不坠仆也。故曰:死生自命也,贫穷自时也。怨夭折者,不知命者也;怨贫穷者,不知时者也。当死不惧,在穷不戚,知命安时也。其使多智之人,量利害,料虚实,度人情,得亦中,亡亦中。其少智之人,不量利害,不料虚实,不度人情,得亦中,亡亦中。量与不量,料与不料,度与不度,奚以异?唯亡所量,亡所不量,则全而亡丧。亦非知全,亦非笑丧。自全也,自亡也,自丧也。

齐景公游于牛山,北临其国城而流涕曰:``美哉国乎!郁郁芊芊,若何滴滴去此国而死乎?使古无死者,寡人将去斯而之何?''史孔梁丘据皆从而泣曰:``臣赖君之赐,疏食恶肉可得而食,怒马棱车,可得而乘也,且犹不欲死,而况吾君乎?''晏子独笑于旁。公雪涕而顾晏子曰:``寡人今日之游悲,孔与据皆从寡人而泣,子之独笑,何也?''晏子对曰:``使贤者常守之,则太公桓公将常守之矣;使有勇者而常守之,则庄公灵公将常守之矣。数君者将守之,吾君方将被蓑笠而立乎畎亩之中,唯事之恤,行假今死乎?则吾君又安得此位而立焉?以其迭处之,迭去之,至于君也,而独为之流涕,是不仁也。见不仁之君,见谄谀之臣;臣见此二者,臣之所为独窃笑也。''景公惭焉,举觞自罚;罚二臣者,各二觞焉。

魏人有东门吴者,其子死而不忧。其相室曰:``公之爱子,天下无有。今子死不忧,何也?''东门吴曰:``吾常无子,无子之时不忧。今子死,乃与向无子同,臣奚忧焉?''

农赴时,商趣利,工追术,仕逐势,势使然也。然农有水旱,商有得失,工有成败,仕有遇否,命使然也。

\hypertarget{header-n113}{%
\subsection{杨朱}\label{header-n113}}

杨朱游于鲁,舍于孟氏。孟氏问曰:``人而已矣,奚以名为?''曰:``以名者为富。''既富矣,奚不已焉?``曰:``为贵''。``既贵矣,奚不已焉?''曰:``为死''。``既死矣,奚为焉?''曰:``为子孙。''``名奚益于子孙?''曰:``名乃苦其身,燋其心。乘其名者,泽及宗族,利兼乡党;况子孙乎?''``凡为名者必廉,廉斯贫;为名者必让,让斯贱。''曰:``管仲之相齐也,君淫亦淫,君奢亦奢,志合言从,道行国霸,死之后,管氏而已。田氏之相齐也,君盈则己降,君敛则己施,民皆归之,因有齐国;子孙享之,至今不绝。''``若实名贫,伪名富。''曰:``实无名,名无实;名者,伪而已矣。昔者尧舜伪以天下让许由善卷,而不失天下,郭祚百年。伯夷叔齐实以孤竹君让,而终亡其国,饿死于首阳之山。实、伪之辩,如此其省也。''

杨朱曰:``百年,寿之大齐。得百年者,千无一焉。设有一者,孩抱以逮昏老,几居其半矣。夜眠之所弭,昼觉之所遣,又几居其半矣。痛疾哀苦,亡失忧惧,又几居其半矣。量十数年之中,逌然而自得,亡介焉之虑者,亦亡一时之中尔。则人之生也奚为哉?奚乐哉?为美厚尔,为声色尔。而美厚复不可常厌足,声色不可常玩闻。乃复为刑赏之所禁劝,名法之所进退;遑遑尔竞一时之虚誉,规死后之余荣;偊々尔慎耳目之观听,惜身意之是非;徒失当年之至乐,不能自肆于一时。重囚累梏,何以异哉?太古之人,知生之暂来,知死之暂往;故从心而动,不违自然所好;当身之娱,非所去也,故不为名所劝。从性而游,不逆万物所好,死后之名,非所取也,故不为刑所及。名誉先后,年命多少,非所量也。''

杨朱曰:``万物所异者生也,所同者死也。生则有贤愚、贵贱,是所异也;死则有臭腐消灭,是所同也。虽然,贤愚、贵贱,非所能也,臭腐、消灭,亦非所能也。故生非所生,死非所死,贤非所贤,愚非所愚,贵非所贵,贱非所贱。然而万物齐生齐死,齐贤齐愚,齐贵齐贱。十年亦死,百年亦死,仁圣亦死,凶愚亦死。生则尧舜,死则腐骨;生则桀纣,死则腐骨。腐骨一矣,熟知其异?且趣当生,奚遑死后?''

杨朱曰:``伯夷非亡欲,矜清之邮,以放饿死。展季非亡情,矜贞之邮,以放寡宗。清贞之误善之若此。''

杨朱曰:``原宪窭于鲁,子贡殖于卫。原宪之窭损生,子贡之殖累身。''``然则窭亦不可,殖亦不可,其可焉在?''曰:``可在乐生,可在逸身。故善乐生者不窭,善逸身者不殖。''

杨朱曰:``古语有之:`生相怜,死相捐。'此语至矣。相怜之道,非唯情也;勤能使逸,饥能使饱,寒能使温,穷能使达也。相捐之道,非不相哀也;不含珠玉,不服文锦,不陈牺牲,不设明器也。''

晏平仲问养生于管夷吾。管夷吾曰:``肆之而已,勿壅勿阏。''晏平仲曰:``其目奈何?''夷吾曰:``恣耳之所欲听,恣目之所欲视,恣鼻之所欲向,恣口之所欲言,恣体之所欲安,恣意之所欲行。夫耳之所欲闻者音声,而不得听,谓之阏聪;目之所欲见者美色,而不得视,谓之阏明;鼻之所欲向者椒兰,而不得嗅,谓之阏颤;口之所欲道者是非,而不得言,谓之阏智;体之所欲安者美厚,而不得从,谓之阏适;意之所为者放逸,而不得行,谓之阏性。凡此诸阏,废虐之主。去废虐之主,熙熙然以俟死,一日、一月,一年、十年,吾所谓养。拘此废虐之主,录而不舍,戚戚然以至久生,百年、千年、万年,非吾所谓养。''管夷吾曰:``吾既告子养生矣,送死奈何?''晏平仲曰:``送死略矣,将何以告焉?''管夷吾曰:``吾固欲闻之。''平仲曰:``既死,岂在我哉?焚之亦可,沈之亦可,瘗之亦可,露之亦可,衣薪而弃诸沟壑亦可,衮衣绣裳而纳诸石椁亦可,唯所遇焉。''管夷吾顾谓鲍叔黄子曰:``生死之道,吾二人进之矣。''

子产相郑,专国之政,三年,善者服其化,恶者畏其禁,郑国以治。诸侯惮之。而有兄曰公孙朝,有弟曰公孙穆。朝好酒,穆好色。朝之室也,聚酒千钟,积麹成封,望门百步,糟浆之气逆于人鼻。方其荒于酒也,不知世道之争危,人理之悔吝,室内之有亡,九族之亲疏,存亡之哀乐也。虽水火兵刃交于前,弗知也。穆之后庭,比房数十,皆择稚齿婑者以盈之。方其耽于色也,屏亲昵,绝交游,逃于后庭,以昼足夜;三月一出,意犹未惬。乡有处子之娥姣者,必贿而招之,媒而挑之,弗获而后已。子产日夜以为戚,密造邓析而谋之,曰:``侨闻治身以及家,治家以及国,此言自于近至于远也。侨为国则治矣,而家则乱矣。其道逆邪?将奚方以救二子?子其诏之!''邓析曰:``吾怪之久矣!未敢先言。子奚不时其治也,喻以性命之重,诱以礼义之尊乎?''了产用邓析之言,因间以谒其兄弟而告之曰:``人之所以贵于禽兽者,智虑。智虑之所将者,礼义。礼义成,则名位至矣。若触情而动,耽于嗜欲,则性命危矣。子纳侨之言,则朝自悔而夕食禄矣。''朝、穆曰:``吾知之久矣,择之亦久矣,岂待若言而后识之哉?凡生之难遇,而死之易及;以难遇之生,俟易及之死,可孰念哉?而欲尊礼义以夸人,矫情性以招名,吾以此为弗若死矣。为欲尽一生之欢,穷当年之乐,唯患腹溢而不得恣口之饮,力惫而不得肆情于色,不遑忧名声之丑,性命之危也。且若以治国之能夸物,欲以说辞乱我之心,荣禄喜我之意,不亦鄙而可怜哉!我又欲与若别之。夫善治外者,物未必治,而身交苦;善治内者,物未必乱,而性交逸。以苦之治外,其法可暂行于一国,未合于人心;以我之治内,可推之于天下,君臣之道息矣。吾常欲以此术而喻之,若反以彼术而教我哉?''子产忙然无以应之。他日以告邓析。邓析曰:``子与真人居而不知也,孰谓子智者乎?郑国之治偶耳,非子之功也。''

卫端木叔者,子贡之世也。藉其先赀,家累万金。不治世故,放意所好。其生民之所欲为,人意之所欲玩者,无不为也,无不玩也。墙屋台榭,园囿池沼,饮食车服,声乐嫔御,拟齐楚之君焉。至其情所欲好,耳所欲听,目所欲视,口所欲尝,虽殊方偏国,非齐土之所产育者,无不必致之;犹藩墙之物也。及其游也,虽山川阻险,途径修远,无不必之,犹人之行咫步也。宾客在庭者日百住,庖厨之下,不绝烟火;堂庑之上,不绝声乐。奉养之余,先散之宗族;宗族之余,次散之邑里;邑里之余,乃散之一国。行年六十,气干将衰,弃其家事,都散其库藏、珍宝、车服、妾媵。一年之中尽焉,不为子孙留财。及其病也,无药石之储;及其死也;无瘗埋之资。一国之人,受其施者,相与赋而藏之,反其子孙之财焉。禽骨厘闻之曰:``端木叔,狂人也,辱其祖矣。''段干生闻之,曰:``端木叔达人也,德过其祖矣。其所行也,其所为也,众意所惊,而诚理所取。卫之君子多以礼教自持,固未足以得此人之心也。''

孟孙阳问杨朱曰:``有人于此,贵生爱身,以蕲不死,可乎?''曰:``理无不死。''``以蕲久生,可乎?''曰:``理无久生。生非贵之所能存,身非爱之所能厚。且久生奚为?五情好恶,古犹今也;四体安危,古犹今也;世事苦乐,古犹今也;变易治乱,古犹今也。既闻之矣,既见之矣,既更之矣,百年犹厌其多,况久生之苦也乎?''孟孙阳曰:`若然,速亡愈于久生;则践锋刃,入汤火,得所志矣。''杨子曰:``不然;既生,则废而任之,究其所欲,以俟于死。将死,则废而任之,究其所之,以放于尽。无不废,无不任,何遽迟速于其间乎?''

杨朱曰:``伯成子高不以一毫利物,舍国而隐耕。大禹不以一身自利,一体偏枯。古之人,损一毫利天下,不与也,悉天下奉一身,不取也。人人不损一毫,人人不利天下,天下治矣。''禽子问杨朱曰:``去子体之一毛,以济一世,汝为之乎?''杨子曰:``世固非一毛之所济。''禽子曰:``假济,为之乎?''杨子弗应。禽子出,语孟孙阳。孟孙阳曰:``子不达夫子之心,吾请言之。有侵苦肌肤获万金者,若为之乎?''曰:``为之。''孟孙阳曰:``有断若一节得一国。子为之乎?''禽子默然有间。孟孙阳曰:``一毛微于肌肤,肌肤微于一节,省矣。然则积一毛以成肌肤,积肌肤以成一节。一毛固一体万分中之一物,奈何轻之乎?''禽子曰:``吾不能所以答子。然则以子之言问老聃、关尹,则子言当矣;以吾言问大禹、墨翟,则吾言当矣。''孟孙阳因顾与其徒说他事。

杨朱曰:``天下之美归之舜、禹、周、孔,天下之恶归之桀纣。然而舜耕于河阳,陶于雷泽,四体不得暂安,口腹不得美厚;父母之所不爱,弟妹之所不亲。行年三十,不告而娶。乃受尧之禅,年已长,智已衰。商钧不才,禅位于禹,戚戚然以至于死:此天人之穷毒者也。鮌治水土,绩用不就,殛诸羽山。禹纂业事仇,惟荒土功,子产不字,过门不入;身体偏枯,手足胼胝。及受舜禅,卑宫室,美绂冕,戚戚然以至于死:此无人之忧苦者也。武王既终,成王幼弱,周公摄天子之政。邵公不悦,四国流言。居东三年,诛兄放弟,仅免其身,戚戚然以至于死:此天人之危惧者也。孔子明帝王之道,应时君之聘,伐树于宋,削迹于卫,穷于商周,围于陈蔡,受屈于季氏,见辱于阳虎,戚戚然以至于死:此天民之遑遽者也。凡彼四圣者,生无一日之欢,死有万世之名。名者,固非实之所取也。虽称之弗知,虽赏之不知,与株块无以异矣。桀藉累世之资,居南面之尊,智足以距群下,威足以震海内;恣耳目之所误,穷意虑之所为,熙熙然从至于死:此天民之逸荡者也。纣亦藉累世之资,居南面之尊;威无不行,志无不从;肆情于倾宫,纵欲于长夜;不以礼义自苦,熙熙然以至于诛:此天民之放纵者也。彼二凶也,生有纵欲之欢,死被愚暴之名。实者,固非名之所与也,虽毁之不知,虽称之弗知,此与株块奚以异矣。彼四圣虽美之所归,苦以至终,同于死矣。彼二凶虽恶之所归,乐以至终,亦同归于死矣。''

杨朱见梁王,言治天下如运诸掌。梁王曰:``先生有一妻一妾,而不能治;三亩之园,而不能芸,而言治天下如运诸掌,何也?''对曰:``君见其牧羊者乎?百羊而群,使五尺童子荷\{竹垂\}而随之,欲东而东,欲西而西。使尧牵一羊,舜荷箠而随之,则不能前矣。且臣闻之:吞舟之鱼,不游枝流;鸿鹄高飞,不集污池。何则?其极远也。黄钟大吕,不可从烦奏之舞,何则?其音疏也。将治大者不治细,成大功者不成小,此之谓矣。''

杨朱曰:``太古之事灭矣,孰志之哉?三皇之事,若存若亡;五帝之事,若觉若梦;三王之事,或隐或显,亿不识一。当身之事,或闻或见,万不识一。目前之事或存或废,千不识一。太古至于今日,年数固不可胜纪。但伏羲已来三十余万岁,贤愚、好丑、成败、是非,无不消灭,但迟速之间耳。矜一时之毁誉,以焦苦其神形,要死后数百年中余名,岂足润枯骨?何生之乐哉?''

杨朱曰:``人肖天地之类,怀五常之性,有生之最灵者也。人者,爪牙不足以供守卫,肌肤不足以自捍御,趋走不足以从利逃害,无毛羽以御寒暑,必将资物以为养,任智而不恃力。故智之所贵,存我为贵;力之所贱,侵物为贱。然身非我有也,既生,不得不全之;物非我有也,既有,不得而去之。身固生之主,物亦养之主。虽全生,不可有其身;虽不去物,不可有其物。有其物有其身,是横私天下之身,横私天下之物。不横私天下文身,不横私天下文物者,其唯圣人乎!公天下之身,公天下之物,其唯至人矣!此之谓至至者也。''

杨朱曰:``生民之不得休息,为四事故:一为寿,二为名,三为位,四为货。有此四者,畏鬼,畏人,畏威,畏刑,此谓之遁民也。可杀可活,制命在外。不逆命,何羡寿?不矜贵,何羡名?不要势,何羡位?不贪富,何羡货?此之谓顺民也。天下无对,制命在内,故语有之曰:人不婚宦,情欲失半;人不衣食,君臣道息。周谚曰:``田父可坐杀。晨出夜入,自以性之恒;啜菽茹藿,自以味之极;肌肉粗厚,筋节\textless{}肉卷\textgreater{}急,一朝处以柔毛绨幕,荐以粱肉兰橘,心\textless{}疒肙\textgreater{}体烦,内热生病矣。商鲁之君与田父侔地,则亦不盈一时而惫矣。故野人之所安,野人之所美,谓天下无过者。昔者宋国有田夫,常衣缊\textless{}麻贲\textgreater{},仅以过冬。暨春东作,自曝于日,不知天下之有广厦隩室,绵纩狐貉。顾谓其妻曰:`负日之暄,人莫知者;以献吾君,将有重赏。'里之富室告之曰:`昔人有美戎菽,甘枲茎芹萍子者,对乡豪称之。乡豪取而尝之,蜇于口,惨于腹,众哂而怨之,其人大惭。子此类也。'''

杨朱曰:``丰屋美服,厚味姣色,有此四者,何求于外?有此而求外者,无厌之性。无厌之性,阴阳之蠹也。忠不足以安君,适足以危身;义不足以利物,适足以害生。安上不由于忠,而忠名灭焉;利物不由于义,而义名绝焉。君臣皆安,物我兼利,古之道也。鬻子曰:`去名者无忧。'老子曰:`名者实之宾。'而悠悠者趋名不已。名固不可去?名固不可宾邪?今有名则尊荣,亡名则卑辱;尊荣则逸乐,卑辱则忧苦。忧苦,犯性者也;逸乐,顺性者也,斯实之所系矣。名胡可去?名胡可宾?但恶夫守名而累实。守名而累实,将恤危亡之不救,岂徒逸乐忧苦之间哉?''

\hypertarget{header-n133}{%
\subsection{说符}\label{header-n133}}

子列子学于壶丘子林。壶丘子林曰:``子知持后,则可言持身矣。''列子曰:``愿闻持后。''曰:``顾若影,则知之。''列子顾而观影:形枉则影曲,形直则影正。然则枉直随形而不在影,屈申任物而不在我,此之谓持后而处先。

关尹谓子列子曰:``言美则响美,言恶则响恶;身长则影长,身短则影短。名也者,响也;身也者,影也。故曰:慎尔言,将有和之;慎尔行,将有随之,是故圣人见出以知入,观往以知来,此其所以先知之理也。度在身,稽在人。人爱我,我必爱之;人恶我,我必恶之。汤武爱天下,故王;桀纣恶天下,故亡,此所稽也。稽度皆明而不道也,譬之出不由门,行不从径也。以是求利,不亦难乎?尝观之《神农有炎》之德,稽之虞、夏、商、周之书,度诸法士贤人之言,所以存亡废兴而非由此道者,未之有也。''

严恢曰:``所为问道者为富,今得珠亦富矣,安用道?''子列子曰:``桀纣唯重利而轻道,是以亡。幸哉余未汝语也!人而无义,唯食而已,是鸡狗也。疆食靡角,胜者为制,是禽兽也。为鸡狗禽兽矣,而欲人之尊己,不可得也。人不尊己,则危辱及之矣。''

列子学射中矣,请于关尹子。尹子曰:``子知子之所以中者乎?''对曰:``弗知也。''关尹子曰:``未可。''退而习之。三年,又以报关尹子。尹子曰:``子知子之所以中乎?''列子曰:``知之矣。''关尹子曰:``可矣;守而勿失也。非独射也,为国与身,亦皆如之。故圣人不察存亡,而察其所以然。''

列子曰:``色盛者骄,力盛者奋,未可以语道也。故不班白语道失,而况行之乎?故自奋则人莫之告。人莫之告,则孤而无辅矣。贤者任人,故年老而不衰,智尽而不乱。故治国之难在于知贤而不在自贤。''

宋人有为其君以玉为楮叶者,三年而成。锋杀茎柯,毫芒繁泽,乱之楮叶中而不可别也。此人遂以巧食宋国。子列子闻之,曰:``使天地之生物,三年而成一叶,则物之叶者寡矣。故圣人恃道化而不恃智巧。''

子列子穷,容貌有饥色。客有言之郑子阳者曰:``列御寇盖有道之士也,居君之国而穷。君无乃为不好士乎?''郑子阳即令官遗之粟。子列子出,见使者,再拜而辞。使者去。子列子入,其妻望之而拊心曰:``妾闻为有道者之妻子,皆得佚乐,今有饥色,君过而遗先生食。先生不受,岂不命也哉?''子列子笑谓之曰:``君非自知我也。以人之言而遗我粟,至其罪我也,又且以人之言,此吾所以不受也。''其卒,民果作难,而杀子阳。

鲁施氏有二子,其一好学,其一好兵。好学者以术干齐侯;齐侯纳之,以为诸公子之傅。好兵者之楚,以法干楚王;王悦之,以为军正。禄富其家,爵荣其亲。施氏之邻人孟氏,同有二子,所业亦同,而窘于贫。羡施氏之有,因从请进趋之方。二子以实告孟氏。孟氏之一子之秦,以术干秦王。秦王曰:``当今诸侯力争,所务兵食而已。若用仁义治吾国,是灭亡之道。''遂宫而放之。其一子之卫,以法干卫侯。卫侯曰:`吾弱国也,而摄乎大国之间。大国吾事之,小国吾抚之,是求安之道。若赖兵权,灭亡可待矣。若全而归之,适于他国。为吾之患不轻矣。''遂刖之,而还诸鲁。既反,孟氏之父子叩胸而让施氏。施氏曰:``凡得时者昌,失时者亡。子道与吾同,而功与吾异,失时者也,非行之谬也。且天下理无常是,事无常非。先日所用,今或弃之;今之所弃,后或用之。此用与不用,无定是非也。投隙抵时,应事无方,属乎智。智苟不足,使若博如孔丘,术如吕尚,焉往而不穷哉?''孟氏父子舍然无愠容,曰:``吾知之矣,子勿重言!''

晋文公出会,欲伐卫,公子锄仰天而笑。公问何笑。曰:``臣笑邻之人有送其妻适私家者,道见桑妇,悦而与言。然顾视其妻,亦有招之者矣。臣窃笑此也。''公寤其言,乃止。引师而还,未至,而有伐其北鄙者矣。

晋国苦盗,有郄雍者,能视盗之貌,察其眉睫之间而得其情。恶侯使视盗,千百无遗一焉。晋侯大喜,告赵文子曰:``吾得一人,而一国盗为尽矣,奚用多为?''文子曰:``吾君恃伺察而得盗,盗不尽矣,且郄雍必不得其死焉。''俄而群盗谋曰:'吾所穷者郄雍也。``遂共盗而残之。晋侯闻而大骇,立召文子而告之曰:``果如子言,郄雍死矣!然取盗何方?''文子曰:``周谚有言:察见渊鱼者不祥,智料隐匿者有殃。且君欲无盗,莫若举贤而任之;使教明于上,化行于下,民有耻心,则何盗之为?''于是用随会知政,而群盗奔秦焉。

孔子自卫反鲁,息驾乎河梁而观焉。有悬水三十仞,圜流九十里,鱼鳖弗能游,鼋鼍弗能居,有一丈夫方将厉之。孔子使人并涯止之,曰:``此悬水三十仞,圜流九十里,鱼鳖弗能游,鼋鼍弗能居也。意者难可以济乎?''丈夫不以错意,遂度而出。孔子问之曰:``巧乎?有道术乎?所以能入而出者,何也?''丈夫对曰:`始吾之入也,先以忠信;及吾之出也,又从以忠信。忠信错吾躯于波流,而吾不敢用私,所以能入而复出者,以此也。''孔子谓弟子曰:``二三子识之!水且犹可以忠信诚身亲之,而况人乎?''

白公问孔子问:``人可与微言乎?''孔子不应。白公问曰:``若以石投水,何如?''孔子曰:``吴之善没者能取之。''曰:``若以水投水何如?''孔子曰:``淄、渑之合,易牙尝而知之。''白公曰:``人故不可与微言乎?''孔子曰:``何为不可?唯知言之谓者乎!夫知言之谓者,不以言言也。争鱼者濡,逐兽者趋,非乐之也。故至言去言,至为无为。夫浅知之所争者,末矣。''白公不得已,遂死于浴室。

赵襄子使新稚穆子攻翟,胜之,取左人中人;使遽人来谒之。襄子方食而有忧色。左右曰:``一朝而两城下,此人之所喜也;今君有忧色,何也?''襄子曰:``夫江河之大也,不过三日;飘风暴雨不终朝,日中不须臾。今赵氏之德行,无所施于积,一朝而两城下,亡其及我哉!''孔子闻之曰:``赵氏其昌乎!夫忧者所以为昌也,喜者所以为亡也。胜非其难者也;持之,其难者也。贤主以此持胜,故其福及后世。齐、楚、吴、越皆尝胜矣,然卒取亡焉,不达乎持胜也。唯有道之主为能持胜。''孔子之劲,能拓国门之关,而不肯以力闻。墨子为守攻,公输般服,而不肯以兵知。故善持胜者以强为弱。

宋人有好行仁义者,三世不懈。家无故黑牛生白犊,以问孔子。孔子曰:``此吉祥也,以荐上帝。''居一年,其父无故而盲,其牛又复生白犊。其父又复令其子问孔子。其子曰:``前问之而失明,又何问乎?''父曰:``圣人之言先迕后合。其事未究,姑复问之。''其子又复问孔子。孔子曰:``吉祥也。''复教以祭。其子归致命。其父曰:``行孔子之言也。''居一年,其子无故而盲。其后楚攻宋,围其城;民易子而食之,析骸而炊之;丁壮者皆乘城而战,死者大半。此人以父子有疾皆免。及围解而疾俱复。

宋有兰子者,以技干宋元。宋元召而使见其技,以双枝长倍其身,属其胫,并趋并驰,弄七剑,迭而跃之,五剑常在空中。元君大惊,立赐金帛。又有兰子又能燕戏者,闻之,复以干元君。元君大怒曰:``昔有异技干寡人者,技无庸,适值寡人有欢心,故赐金帛。彼必闻此而进,复望吾赏。''拘而拟戳之,经月乃放。

秦穆公谓伯乐曰:``子之年长矣,子姓有可使求马者乎?''伯乐对曰:``良马可形容筋骨相也。天下之马者,若灭若没,若亡若失,若此者绝尘弭辙。臣之子皆下才也,可告以良马,不可告以天下之马也。臣有所与共担纆薪菜者,有九方皋,此其于马,非臣之下也。请见之。''穆公见之,使行求马。三月而反,报曰:``已得之矣,在沙丘。''穆公曰:``何马也?''对曰:``牝而黄。''使人往取之,牡而骊。穆公不说,召伯乐而谓之曰:``败矣,子所使求马者!色物、牝牡尚弗能知,又何马之能知也?''伯乐喟然太息曰:``一至于此乎!是乃其所以千万臣而无数者也。若皋之所观,天机也,得其精而忘其粗,在其内而忘其外;见其所见,不见其所不见;视其所视,而遗其所不视。若皋之相者,乃有贵乎马者也。''马至,果天下之马也。

楚庄王问詹何曰:``治国奈何?''詹何对曰:``臣明于治身而不明于治国也。''楚庄王曰:``寡人得奉宗庙社稷,愿学所以守之。''詹何对曰:``臣未尝闻身治而国乱者也,又未尝闻身乱而国治者也。故本在身,不敢对以末。''楚王曰:``善。''

狐丘丈人谓孙叔敖曰:``人有三怨,子知之乎?''孙叔敖曰:``何谓也?''对曰:``爵高者人妒之,官大者主恶之,禄厚者怨逮之。''孙叔敖曰:``吾爵益高,吾志益下;吾官益大,吾心益小;吾禄益厚,吾施益博。以是免于三怨,可乎?''

孙叔敖疾将死,戒其子曰:``王亟封我矣,吾不受也,为我死,王则封汝。汝必无受利地!楚越之间有寝丘者,此地不利而名甚恶。楚人鬼而越人禨,可长有者唯此也。''孙叔敖死,王果以美地封其子。子辞而不受,请寝丘。与之,至今不失。

牛缺者,上地之大儒也,下之邯郸,遇盗于耦沙之中,尽取其衣装车,牛步而去。视之欢然无忧厷之色。盗追而问其故。曰:``君子不以所养害其所养。''盗曰:``嘻!贤矣夫!''既而相谓曰:``以彼之贤,往见赵君。使以我为,必困我。不如杀之。''乃相与追而杀之。燕人闻之,聚族相戒,曰:``遇盗,莫如上地之牛缺也!''皆受教。俄而其弟适秦,至关下,果遇盗;忆其兄之戒,因与盗力争;既而不如,又追而以卑辞请物。盗怒曰:``吾活汝弘矣,而追吾不已,迹将著焉。既为盗矣,仁将焉在?''遂杀之,又傍害其党四五人焉。

虞氏者,梁之富人也,家充殷盛,钱帛无量,财货无訾。登高楼,临大路,设乐陈酒,击博楼上,侠客相随而行,楼上博者射,明琼张中,反两翕鱼而笑。飞鸢适坠其腐鼠而中之。侠客相与言曰:``虞氏富氏之日久矣,而常有轻易人之志。吾不侵犯之,而乃辱我以腐鼠。此而不报,无以立慬于天下。请与若等戮力一志,率徒属,必灭其家为等伦。''皆许诺。至期日之夜,聚众积兵,以攻虞氏,大灭其家。

东方有人焉,曰爰旌目,将有适也,而饿于道。狐父之盗曰丘,见而下壶餐以餔之。爰旌目三餔而后能视,曰:``子何为者也?''曰:``我狐父之人丘也。''爰旌目曰:``譆!汝非盗耶?胡为而食我?吾义不食子之食也。''两手据地而欧之,不出,喀喀然遂伏而死。狐父之人则盗矣,而食非盗也。以人之盗,因谓食为盗而不敢食,是失名实者也。

柱厉叔事莒敖公,自为不知己,去居海上。夏日则食菱芰,冬日则食橡栗。莒敖公有难,柱厉叔辞其友而往死之。其友曰:``子自以为不知己,故去。今往死之,是知与不知无辨也。''柱厉叔曰:``不然;自以为不知,故去。今死,是果不知我也。吾将死之,以丑后世之人主不知其臣者也。''凡知则死之,不知则弗死,此直道而行者也。柱厉叔可谓怼以忘其身者也。

杨朱曰:``利出者实及,怨往者害来。发于此而应于外者唯请,是故贤者慎所出。''

杨子之邻人亡羊,既率其党,又请杨子之竖追之。杨子曰:``嘻!亡一羊何追者之众?''邻人曰:``多歧路。''既反,问:``获羊乎?''曰:``亡之矣。''曰:``奚亡之?''曰:``歧路之中又有歧焉。吾不知所之,所以反也。''杨子戚然变容,不言者移时,不笑者竟日。门人怪之,请曰:``羊贱畜,又非夫子之有,而损言笑者何哉?''杨子不答。门人不获所命。弟子孟孙阳出,以告心都子。心都子他日与孟孙阳偕入,而问曰:`昔有昆弟三人,游齐鲁之间,同师而学,进仁义之道而归。其父曰:`仁义之道若何?'伯曰:`仁义使我爱身而后名。'仲曰:`仁义使我杀身以成名。'叔曰:`仁义使我身名并全。'彼三术相反,而同出于儒。孰是孰非邪?''杨子曰:``人有滨河而居者,习于水,勇于泅,操舟鬻渡,利供百口。裹粮就学者成徒,而溺死者几半。本学泅,不学溺,而利害如此。若以为孰是孰非?''心都子嘿然而出。孟孙阳让之曰:``何吾子问之迂,夫子答之僻?吾惑愈甚。''心都子曰:``大道以多歧亡羊,学者以多方丧生。学非本不同,非本不一,而末异若是。唯归同反一,为亡得丧。子长先生之门,习先生之道,而不达先生之况也,哀哉!''

杨朱之弟曰布,衣素衣而出。天雨,解素衣,衣缁衣而反。其狗不知,迎而吠之。杨而怒,将扑之。杨朱曰:``子无扑矣!子亦犹是也。向者使汝狗白而往,黑而来,岂能无怪哉?''

杨朱曰:``行善不以为名,而名从之;名不与利期,而利归之;利不与争期,而争及之:故君子必慎为善。''

昔人言有知不死之道者,燕君使人受之,不捷,而言者死。燕君甚怒其使者,将加诛焉。幸臣谏曰:``人所忧者莫急乎死,己所重者莫过乎生。彼自丧其生,安能令君不死也?''乃不诛。有齐子亦欲学其道,闻言者之死,乃抚膺而恨。富子闻而笑之曰:``夫所欲学不死,其人已死而犹恨之,是不知所以为学。''胡子曰:``富子之言非也。几人有术不能行者有矣,能行而无其术者亦有矣。卫人有善数者,临死,以诀喻其子。其子志其言而不能行也。他人问之,以其父所言告之。问者用其言而行其术,与其父无差焉。若然,死者奚为不能言生术哉?''

邯郸之民,以正月之旦献鸠于简子,简子大悦,厚赏之。客问其故。简子曰:``正旦放生,示有恩也。''客曰:``民知君之欲放之,故竞而捕之,死者众矣。君如欲生之,不若禁民勿捕。捕而放之,恩过不相补矣。''简子曰:``然。''

齐田氏祖于庭,食客千人。中坐有献鱼雁者,田氏视之,乃叹曰:``天之于民厚矣!殖五谷,生鱼鸟,以为之用。众客和之如响。鲍氏之子年十二,预于次,进曰:``不如君言。天地万物与我并生,类也。类无贵贱,徒以小大智力而相制,迭相食;非相为而生之。人取可食者而食之,岂天本为人生之?且蚊蚋替肤,虎狼食肉,非天本为蚊蚋生人、虎狼生肉者哉?''

齐有贫者,常乞于城市。城市患其亟也,众莫之与。遂适田氏之厩,从马医作役,而假食。郭中人戏之曰:``从马医而食,不以辱乎?''乞儿曰:``天下之辱莫过于乞。乞犹不辱,岂辱马医哉?''

宋人有游于道,得人遗契者,归而藏之,密数其齿。告邻人曰:``吾富可待矣。''

人有枯梧树者,其邻父言枯梧之树不祥。其邻人遽而伐之。邻人父因请以为薪。其人乃不悦,曰:``邻人之父徒欲为薪,而教吾伐之也。与我邻若此,其险岂可哉?''

人有亡鈇者,意者邻之子,视其行步,窃鈇也;颜色,窃鈇也;言语,窃鈇也;作动态度,无为而不窃鈇也。俄而抇其谷而得其鈇,他日复见其邻人之子,动作态度,无似窃鈇者。

白公胜虑乱,罢朝而立,倒仗策,錣上贯颐,血流至地而弗知也。郑人闻之曰:``颐之忘,将何不忘哉?''意之所属著,其行足踬株埳,头抵植木,而不自知也。

昔齐人有欲金者,清旦请冠而之市,适鬻金者之所,因攫其金而去。吏捕得之,问曰:``人皆在焉,子攫人之金何?''对曰:``取金之时,不见人,徒见金。''

\end{document}
