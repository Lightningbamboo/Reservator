\PassOptionsToPackage{unicode=true}{hyperref} % options for packages loaded elsewhere
\PassOptionsToPackage{hyphens}{url}
%
\documentclass[]{article}
\usepackage{lmodern}
\usepackage{amssymb,amsmath}
\usepackage{ifxetex,ifluatex}
\usepackage{fixltx2e} % provides \textsubscript
\ifnum 0\ifxetex 1\fi\ifluatex 1\fi=0 % if pdftex
  \usepackage[T1]{fontenc}
  \usepackage[utf8]{inputenc}
  \usepackage{textcomp} % provides euro and other symbols
\else % if luatex or xelatex
  \usepackage{unicode-math}
  \defaultfontfeatures{Ligatures=TeX,Scale=MatchLowercase}
\fi
% use upquote if available, for straight quotes in verbatim environments
\IfFileExists{upquote.sty}{\usepackage{upquote}}{}
% use microtype if available
\IfFileExists{microtype.sty}{%
\usepackage[]{microtype}
\UseMicrotypeSet[protrusion]{basicmath} % disable protrusion for tt fonts
}{}
\IfFileExists{parskip.sty}{%
\usepackage{parskip}
}{% else
\setlength{\parindent}{0pt}
\setlength{\parskip}{6pt plus 2pt minus 1pt}
}
\usepackage{hyperref}
\hypersetup{
            pdfborder={0 0 0},
            breaklinks=true}
\urlstyle{same}  % don't use monospace font for urls
\setlength{\emergencystretch}{3em}  % prevent overfull lines
\providecommand{\tightlist}{%
  \setlength{\itemsep}{0pt}\setlength{\parskip}{0pt}}
\setcounter{secnumdepth}{0}
% Redefines (sub)paragraphs to behave more like sections
\ifx\paragraph\undefined\else
\let\oldparagraph\paragraph
\renewcommand{\paragraph}[1]{\oldparagraph{#1}\mbox{}}
\fi
\ifx\subparagraph\undefined\else
\let\oldsubparagraph\subparagraph
\renewcommand{\subparagraph}[1]{\oldsubparagraph{#1}\mbox{}}
\fi

% set default figure placement to htbp
\makeatletter
\def\fps@figure{htbp}
\makeatother


\date{}

\begin{document}

\hypertarget{header-n0}{%
\section{管子}\label{header-n0}}

\begin{center}\rule{0.5\linewidth}{\linethickness}\end{center}

\tableofcontents

\begin{center}\rule{0.5\linewidth}{\linethickness}\end{center}

\hypertarget{header-n6}{%
\subsection{牧民}\label{header-n6}}

国颂

凡有地牧民者,务在四时,守在仓廪。国多财,则远者来,地辟举,则民留处;仓廪实,则知礼节;衣食足,则知荣辱;上服度,则六亲固。四维张,则君令行。故省刑之要,在禁文巧,守国之度,在饰四维,顺民之经,在明鬼神,只山川,敬宗庙,恭祖旧。不务天时,则财不生;不务地利,则仓廪不盈;野芜旷,则民乃菅,上无量,则民乃妄。文巧不禁,则民乃淫,不璋两原,则刑乃繁。不明鬼神,则陋民不悟;不只山川,则威令不闻;不敬宗庙,则民乃上校;不恭祖旧,则孝悌不备;四维不张,国乃灭亡。

四维

国有四维,一维绝则倾,二维绝则危,三维绝则覆,四维绝则灭。倾可正也,危可安也,覆可起也,灭不可复错也。何谓四维?一曰礼、二曰义、三曰廉、四曰耻。礼不踰节,义不自进。廉不蔽恶,耻不从枉。故不踰节,则上位安;不自进,则民无巧轴;不蔽恶,则行自全;不从枉,则邪事不生。

四顺

政之所兴,在顺民心。政之所废,在逆民心。民恶忧劳,我佚乐之。民恶贫贱,我富贵之,民恶危坠,我存安之。民恶灭绝,我生育之。能佚乐之,则民为之忧劳。能富贵之,则民为之贫贱。能存安之,则民为之危坠。能生育之,则民为之灭绝。故刑罚不足以畏其意,杀戮不足以服其心。故刑罚繁而意不恐,则令不行矣。杀戮众而心不服,则上位危矣。故从其四欲,则远者自亲;行其四恶,则近者叛之,故知``予之为取者,政之宝也''。

十一经

错国于不倾之地,积于不涸之仓,藏于不竭之府,下令于流水之原,使民于不争之官,明必死之路,开必得之门。不为不可成,不求不可得,不处不可久,不行不可复。错国于不倾之地者,授有德也;积于不涸之仓者,务五谷也;藏于不竭之府者,养桑麻育六畜也;下令于流水之原者,令顺民心也;使民于不争之官者,使各为其所长也;明必死之路者,严刑罚也;开必得之门者,信庆赏也;不为不可成者,量民力也;不求不可得者,不彊民以其所恶也;不处不可久者,不偷取一世也;不行不可复者,不欺其民也;故授有德,则国安;务五谷,则食足;养桑麻,育六畜,则民富;令顺民心,则威令行;使民各为其所长,则用备;严刑罚,则民远邪;信庆赏,则民轻难;量民力,则事无不成;不彊民以其所恶,则轴伪不生;不偷取一世,则民无怨心;不欺其民,则下亲其上。

六亲五法

以家为乡,乡不可为也。以乡为国,国不可为也。以国为天下,天下不可为也。以家为家,以乡为乡,以国为国,以天下为天下。毋曰不同生,远者不听。毋曰不同乡,远者不行。毋曰不同国,远者不从。如地如天,何私何亲?如月如日,唯君之节。御民之辔,在上之所贵。道民之门,在上之所先。召民之路,在上之所好恶。故君求之,则臣得之。君嗜之,则臣食之。君好之,则臣服之。君恶之,则臣匿之。毋蔽汝恶,毋异汝度,贤者将不汝助。言室满室,言堂满堂,是谓圣王。城郭沟渠,不足以固守;兵甲彊力,不足以应敌;博地多财,不足以有众。惟有道者,能备患于未形也,故祸不萌。天下不患无臣,患无君以使之。天下不患无财,患无人以分之。故知时者,可立以为长。无私者,可置以为政。审于时而察于用,而能备官者,可奉以为君也。缓者后于事。吝于财者失所亲,信小人者失士。

\hypertarget{header-n19}{%
\subsection{形势 }\label{header-n19}}

山高而不崩,则祈羊至矣;渊深而不涸,则沈玉极矣,天不变其常,地不易其则,春秋冬夏,不更其节,古今一也。蛟龙得水,而神可立也;虎豹得幽,而威可载也。风雨无乡,而怨怒不及也。贵有以行令,贱有以忘卑,寿夭贫富,无徒归也。

衔命者,君之尊也。受辞者,名之鉉也。上无事,则民自试。抱蜀不言,而庙堂既修。槛鹄锵锵,唯民歌之。济济多士,殷民化之,纣之失也。飞蓬之问,不在所宾;燕雀之集,道行不顾。牺牷圭璧,不足以飨鬼神。主功有素,宝币奚为?羿之道,非射也;造父之术,非驭也;奚仲之巧,非斫削也。召远者使无为焉,亲近者言无事焉,唯夜行者独有也。

平原之隰,奚有于高?大山之隈,奚有于深?訾讆之人,勿与任大。譕臣者可以远举。顾忧者可与致道。其计也速而忧在近者,往而勿召也举长者可远见也;裁大者众之所比也。美人之怀,定服而勿厌也。必得之事,不足赖也;必诺之言,不足信也。小谨者不大立,訾食者不肥体;有无弃之言者,必参于天地也。坠岸三仞,人之所大难也,而猿猱饮焉,故曰伐矜好专,举事之祸也。不行其野,不违其马;能予而无取者,天地之配也。

怠倦者不及,无广者疑神,神者在内,不及者在门,在内者将假,在门者将待。曙戒勿怠,后稚逢殃。朝忘其事,夕失其功。邪气入内,正色乃衰。君不君,则臣不臣。父不父,则子不子。上失其位,则下踰其节。上下不和,令乃不行。衣冠不正,则宾者不肃;进退无仪,则政令不行。且怀且威,则君道备矣。莫乐之,则莫哀之。莫生之,则莫死之。往者不至,来者不极。

道之所言者一也,而用之者异。有闻道而好为家者,一家之人也;有闻道而好为乡者,一乡之人也;有闻道而好为国者,一国之人也;有闻道而好为天下者,天下之人也;有闻道而好定万物者,天下之配也。道往者,其人莫来;道来者,其人莫往;道之所设,身之化也。持满者与天,安危者与人。失天之度,虽满必涸。上下不和,虽安必危。欲王天下,而失天之道,天下不可得而王也。得天之道。其事若自然。失天之道,虽立不安。其道既得,莫知其为之。其功既成,莫知其释之。藏之无刑,天之道也。疑今者,察之古不知来者,视之往,万事之生也,异趣而同归,古今一也。

生栋覆屋。怨怒不及;弱子下瓦,慈母操棰。天道之极,远者自亲。人事之起,近亲造怨。万物之于人也,无私近也,无私远也;巧者有余,而拙者不足;其功顺天者天助之,其功逆天者天违之;天之所助,虽小必大;天之所违,虽成必败;顺天者有其功,逆天者怀其兇,不可复振也。

乌鸟之狡,虽善不亲。不重之结,虽固必解;道之用也,贵其重也。毋与不可,毋彊不能,毋告不知;与不可,彊不能,告不知,谓之劳而无功。见与之交,几于不亲;见哀之役,几于不结;见施之德,几于不报;四方所归,心行者也。独王之国,劳而多祸;独国之君,卑而不威;自媒之女,丑而不信,未之见而亲焉,可以往矣;久而不忘焉,可以来矣。日月不明,天不易也;山高而不见,地不易也。言而不可复者,君不言也;行而不可再者,君不行也。凡言而不可复,行而不可再者,有国者之大禁也。

\hypertarget{header-n29}{%
\subsection{权修 }\label{header-n29}}

万乘之国,兵不可以无主,土地博大,野不可以无吏,百姓殷众,官不可以无长,操民之命,朝不可以无政。

地博而国贫者,野不辟也,民众而兵弱者,民无取也。故末产不禁,则野不辟。赏罚不信,则民无取。野不辟,民无取,外不可以应敌,内不可以固守,故曰有万乘之号,而无千乘之用,而求权之无轻,不可得也。

地辟而国贫者,舟舆饰,台榭广也。赏罚信而兵弱者,轻用众,使民劳也。舟车饰,台榭广,则赋敛厚矣。轻用众,使民劳,则民力竭矣。赋敛厚,则下怨上矣。民力竭,则令不行矣。下怨上,令不行,而求敌之勿谋己,不可得也。

欲为天下者,必重用其国,欲为其国者,必重用其民,欲为其民者,必重尽其民力。无以畜之,则往而不可止也;无以牧之,则处而不可使也;远人至而不去,则有以畜之也。民众而可一,则有以牧之也。见其可也,喜之有征。见其不可也,恶之有刑。赏罚信于其所见,虽其所不见,其敢为之乎?见其可也,喜之无征;见其不可也,恶之无刑;赏罚不信于其所见,而求其所不见之为之化,不可得也。厚爱利,足以亲之。明智礼,足以教之。上身服以先之。审度量以闲之。乡置师以说道之,然后申之以宪令,劝之以庆赏,振之以刑罚,故百姓皆说为善,则暴乱之行无由至矣。

地之生财有时,民之用力有倦,而人君之欲无穷,以有时与有倦,养无穷之君,而度量不生于其间,则上下相疾也。是以臣有杀其君,子有杀其父者矣。故取于民有度,用之有止,国虽小必安;取于民无度,用之不止,国虽大必危。

地之不辟者,非吾地也。民之不牧者,非吾民也。凡牧民者。以其所积者食之。不可不审也。其积多者其食多,其积寡者其食寡,无积者不食。或有积而不食者,则民离上;有积多而食寡者,则民不力;有积寡而食多者,则民多轴;有无积而徒食者,则民偷幸;故离上不力,多轴偷幸,举事不成,应敌不用。故曰:察能授官,班禄赐予,使民之机也。

野与市争民。家与府争货,金与粟争贵,乡与朝争治;故野不积草,农事先也;府不积货,藏于民也;市不成肆,家用足也;朝不合众,乡分治也。故野不积草,府不积货,市不成肆。朝不合众,治之至也。

人情不二,故民情可得而御也。审其所好恶,则其长短可知也;观其交游,则其贤不肖可察也;二者不失,则民能可得而官也。

地之守在城,城之守在兵,兵之守在人,人之守在粟;故地不辟,则城不固。有身不治,奚待于人?有人不治,奚待于家?有家不治,奚待于乡?有乡不治,奚待于国?有国不治,奚待于天下?天下者,国之本也;国者,乡之本也;乡者,家之本也;家者,人之本也;人者,身之本也;身者,治之本也。故上不好本事,则末产不禁;末产不禁,则民缓于时事而轻地利;轻地利,而求田野之辟,仓廪之实,不可得也。

商贾在朝,则货财上流;妇言人事,则赏罚不信;男女无别,则民无廉耻;货财上流,赏罚不信,民无廉耻,而求百姓之安难,兵士之死节,不可得也。朝廷不肃,贵贱不明,长幼不分,度量不审,衣服无等,上下凌节,而求百姓之尊主政令,不可得也。上好轴谋闲欺,臣下赋敛竞得,使民偷壹,则百姓疾怨,而求下之亲上,不可得也。有地不务本事,君国不能壹民,而求宗庙社稷之无危,不可得也。上恃龟筮,好用巫医,则鬼神骤祟;故功之不立,名之不章,为之患者三:有独王者、有贫贱者、有日不足者。

一年之计,莫如树谷;十年之计,莫如树木;终身之计,莫如树人。一树一获者,谷也;一树十获者,木也;一树百获者,人也。我茍种之,如神用之,举事如神,唯王之门。

凡牧民者,使士无邪行,女无淫事。士无邪行,教也。女无淫事,训也。教训成俗,而刑罚省,数也。凡牧民者,欲民之正也;欲民之正,则微邪不可不禁也;微邪者,大邪之所生也;微邪不禁,而求大邪之无伤国,不可得也。

凡牧民者,欲民之有礼也;欲民之有礼,则小礼不可不谨也;小礼不谨于国,而求百姓之行大礼,不可得也。凡牧民者,欲民之有义也;欲民之有义,则小义不可不行;小义不行于国,而求百姓之行大义,不可得也。

凡牧民者,欲民之有廉也;欲民之有廉,则小廉不可不修也;小廉不修于国,而求百姓之行大廉,不可得也。凡牧民者,欲民之有耻也,欲民之有耻,则小耻不可不饰也。小耻不饰于国,而求百姓之行大耻,不可得也。凡牧民者,欲民之修小礼、行小义、饰小廉、谨小耻、禁微邪、此厉民之道也。民之修小礼、行小义、饰小廉、谨小耻、禁微邪、治之本也。

凡牧民者,欲民之可御也;欲民之可御,则法不可不审;法者,将立朝廷者也;将立朝廷者,则爵服不可不贵也;爵服加于不义,则民贱其爵服;民贱其爵服,则人主不尊;人主不尊,则令不行矣。法者,将用民力者也;将用民力者,则禄赏不可不重也;禄赏加于无功,则民轻其禄赏;民轻其禄赏,则上无以劝民;上无以劝民,则令不行矣。法者,将用民能者也;将用民能者,则授官不可不审也;授官不审,则民闲其治;民闲其治,则理不上通;理不上通,则下怨其上;下怨其上,则令不行矣。法者,将用民之死命者也;用民之死命者,则刑罚不可不审;刑罚不审,则有辟就;有辟就,则杀不辜而赦有罪;杀不辜而赦有罪,则国不免于贼臣矣。故夫爵服贱、禄赏轻、民闲其治、贼臣首难,此谓败国之教也。

\hypertarget{header-n47}{%
\subsection{立政}\label{header-n47}}

国之所以治乱者三,杀戮刑罚,不足用也。国之所以安危者四,城郭险阻,不足守也。国之所以富贫者五,轻税租,薄赋敛,不足恃也。治国有三本,而安国有四固,而富国有五事,五事五经也。

三本

君之所审者三:一曰德不当其位;二曰功不当其禄;三曰能不当其官;此三本者,治乱之原也;故国有德义未明于朝者,则不可加以尊位;功力未见于国者,则不可授与重禄;临事不信于民者,则不可使任大官;故德厚而位卑者谓之过;德薄而位尊者谓之失;宁过于君子,而毋失于小人;过于君子,其为怨浅;失于小人,其为祸深;是故国有德义未明于朝而处尊位者,则良臣不进;有功力未见于国而有重禄者,则劳臣不劝;有临事不信于民而任大官者,则材臣不用;三本者审,则下不敢求;三本者不审,则邪臣上通,而便辟制威;如此,则明塞于上,而治壅于下,正道捐弃,而邪事日长。三本者审,则便辟无威于国,道涂无行禽,疏远无蔽狱,孤寡无隐治,故曰:``刑省治寡,朝不合众''。

四固

君之所慎者四:一曰大德不至仁,不可以授国柄。二曰见贤不能让,不可与尊位。三曰罚避亲贵,不可使主兵。四曰不好本事,不务地利,而轻赋敛,不可与都邑。此四务者,安危之本也。故曰:``卿相不得众,国之危也。大臣不和同,国之危也。兵主不足畏,国之危也。民不怀其产,国之危也。''故大德至仁,则操国得众。见贤能让,则大臣和同。罚不避亲贵,则威行于邻敌。好本事,务地利,重赋敛,则民怀其产。

五事

君之所务者五:一曰山泽不救于火,草木不植成,国之贫也。二曰沟渎不遂于隘,鄣水不安其藏,国之贫也。三曰桑麻不植于野,五谷不宜其地,国之贫也。四曰六畜不育于家,瓜瓠荤菜百果不备具,国之贫也。五曰工事竞于刻镂,女事繁于文章,国之贫也。故曰:``山泽救于火,草木植成,国之富也。沟渎遂于隘,鄣水安其藏,国之富也。桑麻植于野,五谷宜其地,国之富也。六畜育于家,瓜瓠荤菜百果备具,国之富也。工事无刻镂,女事无文章,国之富也。''

首宪

分国以为五乡,乡为之师,分乡以为五州,州为之长。分州以为十里,里为之尉。分里以为十游,游为之宗。十家为什,五家为伍,什伍皆有长焉。筑障塞匿,一道路,博出入,审闾闬,慎筦键,筦藏于里尉。置闾有司,以时开闭。闾有司观出入者,以复于里尉。凡出入不时,衣服不中,圈属群徒,不顺于常者,闾有司见之,复无时。

若在长家子弟臣妾属役宾客,则里尉以谯于游宗,游宗以谯于什伍,什伍以谯于长家,谯敬而勿复。一再则宥,三则不赦。凡孝悌忠信、贤良俊材,若在长家子弟臣妾属役宾客,则什伍以复于游宗,游宗以复于里尉。里尉以复于州长。州长以计于乡师。乡师以著于士师。凡过党,其在家属,及于长家。其在长家,及于什伍之长。其在什伍之长,及于游宗。其在游宗,及于里尉。其在里尉,及于州长。其在州长,及于乡师,其在乡师,及于士师。三月一复,六月一计,十二月一著。凡上贤不过等,使能不兼官,罚有罪不独及,赏有功不专与。孟春之朝,君自听朝,论爵赏校官,终五日。季冬之夕,君自听朝,论罚罪刑杀,亦终五日。正月之朔,百吏在朝,君乃出令布宪于国,五乡之师,五属大夫,皆受宪于太史。大朝之日,五乡之师,五属大夫,皆身习宪于君前。太史既布宪,入籍于太府。宪籍分于君前。五乡之师出朝,遂于乡官致于乡属,及于游宗,皆受宪。宪既布,乃反致令焉,然后敢就舍;宪未布,令未致,不敢就舍。就舍,谓之留令。罪死不赦。五属大夫,皆以行车朝,出朝不敢就舍,遂行至都之日。遂于庙致属吏,皆受宪。宪既布,乃发使者致令以布宪之日蚤晏之时,宪既布,使者以发,然后敢就舍;宪未布。使者未发,不敢就舍;就舍,谓之留令,罪死不赦。宪既布,有不行宪者,谓之不从令,罪死不赦。考宪而有不合于太府之籍者,侈曰专制,不足曰亏令,罪死不赦。首宪既布,然后可以布宪。

首事

凡将举事,令必先出,曰事将为。其赏罚之数,必先明之,立事者,谨守令以行赏罚,计事致令,复赏罚之所加,有不合于令之所谓者,虽有功利,则谓之专制,罪死不赦。首事既布,然后可以举事。

省官

修火宪,敬山泽,林薮积草,夫财之所出,以时禁发焉。使民足于宫室之用,薪蒸之所积,虞师之事也,决水潦,通沟渎,修障防,安水藏,使时水虽过度,无害于五谷。岁虽兇旱,有所秎获,司空之事也。相高下,视肥墝,观地宜,明诏期,前后农夫,以时均修焉,使五谷桑麻,皆安其处,由田之事也。行乡里,视宫室,观树艺,简六畜,以时钧修焉。劝勉百姓,使力作毋偷。怀乐家室,重去乡里,乡师之事也。论百工,审时事,辨功苦,上完利,监壹五乡,以时钧修焉。使刻镂文采,毋敢造于乡,工师之事也。

服制

度爵而制服,量禄而用财,饮食有量,衣服有制,宫室有度,六畜人徒有数,舟车陈器有禁,修生则有轩冕服位谷禄田宅之分,死则有棺槨绞衾圹垄之度。虽有贤身贵体,毋其爵,不敢服其服。虽有富家多资,毋其禄,不敢用其财。天子服文有章,而夫人不敢以燕以飨庙,将军大夫不敢以朝官吏,以命士,止于带缘,散民不敢服杂采,百工商贾不得服长鬈貂,刑余戮民不敢服絻,不敢畜连乘车。

九败

寝兵之说胜,则险阻不守;兼爱之说胜,则士卒不战。全生之说胜,则廉耻不立。私议自贵之说胜,则上令不行。群徒比周之说胜,则贤不肖不分。金玉货财之说胜。则爵服下流,观乐玩好之说胜。则奸民在上位。请谒任举之说胜,则绳墨不正,谄谀饰过之说胜,则巧佞者用。

七观

期而致,使而往,百姓舍己以上为心者,教之所期也。始于不足见,终于不可及,一人服之,万人从之,训之所期也。未之令而为,未之使而往,上不加勉,而民自尽,竭俗之所期也。好恶形于心,百姓化于下,罚未行而民畏恐,赏未加而民劝勉,诚信之所期也。为而无害,成而不议,得而莫之能争,天道之所期也。为之而成,求之而得,上之所欲,小大必举,事之所期也。令则行,禁则止,宪之所及,俗之所被,如百体之从心,政之所期也。

\hypertarget{header-n70}{%
\subsection{乘马 }\label{header-n70}}

立国

凡立国都,非于大山之下,必于广川之上;高毋近旱,而水用足;下毋近水,而沟防省;因天材,就地利,故城郭不必中规矩,道路不必中准绳。

大数

无为者帝,为而无以为者王,为而不贵者霸,不自以为所贵,则君道也。贵而不过度,则臣道也。

地政

地者,政之本也。朝者,义之理也。市者,货之准也。黄金者,用之量也。诸侯之地,千乘之国者,器之制也。五者其理可知也,为之有道。地者政之本也,是故地可以正政也,地不平均和调,则政不可正也;政不正,则事不可理也。

阴阳

春秋冬夏,阴阳之推移也。时之短长,阴阳之利用也;日夜之易,阴阳之化也;然则阴阳正矣,虽不正,有余不可损,不足不可益也。天地莫之能损益也。然则可以正政者地也。故不可不正也,正地者,其实必正,长亦正,短亦正;小亦正,大亦正;长短大小尽正。正不正,则官不理;官不理,则事不治;事不治,则货不多;是故何以知货之多也?曰:事治。何以知事之治也?曰:货多。货多事治,则所求于天下者寡矣,为之有道。

爵位

朝者,义之理也。是故爵位正而民不怨;民不怨,则不乱,然后义可理。理不正,则不可以治;而不可不理也,故一国之人,不可以皆贵;皆贵,则事不成而国不利也。为事之不成,国之不利也。使无贵者,则民不能自理也,是故辨于爵列之尊卑,则知先后之序,贵贱之义矣,为之有道。

务市事

市者,货之准也。是故百货贱,则百利不得。百利不得,则百事治。百事治,则百用节矣;是故事者生于虑,成于务,失于傲。不虑则不生,不务则不成,不傲则不失,故曰:市者可以知治乱,可以知多寡,而不能为多寡,为之有道。

黄金

黄金者,用之量也。辨于黄金之理,则知侈俭。知侈俭,则百用节矣,故俭则伤事,侈则伤货;俭则金贱,金贱则事不成,故伤事。

侈则金贵,金贵则货贱,故伤货。货尽而后知不足,是不知量也,事已,而后知货之有余,是不知节也,不知量,不知节不可,为之有道。

诸侯之地千乘之国

诸侯之地,千乘之国者,器之制也。天下乘马服牛,而任之轻重有制,有壹宿之行,道之远近有数矣。是知诸侯之地千乘之国者,所以知地之小大也,所以知任之轻重也;重而后损之,是不知任也;轻而后益之,是不知器也。不知任不知器不可,为之有道。

士农工商

地之不可食者,山之无木者,百而当一。涸泽,百而当一。地之无草木者,百而当一。樊棘杂处,民不得入焉,百而当一。薮,镰缠得入焉,九而当一。蔓山,其木可以为材,可以为轴,斤斧得入焉,九而当一。汎山,其木可以为棺,可以为车,斤斧得入焉,十而当一。流水,网罟得入焉,五而当一。林,其木可以为棺,可以为车,斤斧得入焉,五而当一。泽,网罟得入焉,五而当一。命之曰地均,以实数。方六里,命之曰暴。五暴命之曰部。五部命之曰聚。聚者有市,无市则民乏。五聚命之曰某乡,四乡命之曰方,官制也。官成而立邑。五家而伍,十家而连,五连而暴。五暴而长,命之曰某乡。四乡命之曰都,邑制也,邑成而制事。四聚为一离,五离为一制,五制为一田,二田为一夫,三夫为一家,事制也。事成而制器,方六里,为一乘之地也。一乘者,四马也。一马其甲七,其蔽五。四乘,其甲二十有八,其蔽二十。白徒三十人奉车两,器制也。方六里,一乘之地也。方一里,九夫之田也。黄金一镒,百乘一宿之尽也,无金则用其绢。季绢三十三制当一镒,无绢则用其布。经暴布百两当一镒,一镒之金,食百乘之一宿,则所市之地,六灸一斗,命之曰中,岁有市无市,则民不乏矣。方六里,名之曰社,有邑焉,名之曰央,亦关市之赋。黄金百镒为一箧,其货一谷笼为十箧。其商茍在市者三十人。其正月十二月,黄金一镒,命之曰正。分春曰书比,立夏曰月程,秋曰大稽。与民数得亡。三岁修封,五岁修界。十岁更制,经正也。十仞见水不大潦,五尺见水不大旱,十一仞见水轻征,十分去二三,二则去三四,四则去四,五则去半,比之于山。五尺见水,十分去一,四则去三,三则去二,二则去一,三尺而见水,比之于泽。距国门以外,穷四竟之内,丈夫二犁,童五尺一犁,以为三日之功。正月,令农始作,服于公田农耕,及雪释,耕始焉,芸卒焉。士闻见博,学意察,而不为君臣者,与功而不与分焉。贾知贾之贵贱,日至于市,而不为官贾者,与功而不与分焉。工治容貌功能,日至于市,而不为官工者,与功而不与分焉。不可使而为工,则视货离之实而出夫粟。是故智者知之,愚者不知,不可以教民。巧者能之,拙者不能,不可以教民。非一令而民服之也,不可以为大善。非夫人能之也,不可以为大功;是故非诚贾不得食于贾,非诚工不得食于工,非诚农不得食于农,非信士不得立于朝。是故官虚而莫敢为之请,君有珍车珍甲而莫之敢有。君举事,臣不敢诬其所不能。君知臣,臣亦知君知己也;故臣莫敢不竭力俱操其诚以来。道曰,均地分力,使民知时也,民乃知时日之蚤晏,日月之不足,饥寒之至于身也;是故夜寝蚤起,父子兄弟,不忘其功。为而不倦,民不惮劳苦。故不均之为恶也:地利不可竭,民力不可殚。不告之以时,而民不知;不道之以事,而民不为。与之分货,则民知得正矣,审其分,则民尽力矣,是故不使而父子兄弟不忘其功。

圣人

圣人之所以为圣人者,善分民也。圣人不能分民,则犹百姓也,于己不足,安得名圣。是故有事则用,无事则归之于民,唯圣人为善讬业于民。民之生也,辟则愚,闭则类,上为一。下为二。

失时

时之处事精矣,不可藏而舍也。故曰,今日不为,明日忘货。昔之日已往而不来矣。

地里

上地方八十里,万室之国一,千室之都四;中地方百里,万室之国一,千室之都四。下地方百二十里,万室之国一,千室之都四。以上地方八十里,与下地方百二十里,通于中地方百里。

\hypertarget{header-n98}{%
\subsection{七法}\label{header-n98}}

言是而不能立,言非而不能废;有功而不能赏,有罪而不能诛,若是而能治民者,未之有也。是必立,非必废,有功必赏,有罪必诛,若是安治矣,未也,是何也?曰:形势器械未具,犹之不治也。形势器械具四者备,治矣。不能治其民,而能彊其兵者,未之有也。能治其民矣,而不明于为兵之数,犹之不可。不能彊其兵,而能必胜敌国者,未之有也;能彊其兵,而不明于胜敌国之理,犹之不胜也。兵不必胜敌国,而能正天下者,未之有也。兵必胜敌国矣,而不明正天下之分,犹之不可,故曰:治民有器,为兵有数,胜敌国有理。正天下有分:则、象、法、化、决塞、心术、计数,根天地之气,寒暑之和,水土之性,人民鸟兽草木之生物,虽不甚多,皆均有焉,而未尝变也,谓之则。义也、名也、时也、似也、类也、比也、状也、谓之象。尺寸也、绳墨也、规矩也、衡石也、斗斛也、角量也、谓之法。

七法

渐也、顺也、靡也、久也、服也、习也、谓之化。予夺也、险易也、利害也、难易也、开闭也、杀生也、谓之决塞。实也、诚也、厚也、施也、度也、恕也、谓之心术。刚柔也、轻重也、大小也、实虚也、远近也、多少也、谓之计数。不明于则,而欲出号令,犹立朝夕于鉉均之上,檐竿而欲定其末。不明于象,而欲论材审用,犹绝长以为短,续短以为长。不明于法,而欲治民一众,犹左书而右息之。不明于化,而欲变俗易教,犹朝揉轮而夕欲乘车。不明于决塞,而欲敺众移民,犹使水逆流。不明于心术,而欲行令于人,犹倍招而必拘之。不明于计数,而欲举大事,犹无舟楫而欲经于水险也。故曰:错仪画制,不知则不可。论材审用,不知象不可。和民一众,不知法不可。变俗易教,不知化不可。敺众移民,不知决塞不可。布令必行,不知心术不可。举事必成,不知计数不可。

四伤百匿

百匿伤上威。奸吏伤官法。奸民伤俗教。贼盗伤国众。威伤,则重在下。法伤,则货上流。教伤,则从令者不辑。众伤,则百姓不安其居。重在下,则令不行。货上流,则官徒毁。从令者不辑,则百事无功。百姓不安其居,则轻民处而重民散,轻民处,重民散,则地不辟;地不辟,则六畜不育;六畜不育,则国贫而用不足;国贫而用不足,则兵弱而士不厉;兵弱而士不厉,则战不胜而守不固;战不胜而守不固,则国不安矣。故曰:常令不审,则百匿胜;官爵不审,则奸吏胜;符籍不审,则奸民胜;刑法不审,则盗贼胜;国之四经败,人君泄见危,人君泄,则言实之士不进;言实之士不进,则国之情伪不竭于上。世主所贵者宝也,所亲者戚也,所爱者民也,所重者爵禄也,亡君则不然,致所贵,非宝也,致所亲,非戚也;致所爱,非民也;致所重,非爵禄也,故不为重宝亏其命,故曰:``令贵于宝''。不为爱亲危其社稷,故曰:``社稷戚于亲''。不为爱人枉其法,故曰:``法爱于人''。不为重爵禄分其威,故曰:``威重于爵禄''。不通此四者,则反于无有。故曰:治人如治水潦,养人如养六畜,用人如用草木。居身论道行理,则群臣服教,百吏严断,莫敢开私焉。论功计劳,未尝失法律也。便辟、左右、大族、尊贵、大臣、不得增其功焉。疏远、卑贱、隐不知之人、不忘其劳,故有罪者不怨上,爱赏者无贪心,则列陈之士,皆轻其死而安难,以要上事,本兵之极也。

为兵之数

为兵之数,存乎聚财,而财无敌。存乎论工,而工无敌。存乎制器,而器无敌。存乎选士,而士无敌。存乎政教,而政教无敌。存乎服习,而服习无敌。存乎遍知天下,而遍知天下无敌。存乎明于机数,而明于机数无敌。故兵未出境,而无敌者八;是以欲正天下,财不盖天下,不能正天下;财盖天下,而工不盖天下,不能正天下;工盖天下,而器不盖天下,不能正天下;器盖天下,而士不盖天下,不能正天下;士盖天下,而教不盖天下,不能正天下;教盖天下,而习不盖天下,不能正天下;习盖天下,而不遍知天下,不能正天下;遍知天下,而不明于机数,不能正天下;故明于机数者,用兵之势也。大者时也,小者计也。王道非废也,而天下莫敢窥者,王者之正也。衡库者,天子之礼也。是故器成卒选,则士知胜矣。遍知天下,审御机数,则独行而无敌矣。所爱之国,而独利之;所恶之国,而独害之;则令行禁止,是以圣王贵之。胜一而服百,则天下畏之矣。立少而观多,则天下怀之矣。罚有罪,赏有功,则天下从之矣。故聚天下之精财,论百工之锐器,春秋角试,以练精锐为右;成器不课不用,不试不藏。收天下之豪杰,有天下之骏雄;故举之如飞鸟,动之如雷电,发之如风雨,莫当其前,莫害其后,独出独入,莫敢禁圉。成功立事,必顺于礼义,故不礼不胜天下,不义不胜人;故贤知之君,必立于胜地,故正天下而莫之敢御也。

选阵

若夫曲制时举,不失天时,毋圹地利。其数多少,其要必出于计数。故凡攻伐之为道也,计必先定于内,然后兵出乎境;计未定于内,而兵出乎境,是则战之自胜,攻之自毁也。是故张军而不能战。围邑而不能攻。得地而不能实,三者见一焉。则可破毁也。故不明于敌人之政,不能加也,不明于敌人之情,不可约也。不明于敌人之将,不先军也。不明于敌人之士,不先陈也。是故以众击寡,以治击乱,以富击贫,以能击不能,以教卒练士击敺众白徒。故十战十胜,百战百胜。故事无备,兵无主,则不蚤知。野不辟,地无吏,则无蓄积。官无常,下怨上,而器械不功。朝无政,则赏罚不明。赏罚不明,则民幸生。故蚤知敌人如独行,有蓄积,则久而不匮。器械功,则伐而不费。赏罚明,则人不幸。人不幸,则勇士劝之。故兵也者。审于地图,谋十官。日量蓄积,齐勇士,遍知天下,审御机数,兵主之事也。故有风雨之行,故能不远道里矣。有飞鸟之举,故能不险山河矣。有雷电之战,故能独行而无敌矣。有水旱之功,故能攻国救邑。有金城之守,故能定宗庙,育男女矣。有一体之治,故能出号令,明宪法矣。风雨之行者,速也。飞鸟之举者,轻也。雷电之战者,士不齐也。水旱之功者,野不收,耕不获也。金城之守者,用货财,设耳目也。一体之治者。去奇说。禁雕俗也。不远道里,故能威绝域之民,不险山河,故能服恃固之国。独行无敌,故令行而禁止。故攻国救邑,不恃权与之国,故所指必听。定宗庙,育男女,天下莫之能伤,然后可以有国。制仪法,出号令,莫不向应,然后可以治民一众矣。

\hypertarget{header-n110}{%
\subsection{版法}\label{header-n110}}

凡将立事,正彼天植,风雨无违。远近高下,各得其嗣。三经既饬,君乃有国。喜无以赏,怒无以杀;喜以赏,怒以杀,怨乃起,令乃废,骤令不行,民心乃外。外之有徒,祸乃始牙。众之所忿,置不能图。举所美,必观其所终。废所恶,必计其所穷。庆勉敦敬以显之,富禄有功以劝之,爵贵有名以休之。兼爱无遗,是谓君心。必先顺教,万民乡风。旦暮利之,众乃胜任。取人以己,成事以质。审用财,慎施报,察称量;故用财不可以嗇,用力不可以苦。用财嗇则费,用力苦则劳。民不足,令乃辱。民苦殃,令不行。施报不得,祸乃始昌。祸昌不寤,民乃自图。正法直度,罪杀不赦。杀僇必信,民畏而惧。武威既明,令不再行,顿卒怠倦以辱之,罚罪宥过以惩之,杀僇犯禁以振之。植固不动,倚邪乃恐。倚革邪化,令往民移。法天合德,象法无亲。参于日月,佐于四时。悦在施有,众在废私。召远在修近,闭祸在除怨。修长在乎任贤,高安在乎同利。

\hypertarget{header-n114}{%
\subsection{幼官}\label{header-n114}}

中方本图

若因处虚守静人物,人物则皇。五和时节,君服黄色,味甘味,听宫声,治和气,用五数,饮于黄后之井,以倮兽之火爨,藏温濡,行敺养,坦气修通,凡物开静,形生理。常至命,尊贤授德,则帝。身仁行义,服忠用信,则王。审谋章礼,选士利械,则霸。定生处死,谨贤修伍,则众。信赏审罚,爵材禄能,则强。计凡付终,务本饬末,则富。明法审数,立常备能,则治。同异分官,则安。

通之以道,畜之以惠,亲之以仁,养之以义,报之以德,结之以信,接之以礼,和之以乐,期之以事,攻之以官,发之以力,威之以诚。一举而上下得终,再举而民无不从,三举而地辟散成,四举而农佚粟十,五举而务轻金九,六举而絜知事变,七举而外内为用,八举而胜行威立,九举而帝事成形
。

九本搏大,人主之守也。八分有职,卿相之守也。七官饰胜备威,将军之守也。六纪审密,贤人之守也。五纪不解,庶人之守也。动而无不从,静而无不同。治乱之本三,尊卑之交四,富贫之经五,盛衰之纪六,安危之机七,强弱之应八,存亡之数九。练之以散群傰署。凡数财署,杀僇以聚财,劝勉以选众,使二分具本。发善必审于密,执威必明于中。此居图方中。

中方副图

必得文威武官习,胜之,务时因,胜之。终无方,胜之。几行义,胜之。理名实,胜之。急时分,胜之。事察伐,胜之。行备具,胜之。原无象,胜之。本定独威,胜。定计财,胜。定闻知,胜。定选士,胜。定制禄,胜。定方用,胜。定纶理,胜。定死生,胜。定成败,胜。定依奇,胜。定实虚,胜。定盛衰,胜。举机诚要,则敌不量。用利至诚,则敌不校。明名章实,则士死节。奇举发不意,则士欢用。交物因方,则械器备。因能利备,则求必得。执务明本,则士不偷。备具无常,无方应也。听于钞,故能闻未极。视于新,故能见未形,思于濬,故能知未始。发于惊,故能至无量。动于昌,故能得其宝。立于谋,故能实不可故也。器成教守,则不远道里。号审教施,则不险山河。博一纯固,则独行而无敌。慎号审章,则其攻不待权与。明必胜,则慈者勇。器无方,则愚者智。攻不守,则拙者巧。数也。动慎十号。明审九章。饰习十器。善习五官。谨修三官。必设常主。计必先定。求天下之精材。论百工之锐器。器成,角试否臧。收天下之豪杰,有天下之称材。说行若风雨,发如雷电。此居于图方中。

东方本图

春行冬政,肃。行秋政,雷。行夏政,阉。十二,地气发,戒春事。十二,小卯,出耕。十二,天气下,赐与。十二,义气至,修门闾。十二,清明,发禁。十二,始卯,合男女。十二,中卯。十二,下卯。三卯同事,八举时节。君服青色,味酸味,听角声,治燥气,用八数,饮于青后之井。以羽兽之火爨。藏不忍,行敺养。坦气修通,凡物开静,形生理。合内空周外。强国为圈,弱国为属。动而无不从,静而无不同。举发以礼,时礼必得。和好不基。贵贱无司,事变日至。此居于图东方方外。

东方副图

旗物尚青,兵尚矛。刑则交寒害釱。器成不守,经不知。教习不著,发不意。经不知,故莫之能圉。发不意,故莫之能应。莫之能应,故全胜而无害。莫之能圉,故必胜而无敌。四机不明,不过九日,而游兵惊军。障塞不审,不过八日,而外贼得闲。由守不慎,不过七日,而内有谗谋。诡禁不修,不过六日,而窃盗者起。死亡不食,不过四日,而军财在敌。此居于图东方方外。

南方本图

夏行春政,风。行冬政,落。重则雨雹。行秋政,水。十二,小郢至,德。十二,绝气下,下爵赏。十二,中郢,赐与。十二,中绝,收聚。十二,大暑至,尽善。十二,中暑。十二,小暑终。三暑同事。七举时节,君服赤色,味苦味,听羽声,治阳气,用七数。饮于赤后之井。以毛兽之火爨。藏薄纯,行笃厚,坦气修通,凡物开静,形生理。定府官,明名分,而审责于群臣有司,则下不乘上,贱不乘贵,法立数得,而无比周之民,则上尊而下卑,远近不乖,此居于图南方方外。

南方副图

旗物尚赤。兵尚戟。刑则烧交彊郊。必明其一,必明其将,必明其政,必明其士。四者备,则以治击乱,以成击败。数战则士疲,数胜则君骄,骄君使疲民,则国危。至善不战,其次一之。大胜者积众。胜无非义者,焉可以为大胜。大胜,无不胜也。此居于图南方方外。

西方本图

秋行夏政,叶。行春政,华。行冬政,秏。十二,期风至,戒秋事。十二,小卯,薄百爵。十二,白露下,收聚。十二,复理,赐与。十二,始节赋事。十二,始卯,合男女。十二,中卯。十二,下卯。三卯同事。九和时节,君服白色,味辛味,听商声,治湿气,用九数。饮于白后之井。以介虫之火爨。藏恭敬,行搏锐,坦气修通,凡物开静,形生理。闲男女之畜,修乡闾之什伍。量委积之多寡,定府官之计数。养老弱而勿通,信利周而无私,此居于图西方方外。

西方副图

旗物尚白,兵尚剑。刑则绍昧断绝。始乎无端,卒乎无穷。始乎无端,道也。卒乎无穷,德也。道不可量,德不可数。不可量,则众强不能图。不可数,则为轴不敢乡。两者备施,动静有功。畜之以道,养之以德。畜之以道,则民和,养之以德,则民合。和合故能习;习故能偕。偕习以悉。莫之能伤也。此居于图西方方外。

北方本图

冬行秋政,雾。行夏政,雷。行春政,烝泄。十二,始寒,尽刑。十二,小榆,赐予。十二,中寒,收聚。十二,中榆,大收。十二,寒至,静。十二,大寒,之阴。十二,大寒终三寒同事。六行时节,君服黑色,味咸味,听征声,治阴气,用六数,饮于黑后之井。以鳞兽之火爨。藏慈厚,行薄纯。坦气修通,凡物开静,形生理。

器成于僇,教行于钞。动静不记,行止无量。戒审四时以别息,异出入以两易,明养生以解固,审取予以总之。

一会诸侯,令曰:``非玄帝之命,毋有一日之师役''。再会诸侯,令曰:养孤老,食常疾,收孤寡。三会诸侯,令曰:田租百取五。市赋百取二。关赋百取一。毋乏耕织之器。四会诸侯,令曰:修道路,偕度量,一称数。薮泽以时禁发之。五会诸侯,令曰:修春秋冬夏之常祭,食。天壤山川之故祀,必以时。六会诸侯,令曰:以尔壤生物共玄官,请四辅,将以礼上帝。七会诸侯,令曰:官处四体而无礼者。流之焉莠命。八会诸侯,令曰:立四义而毋议者,尚之于玄官,听于三公。九会诸侯,令曰:以尔封内之财物,国之所有为币。

九会,大命焉出,常至。千里之外,二千里之内。诸侯三年而朝习命。二年,三卿使四辅。一年正月朔日,令大夫来修。受命三公。二千里之外,三千里之内,诸侯五年而会至习命。三年,名卿请事。二年,大夫通吉兇。十年,重适入,正礼义。五年,大夫请受变。三千里之外,诸侯世一至,置大夫以为廷安,入,共受命焉。此居于图北方方外。

北方副图

旗物尚黑,兵尚胁盾。刑则游仰灌流。察数而知治,审器而识胜。明谋而适胜。通德而天下定。定宗庙。育男女。官四分,则可以立威、行、德、制法仪、出号令。至善之为兵也,非地是求也,罚人是君也。立义而加之以胜,至威而实之以德。守之而后修,胜心焚海内。民之所利立之,所害除之,则民人从。立为六千里之侯。则大人从。使国君得其治。则人君从会。请命于天地,知气和,则生物从。计缓急之事。则危危而无难。明于器械之利,则涉难而不变。察于先后之理,则兵出而不困。通于出入之度,则深入而不危。审于动静之务,则功得而无害。著于取与之分,则得地而不执。慎于号令之官。则举事而有功。此居于图北方方外。

\hypertarget{header-n142}{%
\subsection{幼官图}\label{header-n142}}

右中方本图

若因处虚守静,人物则皇。五和时节,君服黄色,味甘味,听宫声,治和气,用五数,饮于黄后之井,以倮兽之火爨。藏漫濡,行驱养,坦气修通。凡物开静,形生理。

常至命,尊贤授德则帝;身仁行义,服忠用信则王;审谋章礼,选士利械则霸;定生处死,谨贤修伍则众;信赏审罚,爵材禄能则强;计凡付终,务本饰末则富;明法审数,立常备能则治;同异分官则安。

通之以道,畜之以惠,亲之以仁,养之以义,报之以德,结之以信,接之以礼,和之以乐,期之以事,攻之以言,发之以力,威之以诚。一举而上下得终,再举而民无不从,三举而地辟谷成,四举而农佚粟十,五举而务轻金九,六举而絜知事变,七举而内外为用,八举而胜行威立,九举而帝事成形。

九本搏大,人主之守也;八分有职,卿相之守也;七胜备威,将军之守也;六纪审密,贤人之守也;五纪不解,庶人之守也;动而无不从,静而无不同。治乱之本三,卑尊之交四,富贫之终五,盛衰之纪六,安危之机七,强弱之应八,存亡之数九。练之以散群傰署,凡数财署。杀僇以聚财,劝勉以迁众,使二分具本。发善必审于密,执威必明于中。

此居图方中。

右中方副图

必得文威武,官习胜之务。时因胜之终,无方胜之几,行义胜之理,名实胜之急,时分胜之事,察伐胜之行,备具胜之原,无象胜之本。定独威胜,定计财胜,定知闻胜,定选士胜,定制禄胜,定方用胜,定纶理胜,定死生胜,定成败胜,定依奇胜,定实虚胜,定盛衰胜。举机诚要,则敌不量;用利至诚,则敌不校。明名章实,则士死节;奇举发不意,则士欢用。交物因方,则械器备;因能利备,则求必得。执务明本,则士不偷;备具无常,无方应也。

听于钞故能闻无极,视于新故能见未形,思于浚故能知未始,发于惊故能至无量,动于昌故能得其宝,立于谋故能实不可故也。器成教守,则不远道里;号审教施,则不险山河;博一纯固,则独行而无敌;慎号审章,则其攻不待权与。明必胜则慈者勇,器无方则愚者智,攻不守则拙者巧,数也。动慎十号,明审九章,饰习十器,善习五教,谨修三官。必设常主,计必先定。求天下之精材,论百工之锐器,器成角试否臧。收天下之豪杰,有天下之称材,说行若风雨,发如雷电。

此居于图方中

右东方本图

春行冬政肃,行秋政霜,行夏政阉。十二地气发,戒春事。十二小卯,出耕。十二天气下,赐与。十二义气至,修门闾。十二清明,发禁。十二始卯,合男女。十二中卯,十二下卯,三卯同事。八举时节,君服青色。味酸味,听角声,治燥气,用八数,饮于青后之井,以羽兽之火爨。藏不忍,行驱养,坦气修通,凡物开静,形生理。

合内空周外,强国为圈,弱国为属。动而无不从,静而无不同。举发以礼,时礼必得。和好不基,贵贱无司,事变日至。

此居于图东方方外。

右东方副图

旗物尚青,兵尚矛,刑则交寒害釱。

器成不守经不知,教习不著发不意。经不知,故莫之能圉;发不意,故莫之能应。莫之能应,故全胜而无害,莫之能圉,故必胜而无敌。

四机不明,不过九日而游兵惊军;障塞不审,不过八日而外贼得间;申守不慎,不过七日而内有谗谋;诡禁不修,不过六日而窃盗者起;死亡不食,不过四日而军财在敌。

此居于图东方方外。

右南方本图

夏行春政风,行冬政落,重则雨雹,行秋政水。十二小郢,至德。十二绝气下,下爵赏。十二中郢,赐与。十二中绝,收聚。十二大暑至,尽善。十二中暑,十二小暑终,三暑同事,七举时节,君服赤色,味苦味,听羽声,治阳气,用七数,饮于赤后之井,以毛兽之火爨。藏薄纯,行笃厚,坦气修通,凡物开静,形生理。

定府官,明名分,而审责于群臣有司,则下不乘上,贱不乘贵。法立数得,而无比周之民,则上尊而下卑,远近不乖。

此居于图南方方外。

右南方副图

旗物尚赤,兵尚戟,刑则烧交疆郊。

必明其情,必明其将,必明其政,必明其士。四者备,则以治击乱,以成击败。数战则士疲,数胜则君骄;骄君使疲民则危国。至善不战,其次一之。

大胜者,积众胜而无非义者焉,可以为大胜。大胜无不胜也。

此居于图南方方外

右西方本图

秋行夏政叶,行春政华,行冬政耗。十二期风至,戒秋事。十二小卯,薄百爵。十二白露下,收聚。十二复理,赐与。十二始节赋事。十二始卯,合男女。十二中卯,十二下卯,三卯同事。九和时节,君服白色,味辛味,听商声,治湿气,用九数,饮于白后之井,以介兽之火爨。藏恭敬,行搏锐,坦气修通,凡物开静,形生理。

间男女之畜,修乡闾之什伍。量委积之多寡,定府官之计数。养老弱而勿通,信利害而无私。

此居于图西方方外。

右西方副图

旗物尚白,兵尚剑,刑则绍味断绝。

始乎无端,卒乎无穷。始乎无端,首也;卒乎无穷,德也。道不可量,德不可数。不可量,则众强不能图;不可数,则为诈不敢乡。两者备施,动静有功。

畜之以道,养之以德。畜之以道则民和,养之以德则民合。和合故能习,习故能偕,借习以悉,莫之能伤也。

此居于图西方方外。

右北方本图

冬行秋政雾。行夏政雷,行春政烝泄。十二始寒,尽刑。十二小榆,赐予。十二中寒,收聚。十二中榆,大收。十二大寒,至静。十二大寒之阴,十二大寒终,三寒同事。六行时节,君服黑色,味咸味,听徵声,汉阴气,用六数,饮于黑后之井,以鳞兽之火爨。藏慈厚,行薄纯,坦气修通,凡物开静,形生理。

器成于僇,教行于钞。动静不记,行止无量。戒四时以别息,异出入以两易,明养生以解固,审取予以总之。一会诸侯令曰:非玄帝之命,毋有一日之师役。再会诸侯令曰:养孤老、食常疾、收孤寡。三会诸侯令曰:田租百取五,市赋百取二,关赋百取一,毋乏耕织之器。四会诸侯令曰:修道路,偕度量,一称数;毋征薮泽以时禁发之。五会诸侯令曰:修春秋冬夏之常祭,食天壤山川之故祀,必以时。六会诸侯令曰:以尔壤生物共玄官,请四辅,将以礼上帝。七会诸侯令曰:官处四体而无礼者,流之焉莠命。八会诸侯令曰:立四义而毋议者,尚之于玄官,听于三公。九会诸侯令曰:以尔封内之财物,国之所有为币。九会大令焉出,常至。千里之外,二千里之内,诸侯三年而朝,习命。二年,三卿使四辅。一年正月朔日,令大夫来修,受命三公。二千里之外,三千里之内,诸侯五年而会至,习命。三年,名卿请事。二年,大夫通吉凶。十年,重适入,正礼义。五年,大夫请受变。三千里之外,诸侯世一至,置大夫以为廷安,入共受命焉。

此居于图北方方外。

右北方副图

旗物尚黑,兵尚胁盾,刑则游仰灌流。

察数而知治,审器而识胜,明谋而适胜,通德而天下定。定宗庙,育男女,官四分,则可以立威行德,制法仪,出号令。至善之为兵也,非地是求也,罚人是君也。立义而加之以胜,至威而实之以德,守之而后修胜,心焚海内。民之所利立之,所害除之,则民人从。立为六千里之侯,则大人从。

使国君得其治,则人君从。会请命于天,地知气和,则生物从。

计缓急之事,则危危而无难。明于器械之利,则涉难而不变。察于先后之理,则兵出而不困。通于出入之度,则深入而不危。审于动静之务,则功得而无害也。著于取与之分,则得地而不执。慎于号令之官,则举事而有功。

此居于图北方方外。

\hypertarget{header-n192}{%
\subsection{五辅 }\label{header-n192}}

古之圣王,所以取明名广誉,厚功大业,显于天下,不忘于后世,非得人者,未之尝闻。暴王之所以失国家,危社稷,覆宗庙,灭于天下,非失人者,未之尝闻。今有士之君,皆处欲安,动欲威,战欲胜,守欲固,大者欲王天下,小者欲霸诸侯。而不务得人,是以小者兵挫而地削,大者身死而国亡,故曰:人不可不务也。此天下之极也。

曰:然则得人之道,莫如利之。利之之道,莫如教之以政,故善为政者,田畴垦而国邑实,朝廷闲而官府治,公法行而私曲止,仓廪实而囹圄空,贤人进而奸民退,其君子上中正而下谄谀。其士民贵武勇而贱得利。其庶人好耕农而恶饮食。于是财用足,而饮食薪菜饶。是故上必宽裕,而有解舍。下必听从,而不疾怨。上下和同,而有礼义,故处安而动威,战胜而守固,是以一战而正诸侯。不能为政者,田畴荒而国邑虚,朝廷兇而官府乱。公法废而私曲行,仓廪虚而囹圄实,贤人退而奸民进,其君子上谄谀而下中正,其士民贵得利而贱武勇,其庶人好饮食而恶耕农,于是财用匮而食饮薪菜乏,上弥残茍,而无解舍,下愈覆鸷而不听从,上下交引而不和同,故处不安而动不威,战不胜而守不固,是以小者兵挫而地削,大者身死而国亡,故以此观之,则政不可不慎也。

德有六兴,义有七体,礼有八经,法有五务,权有三度,所谓六兴者何?曰:辟田畴,利坛宅。修树艺,劝士民,勉稼穑,修墙屋,此谓厚其生。发伏利,输墆积修道途,便关市,慎将宿,此谓输之以财。导水潦,利陂沟,决潘渚,溃泥滞,通郁闭,慎津梁,此谓遗之以利,薄征敛,轻征赋,弛刑罚,赦罪戾,宥小过,此谓宽其政。养长老,慈幼孤,恤鳏寡,问疾病,吊祸丧,此谓匡其急。衣冻寒。食饥渴,匡贫窭,振罢露。资乏绝,此谓振其穷。凡此六者,德之兴也。六者既布,则民之所欲,无不得矣。夫民必得其所欲,然后听上,听上,然后政可善为也,故曰德不可不兴也。

曰:民知德矣,而未知义,然后明行以导之义,义有七体,七体者何?曰:孝悌慈惠,以养亲戚。恭敬忠信,以事君上。中正比宜,以行礼节。整齐撙诎,以辟刑僇。纤嗇省用,以备饥馑。敦懞纯固,以备祸乱。和协辑睦,以备寇戎。凡此七者,义之体也。夫民必知义然后中正,中正然后和调,和调乃能处安,处安然后动威,动威乃可以战胜而守固,故曰义不可不行也。

曰:民知义矣,而未知礼,然后饰八经以导之礼。所谓八经者何?曰:上下有义,贵贱有分,长幼有等贫富有度,凡此八者,礼之经也。故上下无义则乱,贵贱无分则争,长幼无等则倍,贫富无度则失。上下乱,贵贱争,长幼倍,贫富失,而国不乱者,未之尝闻也。是故圣王饬此八礼,以导其民;八者各得其义,则为人君者,中正而无私。为人臣者,忠信而不党。为人父者,慈惠以教。为人子者,孝悌以肃。为人兄者,宽裕以诲。为人弟者,比顺以敬。为人夫者,敦懞以固。为人妻者,劝勉以贞。夫然则下不倍上,臣不杀君,贱不踰贵,少不陵长,远不闲亲,新不闲旧,小不加大,淫不破义,凡此八者,礼之经也。夫人必知礼然后恭敬,恭敬然后尊让,尊让然后少长贵贱不相踰越,少长贵贱不相踰越,故乱不生而患不作,故曰礼不可不谨也。

曰:民知礼矣,而未知务,然后布法以任力,任力有五务,五务者何?曰:君择臣而任官,大夫任官辩事,官长任事守职,士修身功材,庶人耕农树艺。君择臣而任官,则事不烦乱。大夫任官辩事,则举措时。官长任事守职,则动作和。士修身功材,则贤良发。庶人耕农树艺,则财用足。故曰:凡此五者,力之务也。夫民必知务,然后心一,心一然后意专,心一而意专,然后功足观也。故曰:力不可不务也。

曰:民知务矣,而未知权,然后考三度以动之;所谓三度者何?曰:上度之天祥,下度之地宜,中度之人顺,此所谓三度。故曰:天时不祥,则有水旱。地道不宜,则有饥馑。人道不顺,则有祸乱;此三者之来也,政召之。曰:审时以举事,以事动民,以民动国,以国动天下。天下动,然后功名可成也,故民必知权然后举错得。举错得则民和辑,民和辑则功名立矣,故曰:权不可不度也。

故曰五经既布,然后逐奸民,诘轴伪,屏谗慝,而毋听淫辞,毋作淫巧。若民有淫行邪性,树为淫辞,作为淫巧,以上谄君上,而下惑百姓,移国动众,以害民务者,其刑死流,故曰:凡人君之所以内失百姓,外失诸侯,兵挫而地削,名卑而国亏,社稷灭覆,身体危殆,非生于谄淫者未之尝闻也。何以知其然也?曰:淫声谄耳,淫观谄目,耳目之所好谄心,心之所好伤民,民伤而身不危者,未之尝闻也。曰:实圹虚,垦田畴,修墙屋,则国家富。节饮食,撙衣服,则财用足。举贤良,务功劳,布德惠,则贤人进。逐奸人,诘轴伪,去谗慝,则奸人止。修饥馑,救灾害,振罢露,则国家定。

明王之务,在于强本事,去无用,然后民可使富。论贤人,用有能,而民可使治。薄税敛,毋茍于民,待以忠爱,而民可使亲;三者,霸王之事也。事有本而仁义其要也,今工以巧矣,而民不足于备用者,其悦在玩好。农以劳矣,而天下饥者,其悦在珍怪,方丈陈于前。女以巧矣,而天下寒者,其悦在文绣。是故博带梨,大袂列,文绣染,刻镂削,雕琢采。关几而不征,市鄽而不税。是故古之良工,不劳其知巧以为玩好,无用之物,守法者不失。

\hypertarget{header-n204}{%
\subsection{宙合}\label{header-n204}}

左操五音,右执五味,怀绳与准钩,多备规轴,减溜大成,是唯时德之节。春采生,秋采蓏,夏处阴,冬处阳,大贤之德长。明乃哲,哲乃明,奋乃苓,明哲乃大行,毒而无怒,怨而无言,欲而无谋。大揆度仪,若觉卧,若晦明,若敖之在尧也。毋访于佞,毋蓄于谄,毋育于凶,毋监于谗,不正广其荒,不用其区区,鸟飞准绳,讂充末衡,易政利民,毋犯其凶,毋迩其求,而远其忧;高为其居,危颠莫之救。可浅可深,可浮可沉,可曲可直,可言可默。天不一时,地不一利;人不一事,可正而视;定而履,深而□,夫天地一险一易,若鼓之有楟,擿挡则击。天地万物之橐,宙合有橐天地。''左操五音,右执五味,''此言君臣之分也。君出令佚,故立于左。臣任力劳,故立于右。夫五音不同声而能调,此言君之所出令无妄也。而无所不顺,顺而令行政成。五味不同物而能和,此言臣之所任力无也,而无所不得,得而力务财多;故君出令,正其国而无齐其欲,一其爱而无独与是。王施而无私,则海内来宾矣。臣任力,同其忠而无争其利,不失其事而无有其名,分敬而无妒,则夫妇和勉矣。君失音则风律必流,流则乱败。臣离味则百姓不养。百姓不养,则众散亡。臣各能其分,则国宁矣。故名之曰不德。

``怀绳与准钩,多备规轴,减溜大成,是唯时德之节。''夫绳扶拨以为正,准坏险以为平,钩入枉而出直,此言圣君贤佐之制举也。博而不失,因以备能而无遗国犹是国也,民犹是民也,桀纣以乱亡,汤武以治。昌章道以教,明法以期,民之兴善也如此,汤武之功是也。多备规轴者,成轴也。夫成轴之多也,其处大也不究,其入小也不塞。犹□求履之宪也。夫焉有不适善﹖适善,备也,仙也是以无乏。故谕教者取辟焉。天淯阳,无计量,地化生,无法□。所谓是而无非,非而无是,是非有,必交来,苟信是,以有不可先规之,必有不可识虑之,然将卒而不戒,故圣人博闻、多见、畜道、以待物。物至而对形,曲均存矣。减、尽也。溜,发也。言□环毕善,莫不备得,故曰减溜大成。成功之术,必有巨获。必周于德,审于时,时德之遇,事之会也,若合符然,故曰是唯时德之节。

``春采生,秋采蓏,夏处阴,冬处阳'',此言圣人之动静开阖,诎信浧儒,取与之必因于时也。时则动,不时则静,是以古之士有意而未可阳也。故愁其治言,含愁而藏之也。贤人之处乱世也,知道之不可行,则沉抑以辟罚,静默以侔免,辟之也犹夏之就清,冬之就温焉。可以无及于寒暑之灾矣。非为畏死而不忠也,夫强言以为僇,而功泽不加,进伤为人君严之义,退害为人臣者之生,其为不利弥甚。故退身不舍端,修业不息版,以待清明。故微子不与于纣之难,而封于宋,以为殷主,先祖不灭,后世不绝,故曰大贤之德长。

``明乃哲,哲乃明,奋乃苓,明哲乃大行'',此言擅美主盛自奋也,以琅汤凌轹人,人之败也常自此;是故圣人着之简策,传以告后进,曰:``奋盛,苓落也。盛而不落者,昧之有也。''故有道者,不平其称,不满其量,不依其乐,不致其度。爵尊则肃士,禄丰则务施,功大而不伐,业明而不矜。夫名实之相怨久矣,是故绝而无交。惠者知其不可两守,乃取一焉,故安而无忧。

``毒而无怒'',此言止忿速,济没法也。''怨而无言'',言不可不慎也;言不周密,反伤其身。故曰''欲而无谋''。言谋不可以泄,谋泄灾极。夫行忿速,遂没法,贼发。言轻谋泄,灾必及于身;故曰毒而无怒,怨而无言,欲而无谋。

``大揆度仪,若觉卧,若晦明'',言渊色以自诘也,静默以审虑,依贤可用也。仁良既明,通于可不利害之理,循发蒙也。故曰,若觉卧,若晦明,若敖之在尧也。

``毋访于佞'',言毋用佞人也,用佞人,则私多行。``毋蓄于谄'',言毋听谄。听谄则欺上。``毋育于凶'',言毋使暴,使暴则伤民``毋监于谗'',言毋听谗,听则失士。夫行私、欺上、伤民、失士、此四者用,所以害君义失正也。夫为君上者,既失其义正,而倚以为名誉。为臣者不忠而邪,以趋爵禄,乱俗败世,以偷安怀乐,虽广其威,可损也。故曰不正广其荒。是以古之人,阻其路,塞其遂,守而物修,故着之简策,传以告后人曰:其为怨也深,是以威尽焉。

``不用其区区'',者虚也,人而无良焉,故曰虚也。凡坚解而不动,陼堤而不行,其于时必失,失则废而不济。失植之正而不谬,不可贤也。植而无能,不可善也。所贤美于圣人者,以其与变随化也。渊泉而不尽,微约而流施。是以德之流润泽均,加于万物。故曰圣人参于天地。

``鸟飞准绳'',此言大人之义也。夫鸟之飞也,必还山集谷;不还山则因,不集谷则死。山与谷之处也,不必正直,而还山集谷,曲则曲矣,而名绳焉。以为鸟起于北,意南而至于南。起于南,意北而至于北。苟大意得,不以小缺为伤。故圣人美而着之,曰:千里之路,不可扶以绳。万家之都,不可平以准。言大人之行,不必以先帝,常义立之谓贤。故为上者之论其下也,不可以失此术也。

``讂充'',言心也,心欲忠。``末衡'',言耳目也,耳目欲端。中正者,治之本也。耳司听,听必顺闻,闻审谓之聪。目司视,视必顺见。见察谓之明。心司虑,虑必顺言,言得谓之知。聪明以知,则博。博而不惛,所以易政也。政易民利,利乃劝,劝则告。听不顺,不审不聪,不审不聪则缪。视不察不明,不察不明则过。虑不得不知,不得不知则□。缪过以□则忧,忧则所以伎苛,伎苛所以险政,政险民害,害乃怨。怨则凶,故曰:讂充末衡,言易政利民也。

``毋犯其凶'',言中正以蓄慎也。``毋迩其求'',言上之败常,贪于金玉马女,而□爱于粟米货财也。厚藉敛于百姓,则万民怼怨。``远其忧'',言上之亡其国也。常迩其乐,立优美,而外淫于驰骋田腊,内纵于美色淫声,下乃解怠惰失,百吏皆失其端。则烦乱以亡其国家矣。``高为其居。危颠莫之救'',此言尊高满大,而好矜人以丽,主盛处贤,而自予雄也;故盛必失而雄必败。夫上既主盛处贤,以操士民,国家烦乱,万民心怨,此其必亡也,犹自万仞之山播而入深渊,其死而不振也必矣。故曰:毋迩其求,而远其忧,高为其居,危颠莫之救也。

``可浅可深,可沉可浮,可曲可直,可言可默'',此言指意要功之谓也。''天不一时,地不一利,人不一事'',是以着业不得不多,人之名位不得不殊方。明者察于事,故不官于物而旁通于道。道也者,通乎无上,详乎无穷,运乎诸生。是故辨于一言,察于一治,攻于一事者,可以曲说,而不可以广举。圣人由此知言之不可兼也,故博为之治,而计其意。知事之不可兼也,故名为之说,而况其功。岁有春秋冬夏,月有上下中旬,日有朝暮,夜有□晨,半星。辰序各有其司,故曰天不一时。山陵岑岩,渊泉闳流,泉逾瀷而不尽,薄承瀷不满。高下肥硗,物有所宜,故曰地不一利。乡有俗,国有法,食饮不同味,衣服异世用器械,规矩绳准,称量数度,品有所成,故曰人不一事。此各事之仪,其详不可尽也。

``可正而视''言察美恶,审别良苦,不可以不审。操分不杂,故政治不悔。``定而履'',言处其位,行其路,为其事,则民守其职而不乱,故葆统而好终。``深而□'',言明墨章书,道德有常,则后世人人修理而不迷,故名声不息。

``夫天地一险一易,若鼓之有桴,擿挡则击'',言苟有唱之,必有和之,和之不差,因以尽天地之道。景不为曲物直,响不为恶声美。是以圣人明乎物之性者必以其类来也,故君子绳绳乎慎其所先。

``天地万物之橐,宙合有橐天地'',天地苴万物,故曰万物之橐。宙合之意,上通于天之上,下泉于地之下,外出于四海之外,合络天地,以为一裹。散之至于无闲。不可名而山。是大之无外,小之无内,故曰有橐天地,其义不传。一典品之不极一薄,然而典品无治也。多内则富。时出则当。而圣人之道,贵富以当。奚谓当,本乎无妄之治,运乎无方之事,应变不失之谓当。变无不至,无有应当本错不敢忿。故言而名之曰宙合。

\hypertarget{header-n222}{%
\subsection{枢言}\label{header-n222}}

管子曰:``道之在天者日也,其在人者心也。''故曰:``有气则生,无气则死,生者以其气。有名则治,无名则乱,治者以其名。''枢言曰:``爱之利之,益之安之。''四者道之出。

帝王者用之而天下治矣。帝王者,审所先所后,先民与地,则得矣。先贵与骄,则失矣。是故先王慎贵在所先所后。人主不可以不慎贵,不可以不慎民,不可以不慎富,慎贵在举贤,慎民在置官,慎富在务地。故人主之卑尊轻重,在此三者,不可不慎。国有宝有器有用,城郭险阻蓄藏,宝也。圣智,器也。

珠玉,末用也。先王重其宝器,而轻其末用。故能为天下生而不死者二,立而不立者四。喜也者、怒也者、恶也者、欲也者、天下之败也。而贤者宝之,为善者非善也故善无以为也,故先王贵善。王主积于民,霸主积于将战士,衰主积于贵人,亡主积于妇女珠玉,故先王慎其所积。疾之疾之,万物之师也。为之为之,万物之时也。强之强之,万物之指也。凡国有三制,有制人者,有为人之所制者,有不能制人,人亦不能制者。何以知其然,德盛义尊,而不好加名于人,人众兵强,而不以其国造难生患。天下有大事,而好以其国后,如此者,制人者也。

德不盛,义不尊,而好加名于人;人不众,兵不强,而好以其国造难生患;恃与国,幸名利,如此者,人之所制也。人进亦进,人退亦退;人劳亦劳,人佚亦佚,进退劳佚,与人相苟,如此者,不能制人,人亦不能制也。爱人甚而不能利也,憎人甚而不能害也。故先王贵当,贵周。周者不出于口,不见于色,一龙一蛇,一日五化之谓周,故先王不以一过二,先王不独举,不擅功。先王不约束,不结纽,约束则解,结纽则绝。故亲不在约束结纽。先王不货交,不列地,以为天下。天下不可改也,而可以鞭棰使也。时也利也。出为之也。余目不明,余耳不聪。

是以能继天子之容。官职亦然。时者得天,义者得人,既时且义,故能得天与人。先王不以勇猛为边竟,则边竟安。边竟安,则邻国亲。邻国亲,则举当矣。人故相憎也,人之心悍。故为之法。法出于礼,礼出于治,治礼道也,万物待治礼而后定。

凡万物,阴阳两生而参视,先王因其参而慎所入所出。以卑为卑,卑不可得,以尊为尊,尊不可得,桀舜是也,先王之所以最重也。得之必生,失之必死者,何也?唯无得之,尧舜禹汤文武孝己,斯待以成,天下必待以生,故先王重之。一日不食,比岁歉。三日不食,比岁饥。五日不食,比岁荒。七日不食,无国土,十日不食,无畴类尽死矣。先王贵诚信,诚信者,天下之结也。贤大夫不恃宗至,士不恃外权。坦坦之利不以功,坦坦之备不为用。故存国家,定社稷,在卒谋之闲耳。圣人用其心,沌沌乎博而圜,豚豚乎莫得其门,纷纷乎若乱丝,遗遗乎若有从治。故曰:欲知者知之,欲利者利之,欲勇者勇之,欲贵者贵之。彼欲贵,我贵之,人谓我有礼。彼欲勇,我勇之,人谓我恭。彼欲利,我利之,人谓我仁。彼欲知,我知之,人谓我愍,戒之戒之,微而异之。动作必思之,无令人识之,卒来者必备之,信之者仁也,不可欺者智也。既智且仁,是谓成人。贱固事贵,不肖固事贤。贵之所以能成其贵者,以其贵而事贱也,贤之所以能成其贤者,以其贤而事不肖也。恶者美之充也,卑者尊之充也,贱者贵之充也,故先王贵之。天以时使,地以材使,人以德使,鬼神以祥使,禽兽以力使。所谓德者,先之之谓也,故德莫如先,应适莫如后。先王用一阴二阳者霸,尽以阳者王,以一阳二阴者削,尽以阴者亡。量之不以少多称之不以轻重,度之不以短长,不审此三者,不可举大事。能戒乎?能敕乎?能隐而伏乎?能而稷乎?能而麦乎?春不生而夏无得乎,众人之用其心也,爱者憎之始也,德者怨之本也,唯贤者不然。先王事以合交,德以合人,二者不合,则无成矣,无亲矣。凡国之亡也,以其长者也。人之自失也,以其所长者也,故善游者死于梁池,善射者死于中野。命属于食,治属于事。无善事而有善治者,自古及今,未尝之有也。众胜寡,疾胜徐,勇胜怯,智胜愚,善胜恶,有义胜无义,有天道胜无天道,凡此七胜者贵众,用之终身者众矣。人主好佚欲,亡其身失其国者殆。其德不足以怀其民者殆。明其刑而贱其士者殆。

诸侯假之威,久而不知极已者殆。身弥老不知敬其适子者殆。

蓄藏积陈朽腐,不以与人者殆。凡人之名三,有治也者,有耻也者,有事也者。事之名二,正之察之,五者而天下治矣。名正则治,名倚则乱,无名则死,故先王贵名。先王取天下,远者以礼,近者以体,体礼者,所以取天下,远近者,所以殊天下之际。日益之而患少者惟忠,日损之而患多者惟欲。多忠少欲,智也,为人臣者之广道也。为人臣者,非有功劳于国也,家富而国贫,为人臣者之大罪也。为人臣者,非有功劳于国也,爵尊而主卑,为人臣者之大罪也。无功劳于国而贵富者,其唯尚贤乎?众人之用其心也,爱者憎之始也,德者怨之本也。生其事亲也,妻子具,则孝衰矣。其事君也,有好业,家室富足,则行衰矣。爵禄满,则忠衰矣,唯贤者不然,故先王不满也。

人主操逆人臣操顺。先王重荣辱,荣辱在为,天下无私爱也,无私憎也,为善者有福,为不善者有祸,祸福在为,故先王重为。明赏不费明刑不暴,赏罚明,则德之至者也,故先王贵明。

天道大而帝王者用爱恶。爱恶天下可秘,爱恶重闭必固。釜鼓满,则人概之,人满,则天概之,故先王不满也。先王之书,心之敬执也,而众人不知也。故有事事也,毋事亦事也。吾畏事,不欲为事,吾畏言,不欲为言,故行年六十而老吃也。

\hypertarget{header-n235}{%
\subsection{八观 }\label{header-n235}}

大城不可以不完,郭周不可以外通,里域不可以横通,闾闬不可以毋阖,宫垣关闭不可以不修。故大城不完,则乱贼之人谋;郭周外通,则奸遁逾越者作;里域横通,则攘夺窃盗者不止;闾闬无阖,外内交通,则男女无别;宫垣不备,关闭不固,虽有良货,不能守也。故形势不得力非,则奸邪之人悫愿;禁罚威严,则简慢之人整齐;宪令著明,则蛮夷之人不敢犯;赏庆信必,则有功者劝;教训习俗者众,则君民化变而不自知也。是故明君在上位,刑省罚寡,非可刑而不刑,非可罪而不罪也;明君者,闭其门,塞其涂,弇其迹,使民毋由接于淫非之地,是以民之道正行善也若性然。故罪罚寡而民以治矣。

行其田野,视其耕芸,计其农事,而饥饱之国可以知也。其耕之不深,芸之不谨,地宜不任,草田多秽,耕者不必肥,荒者不必墝,以人猥计其野,草田多而辟田少者,虽不水旱,饥国之野也。若是而民寡,则不足以守其地,若是而民众,则国贫民饥。以此遇水旱,则众散而不收;彼民不足以守者,其城不固。民饥者不可以使战。众散而不收,则国为丘墟。故曰:有地君国,而不务耕耘,寄生之君也。故曰:行其田野,视其耕芸,计其农事,而饥饱之国可知也。

行其山泽,观其桑麻,计其六蓄之产,而贫富之国可知也。夫山泽广大,则草木易多也。壤地肥饶,则桑麻易植也。荐草多衍,则六畜易繁也。山泽虽广,草木毋禁,壤地虽肥。桑麻毋数;荐草虽多,六畜有征,闭货之门也。故曰:``时货不遂''。金玉虽多,谓之贫国也。故曰:``行其山泽,观其桑麻,计其六畜之产,而贫富之国可知也。''

入国邑,视宫室,观车马衣服,而侈俭之国可知也。夫国城大而田野浅狭者,其野不足以养其民。城域大而人民寡者,其民不足以守其城。宫营大而室屋寡者,其室不足以实其宫。室屋众而人徒寡者,其人不足以处其室。囷仓寡而台榭繁者,其藏不足以共其费。故曰:``主上无积而宫室美,氓家无积而衣服修,乘车者饰观望,灸行者杂文采,本资少而末用多者,侈国之俗也。''国侈则用费,用费则民贫,民贫则奸智生,奸智生则邪巧作;故奸邪之所生,生于匮不足;匮不足之所生,生于侈;侈之所生,生于毋度;故曰:``审度量,节衣服,俭财用,禁侈泰,为国之急也。''不通于若计者,不可使用国。故曰:``入国邑,视宫室,观车马衣服,而侈俭之国可知也。''

课兇饥,计师役,观台榭,量国费,而实虚之国可知也。凡田野万家之众,可食之地,方五十里,可以为足矣。万家以下,则就山泽可矣。万家以上,则去山泽可矣。彼野悉辟而民无积者,国地小而食地浅也。田半垦而民有余食而粟米多者,国地大而食地博也。国地大而野不辟者,君好货而臣好利者也。辟地广而民不足者,上赋重,流其藏者也,故曰:``粟行于三百里,则国毋一年之积;粟行于四百里,则国毋二年之积;粟行于五百里,则众有饥色;''其稼亡三之一者,命曰小兇。小兇三年而大兇,大兇,则众有大遗苞矣。什一之师,什三毋事,则稼亡三之一。稼亡三之一,而非有故盖积也,则道有损瘠矣。什一之师,三年不解,非有余食也,则民有鬻子矣。故曰:``山林虽近。草木虽美,宫室必有度,禁发必有时,是何也?曰:``大木不可独伐也,大木不可独举也,大木不可独鉉也,大木不可加之薄墙之上。''故曰:``山林虽广,草木虽美,禁发必有时;国虽充盈,金玉虽多,宫室必有度;江海虽广,池泽虽博,鱼鳖虽多,罔罟必有正。''船网不可一财而成也。非私草木爱鱼鳖也,恶废民于生谷也。故曰:``先王之禁山泽之作者,博民于生谷也。''彼民非谷不食,谷非地不生,地非民不动,民非作力毋以致财,天下之所生,生于用力;力之所生,生于劳身,是故主上用财毋已,是民用力毋休也,故曰:``台榭相望者,其上下相怨也''。民毋余积者,其禁不必止,众有遗苞者,其战不必胜。道有损瘠者,其守不必固。故令不必行,禁不必止,战不必胜,守不必固,则危亡随其后矣;故曰:``课兇饥,计师役,观台榭,量国费,实虚之国可知也。''

入州里,观习俗,听民之所以化其上。而治乱之国可知也。州里不鬲,闾闬不设,出入毋时,早晏不禁,则攘夺窃盗,攻击残贼之民,毋自胜矣。食谷水,巷凿井,场容接,树木茂,宫墙毁坏,门户不闭,外内交通,则男女之别毋自正矣。乡毋长游,里毋士舍,时无会同,丧烝不聚,禁罚不严,则齿长辑睦,毋自生矣。故帐礼不谨,则民不修廉,论贤不乡举,则士不及行,货财行于国,则法令毁于官。请谒得于上,则党与成于下。乡官毋法制,百姓群徒不从;此亡国弒君之所自生也。故曰:``入州里,观习俗,听民之所以化其上者,而治乱之国可知也。''

入朝廷,观左右,求本朝之臣,论上下之所贵贱者,而彊弱之国可知也。功多为上,禄赏为下,则积劳之臣,不务尽力。治行为上,爵列为下,则豪桀材臣,不务竭能。便辟左右,不论功能,而有爵禄,则百姓疾怨。非上贱爵轻禄。金玉货财商贾之人,不论志行,而有爵禄也,则上令轻,法制毁。权重之人,不论才能,而得尊位,则民倍本行而求外势。彼积劳之臣,不务尽力。则兵士不战矣。豪桀材人不务竭能,则内治不别矣。百姓疾怨,非上贱爵轻禄,则上毋以劝众矣。上令轻,法制毁,则君毋以使臣,臣毋以事君矣。民倍本行而求外势,则国之情伪竭在敌国矣。故曰:``入朝廷,观左右,求本朝之臣,论上下之所贵贱者,而彊弱之国可知也。''

置法出令,临众用民,计其威严宽惠,行于其民与不行于其民可知也。法虚立而害疏远,令一布而不听者存,贱爵禄而毋功者富,然则众必轻令,而上位危。故曰:``良田不在战士,三年而兵弱。赏罚不信,五年而破。上卖官爵,十年而亡。倍人伦而禽兽行,十年而灭。''战不胜,弱也。地四削,入诸侯,破也。离本国,徙都邑,亡也。有者异姓,灭也。故曰:``置法出令,临众用民,计其威严宽惠,而行于其民不行于其民可知也。''

计敌与,量上意,察国本,观民产之所有余不足,而存亡之国可知也。敌国彊而与国弱,谏臣死而谀臣尊,私情行而公法毁,然则与国不恃其亲,而敌国不畏其彊,豪杰不安其位,而积劳之人不怀其禄。悦商贩而不务本货,则民偷处而不事积聚。豪杰不安其位,则良臣出,积劳之人不怀其禄,则兵士不用。民偷处而不事积聚,则囷仓空虚,如是而君不为变。然则攘夺窃盗,残贼进取之人起矣。内者廷无良臣,兵士不用,囷仓空虚,而外有彊敌之忧,则国居而自毁矣。故曰:``计敌与,量上意,察国本,观民产之所有余不足,而存亡之国可知也。

故以此八者观人主之国,而人主毋所匿其情矣。''

\hypertarget{header-n248}{%
\subsection{法禁 }\label{header-n248}}

法制不议,则民不相私。刑杀毋赦,则民不偷于为善。爵禄毋假。则下不乱其上。三者藏于官则为法,施于国则成俗,其余不彊而治矣。

君壹置则仪,则百官守其法。上明陈其制,则下皆会其度矣。君之置其仪也不一,则下之倍法而立私理者必多矣。是以人用其私,废上之制,而道其所闻,故下与官列法,而上与君分威。国家之危,必自此始矣。昔者圣王之治其民也不然,废上之法制者,必负以耻。财厚博惠,以私亲于民者,正经而自正矣。乱国之道,易国之常,赐赏恣于己者,圣王之禁也。圣王既殁,受之者衰,君人而不能知立君之道,以为国本,则大臣之赘下而射人心者必多矣,君不能审立其法,以为下制。则百姓之立私理而径于利者必众矣。

昔者圣王之治人也,不贵其人博学也,欲其人之和同以听令也。《泰誓》曰:``纣有臣亿万人,亦有亿万之心,武王有臣三千而一心,故纣以亿万之心亡,武王以一心存''。故有国之君,茍不能同人心,一国威,齐士义,通上之治,以为下法,则虽有广地众民,犹不能以为安也。君失其道,则大臣比权重,以相举于国,小臣必循利以相就也。故举国之士,以为亡党,行公道以为私惠。进则相推于君,退则相誉于民,各便其身,而忘社稷。以广其居,聚徒威群。上以蔽君,下以索民。此皆弱君乱国之道也,故国之危也。

乱国之道,易国之常,赐赏恣于己者,圣王之禁也。擅国权以深索于民者,圣王之禁也。其身毋任于上者,圣王之禁也。

进则受禄于君,退则藏禄于室,毋事治职,但力事属,私王官,私君事,去非其人而人私行者,圣王之禁也。

修行则不以亲为本,治事则不以官为主,举毋能、进毋功者,圣王之禁也。

交人则以为己赐,举人则以为己劳,仕人则与分其禄者,圣王之禁也。

交于利通而获于贫穷,轻取于其民而重致于其君,削上以附下,枉法以求于民者,圣王之禁也。

用不称其人,家富于其列,其禄甚寡而资财甚多者,圣王之禁也。

拂世以为行,非上以为名,常反上之法制以成群于国者,圣王之禁也。

饰于贫穷,而发于勤劳,权于贫贱,身无职事,家无常姓,列上下之闲,议言为民者,圣王之禁也。

壶士以为亡资,修田以为亡本,则生之养私不死然后失缫以深与上为市者。圣王之禁也。

审饰小节以示民时言大事以动上远交以踰群,假爵以临朝者,圣王之禁也。

卑身杂处隐行辟倚,侧入迎远,遁上而遁民者,圣王之禁也。

诡俗异礼,大言法行,难其所为,而高自错者,圣王之禁也。

守委闲居,博分以致众。勤身遂行,说人以货财。济人以买誉,其身甚静,而使人求者,圣王之禁也。

行辟而坚言诡而辩,术非而博,顺恶而泽者,圣王之禁也。

以朋党为友,以蔽恶为仁,以数变为智,以重敛为忠,以遂忿为勇者,圣王之禁也。

固国之本,其身务往于上,深附于诸侯者,圣王之禁也。

圣王之身,治世之时,德行必有所是,道义必有所明;故士莫敢诡俗异礼,以自见于国,莫敢布惠缓行,修上下之交,以和亲于民。故莫敢超等踰官,渔利苏功。以取顺其君。圣王之治民也,进则使无由得其所利,退则使无由避其所害,必使反乎安其位,乐其群,务其职,荣其名,而后止矣。故踰其官而离其群者,必使有害。不能其事而失其职者,必使有耻;是故圣王之教民也,以仁错之,以耻使之,修其能,致其所成而止。故曰:``绝而定,静而治,安而尊,举错而不变者,圣王之道也。''

\hypertarget{header-n272}{%
\subsection{重令 }\label{header-n272}}

凡君国之重器,莫重于令。令重则君尊,君尊则国安;令轻则君卑,君卑则国危。故安国在乎尊君,尊君在乎行令,行令在乎严罚。罚严令行,则百吏皆恐;罚不严,令不行,则百吏皆喜。故明君察于治民之本,本莫要于令。故曰:亏令者死,益令者死,不行令者死,留令者死,不从令者死。五者死而无赦,唯令是视。故曰:令重而下恐。

为上者不明,令出虽自上,而论可与不可者在下。夫倍上令以为威,则行恣于己以为私,百吏奚不喜之有?且夫令出虽自上,而论可与不可者在下,是威下系于民也。威下系于民,而求上之毋危,不可得也。令出而留者无罪,则是教民不敬也。令出而不行者毋罪,行之者有罪,是皆教民不听也。令出而论可与不可者在官,是威下分也。益损者毋罪,则是教民邪途也。如此,则巧佞之人,将以此成私为交;比周之人,将以此阿党取与;贪利之人,将以此收货聚财;懦弱之人,将以此阿贵事富便辟;伐矜之人,将以此买誉成名。故令一出,示民邪途五衢,而求上之毋危,下之毋乱,不可得也。

菽粟不足,末生不禁,民必有饥饿之色,而工以雕文刻镂相稚也,谓之逆。布帛不足,衣服毋度,民必有冻寒之伤,而女以美衣锦绣綦组相稚也,谓之逆。万乘藏兵之国,卒不能野战应敌,社稷必有危亡之患,而士以毋分役相稚也,谓之逆。爵人不论能,禄人不论功,则士无为行制死节,而群臣必通外请谒,取权道,行事便辟,以贵富为荣华以相稚也,谓之逆。

朝有经臣,国有经俗,民有经产。何谓朝之经臣?察身能而受官,不诬于上;谨于法令以治,不阿党;竭能尽力而不尚得,犯难离患而不辞死;受禄不过其功,服位不侈其能,不以毋实虚受者,朝之经臣也。何谓国之经俗?所好恶不违于上,所贵贱不逆于令;毋上拂之事,毋下比之说,毋侈泰之养,毋逾等之服;谨于乡里之行,而不逆于本朝之事者,国之经俗也。何谓民之经产?畜长树艺,务时殖谷,力农垦草,禁止末事者,民之经产也。故曰:朝不贵经臣,则便辟得进,毋功虚取;奸邪得行,毋能上通。国不服经俗,则臣下不顺,而上令难行。民不务经产,则仓廪空虚,财用不足。便辟得进,毋功虚取,奸邪得行,毋能上通,则大臣不和。臣下不顺,上令难行,则应难不捷。仓廪空虚,财用不足,则国毋以固守。三者见一焉,则敌国制之矣。

故国不虚重,兵不虚胜,民不虚用,令不虚行。凡国之重也,必待兵之胜也,而国乃重。凡兵之胜也,必待民之用也,而兵乃胜;凡民之用也,必待令之行也,而民乃用。凡令之行也、必待近者之胜也,而令乃行。故禁不胜于亲贵,罚不行于便辟,法禁不诛于严重,而害于疏远,庆赏不施于卑贱,二三而求令之必行,不可得也。能不通于官受,禄赏不当于功,号令逆于民心,动静诡于时变,有功不必赏,有罪不必诛,令焉不必行,禁焉不必止,在上位无以使下,而求民之必用,不可得也。将帅不严威,民心不专一,阵士不死制,卒士不轻敌,而求兵之必胜,不可得也。内守不能完,外攻不能服,野战不能制敌,侵伐不能威四邻,而求国之重,不可得也。德不加于弱小,威不信于强大,征伐不能服天下,而求霸诸侯,不可得也。威有与两立,兵有与分争,德不能怀远国,令不能一诸侯,而求王天下,不可得也。

地大国富,人众兵强,此霸王之本也,然而与危亡为邻矣。天道之数,人心之变。天道之数,至则反,盛则衰。人心之变,有余则骄,骄则缓怠。夫骄者,骄诸侯,骄诸侯者,诸侯失于外;缓怠者,民乱于内。诸侯失于外,民乱于内,天道也。此危亡之时也。若夫地虽大,而不并兼,不攘夺;人虽众,不缓怠,不傲下;国虽富,不侈泰,不纵欲;兵虽强,不轻侮诸侯,动众用兵必为天下政理,此正天下之本而霸王之主也。

凡先王治国之器三,攻而毁之者六。明王能胜其攻,故不益于三者,而自有国、正天下。乱王不能胜其攻,故亦不损于三者,而自有天下而亡。三器者何也?曰:号令也,斧钺也,禄赏也。六攻者何也?曰:亲也,贵也,货也,色也,巧佞也,玩好也。三器之用何也?曰:非号令毋以使下,非斧钺毋以威众,非禄赏毋以劝民。六攻之败何也?曰:虽不听,而可以得存者;虽犯禁,而可以得免者;虽毋功,而可以得富者。凡国有不听而可以得存者,则号令不足以使下;有犯禁而可以得免者,则斧钺不足以威众;有毋功而可以得富者,则禄赏不足以劝民。号令不足以使下,斧钺不足以威众,禄赏不足以劝民,若此,则民毋为自用。民毋为自用,则战不胜;战不胜,而守不固;守不固,则敌国制之矣。然则先王将若之何?曰,不为六者变更于号令,不为六者疑错于斧钺,不为六者益损于禄赏。若此,则远近一心;远近一心,则众寡同力;众寡同力;则战可以必胜,而守可以必固。非以并兼攘夺也,以为天下政治也,此正天下之道也。

\hypertarget{header-n282}{%
\subsection{法法}\label{header-n282}}

不法法,则事毋常;法不法,则令不行。令而不行,则令不法也;法而不行,则修令者不审也;审而不行,则赏罚轻也;重而不行,则赏罚不信也;信而不行,则不以身先之也。故曰:禁胜于身,则令行于民矣。

闻贤而不举,殆;闻善而不索,殆;见能而不使,殆;亲人而不固,殆;同谋而离,殆;危人而不能,殆;废人而复起,殆;可而不为,殆;足而不施,殆;几而不密,殆。人主不周密,则正言直行之士危;正言直行之士危,则人主孤而毋内;人主孤而毋内,则人臣党而成群。使人主孤而毋内、人臣党而成群者,此非人臣之罪也,人主之过也。

民毋重罪,过不大也,民毋大过,上毋赦也。上赦小过,则民多重罪,积之所生也。故曰:赦出则民不敬,惠行则过日益。惠赦加于民,而囹圄虽实,杀戮虽繁,奸不胜矣。故曰:邪莫如蚤禁之。赦过遗善,则民不励。有过不赦,有善不遗,励民之道,于此乎用之矣。故曰:明君者,事断者也。

君有三欲于民,三欲不节,则上位危。三欲者何也?一曰求,二曰禁,三曰令。求必欲得,禁必欲止,令必欲行。求多者,其得寡;禁多者,其止寡;令多者,其行寡。求而不得,则威日损;禁而不止,则刑罚侮;令而不行,则下凌上。故未有能多求而多得者也,未有能多禁而多止者也,未有能多令而多行者也。故曰:上苛则下不听,下不听而强以刑罚,则为人上者众谋矣。为人上而众谋之,虽欲毋危,不可得也。号令已出又易之,礼义已行又止之;度量已制又迁之,刑法已错又移之。如是,则庆赏虽重,民不劝也;杀戮虽繁,民不畏也。故曰:上无固植,下有疑心。国无常经,民力必竭,数也。

明君在上位,民毋敢立私议自贵者,国毋怪严,毋杂俗,毋异礼,士毋私议。倨傲易令,错仪画制,作议者尽诛。故强者折,锐者挫,坚者破。引之以绳墨,绳之以诛僇,故万民之心皆服而从上,推之而往,引之而来。彼下有立其私议自贵,分争而退者,则令自此不行矣。故曰:私议立则主道卑矣。况主倨傲易令,错仪画制,变易风俗,诡服殊说犹立。上不行君令,下不合于乡里,变更自为,易国之成俗者,命之曰不牧之民。不牧之民,绳之外也;绳之外诛。使贤者食于能,斗士食于功。贤者食于能,则上尊而民从;斗士食于功,则卒轻患而傲敌。上尊而民从,卒轻患而傲敌。二者设于国,则天下治而主安矣。

凡赦者,小利而大害者也,故久而不胜其祸。毋赦者,小害而大利者也,故久而不胜其福。故赦者,奔马之委辔;毋赦者,痤雎之矿石也。爵不尊、禄不重者,不与图难犯危,以其道为未可以求之也。是故先王制轩冕所以著贵贱,不求其美;设爵禄所以守其服,不求其观也。使君子食于道,小人食于力。君子食于道,则上尊而民顺;小人食于力,则财厚而养足。上尊而民顺,财厚而养足,四者备体,则胥足上尊时而王不难矣。文有三侑,武毋一赦。惠者,多赦者也,先易而后难,久而不胜其祸:法者,先难而后易,久而不胜其福。故惠者,民之仇雠也;法者,民之父母也。太上以制制度,其次失而能追之,虽有过,亦不甚矣。

明君制宗庙,足以设宾祀,不求其美;为宫室台榭,足以避燥湿寒暑,不求其大;为雕文刻镂,足以辨贵贱,不求其观。故农夫不失其时,百工不失其功,商无废利,民无游日,财无砥墆。故曰:俭其道乎!

令未布而民或为之,而赏从之,则是上妄予也。上妄予,则功臣怨;功臣怨,而愚民操事于妄作;愚民操事于妄作,则大乱之本也。令未布而罚及之,则是上妄诛也。上妄诛,则民轻生;民轻生,则暴人兴、曹党起而乱贼作矣。令已布而赏不从,则是使民不劝勉、不行制、不死节。民不劝勉、不行制、不死节,则战不胜而守不固;战不胜而守不固,则国不安矣。令已布而罚不及,则是教民不听。民不听,则强者立;强者立,则主位危矣。故曰:宪律制度必法道,号令必著明,赏罚必信密,此正民之经也。

凡大国之君尊,小国之君卑。大国之君所以尊者,何也?曰:为之用者众也。小国之君所以卑者,何也?曰:为之用者寡也。然则为之用者众则尊,为之用者寡则卑,则人主安能不欲民之众为己用也?使民众为己用,奈何?曰:法立令行,则民之用者众矣;法不立,令不行,则民之用者寡矣。故法之所立、令之所行者多,而所废者寡,则民不诽议;民不诽议,则听从矣。法之所立,令之所行,与其所废者钧,则国毋常经;国毋常经,则民妄行矣。法之所立、令之所行者寡,而所废者多,则民不听;民不听,则暴人起而奸邪作矣。

计上之所以爱民者,为用之爱之也。为爱民之故,不难毁法亏令,则是失所谓爱民矣。夫以爱民用民,则民之不用明矣。夫至用民者,杀之危之,劳之苦之,饥之渴之;用民者将致之此极也,而民毋可与虑害己者,明王在上,道法行于国,民皆舍所好而行所恶。故善用民者,轩冕不下拟,而斧钺不上因。如是,则贤者劝而暴人止。贤者劝而暴人止,则功名立其后矣。蹈白刃,受矢石,入水火,以听上令;上令尽行,禁尽止。引而使之,民不敢转其力;推而战之,民不敢爱其死。不敢转其力,然后有功;不敢爱其死,然后无敌。进无敌,退有功,是以三军之众皆得保其首领,父母妻子完安于内。故民未尝可与虑始,而可与乐成功。是故仁者、知者、有道者,不与大虑始。

国无以小与不幸而削亡者,必主与大臣之德行失于身也,官职、法制、政教失于国也,诸侯之谋虑失于外也,故地削而国危矣。国无以大与幸而有功名者,必主与大臣之德行得于身也。官职、法制、政教得于国也,诸侯之谋虑得于外也。然后功立而名成。然则国何可无道?人何可无求?得道而导之,得贤而使之,将有所大期于兴利除害。期于兴利除害莫急于身,而君独甚。伤也,必先令之失。人主失令而蔽,已蔽而劫,已劫而弑。

凡人君之所以为君者,势也。故人君失势,则臣制之矣。势在下,则君制于臣矣;势在上,则臣制于君矣。故君臣之易位,势在下也。在臣期年,臣虽不忠,君不能夺也;在子期年,子虽不孝,父不能服也。故《春秋》之记,臣有弑其君、子有弑其父者矣。故曰:堂上远于百里,堂下远于千里,门庭远于万里。今步者一日,百里之情通矣;堂上有事,十日而君不闻,此所谓远于百里也。步者十日,千里之情通矣;堂下有事,一月而君不闻,此所谓远于千里也。步者百日,万里之情通矣,门庭有事,期年而君不闻,此所谓远于万里也。故请入而不出谓之灭,出而不入谓之绝,入而不至谓之侵,出而道止谓之壅。灭绝侵壅之君者,非杜其门而守其户也、为政之有所不行也。故曰:令重于宝,社稷先于亲戚,法重于民,威权贵于爵禄。故不为重宝轻号令,不为亲戚后社稷,不为爱民枉法律,不为爵禄分威权。故曰:势非所以予人也。

政者,正也。正也者,所以正定万物之命也。是故圣人精德立中以生正,明正以治国。故正者,所以止过而逮不及也。过与不及也,皆非正也;非正,则伤国一也。勇而不义伤兵,仁而不法伤正。故军之败也,生于不义;法之侵也,生于不正。故言有辨而非务者,行有难而非善者。故言必中务,不苟为辩;行必思善,不苟为难。

规矩者,方圜之正也。虽有巧目利手,不如拙规矩之正方圜也。故巧者能生规矩,不能废规矩而正方圜。虽圣人能生法,不能废法而治国。故虽有明智高行,倍法而治,是废规矩而正方圜也。

一曰:凡人君之德行威严,非独能尽贤于人也;曰人君也,故从而贵之,不敢论其德行之高卑有故。为其杀生,急于司命也;富人贫人,使人相畜也;良人贱人,使人相臣也。人主操此六者以畜其臣,人臣亦望此六者以事其君,君臣之会,六者谓之谋。六者在臣期年,臣不忠,君不能夺;在子期年,子不孝,父不能夺。故《春秋》之记,臣有弑其君,子有弑其父者,得此六者,而君父不智也。六位在臣,则主蔽矣;主蔽者,失其令也。故曰:令入而不出谓之蔽,令出而不入谓之壅,令出而不行谓之牵,令入而不至谓之瑕。牵瑕蔽壅之事君者,非敢杜其门而守其户也,为令之有所不行也。此其所以然者,在贤人不至而忠臣不用也。故人主不可以不慎其令。令者,人主之大宝也。

一曰:贤人不至谓之蔽,忠臣不用谓之塞,令而不行谓之障,禁而不止谓之逆。蔽塞障逆之君者,不敢杜其门而守其户也,为贤者之不至、令之不行也。

凡民从上也,不从口之所言,从情之所好者也;上好勇,则民轻死;上好仁,则民轻财。故上之所好,民必甚焉。是故明君知民之必以上为心也,故置法以自治,立仪以自正也。故上不行,则民不从;彼民不服法死制,则国必乱矣。是以有道之君,行法修制,先民服也。

凡论人有要:矜物之人,无大士焉。彼矜者,满也;满者,虚也。满虚在物,在物为制也。矜者,细之属也。凡论人而远古者,无高士焉。既不知古而易其功者,无智土焉。德行成于身而远古,卑人也。事无资,遇时而简其业者,愚士也。钓名之人,无贤士焉。钓利之君,无王主焉。贤人之行其身也,忘其有名也;王主之行其道也,忘其成功也。贤人之行,王主之道,其所不能已也。

明君公国一民以听于世,忠臣直进以论其能。明君不以禄爵私所爱,忠臣不诬能以干爵禄。君不私国,臣不诬能,行此道者,虽未大治,正民之经也。今以诬能之臣事私国之君,而能济功名者,古今无之。诬能之人易知也。臣度之先王者,舜之有天下也,禹为司空,契为司徒,皋陶为李,后稷为田。此四士者,天下之贤人也,犹尚精一德以事其君。今诬能之人,服事任官,皆兼四贤之能。自此观之,功名之不立,亦易知也。故列尊禄重,无以不受也;势利官大,无以不从也;以此事君,此所谓诬能篡利之臣者也。世无公国之君,则无直进之士;无论能之主,则无成功之臣。昔者三代之相授也,安得二天下而杀之。

贫民伤财,莫大于兵;危国忧主,莫速于兵。此四患者明矣,古今莫之能废也。兵当废而不废,则古今惑也;此二者不废而欲废之,则亦惑也。此二者伤国一也。黄帝唐虞,帝之隆也,资有天下,制在一人。当此之时也,兵不废。今德不及三帝,天下不顺,而求废兵,不亦难乎?故明君知所擅,知所患。国治而民务积,此所谓擅也。动与静,此所患也。是故明君审其所擅,以备其所患也。

猛毅之君,不免于外难;懦弱之君,不免于内乱。猛毅之君者轻诛,轻诛之流,道正者不安;道正者不安、则材能之臣去亡矣。彼智者知吾情伪,为敌谋我,则外难自是至矣。故曰:猛毅之君,不免于外难。懦弱之君者重诛,重诛之过,行邪者不革;行邪者久而不革,则群臣比周;群臣比周,则蔽美扬恶;蔽美扬恶,则内乱自是起。故曰:懦弱之君,不免于内乱。

明君不为亲戚危其社稷,社稷戚于亲;不为君欲变其令、令尊于君;不为重宝分其威,威贵于宝;不为爱民亏其法,法爱于民。

\hypertarget{header-n307}{%
\subsection{兵法}\label{header-n307}}

明一者皇,察道者帝,通德者王,谋得兵胜者霸。故夫兵,虽非备道至德也,然而所以辅王成霸。今代之用兵者不然,不知兵权者也。故举兵之日而境内贫,战不必胜,胜则多死,得地而国败。此四者,用兵之祸者也。四祸其国而无不危矣。

大度之书曰:举兵之日而境内不贫,战而必胜,胜而不死,得地而国不败。为此四者若何?举兵之日而境内不贫者,计数得也。战而必胜者,法度审也。胜而不死者,教器备利,而敌不敢校也。得地而国不败者,因其民也。因其民,则号制有发也。教器备利,则有制也。法度审,则有守也。计数得,则有明也。治众有数,胜敌有理。察数而知理,审器而识胜,明理而胜敌。定宗庙,遂男女,官四分,则可以定威德;制法仪,出号令,然后可以一众治民。

兵无主,则不蚤知敌。野无吏,则无蓄积。官无常,则下怨上,器械不巧。则朝无定,赏罚不明,则民轻其产。故曰:蚤知敌,则独行;有蓄积,则久而不匮;器械巧,则伐而不费:赏罚明,则勇士劝也。

三官不缪,五教不乱,九章著明,则危危而无害,穷穷而无难。故能致远以数,纵强以制。三官:一曰鼓棗鼓所以任也,所以起也,所以进也;二曰金棗金所以坐也,所以退也,所以免也;三曰旗棗旗所以立兵也,所以利兵也,所以偃兵也。此之谓三官。有三令,而兵法治也。五教:一曰教其目以形色之旗,二曰教其身以号令之数,三曰教其足以进退之度,四曰教其手以长短之利,五曰教其心以赏罚之诚。五教各习,而士负以勇矣。九章:一曰举日章,则昼行;二曰举月章,则夜行;三曰举龙章,则行水;四曰举虎章,则行林;五曰举鸟章,则行陂;六曰举蛇章,则行泽;七曰举鹊章,则行陆;八曰举狼章,则行山;九曰举韟章,则载食而驾。九章既定,而动静不过。

三官、五教、九章,始乎无端,卒乎无穷。始乎无端者,道也;卒乎无穷者,德也。道不可量,德不可数也。故不可量,则众强不能图;不可数,则伪诈不敢向。两者备施,则动静有功。径乎不知,发乎不意。径乎不知,故莫之能御也;发乎不意,故莫之能应也。故全胜而无害。因便而教,准利而行。教无常,行无常。两者备施,动乃有功。

器成教施,追亡逐遁若飘风,击刺若雷电。绝地不守,恃固不拔,中处而无敌,令行而不留。器成教施,散之无方,聚之不可计。教器备利,进退若雷电,而无所疑匮。一气专定,则傍通而不疑;厉士利械,则涉难而不匮。进无所疑,退无所匮,敌乃为用。凌山坑,不待钩梯;历水谷,不须舟辑。径于绝地,攻于恃固,独出独入而莫之能止。宝不独入,故莫之能止;宝不独见,故莫之能敛。无名之至,尽尽而不意。故不能疑神。

畜之以道,则民和;养之以德,则民合。和合故能谐,谐故能辑,谐辑以悉,莫之能伤。定一至,行二要,纵三权,施四教,发五机,设六行,论七数,守八应,审九器,章十号。故能全胜大胜。

无守也,故能守胜。数战则士罢,数胜则君骄,夫以骄君使罢民,则国安得无危?故至善不战,其次一之。破大胜强,一之至也。乱之不以变,乘之不以诡,胜之不以诈,一之实也。近则用实,远则施号;力不可量,强不可度,气不可极,德不可测,一之原也。众若时雨,寡若飘风,一之终也。

利適,器之至也;用敌,教之尽也。不能致器者,不能利適;不能尽教者,不能用敌。不能用敌者穷,不能致器者困。远用兵,则可以必胜。出入异涂,则伤其敌。深入危之,则士自修;士自修,则同心同力。善者之为兵也,使敌若据虚,若搏景。无设无形焉,无不可以成也;无形无为焉,无不可以化也,此之谓道矣。若亡而存,若后而先,威不足以命之。

\hypertarget{header-n319}{%
\subsection{匡君大匡}\label{header-n319}}

齐僖公生公子诸儿、公子纠、公子小白。使鲍叔傅小白,鲍叔辞,称疾不出。管仲与召忽往见之,曰:``何故不出?''鲍叔曰:``先人有言曰:`知子奠若父,知臣莫若君。'今君知臣不肖也,是以使贱臣傅小白也。贱臣知弃矣。''召忽曰:``子固辞,无出,吾权任子以死亡,必免子。''鲍叔曰:``子如是,何不免之有乎?''管仲曰:``不可。持社稷宗庙者,不让事,不广闲。将有国者未可知也。子其出乎。''召忽曰:``不可。吾三人者之于齐国也,譬之犹鼎之有足也,去一焉,则必不立矣。吾观小白,必不为后矣。''管仲曰,``不然也。夫国人憎恶纠之母,以及纠之身,而怜小白之无母也。诸儿长而贱,事未可知也。夫所以定齐国者,非此二公子者,将无已也。小白之为人无小智,惕而有大虑,非夷吾莫容小白。天下不幸降祸加殃于齐,纠虽得立,事将不济,非子定社稷,其将谁也?''召忽曰:``百岁之后:吾君卜世,犯吾君命,而废吾所立,夺吾纠也,虽得天下,吾不生也。兄与我齐国之政也,受君令而不改,奉所立而不济,是吾义也。''管仲曰:``夷吾之为君臣也,将承君命,奉社稷,以持宗庙,岂死一纠哉?夷吾之所死者,社稷破,宗庙灭,祭祀绝,则夷吾死之;非此三者,则夷吾生。夷吾生,则齐国利;夷吾死,则齐国不利。''鲍叔曰:``然则奈何?''管子曰:``子出奉令则可。''鲍叔许诺。乃出奉令,邀傅小白。鲍叔谓管仲曰:``何行?''管仲曰;``为人臣者,不尽力于君则不亲信,不亲信则言不听,言不听则社稷不定。大事君者无二心。''鲍叔许诺。

僖公之母弟夷仲年生公孙无知,有宠于僖公,衣服札秩如適。僖公卒,以诸儿长,得为君,是为襄公。襄公立后,绌无知,无知怒。公令连称、管至父戍葵丘曰:``瓜时而往,及瓜时而来。''期戍,公问不至,请代,不许,故二人因公孙无知以作乱。

鲁桓公夫人文姜,齐女也。公将如齐,与夫人皆行。申俞谏曰:``不可,女有家,男有室,无相渎也,谓之有礼。''公不听,遂以文姜会齐侯于泺。文姜通于齐侯,桓公闻,责文姜。文姜告齐侯,齐侯怒,飨公,使公子彭生乘鲁侯胁之,公薨于车。竖曼曰:``贤者死忠以振疑,百姓寓焉;智者究理而长虑,身得免焉。今彭生二于君,无尽言。而谀行以戏我君,使我君失亲戚之礼命,又力成吾君之祸,以构二国之怨,彭生其得免乎?祸理属焉。夫君以怒遂祸,不畏恶亲闻容,昏生无丑也。岂及彭生而能止之哉?鲁若有诛,必以彭生为说。''二月,鲁人告齐曰:``寡君畏君之威,不敢宁居,来修旧好。礼成而不反,无所归死,请以彭生除之。''齐人为杀彭生,以谢于鲁。五月,襄公田于贝丘、见豕彘。从者曰:``公子彭生也。''公怒曰:``公子彭生安敢见!''射之,豕人立而啼。公惧,坠于车下,伤足亡屦。反,诛屦于徒人费,不得也,鞭之见血。费走而出,遇贼于门,胁而束之,费袒而示之背,贼信之,使费先入,伏公而出,斗死于门中。石之纷如死于阶下。孟阳代君寝于床,贼杀之。曰:``非君也,不类。''见公之足于户下,遂杀公,而立公孙无知也。

鲍叔牙奉公子小白奔莒,管夷吾、召忽奉公子纠奔鲁。九年,公孙无知虐于雍廪,雍廪杀无知也。桓公自莒先入,鲁人伐齐,纳公子纠。战于乾时,管仲射桓公中钩,鲁师败绩,桓公贱位。于是劫鲁,使鲁杀公子纠。桓公问于鲍叔曰:``将何以定社稷?''鲍叔曰:``得管仲与召忽,则社稷定矣。''公曰:``夷吾与召忽,吾贼也。''鲍叔乃告公其故图。公曰:``然则可得乎?''鲍叔曰:``若亟召,则可得也;不亟,不可得也。夫鲁施伯知夷吾为人之有慧也,其谋必将令鲁致政于夷吾、夷吾受之,则彼知能弱齐矣;夷吾不受,彼知其将反于齐也,必将杀之。''公曰:``然则夷吾将受鲁之政乎?其否也?''鲍叔对曰:``不受。夫夷吾之不死纠也,为欲定齐国之社稷也,今受鲁之政,是弱齐也。夷吾之事君无二心,虽知死,必不受也。''公曰:``其于我也,曾若是乎?''鲍叔对曰:``非为君也,为先君也。其于君不如亲纠也,纠之不死,而况君乎?君若欲定齐之社稷,则亟迎之。''公曰:``恐不及,奈何?''鲍叔曰:``夫施伯之为人也,敏而多畏。公若先反,恐注怨焉,必不杀也。''公曰:``诺。''施伯进对鲁君曰:``管仲有急,其事不济,今在鲁,君其致鲁之政焉。若受之,则齐可弱也;若不受,则杀之。杀之,以悦于齐也,与同怒,尚贤于已。''君曰:``诺。''鲁未及致政,而齐之使至,曰:``夷吾与召忽也,寡人之贼也,今在鲁,寡人愿生得之。若不得也,是君与寡人贼比也。''鲁君问施伯,施伯曰:``君与之。臣闻齐君惕而亟骄,虽得贤,庸必能用之乎?及齐君之能用之也,管子之事济也。夫管仲天下之大圣也,今彼反齐、天下皆乡之,岂独鲁乎!今若杀之,此鲍叔之友也,鲍叔因此以作难,君必不能待也,不如与之。''鲁君乃遂束缚管仲与召忽。管仲谓召忽曰:``子惧乎?''召忽曰:``何惧乎?吾不蚤死,将胥有所定也;今既定矣,令子相齐之左,必令忽相齐之右。虽然,杀君而用吾身,是再辱我也。子为生臣,忽为死臣。忽也知得万乘之政而死,公子纠可谓有死臣矣。子生而霸诸侯,公子纠可谓有生臣矣。死者成行,生者成名,名不两立,行不虚至。子其勉之,死生有分矣。''乃行,入齐境,自刎而死。管仲遂入。君子闻之曰:``召忽之死也,贤其生也:管仲之生也,贤其死也。''

或曰:明年,襄公逐小白,小白走莒。三年,襄公薨,公子纠践位。国人召小白。鲍叔曰:``胡不行矣?''小白曰:``不可。夫管仲知,召忽强武,虽国人召我,我犹不得入也。''鲍叔曰:``管仲得行其知于国,国可谓乱乎?召忽强武,岂能独图我哉?''小白曰:``夫虽不得行其知,岂且不有焉乎?召忽虽不得众,其及岂不足以图我哉?''鲍叔对曰:``夫国之乱也,智人不得作内事,朋友不能相合摎,而国乃可图也。''乃命车驾,鲍叔御小白乘而出于莒。小白曰:``夫二人者奉君令,吾不可以试也。''乃将下,鲍叔履其足曰:``事之济也,在此时;事若不济,老臣死之,公于犹之免也。''乃行。至于邑郊,鲍叔令车二十乘先,十乘后。鲍叔乃告小白曰:``夫国之疑二三子,莫忍老臣。事之未济也,老臣是以塞道。``鲍叔乃誓曰:``事之济也,听我令;事之不济也,免公子者为上,死者为下,吾以五乘之实距路。''鲍叔乃为前驱,遂入国,逐公子纠。管仲射小自,中钩。管仲与公子纠、召忽遂走鲁。桓公践位,鲁伐齐,纳公子纠而不能。

桓公二年践位,召管仲。管仲至,公问曰:``社稷可定乎?''管仲对曰:``君霸王,社稷定;君不霸王,社稷不定。''公曰:``吾不敢至于此其大也,定社稷而已。''管仲又请,君曰:``不能。''管仲辞于君曰:``君免臣于死,臣之幸也;然臣之不死纠也,为欲定社稷也。社稷不定,臣禄齐国之政而不死纠也,臣不敢。''乃走出,至门,公召管仲。管仲反,公汗出曰:``勿已,其勉霸乎。''管仲再拜稽首而起曰:``今日君成霸,臣贪承命,趋立于相位。''乃令五官行事。异日,公告管仲曰:``欲以诸侯之间无事也,小修兵革。''管仲曰:``不可。百姓病,公先与百姓,而藏其兵。与其厚于兵,不如厚于人。齐国之社稷未定,公未始于人而始于兵,外不亲于诸侯,内不亲于民。''公曰:``诺。''政未能有行也。

二年,桓公弥乱,又告管仲曰:``欲缮兵。''管仲又曰:``不可。''公不听,果为兵。桓公与宋夫人饮船中。夫人荡船而惧公。公怒,出之,宋受而嫁之蔡侯。明年,公怒告管仲曰:``欲伐宋。''管仲曰:``不可。臣闻内政不修,外举事不济。''公不听,果伐宋。诸侯兴兵而救宋,大败齐师。公怒,归告管仲曰:``请修兵革。吾士不练,吾兵不实,诸侯故敢救吾仇。内修兵革!''管仲曰:``不可,齐国危矣。内夺民用,士劝于勇,外乱之本也。外犯诸侯,民多怨也。为义之士,不入齐国,安得无危?''鲍叔曰:``公必用夷吾之言。''公不听,乃令四封之内修兵。关市之征侈之,公乃遂用以勇授禄。鲍叔谓管仲曰:``异日者,公许子霸,今国弥乱,子将何如?''管仲曰:``吾君惕,其智多诲,姑少胥其自及也。''鲍叔曰:``比其自及也,国无阙亡乎?''管仲曰:``未也。国中之政,夷吾尚微为焉,''乱乎尚可以待。外诸侯之佐,既无有吾二人者,未有敢犯我者。''明年,朝之争禄相刺,裚领而刎颈者不绝。鲍叔谓管仲曰:``国死者众矣,毋乃害乎?''管仲曰:``安得已然,此皆其贪民也。夷吾之所患者,诸侯之为义者莫肯入齐,齐之为义者莫肯仕。此夷吾之所患也。若夫死者,吾安用而爱之?''

公又内修兵。三年,桓公将伐鲁,曰:``鲁与寡人近,于是其救宋也疾,寡人且诛焉。''管仲曰:``不可。臣闻有土之君,不勤于兵,不忌于辱,不辅其过,则社稷安。勤于兵,忌于辱,辅其过,则社稷危。''公不听。兴师伐鲁,造于长勺。鲁庄公兴师逆之,大败之。桓公曰:``吾兵犹尚少,吾参围之,安能圉我!''

四年,修兵,同甲十万,车五千乘。谓管仲曰:``吾士既练,吾兵既多,寡人欲服鲁。''管仲喟然叹曰:``齐国危矣。君不竞于德而竞于兵。天下之国带甲十万者不鲜矣,吾欲发小兵以服大兵。内失吾众,诸侯设备,吾人设诈,国欲无危,得已乎?''公不听,果伐鲁。鲁不敢战,去国五十里而为之关。鲁请比于关内,以从于齐,齐亦毋复侵鲁。桓公许诺。鲁人请盟曰:``鲁小国也,固不带剑,今而带剑,是交兵闻于诸侯,君不如已。请去兵。''桓公曰:``诺。''乃令从者毋以兵。管仲曰:``不可。诸侯加忌于君,君如是以退可。君果弱鲁君,诸侯又加贪于君,后有事,小国弥坚,大国设备,非齐国之利也。''桓公不听。管仲又谏曰:``君必不去鲁,胡不用兵?曹刿之为人也,坚强以忌,不可以约取也。''桓公不听,果与之遇。庄公自怀剑,曹刿亦怀剑,践坛,庄公抽剑其怀曰:``鲁之境去国五十里,亦无不死而已。''左揕桓公,右自承曰:``均之死也,戮死于君前。''管仲走君,曹刿抽剑当两阶之间,曰:``二君将改图,无有进者!''管仲曰:``君与地,以汶为竟。''桓公许诺,以汶为竟而归。桓公归而修于政,不修于兵革,自圉,辟人,以过,弭师。

五年,宋伐杞。桓公谓管仲与鲍叔曰,``夫宋,寡人固欲伐之,无若诸侯何?夫杞,明王之后也。今宋伐之,予欲救之,其可乎?''管仲对曰:``不可。臣闻内政之不修,外举义不信。君将外举义,以行先之,则诸侯可令附。''桓公曰:``于此不救,后无以伐宋。''管仲曰:``诸侯之君,不贪于土。贪于土必勤于兵、勤于兵必病于民,民病则多诈。夫诈密而后动者胜,诈则不信于民。夫不信于民则乱,内动则危于身。是以古之人闻先王之道者,不竞于兵。''桓公曰:``然则奚若?''管仲对曰:``以臣则不而,令人以重市使之。使之而不可,君受而封之。''桓公问鲍叔曰:``奚若?''鲍叔曰:``公行夷吾之言。''公乃命曹孙宿使于宋。宋不听,果伐杞。桓公筑缘陵以封之,予车百乘,甲一千。明年,狄人伐邢,邢君出致于齐,桓公筑夷仪以封之,予车百乘,卒干人。明年,狄人伐卫,卫君出致于虚,桓公且封之,隰朋、宾胥无谏曰:``不可。三国所以亡者,绝以小。今君封亡国,国尽若何?''桓公问管仲曰:``奚若?''管仲曰:''君有行之名,安得有其实。君其行也。''公又间鲍叔,鲍叔曰:``君行夷吾之言。''桓公筑楚丘以封之,与车三百乘,甲五千。既以封卫,明年,桓公问管仲:将何行?管仲对曰:``公内修政而劝民,可以信于诸侯矣。''君许诺。乃轻税,弛关市之征,为赋禄之制,既已,管仲又请曰:``问病。臣愿赏而无罚,五年,诸侯可令傅。''公曰,``诺。''既行之,管仲又请曰:``诸侯之礼,令齐以豹皮往,小侯以鹿皮报;齐以马往,小侯以犬报。''桓公许诺,行之。管仲又请赏于国以及诸侯,君曰:``诺。''行之。管仲赏于国中,君赏于诸侯。诸侯之君有行事善者,以重币贺之;从列士以下有善者,衣裳贺之;凡诸侯之臣有谏其君而善者,以玺问之、以信其言。公既行之,又问管仲曰:``何行?''管仲曰:``隰朋聪明捷给,可令为东国。宾胥无坚强以良,可以为西士。卫国之教,危傅以利。公子开方之为人也,慧以给,不能久而乐始,可游于卫。鲁邑之教,好迩而训于礼。季友之为人也,恭以精,博于粮,多小信,可游于鲁。楚国之教,巧文以利,不好立大义,而好立小信。蒙孙博于教,而文巧于辞,不好立大义,而好结小信,可游于楚。小侯既服,大侯既附,夫如是,则始可以施政矣。''君曰:``诺。''乃游公子开方于卫,游季友于鲁,游蒙孙于楚。五年,诸侯附。

狄人伐,桓公告诸侯曰:``请救伐。诸侯许诺,大侯车二百乘,卒二千人;小侯车百乘,卒于人。''诸侯皆许诺。齐车千乘,卒先致缘陵,战于后。故败狄。其车甲与货,小侯受之,大侯近者,以其县分之,不践其国。北州侯莫来,桓公遇南州侯于召陵,曰:``狄为无道,犯天子令,以伐小国;以天子之故,敬天之命,令以救伐。北州侯莫至,上不听天子令,下无礼诸侯,寡人请诛于北州之侯。''诸侯许诺。桓公乃北伐令支,下凫之山,斩孤竹,遇山戎,顾问管仲曰:``将何行?''管仲对曰:``君教诸侯为民聚食,诸侯之兵不足者,君助之发。如此,则始可以加政矣。''桓公乃告诸侯,必足三年之食,安以其余修兵革。兵革不足,以引其事告齐,齐助之发。既行之,公又问管仲曰:``何行?''管仲对曰:``君会其君臣父子,则可以加政矣。''公曰:``会之道奈何?''曰:``诸侯毋专立妾以为妻,毋专杀大臣,无国劳毋专予禄;士庶人毋专弃妻,毋曲堤,毋贮粟,毋禁材。行此卒岁,则始可以罚矣。''君乃布之于诸侯,诸侯许诺,受而行之。卒岁,吴人伐穀,桓公告诸侯未遍,诸侯之师竭至,以待桓公。桓公以车千乘会诸侯于竟,都师未至,吴人逃。诸侯皆罢。桓公归,问管仲曰:``将何行?''管仲曰:``可以加政矣。''曰:``从今以往二年,嫡子不闻孝,不闻爱其弟,不闻敬老国良,三者无一焉,可诛也。诸侯之臣及国事,三年不闻善,可罚也。君有过,大夫不谏;士庶人有善,而大夫不进,可罚也。士庶人闻之吏贤、孝、悌,可赏也。''桓公受而行之,近侯莫不请事,兵车之会六,乘车之会三,飨国四十有二年。

桓公践位十九年,弛关市之征,五十而取一。赋禄以粟,案田而税。二岁而税一,上年什取三,中年什取二,下年什取一;岁饥不税,岁饥弛而税。

桓公使鲍叔识君臣之有善者,晏子识不仕与耕者之有善者;高子识工贾之有善者,国子为李,隰朋为东国,宾胥无为西土,弗郑为宅。凡仕者近宫,不仕与耕者近门,工贾近市。三十里置遽,委焉,有司职之。从诸侯欲通,吏从行者,令一人为负以车;若宿者,令人养其马,食其委。客与有司别契,至国八契费。义数而不当,有罪。凡庶人欲通,乡吏不通,七日,囚,出欲通,吏不通,五日,囚。贵人子欲通,吏不通,二日,囚。凡具吏进诸侯士而有善,观其能之大小以为之赏,有过无罪。令鲍叔进大夫,劝国家,得之成而不悔,为上举。从政治为次。野为原,又多不发,起讼不骄,次之。劝国家,得之成而悔;从政虽治而不能,野原又多发;起讼骄,行此三者为下。令晏子进贵人之子,出不仕,处不华,而友有少长,为上举;得二,为次;得一,为下。士处靖,敬老与贵,交不失礼,行此三者,为上举;得二,为次;得一,为下。耕者农农用力,应于父兄,事贤多,行此三者,为上举;得二,为次;得一,为下。令高子进工贾,应于父兄,筝长养老,承事敬,行此三者,为上举;得二者,为次;得一者,为下。令国子以情断狱。三大夫既已选举,使县行之。管仲进而举言,上而见之于君,以卒年君举。管仲告鲍叔曰:``劝国家,不得成而悔,从政不治不能、野原又多而发,讼骄,凡三者,有罪元赦。''告晏子曰:``贵人子处华,下交,好饮食,行此三者,有罪无赦。士出入无常,不敬老而营富,行此三者,有罪无赦。耕者出入不应于父兄,用力不农,不事贤,行此三者,有罪无赦。''告国子曰:``工贾出入不应父兄,承事不敬,而违老治危,行此三者,有罪无赦,凡于父兄无过,州里称之,吏进之,君用之。有善无赏,有过无罚。吏不进,廉意。于父兄无过,于州里莫称,吏进之,君用之。善,为上赏;不善,吏有罚。''君谓国子:``凡贵贱之义,入与父俱,出与师俱,上与君俱。凡三者,遇贼不死,不知贼,则无赦。断狱,情与义易,义与禄易,易禄可无敛,有可无赦。''

\hypertarget{header-n335}{%
\subsection{匡君中匡}\label{header-n335}}

管仲会国用,三分二在宾客,其一在国,管仲惧而复之。公曰:``吾子犹如是乎?四邻宾客,入者说,出者誉,光名满天下。入者不说,出者不誉,污名满天下。壤可以为粟,木可以为货。粟尽则有生,货散则有聚。君人者,名之为贵,财安可有?''管仲曰:``此君之明也。''公曰:``民办军事矣,则可乎?''对曰:``不可,甲兵未足也。请薄刑罚,以厚甲兵。''于是死罪不杀,刑罪不罚,使以甲兵赎。死罪以犀甲一戟,刑罚以胁盾一戟,过罚以金军,无所计而讼者,成以束矢。公曰:``甲兵既足矣,吾欲诛大国之不道者,可乎?''对曰:``爱四封之内,而后可以恶竟外之不善者;安卿大夫之家,而后可以危救敌之国;赐小国地,而后可以诛大国之不道者;举贤良,而后可以废慢法鄙贱之民。是故先王必有置也,而后必有废也;必有利也,而后必有害也。''桓公曰:``昔三王者,既弑其君,今言仁义,则必以三王为法度,不识其故何也?''对曰:``昔者禹平治天下,及桀而乱之,汤放桀,以定禹功也。汤平治天下,及纣而乱之,武王伐纣,以定汤功也。且善之伐不善也,自古至今,未有改之。君何疑焉?''公又问曰:``古之亡国其何失?''对曰:``计得地与宝,而不计失诸侯;计得财委,而不计失百姓;计见亲,而不计见弃。三者之属一,足以削;遍而有者,亡矣。古之隳国家,陨社稷者,非故且为之也,必少有乐焉,不知其陷于恶也。''

桓公谓管仲曰:``请致仲父。''公与管仲父而将饮之,掘新井而柴焉。十日斋戒,召管仲。管仲至,公执爵,夫人执尊,觞三行,管仲趋出。公怒曰:``寡人斋戒十日而饮仲父,寡人自以为修矣。仲父不告寡人而出,其故何也?''鲍叔、隰朋趋而出,及管仲于途,曰:``公怒。''管仲反,入,倍屏而立,公不与言。少进中庭,公不与言。少进傅堂,公曰:``寡人斋戒十日而饮仲父,自以为脱于罪矣。仲父不告寡人而出,未知其故也。''对曰:``臣闻之,沉于乐者洽于忧,厚于味者薄于行,慢于朝者缓于政,害于国家者危于社稷,臣是以敢出也。''公遽下堂曰:``寡人非敢自为修也,仲父年长,虽寡人亦衰矣,吾愿一朝安仲父也。''对曰:``臣闻壮者无怠,老者无偷,顺天之道,必以善终者也。三王失之也,非一朝之萃,君奈何其偷乎?''管仲走出,君以宾客之礼再拜送之。明日,管仲朝,公曰:``寡人愿闻国君之信。''对曰:``民爱之,邻国亲之,天下信之,此国君之信。''公曰:``善。请间信安始而可?''对曰:``始于为身,中于为国,成于为天下。''公曰:``请问为身。''对曰:``道血气,以求长年、长心、长德。此为身也。''公曰:``请问为国。''对曰:``远举贤人,慈爱百姓,外存亡国,继绝世,起诸孤;薄税敛,轻刑罚,此为国之大礼也。''公曰:``请问为天下。''对曰:``法行而不苛,刑廉而不赦,有司宽而不凌;菀浊困滞皆,法度不亡,往行不来,而民游世矣,此为天下也。''

\hypertarget{header-n340}{%
\subsection{匡君小匡}\label{header-n340}}

桓公自莒反于齐,使鲍叔牙为宰。鲍叔辞曰:``臣,君之庸臣也。君有加惠于其臣,使臣不冻饥,则是君之赐也。若必治国家,则非臣之所能也,其唯管夷吾乎。臣之所不如管夷吾者五:宽惠爱民,臣不如也;治国不失秉,臣不如也;忠信可结于诸侯,臣不如也;制礼义可法于四方,臣不如也;介胃执枹,立于军门,使百姓皆加勇,臣不如也。夫管仲,民之父母也,将欲治其子,不可弃其父母。''公曰:``管夷吾亲射寡人,中钩,殆于死,今乃用之,可乎?''鲍叔曰:``彼为其君动也,君若宥而反之,其为君亦犹是也。''公曰:``然则为之奈何?''鲍叔曰:``君使人请之鲁。''公曰:``施伯,鲁之谋臣也。彼知吾将用之,必不吾予也。''鲍叔曰:``君诏使者曰:`寡君有不令之臣在君之国,愿请之以戮群臣。'鲁君必诺。且施伯之知夷吾之才,必将致鲁之政。夷吾受之,则鲁能弱齐矣。夷吾不受,彼知其将反于齐。必杀之。''公曰:``然则夷吾受乎?''鲍叔曰:``不受也。夷吾事君无二心。''公曰:``其于寡人犹如是乎?''对曰:``非为君也,为先君与社稷之故。君若欲定宗庙,则亟请之,不然,无及也。''公乃使鲍叔行成,曰:``公子纠,亲也。请君讨之。''鲁人为杀公子纠。又曰:``管仲,仇也。请受而甘心焉。''鲁君许诺。施伯谓鲁侯曰:``勿予。非戮之也,将用其政也。管仲者,天下之贤人也,大器也。在楚则楚得意于天下,在晋则晋得意于天下,在狄则狄得意于天下。今齐求而得之,则必长为鲁国忧,君何不杀而受之其尸。''鲁君曰:``诺。''将杀管仲。鲍叔进曰:``杀之齐,是戮齐也。杀之鲁,是戮鲁也。弊邑寡君愿生得之,以徇于国,为群臣僇;若不生得,是君与寡君贼比也。非弊邑之君所谓也,使臣不能受命。''于是鲁君乃不杀,遂生束缚而柙以予齐。鲍叔受而哭之,三举。施伯从而笑之,谓大夫曰:``管仲必不死。夫鲍叔之,忍不僇贤人,其智称贤以自成也。鲍叔相公子小白先入得国,管仲、召忽奉公子纠后入,与鲁以战,能使鲁败,功足以。得天与失天,其人事一也。今鲁惧,杀公子纠、召忽,囚管仲以予齐,鲍叔知无后事,必将勤管仲以劳其君愿,以显其功。众必予之有得。力死之功,犹尚可加也,显生之功将何如?是昭德以贰君也,鲍叔之知,不是失也。''

至于堂阜之上,鲍叔祓而浴之三。桓公亲迎之郊。管仲诎缨插衽,使人操斧而立其后。公辞斧三,然后退之。公曰:``垂缨下衽,寡人将见。''管仲再拜稽首曰:``应公之赐,杀之黄泉,死且不朽。''公遂与归,礼之于庙,三酌而问为政焉,曰:``首先君襄公,高台广池,湛乐饮酒,田猎罼弋,不听国政。卑圣侮士,唯女是崇,九妃六嫔,陈妾数千。食必粱肉,衣必文绣,而戎士冻饥。戎马待游车之弊,戎士待陈妾之余。倡优侏儒在前,而贤大夫在后。是以国家不日益,不月长。吾恐宗庙之不扫除,社稷之不血食,敢问为之奈何?''管子对曰:``昔吾先王周昭王、穆王世法文武之远迹,以成其名。合群国,比校民之有道者,设象以为民纪、式美以相应,比缀以书,原本穷末。劝之以庆赏,纠之以刑罚,粪除其颠旄,赐予以镇抚之,以为民终始。''公曰:``为之奈何?''管子对曰:``昔者圣王之治其民也,参其国而伍其鄙,定民之居,成民之事,以为民纪,谨用其六秉;如是而民情可得,而百姓可御。''桓公曰:``六秉者何也?''管子曰:``杀、生、贵、贱、贫、富,此六秉也。''桓公曰:``参国奈何?''管子对曰:``制国以为二十一乡:商工之乡六,士农之乡十五。公帅十一乡,高子帅五乡,国子帅五乡。参国故为三军。公立三官之臣:市立三乡,工立三族,泽立三虞,山立三衡。制五家为轨,轨有长;十轨为里,里有司;四里为连,连有长;十连为乡,乡有良人;三乡一帅。''桓公曰:``五鄙奈何?''管子对曰:``制五家为轨,轨有长;六轨为邑,邑有司;十邑为率,率有长;十率为乡,乡有良人;三乡为属,属有帅。五属一五大夫。武政听属,文政听乡,各保而听,毋有淫佚者。''桓公曰:``定民之居,成民之事奈何?''管子对曰:``士农工商四民者,国之石民也,不可使杂处,杂处则其言哤,其事乱。是故圣王之处士必于闲燕,处农必就田野,处工必就官府,处商必就市井。今夫士群萃而州处,闲燕则父与父言义,子与子言孝,其事君者言敬,长者言爱,幼者言弟。旦昔从事于此,以教其子弟,少而习焉,其心安焉,不见异物而迁焉。是故其父兄之教不肃而成;其子弟之学不劳而能。夫是故士之子常为士。今夫农群萃而州处,审其四时,权节具,备其械器用,比耒耜谷芨。及寒击槁除田,以待时乃耕,深耕、均种、疾耰。先雨芸耨,以待时雨。时雨既至,挟其枪刈耨镈,以旦暮从事于田野,税衣就功,别苗莠,列疏遬。首戴苎蒲,身服袯襫,沾体涂足,暴其发肤,尽其四支之力,以疾从事于田野。少而习焉,其心安焉,不见异物而迁焉。是故其父兄之教不肃而成;其子弟之学不劳而能。是故农之子常为农,朴野而不慝,其秀才之能为士者,则足赖也,故以耕则多粟,以仕则多贤,是以圣王敬畏戚农。今夫工群萃而州处,相良材,审其四时,辨其功苦,权节其用,论比计制,断器尚完利。相语以事,相示以功,相陈以巧,相高以知事。旦昔从事于此,以教其子弟。少而习焉,其心安焉,不见异物而迁焉。是故其父兄之教不肃而成,其子弟之学不劳而能。夫是故工之子常为工。今夫商群萃而州处,观凶饥,审国变,察其四时而监其乡之货,以知其市之贾。负任担荷,服牛辂马,以周四方;料多少,计贵贱,以其所有,易其所无,买贱鬻贵。是以羽旄不求而至,竹筋有余于国;奇怪时来,珍异物聚。旦昔从事于此,以教其子弟。相语以利,相示以时,相陈以知贾。少而习焉,其心安焉,不见异物而迁焉。是故其父兄之教不肃而成;其子弟之学不劳而能。夫是故商之子常为商。相地而衰其政,则民不移矣。正旅旧,则民不惰。山泽各以其时至,则民不苟。陵陆、丘井、田畴均,则民不惑。无夺民时,则百姓富;牺牲不劳,则牛马育。''

桓公又问曰:``寡人欲修政以干时于天下,其可平?''管子对曰:``可。''公曰:``安始而可?''管子对曰:``始于爱民。''公曰:``爱民之道奈何?''管子对曰:``公修公族,家修家族,使相连以事,相及以禄,则民相亲矣。放旧罪,修旧宗,立无后,则民殖矣。省刑罚,薄赋敛,则民富矣。乡建贤士,使教于国,则民有礼矣。出令不改,则民正矣。此爱民之道也。''公曰:``民富而以亲,则可以使之乎?''管于对曰:``举财长工,以止民用;陈力尚贤,以劝民知;加刑无苛,以济百姓。行之无私,则足以容众矣;出言必信,则令不穷矣。此使民之道也。''

桓公曰:``民居定矣,事已成矣,吾欲从事于天下诸侯,其可乎?''管子对曰:``未可。民心未吾安。''公曰:``安之奈何?''管子对曰:``修旧法,择其善者,举而严用之;慈于民,予无财,宽政役,敬百姓,则国富而民安矣。''公曰:``民安矣,其可乎?''管仲对曰:``未可。君若欲正卒伍,修甲兵,则大国亦将正卒伍,修甲兵。君有征战之事,则小国诸侯之臣有守圉之备矣。然则难以速得意于天下。公欲速得意于天下诸侯,则事有所隐,而政有所寓。''公曰,``为之奈何?''管子对曰:``作内政而寓军令焉。为高子之里,为国子之里,为公里,三分齐国,以为三军。择其贤民,使为里君。乡有行伍,卒长则其制令,且以田猎,因以赏罚,则百姓通于军事矣。''桓公曰:``善。''于是乎管子乃制五家以为轨,轨为之长;十轨为里,里有司;四里为连,连为之长;十连为乡,乡有良人,以为军令。是故五家为轨,五人为伍,轨长率之。十轨为里,故五十人为小戎,里有司率之。四里为连,故二百人为卒,连长率之。十连为乡,故二千人为旅,乡良人率之。五乡一师,故万人一军,五乡之师率之。三军故有中军之鼓,有高子之鼓,有国子之鼓。春以田,曰蒐①,振旅。秋以田,曰獼,治兵。是故卒伍政定于里,军旅政定于郊。内教既成,令不得迁徙。故卒伍之人,人与人相保,家与家相爱,少相居,长相游,祭祀相福,死丧相恤,祸福相忧,居处相乐,行作相和,哭泣相哀。是故夜战其声相闻,足以无乱;昼战其目相见,足以相识;欢欣足以相死,是故以守则固,以战则胜。君有此教士三万人,以横行于天下,诛无道,以定周室,天下大国之君莫之能圉也。

正月之朝,乡长复事,公亲问焉,曰:``于子之乡,有居处为义好学、聪明质仁、慈孝于父母、长弟闻于乡里者,有则以告。有而不以告,谓之蔽贤,其罪五。''有司已于事而竣。公又问焉,曰:``于子之乡,有拳勇、股肱之力、筋骨秀出于众者,有则以告。有而不以告,谓之蔽才,其罪五。''有司已于事而竣。公又问焉,曰:``于子之乡,有不慈孝于父母,不长弟于乡里,骄躁淫暴,不用上令者,有则以告。有而不以告,谓之下比,其罪五。''有司已于事而竣。于是乎乡长退而修德进贤。桓公亲见之,遂使役之官。公令官长,期而书伐以告,且令选官之贤者而复之。曰:``有人居我官有功,休德维顺,端悫以待时使。使民恭敬以劝。其称秉言,则足以补官之不善政。''公宣问其乡里,而有考验。乃召而与之坐,省相其质,以参其成功成事。可立而时。设问国家之患而不肉,退而察问其乡里,以观其所能,而无大过,登以为上卿之佐。名之曰三选。高子、国子退而修乡,乡退而修连,连退而修里,里退而修轨,轨退而修家。是故匹夫有善,故可得而举也;匹夫有不善,故可得而诛也。政既成,乡不越长,朝不越爵。罢士无伍,罢女无家。士三出妻,逐于境外。女三嫁,入于舂谷。是故民皆勉为善。士与其为善于乡,不如为善于里;与其为善于里,不如为善于家。是故士莫敢言一朝之便,皆有终岁之汁;莫敢以终岁为议,皆有终身之功。

正月之朝,五属大夫复事于公,择其寡功者而谯之曰:``列地分民者若一,何故独寡功?何以不及人?教训不善,政事其不治,一再则宥,三则不赦。''公又问焉,曰,``于子之属,有居处为义好学、聪明质仁、慈孝于父母、长弟闻于乡里者,有则以告。有而不以告,谓之蔽贤,其罪五。''有司已事而竣。公又问焉,曰:``于子之属,有拳勇、股肱之力秀出于众者,有则以告。有而不以告,谓之蔽才,其罪五。''有司已事而竣。公又问焉,曰:``于子之属,有不慈孝于父母,不长弟于乡里,骄躁淫暴,不用上令者,有则以告。有而不以告者,谓之下比,其罪五。''有司已事而竣。于是乎五属大夫退而修属,属退而修连,连退而修乡,乡退而修卒,卒退而修邑,邑退而修家。是故匹夫有善,可得而举;匹夫有不善,可得而诛。政成国安,以守则固,以战则强。封内治,百姓亲,可以出征四方,立一霸王矣。

桓公曰:``卒伍定矣,事已成矣,''吾欲从事于诸侯,其可乎?''管子对曰:``未可。若军令则吾既寄诸内政矣,夫齐国寡甲兵,吾欲轻重罪而移之于甲兵。''公曰:``为之奈何?''管子对曰:``制重罪入以兵甲、犀胁、二戟,轻罪入兰、盾、鞈革、二戟,小罪入以金钧分,宥薄罪入以半钧,无坐抑而讼狱者,正三禁之而不直,则入一束矢以罚之。美金以铸戈、剑、矛、戟,试诸狗马;恶金以铸斤、斧、鉏、夷、锯、欘,试诸木土。''

桓公曰,``甲兵大足矣,吾欲从事于诸侯,可乎?''管仲对曰:``未可。治内者未具也,为外者未备也。''故使鲍叔牙为大谏,王子城父为将,弦子旗为理,宁戚为田,隰朋为行,曹孙宿处楚,商容处宋,季劳处鲁,徐开封处卫,匽尚处燕,审友处晋。又游士八千人,奉之以车马衣裘,多其资粮,财币足之,使出周游于四方,以号召收求天下之贤士。饰玩好,使出周游于四方,鬻之诸侯,以观其上下之所贵好,择其沈乱者而先政之。公曰:``外内定矣,可乎?''管子对曰:``未可。邻国未吾亲也。''公曰:``亲之奈何?''管子对曰:``审吾疆场,反其侵地,正其封界;毋受其货财,而美为皮弊,以极聘覜于诸侯,以安四邻,则邻国亲我矣。''桓公曰:``甲兵大足矣,吾欲南伐,何主?''管子对曰:``以鲁为主。反其侵地常、潜,使海于有弊,渠弥于河有陼,纲山于有牢。''桓公曰:``吾欲西伐,何主?''管子对曰:``以卫为主。反其侵地吉台、原、姑与柒里,使海于有弊,渠弥于有陼,纲山于有牢。''桓公曰:``吾欲北伐,何主?''管子对曰:``以燕为主,反其侵地柴夫、吠狗。使海于有弊,渠弥于有陼,纲山于有牢。''四邻大亲。既反其侵地,正其封疆,地南至于岱阴,西至于济,北至于海,东至于纪随,地方三百六十里。三岁治定,四岁教成,五岁兵出。有教士三万人,革车八百乘。诸侯多沈乱,不服于天子。于是乎桓公东救徐州,分吴半。存鲁蔡陵陵蔡,割越地。南据宋、郑,征伐楚,济汝水,逾方地。望文山,使贡丝于周室。成周反胙于隆岳,荆州诸侯莫不来服。中救晋公,禽狄王,败胡貉,破屠何,而骑寇始服。北伐山戎,制泠支,斩孤竹,而九夷始听。海滨诸侯,莫不来服。西征攘白狄之地,遂至于西河,方舟投柎,乘桴济河,至于石沈。县车柬马,逾大行与卑耳之貉,拘秦夏,西服流沙西虞,而秦戎始从。故兵一出而大功十二。故东夷、西戎、南蛮、北狄、中诸侯国,莫不宾服。与诸侯饰牲为载书,以誓要于上下荐神。然后率天下定周室,大朝诸侯于阳谷。故兵车之会六,乘车之会三,九合诸侯,一匡天下。甲不解垒,兵不解翳,弢无弓,服无矢,寝武事,行文道,以朝天子。

葵丘之会,天子使大夫宰孔致胙于桓公曰:``余一人之命有事于文武。使宰孔致胙。''且有后命曰:``以尔自卑劳,实谓尔伯舅毋下拜。''桓公召管仲而谋,管仲对曰:``为君不君,为臣不臣,乱之本也。''桓公曰:``余乘车之会三,兵车之会六,九合诸侯,一匡天下。北至于孤竹、山戎、秽貉,拘秦夏;西至流沙、西虞;南至吴、越、巴、牂牁、{[}{]}、不庾、雕题、黑齿。荆夷之国,莫违寡人之命,而中国卑我,昔三代之受命者,其异于此乎?''管子对曰:``夫凤凰鸾鸟不降,而鹰隼鸱枭丰,庶神不格,守龟不兆,握粟而筮者屡中。时雨甘露不降,飘风暴雨数臻。五谷不蕃,六畜不育,而蓬蒿藜藋并兴。夫凤凰之文,前德义,后日昌,昔人之受命者,龙龟假,河出图,雒出书,地出乘黄。今三祥未见有者,虽曰受命,无乃失诸乎?''桓公惧,出见客曰:``天威不违颜咫尺,小白承天子之命而毋下拜,恐颠蹶于下,以为天子羞。''遂下拜,登受赏服、大路、龙旗九游、渠门赤旗。天子致胙于桓公而不受,天下诸侯称顺焉。

恒公忧天下诸侯。鲁有夫人庆父之乱,而二君弑死,国绝无后。桓公闻之,使高子存之。男女不淫,马牛选具。执玉以见,请为关内之侯,而桓公不使也。狄人攻邢,桓公筑夷仪以封之。男女不淫,马牛选具。执玉以见,请为关内之侯,而桓公不使也。狄人攻卫,卫人出旅干曹,桓公城楚丘封之。其畜以散亡,故桓公予之系马三百匹,天下诸侯称仁焉。于是天下之诸侯知桓公之为己勤也,是以诸侯之归之也譬若市人。桓公知诸侯之归己也,故使轻其币而重其礼。故使天下诸侯以疲马犬羊为币,齐以良马报。诸侯以缕帛布鹿皮四分以为币,齐以文锦虎豹皮报。诸侯之使垂橐而入,载而归。故钧之以爱,致之以利,结之以信,示之以武。是故天下小国诸侯,既服桓公,莫之敢倍而归之。喜其爱而贪其利,信其仁而畏其武。桓公知天下小国诸侯之多与己也,于是又大施忠焉。可为忧者为之忧,可为谋者为之谋,可为动者为之动。伐谭莱而不有也,诸侯称仁焉。通齐国之鱼盐东莱,使关市几而不正,壥而不税,以为诸侯之利,诸侯称宽焉。筑蔡、鄢陵、培夏、灵父丘,以卫戎狄之地,所以禁暴于诸侯也。筑五鹿、中牟、邺、盖与、社丘,以卫诸夏之地,所以示劝于中国也。教大成。是故天下之于桓公,远国之民望如父母,近国之民从如流水。故行地滋远,得人弥众,是何也?怀其文而畏其武。故杀无道,定周室,天下莫之能圉,武事立也。定三革,偃五兵,朝服以济河,而无怵惕焉,文事胜也。是故大国之君惭愧,小国诸侯附比。是故大国之君事如臣仆,小国诸侯欢如父母。夫然,故大国之君不尊,小国诸侯不卑。是故大国之君不骄,小国诸侯不慑。于是列广地以益狭地,损有财以与无财。周其君子,不失成功;周其小人,不失成命。夫如是,居处则顺,出则有成功。不称动甲兵之事,以遂文武之迹于天下。

桓公能假其群臣之谋以益其智也。其相曰夷吾,大夫曰宁戚、隰朋、宾胥无、鲍叔牙。用此五子者何功?度义光德,继法绍终,以遗后嗣,贻孝昭穆,大霸天下,名声广裕,不可掩也。则唯有明君在上,察相在下也。初,桓公郊迎管子而问焉。管仲辞让,然后对以参国伍鄙,立五乡以崇化,建五属以厉武,寄兵于政,因罚,备器械,加兵无道诸侯,以事周室。桓公大说。于是斋戒十日,将相管仲。管仲曰:``斧钺之人也,幸以获生,以属其腰领,臣之禄也。若知国政,非臣之任也。''公曰:``子大夫受政,寡人胜任;子大夫不受政,寡人恐崩。''管仲许诺,再拜而受相。三日,公曰:``寡人有大邪三,其犹尚可以为国乎?''对曰:``臣未得闻。''公曰:``寡人不幸而好田,晦夜而至禽侧,田莫不见禽而后反。诸侯使者无所致,百官有司无所复。''对曰:``恶则恶矣,然非其急者也。''公曰:``寡人不幸而好酒,日夜相继,诸侯使者无所致、百官有司无所复。''对曰:``恶则恶矣,然非其急者也。''公曰、``寡人有污行,不幸而好色,而姑姊有不嫁者。''对曰:``恶则恶矣,然非其急者也。''公作色曰:``此三者且可,则恶有不可者矣?''对曰:``人君唯优与不敏为不可,优则亡众,不敏不及事。''公曰:``善。吾子就舍,异日请与吾子图之。''对曰:``时可将与夷吾,何待异日乎?''公曰:``奈何?''对曰:``公子举为人博闻而知礼,好学而辞逊,请使游于鲁,以结交焉。公子开方为人巧转而兑利,请使游于卫,以结交焉,曹孙宿其为人也小廉而苛忕、足恭而辞结,正荆之则也,请使往游,以结交焉。''遂立行三使者,而后退。相三月,请论百官。公曰;``诺。''管仲曰:``升降揖让,进退闲习,辨辞之刚柔,臣不如隰朋,请立为大行。垦草入邑,辟土聚粟多众,尽地之利,臣不如宁戚,请立为大司田。平原广牧,车不结辙,士不旋踵,鼓之而三军之士视死如归,臣不如王子城父,请立为大司马。决狱折中,不杀不辜,不诬无罪,臣不如宾胥无,请立为大司理。犯君颜色,进谏必忠,不辟死亡,不挠富贵,臣不如东郭牙,请立以为大谏之官。此五子者,夷吾一不如;然而以易夷吾,夷吾不为也。君若欲治国强兵,则五子者存矣;若欲霸王,夷吾在此。''桓公曰:``善。''

\hypertarget{header-n354}{%
\subsection{霸形}\label{header-n354}}

桓公在位,管仲、隰朋见。立有间,有贰鸿飞而过之。桓公叹曰:``仲父,今彼鸿鹄有时而南,有时而北,有时而往,有时而来,四方无远,所欲至而至焉,非唯有羽翼之故,是以能通其意于天下乎?''管仲、隰朋不对。桓公曰:``二子何故不对?''管子对曰:``君有霸王之心,而夷吾非霸王之臣也,是以不敢对。''桓公曰:``仲父胡为然?盍不当言,寡人其有乡乎??寡人之有仲父也,犹飞鸿之有羽翼也,若济大水有舟楫也。仲父不一言教寡人,寡人之有耳,将安闻道而得度哉。''管子对曰:``君若将欲霸王举大事乎?则必从其本事矣。''桓公变躬迁席,拱手而问曰:``敢问何谓其本?''管子对曰:``齐国百姓,公之本也。人甚忧饥,而税敛重;人甚惧死,而刑政险;人甚伤劳,而上举事不时。公轻其税敛,则人不忧饥;缓其刑政,则人不惧死;举事以时,则人不伤劳。''桓公曰:``寡人闻仲父之言此三者,闻命矣,不敢擅也,将荐之先君。''于是令百官有司,削方墨笔。明日,皆朝于太庙之门朝,定令于百吏。使税者百一钟,孤幼不刑,泽梁时纵,关讥而不征,市书而不赋;近者示之以忠信,远者示之以礼义。行此数年,而民归之如流水。

此其后,宋伐杞,狄伐邢、卫。桓公不救,裸体纫胸称疾。召管仲曰:``寡人有千岁之食,而无百岁之寿,今有疾病,姑乐乎!''管子曰:``诺。''于是令之县钟磬之榬,陈歌舞竽瑟之乐,日杀数十牛者数旬。群臣进谏曰:``宋伐杞,狄伐邢、卫,君不可不救。''桓公曰:``寡人有千岁之食,而无百岁之寿,今又疾病,姑乐乎!且彼非伐寡人之国也,伐邻国也,子无事焉。''宋已取杞,狄已拔邢、卫矣。桓公起,行笋虡之间,管子从。至大钟之西,桓公南面而立,管仲北乡对之,大钟鸣。桓公视管仲曰:``乐夫,仲父?''管子对曰:``此臣之所谓哀,非乐也。臣闻之,古者之言乐于钟磬之间者不如此。言脱于口,而令行乎天下;游钟磬之间,而无四面兵革之忧。今君之事,言脱于口,令不得行于天下;在钟磬之间,而有四面兵革之忧。此臣之所谓哀,非乐也。''桓公曰:``善。''于是伐钟磬之县,并歌舞之乐。宫中虚无人。桓公曰:``寡人以伐钟磬之县,并歌舞之乐矣,请问所始于国,将为何行?''管子对曰:``宋伐杞,狄伐邢、卫,而君之不救也,臣请以庆。臣闻之,诸侯争于强者,勿与分于强。今君何不定三君之处哉?''于是桓公曰:``诺。''因命以车百乘、卒千人,以缘陵封杞;车百乘、卒千人,以夷仪封邢;车五百乘、卒五千人,以楚丘封卫。桓公曰:``寡人以定三君之居处矣,今又将何行?''管子对曰:``臣闻诸侯贪于利,勿与分于利。君何不发虎豹之皮、文锦以使诸侯,令诸侯以缦帛鹿皮报?''桓公曰:``诺。''于是以虎豹皮、文锦使诸侯,诸侯以缦帛、鹿皮报。则令固始行于天下矣。此其后,楚人攻宋、郑。烧焫熯焚郑地,使城坏者不得复筑也,屋之烧者不得复葺也;令其人有丧雌雄,居室如鸟鼠处穴。要宋田,夹塞两川,使水不得东流,东山之西,水深灭垝,四百里而后可田也。楚欲吞宋、郑而畏齐,曰思人众兵强能害己者,必齐也。于是乎楚王号令于国中曰:``寡人之所明于人君者,莫如桓公;所贤于人臣者,莫如管仲。明其君而贤其臣,寡人愿事之。谁能为我交齐者,寡人不爱封侯之君焉。''于是楚国之贤士皆抱其重宝币帛以事齐。桓公之左右,无不受重宝币帛者。于是桓公召管仲曰:``寡人闻之,善人者人亦善之。今楚王之善寡人一甚矣,寡人不善,将拂于道。仲父何不遂交楚哉?''管子对曰:``不可。楚人攻宋、郑,烧焫熯焚郑地,使城坏者不得复筑也,屋之烧者不得复葺也,令人有丧雌雄,居室如鸟鼠处穴。要宋田,夹塞两川,使水不得东流,东山之西,水深灭垝,四百里而后可田也。楚欲吞宋。郑,思人众兵强而能害己者,必齐也。是欲以文克齐,而以武取宋、郑也,楚取宋、郑而不知禁,是失宋、郑也;禁之,则是又不信于楚也。知失于内,兵困于外,非善举也。''桓公曰:``善。然则若何?''管子对曰:``请兴兵而南存宋、郑,而令曰:`无攻楚,言与楚王遇。'至于遇上,而以郑城与宋水为请,楚若许,则是我以文令也;楚若不许,则遂以武令焉。''桓公曰:``善。''于是遂兴兵而南存宋、郑,与楚王遇于召陵之上,而令于遇上曰:``毋贮粟,毋曲堤,无擅废嫡子,无置妾以为妻。''因以郑城与宋水为请于楚,楚人不许。遂退七十里而舍。使军人城郑南之地,立百代城焉。曰:自此而北至于河者,郑自城之,而楚不敢隳也。东发宋田,夹两川,使水复东流,而楚不敢塞也。遂南伐,及逾方城,济于汝水,望汶山,南致楚越之君,而西伐秦,北伐狄,东存晋公于南,北伐孤竹,还存燕公。兵车之会六,乘车之会三,九合诸侯,反位已霸。修钟磬而复乐。管子曰:``此臣之所谓乐也。''

\hypertarget{header-n359}{%
\subsection{霸言}\label{header-n359}}

霸王之形;象天则地,化人易代,创制天下,等列诸侯,宾属四海,时匡天下;大国小之,曲国正之,强国弱之,重国轻之;乱国并之,暴工残之:僇其罪,卑其列,维其民,然后王之。夫丰国之谓霸,兼正之国之谓王。夫王者有所独明。德共者不取也,道同者不王也。夫争天下者,以威易危暴,王之常也。君人者有道,霸王者有时。国修而邻国无道,霸王之资也。夫国之存也,邻国有焉;国之亡也,邻国有焉。邻国有事,邻国得焉;邻国有事,邻国亡焉。天下有事,则圣王利也。国危,则圣人知矣。夫先王所以王者,资邻国之举不当也。举而不当,此邻敌之所以得意也。

夫欲用天下之权者,必先布德诸侯。是故先王有所取,有所与,有所诎,有所信,然后能用天下之权。夫兵幸于权,权幸于地。故诸侯之得地利者,权从之;失地利者,权去之,夫争天下者,必先争人。明大数者得人,审小计者失人。得天下之众者王,得其半者霸。是故圣王卑礼以下天下之贤而王之,均分以钓天下之众而臣之。故贵为天子,富有天下,而伐不谓贪者,其大计存也。以天下之财,利天下之人;以明威之振,合天下之权;以遂德之行,结诸侯之亲;以好佞之罪,刑天下之心;因天下之威,以广明王之伐;攻逆乱之国,赏有功之劳;封贤圣之德,明一人之行,而百姓定矣。夫先王取天下也,术术乎大德哉,物利之谓也。夫使国常无患,而名利并至者,神圣也;国在危亡,而能寿者,明圣也。是故先王之所师者,神圣也;其所赏者,明圣也。夫一言而寿国,不听而国亡,若此者,大圣之言也。夫明王之所轻者马与玉,其所重者政与军。若失主不然,轻予人政,而重予人马;轻予人军,而重与人玉;重宫门之营,而轻四境之守,所以削也。

夫权者,神圣之所资也;独明者,天下之利器也;独断者,微密之营垒也。此三者,圣人之所则也,圣人畏微,而愚人畏明;圣人之憎恶也内,愚人之憎恶也外;圣人将动必知,愚人至危易辞。圣人能辅时,不能违时。知者善谋,不如当时。精时者,日少而功多。夫谋无主则困,事无备则废。是以圣王务具其备。而慎守其时。以备待时,以时兴事,时至而举兵。绝坚而攻国,破大而制地,大本而小标,埊近而攻远。以大牵小,以强使弱,以众致寡,德利百姓,威振天下;令行诸侯而不拂,近无不服,远无不听。夫明王为天下正,理也。按强助弱,圉暴止贪,存亡定危,继绝世,此天下之所载也,诸侯之所与也,百姓之所利也,是故天下王之。知盖天下,继最一世,材振四海,王之佐也。

千乘之国得其守,诸侯可得而臣,天下可得而有也。万乘之国失其守,国非其国也。天下皆理己独乱,国非其国也;诸侯皆令己独孤,国非其国也;邻国皆险己独易,国非其国也。此三者,亡国之徵也。夫国大而政小者,国从其政;国小而政大者,国益大。大而不为者,复小;强而不理者,复弱;众而不理者,复寡;贵而无礼者,复贱;重而凌节者,复轻,富而骄肆者,复贫。故观国者观君,观军者观将,观备者观野。其君如明而非明也,其将如贤而非贤也,其人如耕者而非耕也,三守既失,国非其国也。地大而不为,命曰土满;人众而不理,命曰人满;兵威而不止,命曰武满。三满而不止,国非其国也。地大而不耕,非其地也;卿贵而不臣,非其卿也;人众而不亲,非其人也。

夫无土而欲富者忧,无德而欲王者危,施薄而求厚者孤。夫上夹而下苴、国小而都大者弑。主尊臣卑,上威下敬,令行人服,理之至也。使天下两天子,天下不可理也:一国而两君,一国不可理也;一家而两父,一家不可理也。夫令,不高不行,不抟不听。尧舜之人,非生而理也;桀纣之人,非生而乱也。故理乱在上也。夫霸王之所始也,以人为本。本理则国固,本乱则国危。故上明则下敬,政平则人安,士教和则兵胜敌,使能则百事理,亲仁则上不危,任贤则诸侯服。

霸王之形,德义胜之,智谋胜之,兵战胜之,地形胜之,动作胜之,故王之。夫善用国者,因其大国之重,以其势小之;因强国之权,以其势弱之;因重国之形,以其势轻之。强国众,合强以攻弱,以图霸。强国少,合小以攻大,以图王。强国众,而言王势者,愚人之智也;强国少,而施霸道者,败事之谋也。夫神圣,视天下之形,知动静之时;视先后之称,知祸福之门。强国众,先举者危,后举者利。强国少,先举者王,后举者亡。战国众,后举可以霸;战国少,先举可以王。

夫王者之心,方而不最,列不让贤,贤不齿第择众,是贪大物也。是以王之形大也。夫先王之争天下也以方心,其立之也以整齐,其理之也以平易。立政出令用人道,施爵禄用地道,举大事用天道。是故先王之伐也,伐逆不伐顺,伐险不伐易,伐过不伐及。四封之内,以正使之;诸侯之会,以权致之。近而不服者,以地患之;远而不听者,以刑危之。一而伐之,武也;服而舍之,文也;文武具满,德也。夫轻重强弱之形,诸侯合则强,孤则弱。骥之材,而百马伐之,骥必罢矣。强最一伐,而天下共之,国必弱矣。强国得之也以收小,其失之也以恃强。小国得之也以制节,其失之也以离强。夫国小大有谋,强弱有形。服近而强远,王国之形也;合小以攻大,敌国之形也;以负海攻负海,中国之形也;折节事强以避罪,小国之形也。自古以至今,未尝有先能作难,违时易形,以立功名者;无有常先作难,违时易形,无不败者也。夫欲臣伐君,正四海者,不可以兵独攻而取也。必先定谋虑,便地形,利权称,亲与国,视时而动,王者之术也。夫先王之伐也,举之必义,用之必暴,相形而知可,量力而知攻,攻得而知时。是故先王之伐也,必先战而后攻,先攻而后取地。故善攻者料众以攻众,料食以攻食,料备以攻备。以众攻众,众存不攻;以食攻食,食存不攻;以备攻备,备存不攻。释实而攻虚,释坚而攻膬,释难而攻易。

夫抟国不在敦古,理世不在善攻,霸王不在成曲。夫举失而国危,刑过而权倒,谋易而祸反,计得而强信,功得而名从,权重而令行,固其数也。

夫争强之国,必先争谋,争刑,争权。令人主一喜一怒者,谋也;令国一轻一重者,刑也;令兵一进一退者,权也。故精于谋,则人主之愿可得,而令可行也;精干刑,则大国之地可夺,强国之兵可圉也;精于权,则天下之兵可齐,诸侯之君可朝也。夫神圣视天下之刑,知世之所谋,知兵之所攻,知地之所归,知令之所加矣。夫兵攻所憎而利之,此邻国之所不亲也。权动所恶,而实寡归者强。擅破一国,强在后世者王。擅破一国,强在邻国者亡。

\hypertarget{header-n371}{%
\subsection{问}\label{header-n371}}

凡立朝廷,问有本纪。爵授有德,则大臣兴义。禄予有功,则士轻死节。上帅士以人之所戴,则上下和。授事以能,则人上功。审刑当罪,则人不易讼。无乱社稷宗庙则人有所宗。毋遗老忘亲,则大臣不怨。举知人急,则众不乱。行此道也,国有常经,人知终始,此霸王之术也。

然后问事:事先大功,政自小始。

问死事之孤其未有田宅者有乎?问少壮而未胜甲兵者几何人?问死事之寡,其餼廪何如问国之有功大者何官之吏也?问州之大夫也何里之士也。今吏亦何以明之矣,问刑论有常以行,不可改也,今其事之久留也何若?问五官有制度,官都有其常断。今事之稽也何待。问独夫寡妇孤寡疾病者几何人也?问国之弃人何族之子弟也?问乡之良家其所牧养者几何人矣。问邑之贫人债而食者几何家?问理园容而食者几何家?人之开田而耕者几何家?士之身耕者几何家?问乡之贫人何族之别也?问宗子之收昆弟者,以贫从昆弟者几何家?余子仕而有田邑,今入者几何人?子弟以孝闻于乡里者几何人?余子父母存,不养而出离者几何人?士之有田而不使者几何人?吏恶何事士之有田而不耕者几何人?身何事。君臣有位而未有田者几何人?外人之来从而未有田宅者几何家?国子弟之游于外者几何人?贫士之受责于大夫者几何人?官贱行书,身士以家臣自代者几何人?官承吏之无田餼而徒理事者几何人?群臣有位事官大夫者几何人?外人来游在大夫之家者几何人?乡子弟力田为人率者几何人?国子弟之无上事,衣食不节;率子弟不田弋猎者几何人?男女不整齐,乱乡子弟者有乎?问人之贷粟米,有别券者几何家?

问国之伏利其可应人之急者几何所也?人之所害于乡里者何物也?问士之有田宅身在陈列者几何人?余子之胜甲兵有行伍者几何人?问男女有巧伎,能利备用者几何人?处女操工事者几何人?冗国所开口而食者几何人?问一民有几年之食也?问兵车之计几何乘也?牵家马軶家车者几何乘。处士修行。足以教人,可使帅众莅百姓者几何人?士之急难可使者几何人?工之巧,出,足以利军伍,处,可以修城郭补守备者几何人?城粟军粮其可以行几何年也。吏之急难可使者几何人?大夫疏器甲兵、兵车、旌旗、鼓铙、帷幕、帅车之载、几何乘?疏藏器弓弩之张、衣夹铗钩弦之造、戈戟之紧,其厉何若?其宜修而不修者故何视?而造修之官,出器处器之具,宜起而未起者何待?乡师车辎造修之具,其缮何若?工尹伐材用,毋于三时,群材乃植,而造器定冬,完良备用必足。人有余兵,诡陈之行,以慎国常。时简稽帅马牛之肥膌,其老而死者皆举之。其就山薮林泽食荐者几何,出入死生之会几何。若夫城郭之厚薄,沟壑之浅深,门闾之尊卑,宜修而不修者,上必几之。守备之伍,器物不失其具,淫雨而各有处藏。问兵官之吏,国之豪士,其急难足以先后者几何人?夫兵事者危物也,不时而胜,不义而得,未为福也。失谋而败,国之危也。慎谋乃保国。

问所以教选人者何事?问执官都者,其位事几何年矣。所辟草莱有益于家邑者几何矣?所封表以益人之生利者何物也?所筑城郭修墙闭绝通道阨阙深防沟以益人之地守者何所也?所捕盗贼,除人害者几何矣?

制地。君曰:理国之道,地德为首,君臣之礼,父子之亲,覆育万人,官府之藏,彊兵保国,城郭之险,外应四极,具取之地。而市者天地之财具也。而万人之所和而利也。正是道也。民荒无苛人,尽地之职,一保其国。各主异位,毋使谗人乱普,而德营九军之亲。关者,诸侯之陬隧也。而外财之门户也。万人之道行也。明道以重告之。征于关者,勿征于市,征于市者,勿征于关。虚车勿索,徒负勿入,以来远人。十六道同身外事谨,则听其名,视其名,视其色,是其事,稽其德。以观其外则,无敦于权人,以困貌德。国则不惑,行之职也。问于边吏曰:小利害信,小怒伤义,边信伤德,厚和构四国以顺貌德。后乡四极,令守法之官日行,度必明,无失经常。

\hypertarget{header-n380}{%
\subsection{戒}\label{header-n380}}

桓公将东游,问于管仲曰:我游犹轴转斛,南至瑯邪。司马曰:``亦先王之游已。''何谓也?管仲对曰:``先王之游也,春出,原农事之不本者,谓之游。秋出,补人之不足者,谓之夕。夫师行而粮食其民者,谓之亡。从乐而不反者,谓之荒。先王有游夕之业于人,无亡荒之行于身。''桓公退再拜命曰:``宝法也。''管仲复于桓公曰:``无翼而飞者声也,无根而固者情也,无方而富者生也,公亦固情谨声,以严尊生。此谓道之荣。桓公退。再拜,请若此言。管仲复于桓公曰:``任之重者莫如身,涂之畏者莫如口,期而远者莫如年。以重任行畏涂至远期。唯君子乃能矣。''桓公退,再拜之曰:``夫子数以此言者教寡人。''管仲对曰:``滋味动静,生之养也。好恶喜怒哀乐,生之变也。聪明当物,生之德也。是故圣人齐滋味而时动静,御正六气之变。禁止声色之淫。邪行亡乎体,违言不存口。静无定生,圣也。仁从中出,义从外作。仁故不以天下为利,义故不以天下为名。仁故不代王,义故七十而致政。是故圣人上德而下功,尊道而贱物。道德当身故不以物惑。是故,身在草茅之中,而无慑意,南面听天下,而无骄色。如此,而后可以为天下王。所以谓德者。不动而疾,不相告而知,不为而成,不召而至,是德也。故天不动,四时云下,而万物化;君不动,政令陈下,而万功成;心不动,使四肢耳目,而万物情。寡交多亲,谓之知人。寡事成功,谓之知用。闻一言以贯万物,谓之知道。多言而不当,不如其寡也。博学而不自反,必有邪。孝弟者,仁之祖也。忠信者,交之庆也。内不考孝弟,外不正忠信,泽其四经而诵学者,是亡其身者也。''

桓公明日弋在廪,管仲隰朋朝,公望二子,弛弓脱釬,而迎之曰:``今夫槛鹄春北而秋南,而不失其时,夫唯有羽翼以通其意于天下乎?今孤之不得意于天下,非皆二子之忧也。''桓公再言,二子不对,桓公曰:``孤既言矣,二子何不对乎?''管仲对曰:``今夫人患劳,而上使不时,人患饥,而上重敛焉。人患死,而上急刑焉,如此,而又近有色,而远有德。虽槛鹄之有翼,济大水之有舟楫也,其将若君何?''桓公蹙然逡遁。管仲曰:``昔先王之理人也,盖人患劳,而上使之以时,则人不患劳也。人患饥,而上薄敛焉,则人不患饥矣。人患死,而上宽刑焉,则人不患死矣。如此,而近有德而远有色,则四封之内,视君其犹父母邪,四方之外,归君其犹流水乎。公辍射援绥而乘,自御,管仲为左,隰朋参乘,朔月三日,进二子于里官。再拜顿首曰:``孤之闻二子之言也,耳加聪而视加明,于孤不敢独听之,荐之先祖。''管仲隰朋再拜顿首曰:``如君之王也,此非臣之言也,君之教也。''于是管仲与桓公盟誓为令曰:``老弱勿刑。参宥而后弊,关箭而不正市正而不布。山林梁泽,以时禁发,而不正也。''草封泽盐者之归之也譬若市人。三年教人,四年选贤以为长,五年始兴车践乘,遂南伐楚,门傅施城。北伐山戎,出冬葱与戎叔,布之天下,果三匡天子而九合诸侯。

桓公外舍,而不鼎馈。中妇诸子谓宫人盍不出从乎?君将有行,宫人皆出从。公怒曰:``庸谓我有行者?''宫人曰:``贱妾闻之中妇诸子。''公召中妇诸子曰:``女焉闻吾有行也?''对曰:``妾人闻之,君外舍而不鼎馈,非有内忧,必有外患。今君外舍而不鼎馈,君非有内忧也,妾是以知君之将有行也。''公曰:``善!此非吾所与女及也。而言乃至焉,吾是以语女。吾欲致诸侯而不至,为之奈何?''中妇诸子曰:``自妾之身之不为人持接也,未尝得人之布织也。意者更容不审耶?''明日,管仲朝,公告之,管仲曰:``此圣人之言也,君必行也。''

管仲寝疾,桓公往问之曰:``仲父之疾甚矣,若不可讳也不幸而不起此疾,彼政我将安移之?''管仲未对。桓公曰:``鲍叔之为人何如?''管子对曰:``鲍叔君子也,千乘之国,不以其道,予之,不受也。虽然,不可以为政,其为人也,好善而恶恶已甚,见一恶终身不忘。''桓公曰:``然则庸可?''管仲对曰:``隰朋可,朋之为人,好上识而下问,臣闻之,以德予人者,谓之仁;以财予人者,谓之良;以善胜人者,未有能服人者也。以善养人者,未有不服人者也。于国有所不知政,于家有所不知事,则必朋乎。且朋之为人也,居其家不忘公门,居公门不忘其家,事君不二其心,亦不忘其身,举齐国之币。握路家五十室,其人不知也,大仁也哉,其朋乎!''公又问曰:``不幸而失仲父也,二三大夫者,其犹能以国宁乎?''管仲对曰:``君请矍已乎,鲍叔牙之为人也好直,宾胥无之为人也好善,宁戚之为人也能事,孙在之为人也善言。''公曰:``此四子者,其庸能一人之上也?寡人并而臣之,则其不以国宁,何也。''对曰:``鲍叔之为人也好直,而不能以国诎,宾胥无之为人也好善,而不能以国诎。宁戚之为人也能事,而不能以足息。孙在之为人也善言,而不能以信默臣闻之,消息盈虚,与百姓诎信,然后能以国宁,勿已者,朋其可乎!朋之为人也,动必量力,举必量技。''言终,喟然而叹曰:``天之生朋,以为夷吾舌也,其身死,舌焉得生哉?''管仲曰:``夫江黄之国近于楚,为臣死乎,君必归之楚而寄之。君不归,楚必私之,私之而不救也,则不可,救之,则乱自此始矣。''桓公曰:``诺。''管仲又言曰:``东郭有狗啀啀,旦暮欲啮我,猳而不使也,今夫易牙,子之不能爱,将安能爱君?君必去之。''公曰:``诺。''管子又言曰:``北郭有狗啀啀,旦暮欲啮我,猳而不使也,今夫竖刁,其身之不爱,焉能爱君,君必去之。''公曰:``诺。''管子又言曰:``西郭有狗啀啀,旦暮欲啮我,猳而不使也,今夫卫公子开方,去其千乘之太子,而臣事君,是所愿也得于君者,将欲过其千乘也,君必去之。''桓公曰:``诺。''管子遂卒。卒十月,隰朋亦卒。桓公去易牙竖刁卫公子开方。五味不至,于是乎复反易牙。宫中乱,复反竖刁。利言卑辞不在侧,复反卫公子开方。桓公内不量力,外不量交,而力伐四邻。公薨,六子皆求立,易牙与卫公子,内与竖刁,因共杀群吏而立公子无亏,故公死七日不敛,九月不葬,孝公奔宋,宋襄公率诸侯以伐齐,战于甗,大败齐师,杀公子无亏,立孝公而还。襄公立十三年,桓公立四十二年。

\hypertarget{header-n387}{%
\subsection{地图}\label{header-n387}}

凡兵主者必先审知地图轘辕之险。滥车之水名山通谷经川陵陆丘阜之所在,苴草林木蒲苇之所茂道里之远近,城郭之大小,名邑废邑困殖之地必尽知之。地形之出入相错者尽藏之然后可以行军袭邑,举错知先后,不失地利,此地图之常也。

人之众寡,士之精粗,器之功苦尽知之,此乃知形者也,知形不如知能,知能不如知意,故主兵必参具者也,主明、相知、将能之谓参具,故将出令发士,期有日数矣,宿定所征伐之国,使群臣大吏父兄便辟左右不能议成败,人主之任也。论功劳,行赏罚,不敢蔽贤有私行,用货财供给军之求索,使百吏肃敬,不敢解怠行邪,以待君之令,相室之任也。缮器械,选练士,为教服,连什伍,遍知天下,审御机数,此兵主之事也。

\hypertarget{header-n392}{%
\subsection{参患}\label{header-n392}}

凡人主者,猛毅则伐,懦弱则杀,猛毅者何也?轻诛杀人之谓猛毅。懦弱者何也,重诛杀人之谓懦弱。此皆有失彼此。凡轻诛者杀不辜,而重诛者失有皋,故上杀不辜,则道正者不安;上失有皋,则行邪者不变。道正者不安,则才能之人去亡;行邪者不变,则群臣朋党;才能之人去亡,则宜有外难,群臣朋党,则宜有内乱。故曰猛毅者伐,懦弱者杀也。

君之所以卑尊,国之所以安危者,莫要于兵。故诛暴国必以兵,禁辟民必以刑。然则兵者外以诛暴,内以禁邪。故兵者尊主安国之经也,不可废也。若夫世主则不然。外不以兵,而欲诛暴,则地必亏矣。内不以刑,而欲禁邪,则国必乱矣。

故凡用兵之计,三惊当一至,三至当一军,三军当一战;故一期之师,十年之蓄积殚;一战之费,累代之功尽;今交刃接兵而后利之,则战之自胜者也。攻城围邑,主人易子而食之,析骸而爨之,则攻之自拔者也。是以圣人小征而大匡,不失天时,不空地利,用日维梦,其数不出于计。故计必先定而兵出于竟,计未定而兵出于竟,则战之自败,攻之自毁者也。

得众而不得其心,则与独行者同实。兵不完利,与无操者同实,甲不坚密,与俴者同实。弩不可以及远,与短兵同实。射而不能中,与无矢者同实。中而不能入,与无鏃者同实。将徒人,与俴者同实。短兵待远矢,与坐而待死者同实。故凡兵有大论。必先论其器,论其士,论其将,论其主,故曰:``器滥恶不利者,以其士予人也,士不可用者,以其将予人也;将不知兵者,以其主予人也;主不积务于兵者,以其国予人也;故一器成,往夫具,而天下无战心。二器成,惊夫具,而天下无守城。三器成,游夫具,而天下无聚众。''所谓无战心者,知战必不胜,故曰无战心。所谓无守城者,知城必拔,故曰无守城。所谓无聚众者,知众必散,故曰无聚众。

\hypertarget{header-n399}{%
\subsection{制分}\label{header-n399}}

凡兵之所以先争,圣人贤士,不为爱尊爵。道术知能,不为爱官职。巧伎勇力,不为爱重禄。聪耳明目,不为爱金财。故伯夷叔齐,非于死之日而后有名也,其前行多修矣。武王非于甲子之朝而后胜也,其前政多善矣。故小征千里遍知之,筑堵之墙,十人之聚,日五闲之。大征遍知天下。日一闲之。散金财,用聪明也,故善用兵者,无沟垒而有耳目。兵不呼儆,不茍聚,不妄行,不强进,呼儆则敌人戒。茍聚则众不用。妄行则群卒困,强进则锐士挫,故凡用兵者,攻坚则軔乘瑕则神,攻坚则瑕者坚乘瑕则坚者瑕。故坚其坚者,瑕其瑕者。屠牛坦朝解九牛,而刀可以莫铁,则刃游闲也。故天道不行,屈不足从。人事荒乱,以十破百。器备不行,以且击倍。故军争者不行于完城,有道者不行于无君。故莫知其将至也,至而不可圉。莫知其将去也,去而不可止。敌人虽众,不能止待。

治者所道富也,而治未必富也,必知富之事,然后能富。富者所道强也,而富未必强也,必知强之数,然后能强。强者所道胜也,而强未必胜也,必知胜之理;然后能胜。胜者所道制也,而胜未必制也,必知制之分,然后能制。是故治国有器,富国有事,强国有数,胜国有理,制天下有分。

\hypertarget{header-n404}{%
\subsection{君臣上}\label{header-n404}}

为人君者,修官上之道,而不言其中;为人臣者,比官中之事,而不言其外。君道不明,则受令者疑;权度不一,则修义者惑。民有疑惑贰豫之心而上不能匡,则百姓之与间,犹揭表而令之止也。是故能象其道于国家,加之于百姓,而足以饰官化下者,明君也。能上尽言于主,下致力于民,而足以修义从令者,忠臣也。上惠其道,下敦其业,上下相希,若望参表,则邪者可知也。

吏啬夫任事,人啬夫任教。教在百姓,论在不挠,赏在信诚,体之以君臣,其诚也以守战。如此,则人啬夫之事究矣。吏啬夫尽有訾程事律,论法辟、衡权、斗斛、文劾,不以私论,而以事为正。如此,则吏啬夫之事究矣。人啬夫成教、吏啬夫成律之后,则虽有敦悫忠信者不得善也;而戏豫怠傲者不得败也。如此,则人君之事究矣。是故为人君者因其业,乘其事,而稽之以度。有善者,赏之以列爵之尊、田地之厚,而民不慕也。有过者,罚之以废亡之辱、僇死之刑,而民不疾也。杀生不违,而民莫遗其亲者,此唯上有明法,而下有常事也。

天有常象,地有常形,人有常礼。一设而不更,此谓三常。兼而一之,人君之道也;分而职之,人臣之事也。君失其道,无以有其国;臣失其事,无以有其位。然则上之畜下不妄,而下之事上不虚矣。上之畜下不妄,则所出法制度者明也;下之事上不虚,则循义从令者审也。上明下审,上下同德,代相序也。君不失其威,下不旷其产,而莫相德也。是以上之人务德,而下之人守节。义礼成形于上,而善下通于民,则百姓上归亲于主,而下尽力于农矣。故曰:君明、相信、五官肃、士廉、农愚、商工愿、则上下体而外内别也,民性因而三族制也。

夫为人君者,荫德于人者也;为人臣者,仰生于上者也。为人上者,量功而食之以足;为人臣者,受任而处之以教。布政有均,民足于产,则国家丰矣。以劳受禄,则民不幸生;刑罚不颇,则下无怨心;名正分明,则民不惑于道。道也者,上之所以导民也。是故道德出于君,制令传于相,事业程于官,百姓之力也,胥令而动者也。是故君人也者,无贵如其言;人臣也者,无爱如其力。言下力上,而臣主之道毕矣。是故主画之,相守之;相画之,官守之;官画之,民役之;则又有符节、印玺、典法、策籍以相揆也。此明公道而灭奸伪之术也。

论材量能,谋德而举之,上之道也;专意一心,守职而不劳,下之事也。为人君者,下及官中之事,则有司不任;为人臣者,上共专于上,则人主失威。是故有道之君,正其德以莅民,而不言智能聪明。智能聪明者,下之职也;所以用智能聪明者,上之道也。上之人明其道,下之人守其职,上下之分不同任,而复合为一体。

是故知善,人君也;身善,人役也。君身善,则不公矣。人君不公,常惠于赏,而不忍于刑,是国无法也。治国无法,则民朋党而下比,饰巧以成其私。法制有常,则民不散而上合,竭情以纳其忠。是以不言智能,而顺事治、国患解,大臣之任也。不言于聪明,而善人举,奸伪诛、视听者众也。

是以为人君者,坐万物之原,而官诸生之职者也。选贤论材,而待之以法。举而得其人,坐而收其福,不可胜收也。官不胜任,奔走而奉其败事,不可胜救也。而国未尝乏于胜任之士,上之明适不足以知之。是以明君审知胜任之臣者也。故曰:主道得,贤材遂,百姓治。治乱在主而已矣。

故曰:主身者,正德之本也;官(治)者,耳目之制也。身立而民化,德正而官治。治官化民,其要在上。是故君子不求于民。是以上及下之事谓之矫,下及上之事谓之胜。为上而矫,悖也;为下而胜,逆也。国家有悖逆反迕之行,有土主民者失其纪也。是故别交正分之谓理,顺理而不失之谓道,道德定而民有轨矣。有道之君者,善明设法,而不以私防者也。而无道之君,既已设法,则舍法而行私者也。为人上者释法而行私,则为人臣者援私以为公。公道不违,则是私道不违者也。行公道而托其私焉,寖久而不知,奸心得无积乎?奸心之积也,其大者有侵逼杀上之祸、其小者有比周内争之乱。此其所以然者,由主德不立,而国无常法也。主德不立,则妇人能食其意;国无常法,则大臣敢侵其势。大臣假于女之能,以规主情;妇人嬖宠假于男之知,以援外权。于是乎外夫人而危太子,兵乱内作,以召外寇。此危君之征也。

是故有道之君,上有五官以牧其民,则众不敢逾轨而行矣;下有五横以揆其官,则有司不敢离法而使矣。朝有定度衡仪,以尊主位,衣服緷絻,尽有法度,则君体法而立矣。君据法而出令,有司奉命而行事,百姓顺上而成俗,著久而为常,犯俗离教者,众共奸之,则为上者佚矣。

天子出令于天下,诸侯受令于天子,大夫受令于君,子受令于父母,下听其上,弟听其兄,此至顺矣。衡石一称,斗斛一量,丈尺一綧制,戈兵一度,书同名,车同轨,此至正也。从顺独逆,从正独辟,此犹夜有求而得火也,奸伪之人,无所伏矣。此先王之所以一民心也。是故天子有善,让德于天;诸侯有善,庆之于天子;大夫有善,纳之于君;民有善,本于父,庆之于长老。此道法之所从来,是治本也。是故岁一言者,君也;时省者,相也;月稽者,官也;务四支之力,修耕农之业以待令者,庶人也。是故百姓量其力于父兄之间,听其言于君臣之义,而官论其德能而待之。大夫比官中之事,不言其外;而相为常具以给之。相总要,者官谋士,量实义美,匡请所疑。而君发其明府之法瑞以稽之,立三阶之上,南面而受要。是以上有余日,而官胜其任;时令不淫,而百姓肃给。唯此上有法制,下有分职也。

道者,诚人之姓也,非在人也。而圣王明君,善知而道之者也。是故治民有常道,而生财有常法。道也者,万物之要也。为人君者,执要而待之,则下虽有奸伪之心,不敢杀也。夫道者虚设,其人在则通,其人亡则塞者也。非兹是无以理人,非兹是无以生财,民治财育,其福归于上。是以知明君之重道法而轻其国也。故君一国者,其道君之也。王天下者,其道王之也。大王天下,小君一国,其道临之也。是以其所欲者能得诸民,其所恶者能除诸民。所欲者能得诸民,故贤材遂;所恶者能除诸民,故奸伪省。如冶之于金,陶之于埴,制在工也。

是故将与之,惠厚不能供;将杀之,严威不能振。严威不能振,惠厚不能供,声实有间也。有善者不留其赏,故民不私其利;有过者不宿其罚,故民不疾其威。威罚之制,无逾于民,则人归亲于上矣。如天雨然,泽下尺,生上尺。

是以官人不官,事人不事,独立而无稽者,人主之位也。先王之在天下也,民比之神明之德。先王善牧之于民者也。夫民别而听之则愚,合而听之则圣。虽有汤武之德,复合于市人之言。是以明君顺人心,安情性,而发于众心之所聚。是以令出而不稽,刑设而不用。先王善与民为一体。与民为一体,则是以国守国,以民守民也。然则民不便为非矣。

虽有明君,百步之外,听而不闻;间之堵墙,窥而不见也。而名为明君者,君善用其臣,臣善纳其忠也。信以继信,善以传善。是以四海之内,可得而治。是以明君之举其下也,尽知其短长,知其所不能益,若任之以事。贤人之臣其主也,尽知短长与身力之所不至,若量能而授官。上以此畜下,下以此事上,上下交期于正,则百姓男女皆与治焉。

\hypertarget{header-n421}{%
\subsection{君臣下}\label{header-n421}}

古者未有君臣上下之别,未有夫妇妃匹之合,兽处群居,以力相征。于是智者诈愚,强者凌弱,老幼孤独不得其所。故智者假众力以禁强虐,而暴人止。为民兴利除害,正民之德,而民师之。是故道术德行,出于贤人。其从义理兆形于民心,则民反道矣。名物处,违是非之分,则赏罚行矣。上下设,民生体,而国都立矣。是故国之所以为国者,民体以为国;君之所以为君者,赏罚以为君。

致赏则匮,致罚则虐。财匮而令虐,所以失其民也。是故明君审居处之教,而民可使居治、战胜、守固者也。夫赏重,则上不给也;罚虐,则下不信也。是故明君饰食饮吊伤之礼,而物属之者也。是故厉之以八政,旌之以衣服,富之以国裹,贵之以王禁,则民亲君可用也。民用,则天下可致也。天下道其道则至,不道其道则不至也。夫水波而上,尽其摇而复下,其势固然者也。故德之以怀也,威之以畏也,则天下归之矣。有道之国,发号出令,而夫妇尽归亲于上矣;布法出宪,而贤人列士尽功能于上矣。千里之内,束布之罚,一亩之赋,尽可知也。治斧钺者不敢让刑,治轩冕者不敢让赏,坟然若一父之子,若一家之实,义礼明也。

夫下不戴其上,臣不戴其君,则贤人不来。贤人不来,则百姓不用。百姓不用,则天下不至。故曰:德侵则君危,论侵则有功者危,令侵则官危,刑侵则百姓危。而明君者,审禁淫侵者也。上无淫侵之论,则下无异幸之心矣。

为人君者,倍道弃法,而好行私,谓之乱。为人臣者,变故易常,而巧官以谄上,谓之腾。乱至则虐,腾至则北。四者有一至,败敌人谋之。则故施舍优犹以济乱,则百姓悦。选贤遂材,而礼孝弟,则奸伪止。要淫佚,别男女,则通乱隔。贵贱有义,伦等不逾,则有功者劝。国有常式,故法不隐,则下无怨心。此五者,兴德匡过、存国定民之道也。

夫君人者有大过,臣人者有大罪,国所有也,民所君也,有国君民而使民所恶制之,此一过也。民有三务,不布其民,非其民也。民非其民,则不可以守战。此君人者二过也。夫臣人者,受君高爵重禄,治大官。倍其官,遗其事,穆君之色,从其欲,阿而胜之,此臣人之大罪也。君有过而不改,谓之倒;臣当罪而不诛,谓之乱。君为倒君,臣为乱臣,国家之衰也,可坐而待之。是故有道之君者执本,相执要,大夫执法以牧其群臣,群臣尽智竭力以役其上。四守者得则治,易则乱。故不可不明设而守固。

昔者,圣王本厚民生,审知祸福之所生。是故慎小事微,违非索辩以根之。然则躁作、奸邪、伪诈之人,不敢试也。此礼正民之道也。

古者有二言:``墙有耳,伏寇在侧。''墙有耳者,微谋外泄之谓也;伏寇在侧者,沈疑得民之道也。微谋之泄也,狡妇袭主之请而资游慝也。沈疑之得民也者,前贵而后贱者为之驱也。明君在上,便僻不能食其意,刑罚亟近也;大臣不能侵其势,比党者诛,明也。为人君者,能远谗谄,废比党,淫悖行食之徒,无爵列于朝者,此止诈拘奸、厚国存身之道也。

为人上者,制群臣百姓通,中央之人和,是以中央之人,臣主之参。制令之布于民也,必由中央之人。中央之人,以缓为急,急可以取威;以急为缓,缓可以惠民。威惠迁于下,则为人上者危矣。贤不肖之知于上,必由中央之人。财力之贡于上,必由中央之人。能易贤不肖而可威党于下。有能以民之财力上陷其主,而可以为劳于下。兼上下以环其私,爵制而不可加,则为人上者危矣。先其君以善者,侵其赏而夺之实者也;先其君以恶者,侵其刑而夺之威者也;讹言于外者,胁其君者也;郁令而不出者,幽其君者也。四者一作而上(下)不知也,则国之危,可坐而待也。

神圣者王,仁智者君,武勇者长,此天之道,人之情也。天道人情,通者质,宠者从,此数之因也。是故始于患者,不与其事;亲其事者,不规其道。是以为人上者患而不劳也,百姓劳而不患也。君臣上下之分索,则礼制立矣。是故以人役上,以力役明,以刑役心,此物之理也。心道进退,而形道滔赶。进退者主制,滔赶者主劳。主劳者方,主制者圆。圆者运,运者通,通则和。方者执,执者固,固则信。君以利和,臣以节信,则上下无邪矣。故曰:君人者制仁,臣人者守信。此言上下之礼也。

君之在国都也,若心之在身体也。道德定于上,则百姓化于下矣。戒心形于内,则容貌动于外矣,正也者,所以明其德。知得诸己,知得诸民,从其理也。知失诸民,退而修诸己,反其本也。所求于己者多,故德行立。所求于人者少,故民轻给之。故君人者上注,臣人者下注。上注者,纪天时,务民力。下注者,发地利,足财用也。故能饰大义,审时节,上以礼神明,下以义辅佐者,明君之道。能据法而不阿,上以匡主之过,下以振民之病者,忠臣之所行也。

君子食于道,则义审而礼明,义审而礼明,则伦等不逾,虽有偏卒之大夫,不敢有幸心,则上无危矣。齐民食于力则作本,作本者众,农以听命。是以明君立世,民之制于上,犹草木之制于时也。故民迂则流之,民流通则迂之。决之则行,塞之则止。虽有明君,能决之,又能塞之。决之则君子行于礼,塞之则小人笃于农。君子行于礼,则上尊而民顺;小民笃于农,则财厚而备足。上尊而民顺,财厚而备足,四者备体,顷时而王不难矣。

四肢六道,身之体也。四正五官,国之体也。四肢不通,六道不达,曰失。四正不正,五官不官,曰乱。是故国君聘妻于异姓,设为姪娣、命妇、宫女,尽有法制,所以治其内也。明男女之别,昭嫌疑之节,所以防其奸也。是以中外不通,谗慝不生;妇言不及官中之事,而诸臣子弟无宫中之交,此先王所以明德圉奸,昭公威私也。

明立宠设,不以逐子伤义。礼私爱欢,势不并论。爵位虽尊,礼无不行。选为都佼,冒之以衣服,旌之以章旗,所以重其威也。然则兄弟无间郄,谗人不敢作矣。

故其立相也,陈功而加之以德,论劳而昭之以法,参伍相德而周举之,尊势而明信之。是以下之人无谏死之誋,而聚立者无郁怨之心,如此,则国平而民无慝矣。其选贤遂材也,举德以就列,不类无德;举能以就官,不类无能;以德弇劳,不以伤年。如此,则上无困,而民不幸生矣。

国之所以乱者四,其所以亡者二。内有疑妻之妾,此宫乱也;庶有疑適之子,此家乱也;朝有疑相之臣,此国乱也;任官无能,此众乱也。四者无别,主失其体。群官朋党,以怀其私,则失族矣;国之几臣,阴约闭谋以相待也,则失援矣。失族于内,失援于外,此二亡也。故妻必定,子必正,相必直立以听,官必中信以敬。故曰:有宫中之乱,有兄弟之乱,有大臣之乱,有中民之乱,有小人之乱。五者一作,则为人上者危矣。宫中乱曰妒纷,兄弟乱曰党偏,大臣乱曰称述、中民乱曰詟谆,小民乱曰财匮。财匮生薄,詟谆生慢,称述、党偏、妒纷生变。

故正名稽疑,刑杀亟近,则内定矣。顺大臣以功,顺中民以行,顺小民以务、则国丰矣。审天时,物地生,以辑民力;禁淫务:劝农功,以职其无事,则小民治矣。上稽之以数,下十伍以征,近其罪伏,以固其意。乡树之师,以遂其学。官之以其能,及年而举,则士反行矣。称德度功,劝其所能,若稽之以众风,若任以社稷之任。若此,则土反于情矣。

\hypertarget{header-n440}{%
\subsection{小称}\label{header-n440}}

管子曰:``身不善之患,毋患人莫己知。丹青在山,民知而取之;美珠在渊,民知而取之。是以我有过为,而民毋过命。民之观也察矣,不可遁逃以为不善。故我有善,则立誉我;我有过,则立毁我。当民之毁誉也,则莫归问于家矣,故先王畏民。操名从人,无不强也;操名去人,无不弱也。虽有天子诸侯,民皆操名而去之,则捐其地而走矣,故先王畏民。在于身者孰为利?气与目为利。圣人得利而托焉,故民重而名遂。我亦托焉,圣人托可好,我托可恶,以来美名,又可得乎?我托可恶,爱且不能为我能也,毛嫱、西施,天下之美人也,盛怨气于面,不能以为可好。我且恶面而盛怨气焉,怨气见于面,恶言出于口,去恶充以求美名,又可得乎?甚矣,百姓之恶人之有余忌也,是以长者断之,短者续之,满者洫之,虚者实之。''

管子曰:``善罪身者,民不得罪也;不能罪身者,民罪之。故称身之过者,强也;洽身之节者,惠也;不以不善归人者,仁也。故明王有过则反之于身,有善则归之于民。有过而反之身则身惧,有善而归之民则民喜。往喜民,来惧身,此明王之所以治民也。今夫桀纣不然,有善则反之于身,有过则归之于民。归之于民则民怒,反之于身则身骄。往怒民,来骄身,此其所以失身也。故明王惧声以感耳,惧气以感目。以此二者有天下矣,可毋慎乎?匠人有以感斤,故绳可得断也,羿有以感弓矢,故彀可得中也。造父有以感辔策,故遬兽可及,远道可致。天下者,无常乱,无常治。不善人在则乱,善人在则治,在于既善,所以感之也。''

管子曰:``修恭逊、敬爱、辞让、除怨、无争以相逆也,则不失于人矣。尝试多怨争利,相为不逊,则不得其身。大哉!恭逊敬爱之道。吉事可以入察,凶事可以居丧。大以理天下而不益也,小以治一人而不损也。尝试往之中国、诸夏、蛮夷之国,以及禽兽昆虫、皆待此而为治乱。泽之身则荣,去之身则辱。审行之身毋怠,虽夷貉之民,可化而使之爱。审去之身,虽兄弟父母,可化而使之恶。故之身者,使之爱恶;名者,使之荣辱。此其变名物也,如天如地,故先王曰道。''

管仲有病,桓公往问之曰:``仲父之病病矣,若不可讳而不起此病也,仲父亦将何以诏寡人?``管仲对曰:``微君之命臣也,故臣且谒之,虽然,君犹不能行也。''公曰:``仲父命寡人东,寡人东;令寡人西,寡人西。仲父之命于寡人,寡人敢不从乎?''管仲摄衣冠起,对曰:``臣愿君之远易牙、竖刁、堂巫、公子开方。夫易牙以调和事公,公曰:惟烝婴儿之未尝。于是烝其首子而献之公。人情非不爱其子也,于子之不爱,将何有于公?公喜宫而妒,竖刁自刑而为公治内。人情非不爱其身也,于身之不爱,将何有于公?公子开方事公,十五年不归视其亲,齐卫之间,不容数日之行。于亲之不爱,焉能有子公?臣闻之,务为不久,盖虚不长。其生不长者,其死必不终。''桓公曰:``善。''管仲死,已葬。公憎四子者,废之官。逐堂巫而苛病起兵,逐易牙而味不至,逐竖刁而宫中乱,逐公子开方而朝不治。桓公曰:``嗟!圣人固有悖乎!''乃复四子者。处期年,四子作难,围公一室不得出。有一妇人、遂从窦入,得至公所。公曰:``吾饥而欲食,渴而欲饮,不可得,其故何也?''妇人对曰:``易牙、竖刁、堂巫、公子开方四人分齐国,涂十日不通矣。公子开方以书社七百下卫矣,食将不得矣。''公曰:``嗟兹乎!圣人之言长乎哉!死者无知则已,若有知,吾何面目以见仲父于地下!''乃援素幭以裹首而绝。死十一日,虫出于户,乃知桓公之死也。葬以杨门之扇。桓公之所以身死十一日,虫出户而不收者,以不终用贤也。

桓公、管仲、鲍叔牙、宁戚四人饮,饮酣,桓公谓鲍叔牙曰:``阖不起为寡人寿乎?``鲍叔牙奉杯而起曰:``使公毋忘出如莒时也,使管子毋忘束缚在鲁也,使宁戚毋忘饭牛车下也。''桓公辟席再拜曰:``寡人与二大夫能无忘夫子之言,则国之社稷必不危矣。''

管子曰:``修恭逊、敬爱、辞让、除怨、无争以相逆也,则不失于人矣。尝试多怨争利,相为不逊,则不得其身。大哉!恭逊敬爱之道。吉事可以入察,凶事可以居丧。大以理天下而不益也,小以治一人而不损也。尝试往之中国、诸夏、蛮夷之国,以及禽兽昆虫、皆待此而为治乱。泽之身则荣,去之身则辱。审行之身毋怠,虽夷貉之民,可化而使之爱。审去之身,虽兄弟父母,可化而使之恶。故之身者,使之爱恶;名者,使之荣辱。此其变名物也,如天如地,故先王曰道。''

管仲有病,桓公往问之曰:``仲父之病病矣,若不可讳而不起此病也,仲父亦将何以诏寡人?``管仲对曰:``微君之命臣也,故臣且谒之,虽然,君犹不能行也。''公曰:``仲父命寡人东,寡人东;令寡人西,寡人西。仲父之命于寡人,寡人敢不从乎?''管仲摄衣冠起,对曰:``臣愿君之远易牙、竖刁、堂巫、公子开方。夫易牙以调和事公,公曰:惟烝婴儿之未尝。于是烝其首子而献之公。人情非不爱其子也,于子之不爱,将何有于公?公喜宫而妒,竖刁自刑而为公治内。人情非不爱其身也,于身之不爱,将何有于公?公子开方事公,十五年不归视其亲,齐卫之间,不容数日之行。于亲之不爱,焉能有子公?臣闻之,务为不久,盖虚不长。其生不长者,其死必不终。''桓公曰:``善。''管仲死,已葬。公憎四子者,废之官。逐堂巫而苛病起兵,逐易牙而味不至,逐竖刁而宫中乱,逐公子开方而朝不治。桓公曰:``嗟!圣人固有悖乎!''乃复四子者。处期年,四子作难,围公一室不得出。有一妇人、遂从窦入,得至公所。公曰:``吾饥而欲食,渴而欲饮,不可得,其故何也?''妇人对曰:``易牙、竖刁、堂巫、公子开方四人分齐国,涂十日不通矣。公子开方以书社七百下卫矣,食将不得矣。''公曰:``嗟兹乎!圣人之言长乎哉!死者无知则已,若有知,吾何面目以见仲父于地下!''乃援素幭以裹首而绝。死十一日,虫出于户,乃知桓公之死也。葬以杨门之扇。桓公之所以身死十一日,虫出户而不收者,以不终用贤也。

桓公、管仲、鲍叔牙、宁戚四人饮,饮酣,桓公谓鲍叔牙曰:``阖不起为寡人寿乎?``鲍叔牙奉杯而起曰:``使公毋忘出如莒时也,使管子毋忘束缚在鲁也,使宁戚毋忘饭牛车下也。''桓公辟席再拜曰:``寡人与二大夫能无忘夫子之言,则国之社稷必不危矣。''

\hypertarget{header-n451}{%
\subsection{四称}\label{header-n451}}

桓公问于管子曰:``寡人幼弱惛愚,不通诸侯四邻之义,仲父不当尽语我昔者有道之君乎?吾亦鉴焉。''管子对曰:``夷吾之所能与所不能,尽在君所矣,君胡有辱令?''桓公又问曰:``仲父,寡人幼弱惛愚,不通四邻诸侯之义,仲父不当尽告我昔者有道之君乎?吾亦鉴焉。''管子对曰:``夷吾闻之于徐伯曰,昔者有道之君,敬其山川、宗庙、社稷,及至先故之大臣,收聚以忠而大富之。固其武臣,宣用其力。圣人在前,贞廉在侧,竟称于义,上下皆饰。形正明察,四时不贷,民亦不忧,五谷蕃殖。外内均和,诸侯臣伏,国家安宁,不用兵革。受币帛,以怀其德;昭受其令,以为法式。此亦可谓昔者有道之君也。''桓公曰:``善哉!''

桓公曰:``仲父既己语我昔者有道之君矣,不当尽语我昔者无道之君乎?吾亦鉴焉。''管子对曰:``今若君之美好而宣通也,既官职美道,又何以闻恶为?''桓公曰:``是何言邪?以繬缁缘繬,吾何以知其美也?以素缘素,吾何以知其善也?仲父已语我其善,而不语我其恶,吾岂知善之为善也?''管子对曰:``夷吾闻之徐伯曰,昔者无道之君,大其宫室,高其台榭,良臣不使,谗贼是舍。有家不治,借人为图,政令不善,墨墨若夜,辟若野兽,无所朝处,不修天道,不鉴四方,有家不治,辟若生狂,众所怨诅,希不灭亡。进其谀优,繁其钟鼓,流于博塞,戏其工瞽。诛其良臣,敖其妇女,撩猎毕弋,暴遇诸父,驰骋无度,戏乐笑语。式政既輮,刑罚则烈。内削其民,以为攻伐,辟犹漏釜,岂能无竭。此亦可谓昔者无道之君矣。''桓公曰:``善哉!''

桓公曰:``仲父既已语我昔者有道之君与昔者无道之君矣,仲父不当尽语我昔者有道之臣乎?吾以鉴焉。''管子对曰:``夷吾闻之于徐伯曰,昔者有道之臣,委质为臣,不宾事左右;君知则仕,不知则已。若有事,必图国家,遍其发挥。循其祖德,辩其顺逆,推育贤人,谗慝不作。事君有义,使下有礼,贵贱相亲,若兄若弟,忠于国家,上下得体。居处则思义,语言则谋谟,动作则事。居国则富,处军则克,临难据事,虽死不悔。近君为拂,远君为辅,义以与交,廉以与处。临官则治,酒食则慈,不谤其君,不毁其辞。君若有过,进谏不疑;君若有忧,则臣服之。此亦可谓昔者有道之臣矣。''桓公曰:``善哉!''

桓公曰:``仲父既以语我昔者有道之臣矣,不当尽语我昔者无道之臣乎?吾亦鉴焉。''管子对曰:``夷吾闻之于徐伯曰,昔者无道之臣,委质为臣,宾事左右;执说以进,不蕲亡己;遂进不退,假宠鬻贵。尊其货贿,卑其爵位;进曰辅之,退曰不可,以败其君,皆曰非我。不仁群处,以攻贤者,见贤若货,见贱若过。贪于货贿,竟于酒食,不与善人,唯其所事。倨敖不恭,不友善士,谗贼与斗,不弥人争,唯趣人诏。湛湎于酒,行义不从。不修先故,变易国常,擅创为令,迷或其君,生夺之政,保贵宠矜。迁损善士,捕援货人,入则乘等,出则党骈,货贿相入,酒食相亲,俱乱其君。君若有过,各奉其身。此亦谓昔者无道之臣。''桓公曰:``善哉!''

\hypertarget{header-n458}{%
\subsection{侈靡}\label{header-n458}}

问曰:``古之时与今之时同乎?''曰:``同。''``其人同乎?不同乎?''曰:``不同。''可与?政其诛。俈尧之时,混吾之美在下,其道非独出人也。山不童而用赡,泽不獘而养足。耕以自养,以其余应良天子,故平。牛马之牧不相及,人民之俗不相知,不出百里而来足,故卿而不理,静也。其狱一踦腓一踦屦而当死。今周公断指满稽,断首满稽,断足满稽,而死民不服,非人性也,敝也。地重人载,毁敝而养不足,事末作而民兴之;是以下名而上实也,圣人者,省诸本而游诸乐,大昏也,博夜也。

问曰:``兴时化若何?''莫善于侈靡;贱有实,敬无用,则人可刑也。故贱粟米而如敬珠玉,好礼乐而如贱事业。本之殆也,珠者阴之阳也,故胜火。玉者阴之阴也,故胜水。其化如神。故天子臧珠玉,诸侯臧金石,大夫畜狗马,百姓臧布帛。不然,则强者能守之,智者能牧之,贱所贵而贵所贱。不然,鳏寡独老不与得焉,均之始也。

政与教庸急?管子曰:夫政教相似而殊方,若夫教者,标然若秋云之远,动人心之悲;蔼然若夏之静云,乃及人之体,□然若謞之静。动人意以怨,荡荡若流水,使人思之。人所生往,教之始也,身必备之。辟之若秋云之始见,贤者不肖者化焉。敬而待之,爱而使之,若樊神山祭之。贤者少。不肖者多。使其贤,不肖恶得不化。今夫政则少则,若夫成形之征者也,去则少可使人乎。

用贫与富,何如而可,曰:甚富不可使,甚贫不知耻,水平而不流,无源则遫竭,云平而雨不甚,无委云,雨则遫已。政平而无威,则不行。爱而无亲则流。亲左有用,无用则辟之,若相为有兆怨。上短下长,无度而用,则危本不称。

而祀谭次祖,犯诅渝盟伤言。敬祖祢,尊始也。齐约之信,论行也。尊天地之理,所以论威也。薄德之君之府囊也。必因成形而论于人,此政行也,可以王乎?

请问用之若何?必辨于天地之道,然后功名可以殖。辨于地利,而民可富。通于侈靡,而士可戚。君亲自好事,强以立断,仁以好任。人君寿以政年,百姓不夭厉,六畜鞍育,五谷鞍熟,然后民力可得用。邻国之君俱不贤,然后得王。

俱贤若何?曰:忽然易卿而移,忽然易事而化,变而足以成名。承獘而民劝之,慈种而民富,应言待感,与物俱长,故日月之明,应风雨而种。天之所覆,地之所载,斯民之良也,不有而丑天地,非天子之事也。民变而不能变,是棁之傅革,有革而不能革,不可服。民死信,诸侯死化。

请问诸侯之化獘,獘也者,家也。家也者,以因人之所重而行之。吾君长来猎君长虎豹之皮用。功力之君上金玉币,好战之君上甲兵。甲兵之本,必先于田宅。今吾君战,则请行民之所重。

饮食者也,侈乐者也,民之所愿也,足其所欲,赡其所愿,则能用之耳。今使衣皮而冠角食野草,饮野水,庸能用之?伤心者不可以致功。故尝至味,而罢至乐。而雕卵然后瀹之,雕橑然后爨之。丹沙之穴不塞,则商贾不处。富者靡之,贫者为之,此百姓之怠生百振而食非,独自为也,为之畜化。

用其臣者,予而夺之,使而辍之,徒以而富之,父系而伏之,予虚爵而骄之。收其春秋之时而消之,有集礼我而居之。时举其强者以誉之。强而可使服事。辩以辩辞,智以招请,廉以摽人,坚强以乘六,广其德以轻上,位不能使之而流徙,此谓国亡之郤。故法而守常,尊礼而变俗,上信而贱文,好缘而好駔,此谓成国之法也。为国者,反民性,然后可以与民戚,民欲佚,而教以劳。民欲生,而教以死。劳教定而国富,死教定而威行。

圣人者,阴阳理,故平外而险中;故信其情者伤其神,美其质者伤其文,化之美者应其名,变其美者应其时,不能兆其端者菑及之。故缘地之利,承从天之指,辱举其死,开国闭辱,知其缘地之利者,所以参天地之吉纲也;承从天之指者,动必明。辱举其死者,与其失人同公事,则道必行。开其国门者,玩之以善言。柰其斝辱,知神次者,操牺牲与其珪璧,以执其斝。家小害,以小胜大。员其中,辰其外。而复畏强,长其虚,而物正以视其中情。

公曰:国门则塞,百姓谁敢敖,胡以备之?择天下之所宥,择鬼之所当,择人天之所戴,而前付其身,此所以安之也。强与短而立,齐国之若何?高予之名而举之,重予之官而危之,因责其能以随之,犹傶则疏之,毋使人图之,犹疏则数之,毋使人曲之,此所以为之也。

大有臣甚大,将反为害,吾欲优患除害,将小能察大,为之奈何?潭根之毋伐,固蒂之毋乂,深黎之毋涸,不仪之毋助,章明之毋灭,生荣之毋失。十言者不胜此一,虽凶必吉,故平以满。

无事而总,以待有事,而为之若何?积者立余日而侈,美车马而驰,多酒醴而靡,千岁毋出食,此谓本事。县人有主,人此治用,然而不治,积之市,一人积之下,一人积之上,此谓利无常。百姓无宝,以利为首。一上一下,唯利所处。利然后能通,通然后成国。利静而不化,观其所出,从而移之。

视其不可使,因以为民等。择其好名,因使长民;好而不已,是以为国纪。功未成者,不可以独名;事未道者,不可以言名。成功然后可以独名,事道然后可以言名,然后可以承致酢。

先其士者之为自犯,后其民者之为自赡。轻国位者国必败,疏贵戚者谋将泄。毋仕异国之人,是为失经。毋数变易,是为败成。大臣得罪,勿出封外,是为漏情。毋数据大臣之家而饮酒,是为使国大消。三尧在,臧于县,返于连,比若是者,必从是儡亡乎!辟之若尊觯,未胜其本,亡流而下不平。令苟下不治,高下者不足以相待,此谓杀。

事立而坏,何也?兵远而畏,何也?民已聚而散,何也?辍安而危,何也?功成而不信者,殆;兵强而无义者,残;不谨于附近而欲求远者,兵不信。略近臣合于其远者,立。亡国之起,毁国之族,则兵远而不畏。国小而修大,仁而不利,犹有争名者,累哉是也!乐聚之力,以兼人之强,以待其害,虽聚必散。大王不恃众而自恃,百姓自聚;供而后利之,成而无害。疏戚而好外,企以仁而谋泄,贱寡而好大,此所以危。

众而约,实取而言让,行阴而言阳,利人之有祸,言人之无患,吾欲独有是,若何?是故之时,陈财之道可以行。今也利散而民察,必放之身然后行。公曰:谓何?长丧以毁其时,重送葬以起身财,一亲往,一亲来,所以合亲也。此谓众约。问,用之若何?巨瘗堷,所以使贫民也;美垄墓,所以使文明也;巨棺椁,所以起木工也;多衣衾,所以起女工也。犹不尽,故有次浮也,有差樊,有瘗藏。作此相食,然后民相利,守战之备合矣。

乡殊俗,国异礼,则民不流矣;不同法,则民不困;乡丘老不通睹,诛流散,则人不眺安乡乐宅,享祭而讴吟称号者皆诛,所以留民俗也。断方井田之数,乘马甸之众,制之。陵溪立鬼神而谨祭。皆以能别以为食数,示重本也。

故地广千里者,禄重而祭尊。其君无余地与他若一者,从而艾之。君始者艾若一者,从乎杀。与于杀若,一者从者艾若一者,从于杀。与于杀若,一者从无封始,王者上事,霸者生功,言重本。是为十禺,分免而不争,言先人而自后也。

官礼之司,昭穆之离先后功器事之治,尊鬼而守故;战事之任,高功而下死;本事,食功而省利;劝臣,上义而不能与小利。五官者,人争其职,然后君闻。

祭之,时上贤者也,故君臣掌。君臣掌则上下均,此以知上贤无益也,其亡兹适。上贤者亡,而役贤者昌。上义以禁暴,尊祖以敬祖,聚宗以朝杀,示不轻为主也。载祭明置,高子闻之,以告中寝诸子,中寝诸子告寡人,舍朝不鼎馈,中寝诸子告宫中女子曰,公将有行,故不送公,公言无行,女安闻之,曰:闻之中寝诸子,索中寝诸子而问之,寡人无行,女安闻之,吾闻之先人,诸侯舍于朝不鼎馈者,非有外事,必有内忧。公曰:吾不欲与汝及若。女言至焉,不得毋与女及若言,吾欲致诸侯,诸侯不至若何哉?女子不辩于致诸侯,自吾不为污杀之事人,布织不可得而衣,故虽有圣人恶用之。

能摩故道新道,定国家,然后化时乎?国贫而鄙富,苴美于朝市国;国富而鄙贫,莫尽如市。市也者,劝也。劝者,所以起。本善而末事起。不侈,本事不得立。

贤举能不可得,恶得伐不服?用百夫无长,不可临也;干乘有道,不可修也。夫纣在上,恶得伐不得?钧则战,守则攻,百盖无筑,千聚无社,谓之陋,一举而取。天下有一事之时也,万诸侯钧,万民无听,上位不能为功更制,其能王乎?

缘故修法,以政治道,则约杀子吾君,故取夷吾谓替。公曰:何若?对曰:以同。其日久临,可立而待。鬼神不明,囊橐之食无报,明厚德也。沈浮,示轻财也。先立象而定期,则民从之;故为祷朝缕绵,明轻财而重名。公曰:同临?所谓同者,其以先后智渝者也。钧同财争,依则说,十则从服,万则化。成功而不能识,而民期然后,成形而更名,则临矣。

请问为边若何?对曰:夫边日变,不可以常知观也。民未始变而是变,是为自乱。请问诸边而参其乱,任之以事,因其谋。方百里之地,树表相望者,丈夫走祸,妇人备食,内外相备。春秋一日,败曰千金,称本而动。候人不可重也,唯交于上,能必于边之辞。行人可不有私,不有私,所以为内因也。使能者有主,矣而内事。

万世之国,必有万世之实。必因天地之道,无使其内使其外,使其小毋使其大。弃其国宝使其大,贵一与而圣;称其宝使其小,可以为道。能则专,专则佚。椽能逾,则椽于逾。能宫,则不守而不散。众能,伯;不然,将见对。

君子者,勉于糺人者也,非见糺者也。故轻者轻,重者重,前后不慈。凡轻者操实也,以轻则可使;重不可起轻,轻重有齐。重以为国,轻以为死。毋全禄,贫国而用不足;毋全赏,好德恶亡使常。

请问先合于天下而无私怨,犯强而无私害,为之若何?对曰:国虽强,令必忠以义;国虽弱,令必敬以哀。强弱不犯,则人欲听矣。先人而自后而无以为仁也,加功于人而勿得,所橐者远矣,所争者外矣。明无私交,则无内怨;与大则胜,私交众则怨杀。

夷吾也,如以予人财者,不如毋夺时;如以予人食者,不如毋夺其事,此谓无外内之患。事故也,君臣之际也;礼义者,人君之神也。且君臣之属,也;亲戚之爱,性也。使君亲之察同索,属故也。使人君不安者,属际也,不可不谨也。

贤不可威,能不可留,杜事之于前,易也。水鼎之汩也,人聚之;壤地之美也,人死之。若江湖之大也,求珠贝者,不令也。逐神而远热,交觯者不处,兄遗利夫!事左中国之人,观危国过君而弋其能者,岂不几于危社主哉!

利不可法,故民流;神不可法,故事之。天地不可留,故动,化故从新。是故得天者高而不崩,得人者卑而不可胜。是故圣人重之,人君重之。故至贞生至信,至言往至绞。生至自有道,不务以文胜情,不务以多胜少,不动则望有廧,旬身行。

法制度量,王者典器也;执故义道,畏变也。天地若夫神之动。化变者也,天地之极也。能与化起而王用,则不可以道山也。仁者善用,智者善用,非其人,则与神往矣。

衣食之于人也,不可以一日违也,亲戚可以时大也。是故圣人万民艰处而立焉。人死则易云,生则难合也。故一为赏,再为常,三为固然。其小行之则俗也,久之则礼义。故无使下当上必行之,然后移商人于国,非用人也,不择乡而处,不择君而使,出则从利,入则不守。国之山林也,则而利之。市塵之所及,二依其本。故上侈而下靡,而君臣相上下相亲,则君臣之财不私藏。然则贪动枳而得食矣。徙邑移市,亦为数一。

问曰:多贤可云?对曰:鱼鳖之不食咡者,不出其渊;树木之胜霜雪者,不听于天;士能自治者,不从圣人,岂云哉?夷吾之闻之也,不欲,强能不服,智而不牧。若旬虚期于月,津若出于一,明然,则可以虚矣。故阨其道而薄其所予,则士云矣。不择人而予之,谓之好人;不择人而取之,谓之好利。审此两者,以为处行,则云矣。

不方之政,不可以为国;曲静之言,不可以为道。节时于政,与时往矣。不动以为道,齐以为行,避世之道,不可以进取。

阳者进谋,几者应感,再杀则齐,然后运可请也。对曰:夫运谋者,天地之虚满也,合离也,春秋冬夏之胜也,然有知强弱之所尤,然后应诸侯取交,故知安危国之所存。以时事天,以天事神,以神事鬼,故国无罪而君寿,而民不杀智运谋而杂橐刃焉。

其满为感,其虚为亡,满虚之合,有时而为实,时而为动。地阳时贷,其冬厚则夏热,其阳厚则阴寒。是故王者谨于日至,故知虚满之所在,以为政令。已杀生,其合而未散,可以决事。将合,可以禺其随行以为兵,分其多少以为曲政。

请问形有时而变乎?对曰:阴阳之分定,则甘苦之草生也。从其宜,则酸咸和焉,而形色定焉,以为声乐。夫阴阳进退,满虚亡时,其散合可以视岁。唯圣人不为岁,能知满虚,夺余满,补不足,以通政事,以赡民常。地之变气,应其所出;水之变气,应之以精,受之以豫;天之变气,应之以正。且夫天地精气有五,不必为沮,其亟而反,其重陔动毁之进退,即此数之难得者也,此形之时变也。

沮平气之阳,若如辞静。余气之潜然而动,爱气之潜然而哀,胡得而治动?对曰:得之衰时,位而观之,佁美然后有辉。修之心,其杀以相待,故有满虚哀乐之气也。故书之帝八,神农不与存,为其无位,不能相用。

问:运之合满安臧?二十岁而可广,十二岁而聂广,百岁伤神。周郑之礼移矣,则周律之废矣,则中国之草木有移于不通之野者。然则人君声服变矣,则臣有依驷之禄,妇人为政,铁之重反旅金。而声好下曲,食好咸苦,则人君日退。亟则溪陵山谷之神之祭更,应国之称号亦更矣。

视之示变,观之风气。古之祭,有时而星,有时而星熺,有时而熰,有时而朐。鼠应广之实,阴阳之数也。华若落之名,祭之号也。是故天子之为国,图具其树物也。

\hypertarget{header-n503}{%
\subsection{心术上}\label{header-n503}}

心之在体,君之位也;九窍之有职,官之分也。心处其道。九窍循理;嗜欲充益,目不见色,耳不闻声。故曰上离其道,下失其事。毋代马走,使尽其力;毋代鸟飞,使弊其羽翼;毋先物动,以观其则。动则失位,静乃自得。

道,不远而难极也,与人并处而难得也。虚其欲,神将入舍;扫除不洁,神乃留处。人皆欲智而莫索其所以智乎。智乎,智乎,投之海外无自夺,求之者不得处之者。夫正人无求之也,故能虚无。

虚无无形谓之道,化育万物谓之德,君臣父子人间之事谓之义,登降揖让、贵贱有等、亲疏之体谓之礼,简物、小未一道。杀僇禁诛谓之法。

大道可安而不可说。直人之言不义不颇,不出于口,不见于色,四海之人,又孰知其则?

天曰虚,地曰静,乃不伐。洁其宫,开其门,去私毋言,神明若存。纷乎其若乱,静之而自治。强不能遍立,智不能尽谋。物固有形,形固有名,名当,谓之圣人。故必知不言,无为之事,然后知道之纪。殊形异埶,不与万物异理,故可以为天下始。

人之可杀,以其恶死也;其可不利,以其好利也。是以君子不休乎好,不迫乎恶,恬愉无为,去智与故。其应也,非所设也;其动也,非所取也。过在自用,罪在变化。是故有道之君,其处也若无知,其应物也若偶之。静因之道也。

``心之在体,君之位也;九窍之有职,官之分也。''耳目者。视听之官也,心而无与于视听之事,则官得守其分矣。夫心有欲者,物过而目不见,声至而耳不闻也。故曰:``上离其道,下失其事。''故曰:心术者,无为而制窍者也。故曰``君''。``毋代马走'',``毋代鸟飞'',此言不夺能能,不与下诚也。``毋先物动''者,摇者不走,趮者不静,言动之不可以观也。``位''者'',谓其所立也。人主者立于阴,阴者静,故曰``动则失位''。阴则能制阳矣,静则能制动矣,攸曰,`静乃自得。''

道在天地之间也,其大无外,其小无内,故曰``不远而难极也''。虚之与人也无间,唯圣人得虚道,故曰``并处而难得''。世人之所职者精也。去欲则宣,宣则静矣,静则精。精则独立矣,独则明,明则神矣。神者至贵也,故馆不辟除,则贵人不舍焉。故曰``不洁则神不处''。``人皆欲知而莫索之'',其所(以)知,彼也;其所以知,此也。不修之此,焉能知彼?修之此,莫能虚矣。虚者,无藏也。故曰去知则奚率求矣,无藏则奚设矣。无求无设则无虑,无虑则反复虚矣。

天之道,虚其无形。虚则不屈,无形则无所位迕,无所位迕,故遍流万物而不变,德者,道之舍,物得以生生,知得以职道之精。故德者得也。得也者,其谓所得以然也。以无为之谓道,舍之之谓德。故道之与德无间,故言之者不别也。间之理者,谓其所以舍也。义者,谓各处其宜也。礼者,因人之情,缘义之理,而为之节文者也,故礼者谓有理也。理也者,明分以谕义之意也。故礼出乎义,义出乎理,理因乎宜者也。法者所以同出,不得不然者也,故杀僇禁诛以一之也。故事督乎法,法出乎权,权出于道。

道也者、动不见其形,施不见其德,万物皆以得,然莫知其极。故曰``可以安而不可说''也。莫人,言至也。不宜,言应也。应也者,非吾所设,故能无宜也。不顾,言因也。因也者,非吾所顾,故无顾也。``不出于口,不见于色'',言无形也;``四海之人,孰知其则'',言深囿也。

天之道虚,地之道静。虚则不屈,静则不变,不变则无过,故曰``不伐''。``洁其宫,阙其门'':宫者,谓心也。心也者,智之舍也,故曰``宫''。洁之者,去好过也。门者,谓耳目也。耳目者,所以闻见也。``物固有形,形固有各'',此言不得过实、实不得延名。姑形以形,以形务名,督言正名,故曰``圣人''。``不言之言'',应也。应也者,以其为之人者也。执其名,务其应,所以成,之应之道也。``无为之道,因也。因也者,无益无损也。以其形因为之名,此因之术也。名者,圣人之所以纪万物也。人者立于强,务于善,未于能,动于故者也。圣人无之,无之则与物异矣。异则虚,虚者万物之始也,故曰``可以为天下始''。

人迫于恶,则失其所好;怵于好,则忘其所恶。非道也。故曰:``不怵乎好,不迫乎恶。''恶不失其理,欲不过其情,故曰:``君子''。``恬愉无为,去智与故'',言虚素也。``其应非所设也,其动非所取也'',此言因也。因也者,舍己而以物为法者也。感而后应,非所设也;缘理而动,非所取也,``过在自用,罪在变化'':自用则不虚,不虚则仵于物矣;变化则为生,为生则乱矣。故道贵因。因者,因其能者,言所用也。``君子之处也若无知'',言至虚也;``其应物也若偶之'',言时适也、若影之象形,响之应声也。故物至则应,过则舍矣。舍矣者,言复所于虚也。

\hypertarget{header-n518}{%
\subsection{心术下}\label{header-n518}}

形不正者,德不来;中不精者,心不冶。正形饰德,万物毕得,翼然自来,神莫知其极,昭知天下,通于四极。是故曰:无以物乱官,毋以官乱心,此之谓内德。是故意气定,然后反正。气者身之充也,行者正之义也。充不美则心不得,行不正则民不服。是故圣人若天然,无私覆也;若地然,无私载也。私者,乱天下者也。

凡物载名而来,圣人因而财之,而天下治。实不伤,不乱于天下,而天下治。专于意,一于心,耳目端,知远之证。能专乎?能一乎?能毋卜筮而知凶吉乎?能止乎?能已乎?能毋问于人而自得之于己乎?故曰,思之。思之不得,鬼神教之。非鬼神之力也。其精气之极也。

一气能变曰精、一事能变曰智。慕选者,所以等事也;极变者,所以应物也。慕选而不乱,极变而不烦,执一之君子执一而不失,能君万物,日月之与同光,天地之与同理。

圣人裁物,不为物使。心安,是国安也;心治,是国治也。治也者心也,安也者心也。治心在于中,治言出于口,治事加于民,故功作而民从,则百姓治矣。所以操者非刑也,所以危者非怒也。民人操,百姓治,道其本至也,至不至无,非所人而乱。

凡在有司执制者之利,非道也。圣人之道,若存若亡,援而用之,殁世不亡。与时变而不化,应物而不移,日用之而不化。

人能正静者,筋肕而骨强;能戴大圆者,体乎大方;镜大清者,视乎大明。正静不失,日新其德,昭知天下,通于四极。金心在中不可匿,外见于形容,可知于颜色。善气迎人,亲如弟兄;恶气迎人,害于戈兵。不言之言,闻于雷鼓。全心之形,明于日月,察于父母。昔者明王之爱天下,故天下可附;暴王之恶天下,故天下可离。故货之不足以为爱,刑之不足以为恶。货者爱之末也,刑者恶之末也。

凡民之生也,必以正平;所以失之者,必以喜乐哀怒,节怒莫若乐,节乐莫若礼,守礼莫若敬。外敬而内静者,必反其性。

岂无利事哉?我无利心。岂无安处哉?我无安心。心之中又有心。意以先言,意然后形,形然后思,思然后知。凡心之形,过知失生。

是故内聚以为原。泉之不竭,表里遂通;泉之不涸,四支坚固。能令用之,被及四固。

是故圣人一言解之,上察于天,下察于地。

\hypertarget{header-n531}{%
\subsection{白心}\label{header-n531}}

建当立有,以靖为宗,以时为宝,以政为仪,和则能久。非吾仪虽利不为,非吾当虽利不行,非吾道虽利不取。上之随天,其次随人。人不倡不和,天不始不随。故其言也不废,其事也不随。

原始计实,本其所生。知其象则索其形,缘其理则知其情,索其端则知其名。故苞物众者,莫大于天地;化物多者,莫多于日月;民之所急,莫急于水火。然而,天不为一物在其时,明君圣人亦不为一人枉其法。天行其所行而万物被其利,圣人亦行其所行而百姓被其利。是故万物均、既夸众百姓平矣。是以圣人之治也,静身以待之,物至而名自治之。正名自治之,奇身名废。名正法备,则圣人无事。不可常居也,不可废舍也。随变断事也,知时以为度。大者宽,小者局,物有所余有所不足。

兵之出,出于人;其人入,入于身。兵之胜,从于适;德之来,从于身。故曰:祥于鬼者义于人,兵不义不可,强而骄者损其强,弱而骄者亟死亡;强而卑义信其强,弱而卑义免于罪。是故骄之余卑,卑之余骄。

道者,一人用之,不闻有余;天下行之,不闻不足。此谓道矣。小取焉则小得福,大取焉则大得福,尽行之而天下服,殊无取焉则民反,其身不免于贼。左者,出者也;右者,人者也。出者而不伤人,入者自伤也。不日不月,而事以从;不卜不筮,而谨知吉凶。是谓宽乎形,徒居而致名。去善之言,为善之事,事成而顾反无名。能者无名,从事无事。审量出入,而观物所载。

孰能法无法乎?始无始乎?终无终乎?弱无弱乎?故曰:美哉岪岪。故曰有中有中,孰能得夫中之衷乎?故曰功成者隳,名成者亏。故曰,孰能弃名与功而还与众人同?孰能弃功与名而还反无成?无成有贵其成也,有成贵其无成也。日极则仄,月满则亏。极之徒仄,满之徒亏,巨之徒灭。孰能己无乎?效夫天地之纪。

人言善亦勿听,人言恶亦勿听,持而待之,空然勿两之,淑然自清。无以旁言为事成,察而征之,无听辩,万物归之,美恶乃自见。

天或维之,地或载之。天莫之维,则天以坠矣;地莫之载,则地以沉矣。夫天不坠,地不沉,夫或维而载之也夫!又况于人?人有治之,辟之若夫雷鼓之动也。夫不能自摇者,夫或摇之。夫或者何?若然者也。视则不见,听则不闻,洒乎天下满,不见其塞。集于颜色,知于肌肤,责其往来,莫知其时。薄乎其方也,韕乎其圜也,韕韕乎莫得其门。故口为声也,耳为听也,目有视也,手有指也,足有履也,事物有所比也。

``当生者生,当死者死'',言有西有东,各死其乡。置常立仪,能守贞乎?常事通道,能官人乎?故书其恶者,言其薄者。上圣之人,口无虚习也,手无虚指也,物至而命之耳。发于名声,凝于体色,此其可谕者也。不发于名声,不凝于体色,此其不可谕者也。及至于至者,教存可也,教亡可也。故曰:济于舟者和于水矣,义于人者祥其神矣。

事有适,而无适,若有适;觿解,不可解而后解。故善举事者,国人莫知其解。为善乎,毋提提;为不善乎,将陷于刑。善不善,取信而止矣。若左若右,正中而已矣。县乎日月无已也。愕愕者不以天下为忧,剌剌者不以万物为策,孰能弃剌剌而为愕愕乎?

难言宪术,须同而出。无益言,无损言,近可以免。故曰:知何知乎?谋何谋乎?审而出者彼自来。自知曰稽,知人曰济。知苟适,可为天下周。内固之,一可为长久。论而用之,可以为天下王。

天之视而精,四璧而知请,壤土而与生。能若夫风与波乎?唯其所欲适。故子而代其父,曰义也,臣而代其君,曰篡也。篡何能歌?武王是也。故曰:孰能去辩与巧,而还与众人同道?故曰:思索精者明益衰,德行修者王道狭,卧名利者写生危,知周于六合之内者,吾知生之有为阻也。持而满之,乃其殆也。名满于天下,不若其已也。名进而身退,天之道也。满盛之国,不可以仕任;满盛之家,不可以嫁子;骄倨傲暴之人,不可与交。

道之大如天,其广如地,其重如石,其轻如羽。民之所以,知者寡。故曰:何道之近而莫之与能服也,弃近而就远何以费力也。故曰:欲爱吾身,先知吾情,君亲六合,以考内身。以此知象,乃知行情。既知行情,乃知养生。左右前后,周而复所。执仪服象,敬迎来者。今夫来者,必道其道,无迁无衍,命乃长久。和以反中,形性相葆。一以无贰,是谓知道。将欲服之,必一其端,而固其所守。责其往来,莫知其时,索之于天,与之为期,不失其期,乃能得之。故曰:吾语若大明之极,大明之明非爱人不予也。同则相从,反则相距也。吾察反相距,吾以故知古从之同也。

\hypertarget{header-n546}{%
\subsection{水地}\label{header-n546}}

地者,万物之本原,诸生之根菀也,美恶、贤不官、愚俊之所生也。水者,地之血气,如筋脉之通流者也。故曰:水,具材也。

何以知其然也?曰:夫水淖弱以清,而好洒人之恶,仁也;视之黑而白,精也;量之不可使概,至满而止,正也;唯无不流,至平而止,义也;人皆赴高,己独赴下,卑也。卑也者,道之室,王者之器也,而水以为都居。

准也者,五量之宗也;素也者,五色之质也;淡也者,五味之中也。是以水者,万物之准也,诸生之淡也,违非得失之质也,是以无不满,无不居也,集于天地而藏于万物,产于金石,集于诸生,故曰水神。集于草木,根得其度,华得其数,实得其量,乌鲁得之,形体肥大,羽毛丰茂,文理明著。万物莫不尽其几、反其常者,水之内度适也。

夫玉之所贵者,九德出焉。夫玉温润以泽,仁也;邻以理者,知也;坚而不蹙,义也;廉而不刿,行也;鲜而不垢,洁也;折而不挠,勇也;瑕适皆见,精也;茂华光泽,并通而不相陵,容也;叩之,其音清搏彻远,纯而不杀,辞也;是以人主贵之,藏以为室,剖以为符瑞,九德出焉。

人,水也。男女精气合,而水流形。三月如咀。咀者何?曰五味。五昧者何?曰五藏。酸主脾,咸主肺,辛主肾,苦主肝,甘主心。五藏已具,而后生肉。脾生隔,肺生骨,肾生脑,肝生革,心生肉。五内已具,而后发为九窍。脾发为鼻,肝发为目,肾发为耳,肺发为窍。五月而成,十月而生。生而目视,耳听,心虑。目之所以视,非特山陵之见也,察于荒忽。耳之所听,非特雷鼓之闻也,察干淑湫。心之所虑,非特知于粗粗也,察于微眇,故修要之精。

是以水集于玉而九德出焉。凝蹇而为人,而九窍五虑出焉。此乃其精也精粗浊蹇能存而不能亡者也。

伏暗能存而能亡者,蓍龟与龙是也。龟生于水,发之于火,于是为万物先,为祸福正。龙生干水,被五色而游,故神。欲小则化如蚕蠋,欲大则藏于天下,欲尚则凌于云气,欲下则入千深泉;变化无日,上下无时,谓之神。龟与龙,伏暗能存而能亡者也。

或世见,或世不见者,牛蟡与庆忌。故涸泽数百岁,谷之不徙,水之不绝者,生庆忌。庆忌者,其状若人,其长四寸,衣黄衣,冠黄冠,戴黄盖,乘小马,好疾驰,以其名呼之,可使千里外一日反报,此涸泽之精也。涸川之精者,生于蟡。蟡者,一头而两身,其形若蛇,其长八尺,以其名呼之,可以取鱼鳖。此涸川水之精也。

是以水之精粗浊蹇,能存而不能亡者,生人与玉。伏暗能存而亡者,蓍龟与龙。或世见或不见者、蟡与庆忌。故人皆服之,而管子则之。人皆有之,而管子以之。

是故具者何也?水是也。万物莫不以生,唯知其托者能为之正。具者,水是也,故曰:水者何也?万物之本原也,诸生之宗室也,美恶、贤不肖、愚俊之所产也。何以知其然也?夫齐之水道躁而复,故其民贪粗而好勇;楚之水淖弱而清,故其民轻果而贼;越之水浊重而洎,故其民愚疾而垢;秦之水泔冣而稽,淤滞而杂,故其民贪戾罔而好事;齐晋之水枯旱而运,淤滞而杂,故其民谄谀葆诈,巧佞而好利;燕之水萃下而弱,沈滞而杂,故其民愚戆而好贞,轻疾而易死;宋之水轻劲而清,故其民闲易而好正。是以圣人之化世也,其解在水。故水一则人心正,水清则民心易。一则欲不污,民心易则行无邪。是以圣人之治于世也,不人告也,不户说也,其枢在水。

\hypertarget{header-n559}{%
\subsection{五行}\label{header-n559}}

一者本也,二者器也,三者充也,治者四也,教者五也,守者六也,立者七也,前者八也,终者九也,十者然后具五官于六府也,五声于六律也。

六月日至,是故人有六多,六多所以街天地也。天道以九制,地理以八制,人道以六制。以天为父,以地为母,以开乎万物,以总一统。通乎九制、六府、三充,而为明天子。修槩水上,以待乎天堇;反五藏,以视不亲;治祀之下,以观地位;货曋神庐,合于精气。已合而有常,有常而有经。审合其声,修十二钟,以律人情。人情已得,万物有极,然后有德。

故通乎阳气,所以事天也,经纬日月,用之于民。通乎阴气,所以事地也,经纬星历,以视其离。通若道然后有行,然则神筮不灵,神龟不卜,黄帝泽参,治之至也。昔者黄帝得蚩尤而明于天道,得大常而察于地利,得奢龙而辩于东方,得祝融而辩于南方,得大封而辩于西方,得后土而辩于北方。黄帝得六相而天地治,神明至。蚩尤明乎天道,故使为当时;大常察乎地利,故使为廪者:奢龙辩乎东方,故使为土师,祝融辩乎南方,故使为司徒;大封辩于西方,故使为司马;后土辩乎北方,故使为李。是故春者土师也,夏者司徒也,秋者司马也,冬者李也。

昔黄帝以其缓急作五声,以政五钟。令其五钟,一曰青钟大音,二曰赤钟重心,三曰黄钟洒光,四曰景钟昧其明,五曰黑钟隐其常。五声既调,然后作立五行以正天时,五官以正人位。人与天调,然后天地之美生。

日至,睹甲子木行御。天子出令,命左右士师内御。总别列爵,论贤不肖士吏。赋秘,赐赏于四境之内,发故粟以田数。出国,衡顺山林,禁民斩木,所以爱草木也。然则冰解而冻释,草木区萌,赎蛰虫卵菱。春辟勿时,苗足本。不疠雏鷇,不夭麑□,毋傅速。亡伤襁褓。时则不调。七十二日而毕。

睹丙子火行御。天子出令,命行人内御。令掘沟浍,津旧涂。发藏,任君赐赏。君子修游驰,以发地气。出皮币,命行人修春秋之礼于天下诸侯,通天下遇者兼和。然则天无疾风,草木发奋,郁气息,民不疾而荣华蕃。七十二日而毕。

睹戊子土行御。天子出令,命左右司徒内御。不诛不贞,农事为敬。大扬惠言,宽刑死,缓罪人。出国,司徒令,命顺民之功力,以养五谷。君子之静居,而农夫修其功力极。然则天为粤宛,草木养长,五谷蕃实秀大,六畜牺牲具,民足财,国富,上下亲,诸侯和。七十二日而毕。

睹庚子金行御。天子出令,命祝宗选禽兽之禁、五谷之先熟者,而荐之祖庙与五祀,鬼神享其气焉,君子食其味焉。然则凉风至,白露下,天子出令,命左右司马(衍)组甲厉兵,合什为伍,以修于四境之内,谀然告民有事,所以待天地之杀敛也。然则昼炙阳,夕下露,地竞环,五谷邻熟,草木茂实,岁农丰年大茂。七十二日而毕。

睹王子水行御。天子出令,命左右使人内御。御其气足,则发而止;其气不足,则发撊渎盗贼。数劋竹箭,伐檀柘,令民出猎,禽兽不释巨少而杀之,所以贵天地之所闭藏也。然则羽卵者不段,毛胎者不贕,孕妇不销弃,草木根本美。七十二日而毕。

睹甲子木行御。天子不赋不赐赏,而大斩伐伤,君危,不杀太子危;家人夫人死,不然则长子死。七十二日而毕。睹丙子火行御。天子敬行急政,旱札,苗死,民厉。七十二日而毕。睹戊子土行御。天子修宫室,筑台榭,君危;外筑城郭,臣死。七十二日而毕。睹庚子金行御。天子攻山击石,有兵作战而败,士死,丧执政。七十二日而毕。睹壬子水行御。天子决塞,动大水,王后夫人薨,不然则羽卵者段,毛胎者贕,孕妇销弃,草木根本不美。七十二日而毕。

\hypertarget{header-n572}{%
\subsection{势}\label{header-n572}}

战而惧水,此谓澹灭。小事不从,大事不吉。战而惧险,此谓迷中。分其师众,人既迷芒,必其将亡之道。

动静者比于死,动作者比于丑,动信者比于距,动诎者比于避。夫静与作,时以为主人,时以为客,贵得度。知静之修,居而自利;知作之从,每动有功。故曰,无为者帝,其此之谓矣。

逆节萌生,天地未形,先为之政,其事乃不成,缪受其刑。天因人,圣人因天。天时不作勿为客,人事不起勿为始。慕和其众,以修天地之从。人先生之,天地刑之,圣人成之,则与天同极。正静不争,动作不贰,素质不留,与地同极。未得天极,则隐于德;已得天极,则致其力。既成其功,顺守其从,人不能代。

成功之道,嬴缩为宝。毋亡天极,究数而止。事若未成,毋改其形,毋失其始,静民观时,待令而起。故曰,修阴阳之从,而道天地之常。嬴嬴缩缩,因而为当;死死生生,因天地之形。天地之形,圣人成之。小取者小利,大取者大利,尽行之者有天下。

故贤者诚信以仁之,慈惠以爱之,端政象不敢以先人,中静不留,裕德无求,形于女色。其所处者,柔安静乐,行德而不争,以待天下之濆作也。故贤者安徐正静,柔节先定,行于不敢,而立于不能,守弱节而坚处之。故不犯天时,不乱民功,秉时养人,先德后刑,顺于天,微度人。

善周者,明不能见也;善明者,周不能蔽也。大明胜大周,则民无大周也;大周胜大明,则民无大明也。大周之先,可以奋信;大明之祖,可以代天。下索而不得,求之招摇之下。

兽厌走,而有伏网罟。一偃一侧,不然不得。大文三曾,而贵义与德;大武三曾,而偃武与力。

\hypertarget{header-n582}{%
\subsection{正 }\label{header-n582}}

制断五刑,各当其名,罪人不怨,善人不惊,曰刑。正之、服之、胜之、饰之,必严其令,而民则之,曰政。如四时之不貣,如垦辰之不变,如宵如昼,如阴如阳,如日月之明,曰法。爱之、生之、养之、成之,利民不德,天下亲之,曰德。无德无怨,无好无恶,万物崇一,阴阳同度,曰道,刑以弊之,政以命之,法以遏之,德以养之,道以明之。刑以弊之,毋失民命;令之以终其欲,明之毋径;遏之以绝其志意,毋使民幸;养之以化其恶,必自身始;明之以察其生,必修其理。致刑,其民庸心以蔽;致政,其民服信以听;致德,其民和平以静;致道,其民付而不争,罪人当名曰刑,出令时当曰政,当故不改曰法,爱民无私曰德,会民所聚曰道。

立常行政,能服信乎?中和慎敬,能日新乎?正衡一静,能守慎乎?废私立公,能举人乎?临政官民,能后其身乎?能服信政,此谓正纪。能服日新,此谓行理。守慎正名,伪诈自止。举人无私,臣德威道。能后其身,上佐天子。

\hypertarget{header-n587}{%
\subsection{九变}\label{header-n587}}

凡民之所以守战至死而不德其上者,有数以至焉。曰:大者亲戚坟墓之所在也,田宅富厚足居也。不然,则州县乡党与宗族足怀乐也。不然,则上之教训、习俗,慈爱之于民也厚,无所往而得之。不然,则山林泽谷之利足生也。不然,则地形险阻,易守而难攻也。不然,则罚严而可畏也。不然,则赏明而足劝也。不然,则有深怨于敌人也。不然,则有厚功干上也。此民之所以守战至死而不德其上者也。

今恃不信之人,而求以智;用不守之民,而欲以固;将不战之卒,而幸以胜,此兵之三暗也。

\hypertarget{header-n592}{%
\subsection{任法}\label{header-n592}}

圣君任法而不任智,任数而不任说,任公而不任私,任大道而不任小物,然后身佚而天下治。失君则不然,合法而任智,故民舍事而好誉;舍数而任说,故民舍实而好言;舍公而好私,故民离法而妄行;舍大道而任小物,故上劳烦,百姓迷惑,而国家不治。圣君则不然,守道要,处佚乐,驰骋弋猎,钟鼓竽瑟,宫中之乐,无禁圉也。不思不虑,不忧不图,利身体,便形躯,养寿命,垂拱而天下治。是敌人主有能用其道者,不事心,不劳意,不动力,而土地自辟,囷仓自实,蓄积自多,甲兵自强,群臣无诈伪,百官无奸邪,奇术技艺之人莫敢高言孟行以过其情、以遇其主矣。

昔者尧之治天下也,犹埴之在埏也,唯陶之所以为;犹金之在垆;恣冶之所以铸。其民引之而来,推之而往,使之而成,禁之而止。故尧之治也,善明法禁之令而已矣。黄帝之治天下也,其民不引而来,不推而往,不使而成,不禁而止。故黄帝之治也,置法而不变,使民安其法者也。

所谓仁义礼乐者,皆出于法。此先圣之所以一民者也。《周书》曰:``国法,法不一,则有国者不祥;民不道法,则不祥;国更立法以典民,则不祥;群臣不用礼义教训,则不祥;百官服事者离法而治,则不祥。''故曰:法者不可不恒也,存亡治乱之所以出,圣君所以为天下大仪也。君臣上下贵贱皆发焉,故曰``法''。

古之法也,世无请谒任举之人,无间识博学辩说之士,无伟服,无奇行,皆囊于法以事其主。故明王之所恒者二:一曰明法而固守之,二曰禁民私而收使之,此二者主之所恒也,夫法者,上之所以一民使下也;私者,下之所以侵法乱主也。故圣君置仪设法而固守之,然故谌杵习士闻识博学之人不可乱也,众强富贵私勇者不能侵也,信近亲爱者不能离也,珍怪奇物不能惑也,万物百事非在法之中者不能动也。故法者,天下之至道也,圣君之实用也。

今天下则不然,皆有善法而不能守也。然故谌杵习士闻识博学之士能以其智乱法惑上,众强富贵私勇者能以其威犯法侵陵,邻国诸侯能以其权置子立相,大臣能以其私附百姓,剪公财以禄私士。凡如是而求法之行,国之治,不可得也。

圣君则不然,卿相不得剪其私,群臣不得辟其所亲爱,圣君亦明其法而固守之,群臣修通辐凑以事其主,百姓辑睦听令道法以从其事。故曰:有生法,有守法,有法于法。夫生法者,君也;守法者,臣也;法于法者,民也。君臣上下贵贱皆从法,此谓为大治。

故主有三术:夫爱人不私赏也,恶人不私罚也,置仪设法以度量断者,上主也。爱人而私赏之,恶人而私罚之,倍大臣,离左右,专以其心断者,中主也。臣有所爱而为私赏之,有所恶而为私罚之,倍其公法,损其正心,专听其大臣者,危主也。故为人主者,不重爱人,不重恶人;重爱曰失德,重恶曰失威。威德皆失,则主危也。

故明王之所操者六:生之、杀之、富之、贫之、贵之、贱之。此六柄者,主之所操也。主之所处者四:一曰文,二曰武,三曰威,四曰德。此四位者,主之所处也。借人以其所操,命曰夺柄;借人以其所处,命曰失位。夺柄失位,而求令之行,不可得也。法不平,令不全,是亦夺柄失位之道也。故有为枉法,有为毁令,此圣君之所以自禁也。故贵不能威,富不能禄,贱不能事,近不能亲,美不能淫也。植固而不动,奇邪乃恐,奇革而邪化,令往而民移。故圣君失度量,置仪法,如天地之坚,如列星之固,如日月之明,如四时之信,然故令往而民从之。而失君则不然,法立而还废之,令出而后反之,枉法而从私,毁令而不全。是贵能威之,富能禄之,贱能事之,近能亲之,美能淫之也。此五者不禁于身,是以群臣百姓人挟其私而幸其主,彼幸而得之,则主日侵。彼幸而不得,则怨日产。夫日侵而产怨,此失君之所慎也。

凡为主而不得用其法,不适其意,顾臣而行,离法而听贵臣,此所谓贵而威之也。富人用金玉事主而来焉,主离法而听之,此所谓富而禄之也。贱人以服约卑敬悲色告诉其主,主因离法而听之,所谓贱而事之也。近者以逼近亲爱有求其主,主因离法而听之,此谓近而亲之也。美者以巧言令色请其主,主因离法而听之,此所谓美而淫之也。不见也;有私听也,故有不闻也;有私虑也,故有不知也。夫私者,壅蔽失位之道也。上舍公法而听私说,故群臣百姓皆设私立方以教于国,群党比周以立其私,请谒任举以乱公法,人用其心以幸于上。上无度量以禁之,是以私说日益,而公法日损,国之不冶,从此产矣。

夫君臣者,天地之位也;民者,众物之象也。各立其所职以待君令,群臣百姓安得各用其心而立私乎?故遵主令而行之,虽有伤败,无罚;非主令而行之,虽有功利,罪死。然故下之事上也,如响之应声也;臣之事主也,如影之从形也。故上令而下应,主行而臣从,此治之道也。夫非主令而行,有功利,因赏之,是教妄举也;遵主令而行之,有伤败,而罚之,是使民虑利害而离法也。群臣百姓人虑利害,而以其私心举措,则法制毁而令不行矣。

\hypertarget{header-n605}{%
\subsection{明法}\label{header-n605}}

所谓治国者,主道明也;所谓乱国者,臣术胜也。夫尊君卑臣,非计亲也,以势胜也;百官识,非惠也,刑罚必也。故君臣共道则乱,专授则失。夫国有四亡:令求不出谓之灭,出而道留谓之拥,下情求不上通谓之塞,下情上而道止谓之侵。故夫灭、侵、塞、拥之所生,从法之不立也。是故先王之治国也,不淫意于法之外,不为惠于法之内也。动无非法者,所以禁过而外私也。威不两错,政不二门。以法治国则举错而已。是故有法度之制者,不可巧以诈伪;有权衡之称者,不可欺以轻重;有寻丈之数者,不可差以长短。今主释法以誉进能,则臣离上而下比周矣;以党举官,则民务交而不求用矣。是故官之失其治也,是主以誉为赏,以毁为罚也。然则喜赏恶罚之人,离公道而行私木矣。比周以相为慝,是忘主私佼,以进其誉。故交众者誉多,外内朋党,虽有大奸,其蔽主多矣。是以忠臣死于非罪,而邪臣起于非功。所死者非罪,所起者非功也,然则为人臣者重私而轻公矣。十至私人之门,不一至于庭;百虑其家,不一图国。属数虽众,非以尊君也;百官虽具,非以任国也;此之谓国无人。国无人者,非朝臣之衰也,家与家务于相益,不务尊君也;大臣务相贵,而不任国;小臣持禄养交,不以官为事,故官失其能。是故先王之治国也,使法择人,不自举也;使法量功,不自度也。故能匿而不可蔽,败而不可饰也;誉者不能进,而诽者不能退也。然则君臣之间明别,明别则易治也,主虽不身下为,而守法为之可也。

\hypertarget{header-n609}{%
\subsection{正世}\label{header-n609}}

古之欲正世调天下者,必先观国政,料事务,察民俗,本治乱之所生,知得失之所在,然后从事。故法可立而治可行。

夫万民不和,国家不安,失非在上,则过在下。今使人君行逆不修道,诛杀不以理,重赋敛,竭民财,急使令,罢民力,财竭则不能毋侵夺,力罢则不能毋堕倪。民已侵夺、堕倪,因以法随而诛之,则是诛罚重而乱愈起。夫民劳苦困不足,则简禁而轻罪,如此则失在上,失在上而上不变,则万民无所托其命。今人主轻刑政,宽百姓,薄赋敛,缓使令,然民淫躁行私而不从制,饰智任诈,负力而争,则是过在下。过在下,人君不廉而变,则暴人不胜,邪乱不止。暴人不胜,邪乱不止,则君人者势伤而威日衰矣。

故为人君者,莫贵于胜。所谓胜者,法立令行之谓胜。法立令行,故群臣奉法守职,百官有常。法不繁匿。万民敦悫,反本而俭力。故赏必足以使,威必足以胜,然后下从。

故古之所谓明君者,非一君也。其设赏有薄有厚,其立禁有轻有重,迹行不必同,非故相反也,皆随时而变,因俗而动。夫民躁而行僻,则赏不可以不厚,禁不可以不重。故圣人设厚赏,非侈也;立重禁,非戾也。赏薄则民不利,禁轻则邪人不畏。设人之所不利,欲以使,则民不尽力;立人之所不畏,欲以禁,则邪人不止。是故陈法出令而民不从。故赏不足劝,则士民不为用;刑罚不足畏,则暴人轻犯禁。民者,服于威杀然后从,见利然后用,被治然后正,得所安然后静者也。夫盗贼不胜,邪乱不止,强劫弱,众暴寡,此天下之所忧,万民之所患也。忧患不除,则民不安其居;民不安其居,则民望绝于上矣。

夫利莫大于治,害莫大于乱。夫五帝三王所以成功立名,显于后世者,以为天下致利除害也。事行不必同,所务一也。夫民贪行躁,而诛罚轻,罪过不发,则是长淫乱而便邪僻也,有爱人之心、而实合于伤民,此二者不可不察也。

夫盗贼不胜则良民危,法禁不立则奸邪繁。故事莫急于当务,治莫贵于得齐。制民急则民迫,民迫则窘,窘则民失其所葆;缓则纵,纵则淫,淫则行私,行私则离公,离公则难用。故治之所以不立者,齐不得也。齐不得则治难行。故治民之齐,不可不察也。圣人者,明于治乱之道,习于人事之终始者也。其治人民也,期于利民而止。故其位齐也,不慕古,不留今,与时变,与俗化。

夫君人之道,莫贵于胜。胜,故君道立;君道立,然后下从;下从,故教可立而化可成也。夫民不心服体从,则不可以礼义之文教也,君人者不可以不察也。

\hypertarget{header-n619}{%
\subsection{治国}\label{header-n619}}

凡治国之道,必先富民。民富则易治也,民贫则难治也。奚以知其然也?民富则安乡重家,安乡重家则敬上畏罪,敬上畏罪则易治也。民贫则危乡轻家,危乡轻家则敢凌上犯禁,凌上犯禁则难治也。故治国常富,而乱国常贫。是以善为国者,必先富民,然后治之。

昔者,七十九代之君,法制不一,号令不同,然俱王天下者,何也?必国富而粟多也。夫富国多粟生于农,故先王贵之。凡为国之急者,必先禁末作文巧,末作文巧禁则民无所游食,民无所游食则必农。民事农则田垦,田垦则粟多,粟多则国富。国富者兵强,兵强者战胜,战胜者地广。是以先王知众民、强兵、广地、富国之必生于粟也,故禁末作,止奇巧,而利农事。今为末作奇巧者,一日作而五日食。农夫终岁之作,不足以自食也。然则民舍本事而事末作。舍本事而事末作,则田荒而国贫矣。

凡农者月不足而岁有余者也,而上征暴急无时,则民倍贷以给上之征矣。耕耨者有时,而泽不必足,则民倍贷以取庸矣。秋籴以五,春粜以束,是又倍贷也。故以上之证而倍取于民者四,关市之租,府库之征粟十一,厮舆之事,此四时亦当一倍贷矣。夫以一民养四主,故逃徙者刑而上不能止者,粟少而民无积也。

嵩山之东,河汝之间,蚤生而晚杀,五谷之所蕃孰也,四种而五获。中年亩二石,一夫为粟二百石。今也仓廪虚而民无积,农夫以粥子者,上无术以均之也。故先王使农、士、商、工四民交能易作,终岁之利无道相过也。是以民作一而得均。民作一则田垦,奸巧不生。田垦则粟多,粟多则国富。奸巧不生则民治。富而治,此王之道也。

不生粟之国亡,粟生而死者霸,粟生而不死者王。粟也者,民之所归也;粟也者,财之所归也;粟也者,地之所归也。粟多则天下之物尽至矣。故舜一徙成邑,二徙成都,参徙成国。舜非严刑罚重禁令,而民归之矣,去者必害,从者必利也。先王者善为民除害兴利,故天下之民归之。所谓兴利者,利农事也;所谓除害者,禁害农事也。农事胜则入粟多,入粟多则国富,国富则安乡重家,安乡重家则虽变俗易习、驱众移民,至于杀之,而民不恶也。此务粟之功也。上不利农则粟少,粟少则人贫,人贫则轻家,轻家则易去、易去则上令不能必行,上令不能必行则禁不能必止,禁不能必止则战不必胜、守不必固矣。夫令不必行,禁不必止,战不必胜,守不必固,命之曰寄生之君。此由不利农少粟之害也。粟者,王之本事也,人主之大务,有人之涂,治国之道也。

\hypertarget{header-n627}{%
\subsection{内业}\label{header-n627}}

凡物之精,此则为生。下生五谷,上为列星。流于天地之间,谓之鬼神;藏于胸中,谓之圣人。是故民气,杲乎如登于天,杳乎如入于渊,淖乎如在于海,卒乎如在于己。是故此气也,不可止以力,而可安以德;不可呼以声,而可迎以音。敬守勿失,是谓成德,德成而智出,万物果得。

凡心之刑,自充自盈,自生自成。其所以失之,必以忧乐喜怒欲利。能去忧乐喜怒欲利,心乃反济。彼心之情,利安以宁,勿烦勿乱,和乃自成。折折乎如在于侧,忽忽乎如将不得,渺渺乎如穷无极。此稽不远,日用其德。

夫道者,所以充形也,而人不能固。其往不复,其来不舍。谋乎莫闻其音,卒乎乃在于心;冥冥乎不见其形,淫淫乎与我俱生。不见其形;不闻其声,而序其成,谓之道。凡道无所,善心安爱。心静气理,道乃可止。彼道不远,民得以产;彼道不离,民因以知。是故卒乎其如可与索,眇眇乎其如穷无所。彼道之情,恶音与声,修心静音,道乃可得。道也者,口之所不能言也,目之所不能视也,耳之所不能听也,所以修心而正形也;人之所失以死,所得以生也;事之所失以败,所得以成也。凡道无根无茎,无叶无荣。万物以生,万物以成,命之曰道。

天主正,地主平,人主安静。春秋冬夏,天之时也;山陵川谷,地之枝也;喜怒取予,人之谋也。是故圣人与时变而不化,从物而不移。能正能静,然后能定。定心在中,耳目聪明,四肢坚固,可以为精舍。精也者,气之精者也。气,道乃生,生乃思,思乃知,知乃止矣。凡心之形,过知失生。

一物能化谓之神,一事能变谓之智。化不易气,变不易智,唯执一之君子能为此乎!执一不失,能君万物。君子使物,不为物使,得一之理。治心在于中,治言出于口,治事加于人,然则天下治矣。一言得而天下服,一言定而天下听,公之谓也。

形不正,德不来;中不静,心不治。正形摄德,天仁地义,则淫然而自至神明之极,照乎知万物。中义守不忒,不以物乱官,不以官乱心,是谓中得。

有神自在身,一往一来,奠之能思。失之必乱,得之必治。敬除其舍,精将自来。精想思之,宁念治之,严容畏敬,精将至定。得之而勿舍,耳目不淫。

心无他图,正心在中,万物得度。道满天下,普在民所,民不能知也。一言之解,上察于天,下极于地,蟠满九州。何谓解之?在于心安。我心治,官乃治,我心安,官乃安。治之者心也,安之者心也。

心以藏心,心之中又有心焉。彼心之心,音以先言。音然后形,形然后言,言然后使,使然后治。不治必乱,乱乃死。

精存自生,其外安荣,内藏以为泉原,浩然和平,以为气渊。渊之不涸,四体乃固;泉之不竭,九窍遂通。乃能穷天地,破四海。中无惑意,外无邪灾,心全于中,形全于外,不逢天灾,不遇人窖,谓之圣人。

人能正静,皮肤裕宽,耳目聪明,筋信而骨强。乃能戴大圜,而履大方,鉴于大清,视干大明。敬慎无忒,日新其德,遍知天下,穷于四极。敬发其充,是谓内得。然而不反,此生之忒。

凡道,必周必密,必宽必舒,必坚必固,守善勿舍,逐淫泽薄,既知其极,反于道德。全心在中,不可蔽匿,和于形容,见于肤色。善气迎人,亲于弟兄;恶气迎人,害于戎兵。不言之声,疾于雷鼓;心气之形,明于日月,察于父母。赏不足以劝善,刑不足以惩过,气意得而天下服,心意定而天下听。

搏气如神,万物备存。能搏乎?能一乎?能无卜筮而知吉凶乎?能止乎?能已乎?能勿求诸人而得之己乎?思之,思之,又重思之。思之而不通,鬼神将通之。非鬼神之力也,精气之极也。

四体既正,血气既静,一意搏心,耳目不淫,虽远若近。思索生知,慢易生忧,暴傲生怨,忧郁生疾,疾困乃死。思之而不舍,内困外薄,不早为图,生将巽舍。食莫若无饱,思莫若勿致,节适之齐,彼将自至。

凡人之生也,天出其精,地出其形,合此以为人。和乃生,不和不生。察和之道,其精不见,其征不丑。平正擅匈,论治在心。此以长寿。忿怒之失度,乃为之图。节其五欲,去其二凶,不喜不怒,平正擅匈。

凡人之生也,必以平正。所以失之,必以喜怒忧患。是故止怒莫若诗,去忧莫若乐,节乐莫若礼,守礼莫若敬,守敬莫若静。内静外敬,能反其性,性将大定。

凡食之道:大充,伤而形不臧;大摄,骨枯而血沍。充摄之间,此谓和成,精之所舍,而知之所生,饥饱之失度,乃为之图。饱则疾动,饥则广思,老则长虑。饱不疾动,气不通于四末;饥不广思,饱而不废;老不长虑,困乃速竭。大心而敢,宽气而广,其形安而不移,能守一而弃万苛,见利不诱,见害不俱,宽舒而仁,独乐其身,是谓云气,意行似天。

凡人之生也,必以其欢。忧则失纪,怒则失端。忧悲喜怒,道乃无处。爱欲静之,遇乱正之,勿引勿推,福将自归。彼道自来,可藉与谋,静则得之,躁则失之。灵气在心,一来一逝,其细无内,其大无外。所以失之,以躁为害。心能执静,道将自定。得道之人,理丞而屯泄,匈中无败。节欲之道,万物不害。

\hypertarget{header-n648}{%
\subsection{封禅}\label{header-n648}}

桓公既霸,会诸侯于葵丘,而欲封禅。管仲曰:``古者封泰山禅梁父者七十二家,而夷吾所记者十有二焉。昔无怀氏封泰山,禅云云;伏羲封泰山,禅云云;神农封泰山,禅云云;炎帝封泰山,禅云云;黄帝封泰山,禅亭亭;颛顼封泰山,禅云云;帝喾封泰山,禅云云;尧封泰山,禅云云;舜封泰山,禅云云;禹封泰山,禅会稽;汤封泰山,禅云云;周成王封泰山,禅社首。皆受命然后得封禅。''桓公曰:``寡人北伐山戎,过孤竹;西伐大夏,涉流沙,束马悬车,上卑耳之山;南伐至召陵,登熊耳山以望江汉。兵车之会三,而乘车之会六,九合诸侯,一匡天下,诸侯莫违我。昔三代受命,亦何以异乎?''于是管仲睹桓公不可穷以辞,因设之以事,曰:``古之封禅,鄗上之黍,北里之禾,所以为盛;江淮之间,一茅三脊,所以为藉也;东海致比目之鱼,西海致比翼之鸟,然后物有不召而自至者十有五焉。今凤凰麒麟不来,嘉谷不生,而蓬蒿藜莠茂,鸱枭数至,而欲封禅,毋乃不可乎?''于是桓公乃止。

\hypertarget{header-n652}{%
\subsection{小问}\label{header-n652}}

桓公问管子曰:``治而不乱,明而不蔽,若何?''管子对曰:``明分任职,则治而不乱,明而不蔽矣。''公曰:``请问富国奈何?''管子对曰:``力地而动于时,则国必富矣。''公又问曰:``吾欲行广仁大义,以利天下,奚为而可?''管子对曰:``诛暴禁非,存亡继绝,而赦无罪,则仁广而义大矣。''公曰:``吾闻之也,夫诛暴禁非,而赦无罪者,必有战胜之器、攻取之数,而后能诛暴禁非,而赦无罪。''公曰:``请问战胜之器?''管子对曰:``选天下之豪杰,致天下之精材,来天下之良工,则有战胜之器矣。''公曰:``攻取之数何如?''管子对曰:``毁其备,散其积,夺之食,则无固城矣。''公曰:``然则取之若何?''管子对曰:``假而礼之,厚而无欺,则天下之士至矣。''公曰:``致天下之精材若何?''管子对曰:``五而六之,九而十之,不可为数。''公曰:``来工若何?''管子对曰:``三倍,不远千里。''桓公曰:``吾已知战胜之器、攻取之数矣。请问行军袭邑,举错而知先后,不失地利若何?''管子对曰:``用货,察图。''公曰:``野战必胜若何?''管子对曰:``以奇。''公曰:``吾欲遍知天下若何?''管子对曰:``小以吾不识,则天下不足识也。''公曰:``守战,远见,有患。夫民不必死,则不可与出乎守战之难;不必信,则不可恃而外知。夫恃不死之民而求以守战,恃不信之人而求以外知,此兵之三暗也。使民必死必信若何?''管子对曰:``明三本。''公曰:``何谓三本?''管子对曰:``三本者,一曰固,二曰尊,三曰质。''公曰:``何谓也?''管子对曰:``故国父母坟墓之所在,固也;田宅爵禄,尊也;妻子,质也。三者备,然后大其威,厉其意,则民必死而不我欺也。''

桓公问治民于管子。管子对曰:``凡牧民者,必知其疾,而忧之以德,勿惧以罪,勿止以力。慎此四者,足以治民也。''桓公曰:``寡人睹其善也,何为其寡也?''管仲对曰:``夫寡非有国者之患也。昔者天子中立,地方千里,四言者该焉,何为其寡也?夫牧民不知其疾则民疾,不忧以德则民多怨,惧之以罪则民多诈,止之以力则往者不反,来者鸷距。故圣王之牧民也,不在其多也。''桓公曰:``善,勿已,如是又何以行之?''管仲对曰:``质信极忠,严以有礼,慎此四者,所以行之也。''桓公曰:``请闻其说。''管仲对曰:``信也者,民信之;忠也者,民怀之;严也者,民畏之;礼也者,民美之。语曰,泽命不渝,信也;非其所欲,勿施于人,仁也;坚中外正,严也;质信以让,礼也。''桓公曰:``善哉!牧民何先?''管子对曰:``(有时先事)有时先政,有时先德(有时先恕)。飘风暴雨不为人害,涸旱不为民患,百川道,年谷熟,籴贷贱,禽兽与人聚食民食,民不疾疫。当此时也,民富且骄。牧民者厚收善岁以充仓廪,禁薮泽,(此谓)先之以事,随之以刑,敬之以礼乐以振其淫。此谓先之以政。飘风暴雨为民害,涸旱为民患,年谷不熟,岁饥,籴贷贵,民疾疫。当此时也,民贫且罢。牧民者发仓廪、山林、薮泽以共其财,后之以事,先之以恕,以振其罢。此谓先之以德。其收之也,不夺民财;其施之也,不失有德。富上而足下,此圣王之至事也。''桓公曰:``善。''

桓公问管仲曰:``寡人欲霸,以二三子之功,既得霸矣。今吾有欲王,其可乎。''管仲对曰:``公当召叔牙而问焉。''鲍叔至,公又问焉。鲍叔对曰:``公当召宾胥无而问焉。''宾胥无趋而进,公又问焉。宾胥无对曰:``古之王者,其君丰,其臣教。今君之臣丰。''公遵遁,缪然远二。三子遂徐行而进。公曰:``昔者大王贤,王季贤,文王贤,武王贤;武王伐殷克之,七年而崩,周公旦辅成王而治天下,仅能制于四海之内矣。今寡人之子不若寡人,寡人不若二三子。以此观之,则吾不王必矣。''

桓公曰:``我欲胜民,为之奈何?''管仲对曰:``此非人君之言也。胜民为易。夫胜民之为道,非天下之大道也。君欲胜民,则使有司疏狱,而谒有罪者偿,数省而严诛,若此,则民胜矣。虽然,胜民之为道,非天下之大道也。使民畏公而不见亲,祸亟及于身,虽能不久,则人待莫之弑也,危哉,君之国岌乎。''

桓公观于厩,问厩吏曰:``厩何事最难?''厩吏未对,管仲对曰:``夷吾尝为圉人矣,傅马栈最难。先傅曲木,曲木又求曲木,曲木已傅,直木无所施矣。先傅直木,直木又求直木,直木已傅,曲木亦无所施矣。''

桓公谓管仲曰:``吾欲伐大国之不服者奈何?''管仲对曰:``先爱四封之内,然后可以恶竟外之不善者;先定卿大夫之家,然后可以危邻之敌国。是故先王必有置也,然后有废也;必有利也,然后有害也。''

桓公践位,令衅社塞祷。祝凫巳疪献胙,祝曰:``除君苛疾与若之多虚而少实。''桓公不说,瞑目而视祝凫巳疪。祝凫巳疪授酒而祭之曰:``又与君之若贤。''桓公怒,将诛之,而未也。以复管仲,管仲于是知桓公之可以霸也。

桓公乘马,虎望见之而伏。桓公问管仲曰:``今者寡人乘马,虎望见寡人而不敢行,其故何也?''管仲对曰:``意者君乘駮马而洀桓,迎日而驰乎?''公曰:``然。''管仲对曰:``此駮象也。駮食虎豹,故虎疑焉。''

楚伐莒,莒君使人求救于齐。桓公将救之,管仲曰:``君勿救也。''公曰,``其故何也?''管仲对曰:``臣与其使者言,三辱其君,颜色不变。臣使官无满其礼,三强其使者,争之以死。莒君,小人也。君勿救。''桓公果不救而莒亡。

桓公放春,三月观于野,桓公曰:``何物可比于君子之德乎?''隰朋对曰:``夫粟,内甲以处,中有卷城,外有兵刃。未敢自恃,自命曰粟,此其可比于君子之德乎!''管仲曰:``苗,始其少也,眴眴乎何其孺子也!至其壮也,庄庄乎何其士也!至其成也,由由乎兹免,何其君子也!天下得之则安,不得则危,故命之曰禾。此其可比于君子之德矣。''桓公曰:``善。''

桓公北伐孤竹,未至卑耳之溪十里,闟然止,瞠然视,援弓将射,引而未敢发也,谓左右曰:``见是前人乎?''左右对曰,``不见也。''公曰:``事其不济乎?寡人大惑。今者寡人见人长尺而人物具焉:冠,右祛衣,走马前疾。事其不济乎?寡人大惑。岂有人若此者乎?''管仲对曰:``臣闻登山之神有俞儿者,长尺而人物具焉。霸王之君兴,而登山神见。且走马前疾,道也。祛衣,示前有水也。右祛衣,示从右方涉也。''至卑耳之溪,有赞水者曰:``从左方涉,其深及冠;从右方涉,其深至膝。若右涉,其大济。''桓公立拜管仲于马前曰:``仲父之圣至若此,寡人之抵罪也久矣。''管仲对曰:``夷吾闻之,圣人先知无形。今已有形,而后知之,臣非圣也,善承教也。''

桓公使管仲求宁戚,宁戚应之曰:``浩浩乎。''管仲不知,至中食而虑之。婢子曰:``公何虑?''管仲曰:``非婢子之所知也。''婢子曰:``公其毋少少,毋贱贱。昔者吴干战,未龀不得人军门。国子擿其齿,遂入,为干国多。百里徯,秦国之饭牛者也,穆公举而相之,遂霸诸侯。由是观之,贱岂可贱,少岂可少哉?''管仲曰:``然,公使我求宁戚,宁戚应我曰:`浩浩乎。'吾不识。''婢子曰:``诗有之:`浩浩者水,育育者鱼,未有室家,而安召我居?'宁子其欲室乎?''

桓公与管仲阖门而谋伐莒,未发也,而已闻于国矣。桓公怒谓管仲曰:``寡人与仲父阖门而谋伐莒,未发也,而已闻于国,其故何也?''管仲曰:``国必有圣人。''桓公曰:``然。夫日之役者,有执席食以视上者,必彼是邪?''于是乃令之复役,毋复相代。少焉,东郭邮至。桓公令傧者延而上,与之分级而上,问焉,曰:``子言伐莒者乎?''东郭邮曰:``然,臣也。''桓公曰:``寡人不言伐莒而子言伐莒,其故何也?''东郭邮对曰:``臣闻之,君子善谋,而小人善意,臣意之也。''桓公曰:``子奚以意之?''东郭邮曰:``夫欣然喜乐者,钟鼓之色也;夫渊然清静者,缞绖之色也;漻然丰满,而手足拇动者,兵甲之色也。日者,臣视二君之在台上也,口开而不阖,是言莒也;举手而指,势当莒也。且臣观小国诸侯之不服者,唯莒,于是臣故曰伐莒。''桓公曰:``善哉,以微射明,此之谓乎!子其坐,寡人与子同之。''

客或欲见齐桓公,请仕上官,授禄千钟。公以告管仲。曰:``君予之。''客闻之曰:``臣不仕矣。''公曰:``何故?''对曰:``臣闻取人以人者,其去人也,亦用人。吾不仕矣。''

\hypertarget{header-n669}{%
\subsection{七主七臣}\label{header-n669}}

或以平虚请论七主之过,得六过一是,以还自镜,以知得失。以绳七臣,得六过一是。呼呜美哉,成事疾。

申主:任势守数以为常,周听近远以续明。皆要审则法令固,赏罚必则下服度。不备待而得和,则民反素也。故主虞而安,吏肃而严,民朴而亲,官无邪吏,朝无奸臣,下无侵争,世无刑民。

惠主:丰赏厚赐以竭藏,赦奸纵过以伤法。藏竭则主权衰,法伤则奸门闿。故曰:泰则反败矣。

侵主:好恶反法以自伤,喜决难知以塞明。从狙而好小察,事无常而法令申曳。不悟,则国失势。

芒主:目伸五色,耳常五声,四邻不计,司声不听,则臣下恣行,而国权大倾。不悟,则所恶及身。

劳主:不明分职,上下相干,臣主同则。刑振以丰,丰振以刻。去之而乱,临之而殆,则后世何得?

振主:喜怒无度,严诛无赦,臣下振怒,不知所错,则人反其故。不悟,则法数日衰而国失固。

芒主:不通人情以质疑,故臣下无信。尽自治其事则事多,多则昏,昏则缓急俱植。不悟,则见所不善,余力自失而罚。

故一人之治乱在其心,一国之存亡在其主。天下得失,道一人出。主好本则民好垦草莱,主好货则人贾市,主好宫室则工匠巧,主好文采则女工靡。夫楚王好小腰而美人省食,吴王好剑而国士轻死。死与不食者,天下之所共恶也,然而为之者何也?从主之所欲也。而况愉乐音声之化乎?夫男不田,女不缁,工技力于无用,而欲土地之毛,仓库满实,不可得也。土地不毛则人不足,人不足则逆气生,逆气生则令不行。然强敌发而起,虽善者不能存。何以效其然也?曰:昔者桀纣是也。诛贤忠,近谗贼之士而贵妇人,好杀而不勇,好富而忘贫。驰猎无穷,鼓乐无厌,瑶台玉餔不足处,驰车千驷不足乘,材女乐三千人,钟石丝竹之音不绝。百姓罢乏,君子无死,卒莫有人,人有反心,遇周武王,遂为周氏之禽。此营于物而失其情者也,愉于淫乐而忘后患者也。故设用无度国家踣,举事不时,必受其灾。夫仓库非虚空也,商宦非虚坏也,法令非虚乱也,国家非虚亡也。彼时有春秋,岁有败凶,政有急缓。政有急缓故物有轻重,岁有败凶故民有义不足,时有春秋故谷有贵贱。而上不调淫,故游商得以什伯其本也。百姓之不田,贫富之不訾,皆用此作。城郭不守,兵士不用,皆道此始。夫亡国踣家者,非无壤土也,其所事者,非其功也。夫凶岁雷旱,非无雨露也,其燥湿非其时也。乱世烦政,非无法令也,其所诛赏者非其人也。暴主迷君,非无心腹也,其所取舍非其术也。故明主有六务四禁。六务者何也?一曰节用,二曰贤佐,三曰法度,四曰必诛,五曰天时,六曰地宜。四禁者何也?春无杀伐,无割大陵,倮大衍,伐大木,斩大山,行大火,诛大臣,收谷赋。夏无遏水达名川,塞大谷,动土功,射鸟兽。秋毋赦过、释罪、缓刑。冬无赋爵赏禄,伤伐五谷。故春政不禁则百长不生,夏政不禁则五谷不成,秋政不禁则奸邪不胜,冬政不禁则地气不藏。四者俱犯,则阴阳不和,风雨不时,大水漂州流邑,大风漂屋折树,火暴焚地燋草;天冬雷,地冬霆,草木夏落而秋荣;蛰虫不藏,宜死者生,宜蛰者鸣;苴多螣蟆,山多虫螟;六畜不蕃,民多夭死;国贫法乱,逆气下生。故曰:台榭相望者,亡国之庑也;驰车充国者,追寇之马也;羽剑珠饰者,斩生之斧也;文采纂组者,燔功之窑也。明王知其然,故远而不近也。能去此取彼,则人主道备矣。夫法者,所以兴功惧暴也;律者,所以定分止争也;令者,所以令人知事也。法律政令者,吏民规矩绳墨也。夫矩不正,不可以求方;绳不信,不可以求直。法令者,君臣之所共立也;权势者,人主之所独守也。故人主失守则危,臣吏失守则乱。罪决于吏则治,权断于主则威,民信其法则亲。是故明王审法慎权,下上有分。夫凡私之所起,必生于主。夫上好本则端正之士在前,上好利则毁誉之士在侧;上多喜善赏,不随其功,则士不为用;数出重法,而不克其罪,则奸不为止。明王知其然,故见必然之政,立必胜之罚。故民知所必就,而知所必去,推则往,召则来,如坠重于高,如渎水于地。故法不烦而吏不劳,民无犯禁,故有百姓无怨于上亦。

法臣:法断名决,无诽誉。故君法则主位安,臣法则货赂止而民无奸。呜呼美哉,名断言泽。

饰臣:克亲贵以为名,恬爵禄以为高。好名则无实,为高则不御。《故记》曰:``无实则无势,失辔则马焉制?''

侵臣:事小察以折法令,好佼反而行私请。故私道行则法度侵,刑法繁则奸不禁。主严诛则失民心。

乱臣:多造钟鼓、众饰妇女以惛上。故上惛则隟不计,而司声直禄。是以谄臣贵而法臣贱,此之谓微孤。

愚臣:深罪厚罚以为行,重赋敛、多兑道以为上,使身见憎而主受其谤。《故记》称之曰:``愚忠谗贼'',此之谓也。

奸臣:痛言人情以惊主,开罪党以为雠除。雠则罪不辜,罪不辜则与雠居。故善言可恶以自信,而主失亲。

乱臣:自为辞功禄,明为下请厚赏。居为非母,动为善栋。以非买名,以是伤上,而众人不知。之谓微攻。

\hypertarget{header-n688}{%
\subsection{禁藏}\label{header-n688}}

禁藏于胸胁之内,而祸避于万里之外。能以此制彼者,唯能以己知人者也。夫冬日之不滥,非爱冰也;夏日之不炀,非爱火也,为不适于身便于体也。夫明王不美宫室,非喜小也;不听钟鼓,非恶乐也,为其伤于本事,而妨于教也。故先慎于己而后彼,官亦慎内而后外,民亦务本而去末。

居民于其所乐,事之于其所利,赏之于其所善,罚之于其所恶,信之于其所余财,功之于其所无诛。于下无诛者,必诛者也;有诛者,不必诛者也。以有刑至无刑者,其法易而民全;以无刑至有刑者,其刑烦而奸多。夫先易者后难,先难而后易,万物尽然。明王知其然,故必诛而不赦,必赏而不迁者,非喜予而乐其杀也,所以为人致利除害也。于以养老长弱,完活万民,莫明焉。

夫不法法则治。法者天下之仪也,所以决疑而明是非也,百姓所县命也。故明王慎之,不为亲戚故贵易其法,吏不敢以长官威严危其命,民不以珠玉重宝犯其禁。故主上视法严于亲戚,吏之举令敬于师长,民之承教重于神宝。故法立而不用,刑设而不行也。夫施功而不钧,位虽高为用者少;赦罪而不一,德虽厚不誉者多;举事而不时,力虽尽其功不成;刑赏不当,断斩虽多,其暴不禁。夫公之所加,罪虽重下无怨气;私之所加,赏虽多士不为欢。行法不道,众民不能顺;举错不当,众民不能成;不攻不备,当今为愚人。

故圣人之制事也,能节宫室、通车舆以实藏,则国必富、位必尊;能適衣服、去玩好以奉本,而用必赡、身必安矣;能移无益之事、无补之费,通币行礼,而党必多、交必亲矣。夫众人者,多营于物,而苦其力、劳其心,故困而不赡,大者以失其国,小者以危其身。凡人之情:得所欲则乐,逢所恶则忧,此贵贱之所同有也。近之不能勿欲,远之不能勿忘,人情皆然,而好恶不同,各行所欲,而安危异焉,然后贤不肖之形见也。夫物有多寡,而情不能等;事有成败,而意不能同;行有进退,而力不能两也。故立身于中,养有节;宫室足以避燥湿,食饮足以和血气,衣服足以适寒温,礼仪足以别贵贱,游虞足以发欢欣,棺椁足以朽骨,衣衾足以朽肉,坟墓足以道记。不作无补之功,不为无益之事,故意定而不营气情。气情不营则耳目穀、衣食足;耳目穀、衣食足,则侵争不生,怨怒无有,上下相亲,兵刃不用矣。故適身行义,俭约恭敬,其唯无福,祸亦不来矣;骄傲侈泰,离度绝理,其唯无祸,福亦不至矣。是故君于上观绝理者以自恐也,下观不及者以自隐也。故曰:誉不虚出,而患不独生,福不择家,祸不索人,此之谓也。能以所闻瞻察,则事必明矣。

故凡治乱之情,皆道上始。故善者圉之以害,牵之以利。能利害者,财多而过寡矣。夫凡人之情,见利莫能勿就,见害莫能勿避。其商人通贾,倍道兼行,夜以续日,千里而不远者,利在前也。渔人之入海,海深万仞,就波逆流乘危百里,宿夜不出者,利在水也。故利之所在,虽千仞之山无所不上,深源之下,无所不入焉。故善者势利之在,而民自美安,不推而往,不引而来,不烦不扰,而民自富。如鸟之覆卵,无形无声,而唯见其成。

夫为国之本,得天之时而为经,得人之心而为纪,法令为维纲,吏为网罟,什伍以为行列,赏诛为文武。缮农具当器械,耕农当攻战,推引铫耨以当剑戟,被蓑以当铠鑐,菹笠以当盾橹。故耕器具则战器备,衣事习则功战巧矣。当春三月,萩室熯造,钻隧易火,杼井易水,所以去兹毒也。举春祭,塞久祷,以鱼为牲,以糵为酒,相召,所以属亲戚也。毋杀畜生,毋拊卵,毋伐木,毋夭英,毋拊竿,所以息百长也。赐鳏寡,振孤独,贷无种,与无赋,所以劝弱民。发五正,赦薄罪,出拘民,解仇雠,所以建时功施生谷也。夏赏五德,满爵禄,迁官位,礼孝弟,复贤力,所以劝功也。秋行五刑,诛大罪,所以禁淫邪,止盗贼。冬收五藏,最万物,所以内作民也。四时事备,而民功百倍矣。故春仁、夏忠、秋急、冬闭,顺天之时,约地之宜,忠人之和,故风雨时,五谷实,草木美多,六畜蕃息,国富兵强,民材而令行,内无烦扰之政,外无强敌之患也。

夫动静顺然后和也,不夫其时然后富,不失其法然后治。故国不虚富,民不虚治。不治而昌,不乱而亡者,自古至今未尝有也。故国多私勇者其兵弱,吏多私智者其法乱,民多私利者其国贫。故德莫若博厚,使民死之;赏罚莫若必成,使民信之。

夫菩牧民者,非以城郭也,辅之以什,司之以伍。伍无非其人,人无非其里,里无非其家。故奔亡者无所匿,迁徙者无所容,不求而约,不召而来。故民无流亡之意,吏无备追之忧。故主政可往于民,民心可系于主。夫法之制民也,犹陶之于埴,冶之于金也。故审利害之所在,民之去就,如火之于燥湿,水之于高下。夫民之所生,衣与食也;食之所生,水与土也。所以富民有要,食民有率,率三十亩而足于卒岁。岁兼美恶,亩取一石,则人有三十石,果蓏素食当十石,糠秕六畜当十石,则人有五十石,布帛麻丝,旁入奇利,未在其中也。故国有余藏,民有余食。夫叙钧者,所以多寡也;权衡者,所以视重轻也;户籍田结者,所以知贫富之不訾也;故善者必先知其田,乃知其人,田备然后民可足也。

凡有天下者,以情伐者帝,以事伐者王,以政伐者霸。而谋有功者五,一曰视其所爱,以分其威,一人两心,其内必衰也。臣不用,其国可危。二曰视其阴所憎,厚其货赂,得情可深,身内情外,其国可知。三曰听其淫乐,以广其心,遗以竽瑟美人,以塞其内;遗以谄臣文马,以蔽其外。外内蔽塞,可以成败。四曰必深亲之,如典之同生。阴内辩士,使图其计;内勇士,使高其气。内人他国,使倍其约,绝其使,拂其意,是必士斗。两国相敌,必承其弊。五曰深察其谋,谨其忠臣,揆其所使,令内不信,使有离意。离气不能令,必内自贼。忠臣已死,故政可夺。此五者,谋功之道也。

\hypertarget{header-n700}{%
\subsection{入国}\label{header-n700}}

入国四旬,五行九惠之教。一曰老老,二曰慈幼,三曰恤孤,四曰养疾,五曰合独,六曰问疾,七曰通穷,八曰振困,九曰接绝。

所谓老老者,凡国、都皆有掌老,年七十已上,一子无征,三月有馈肉;八十已上,二子无征,月有馈肉;九十已上,尽家无征,日有酒肉。死,上共棺椁。劝子弟:精膳食,问所欲,求所嗜。此之谓老老。

所谓慈幼者,凡国、都皆有掌幼,士民有子,子有幼弱不胜养为累者,有三幼者无妇征,四幼者尽家无征,五幼又予之葆,受二人之食,能事而后止。此之谓慈幼。

所谓恤孤者,凡国、都皆有掌孤,士人死,子孤幼,无父母所养,不能自生者,属之其乡党、知识、故人。养一孤者一子无征,养二孤者二子无征,养三孤者尽家无征。掌孤数行问之,必知其食饮饥寒身之膌胜而哀怜之。此之谓恤孤。

所谓养疾者,凡国、都皆有掌养疾,聋、盲、喑、哑、跛辟、偏枯、握递,不耐自生者,上收而养之疾官,而衣食之,殊身而后止。此之谓养疾。

所谓合独者,凡国、都皆有掌媒,丈夫无妻曰鳏,妇人无夫曰寡,取鳏寡而合和之,予田宅而家室之,三年然后事之。此之谓合独。

所谓问疾者,凡国、都皆有掌病,士人有病者,掌病以上令问之。九十以上,日一问;八十以上,二日一问;七十以上,三日一问;众庶五日一问。疾甚者,以告上,身问之。掌病行于国中,以问病为事。此之谓问病。

所谓通穷者,凡国、都皆有通穷,若有穷夫妇无居处,穷宾客绝粮食,居其乡党以闻者有赏,不以闻者有罚,此之谓通穷。

\hypertarget{header-n711}{%
\subsection{九守}\label{header-n711}}

主位

安徐而静,柔节先定,虚心平意以待须。

主明

目贵明,耳贵聪,心贵智。以天下之目视则无不见也,以天下之耳听则无不闻也,以天下之心虑则无不知也。辐凑并进,则明不塞矣。

主听

听之术,曰:勿望而距,勿望而许。许之则失守,距之则闭塞。高山,仰之不可极也;深渊,度之不可测也。神明之德,正静其极也。

主赏

用赏者贵诚,用刑者贵必。刑赏信必于耳目之所见,则其所不见,莫不暗化矣。诚,畅乎天地,通于神明,见奸伪也?

主问

一曰天之,二曰地之,三曰人之,四(曰)上下,左右前后,荧惑其处安在?

主因

心不为九窍,九窍治;君不为五官,五官治。为善者,君予之赏;为非者,君予之罚。君因其所以来,因而予之,则不劳矣。圣人因之,故能掌之。因之修理,故能长久。

主周

人主不可不周。人主不周则群臣下乱。寂乎其无端也。外内不通,安知所怨?关闬不开,善否无原。

主参

一曰长目,二曰飞耳,三曰树明。明知千里之外,隐微之中,曰动奸。奸动则变更矣。

督名

修名而督实,按实而定名。名实相生,反相为情。名实当则治,不当则乱。名生于实,实生于德,德生于理,理生于智,智生于当。

\hypertarget{header-n732}{%
\subsection{桓公问}\label{header-n732}}

齐桓公问管子曰:``吾念有而勿失,得而勿忘,为之有道乎?''对曰:``勿创勿作,时至而随。毋以私好恶害公正,察民所恶,以自为戒。黄帝立明台之议者,上观于贤也;尧有衢室之问者,下听于人也;舜有告善之旌,而主不蔽也;禹立谏鼓于朝,而备讯唉;汤有总街之庭,以观人诽也;武王有灵台之复,而贤者进也。此古圣帝明王所以有而勿失,得而勿忘者也。''桓公曰:``吾欲效而为之,其名云何?''对曰:``名曰啧室之议。曰:法简而易行,刑审而不犯,事约而易从,求寡而易足。人有非上之所过,谓之正士,内于啧室之议。有司执事者咸以厥事奉职,而不忘为。此啧室之事也,请以东郭牙为之。此人能以正事争于君前者也。''桓公曰:``善。''\\
~\\

\hypertarget{header-n734}{%
\subsection{度地}\label{header-n734}}

昔者,桓公问管仲曰:``寡人请问度地形而为国者,其何如而可?''管仲对曰:``夷吾之所闻,能为霸王者,盖天子圣人也。故圣人之处国者,必于不倾之地,而择地形之肥饶者。乡山,左右经水若泽。内为落渠之写,因大川而注焉。乃以其天材、地之所生,利养其人,以育六畜。天下之人,皆归其德而惠其义。乃别制断之,州者谓之术,不满术者谓之里。故百家为里,里十为术,术十为州,州十为都,都十为霸国。不如霸国者,国也。以奉天子,天子有万诸侯也,其中有公侯伯子男焉。天子中而处,此谓因天之固,归地之利。内为之城,城外为之郭,郭外为之土阆,地高则沟之,下则堤之,命之曰金城。树以荆棘,上相穑著者,所以为固也。岁修增而毋已,时修增而毋已,福及孙子,此谓人命万世无穷之利,人君之葆守也。臣服之以尽忠于君,君体有之以临天下,故能为天下之民先也。此宰之任,则臣之义也。故善为国者,必先除其五害,人乃终身无患害而孝慈焉。''

桓公曰:``愿闻五害之说。''管仲对曰:``水,一害也;旱,一害也;风雾雹霜,一害也;厉,一害也;虫,一害也。此谓五害。五害之属,水最为大。五害已除,人乃可治。''桓公曰,``愿闻水害。''管仲对曰:``水有大小,又有远近。水之出于山,而流入于海者,命曰经水;水别于他水,入于大水及海者,命曰枝水;山之沟,一有水一毋水者,命曰谷水;水之出于他水沟,流于大水及海者,命曰川水;出地而不流者,命曰渊水。此五水者,因其利而往之可也,因而扼之可也,而不久常有危殆矣。''桓公曰:``水可扼而使东西南北及高乎?''管仲对曰:``可。夫水之性,以高走下则疾,至于石;而下向高,即留而不行,故高其上。领瓴之,尺有十分之三,里满四十九者,水可走也。乃迂其道而远之,以势行之。水之性,行至曲必留退,满则后推前,地下则平行,地高即控,杜曲则捣毁。杜曲激则跃,跃则倚,倚则环,环则中,中则涵,涵则塞,塞则移,移则控,控则水妄行;水妄行则伤人,伤人则困,困则轻法,轻法则难治,难治则不孝,不孝则不臣矣。故五害之属,伤杀之类,祸福同矣。知备此五者,人君天地矣。''

桓公曰:``请问备五害之道?''管子对曰:``请除五害之说,以水为始。请为置水官,令习水者为吏:大夫、大夫佐各一人,率部校长、官佐各财足。乃取水左右各一人,使为都匠水工。令之行水道、城郭、堤川、沟池、官府、寺舍及州中,当缮治者,给卒财足。令曰:常以秋岁末之时,阅其民,案家人比地,定什伍口数,别男女大小。其不为用者辄免之,有锢病不可作者疾之,可省作者半事之。并行以定甲士,当被兵之数,上其都。都以临下,视有余不足之处,辄下水官。水官亦以甲士当被兵之数,与三老、里有司、伍长行里,因父母案行。阅具备水之器,以冬无事之时。笼、臿、板、筑,各什六,土车什一,雨輂什二。食器两具,人有之,铜藏里中,以给丧器。后常令水官吏与都匠,因三老、里有司、伍长案行之。常以朔日始,出具阅之,取完坚,补弊久,去苦恶。常以冬少事之时,令甲士以更次益薪,积之水旁。州大夫将之,唯毋后时。其积薪也,以事之已;其作土也,以事未起。天地和调,日有长久,以此观之,其利百倍。故常以毋事具器,有事用之,水常可制,而使毋败。此谓素有备而豫具者也。''

桓公曰:``当何时作之?''管子曰:``春三月,天地乾燥,水纠列之时也。山川涸落,天气下,地气上,万物交通。故事已,新事未起,草木荑生可食。寒暑调,日夜分,分之后,夜日益短,昼日益长。利以作土功之事,土乃益刚。令甲士作堤大水之旁,大其下,小其上,随水而行。地有不生草者,必为之囊。大者为之堤,小者为之防,夹水四道,禾稼不伤。岁埤增之,树以荆棘,以固其地,杂之以柏杨,以备决水。民得其饶,是谓流膏,令下贫守之,往往而为界,可以毋败。当夏三月,天地气壮,大暑至,万物荣华,利以疾杀草薉,使令不欲扰,命曰不长。不利作土功之事,放农焉,利皆耗十分之五,土功不成。当秋三月,山川百泉踊,下雨降,山水出,海路距,雨露属,天地凑汐。利以疾作,收敛毋留,一日把,百日餔。民毋男女,皆行于野。不利作土功之事,濡湿日生,土弱难成。利耗什分之六,土工之事亦不立。当冬三月,天地闭藏,暑雨止,大寒起,万物实熟。利以填塞空郄,缮边城,涂郭术,平度量,正权衡,虚牢狱,实廥仓,君修乐,与神明相望。凡一年之事毕矣,举有功,赏贤,罚有罪,迁有司之吏而第之。不利作土工之事,利耗什分之七,土刚不立。昼日益短,而夜日益长,利以作室,不利以作堂。四时以得,四害皆服。''

桓公曰:``寡人悖,不知四害之服奈何?''管仲对曰:``冬作土功,发地藏,则夏多暴雨,秋霖不止。春不收枯骨朽脊,伐枯木而去之,则夏旱至矣。夏有大露原烟,噎下百草,人采食之伤人。人多疾病而不止、民乃恐殆。君令五官之吏,与三老、里有司、伍长行里顺之,令之家起火为温,其田及宫中皆盖井,毋令毒下及食器,将饮伤人。有下虫伤禾稼。凡天灾害之下也,君子谨避之,故不八九死也。大寒、大暑、大风、大雨,其至不时者,此谓四刑。或遇以死,或遇以生,君子避之,是亦伤人。故吏者所以教顺也,三老、里有司、伍长者,所以为率也。五者已具,民无愿者,愿其毕也:故常以冬日顺三老、里有司、伍长,以冬赏罚,使各应其赏而服其罚。五者不可害,则君之法犯矣。此示民而易见,故民不比也。''

桓公曰:``凡一年之中十二月,作土功,有时则为之,非其时而败,将何以待之?''管仲对曰:``常令水官之吏,冬时行堤防,可治者章而上之都。都以春少事作之。已作之后,常案行。堤有毁作,大雨,各葆其所,可治者趣治,以徒隶给。大雨,堤防可衣者衣之。冲水,可据者据之。终岁以毋败为固。此谓备之常时,祸何从来?所以然者,独水蒙壤,自塞而行者,江河之谓也。岁高其堤,所以不没也。春冬取土于中,秋夏取土于外,浊水入之不能为败。''桓公曰:``善。仲父之语寡人毕矣,然则寡人何事乎哉?亟为寡人教侧臣。''

\hypertarget{header-n743}{%
\subsection{地员}\label{header-n743}}

夫管仲之匡天下也,其施七尺。

渎田悉徙,五种无不宜,其立后而手实。其木宜蚖、菕与杜、松,其草宜楚棘。见是土也,命之曰五施,五七三十五尺而至于泉。呼音中角。其水仓,其民强。

赤垆,历强肥,五种无不宜。其麻白,其布黄,其草宜白茅与雚,其木宜赤棠。见是土也,命之曰四施,四七二十八尺而至于泉。呼音中商。其水白而甘,其民寿。

黄唐,无宜也,唯宜黍秫也。宜县泽。行廧落,地润数毁,难以立邑置廧。其草宜黍秫与茅,其木宜櫄、桑。见是土也,命之曰三施,三七二十一尺而至于泉。呼音中宫。其泉黄而糗,流徙。

斥埴,宜大菽与麦。其草宜萯、雚,其木宜杞。见是土也,命之曰再施,二七一十四尺而至于泉。呼音中羽。其泉咸,水流徙。

黑埴,宜稻麦。其草宜苹、蓨,其木宜白棠。见是土也,命之曰一施,七尺而至于泉。呼音中徵。其水黑而苦。

凡听徵,如负猪豕觉而骇。凡听羽,如鸣马在野。凡听宫,如牛鸣窌中。凡听商,如离群羊。凡听角,如雉登木以鸣,音疾以清。凡将起五音凡首,先主一而三之,四开以合九九,以是生黄钟小素之首,以成宫。三分而益之以一,为百有八,为徵。不无有三分而去其乘,适足,以是生商。有三分,而复于其所,以是成羽。有三分,去其乘,适足,以是成角。

坟延者,六施,六七四十二尺而至于泉。陕之芳七施,七七四十九尺而至于泉。祀陕八施,七八五十六尺而至于泉。杜陵九施,七九六十三尺而至于泉。延陵十施,七十尺而至于泉。环陵十一施,七十七尺而至于泉。蔓山十二施,八十四尺而至于泉。付山十三施,九十一尺而至于泉。付山白徒十四施,九十八尺而至于泉。中陵十五施,百五尺而至于泉。青山十六施,百一十二尺而至于泉,青龙之所居,庚泥,不可得泉,赤壤磝山十七施,百一十九尺而至于泉,其下清商,不可得泉。□山白壤十八施,百二十六尺而至于泉,其下骈石,不可得泉。徙山十九施,百三十三尺而至于泉,其下有灰壤,不可得泉。高陵土山二十施,百四十尺而至于泉。

山之上,命之曰悬泉,其地不干,其草如茅与走,其木乃樠,凿之二尺,乃至于泉。山之上,命曰复吕,其草鱼肠与莸,其木乃柳,凿之三尺而至于泉。山之上,命之曰泉英,其草蕲、白昌,其木乃杨,凿之五尺而至于泉。山之材,其草兢与蔷,其木乃格,凿之二七十四尺而至于泉。山之侧,其草葍与蒌,其木乃品榆,凿之三七二十一尺而至于泉。

凡草土之道,各有榖造。或高或下,各有草土。叶下于□,□下于苋,苋下于蒲,蒲下于苇,苇下于雚,雚下于蒌,蒌下于荓,荓下于萧,萧下于薜(薛),薜下于萑(蓷),萑下于茅。凡彼草物,有十二衰,各有所归。

九州之土,为九十物。每州有常,而物有次。

群土之长,是唯五粟。五粟之物,或赤或青或白或黑或黄,五粟五章。五粟之状,淖而不肕,刚而不觳,不泞车轮,不污手足。其种,大重细重,白茎白秀,无不宜也。五粟之土,若在陵在山,在隫在衍,其阴其阳,尽宜桐柞,莫不秀长。其榆其柳,其檿其桑,其柘其栎,其槐其杨,群木蕃滋数大,条直以长。其泽则多鱼,牧则宜牛羊。其地其樊,俱宜竹、箭、藻、龟、楢、檀。五臭生之:薜荔、白芷,麋芜、椒、连。五臭所校,寡疾难老,士女皆好,其民工巧。其泉黄白,其人夷姤。五粟之土,干而不格,湛而不泽,无高下,葆泽以处。是谓粟土。

粟土之次,曰五沃。五沃之物,或赤或青或黄或白或黑。五沃五物,各有异则。五沃之状,剽怷橐土,虫易全处,怷剽不白,下乃以泽。其种,大苗细苗,赨茎黑秀箭长。五沃之土,若在丘在山,在陵在冈,若在陬,陵之阳,其左其右,宜彼群木,桐、柞、枎、櫄,及彼白梓。其梅其杏,其桃其李,其秀生茎起,其棘其棠,其槐其杨,其榆其桑,其札其枋,群木数大,条直以长。其阴则生之楂蔾,其阳则安树之五麻,若高若下,不择畴所。其麻大者,如箭如苇,大长以美;其细者,如雚如蒸。欲有与各,大者不类,小者则治,揣而藏之,若众练丝,五臭畴生,莲、与、蘼芜,藁本、白芷。其泽则多鱼,牧则宜牛羊。其泉白青;其人坚劲,寡有疥骚,终无痟酲。五沃之土,干而不斥,湛而不泽,无高下,葆泽以处。是谓沃土。

沃土之次,曰五位。五位之物,五色杂英,各有异章。五位之状,不塥不灰,青怷以菭。及其种:大苇无、细苇无,赨茎白秀。五位之土,若在冈在陵,在隫在衍,在丘在山,皆宜竹、箭、求、黾、楢、檀。其山之浅,有茏与斥。群木安逐,条长数大:其桑其松,其杞其茸,种木胥容,榆、桃、柳、楝。群药安生,姜与桔梗,小辛、大蒙。其山之枭(注:或``阜''),多桔、符、榆;其山之末,有箭(注:或``葥'')与苑;其山之旁,有彼黄蝱,及彼白昌,山蔾、苇、芒。群药安聚,以圉民殃。其林其漉,其槐其楝,其柞其穀,群木安逐,鸟兽安施。既有麋麃,又且多鹿。其泉青黑,其人轻直,省事少食。无高下,葆泽以处。是谓位土。

位土之次,曰五蘟。五蔭之状,黑土黑菭,青怵以肥,芬然若灰,其种櫑葛,赨茎黄秀恚目,其叶若苑。以蓄殖果木,不若三土以十分之二,是谓蔭土。

蔭土之次,曰五壤。五壤之状,芬然若泽、若屯土。其种,大水肠、细水肠,赨茎黄秀以慈。忍水旱,无不宜也。蓄殖果木,不若三土以十分之二,是谓壤土。

壤土之次,曰五浮。五浮之状,捍然如米以葆泽,不离不坼。其种,忍蔭。忍叶如雚叶,以长狐茸。黄茎黑茎黑秀,其粟大,无不宜也。蓄殖果木,不如三上以十分之二。

凡上土三十物,种十二物。

中土曰五怷。五怷之状,廪焉如壏,润湿以处。其种,大稷、细稷,赨茎黄秀以慈。忍水旱,细粟如麻。蓄殖果木,不若三土以十分之三。

怷土之次,曰五纑。五纑之状,强力刚坚。其种,大邯郸、细邯郸,茎叶如枎櫄,其粟大。蓄殖果木,不若三土以十分之三。

纑土之次,曰五壏。五壏之状,芬焉若糠以肥。其种,大荔、细荔,青茎黄秀。蓄殖果木,不若三土以十分之三。

壏土之次,曰五剽。五剽之状,华然如芬以脆。其种,大秬、细秬,黑茎青秀。蓄殖果木,不若三土以十分之四。

剽土之次,曰五沙。五沙之状,粟焉如屑尘厉。其种,大萯、细萯,白茎青秀以蔓。蓄殖果木,不如三土以十分之四。

沙土之次,曰五塥。五塥之状,累然如仆累,不忍水旱。其种,大樛杞、细樛杞,黑茎黑秀。蓄殖果木,不若三土以十分之四。

凡中土三十物,种十二物。

下土曰五犹。五犹之状如粪。其种,大华、细华、白茎黑秀。蓄殖果木,不如三土以十分之五。

犹土之次,曰五壮。五壮之状如鼠肝。其种,青梁,黑茎黑秀。蓄殖果木,不如三土以十分之五。

壮土之次,曰五殖。五殖之状,甚泽以疏,离坼以臞塉。其种,雁膳黑实,朱跗黄实。蓄殖果木,不如三土以十分之六。

五殖之次,曰五觳。五觳之状娄娄然,不忍水旱。其种,大菽、细菽,多白实。蓄殖果木,不如三土以十分之六。

觳土之次,曰五凫。五凫之状,坚而不骼。其种,陵稻、黑鹅、马夫。蓄殖果木,不如三土以十分之七。

凫土之次,曰五桀,五桀之状,甚咸以苦,其物为下。其种,白稻长狭。蓄殖果木,不如三土以十分之七。

凡下土三十物,其种十二物。

凡土物九十,其种三十六。

\hypertarget{header-n779}{%
\subsection{形势解}\label{header-n779}}

山者,物之高者也。惠者,主之高行也。慈者,父母之高行也。忠者,臣之高行也。孝者,子妇之高行也。故山高而不崩则祈羊至,主惠而不解则民奉养,父母慈而不解则子妇顺,臣下忠而不解则爵禄至,子妇孝而不解则美名附。故节高而不解,则所欲得矣;解,则不得。故曰:``山高而不崩则祈羊至矣。''

渊者,众物之所生也,能深而不涸,则沈玉至。主者,人之所仰而生也,能宽裕纯厚而不苛忮,则民人附。父母者,子妇之所受教也,能慈仁教训而不失理,则子妇孝。臣下者,主之所用也,能尽力事上,则当于主。子妇者,亲之所以安也,能孝弟顺亲,则当于亲。故渊涸而无水则沈玉不至,主苛而无厚则万民不附,父母暴而无恩则子妇不亲,臣下随而不忠则卑辱困穷,子妇不安亲则祸忧至。故渊不涸,则所欲者至;涸,则不至。故曰:``渊深而不涸则沈玉极。''

天,覆万物,制寒暑,行日月,次星辰,天之常也。治之以理,终而复始。主,牧万民,治天下,莅百官,主之常也。治之以法,终而复始。和子孙,属亲戚,父母之常也。治之以义,终而复治。敦敬忠信,臣下之常也。以事其主,终而复始。爱亲善养,思敬奉教,子妇之常也。以事其亲,终而复始。故天不失其常,则寒暑得其时,日月星辰得其序。主不失其常,则群臣得其义,百官守其事。父母不失其常,则子孙和顺,亲戚相欢。臣下不失其常,则事无过失,而官职政治。子妇不失其常,则长幼理而亲疏和。故用常者治,失常者乱。天未尝变,其所以治也。故曰:``天不变其常。''

地生养万物,地之则也。治安百姓,主之则也。教护家事,父母之则也。正谏死节,臣下之则也。尽力共养,子妇之则也。地不易其则,故万物生焉。主不易其则,故百姓安焉。父母不易其则,故家事辨焉。臣下不易其则,故主无过失。子妇不易其则,故亲养备具。故用则者安,不用则者危。地未尝易,其所以安也。故曰:``地不易其则。''

春者,阳气始上,故万物生。夏者,阳气毕上,故万物长。秋者,阴气始下,故万物收。冬者,阴气毕下,故万物藏。故春夏生长,秋冬收藏,四时之节也。赏赐刑罚,主之节也。四时未尝不生杀也,主未尝不赏罚也。故曰:``春秋冬夏不更其节也。''

天,覆万物而制之;地,载万物而养之;四时,生长万物而收藏之。古以至今,不更其道。故曰:``古今一也。''

蛟龙,水虫之神者也。乘于水则神立,失于水则神废。人主,天下之有威者也。得民则威立,失民则威废。蛟龙待得水而后立其神,人主待得民而后成其威。故曰:``蛟龙得水而神可立也。''

虎豹,兽之猛者也,居深林广泽之中则人畏其威而载之。人主,天下之有势者也,深居则人畏其势。故虎豹去其幽而近于人,则人得之而易其威。人主去其门而迫于民,则民轻之而傲其势。故曰:``虎豹托幽而威可载也。''

风,漂物者也。风之所漂,不避贵贱美恶。雨,濡物者也。雨之所堕,不避小大强弱。风雨至公而无私,所行无常乡,人虽遇漂濡而莫之怨也。故曰:``风雨无乡而怨怒不及也。''

人主之所以令则行禁则止者,必令于民之所好而禁于民之所恶也。民之情莫不欲生而恶死,莫不欲利而恶害。故上令于生、利人,则令行;禁于杀、害人,则禁止。令之所以行者,必民乐其政也,而令乃行。故曰:``贵有以行令也。''

人主之所以使下尽力而亲上者,必为天下致利除害也。故德泽加于天下,惠施厚于万物,父子得以安,群生得以育,故万民欢尽其力而乐为上用。入则务本疾作以实仓廪,出则尽节死敌以安社稷,虽劳苦卑辱而不敢告也。此贱人之所以亡其卑也。故曰``贱有以亡卑。''

起居时,饮食节,寒暑适,则身利而寿命益,起居不时,饮食不节,寒暑不适,则形体累而寿命损。人惰而侈则贫,力而俭则富。夫物莫虚至,必有以也。故曰:``寿夭贫富无徒归也。''

法立而民乐之,令出而民衔之,法令之合于民心如符节之相得也,则主尊显。故曰:``衔令者君之尊也。''

人主出言,顺于理,合于民情,则民受其辞。民受其辞则名声章。故曰:``受辞者名之运也。''

明主之治天下也,静其民而不扰,佚其民而不劳。不扰则民自循;不劳则民自试。故曰:``上无事而民自试。''

人主立其度量,陈其分职,明其法式,以莅其民,而不以言先之,则民循正。所谓抱蜀者,祠器也。故曰:``抱蜀不言而庙堂既修。''

将将鸿鹄,貌之美者也。貌美,故民歌之。德义者,行之美者也。德义美,故民乐之。民之所歌乐者,美行德义也,而明主鸿鹄有之。故曰:``鸿鹄将将,维民歌之。''\\
济济者,诚庄事断也;多士者,多长者也。周文王诚庄事断,故国治。其群臣明理以佐主,故主明。主明而国治,竟内被其利泽,殷民举首而望文王,愿为文王臣。故曰:``济济多士,殷民化之。''

纣之为主也,劳民力,夺民财,危民死,冤暴之令,加于百姓;憯毒之使,施于天下。故大臣不亲,小民疾怨,天下叛之而愿为文王臣者,纣自取之也。故曰:``纣之失也。''

无仪法程式,蜚摇而无所定,谓之蜚蓬之间。蜚蓬之间,明主不听也。无度之言,明主不许也。故曰:``蜚蓬之间,不在所宾。''

道行则君臣亲,父子安,诸生育。故明主之务,务在行道,不顾小物。燕爵,物之小者也。故曰:``燕爵之集,道行不顾。''

明主之动静得理义,号令顺民心,诛杀当其罪,赏赐当其功,故虽不用牺牲珪璧祷于鬼神,鬼神助之,天地与之,举事而有福。乱主之动作失义理,号令逆民心,诛杀不当其罪,赏赐不当其功,故虽用牺牲珪璧祷于鬼神,鬼神不助,天地不与,举事而有祸。故曰:``牺牲珪璧不足以享鬼神。''

主之所以为功者,富强也。故国富兵强,则诸侯服其政,邻敌畏其威,虽不用宝币事诸侯,诸侯不敢犯也。主之所以为罪者,贫弱也。故国贫兵弱,战则不胜,守则不固,虽出名器重宝以事邻敌,不免于死亡之患。故曰:``主功有素,宝币奚为?''

羿,古之善射者也。调和其弓矢而坚守之。其操弓也,审其高下,有必中之道,故能多发而多中。明主,犹羿也,平和其法,审其废置而坚守之,有必治之道,故能多举而多当。道者,羿之所以必中也,主之所以必治也。射者,弓弦发矢也。故曰:``羿之道非射也。''

造父,善驭马者也。善视其马,节其饮食,度量马力,审其足走,故能取远道而马不罢,明主,犹造父也。善治其民,度量其力,审其技能,故立功而民不困伤。故术者,造父之所以取远道也,主之所以立功名也。驭者,操辔也。故曰,``造父之术非驭也。''

奚仲之为车器也,方圜曲直皆中规矩钩绳,故机旋相得,用之牢利,成器坚固。明主,犹奚仲也,言辞动作,皆中术数,故众理相当,上下相亲。巧者,奚仲之所以为器也,主之所以为治也。斫削者,斤刀也。故曰:``奚仲之巧非斫削也。''

民,利之则来,害之则去。民之从利也,如水之走下,于四方无择也。故欲来民者,先起其利,虽不召而民自至。设其所恶,虽召之而民不来也。故曰:``召远者使无为焉。''

莅民如父母,则民亲爱之。道之纯厚,遇之(有)【真】实,虽不言曰吾亲民,而民亲矣。莅民如仇雠,则民疏之。道之不厚,遇之无实,诈伪并起,虽言曰吾亲民,民不亲也。故曰:``亲近者言无事焉。''

明主之使远者来而近者亲也,为之在心。所谓夜行者,心行也。能心行德,则天下莫能与之争矣。故曰:``唯夜行者独有之乎。''

为主而贼,为父母而暴,为臣下而不忠,为子妇而不孝,四者人之大失也。大失在身,虽有小善,不得力贤。所谓平原者,下泽也,虽有小封,不得为高。故曰:``平原之隰,奚有于高?''

为主而惠,为父母而慈,为臣下而忠,为子妇而孝,四者人之高行也。高行在身,虽有小过,不为不肖。所谓大山者,山之高者也,虽有小隈,不以为深。故曰:``大山之隈,奚有于深?''

毁訾贤者之谓訾,推誉不肖之谓讆。訾讆之人得用,则人主之明蔽,而毁誉之言起。任之大事,则事不成而祸患至。故曰:``訾讆之人,勿与任大。''

明主之虑事也,为天下计者,谓之譕臣。譕臣则海内被其泽,泽布于天下,后世享其功久远而利愈多。故曰:``譕臣者可与远举。''

圣人择可言而后言,择可行而后行。偷得利而后有害,偷得乐而后有忧者,圣人不为也。故圣人择言必顾其累,择行必顾其忧。故曰:``顾忧者可与致道。''

小人者,枉道而取容,适主意而偷说,备利而偷得。如此者,其得之虽速,祸患之至亦急。故圣人去而不用也。故曰:``其计也速而忧在近者,往而勿召也。''

举一而为天下长利者,谓之举长。举长则被其利者众,而德义之所见远。故曰:``举长者可远见也。''

天之裁大,故能兼覆万物;地之裁大,故能兼载万物;人主之裁大,故容物多而众人得比焉。故曰:``裁大者众之所比也。''

贵富尊显,民归乐之,人主莫不欲也。故欲民之怀乐己者,必服道德而勿厌也,而民怀乐之。故曰:``美人之怀,定服而勿厌也。''

圣人之求事也,先论其理义,计其可否。故义则求之,不义则止。可则求之,不可则止。故其所得事者,常为身宝。小人之求事也,不论其理义,不计其可否,不义亦求之,不可亦求之。故其所得事者,未尝为赖也。故曰:``必得之事,不足赖也。''

圣人之诺已也,先论其理义,计其可否。义则诺,不义则已;可则诺,不可则已。故其诺未尝不信也。小人不义亦诺,不可亦诺,言而必诺。故其诺未必信也。故曰:``必诺之言,不足信也。''

谨于一家,则立于一家;谨于一乡,则立于一乡;谨于一国,则立于一国;谨于天下,则立于天下。是故其所谨者小,则其所立亦小;其所谨者大,则其所立亦大。故曰:``小谨者不大立。''

海不辞水,故能成其大;山不辞土石,故能成其高;明主不厌人,故能成其众;士不厌学,故能成其圣。飺者,多所恶也。谏者,所以安主也;食者,所以肥体也。主恶谏则不安,人飺食则不肥。故曰:``飺食者不肥体也。''

言而语道德忠信孝弟者,此言无弃者。天公平而无私,故美恶莫不覆;地公平而无私,故小大莫不载。无弃之言,公平而无私,故贤不肖莫不用。故无弃之言者,参伍于天地之无私也。故曰:``有无弃之言者,必参之于天地矣。''

明主之官物也,任其所长,不任其所短,故事无不成而功无不立。乱主不知物之各有所长所短也,而责必备。夫虑事定物,辩明礼义,人之所长而蝚蝯之所短也;缘高出险,蝚蝯之所长而人之所短也。以蝚蝯之所长责人,故其令废而责不塞。故曰:``坠岸三仞,人之所大难也,而蝚蝯饮焉。''

明主之举事也,任圣人之虑,用众人之力,而不自与焉。故事成而福生。乱主自智也,而不因圣人之虑;矜奋自功,而不因众人之力;专用己,而不听正谏,故事败而祸生。故曰:``伐矜好专,举事之祸也。''

马者,所乘以行野也。故虽不行于野,其养食马也,未尝解惰也。民者,所以守战也。故虽不守战,其治养民也,未尝解惰也。故曰:``不行其野,不违其马。''

天生四时,地生万财,以养万物而无取焉。明主配天地者也,教民以时,劝之以耕织,以厚民养,而不伐其功,不私其利。故曰:``能予而无取者,天地之配也。''

解惰简慢,以之事主则不忠,以之事父母则不孝,以之起事则不成。故曰:``怠倦者不及也。''

以规矩为方圜则成,以尺寸量长短则得,以法数治民则安。故事不广于理者,其成若神。故曰:``无广者疑神。''

事主而不尽力则有刑,事父母而不尽力则不亲,受业问学而不加务则不成。故朝不勉力务进,夕无见功。故曰:``朝忘其事,夕失其功。''

中情信诚则名誉美矣,修行谨敬则尊显附矣。中无情实则名声恶矣,修行慢易则污辱生矣。故曰:``邪气袭内。正色乃衰也。''

为人君而不明君臣之义以正其臣,则臣不知于为臣之理以事其主矣。故曰:``君不君则臣不臣。''

为人父而不明父子之义以教其子而整齐之,则子不知为人子之道以事其父矣。故曰:``父不父则子不子。''

君臣亲,上下和,万民辑,故主有令则民行之,上有禁则民不犯。君臣不亲,上下不和,万民不辑,故令则不行,禁则不止。故曰:``上下不和,令乃不行。''

言辞信,动作庄,衣冠正,则臣下肃。言辞慢,动作亏,衣冠惰,则臣下轻之。故曰:``衣冠不正则宾者不肃。''

仪者,万物之程式也。法度者,万民之仪表也。礼义者,尊卑之仪表也。故动有仪则令行,无仪则令不行。故曰:``进退无仪则政令不行。''

人主者,温良宽厚则民爱之,整齐严庄则民畏之。故民爱之则亲,畏之则用。夫民亲而为用,王之所急也。故曰:``且怀且威则君道备矣。''

人主能安其民,则事其主如事其父母。故主有忧则忧之,有难则死之。主视民如土,则民不为用,主有忧则不忧,有难则不死。故曰:``莫乐之则莫哀之,莫生之则莫死之。''

民之所以守战至死而不衰者,上之所以加施于民者厚也。故上施厚,则民之报上亦厚;上施薄,则民之报上亦薄。故薄施而厚责,君不能得之于臣,父不能得之于子。故曰:``往者不至,来者不极。''

道者,扶持众物,使得生育,而各终其性命者也。故或以治乡,或以治国,或以治天下。故曰:``道之所言者一也,而用之者异。''

闻道而以治一乡,亲其父子,顺其兄弟,正其习俗,使民乐其上,安其土,为一乡主干者,乡之人也。故曰:``有闻道而好为乡者,一乡之人也。''

民之从有道也,如饥之先食也,如寒之先衣也,如暑之先阴也。故有道则民归之,无道则民去之。故曰:``道往者其人莫来,道来者其人莫往。''

道者,所以变化身而之正理者也,故道在身则言自顺,行自正,事君自忠,事父自孝,遇人自理。故曰:``道之所设,身之化也。''

天之道,满而不溢,盛而不衰。明主法象天道,故贵而不骄,富而不奢,行理而不惰。故能长守贵富,久有天下而不失也。故曰:``持满者与天。''

明主救天下之祸,安天下之危者也。夫救祸安危者,必待万民之为用也,而后能为之。故曰:``安危者与人。''

地大国富,民众兵强,此盛满之国也。虽已盛满,无德厚以安之,无度数以治之,则国非其国,而民无其民也。故曰,``失天之度,虽满必涸。''

臣不亲其主,百姓不信其吏,上下离而不和,故虽自安,必且危之。故曰:``上下不和,虽安必危。''

主有天道,以御其民,则民一心而奉其上,故能贵富而久王天下。失天之道,则民离叛而不听从,故主危而不得久王天下。故曰:``欲王天下而失天之道,天下不可得而王也。''

人主务学术数,务行正理,则化变日进,至于大功,而愚人不知也。乱主淫佚邪枉,日为无道,至于灭亡,而不自知也。故曰:``莫知其为之,其功既成;莫知其舍之也,藏之而无形。''

古者三王五伯皆人主之利天下者也,故身贵显而子孙被其泽。桀,纣、幽、厉皆人主之害天下者也,故身困伤而子孙蒙其祸。故曰:``疑今者察之古,不知来者视之往。''

神农教耕生谷,以致民利。禹身决渎,斩高桥下,以致民利。汤武征伐无道,诛杀暴乱,以致民利。故明王之动作虽异,其利民同也。故曰:``万事之任也,异起而同归,古今一也。''

栋生桡不胜任则屋覆,而人不怨者,其理然也。弱子,慈母之所爱也,不以其理动者,下瓦则慈母笞之。故以其理动者,虽覆屋不为怨;不以其理动者,下瓦必笞。故曰:``生栋覆屋,怨怒不及;弱子下瓦,慈母操箠。''

行天道,出公理,则远者自亲;废天道,行私为,则子母相怨。故曰:``天道之极,远者自亲;人事之起,近亲造怨。''

古者,武王地方不过百里,战卒之众不过万人,然能战胜攻取,立为天子,而世谓之圣王者,知为之之术也。桀、纣贵为天子,富有海内,地方甚大,战卒甚众,而身死国亡,为天下僇者,不知为之之术也。故能为之,则小可为大,贱可为贵。不能为之,则虽为天子,人犹夺之也。故曰:``巧者有余而拙者不足也。''

明主上不逆天,下不圹地,故天予之时,地生之财。乱主上逆天道,下绝地理,故天不予时,地不生财。故曰:``其功顺天者,天助之;其功逆天者,天违之。''

古者,武王,天之所助也。故虽地小而民少,犹之为天子也。桀纣,天之所违也,故虽地大民众,犹之困辱而死亡也。故曰:``天之所助,虽小必大;天之所违,虽大必削。''

与人交,多诈伪无情实,偷取一切,谓之乌集之交。乌集之交,初虽相欢,后必相咄。故曰:``乌集之交,虽善不亲。''

圣人之与人约结也,上观其事君也,内观其事亲也,必有可知之理,然后约结。约结而不袭于理,后必相倍。故曰:``不重之结,虽固必解。道之用也,贵其重也。''

明主与圣人谋,故其谋得;与之举事,故其事成。乱主与不肖者谋,故其计失;与之举事,故其事败。夫计失而事败,此与不可之罪。故曰:``毋与不可。''

主度量人力之所能为,而后使焉。故令于人之所能为,则令行;使于人之所能为,则事成。乱主不量人力,令于人之所不能为,故其令废;使于人之所不能为,故其事败。夫令出而废,举事而败,此强不能之罪也。故曰:``毋强不能。''

狂惑之人,告之以君臣之义、父子之理、贵贱之分,不信圣人之言也,而反害伤之。故圣人不告也。故曰:``毋告不知。''

与不肖者举事,则事败;使于人之所不能为,则令废;告狂惑之人,则身害。故曰:``与不可,强不能,告不知,谓之劳而无功。''

常以言翘明,其与人也,其爱人也,其有德于人也,以此为友则不亲,以此为交则不结,以此有德于人则不报。故曰:``见与之友,几于不亲;见爱之交,几于不结;见施之德,几于不报。四方之所归,心行者也。''

明主不用其智,而任圣人之智;不用其力,而任众人之力。故以圣人之智思虑者,无不知也;以众人之力起事者,无不成也。能自去而因天下之智力起,则身逸而福多。乱主独用其智,而不任圣人之智;独用其力,而不任众人之力,故其身劳而祸多。故曰:``独任之国,劳而多祸。''

明主内行其法度,外行其理义,故邻国亲之,与国信之,有患则邻国忧之,有难则邻国救之。乱主内失其百姓,外不信于邻国,故有患则莫之忧也,有难则莫之救也,外内皆失,孤特而无党,故国弱而主辱。故曰,``独国之君,卑而不威。''

明主之治天下也,必用圣人,而后天下治;妇人之求夫家也,必用媒,而后家事成。故治天下而不用圣人,则天下乖乱而民不亲也;求夫家而不用媒,则丑耻而人不信也。故曰,``自媒之女,丑而不信。''

明主者,人未之见而有亲心焉者,有使民亲之之道也。故其位安而民往之。故曰:``未之见而亲焉,可以往矣。''

尧舜,古之明主也。天下推之而不倦,誉之而不厌,久远而不忘者,有使民不忘之道也。故其位安而民来之。故曰:``久而不忘焉,可以来矣。''

日月,昭察万物者也,天多云气,蔽盖者众,则日月不明。人主,犹日月也,群臣多奸立私,以拥蔽主,则主不得昭察其臣下,臣下之情不得上通。故奸邪日多而人主愈蔽。故曰:``日月不明,天不易也。''

山,物之高者也。地险秽不平易,则山不得见。人主,犹山也,左右多党比周以壅其主,则主不得见。故曰:``山高而不见,地不易也。''

人主出言不逆于民心,不悖于理义,其所言足以安天下者也,人唯恐其不复言也。出言而离父子之亲,疏君臣之道,害天下之众,此言之不可复者也,故明主不言也。故曰:``言而不可复者,君不言也。''

人主身行方正,使人有礼,遇人有理,行发于身而为天下法式者,人唯恐其不复行也。身行不正,使人暴虐,遇人不信,行发于身而为天下笑者,此不可复之行,故明主不行也。故曰:``行而不可再者,君不行也。''

言之不可复者,其言不信也;行之不可再者,其行贼暴也。故言而不信则民不附,行而贼暴则天下怨。民不附,天下怨,此灭亡之所从生也,故明主禁之。故曰,``凡言之不可复,行之不可再者,有国者之大禁也。''

\hypertarget{header-n874}{%
\subsection{立政九败解}\label{header-n874}}

人君唯毋听寝兵,则群臣宾客莫敢言兵。然则内之不知国之冶乱,外之不知诸侯强弱,如是则城郭毁坏,莫之筑补;甲弊兵簓,莫之修缮。如是则守圉之备毁矣,辽远之地谋,边竟之士修,百姓无圉敌之心。故曰,``寝兵之说胜,则险阻不守。''

人君唯毋听兼爱之说,则视天下之民如其民,视国如吾国。如是则无并兼攘夺之心,无覆军败将之事。然则射御勇力之士不厚禄,覆军杀将之臣不贵爵,如是则射御勇力之士出在外矣。我能毋攻人可也,不能令人毋攻我。彼求地而予之,非吾所欲也,不予而与战,必不胜也。彼以教士,我以驱众;彼以良将,我以无能,其败必覆军杀将。故曰:``兼爱之说胜,则士卒不战。''

人君唯无好全生,则群臣皆全其生,而生又养。生养何也?曰:滋味也,声色也,然后为养生。然则从欲妄行,男女无别,反于禽兽。然则礼义廉耻不立,人君无以自守也。故曰:``全生之说胜,则廉耻不立。''

人君唯无听私议自贵,则民退静隐伏,窟穴就山,非世间上,轻爵禄而贱有司。然则令不行禁不止。故曰:``私议自贵之说胜,则上令不行。''

人君唯无好金玉货财,必欲得其所好,然则必有以易之。所以易之者何也?大官尊位,不然则尊爵重禄也。如是则不肖者在上位矣。然则贤者不为下,智者不为谋,信者不为约,勇者不为死。如是则驱国而捐之也。故曰:``金玉货财之说胜,则爵服下流。''

人君唯毋听群徒比周,则群臣朋党,蔽美扬恶。然则国之情伪不见于上。如是则朋党者处前,寡党者处后。夫朋党者处前,贤、不肖不分,则争夺之乱起,而君在危殆上,党与成于乡。如是则货财行于国,法制毁于官,群臣务佼而不求用,然则无爵而贵,无禄而富。故曰:``请谒任举之说胜,则绳墨不正。''

人君唯无听谄谀饰过之言,则败。奚以知其然也?夫谄臣者,常使其主不悔其过不更其失者也,故主惑而不自知也,如是则谋臣死而谄臣尊矣。故曰:``谄谗饰过之说胜,则巧佞者用。''

\hypertarget{header-n884}{%
\subsection{版法解}\label{header-n884}}

版法者,法天地之位,象四时之行,以治天下。四时之行,有寒有暑,圣人法之,故有文有武。天地之位,有前有后,有左有右,圣人法之,以建经纪。春生于左,秋杀于右;夏长于前,冬藏于后。生长之事,文也;收藏之事,武也。是故文事在左,武事在右,圣人法之,以行法令,以治事理。凡法事者,操持不可以不正,操持不正则听治不公;听治不公则治不尽理,事不尽应。治不尽理,则疏远微贱者无所告;事不尽应,则功利不尽举。功利不尽举则国贫,疏远微贱者无所告则下饶。故臼:``凡将立事,正彼天植。''

天植者,心也。天棺正,则不私近亲,不孽疏远。不私近亲,不孽疏远,则无遗利,无隐治。无遗利,无隐治,则事无不举,物无遗者。欲见天心,明以风雨。故曰:``风雨无违,远近高下,各得其嗣。''

万物尊天而贵风雨。所以尊天者,为其莫不受命焉也;所以贵风雨者,为其莫不待风而动待雨而濡也。若使万物释天而更有所受命,释风而更有所仰动,释雨而更有所仰濡,则无为尊天而贵风雨矣。今人君之所尊安者,为其威立而令行也。其所以能立威行令者,为其威利之操莫不在君也。若使威利之操不专在君,而有所分散,则君日益轻而威利日衰,侵暴之道也。故曰:``三经既饬,君乃有国。''

乘夏方长,审治刑赏,必明经纪,陈义设法。断事以理,虚气平心,乃去怒喜。若倍法弃令而行怒喜,祸乱乃生,上位乃殆。故曰:``喜无以赏,怒无以杀。喜以赏,怒以杀,怨乃起,令乃废。骤令而不行,民心乃外,外之有徒,祸乃始牙。众之所忿,寡不能图。''

冬既闭藏,百事尽止,往事毕登,来事未起。方冬无事,慎观终始,审察事理。事有先易而后难者,有始不足见而终不可及者;此常利之所以不举,事之所以困者也。事之先易者,人轻行之,人轻行之,则必困难成之事;始不足见者,人轻弃之,人轻弃之,则必失不可及之功。夫数困难成之事,而时失不可及之功,衰耗之道也。是故明君审察事理,慎观终始,为必知其所成,成必知其所用,用必知其所利害。为而不知所成,成而不知所用,用而不知所利害,谓之妄举。妄举者,其事不成,其功不立。故曰:``举所美必观其所终,废所恶必计其所穷。''

凡人君者,欲民之有礼义也。夫民无礼义,则上下乱而贵贱争。故曰:``庆勉敦敬以显之,富禄有功以劝之,爵贵有名以休之。''

凡人君者,欲众之亲上乡意也,欲其从事之胜任也。而众者,不爱则不亲,不亲则不明,不教顺则不乡意。是故明君兼爱以亲之,明教顺以道之,便其势,利其备,爱其力,而勿夺其时以利之。如此则众亲上乡意,从事胜任矣。故曰:``兼爱无遗,是谓君心。必先顺教,万民乡风。旦暮利之,众乃胜任。''

治之本二:一曰人,二曰事。人欲必用,事欲必工。人有逆顺,事有称量。人心逆则人不用,事失称量则事不工。事不工则伤,人不用则怨。故曰:``取人以己,成事以质。''

成事以质者,用称量也。取人以己者,度恕而行也。度恕者,度之于己也,己之所不安,勿施于人。故曰:``审用财,慎施报,察称量。故用财不可以啬,用力不可以苦。用财啬则费,用力苦则劳矣。''

奚以知其然也?用力苦则事不工,事不工而数复之,故曰劳矣。用财啬则不当人心,不当人心则怨起。用财而生怨,故曰费。怨起而不复反,众劳而不得息,则必有崩阤堵坏之心。故曰:``民不足,令乃辱;民苦殃,令不行。施报不得,祸乃始昌;祸昌而不悟,民乃自图。''

凡国无法则众不知所为,无度则事无机,有法不正,有度不直,则治辟。治辟则国乱。故曰:``正法直度,罪杀不赦。杀僇必信,民畏而惧。武威既明,令不再行。''

凡民者,莫不恶罚而畏罪。是以人君严教以示之,明刑罚以致之。故曰:``顿卒怠倦以辱之,罚罪有过以惩之,杀僇犯禁以振之。''

治国有三器,乱国有六攻。明君能胜六攻而立三器,则国治;不肖之君不能胜六攻而立三器,故国不治。三器者何也?曰:号令也、斧钺也、禄赏也。六攻者何也?亲也、贵也、货也、色也、巧佞也、玩好也。三器之用何也?曰:非号令无以使下,非斧钺无以畏众,非禄赏无以劝民。六攻之败何也?曰:虽不听而可以得存,虽犯禁而可以得免,虽无功而可以得富。夫国有不听而可以得存者,则号令不足以使下;有犯禁而可以得免者,则斧钺不足以畏众;有无功而可以得富者,则禄赏不足以劝民。号令不足以使下,斧钺不足以畏众,禄赏不足以劝民,则人君无以自守也。然则明君奈何?明君不为六者变更号令,不为六者疑错斧钺,不为六者益损禄赏。故曰:``植固而不动,奇邪乃恐。奇革邪化,令往民移。''

凡人君者,覆载万民而兼有之,烛临万族而事使之。是故以天地、日月、四时为主、为质,以治天下。天覆而无外也,其德无所不在;地载而无弃也,安固而不动,故莫不生殖。圣人法之以覆载万民,故莫不得其职姓,得其职姓,则莫不为用。故曰:``法天合德,象地无亲。''

日月之明无私,故莫不得光。圣人法之,以烛万民,故能审察,则无遗善,无隐奸。无遗善,无隐奸,则刑赏信必。刑赏信必,则善劝而奸止。故曰:``参于日月。''

四时之行,信必而著明。圣人法之,以事万民,故不失时功。故曰:``伍于四时。''

凡众者,爱之则亲,利之则至。是故明君设利以致之,明爱以亲之。徒利而不爱,则众至而不亲;徒爱而不利,则众亲而不至。爱施俱行,则说君臣、说朋友、说兄弟、说父子。爱施所设,四固不能守。故曰:``说在爱施。''

凡君所以有众者,爱施之德也。爱有所移,利有所并,则不能尽有。故曰:``有众在废私。''

爱施之德虽行而无私,内行不修,则不能朝远方之君。是故正君臣上下之义,饰父子兄弟夫妻之义,饰男女之别,别疏数之差,使君德臣忠,父慈子孝,兄爱弟敬,礼义章明。如此则近者亲之,远者归之。故曰:``召远在修近。''

闭祸在除怨,非有怨乃除之,所事之地常无怨也。凡祸乱之所生,生于怨咎;怨咎所生,生于非理。是以明君之事众也必经,使之必道,施报必当,出言必得,刑罚必理。如此则众无郁怨之心,无憾恨之意,如此则祸乱不生,上位不殆。故曰:``闭祸在除怨也。''

凡人君所以尊安者,贤佐也。佐贤则君尊、国安、民治,无佐则君卑、国危、民乱。故曰:``备长存乎任贤。''

凡人者,莫不欲利而恶害,是故与天下同利者,天下持之;擅天下之利者,天下谋之。天下所谋,虽立必隳;天下所持,虽高不危。故曰:``安高在乎同利。''

\hypertarget{header-n909}{%
\subsection{明法解 }\label{header-n909}}

明主者,有术数而不可得欺也,审于法禁而不可犯也,察于分职而不可乱也。故群臣不敢行其私,贵臣不得蔽贱,近者不得塞远,孤寡老弱不失其(所)职,竟内明辨而不相逾越。此之谓治国。故《明法》曰:``所谓治国者,主道明也。''

明主者,上之所以一民使下也。私术者,下之所以侵上乱主也。故法废而私行,则人主孤特而独立,人臣群党而成朋。如此则主弱而臣强,此之谓乱国。故《明法》曰:``所谓乱国者,臣术胜也。''

明主在上位,有必治之势,则群臣不敢为非。是故群臣之不敢欺主者,非爱主也,以畏主之威势也;百姓之争用,非以爱主也,以畏主之法令也。故明主操必胜之数,以治必用之民;处必尊之势,以制必服之臣。故令行禁止,主尊而臣卑。故《明法》曰:``尊君卑臣,非计亲也,以势胜也。''

明主之治也,县爵禄以劝其民,民有利于上,故主有以使之;立刑罚以威其下,下有畏于上,故主有以牧之。故无爵禄则主无以劝民,无刑罚则主无以威众。故人臣之行理奉命者,非以爱主也,且以就利而避害也;百官之奉法无奸者,非以爱主也,欲以爱爵禄而避罚也。故《明法》曰:``百官论职,非惠也,刑罚必也。''

人主者,擅生杀,处威势,操令行禁止之柄以御其群臣,此主道也。人臣者,处卑贱,奉主令,守本任,治分职,此臣道也。故主行臣道则乱,臣行主道则危。故上下无分,君臣共道,乱之本也。故《明法》曰:``君臣共道则乱。''

人臣之所以畏恐而谨事主者,以欲生而恶死也。使人不欲生,不恶死,则不可得而制也。夫生杀之柄,专在大臣,而主不危者,未尝有也。故治乱不以法断而决于重臣,生杀之柄不制于主而在群下,此寄生之主也。故人主专以其威势予人,则必有劫杀之患;专以其法制予人,则必有乱亡之祸。如此者,亡主之道也。故《明法》曰:``专授则失。''

凡为主而不得行其令,废法而恣群臣,威严已废,权势已夺,令不得出,群臣弗为用,百姓弗为使,竟内之众不制,则国非其国而民非其民。如此者,灭主之道也。故《明法》曰:``令本不出谓之灭。''

明主之道,卑贱不待尊贵而见,大臣不因左右而进,百官条通,群臣显见,有罚者主见其罪,有赏者主知其功。见知不悖,赏罚不差。有不蔽之术,故无壅遏之患。乱主则不然,法令不得至于民,疏远隔闭而不得闻。如此者,壅遏之道也。故《明法》曰:``令出而留谓之壅。''

人臣之所以乘而为奸者,擅主也。臣有擅主者,则主令不得行,而下情不上通。人臣之力,能鬲君臣之间,而使美恶之情不扬闻,祸福之事不通彻,人主迷惑而无从悟,如此者,塞主之道也。故《明法》曰:``下情不上通谓之塞。''

明主者,兼听独断,多其门户。群臣之道,下得明上,贱得言贵,故奸人不敢欺。乱主则不然,听无术数,断事不以参伍。故无能之士上通,邪枉之臣专国,主明蔽而聪塞,忠臣之欲谋谏者不得进。如此者,侵主之道也。故《明法》曰:``下情上而道止,谓之侵。''

人主之治国也,莫不有法令赏罚。具故其法令明而赏罚之所立者当,则主尊显而奸不生;其法令逆而赏罚之所立者不当,则群臣立私而壅塞之,朋党而劫杀之。故《明法》曰:``灭、塞、侵、壅之所生,从法之不立也。''

法度者,主之所以制天下而禁奸邪也,所以牧领海内而奉宗庙也。私意者,所以生乱长奸而害公正也,所以壅蔽失正而危亡也。故法度行则国治,私意行则国乱。明主虽心之所爱而无功者不赏也,虽心之所憎而无罪者弗罚也。案法式而验得失,非法度不留意焉。故《明法》曰:``先王之治国也,不淫意于法之外。''

明主之治国也,案其当宜,行其正理。故其当赏者,群臣不得辞也;其当罚者,群臣不敢避也。夫赏功诛罪,所以为天下致利除害也。草茅弗去,则害禾谷;盗贼弗诛,则伤良民。夫舍公法而行私惠,则是利奸邪而长暴乱也。行私惠而赏无功,则是使民偷幸而望于上也;行私惠而赦有罪,则是使民轻上而易为非也。夫舍公法用私惠,明主不为也。故《明法》曰:``不为惠于法之内。''

凡人主莫不欲其民之用也。使民用者,必法立而令行也。故治国使众莫如法,禁淫止暴莫如刑。故贫者非不欲夺富者财也,然而不敢者,法不使也;强者非不能暴弱也,然而不敢者,畏法诛也。故百官之事,案之以法,则奸不生;暴慢之人,诛之以刑,则祸不起;群臣并进,策之以数,则私无所立。故《明法》曰:``动无非法者,所以禁过而外私也。''

人主之所以制臣下者,威势也。故威势在下,则主制于臣;威势在上,则臣制于主。夫蔽主者,非塞其门守其户也,然而令不行、禁不止、所欲不得者,失其威势也。故威势独在于主,则群臣畏敬;法政独出于主,则天下服德。故威势分于臣则令不行,法政出于臣则民不听。故明主之治天下也,威势独在于主而不与臣共,法政独制于主而不从臣出。故《明法》曰:``威不两错,政不二门。''

明主者,一度量,立表仪,而坚守之。故令下而民从。法者,天下之程式也,万事之仪表也;吏者,民之所悬命也。故明主之治也,当于法者赏之,违于法者诛之。故以法诛罪,则民就死而不怨;以法量功,则民受赏而无德也。此以法举错之功也。故《明法》曰;``以法治国,则举错而已。''

明主者,有法度之制、故群臣皆出于方正之治而不敢为奸,百姓知主之从事于法也,故吏之所使者,有法则民从之,无法则止,民以法与吏相距,下以法与上从事。故诈伪之人不得欺其主,嫉妒之人不得用其贼心,谗谀之人不得施其巧。千里之外,不敢擅为非。故《明法》曰:``有法度之制者,不可巧以诈伪。''

权衡者,所以起轻重之数也。然而人不事者,非心恶利也,权不能为之多少其数,而衡不能为之轻重其量也。人知事权衡之无益,故不事也。故明主在上位,则官不得枉法,吏不得为私。民知事吏之无益,故财货不行于吏,权衡平正而待物,故奸诈之人不得行其私。故《明法》曰:``有权衡之称者,不可欺以轻重。''

尺寸寻丈者,所以得长短之情也。故以尺寸量短长,则万举而万不失矣。是故尺寸之度,虽富贵众强,不为益长;虽贫贱卑辱,不为损短。公平而无所偏,故奸诈之人不能误也。故《明法》曰:``有寻丈之数者,不可差以长短。''

国之所以乱者,废事情而任非誉也。故明主之听也,言者责之以其实,誉人者试之以其官。言而无实者,诛;吏而乱官者,诛。是故虚言不敢进,不肖者不敢受官。乱主则不然,听言而不督其实,故群臣以虚誉进其党;任官而不责其功,故愚污之吏在庭。如此则群臣相推以美名,相假以功伐,务多其佼而不为主用。故《明法》曰:``主释法以誉进能,则臣离上而下比周矣;以党举官,则民务佼而不求用矣。''

乱主不察臣之功劳,誉众者,则赏之;不审其罪过,毁众者,则罚之。如此者,则邪臣无功而得赏,忠正无罪而有罚。故功多而无赏,则臣不务尽力:行正而有罚,则贤圣无从竭能;行货财而得爵禄,则污辱之人在官;寄托之人不肖而位尊,则民倍公法而趋有势。如此,则悫愿之人失其职,而廉洁之吏失其治。故《明法》曰:``官之失其治也,是主以誉为赏而以毁为罚也。''

平吏之治官也,行法而无私,则奸臣不得其利焉。此奸臣之所务伤也。人主不参验其罪过,以无实之言诛之,则奸臣不能无事贵重而求推誉,以避刑罚而受禄赏焉。故《明法》曰:``喜赏恶罚之人,离公道而行私术矣。''

奸臣之败其主也,积渐积微,使主迷惑而不自知也。上则相为候望于主,下则买誉于民。誉其党而使主尊之,毁不誉者而使主废之。其所利害者,主听而行之,如此,则群臣皆忘主而趋私佼矣。故《明法》曰:``比周以相为慝,是故忘主私佼,以进其誉。''

主无术数,则群臣易欺之;国无明法,则百姓轻为非。是故奸邪之人用国事,则群臣仰利害也。如此,则奸人为之视听者多矣。虽有大义,主无从知之。故《明法》曰:``佼众誉多,外内朋党,虽有大奸,其蔽主多矣。''

凡所谓忠臣者,务明法术,日夜佐主明于度数之理,以治天下者也。奸邪之臣知法术明之必治也,治则奸臣困而法术之士显。是故邪之所务事者,使法无明,主无悟,而己得所欲也。故方正之臣得用则奸邪之臣困伤矣,是方正之与奸邪不两进之势也。奸邪在主之侧者,不能勿恶也。唯恶之,则必候主间而日夜危之。人主不察而用其言,则忠臣无罪而困死,奸臣无功而富贵。故《明法》曰:``忠臣死于非罪,而邪臣起于非功。''

富贵尊显,久有天下,人主莫不欲也。令行禁止,海内无敌,人主莫不欲也。蔽欺侵凌,人主莫不恶也。失天下,灭宗庙,人主莫不恶也。忠臣之欲明法术以致主之所欲而除主之所恶者,奸臣之擅主者,有以私危之,则忠臣无从进其公正之数矣。故《明法》曰:``所死者非罪,所起者非功,然则为人臣者重私而轻公矣。''

乱主之行爵禄也,不以法令案功劳;其行刑罚也,不以法令案罪过。而听重臣之所言。故臣有所欲赏,主为赏之;臣欲有所罚,主为罚之。废其公法,专听重臣。如此,故群臣皆务其党,重臣而忘其主,趋重臣之门而不庭。故《明法》曰:``十至于私人之门,不一至于庭。''

明主之治也,明于分职,而督其成事。胜其任者处官,不胜其任者废免。故群臣皆竭能尽力以治其事。乱主则不然。故群臣处官位,受厚禄,莫务治国者,期于管国之重而擅其利,牧渔其民以富其家。故《明法》曰:``百虑其家,不一图其国。''

明主在上位,则竟内之众尽力以奉其主,百官分职致治以安国家。乱主则不然,虽有勇力之士,大臣私之,而非以奉其主也;虽有圣智之士,大臣私之,非以治其国也。故属数虽众,不得进也;百官虽具,不得制也。如此者,有人主之名而无其实。故《明法》曰:``属数虽众,非以尊君也;百官虽具,非以任国也。此之谓国无人。''

明主者,使下尽力而守法分,故群臣务尊主而不敢顾其家;臣主之分明,上下之位审,故大臣各处其位而不敢相贵。乱主则不然,法制废而不行,故群臣得务益其家;君臣无分,上下无别,故群臣得务相贵。如此者,非朝臣少也,众不为用也。故《明法》曰:``国无人者,非朝臣衰也,家与家务相益,不务尊君也;大臣务相贵,而不任国也。''

人主之张官置吏也,非徒尊其身厚奉之而已也,使之奉主之法,行主之令,以治百姓而诛盗贼也。是故其所任官者大,则爵尊而禄厚;其所任官者小,则爵卑而禄薄。爵禄者,人主之所以使吏治官也。乱主之治也,处尊位,受厚禄,养所与佼,而不以官为务。如此者,则官失其能矣。故《明法》曰:``小臣持禄养佼,不以官为事,故官失职。''

明主之择贤人也,言勇者试之以军,言智者试之以官。试于军而有功者则举之,试于官而事治者则用之。故以战功之事定勇怯,以官职之治定愚智;故勇怯愚智之见也,如白黑之分。乱主则不然,听言而不试,故妄言者得用;任人而不官,故不肖者不困。故明主以法案其言而求其实,以官任其身而课其功,专任法不自举焉。故《明法》曰:``先王之治国也,使法择人,不自举也。''

凡所谓功者,安主上,利万民者也。夫破军杀将,战胜攻取,使主无危亡之忧,而百姓无死虏之患,此军士之所以为功者也。奉主法,治竟内,使强不凌弱,众不暴寡,万民欢尽其力而奉养其主,此吏之所以为功也。匡主之过,救主之失,明理义以道其主,主无邪僻之行,蔽欺之患,此臣之所以为功也。故明主之治也,明分职而课功劳,有功者赏,乱治者诛,诛赏之所加,各得其宜,而主不自与焉。故《明法》曰:``使法量功,不自度也。''

明主之治也,审是非,察事情,以度量案之。合于法则行,不合于法则止。功充其言则赏,不充其言则诛。故言智能者,必有见功而后举之;言恶败者,必有见过而后废之。如此则士上通而莫之能妒,不肖者困废而莫之能举。故《明法》曰:``能不可蔽而败不可饰也。''

明主之道,立民所欲而求其功,故为爵禄以劝之;立民所恶以禁其邪,故为刑罚以畏之。故案其功而行赏,案其罪而行罚,如此则群臣之举无功者,不敢进也;毁无罪者,不能退也。故《明法》曰:``誉者不能进而诽者不能退也。''

制群臣,擅生杀,主之分也;县令仰制,臣之分也。威势尊显,主之分也;卑贱畏敬,臣之分也。令行禁止,主之分也;奉法听从,臣之分也。故君臣相与,高下之处也,如天之与地也;其分画之不同也,如白之与黑也。故君臣之间明别,则主尊臣卑。如此,则下之从上也,如响之应声;臣之法主也,如景之随形。故上令而下应,主行而臣从,以令则行,以禁则止,以求则得。此之谓易治。故《明法》曰:``君臣之间明别,则易治。''

明主操术任臣下,使群臣效其智能,进其长技。故智者效其计,能者进其功。以前言督后事,所效当则赏之,不当则诛之,张官任吏治民,案法试课成功。守法而法之,身无烦劳而分职。故《明法》曰:``主虽不身下为而守法为之可也。''

\hypertarget{header-n949}{%
\subsection{巨乘马}\label{header-n949}}

桓公问管子曰:``请问乘马。''管子对曰:``国无储在令。''桓公曰:``何谓国无储在令?''管子对曰:``一农之量壤百亩也,春事二十五日之内。''桓公曰:``何谓春事二十五日之内?''管子对曰:``日至六十日而阳冻释,七十〔五〕日而阴冻释。阴冻释而秇稷,百日不秇稷,故春事二十五日之内耳也。今君立扶台、五衢之众皆作。君过春而不止,民失其二十五日,则五衢之内阻弃之地也。起一人之繇,百亩不举;起十人之繇,千亩不举;起百人之繇,万亩不举;起千人之繇,十万亩不举。春已失二十五日,而尚有起夏作,是春失其地,夏失其苗,秋起繇而无止,此之谓谷地数亡。谷失于时,君之衡藉而无止,民食什伍之谷,则君已籍九矣,有衡求币焉,此盗暴之所以起,刑罚之所以众也。随之以暴,谓之内战。''桓公曰:``善哉!''

管子曰:``策乘马之数求尽也,彼王者不夺民时,故五谷兴丰。五谷兴丰,则士轻禄,民简赏。彼善为国者,使农夫寒耕暑耘,力归于上,女勤于纤微而织归于府者,非怨民心伤民意,高下之策,不得不然之理也。''

桓公曰:``为之奈何?''管子曰:``虞国得策乘马之数矣。''桓公曰:``何谓策乘马之数?''管子曰:``百亩之夫,予之策:`率二十七日为子之春事,资子之币。'春秋,子谷大登,国谷之重去分。谓农夫曰:`币之在子者以为谷而廪之州里。'国谷之分在上,国谷之重再十倍。谓远近之县,里、邑百官,皆当奉器械备,曰:`国无币,以谷准币。'国谷之櫎,一切什九。还谷而应谷,国器皆资,无籍于民。此有虞之策乘马也。''

\hypertarget{header-n955}{%
\subsection{乘马数}\label{header-n955}}

桓公问管子曰:``有虞策乘马已行矣,吾欲立策乘马,为之奈何?''管子对曰:``战国修其城池之功,故其国常失其地用。王国则以时行也。''桓公曰:``何谓以时行?''管子对曰:``出准之令,守地用人策,故开阖皆在上,无求于民。''

``霸国守分,上分下游于分之间而用足。王国守始,国用一不足则加一焉,国用二不足则加二焉,国用三不足则加三焉,国用四不足则加四焉,国用五不足则加五焉,国用六不足则加六焉,国用七不足则加七焉,国用八不足则加八焉,国用九不足则加九焉,国用十不足则加十焉。人君之守高下,岁藏三分,十年则必有五年之余。若岁凶旱水泆,民失本,则修宫室台榭,以前无狗后无彘者为庸。故修宫室台榭,非丽其乐也,以平国策也。今至于其亡策乘马之君,春秋冬夏,不知时终始,作功起众,立宫室台榭。民失其本事,君不知其失诸春策,又失诸夏秋之策数也。民无卖子数矣。猛毅之人淫暴,贫病之民乞请,君行律度焉,则民被刑僇而不从于主上。此策乘马之数亡也。''

``乘马之准,与天下齐准。彼物轻则见泄,重则见射。此斗国相泄,轻重之家相夺也。至于王国,则持流而止矣。''桓公曰:``何谓持流?''管子对曰:``有一人耕而五人食者,有一人耕而四人食者,有一人耕而三人食者,有一人耕而二人食者。此齐力而功地。田策相圆,此国策之时守也。君不守以策,则民且守于下,此国策流已。''

桓公曰:``乘马之数尽于此平?''管子对曰:``布织财物,皆立其赀。财物之货与币高下,谷独贵独贱。''桓公曰:``何谓独贵独贱?''管子对曰:``谷重而万物轻,谷轻而万物重。''

公曰:``贱策乘马之数奈何?''管子对曰:``郡县上臾之壤守之若干,间壤守之若干,下壤守之若干。故相壤定籍而民不移,振贫补不足,下乐上。故以上壤之满补下壤之众,章四时,守诸开阖,民之不移也,如废方于地。此之谓策乘马之数也。''

\hypertarget{header-n961}{%
\subsection{事语}\label{header-n961}}

桓公问管子曰:``事之至数可闻乎?''管子对曰:``何谓至数?''桓公曰:``秦奢教我曰:`帷盖不修,衣服不众,则女事不泰。俎豆之礼不致牲,诸侯太牢,大夫少牢,不若此,则六畜不育。非高其台榭,美其宫室,则群材不散。'此言何如?''管子曰:``非数也。''桓公曰:``何谓非数?''管子对曰:``此定壤之数也。彼天子之制,壤方千里,齐诸侯方百里,负海子七十里,男五十里,若胸臂之相使也。故准徐疾、赢不足,虽在下也,不为君忧。彼壤狭而欲举与大国争者,农夫寒耕暑耘,力归于上,女勤于缉绩徽织,功归于府者,非怨民心伤民意也,非有积蓄不可以用人,非有积财无以劝下。泰奢之数,不可用于危隘之国。''桓公曰:``善。''

桓公又问管子曰:``佚田谓寡人曰:`善者用非其有,使非其人,何不因诸侯权以制夭下?'''管子对曰:``佚田之言非也,彼善为国者,壤辟举则民留处,仓廪实则知礼节。且无委致围,城脆致冲。夫不定内,不可以持天下。佚田之言非也。''管子曰:``岁藏一,十年而十也。岁藏二,五年而十也。谷十而守五,绨素满之,五在上。故视岁而藏,县时积岁,国有十年之蓄,富胜贫,勇胜怯,智胜愚,微胜不微,有义胜无义,练士胜驱众。凡十胜者尽有之,故发如风雨,动如雷霆,独出独入,莫之能禁止,不待权舆。故佚田之言非也。''桓公曰:``善。''

\hypertarget{header-n966}{%
\subsection{海王 }\label{header-n966}}

桓公问于管子曰:``吾欲藉于台雉何如?''管子对曰:``此毁成也。''``吾欲藉于树木?''管子对曰:``此伐生也。''``吾欲藉于六畜?''管子对曰:``此杀生也。''``吾欲藉于人,何如?''管子对曰:``此隐情也。''桓公曰:``然则吾何以为国?''管子对曰:``唯官山海为可耳。''

桓公曰:``何谓官山海?''管子对曰:``海王之国,谨正盐策。''桓公曰:``何谓正盐策?''管子对曰:``十口之家十人食盐,百口之家百人食盐。终月,大男食盐五升少半,大女食盐三升少半,吾子食盐二升少半,此其大历也。盐百升而釜。令盐之重升加分强,釜五十也;升加一强,釜百也;升加二强,釜二百也。钟二千,十钟二万,百钟二十万,千钟二百万。万乘之国,人数开口千万也,禺策之,商日二百万,十日二千万,一月六千万。万乘之国,正九百万也。月人三十钱之籍,为钱三千万。今吾非籍之诸君吾子,而有二国之籍者六千万。使君施令曰:吾将籍于诸君吾子,则必嚣号。今夫给之盐策,则百倍归于上,人无以避此者,数也。''

``今铁官之数曰:一女必有一针一刀,若其事立;耕者必有一耒一耜一铫,若其事立;行服连轺輂者必有一斤一锯一锥一凿,若其事立。不尔而成事者天下无有。令针之重加一也,三十针一人之籍;刀之重加六,五六三十,五刀一人之籍也;耜铁之重加七,三耜铁一人之籍也。其余轻重皆准此而行。然则举臂胜事,无不服籍者。''

桓公曰:``然则国无山海不王乎?''管子曰:``因人之山海假之。名有海之国雠盐于吾国,釜十五,吾受而官出之以百。我未与其本事也,受人之事,以重相推。此人用之数也。''

\hypertarget{header-n973}{%
\subsection{国蓄}\label{header-n973}}

国有十年之蓄,而民不足于食,皆以其技能望君之禄也;君有山海之金,而民不足于用,是皆以其事业交接于君上也。故人君挟其食,守其用,据有余而制不足,故民无不累于上也。五谷食米,民之司命也;黄金刀币,民之通施也。故善者执其通施以御其司命,故民力可得而尽也。

夫民者亲信而死利,海内皆然。民予则喜,夺则怒,民情皆然。先王知其然,故见予之形,不见夺之理。故民爱可洽于上也。租籍者,所以强求也:租税者,所虑而请也。王霸之君去其所以强求,废其所虑而请,故天下乐从也。

利出于一孔者,其国无敌;出二孔者,其兵不诎;出三孔者,不可以举兵;出四孔者,其国必亡。先王知其然,故塞民之养,隘其利途。故予之在君,夺之在君,贫之在君,富之在君。故民之戴上如日月,亲君若父母。

凡将为国,不通于轻重,不可为笼以守民;不能调通民利,不可以语制为大治。是故万乘之国有万金之贾,千乘之国有千金之贾,然者何也?国多失利,则臣不尽其忠,士不尽其死矣。岁有凶穰,故谷有贵贱;令有缓急,故物有轻重。然而人君不能治,故使蓄贾游市,乘民之不给,百倍其本。分地若一,强者能守;分财若一,智者能收。智者有什倍人之功,愚者有不赓本之事。然而人君不能调,故民有相百倍之生也。夫民富则不可以禄使也,贫则不可以罚威也。法令之不行,万民之不治,贫富之不齐也。且君引錣量用,耕田发草,上得其数矣。民人所食,人有若干步亩之数矣,计本量委则足矣。然而民有饥饿不食者何也?谷有所藏也。人君铸钱立币,民庶之通施也,人有若干百千之数矣。然而人事不及、用不足者何也?利有所并藏也。然则人君非能散积聚,钧羡不足,分并财利而调民事也,则君虽强本趣耕,而自为铸币而无已,乃今使民下相役耳,恶能以为治乎?

岁适美,则市粜无予,而狗彘食人食。岁适凶,则市籴釜十繦,而道有饿民。然则岂壤力固不足而食固不赡也哉?夫往岁之粜贱,狗彘食人食,故来岁之民不足也。物适贱,则半力而无予,民事不偿其本;物适贵,则什倍而不可得,民失其用。然则岂财物固寡而本委不足也哉?夫民利之时失,而物利之不平也。故善者委施于民之所不足,操事于民之所有余。夫民有余则轻之,故人君敛之以轻;民不足则重之,故人君散之以重。敛积之以轻,散行之以重,故君必有十倍之利,而财之櫎可得而平也。

凡轻重之大利,以重射轻,以贱泄平。万物之满虚随财,准平而不变,衡绝则重见。人君知其然,故守之以准平,使万室之都必有万钟之藏,藏繦千万;使千室之都必有千钟之藏,藏繦百万。春以奉耕,夏以奉芸。耒耜械器,种镶粮食,毕取赡于君。故大贾蓄家不得豪夺吾民矣。然则何?君养其本谨也。春赋以敛缯帛,夏贷以收秋实,是故民无废事而国无失利也。

凡五谷者,万物之主也。谷贵则万物必贱,谷贱则万物必贵。两者为敌,则不俱平。故人君御谷物之秩相胜,而操事于其不平之间。故万民无籍而国利归于君也。夫以室庑籍,谓之毁成;以六畜籍,谓之止生;以田亩籍,谓之禁耕;以正人籍,谓之离情;以正户籍,谓之养赢。五者不可毕用,故王者遍行而不尽也。故天子籍于币,诸侯籍于食。中岁之谷,粜石十钱。大男食四石,月有四十之籍;大女食三石,月有三十之籍:吾子食二石,月有二十之籍。岁凶谷贵,籴石二十钱,则大男有八十之籍,大女有六十之籍,吾子有四十之籍。是人君非发号令收啬而户籍也,彼人君守其本委谨,而男女诸君吾子无不服籍者也。一人廪食,十人得余;十人廪食,百人得余;百人廪食,千人得余。夫物多则贱,寡则贵,散则轻,聚则重。人君知其然,故视国之羡不足而御其财物。谷贱则以币予食,布帛贱则以币予衣。视物之轻重而御之以准,故贵贱可调而君得其利。

前有万乘之国,而后有千乘之国,谓之抵国。前有千乘之国,而后有万乘之国,谓之距国。壤正方,四面受敌,谓之衢国。以百乘衢处,谓之托食之君。千乘衢处,壤削少半。万乘衢处,壤削太半。何谓百乘衢处托食之君也?夫以百乘衢处,危慑围阻千乘万乘之间,夫国之君不相中,举兵而相攻,必以为捍挌蔽圉之用。有功利不得乡。大臣死于外,分壤而功;列陈系累获虏,分赏而禄。是壤地尽于功赏,而税臧殚于继孤也。是特名罗于为君耳,无壤之有;号有百乘之守,而实无尺壤之用,故谓托食之君。然则大国内款,小国用尽,何以及此?曰:百乘之国,官赋轨符,乘四时之朝夕,御之以轻重之准,然后百乘可及也。千乘之国,封天财之所殖,诫器之所出,财物之所生,视岁之满虚而轻重其禄,然后千乘可足也。万乘之国,守岁之满虚,乘民之缓急,正其号令而御其大准,然后万乘可资也。

玉起于禺氏,金起于汝汉,珠起于赤野,东西南北距周七千八百里。水绝壤断,舟车不能通。先王为其途之远,其至之难,故托用于其重,以珠玉为上币,以黄金为中币,以刀布为下币。三币握之则非有补于暖也,食之则非有补于饱也,先王以守财物,以御民事,而平天下也。今人君籍求于民,令曰十日而具,则财物之贾什去一;令曰八日而具,则财物之贾什去二;令曰五日而具,则财物之贾什去半;朝令而夕具,则财物之贾什去九。先王知其然,故不求于万民而籍于号令也。

\hypertarget{header-n985}{%
\subsection{山国轨}\label{header-n985}}

桓公问管子曰:``请问官国轨。''管子对曰:``田有轨,人有轨,用有轨,乡有轨,人事有轨,币有轨,县有轨,国有轨。不通于轨数而欲为国,不可。''

桓公曰,``行轨数奈何?''对曰,``某乡田若干?人事之准若干?谷重若干?曰:某县之人若干?田若干?币若干而中用?谷重若干而中币?终岁度人食,其余若干?曰:某乡女胜事者终岁绩,其功业若干?以功业直时而櫎之,终岁,人已衣被之后,余衣若干?别群轨,相壤宜。''

桓公曰:``何谓别群轨,相壤宜?''管子对曰:``有莞蒲之壤,有竹箭檀柘之壤,有汜下渐泽之壤,有水潦鱼鳖之壤。今四壤之数,君皆善官而守之,则籍于财物,不籍于人。亩十鼓之壤,君不以轨守,则民且守之。民有过移长力,不以本为得,此君失也。''

桓公曰:``轨意安出?''管子对曰:``不阴据其轨,皆下制其上。''桓公曰:``此若言何谓也?''管子对曰:``某乡田若干?食者若干?某乡之女事若干?余衣若干?谨行州里,曰:`田若干,人若干,人众田不度食若干。'曰:`田若干,余食若干。'必得轨程,此谓之泰轨也。然后调立环乘之币。田轨之有余于其人食者,谨置公币焉。大家众,小家寡。山田、间田,曰终岁其食不足于其人若干,则置公币焉,以满其准。重岁,丰年,五谷登,谓高田之萌曰:`吾所寄币于子者若干,乡谷之櫎若干,请为子什减三。'谷为上,币为下。高田抚间田山不被,谷十倍。山田以君寄币,振其不赡,未淫失也。高田以时抚于主上,坐长加十也。女贡织帛,苟合于国奉者,皆置而券之。以乡櫎市准曰:`上无币,有谷。以谷准币。'环谷而应策,国奉决。谷反准,赋轨币,谷廪重有加十。谓大家委赀家曰:`上且修游,人出若干币。'谓邻县曰:`有实者皆勿左右。不赡,则且为人马假其食民。'邻县四面皆櫎,谷坐长而十倍。上下令曰:`赀家假币,皆以谷准币,直币而庚之。'谷为下,币为上。百都百县轨据,谷坐长十倍。环谷而应假币。国币之九在上,一在下,币重而万物轻。敛万物,应之以币。币在下,万物皆在上,万物重十倍。府官以市櫎出万物,隆而止。国轨,布于未形,据其已成,乘令而进退,无求于民。谓之国轨。''

桓公间于管子曰:``不籍而赡国,为之有道乎?''管子对曰:``轨守其时,有官天财,何求于民。''桓公曰:``何谓官天财?''管子对曰:``泰春民之功繇;泰夏民之令之所止,令之所发;泰秋民令之所止,令之所发;泰冬民令之所止,令之所发。此皆民所以时守也,此物之高下之时也,此民之所以相并兼之时也。君守诸四务。''

桓公曰:``何谓四务?''管子对曰:``泰春,民之且所用者,君已廪之矣;泰夏,民之且所用者,君已廪之矣;泰秋,民之且所用者,君已廪之矣;泰冬,民之且所用者,君已廪之矣。泰春功布日,春缣衣、夏单衣、捍、宠、累箕、胜、籯、屑、,若干日之功,用人若干,无赀之家皆假之械器,幐、籯、筲、、公衣,功已而归公衣,折券。故力出于民,而用出于上。春十日不害耕事,夏十日不害芸事,秋十日不害敛实,冬二十日不害除田。此之谓时作。''

桓公曰:``善。吾欲立轨官,为之奈何?''管子对曰:``盐铁之策,足以立轨官。''桓公曰:``奈何?''管子对曰:``龙夏之地,布黄金九千,以币赀金,巨家以金,小家以币。周岐山至于峥丘之西塞丘者,山邑之田也,布币称贫富而调之。周寿陵而东至少沙者,中田也,据之以币、巨家以金、小家以币。三壤已抚,而国谷再什倍。粱渭、阳琐之牛马满齐衍,请驱之颠齿,量其高壮,曰:`国为师旅,战车驱就敛子之牛马,上无币,请以谷视市櫎而庚子。'牛马在上,粟二家。二家散其粟,反准。牛马归于上。''

管子曰:``请立赀于民,有田倍之。内毋有,其外外皆为赀壤。被鞍之马千乘,齐之战车之具,具于此,无求于民。此去丘邑之籍也。''

``国谷之朝夕在上,山林廪械器之高下在上,春秋冬夏之轻重在上。行田畴,田中有木者,谓之谷贼。宫中四荣,树其余曰害女功。宫室械器非山无所仰。然后君立三等之租于山,曰:握以下者为柴楂,把以上者为室奉,三围以上为棺椁之奉;柴楂之租若干,室奉之租若干,棺椁之租若干。''

管子曰:``盐铁抚轨,谷一廪十,君常操九,民衣食而繇,下安无怨咎。去其田赋,以租其山:巨家重葬其亲者服重租,小家菲葬其亲者服小租;巨家美修其宫室者服重租,小家为室庐者服小租。上立轨于国,民之贫富如加之以绳,谓之国轨。''

\hypertarget{header-n998}{%
\subsection{山权数}\label{header-n998}}

桓公问管子曰:``请问权数。''管子对曰:``天以时为权,地以财为权,人以力为权,君以令为权。失天之权,则人地之权亡。''桓公曰:``何为失天之权则人地之权亡?''管子对曰:``汤七年旱,禹五年水,民之无(米亶)卖子者。汤以庄山之金铸币,而赎民之无(米亶)卖子者;禹以历山之金铸币,而赎民之无卖子者。故天权失,人地之权皆失也。故王者岁守十分之参,三年与少半成岁,三十一年而藏十一年与少半。藏三之一不足以伤民,而农夫敬事力作。故天毁埊,凶旱水泆,民无入于沟壑乞请者也。此守时以待天权之道也。''桓公曰:``善。吾欲行三权之数,为之奈何?''管子对曰:``梁山之阳綪、夜石之币,天下无有。''管子曰:``以守国谷,岁守一分,以行五年,国谷之重什倍异日。''管子曰:``请立币,国铜以二年之粟顾之,立黔落。力重与天下调。彼重则见射,轻则见泄,故与天下调。泄者,失权也;见射者,失策也。不备天权,下相求备,准下阴相隶。此刑罚之所起而乱之之本也。故平则不平,民富则不如贫,委积则虚矣。此三权之失也已。''桓公曰:``守三权之数奈何?''管子对曰:``大丰则藏分,阨亦藏分。''桓公曰:``阨者,所以益也。何以藏分?''管子对曰:``隘则易益也,一可以为十,十可以为百。以阨守丰,阨之准数一上十,丰之策数十去九,则吾九为余。于数策丰,则三权皆在君,此之谓国权。''

桓公问于管子曰:``请问国制。''管子对曰:``国无制,地有量。''桓公曰,``何谓国无制,地有量?''管子对曰:``高田十石,间田五石,庸田三石,其余皆属诸荒田。地量百亩,一夫之力也。粟贾一,粟贾十,粟贾三十,粟贾百。其在流策者,百亩从中千亩之策也。然则百乘从千乘也,千乘从万乘也。故地有量,国无策。''桓公曰:``善。今欲为大国,大国欲为天下,不通权策,其无能者矣。''

桓公曰:``今行权奈何?''管子对曰:``君通于广狭之数,不以狭畏广;通于轻重之数,不以少畏多。此国策之大者也。''桓公曰:``善。盖天下,视海内,长誉而无止,为之有道乎?''管子对曰:``有。曰:轨守其数,准平其流,动于未形,而守事已成。物一也而十,是九为用。徐疾之数,轻重之策也,一可以为十,十可以为百。引十之半而藏四,以五操事,在君之决塞。''桓公曰:``何谓决塞?''管子曰:``君不高仁,则国不相被;君不高慈孝,则民简其亲而轻过。此乱之至也。则君请以国策十分之一者树表置高,乡之孝子聘之币,孝子兄弟众寡不与师旅之事。树表置高而高仁慈孝,财散而轻。乘轻而守之以策,则十之五有在上。运五如行事,如日月之终复。此长有天下之道,谓之准道。''

桓公问于管子曰:``请问教数。''管子对曰:``民之能明于农事者,置之黄金一斤,直食八石。民之能蕃育六畜者,置之黄金一斤,直食八石。民之能树艺者,置之黄金一斤,直食八石。民之能树瓜瓠荤菜百果使蕃衮者,置之黄金一斤,直食八石。民之能已民疾病者,置之黄金一斤,直食八石。民之知时:曰`岁旦阨',曰`某谷不登'曰`某谷丰'者,置之黄金一斤,直食八石。民之通于蚕桑,使蚕不疾病者,皆置之黄金一斤,直食八石。谨听其言而藏之官,使师旅之事无所与,此国策之者也。国用相靡而足,相困揲而(上次下吉),然后置四限高下,令之徐疾,驱屏万物,守之以策,有五官技。''桓公曰:``何谓五官技?''管子曰:``诗者所以记物也,时者所以记岁也,春秋者所以记成败也,行者道民之利害也,易者所以守凶吉成败也,卜者卜凶吉利害也。民之能此者皆一马之田,一金之衣。此使君不迷妄之数也。六家者,即见:其时,使豫先蚤闲之日受之,故君无失时,无失策,万物兴丰;无失利,远占得失,以为末教;诗,记人无失辞;行,殚道无失义;易,守祸福凶吉不相乱。此谓君棅。''

桓公问于管子曰:``权棅之数吾已得闻之矣,守国之固奈何?''曰:``能皆已官,时皆已官,得失之数,万物之终始,君皆已官之矣。其余皆以数行。''桓公曰:``何谓以数行?''管子对曰:``谷者民之司命也,智者民之辅也。民智而君愚,下富而君贫,下贫而君富,此之谓事名二。国机,徐疾而已矣。君道,度法而已矣。人心,禁缪而已矣。''桓公曰:``何谓度法?何谓禁缪?''管子对曰:``度法者,量人力而举功;禁缪者,非往而戒来。故祸不萌通而民无患咎。''桓公曰:``请闻心禁。''管子对曰:``晋有臣不忠于其君,虑杀其主,谓之公过。诸公过之家毋使得事君。此晋之过失也。齐之公过,坐立长差。恶恶乎来刑,善善乎来荣,戒也。此之谓国戒。''

桓公问管子曰:``轻重准施之矣,策尽于此乎?''管子曰:``未也,将御神用宝。''桓公曰:``何谓御神用宝?''管子对曰:``北郭有掘阙而得龟者,此检数百里之地也。''桓公曰:``何谓得龟百里之地?''管子对曰:``北郭之得龟者,令过之平盘之中。君请起十乘之使,百金之提,命北郭得龟之家曰:`赐若服中大夫。'曰:`东海之子类于龟,托舍于若。赐若大夫之服以终而身,劳若以百金。'之龟为无赀,而藏诸泰台,一日而衅之以四牛,立宝曰无赀。还四年,伐孤竹。丁氏之家粟可食三军之师行五月,召丁氏而命之曰;`吾有无赀之宝于此。吾今将有大事,请以宝为质于子,以假子之邑粟。'丁氏北乡再拜,入粟,不敢受宝质。桓公命丁氏曰:`寡人老矣,为子者不知此数。终受吾质!'丁氏归,革筑室,赋籍藏龟。还四年,伐孤竹,谓丁氏之粟中食三军五月之食。桓公立贡数:文行中七,年龟中四千金,黑白之子当千金。凡贡制,中二齐之壤策也,用贡:国危出宝,国安行流。''桓公曰:``何谓流?''管子对曰:``物有豫,则君失策而民失生矣。故善为天下者,操于二豫之外。''桓公曰:``何谓二豫之外?''管子对曰:``万乘之国,不可以无万金之蓄饰;千乘之国,不可以无千金之蓄饰;百乘之国,不可以无百金之蓄饰。以此与令进退,此之谓乘时。''

\hypertarget{header-n1007}{%
\subsection{山至数}\label{header-n1007}}

桓公问管子曰:``梁聚谓寡人曰:`古者轻赋税而肥籍敛,取下无顺于此者矣。'梁聚之言如何?''管子对曰:``梁聚之言非也。彼轻赋税则仓廪虚,肥籍敛则械器不奉。械器不奉,而诸侯之皮币不衣;仓廪虚则倳贱无禄。外,皮币不衣于天下;内,国倳贱。梁聚之言非也。君有山,山有金,以立币,以币准谷而授禄,故国谷斯在上,谷贾什倍。农夫夜寝蚤起,不待见使,五谷什倍。士半禄而死君,农夫夜寝蚤起,力作而无止;彼善为国者,不曰使之,使不得不使;不曰贫之,使不得不用。故使民无有不得不使者。夫梁聚之言非也。''桓公曰:``善。''

桓公又问于管子曰,``有人教我,谓之请士。曰:`何不官百能?'''管子对曰:``何谓百能?''桓公曰:``使智者尽其智,谋士尽其谋,百工尽其巧。若此则可以为国乎?''管子对曰:``请士之言非也。禄肥则士不死,币轻则士简赏,万物轻则士偷幸。三怠在国,何数之有?彼谷十藏于上,三游于下,谋士尽其虑,智士尽其知,勇士轻其死。请士所谓妄言也。不通于轻重,谓之妄言。''

桓公问于管子曰:``昔者周人有天下,诸侯宾服,名教通于天下,而夺于其下。何数也?''管子对曰:``君分壤而贡入,市朝同流。黄金,一策也;江阳之珠,一策也;秦之明山之曾青,一策也。此谓以寡为多,以狭为广,轨出之属也。''桓公曰:``天下之数尽于轨出之属也?'':``今国谷重什倍而万物轻,大夫谓贾之:`子为吾运谷而敛财。'谷之重一也,今九为余。谷重而万物轻,若此,则国财九在大夫矣。国岁反一,财物之九者皆倍重而出矣。财物在下,币之九在大夫。然则币谷羡在大夫也。天子以客行,令以时出。熟谷之人亡,诸侯受而官之。连朋而聚与,高下万物以合民用。内则大夫自还而不尽忠,外则诸侯连朋合与,熟谷之人则去亡,故天子失其权也。''桓公曰:``善。''

桓公又问管子曰:``终身有天下而勿失,为之有道乎?''管子对曰:``请勿施于天下,独施之于吾国。''桓公曰:``此若言何谓也?''管子对曰:``国之广狭、壤之肥墝有数,终岁食余有数。彼守国者,守谷而已矣。曰:某县之壤广若干,某县之壤狭若干,则必积委币,于是县州里受公钱。泰秋,国谷去参之一,君下令谓郡、县、属大夫里邑皆籍粟入若干。谷重一也,以藏于上者,国谷三分则二分在上矣。泰春,国谷倍重,数也。泰夏,赋谷以市櫎,民皆受上谷以治田土。泰秋,田:`谷之存予者若干,今上敛谷以币。'民曰:`无币以谷。'则民之三有归于上矣。重之相因,时之化举,无不为国策。君用大夫之委,以流归于上。君用民,以时归于君。藏轻,出轻以重,数也。则彼安有自还之大夫独委之?彼诸侯之谷十,使吾国谷二十,则诸侯谷归吾国矣;诸侯谷二十,吾国谷十,则吾国谷归于诸侯矣。故善为天下者,谨守重流,而天下不吾泄矣。彼重之相归,如水之就下。吾国岁非凶也,以币藏之,故国谷倍重,故诸侯之谷至也。是藏一分以致诸侯之一分。利不夺于天下,大夫不得以富侈。以重藏轻,国常有十国之策也。故诸侯服而无正,臣櫎从而以忠,此以轻重御天下之道也,谓之数应。''

桓公问管子曰:``请问国会。''管子对曰:``君失大夫为无伍,失民为失下。故守大夫以县之策,守一县以一乡之策,守一乡以一家之策,守家以一人之策。''桓公曰:``其会数奈何?''管子对曰:``币准之数,一县必有一县中田之策,一乡必有一乡中田之策,一家必有一家直人之用。故不以时守郡为无与,不以时守乡为无伍。''桓公曰:``行此奈何?''管子对曰:''王者藏于民,霸者藏于大夫,残国亡家藏于箧。''桓公曰:``何谓藏于民?'':``请散,栈台之钱,散诸城阳;鹿台之布,散诸济阴。君下令于百姓曰:`民富君无与贫,民贫君无与富。故赋无钱布,府无藏财,赀藏于民。'岁丰,五谷登,五谷大轻,谷贾去上岁之分,以币据之,谷为君,币为下。国币尽在下,币轻,谷重上分。上岁之二分在下,下岁之二分在上,则二岁者四分在上,则国谷之一分在下,谷三倍重。邦布之籍,终岁十钱。人家受食,十亩加十,是一家十户也。出于国谷策而藏于币者也。以国币之分复布百姓,四减国谷,三在上,一在下。复策也。大夫聚壤而封,积实而骄上,请夺之以会。''桓公曰:``何谓夺之以会?''管子对曰:``粟之三分在上,谓民萌皆受上粟,度君藏焉。五谷相靡而重去什三,为余以国币谷准反行,大夫无什于重。君以币赋禄,什在上。君出谷,什而去七。君敛三,上赋七,散振不资者,仁义也。五谷相靡而轻,数也;以乡完重而籍国,数也;出实财,散仁义,万物轻,数也。乘时进退。故曰:王者乘时,圣人乘易。''桓公曰:``善。''

桓公问管子曰:``特命我曰:`天子三百领,泰啬。而散大夫准此而行。'此如何?''管子曰:``非法家也。大夫高其垄,美其室,此夺农事及市庸,此非便国之道也。民不得以织为绡而貍之于地。彼善为国者乘时徐疾而已矣。谓之国会。''

桓公问管子曰:``请问争夺之事何如?''管子曰:``以戚始。''桓公曰:``何谓用戚始?''管子对曰:``君人之主,弟兄十人,分国为十;兄弟五人,分国为五。三世则昭穆同祖,十世则为祏。故伏尸满衍,兵决而无止。轻重之家复游于其间。故曰:毋予人以壤,毋授人以财。财终则有始,与四时废起。圣人理之以徐疾,守之以决塞,夺之以轻重,行之以仁义,故与天壤同数,此王者之大辔也。''

桓公问管子曰:``请问币乘马。''管子对曰:``始取夫三大夫之家,方六里而一乘,二十六人而奉一乘。币乘马者,方六里,田之美恶若干,谷之多寡若干,谷之贵贱若干,凡方六里用币若干,谷之重用币若干。故币乘马者,布币于国,币为一国陆地之数。谓之币乘马。''桓公曰:``行币乘马之数奈何?''管子对曰:``士受资以币,大夫受邑以币,人马受食以币,则一国之谷资在上:币赀在下。国谷什倍,数也。万物财物去什二,策也。皮革、筋角、羽毛、竹箭、器械、财物,苟合于国器君用者,皆有矩券于上。君实乡州藏焉,曰:`某月某日,苟从责者,乡决州决'。故曰:就庸一日而决。国策出于谷轨,国之策货,币乘马者也。今刀布藏于官府,巧币、万物轻重皆在贾人,彼币重而万物轻,币轻而万物重,彼谷重而。人君操谷、币金衡,而天下可定也。此守天下之数也。''

桓公问于管子曰:``准衡、轻重、国会,吾得闻之矣。请问县数。''管子对曰:``狼牡以至于冯会之日,龙夏以北至于海庄,禽兽羊牛之地也,何不以此通国策哉?''桓公曰:``何谓通国策?''管子对曰:``冯市门一吏书赘直事。若其事唐【谀〕圉牧食之人养视不失捍殂者,去其都秩,与其县秩。大夫不乡赘合游者,谓之无礼义,大夫幽其春秋,列民幽其门、山之祠。冯会、龙夏牛羊牺牲月价十倍异日。此出诸礼义,籍于无用之地,因扪牢策也。谓之通。''

桓公问管子曰:``请问国势。''管子对曰:``有山处之国,有氾下多水之国,有山地分之国,有水泆之国,有漏壤之国。此国之五势,人君之所忧也。山处之国常藏谷三分之一,氾下多水之国常操国谷三分之一,山地分之国常操国谷十分之三,水泉之所伤,水泆之国常操十分之二,漏壤之国谨下诸侯之五谷,与工雕文梓器以下天下之五谷。此准时五势之数也。''

桓公问管子曰:``今有海内,县诸侯,则国势不用已乎?''管子对曰:``今以诸侯为公州之饰焉,以乘四时,行扪牢之策。以东西南北相彼,用平而准。故曰:为诸俟,则高下万物以应诸侯;遍有天下,则赋币以守万物之朝夕,调而已。利有足则行,不满则有止。王者乡州以时察之,故利不相倾,县死其所。君守大奉一,谓之国簿。''

\hypertarget{header-n1021}{%
\subsection{地数}\label{header-n1021}}

桓公曰:``地数可得闻乎?''管子对曰:``地之东西二万八千里,南北二万六千里。其出水者八千里,受水者八千里,出铜之山四百六十七山,出铁之山三千六百九山。此之所以分壤树谷也,戈矛之所发,刀币之所起也。能者有余,拙者不足。封于泰山,禅于梁父,封禅之王七十二家,得失之数,皆在此内。是谓国用。''桓公曰:``何谓得失之数皆在此?''管子对曰:``昔者桀霸有天下而用不足,汤有七十里之薄而用有余。天非独为汤雨菽粟,而地非独为汤出财物也。伊尹善通移、轻重、开阖、决塞,通于高下徐疾之策,坐起之费时也。黄帝问于伯高曰:`吾欲陶天下而以为一家,为之有道乎?'伯高对曰:`请刈其莞而树之,吾谨逃其蚤牙,则天下可陶而为一家。'黄帝曰:`此若言可得闻乎?'伯高对曰:`上有丹砂者下有黄金,上有慈石者下有铜金,上有陵石者下有铅、锡、赤铜,上有赭者下有铁,此山之见荣者也。苟山之见其荣者,君谨封而祭之。距封十里而为一坛,是则使乘者下行,行者趋。若犯令者,罪死不赦。然则与折取之远矣。'修教十年,而葛卢之山发而出水,金从之。蚩尤受而制之,以为剑、铠、矛、戟,是岁相兼者诸侯九。雍狐之山发而出水,金从之。蚩尤受而制之,以为雍狐之戟、芮戈,是岁相兼者诸侯十二。故天下之君顿戟一怒,伏尸满野。此见戈之本也。''

桓公问于管子曰:``请问天财所出?地利所在?''管子对曰:``山上有赭者其下有铁,上有铅者其下有银。一曰:`上有铅者其下有鉒银,上有丹砂者其下有鉒金,上有慈石者其下有铜金。'此山之见荣者也。苟山之见荣者,谨封而为禁。有动封山者,罪死而不赦。有犯令者,左足入、左足断;右足入,右足断。然则其与犯之远矣。此天财地利之所在也。''桓公问于管子曰:``以天财地利立功成名于天下者谁子也?''管子对曰:``文武是也。''桓公曰:``此若言何谓也?''管子对曰:``夫玉起于牛氏边山,金起于汝汉之右洿,珠起于赤野之末光。此皆距周七千八百里,其涂远而至难。故先王各用于其重,珠玉为上币,黄金为中币,刀布为下币。令疾则黄金重,令徐则黄金轻。先王权度其号令之徐疾,高下其中币而制下上之用,则文武是也。''

桓公问于管子曰:``吾欲守国财而毋税于天下,而外因天下,可乎?''管子对曰:``可。夫水激而流渠,令疾而物重。先王理其号令之徐疾,内守国财而外因天下矣。''桓公问于管子曰:``其行事奈何?''管子对曰:``夫昔者武王有巨桥之粟贵籴之数。''桓公曰:``为之奈何?''管子对曰:``武王立重泉之戍,令曰:`民自有百鼓之粟者不行。'民举所最粟以避重泉之戍,而国谷二什倍,巨桥之粟亦二什倍。武王以巨桥之粟二什倍而市缯帛,军五岁毋籍衣于民。以巨桥之粟二什倍而衡黄金百万,终身无籍于民。准衡之数也。''桓公问于管子曰:``今亦可以行此乎?''管子对曰:``可。夫楚有汝汉之金,齐有渠展之盐,燕有辽东之煮。此三者亦可以当武王之数。十口之家,十人咶盐,百口之家,百人咶盐。凡食盐之数,一月丈夫五升少半,妇人三升少半,婴儿二升少半。盐之重,升加分耗而釜五十,升加一耗而釜百,升加十耗而釜千。君伐菹薪煮泲水为盐,正而积之三万钟,至阳春请籍于时。''桓公曰:``何谓籍于时?''管子曰:``阳春农事方作,令民毋得筑垣墙,毋得缮冢墓;丈夫毋得治宫室,毋得立台榭;北海之众毋得聚庸而煮盐。然盐之贾必四什倍。君以四什之贾,修河、济之流,南输梁、赵、宋、卫、濮阳。恶食无盐则肿,守圉之本,其用盐独重。君伐菹薪煮泲水以籍于天下,然则天下不减矣。''

桓公问于管子曰:``吾欲富本而丰五谷,可乎?''管子对曰:``不可。夫本富而财物众,不能守,则税干天下;五谷兴丰,巨钱而天下贵,则税于天下。然则吾民常为天下虏矣。夫善用本者,若以身济于大海,观风之所起。天下高则高,天下下则下。天下高我下,则财利税于天下矣。''

桓公问于管子曰,``事尽于此乎?''管子对曰:``未也。夫齐衢处之本,通达所出也,游子胜商之所道。人求本者,食吾本粟,因吾本币,骐骥黄金然后出。令有徐疾,物有轻重,然后天下之宝壹为我用。善者用非有,使非人。''

\hypertarget{header-n1029}{%
\subsection{揆度}\label{header-n1029}}

齐桓公问于管子曰:``自燧人以来,其大会可得而闻乎?''管子对曰:``燧人以来,未有不以轻重为天下也。共工之王,水处什之七,陆处什之三,乘天势以隘制天下。至于黄帝之王,谨逃其爪牙,不利其器,烧山林,破增薮,焚沛泽,逐禽兽,实以益人,然后天下可得而牧也。至于尧舜之王,所以化海内者,北用禺氏之玉,南贵江汉之珠,其胜禽兽之仇,以大夫随之。''桓公曰:``何谓也?''管子对曰:``令:`诸侯之子将委质者,皆以双武之皮,卿大夫豹饰,列大夫豹幨。'大夫散其邑粟与其财物以市虎豹之皮,故山林之人刺其猛兽若从亲戚之仇,此君冕服于朝,而猛鲁胜于外;大夫已散其财物,万人得受其流。此尧舜之数也。''

桓公曰,```事名二、正名五而天下治',何谓`事名二'?''对曰:``天策阳也,壤策阴也,此谓`事名二'。''``何谓`正名五'?''对曰:``权也,衡也,规也,矩也,准也,此谓`正名五'。其在色者,青黄白黑赤也;其在声者,宫商羽徵角也;其在味者,酸辛咸苦甘也。二五者,童山竭泽,人君以数制之人。味者所以守民口也,声者所以守民耳也,色者所以守民目也。人君失二五者亡其国,大夫失二五者亡其势,民失二五者亡其家。此国之至机也,谓之国机。''

轻重之法曰:``自言能为司马不能为司马者,杀其身以衅其鼓;自言能治田土不能治田土者,杀其身以衅其社;自言能为官不能为官者,劓以为门父。''故无敢奸能诬禄至于君者矣。故相任寅为官都,重门击柝不能去,亦随之以法。

桓公问于管子曰,``请问大准。''管子对曰:``大准者,天下皆制我而无我焉;此谓大准。''桓公曰:``何谓也?''管子对曰:``今天下起兵加我,臣之能谋厉国定名者,割壤而封;臣之能以车兵进退成功立名者,割壤而封。然则是天下尽封君之臣也,非君封之也。天下已封君之臣十里矣,天下每动,重封君之民二十里。君之民非富也,邻国富之。邻国每动,重富君之民,贫者重贫,富者重富。失准之数也。''桓公曰:``何谓也?''管子对曰:``今天下起兵加我,民弃其耒耜,出持戈于外,然则国不得耕。此非天凶也,此人凶也。君朝令而夕求具,民肆其财物与其五谷为雠,厌而去。贸人受而廪之,然则国财之一分在贾人。师罢,民反其事,万物反其重。贾人出其财物,国币之少分廪于贾人。若此则币重三分,财物之轻重三分,贾人市于三分之间,国之财物尽在贾人,而君无策焉。民更相制,君无有事焉。此轻重之大准也。''

管子曰:``人君操本,民不得操末;人君操始,民不得操卒。其在涂者,籍之于衢塞;其在谷者,守之春秋;其在万物者,立赀而行。故物动则应之。故豫夺其涂,则民无遵;君守其流。则民失其高。故守四方之高下,国无游贾,贵贱相当,此谓国衡;以利相守,则数归于君矣。''

管子曰:``善正商任者省有肆,省有肆则市朝闲,市朝闲则田野充,田野充则民财足,民财足则君赋敛焉不穷。今则不然,民重而君重,重而不能轻;民轻而君轻,轻而不能重。天下善者不然,民重则君轻,民轻则君重,此乃财余以满不足之数也。故凡不能调民利者,不可以为大治。不察于终始,不可以为至矣。动左右以重相因,二十国之策也;盐铁二十国之策也;锡金二十国之策也。五官之数;不籍于民。''

桓公问于管子曰:``轻重之数恶终?''管子对曰:``若四时之更举,无所终。国有患忧,轻重五谷以调用,积余臧羡以备赏。天下宾服,有海内,以富诚信仁义之士,故民高辞让,无为奇怪者,彼轻重者,诸侯不服以出战,诸侯宾服以行仁义。''

管子曰:``一岁耕,五岁食,粟贾五倍。一岁耕,六岁食,粟贾六倍。二年耕而十一年食。夫富能夺,贫能予,乃可以为天下。且天下者,处兹行兹,若此而天下可壹也。夫天下者,使之不使,用之不用。故善为天下者,毋曰使之,使不得不使;毋曰用之,用不得不用也。''

管子曰:``善为国者,如金石之相举,重钧则金倾。故治权则势重,治道则势羸。今谷重于吾国,轻于天下,则诸侯之自泄,如原水之就下。故物重则至,轻则去。有以重至而轻处者,我动而错之,天下即已于我矣。物臧则重,发则轻,散则多。币重则民死利,币轻则决而不用,故轻重调于数而止。''

``五谷者,民之司命也;刀币者,沟渎也;号令者,徐疾也。``令重于宝,社稷重于亲戚',胡谓也?''对曰:``夫城郭拔,社稷不血食,无生臣。亲没之后,无死子。此社稷之所重于亲戚者也。故有城无人,谓之守平虚;有人而无甲兵而无食,谓之与祸居。''

桓公问管子曰:``吾闻海内玉币有七策,可得而闻乎?''管子对曰:``阴山之礝碈,一策也;燕之紫山白金,一策也;发、朝鲜之文皮,一策也;汝、汉水之右衢黄金,一策也;江阳之珠,一策也;秦明山之曾青,一策也;禺氏边山之玉,一策也。此谓以寡为多,以狭为广。天下之数尽于轻重矣。''

桓公问于管子曰:``阴山之马具驾者千乘,马之平贾万也,金之平贾万也。吾有伏金千斤,为此奈何?''管子对曰:``君请使与正籍者,皆以币还于金,吾至四万,此一为四矣。吾非埏埴摇炉櫜而立黄金也,今黄金之重一为四者,数也。珠起于赤野之末光,黄金起于汝汉水之右衢,玉起于禺氏之边山。此度去周七千八百里,其涂远,其至阨。故先王度用其重而因之,珠玉为上币,黄金为中币,刀布为下币。先王高下中币,利下上之用。''

百乘之国,中而立市,东西南北度五十里。一日定虑,二日定载,三日出竟,五日而反。百乘之制轻重,毋过五日。百乘为耕田万顷,为户万户,为开口十万人,为分者万人,为轻车百乘,为马四百匹。千乘之国,中而立市,东西南北度百五十余里。二日定虑,三日定载,五日出竟,十日而反。千乘之制轻重,毋过一旬。千乘为耕田十万顷,为户十万户,为开口百万人,为当分者十万人,为轻车千乘,为马四千匹。万乘之国,中而立市,东西南北度五百里。三日定虑,五日定载,十日出竟,二十日而反。万乘之制轻重,毋过二旬。万乘为耕田百万顷,为户百万户,为开口千万人,为当分者百万人,为轻车万乘,为马四万匹。

管子曰:``匹夫为鳏,匹妇为寡,老而无子者为独。君问其若有子弟师役而死者,父母为独,上必葬之:衣衾三领,木必三寸,乡吏视事,葬于公壤。若产而无弟兄,上必赐之匹马之壤。故亲之杀其子以为上用,不苦也。君终岁行邑里,其人力同而宫室美者,良萌也,力作者也,脯二束、酒一石以赐之;力足荡游不作,老者谯之,当壮者遣之边戍:民之无本者贷之圃强。故百事皆举,无留力失时之民。此皆国策之数也。''

上农挟五,中农挟四,下农挟三。上女衣五,中女衣四,下女衣三。农有常业,女有常事。一农不耕,民有为之饥者;一女不织,民有为之寒者。饥寒冻饿,必起于粪土。故先王谨于其始,事再其本,民无者卖其子。三其本,若为食。四其本,则乡里给。五其本,则远近通,然后死得葬矣。事不能再其本,而上之求焉无止,然则奸涂不可独遵,货财不安于拘。随之以法,则中内民也,轻重不调,无(米亶)之民不可责理,鬻子不可得使,君失其民,父失其子,亡国之数也。

管子曰:``神农之数曰:`一谷不登,减一谷,谷之法什倍。二谷不登,减二谷,谷之法再十倍。'夷疏满之,无食者予之陈,无种者贷之新,故无什倍之贾,无倍称之民。''

\hypertarget{header-n1048}{%
\subsection{国准}\label{header-n1048}}

桓公问于管子曰:``国准可得闻乎?''管子对曰,``国准者,视时而立仪。''桓公曰:``何谓视时而立仪?''对曰:``黄帝之王,谨逃其爪牙。有虞之王,枯泽童山。夏后之王,烧增薮,焚沛泽,不益民之利。殷人之王,诸侯无牛马之牢,不利其器。周人之王,官能以备物。五家之数殊而用一也。''

桓公曰:``然则五家之数,籍何者为善也?''管子对曰:``烧山林,破增薮,焚沛泽,猛兽众也。童山竭泽者,君智不足也。烧增薮,焚沛泽,不益民利,逃械器,闭智能者,辅己者也。诸侯无牛马之牢,不利其器者,曰淫器而壹民心者也。以人御人,逃戈刃,高仁义,乘天固以安己者也。五家之数殊而用一也。''

桓公曰:``今当时之王者立何而可?''管子对曰:``请兼用五家而勿尽。''桓公曰,``何谓?''管子对曰:``立祈祥以固山泽,立械器以使万物,天下皆利而谨操重策。童山竭泽,益利搏流。出山金立币,存菹丘,立骈牢,以为民饶。彼菹菜之壤,非五谷之所生也,麋鹿牛马之地。春秋赋生杀老,立施以守五谷,此以无用之壤臧民之羸。五家之数皆用而勿尽。''

桓公曰:``五代之王以尽天下数矣,来世之王者可得而闻乎?''管子对曰:``好讥而不乱,亟变而不变,时至则为,过则去。王数不可豫致。此五家之国准也。''

\hypertarget{header-n1055}{%
\subsection{轻重甲}\label{header-n1055}}

桓公曰:``轻重有数乎?''管子对曰:``轻重无数,物发而应之,闻声而乘之。故为国不能来大下之财,致天下之民,则国不可成。''桓公曰:``何谓来天下之财?''管子对曰:``昔者桀之时,女乐三万人,端譟晨,乐闻于三衢,是无不服文绣衣裳者。伊尹以薄之游女工文绣篡组,一纯得粟百钟于桀之国。夫桀之国者,天子之国也,桀无天下忧,饰妇女钟鼓之乐,故伊尹得其粟而夺之流。此之谓来天下之财。''桓公曰:``何谓致天下之民?''管子对曰:``请使州有一掌,里有积五窌。民无以与正籍者予之长假,死而不葬者予之长度。饥者得食,寒者得衣,死者得葬,不资者得振,则天下之归我者若流水,此之谓致天下之民。故圣人善用非其有,使非其人,动言摇辞,万民可得而亲。''桓公曰:``善。''

桓公问管子曰:``夫汤以七十里之薄,兼桀之天下,其故何也?''管子对曰:``桀者冬不为杠,夏不束柎,以观冻溺。弛牝虎充市,以观其惊骇。至汤而不然。夷兢而积粟,饥者食之,寒者衣之,不资者振之,天下归汤若流水。此桀之所以失其天下也。''桓公曰:``桀使汤得为是,其故何也?''管子曰:``女华者,桀之所爱也,汤事之以千金;曲逆者,桀之所善也,汤事之以千金。内则有女华之阴,外则有曲逆之阳,阴阳之议合,而得成其天子。此汤之阴谋也。''

桓公曰:``轻重之数,国准之分,吾已得而闻之矣,请问用兵奈何?''管子对曰:``五战而至于兵。''桓公曰:``此若言何谓也?''管子对曰:``请战衡,战准,战流,战权,战势。此所谓五战而至于兵者也。''桓公曰:``善。''

桓公欲赏死事之后,曰:``吾国者,衢处之国,馈食之都,虎狼之所栖也,今每战舆死扶伤,如孤,茶首之孙,仰倳戟之宝,吾无由与之,为之奈何?''管子对曰:``吾国之豪家,迁封、食邑而居者,君章之以物则物重,不章以物则物轻;守之以物则物重,不守以物则物轻。故迁封、食邑、富商、蓄贾、积余、藏羡、跱蓄之家,此吾国之豪也,故君请缟素而就士室,朝功臣、世家、迁封、食邑、积余、藏羡、跱蓄之家曰:`城肥致冲,无委致围。天下有虑,齐独不与其谋?子大夫有五谷菽粟者勿敢左右,请以平贾取之子。'与之定其券契之齿。釜鏂之数,不得为侈弇焉。困穷之民闻而籴之,釜鏂无止,远通不推。国粟之贾坐长而四十倍。君出四十倍之粟以振孤寡,牧贫病,视独老穷而无子者;靡得相鬻而养之,勿使赴于沟浍之中,若此,则士争前战为颜行,不偷而为用,舆死扶伤,死者过半。此何故也?士非好战而轻死,轻重之分使然也。''

桓公曰:``皮、干、筋、角之征甚重。重籍于民而贵市之皮、干、筋、角,非为国之数也。''管子对曰:``请以令高杠柴池,使东西不相睹,南北不相见。''桓公曰:``诺。''行事期年,而皮、干、筋、角之征去分,民之籍去分。桓公召管子而问曰:``此何故也?''管子对曰:``杠、池平之时,夫妻服簟,轻至百里,今高杠柴池,东西南北不相睹,天酸然雨,十人之力不能上;广泽遇雨,十人之力不可得而恃。夫舍牛马之力所无因。牛马绝罢,而相继死其所者相望,皮、干、筋、角徒予人而莫之取。牛马之贾必坐长而百倍。天下闻之,必离其牛马而归齐若流。故高杠柴池,所以致天下之牛马而损民之籍也,《道若秘》云:`物之所生,不若其所聚。'''

桓公曰:``弓弩多匡(车多)者,而重籍于民,奉缮工,而使弓弩多匡(车多)者,其故何也?''管子对曰:``鹅骛之舍近,鹍鸡鹄(鸟包)之通远。鹄鹍之所在,君请式璧而聘之。''桓公曰:``诺。''行事期年,而上无阙者,前无趋人。三月解(去勹),弓弩无匡(车多)者。召管子而问曰,``此何故也?''管子对曰:``鹄鹍之所在,君式璧而聘之。菹泽之民闻之,越平而射远,非十钧之弩不能中鹍鸡鹄(鸟包)。彼十钧之弩,不得(上非下束)擏不能自正。故三月解医而弓弩无匡(车多)者,此何故也?以其家习其所也。''

桓公曰:``寡人欲藉于室屋。''管子对曰:``不可,是毁成也。''``欲藉于万民。''管子曰:``不可,是隐情也。''``欲藉于六畜。''管子对曰:``不可,是杀生也。''``欲藉于树木。''管子对曰:``不可,是伐生也。''``然则寡人安藉而可?''管子对曰:``君请藉于鬼神。''桓公忽然作色曰:``万民、室屋、六畜、树木且不可得藉:鬼神乃可得而藉夫?''管子对曰:``厌宜乘势,事之利得也;计议因权,事之囿大也。王者乘势,圣人乘幼,与物皆宜。''桓公曰:``行事奈何?''管子对曰:``昔尧之五吏五官无所食,君请立五厉之祭,祭尧之五吏,春献兰,秋敛落;原鱼以为脯,鲵以为殽。若此,则泽鱼之正,伯倍异日,则无屋粟邦布之藉。此之谓设之以祈祥,推之以礼义也。然则自足,何求于民也?''

桓公曰:``天下之国,莫强于越,今寡人欲北举事孤竹、离枝,恐越人之至,为此有道乎?''管子对曰:``君请遏原流,大夫立沼池,令以矩游为乐,则越人安敢至?''桓公曰:``行事奈何?''管子对曰:``请以令隐三川,立员都,立大舟之都。大身之都有深渊,垒十仞。令曰:`能游者赐千金。'未能用金千,齐民之游水,不避吴越。''桓公终北举事于孤竹、离校。越人果至,隐曲蔷以水齐。管子有扶身之士五万人,以待战于曲菑,大败越人。此之谓水豫。

齐之北泽烧,火光照堂下。管子入贺桓公曰:``吾田野辟,农夫必有百倍之利矣。''是岁租税九月而具,粟又美。桓公召管子而问曰:``此何故也?''管子对曰:``万乘之国、千乘之国,不能无薪而炊。今北泽烧。莫之续,则是农夫得居装而卖其薪荛,一束十倍。则春有以倳耜,夏有以决芸。此租税所以九月而具也。''

桓公忧北郭民之贫、召管子而问曰、``北郭者,尽屦缕之甿也,以唐园为本利,为此有道乎?''管子对曰:``请以令:禁百钟之家不得事鞒,千钟之家不得为唐园,去市三百步者不得树葵菜,若此,则空闲有以相给资,则北郭之甿有所雠。其手搔之功,唐园之利,故有十倍之利。''

管子曰:``阴王之国有三,而齐与在焉。''桓公曰:``此若言可得闻平?''管子对曰:``楚有汝、汉之黄金,而齐有渠展之盐,燕有辽东之煮,此阴王之国也。且楚之有黄金,中齐有蔷石也。苟有操之不工,用之不善,天下倪而是耳。使夷吾得居楚之黄金,吾能令农毋耕而食,女毋织而衣。今齐有渠展之盐,请君伐菹薪,煮沸火水为盐,正而积之。''桓公曰:``诺。''十月始正,至于正月,成盐三万六千钟。召管子而问曰:``安用此盐而可?''管子对曰:``孟春既至,农事且起。大夫无得缮冢墓,理宫室,立台榭,筑墙垣。北海之众无得聚庸而煮盐。若此,则盐必坐长而十倍。''桓公曰:``善。行事奈何?''管子对曰:``请以令粜之梁、赵、宋、卫、濮阳,彼尽馈食之也。国无盐则肿,守圉之国,用盐独甚。''桓公曰:``诺。''乃以令使粜之,得成金万一千余斤。桓公召管子而问曰:``安用金而可?''管子对曰:``请以令使贺献、出正籍者必以金,金坐长而百倍。运金之重以衡万物,尽归于君。故此所谓用若挹于河海,若输之给马。此阴王之业。''

管子曰:``万乘之国必有万金之贾,千乘之国必有干金之贾,百乘之国必有百金之贾,非君之所赖也,君之所与。故为人君而不审其号令,则中一国而二君二王也。''桓公曰:``何谓一国而二君二王?''管子对曰:``今君之籍取以正,万物之贾轻去其分,皆入于商贾,此中一国而二君二王也。故贾人乘其弊以守民之时,贫者失其财,是重贫也;农夫失其五谷,是重竭也。故为人君而不能谨守其山林、菹泽、草莱,不可以立为天下王。''桓公曰:``此若言何谓也?''管子对曰:``山林、菹泽、草莱者,薪蒸之所出,牺牲之所起也。故使民求之,使民藉之,因此给之。私爱之于民,若弟之与兄,子之与父也,然后可以通财交殷也,故请取君之游财,而邑里布积之。阳春,蚕桑且至,请以给其口食筐曲之强。若此,则絓丝之籍去分而敛矣。且四方之不至,六时制之:春日倳耜,次日获麦,次日薄芋,次日树麻,次日绝菹,次日大雨且至,趣芸壅培。六时制之,臣给至于国都。善者乡因其轻重,守其委庐,故事至而不妄。然后可以立为天下王。''

管子曰:``一农不耕,民或为之饥;一女不织,民或为之寒。故事再其本,则无卖其子者;事三其本,则衣食足;事四其本,则正籍给;事五其本,则远近通,死得藏。今事不能再其本,而上之求焉无止,是使奸涂不可独行,遗财不可包止。随之以法,则是下艾民。食三升,则乡有正食而盗;食二升,则里有正食而盗;食一升,则家有正食而盗。今操不反之事,而食四十倍之粟,而求民之毋失,不可得矣。且君朝令而求夕具,有者出其财,无有者卖其衣屦,农夫粜其五谷,三分贾而去。是君朝令一怒,布帛流越而之天下。君求焉而无止,民无以待之,走亡而栖山阜。持戈之士顾不见亲,家族失而不分,民走于中而士遁于外。此不待战而内败。''

管子曰:``今为国有地牧民者,务在四时,守在仓廪。国多财则远者来,地辟举则民留处;仓廪实则知礼节,衣食足则知荣辱。今君躬犁垦田,耕发草土,得其谷矣。民人之食,有人若干步亩之数,然而有饿馁于衢闾者何也?谷有所藏也。今君铸钱立币,民通移,人有百十之数,然而民有卖子者何也?财有所并也。故为人君不能散积聚,调高下,分并财,君虽强本趣耕,发草立币而无止,民犹若不足也。''桓公问于管子曰:``今欲调高下,分并财,散积聚。不然,则世且并兼而无止,蓄余藏羡而不息,贫贱鳏寡独老不与得焉。散之有道,分之有数乎?''管子对曰:``唯轻重之家为能散之耳,请以令轻重之家。''恒公曰:``诺。''东车五乘,迎癸乙于周下原。桓公问四因与癸乙、管子、宁戚相与四坐,桓公曰:``请间轻重之数。''癸乙曰:``重籍其民者失其下,数欺诸侯者无权与。''管子差肩而问曰:``吾不籍吾民,何以奉车革?不籍吾民,何以待邻国?''癸乙曰:``唯好心为可耳!夫好心则万物通,万物通则万物运,万物运则万物贱,万物贱则万物可因。知万物之可因而不因者,夺于天下。夺于天下者,国之大贼也。''桓公曰,``请问好心万物之可因?''癸乙曰:``有余富无余乘者,责之卿诸侯;足其所,不赂其游者,责之令大夫。若此则万物通,万物通则万物运,万物运则万物贱,万物贱则万物可因矣。故知三准同策者能为天下,不知三准之同策者不能为天下。故申之以号令,抗之以徐疾也,民乎其归我若流水。此轻重之数也。''

桓公问于管子曰:``今倳戟十万,薪菜之靡日虚十里之衍;顿戟一譟,而靡币之用日去千金之积。久之,且何以待之?''管子对曰:``粟贾平四十,则金贾四千。粟贾釜四十则钟四百也,十钟四千也,二十钟者为八千也。金贾四千,则二金中八千也。然则一农之事,终岁耕百亩,百亩之收不过二十钟,一农之事乃中二金之财耳。故粟重黄金轻,黄金重而粟轻,两者不衡立,故善者重粟之贾。釜四百,则是钟四千也,十钟四万,二十钟者八万。金贾四千,则是十金四万也,二十金者为八万。故发号出令,曰一农之事有二十金之策。然则地非有广狭,国非有贫富也,通于发号出令,审于轻重之数然。''

管子曰:``湩然击鼓,士愤怒;枪然击金,士帅然。策桐鼓从之,舆死扶伤,争进而无止。口满用,手满钱,非大父母之仇也,重禄重赏之所使也。故轩冕立于朝,爵禄不随,臣不为忠;中军行战,委予之赏不随,士不死其列陈。然则是大臣执于朝,而列陈之士执于赏也。故使父不得子其子,兄不得弟其弟,妻不得有其夫,唯重禄重赏为然耳,故不远道里而能威绝域之民,不险山川而能服有恃之国,发若雷霆,动若风雨,独出独入,莫之能圉。''

桓公曰:``四夷不服,恐其逆政游于天下而伤寡人,寡人之行为此有道乎?''管子对曰:``吴越不朝,珠象而以为币乎?发、朝鲜不朝,请文皮、服而为币乎?。禺氏不朝,请以白璧为币乎?昆仑之虚不朝,请以璆琳、琅玕为币乎?故夫握而不见于手,含而不见于口,而辟千金者,珠也;然后,八千里之吴越可得而朝也。一豹之皮,容金而金也;然后,八千里之发、朝鲜可得而朝也。怀而不见于抱,挟而不见于掖,而辟千金者,白璧也;然后,八千里之禺氏可得而朝也。簪珥而辟千金者,璆琳、琅玕也;然后,八千里之昆仑之虚可得而朝也。故物无主,事无接,远近无以相因,则四夷不得而朝矣。''

\hypertarget{header-n1075}{%
\subsection{轻重乙}\label{header-n1075}}

桓公曰,``天下之朝夕可定乎?''管子对曰:``终身不定。''桓公曰:``其不定之说,可得闻乎?''管子对曰:``地之东西二万八千里,南北二万六千里。天子中而立,国之四面,面万有余里。民之入正籍者亦万有余里。故有百倍之力而不至者,有十倍之力而不至者,有倪而是者。则远者疏,疾怨上。边境诸侯受君之怨民,与之为善,缺然不朝,是无子塞其涂。熟谷者去,天下之可得而霸?''桓公曰:``行事奈何?''管子对曰:``请与之立壤列天下之旁,天子中立,地方千里,兼霸之壤三百有余里,佌诸侯度百里,负海子男者度七十里,若此则如胸之使臂,臂之使指也。然则小不能分于民,准徐疾羡不足,虽在下不为君忧。夫海出泲无止,山生金木无息,草木以时生,器以时靡币,泲水之盐以日消。终则有始,与天壤争,是谓立壤列也。''

武王问于癸度曰:``贺献不重,身不亲于君;左右不足,友不善于群臣。故不欲收穑户籍而给左右之用,为之有道乎?''癸度对曰:``吾国者衢处之国也,远秸之所通、游客蓄商之所道,财物之所遵。故苟入吾国之粟,因吾国之币,然后,载黄金而出。故君请重重而衡轻轻,运物而相因,则国策可成。故谨毋失其度,未与民,可治?''武王曰:``行事奈何?''癸度曰:``金出于汝、汉之右衢,珠出于赤野之末光,玉出于禺氏之旁山。此皆距周七千八百余里,其涂远,其至阨。故先王度用于其重,因以珠玉为上币,黄金为中币,刀布为下币。故先王善高下中币,制下上之用,而天下足矣。''

桓公曰,``衡谓寡人曰:`一农之事必有一耜、一铫。一镰、一鎒、一椎、一铚,然后成为农。一车必有一斤、一锯、一釭、一钻、一凿、一銶、一轲,然后成为车。一女必有一刀、一锥、一箴、一鉥,然后成为女。请以令断山木,鼓山铁。是可以无籍而用尽。'''管子对曰:``不可。今发徒隶而作之,则逃亡而不守;发民,则下疾怨上,边竟有兵则怀宿怨而不战。未见山铁之利而内败矣。故善者不如与民,量其重,计其赢,民得其十,君得其三。有杂之以轻重,守之以高下。若此,则民疾作而为上虏矣。''

桓公曰:``请问壤数。''管子对曰:``河(土於)诸侯,亩钟之国也。皟,山诸侯之国也。河(土於)诸侯常不胜山诸侯之国者,豫戒者也。''桓公曰:``此若言何谓也?''管子对曰:``夫河(土於)诸侯,亩钟之国也,故谷众多而不理,固不得有。至于山诸侯之国、则敛蔬藏菜,此之谓豫戒。''桓公曰:``壤数尽于此乎?''管子对曰:``未也。昔狄诸侯,亩钟之国也,故粟十钟而锱金,程诸侯,山诸侯之国也,故粟五釜而锱金。故狄诸侯十钟而不得倳戟,程诸侯五釜而得倳戟,十倍而不足,或五分而有余者,通于轻重高下之数。国有十岁之蓄,而民食不足者,皆以其事业望君之禄也。君有山海之财,而民用不足者,皆以其事业交接于上者也。故租籍,君之所宜得也;正籍者,君之所强求也。亡君废其所宜得而敛其所强求,故下怨上而令不行。民,夺之则怒,予之则喜。民情固然。先王知其然,故见予之所,不见夺之理。故五谷粟米者,民之司命也;黄金刀布者,民之通货也。先王善制其通货以御其司命,故民力可尽也。''

管子曰:``泉雨五尺,其君必辱;食称之国必亡,待五谷者众也。故树木之胜霜露者不受令于天,家足其所者不从圣人。故夺然后予,高然后下,喜然后怒,天下可举。''

桓公曰:``强本节用,可以为存乎?''管子对曰,``可以为益愈,而未足以为存也。昔者纪氏之国强本节用者,其五谷丰满而不能理也,四流而归于天下。若是,则纪氏其强本节用,适足以使其民谷尽而不能理,为天下虏。是以其国亡而身无所处。故可以益愈而不足以为存,故善为国者,天下下,我高;天下轻,我重;天下多,我寡。然后可以朝天下。''

桓公曰:``寡人欲毋杀一士,毋顿一戟,而辟方都二,为之有道乎?''管子对曰:``泾水十二空,汶、渊、洙浩满,三之於。乃请以令使九月种麦,日至日获,则时雨未下而利农事矣。''桓公曰:``诺。''令以九月种麦,日至而获。量其艾、一收之积中方都二。故此所谓善因天时,辨于地利而辟方都之道也。

管子入复桓公曰:``终岁之租金四万二千金,请以一朝素赏军士。''桓公曰:``诺。''以令至鼓期于泰舟之野期军士。桓公乃即坛而立,宁戚、鲍叔、隰朋、易牙,宾须无皆差肩而立。管子执枹而揖军士曰:``谁能陷陈破众者,赐之百金。''三问不对。有一人秉剑而前,问曰:``几何人之众也?''管子曰:``千人之众。''``千人之众,臣能陷之。''赐之百金。管子又曰:``兵接弩张,谁能得卒长者,赐之百金。''问曰:``几何人卒之长也?''管子曰:``千人之长。''``千人之长,臣能得之。''赐之百金。管子又曰:``谁能听旌旗之所指,而得执将首者,赐之千金。''言能得者垒千人,赐之人千金。其余言能外斩首者,赐之人十金。一朝素赏,四万二千金廓然虚。桓公惕然太息曰:``吾曷以识此?''管子对曰:``君勿患。且使外为名于其内,乡为功于其亲,家为德于其妻子。若此,则士必争名报德,无北之意矣。吾举兵而攻,破其军,并其地,则非特四万二千金之利也。''五子曰:``善。''桓公曰:``诺。''乃诫大将曰:``百人之长,必为之朝礼;干人之长,必拜而送之,降两级。其有亲戚者,必遗之酒四石,肉四鼎;其无亲戚者,必遗其妻子酒三石,肉三鼎。''行教半岁,父教其子,兄教其弟,妻谏其夫,曰:``见其若此其厚,而不死列陈,可以反于乡乎?''桓公终举兵攻莱,战于莒必市里。鼓旗未相望,众少未相知,而莱人大遁。故遂破其军,兼其地,而虏其将。故未列地而封,未出金而赏,破莱军,并其地,擒其君。此素赏之计也。

桓公曰:``曲防之战,民多假贷而给上事者。寡人欲为之出赂,为之奈何?''管子对曰:``请以令:令富商蓄贾百符而一马,无有者取于公家。若此,则马必坐长而百倍其本矣。是公家之马不离其牧皂,而曲防之战赂足矣。''

桓公问于管子曰:``崇弟、蒋弟,丁、惠之功世,吾岁罔,寡人不得籍斗升焉,去。菹菜、咸卤、斥泽、山间(土畏)(土垒)不为用之壤,寡人不得籍斗升焉,去一。列稼缘封十五里之原,强耕而自以为落,其民寡人不得籍斗升焉。则是寡人之国,五分而不能操其二,是有万乘之号而无干乘之用也。以是与天子提衡,争秩于诸侯,为之有道乎?''管子对曰:``唯籍于号令为可耳。''桓公曰,``行事奈何?''管于对曰:``请以令发师置屯籍农,十钟之家不行,百钟之家不行,千钟之家不行。行者不能百之一,千之十,而囷窌之数皆见于上矣。君案囷窌之数,令之曰:`国贫而用不足,请以平价取之子,皆案囷窌而不能挹损焉。'君直币之轻重以决其数,使无券契之责,则积藏囷窌之粟皆归于君矣。故九州无敌,竟上无患。''令曰:``罢兵归农,无所用之。''管子曰:``天下有兵,则积藏之粟足以备其粮;天下无兵,则以赐贫甿,若此则菹菜、咸卤、斥泽、山间之壤无不发草:此之谓籍于号令。''

管子曰:``滕鲁之粟釜百,则使吾国之粟釜千;滕鲁之粟四流而归我、若下深谷者。非岁凶而民饥也,辟之以号令,引之以徐疾,施平其归我若流水。''

桓公曰:``吾欲杀正商贾之利而益农夫之事,为此有道乎?''管子对曰:``粟重而万物轻,粟轻而万物重,两者不衡立。故杀正商贾之利而益农夫之事,则请重粟之价金三百。若是则田野大辟,而农夫劝其事矣。''桓公曰:``重之有道乎?''管子对曰:``请以令与大夫城藏,使卿、诸侯藏千钟,令大夫藏五百钟,列大夫藏百钟,富商蓄贾藏五十钟,内可以为国委,外可以益农夫之事。''桓公曰:``善。''下令卿诸侯令大夫城藏。农夫辟其五谷,三倍其贾。则正商失其事,而农夫有百倍之利矣。

桓公问于管子曰:``衡有数乎?''管子对曰:``衡无数也。衡者使物一高一下,不得常固。''桓公曰:``然则衡数不可调耶?''管子对曰:``不可调。调则澄。澄则常,常则高下不贰,高下不贰则万物不可得而使固。''桓公曰:``然则何以守时?''管子对曰:``夫岁有四秋,而分有四时。故曰:农事且作,请以什伍农夫赋耜铁,此之谓春之秋。大夏且至,丝纩之所作,此之谓夏之秋。而大秋成,五谷之所会,此之谓秋之秋。大冬营室中,女事纺织缉缕之所作也,此之谓冬之秋。故岁有四秋,而分有四时。已有四者之序,发号出令,物之轻重相什而相伯。故物不得有常固。故曰衡无数。''

桓公曰,``皮干筋角竹箭羽毛齿革不足,为此有道乎?''管子曰:``惟曲衡之数为可耳。''桓公曰,``行事奈何?''管子对曰:``请以令为诸侯之商贾立客舍,一乘者有食,三乘者有刍菽,五乘者有伍养。天下之商贾归齐若流水。''

\hypertarget{header-n1092}{%
\subsection{轻重丁}\label{header-n1092}}

石璧谋

桓公曰:``寡人欲西朝天子而贺献不足,为此有数乎?''管子对曰:``请以令城阴里,使其墙三重而门九袭。因使玉人刻石而为璧,尺者万泉,八寸者八千,七寸者七千,珪中四千,瑗中五百。''璧之数已具,管子西见天子曰:``弊邑之君欲率诸侯而朝先王之庙,观于周室。请以令使天下诸侯朝先王之庙,观于周室者,不得不以彤弓石璧。不以彤弓石璧者,不得入朝。''天子许之曰:``诺。''号令于天下。天下诸侯载黄金珠玉五谷文采布泉输齐以收石璧。石璧流而之天下,天下财物流而之齐。故国八岁而无籍,阴里之谋也。

菁茅谋

桓公曰:``天子之养不足,号令赋于天下则不信诸侯,为此有道乎?''管子对曰:``江淮之间有一茅而三脊母至其本,名之曰菁茅。请使天子之吏环封而守之。夫天子则封于太山、禅于梁父。号令天下诸侯曰:`诸从天子封于太山、禅于梁父者,必抱菁茅一束以为禅籍。不如令者不得从。'''天子下诸侯载其黄金。争秩而走,江淮之菁茅坐长而十倍,其贾一柬而百金。故天子三日即位,天下之金四流而归周若流水。故周天子七年不求贺献者,菁茅之谋也。

桓公曰:``寡人多务,令衡籍吾国之富商蓄贾称贷家,以利吾贫萌、农夫,不失其本事。反此有道乎?''管子对曰:``唯反之以号令为可耳。''桓公说:``行事奈何?''管子对曰:``请使宾胥无驰而南,隰朋驰而北,宁戚驰而东,鲍叔驰而西。四子之行定,夷吾请号令谓四子曰:`子皆为我君视四方称贷之间,其受息之氓几何千家,以报吾。'''鲍叔驰而西,反报曰:``西方之氓者,带济负河,菹泽之萌也。渔猎取薪蒸而为食。其称贷之家多者千钟,少者六、七百钟。其出之,钟也一钟。其受息之萌九百余家。''宾胥无驰而南。反报曰:``南方之萌者,山居谷处,登降之萌也。上斫轮轴,下采杼栗,田猎而为食。其称贷之家多者千万,少者六、七百万。其出之,中伯伍也。其受息之萌八百余家。''宁戚驰而东。反报曰:``东方之萌,带山负海,若处,上断福,渔猎之萌也。治葛缕而为食。其称贷之家棗丁、惠、高、国,多者五千钟,少者三千钟。其出之,中钟五釜也。其受息之萌八、九百家。''隰朋驰而北。反报曰:``北方之萌者,衍处负海,煮泲水为盐,梁济取鱼之萌也。薪食。其称贷之家多者千万,少者六、七百万。其出之,中伯二十也。受息之萌九百余家。''凡称贷之家出泉三千万,出粟三数千万钟,受子息民三万家。四子已报,管子曰:``不弃我君之有萌中一国而五君之正也,然欲国之无贫,兵之无弱,安可得哉?''桓公曰:``为此有道乎?''管子曰:``惟反之以号令为可。请以令贺献者皆以鐻枝兰鼓,则必坐长什倍其本矣,君之栈台之职亦坐长什倍。请以令召称贷之家,君因酌之酒,太宰行觞。桓公举衣而问曰:`寡人多务,令衡籍吾国。闻子之假贷吾贫萌,使有以终其上令。寡人有鐻枝兰鼓,其贾中纯万泉也。愿以为吾贫萌决其子息之数,使无券契之责。'称贷之家皆齐首而稽颡曰:`君之忧萌至于此!请再拜以献堂下。'桓公曰:`不可。子使吾萌春有以倳耜,夏有以决芸。寡人之德子无所宠,若此而不受,寡人不得于心。'故称贷之家曰皆:`再拜受。'所出栈台之织未能三千纯也,而决四方子息之数,使无券契之责。四方之萌闻之,父教其子,兄教其弟曰:`夫垦田发务,上之所急,可以无庶乎?君之忧我至于此!'此之谓反准。''

管子曰:``昔者癸度居人之国,必四面望于天下,天下高亦高。天下高我独下,必失其国于天下。''桓公曰:``此若言曷谓也?''管子对曰:``昔莱人善染。练茈之于莱纯锱,緺绶之于莱亦纯锱也。其周中十金。莱人知之,闻纂茈空。周且敛马作见于莱人操之,莱有推马。是自莱失纂茈而反准于马也。故可因者因之,乘者乘之,此因天下以制天下。此之谓国准。''

桓公曰:``齐西水潦而民饥,齐东丰庸而粟贱,欲以东之贱被西之贵,为之有道乎?''管子对曰:``今齐西之粟釜百泉,则鏂二十也。齐东之粟釜十泉,则鏂二钱也。请以令籍人三十泉,得以五谷菽粟决其籍。若此,则齐西出三斗而决其籍,齐东出三釜而决其籍。然则釜十之粟皆实子仓廪,西之民饥者得食,寒者得衣;无本者予之陈,无种者予之新。若此,则东西之相被,远近之准平矣。''

桓公曰,``衡数吾已得闻之矣,请问国准。''管子对曰:``孟春且至,沟渎阮而不遂,溪谷报上之水不安于藏,内毁室屋,坏墙垣,外伤田野,残禾稼。故君谨守泉金之谢物,且为之举。大夏,帷盖衣幕之奉不给,谨守泉布之谢物,且为之举。大秋,甲兵求缮,弓弩求弦,谨丝麻之谢物,且为之举。大冬,任甲兵,粮食不给,黄金之赏不足,谨守五谷黄金之谢物,且为之举。已守其谢,富商蓄贾不得如故。此之谓国准。''

龙斗于马谓之阳,牛山之阴。管子入复于桓公曰:``天使使者临君之郊,请使大夫初饬、左右玄服天之使者乎!''天下闻之曰:``神哉齐桓公,天使使者临其郊。''不待举兵,而朝者八诸侯。此乘天威而动天下之道也。故智者役使鬼神而愚者信之。

桓公终神,管子入复桓公曰:``地重,投之哉兆,国有恸。风重,投之哉兆。国有枪星,其君必辱;国有彗星,必有流血。浮丘之战,彗之所出,必服天下之仇。今彗星见于齐之分,请以令朝功臣世家,号令于国中曰:`彗星出,寡人恐服天下之仇。请有五谷菽粟布帛文采者,皆勿敢左右。国且有大事,请以平贾取之。'功臣之家、人民百姓皆献其谷菽粟泉金,归其财物,以佐君之大事。此谓乘天灾而求民邻财之道也。''

桓公曰:``大夫多并其财而不出,腐朽五谷而不散。''管子对曰:``请以令召城阳大夫而请之。''桓公曰:``何哉?''管子对曰:```城阳大夫,嬖宠被絺,鹅骛含余,齐钟鼓之声,吹笙篪,同姓不入,伯叔父母远近兄弟皆寒而不得衣,饥而不得食。子欲尽忠于寡人,能乎?故子毋复见寡人。'灭其位,杜其门而不出。''功臣之家皆争发其积藏,出其资财,以予其远近兄弟。以为未足,又收国中之贫病孤独老不能自食之萌,皆与得焉。故桓公推仁立义、功臣之家兄弟相戚,骨肉相亲,国无饥民。此之谓缪数。

桓公曰:``峥丘之战,民多称贷负子息,以给上之急,度上之求。寡人欲复业产、此何以洽?''管子对曰:``惟缪数为可耳。''桓公曰:``诺。''令左右州曰,``表称贷之家,皆垩白其门而高其闾。''州通之师执折箓曰:``君且使使者。''桓公使八使者式璧而聘之,以给盐菜之用。称贷之家皆齐首稽颡而问曰:``何以得此也?''使者曰:``君令曰:`寡人闻之《诗》曰:恺悌君子,民之父母也。寡人有峥丘之战,吾闻子假贷吾贫萌,使有以给寡人之急,度寡人之求,使吾萌春有以倳耜,夏有以决芸,而给上事,子之力也。是以式璧而聘子,以给盐菜之用。故子中民之父母也。'''称贷之家皆折其券而削其书,发其积藏,出其财物,以赈贫病,分其故赀,故国中大给,峥丘之谋也。此之谓缪数。

桓公曰:``四郊之民贫,商贾之民富,寡人欲杀商贾之民以益四郊之民,为之奈何?''管子对曰:``请以令决洛之水,通之杭庄之间。''桓公曰:``诺。''行令未能一岁,而郊之民殷然益富,商贸之民廓然益贫。桓公召管子而问曰:``此其故何也?''管子对曰:``洛之水通之杭庄之间,则屠酤之汁肥流水,则蚊虻巨雄、翡燕小鸟皆归之,宜昏饮,此水上之乐也。贾人蓄物而卖为雠,买为取,市未央毕,而委舍其守列,投蚊虵巨雄;新冠五尺请挟弹怀丸游水上,弹翡燕小鸟,被于暮。故贱卖而贵买,四郊之民卖贱,何为不富哉?商贾之人,何为不贫乎?''桓公曰:``善。''

桓公曰:``五衢之民,衰然多衣弊而屦穿,寡人欲使帛、布、丝、纩之贾贱,为之有道乎?''管子曰:``请以令沐途旁之树枝,使无尺寸之阴。''桓公曰:``诺。''行令未能一岁,五衢之民皆多衣帛完屦。桓公召管子而问曰:``此其何故也?''管子对曰:``途旁之树未沐之时,五衢之民,男女相好往来之市者,罢市相睹树下,谈语终日不归。男女当壮,扶辇推舆,相睹树下,戏笑超距,终日不归。父兄相睹树下,论议玄语,终日不归。是以田不发,五谷不播,桑麻不种,茧缕不治。内严一家而三不归,则帛、布、丝、纩之贾安得不贵?''桓公曰:``善。''

桓公曰:``粜贱,寡人恐五谷之归于诸侯,寡人欲为百姓万民藏之,为此有道乎?''管子曰:``今者夷吾过市,有新成囷京者二家,君请式璧而聘之。''恒公曰:``诺。''行令半岁,万民闻之,舍其作业而为囷京以藏菽粟五谷者过半。桓公问管于曰:``此其何故也?''管子曰:``成囷京者二家,君式璧而聘之,名显于国中,国中莫不闻。是民上则无功显名于百姓也,功立而名成;下则实其囷京,上以给上为君。一举而名实俱在也,民何为也?''

桓公问管子曰:``请问王数之守终始,可得闻乎?''管子曰:``正月之朝,谷始也;日至百日,黍秫之始也;九月敛实,平麦之始也。''

管子问于桓公:``敢问齐方于几何里?''桓公曰:``方五百里。''管子曰:``阴雍长城之地,其于齐国三分之一,非谷之所生也。、龙夏,其于齐国四分之一也;朝夕外之,所墆齐地者五分之一,非谷之所生也。然则吾非托食之主耶?''桓公遽然起曰:``然则为之奈何?''管子对曰:``动之以言,溃之以辞,可以为国基。且君币籍而务,则贾人独操国趣;君谷籍而务,则农人独操国固。君动言操辞,左右之流君独因之,物之始吾已见之矣,物之终吾已见之矣,物之贾吾已见之矣。''管子曰:``长城之阳,鲁也;长城之阴,齐也。三败杀君二重臣定社稷者,吾此皆以孤突之地封者也。故山地者山也,水地者泽也,薪刍之所生者斥也。''公曰:``托食之主及吾地亦有道乎?''管子对曰:``守其三原。''公曰:``何谓三原?''管子对曰:``君守布则籍于麻,十倍其贾,布五十倍其贾。此数也。君以织籍,籍于系。未为系籍,系抚织,再十倍其价。如此,则云五谷之籍。是故籍于布则抚之系,籍于谷则抚之山,籍于六畜则抚之术。籍于物之终始而善御以言。''公曰:``善。''

管子曰:``以国一籍臣右守布万两而右麻籍四十倍其贾术。布五十倍其贾。公以重布决诸侯贾,如此而有二十齐之故。是故轻轶于贾谷制畜者则物轶于四时之辅。善为国者守其国之财,汤之以高下,注之以徐疾,一可以为百。未尝籍求于民,而使用若河海,终则有始。此谓守物而御天下也。''公曰:``然则无可以为有乎?贫可以为富乎?''管子对曰:``物之生未有刑,而王霸立其功焉。是故以人求人,则人重矣;以数求物,则物重矣。''公曰:``此若言何谓也?''管子对曰:``举国而一则无赀,举国而十则有百。然则吾将以徐疾御之,若左之授右,若右之授左,是以外内不踡,终身无咎。王霸之不求于人而求之终始,四时之高下,令之徐疾而已矣。源泉有竭,鬼神有歇,守物之终始,身不竭。此谓源究。''

\hypertarget{header-n1113}{%
\subsection{轻重戊}\label{header-n1113}}

桓公问于管子曰:``轻重安施?''管子对曰:``自理国戏以来,未有不以轻重而能成其王者也。''公曰:``何谓?''管子对曰:``虙戏作,造六峜以迎阴阳,作九九之数以合天道,而天下化之。神农作,树五谷淇山之阳,九州之民乃知谷食,而天下化之。黄帝作,钻燧生火,以熟荤臊,民食之无兹胃之病,而天下化之。黄帝之王,童山竭泽。有虞之王,烧曾薮,斩群害,以为民利,封土为社,置木为闾,始民知礼也。当是其时,民无愠恶不服,而天下化之。夏人之王,外凿二十虻,韘十七湛,疏三江,凿五湖,道四泾之水,以商九州之高,以治九薮,民乃知城郭、门闾、室屋之筑,而天下化之。殷人之王,立皂牢,服牛马,以为民利,而天下化之。周人之王,循六*(上山下念),合阴阳,而天下化之。''公曰:``然则当世之王者何行而可?''管子对曰:``并用而勿俱尽也。''公曰:``何谓?''管子对曰:``帝王之道备矣,不可加也。公其行义而已矣。''公曰:``其行义奈何?''管子对曰:``天子幼弱,诸侯亢强,聘享不上。公其弱强继绝,率诸侯以起周室之祀。''公曰:``善。''

桓公曰:``鲁粱之于齐也,千榖也,蜂螫也,齿之有唇也。今吾欲下鲁梁,何行而可?''管子对曰:``鲁粱之民俗为绨。公服绨,令左右服之,民从而眼之。公因令齐勿敢为,必仰于鲁梁,则是鲁梁释其农事而作绨矣。''桓公曰:``诺。''即为服于泰山之阳,十日而服之。管子告鲁梁之贾人曰:``子为我致绨千匹,赐子金三百斤;什至而金三千斤。''则是鲁梁不赋于民,财用足也。鲁梁之君闻之,则教其民为绨。十三月,而管子令人之鲁梁,鲁梁郭中之民道路扬尘,十步不相见,绁繑而踵相随,车毂齺,骑连伍而行。管子曰:``鲁梁可下矣。''公曰,``奈何?''管子对曰:``公宜服帛,率民去绨。闭关,毋与鲁粱通使。''公曰:``诺。''后十月,管子令人之鲁梁,鲁梁之民饿馁相及,应声之正无以给上。鲁梁之君即令其民去绨修农。谷不可以三月而得,鲁梁之人籴十百,齐粜十钱。二十四月,鲁梁之民归齐者十分之六;三年,鲁梁之君请服。

桓公问管子曰:``民饥而无食,寒而无衣,应声之正无以给上,室屋漏而不居,墙垣坏而不筑,为之奈何?''管子对曰:``沐涂树之枝也。''桓公曰:``诺。''令谓左右伯沐涂树之枝。左右伯受沐,涂树之枝阔。其年,民被白布,清中而浊,应声之正有以给上,室屋漏者得居,墙垣坏者得筑。公召管子问曰,``此何故也?''管子对曰,``齐者,夷莱之国也。一树而百乘息其下者,以其不也。众鸟居其上,丁壮者胡丸操弹居其下,终日不归。父老柎枝而论,终日不归。归市亦惰倪,终日不归。今吾沐涂树之枝,日中无尺寸之阴,出入者长时,行者疾走,父老归而治生,丁壮者归而薄业。彼臣归其三不归,此以乡不资也。''

桓公问于管子曰:``莱、莒与柴田相并,为之奈何?''管子对曰:``莱、莒之山生柴,君其率白徒之卒铸庄山之金以为币,重莱之柴贾。''莱君闻之,告左右曰:``金币者,人之所重也。柴者,吾国之奇出也。以吾国之奇出,尽齐之重宝,则齐可并也。''莱即释其耕农而治柴。管子即令隰朋反农。二年,桓公止柴。莱:莒之籴三百七十,齐粜十钱,莱、莒之民降齐者十分之七。二十八月,莱、莒之君请服。

桓公问于管子曰:``楚者,山东之强国也,其人民习战斗之道。举兵伐之,恐力不能过。兵弊于楚,功不成于周,为之奈何?''管子对曰:``即以战斗之道与之矣。''公曰:``何谓也?''管子对曰:``公贵买其鹿。''桓公即为百里之城,使人之楚买生鹿。楚生鹿当一而八万。管子即令桓公与民通轻重,藏谷什之六。令左司马伯公将白徒而铸钱于庄山,令中大夫王邑载钱二千万,求生鹿于楚。楚王闻之,告其相曰:``彼金钱,人之所重也,国之所以存,明王之所以赏有功。禽兽者群害也,明王之所弃逐也。今齐以其重宝贵买吾群害,则是楚之福也,天且以齐私楚也。子告吾民急求生鹿,以尽齐之宝。''楚人即释其耕农而田鹿。管子告楚之贾人曰:``子为我致生鹿二十,赐子金百斤。什至而金干斤也。''则是楚不赋于民而财用足也。楚之男于居外,女子居涂。隰朋教民藏粟五倍,楚以生鹿藏钱五倍。管子曰:``楚可下矣。''公曰:``奈何?''管子对曰:``楚钱五倍,其君且自得而修谷。钱五倍,是楚强也。''桓公曰:``诺。''因令人闭关,不与楚通使。楚王果自得而修谷,谷不可三月而得也,楚籴四百,齐因令人载粟处芊之南,楚人降齐者十分之四。三年而楚服。

桓公问于管子曰:``代国之出,何有?''管子对曰:``代之出,狐白之皮,公其贵买之。''管子曰:``狐白应阴阳之变,六月而壹见。公贵买之,代人忘其难得,喜其贵买,必相率而求之。则是齐金钱不必出,代民必去其本而居山林之中。离枝闻之,必侵其北。离枝侵其北,代必归于齐。公因令齐载金钱而往。''桓公曰,``诺。''即令中大夫王师北将人徒载金钱之代谷之上,求狐白之皮。代王闻之,即告其相曰:``代之所以弱于离枝者,以无金钱也。今齐乃以金钱求狐白之皮,是代之福也。子急令民求狐臼之皮以致齐之币,寡人将以来离枝之民。''代人果去其本,处山林之中,求狐白之皮。二十四月而不得一。离枝闻之,则侵其北。代王闻之,大恐,则将其士卒葆于代谷之上。离枝遂侵其北,王即将其士卒愿以下齐。齐未亡一钱币,修使三年而代服。

桓公问于管子曰:``吾欲制衡山之术,为之奈何?''管子对曰:``公其令人贵买衡山之械器而卖之。燕、代必从公而买之,秦、赵闻之,必与公争之。衡山之械器必倍其贾,天下争之,衡山械器必什倍以上。''公曰:``诺。''因令人之衡山求买械器,不敢辩其贵贾。齐修械器于衡山十月,燕、代闻之,果令人之衡山求买械器,燕、代修三月,秦国闻之,果令人之衡山求买械器。衡山之君告其相曰,``天下争吾械器,令其买再什以上。''衡山之民释其本,修械器之巧。齐即令隰朋漕粟千赵。赵籴十五,隰朋取之石五十。天下闻之,载粟而之齐。齐修械器十七月,修粜五月,即闭关不与衡山通使。燕、代、秦、赵即引其使而归。衡山械器尽,鲁削衡山之南,齐削衡山之北。内自量无械器以应二敌,即奉国而归齐矣。

\hypertarget{header-n1123}{%
\subsection{轻重己}\label{header-n1123}}

清神生心,心生规,规生矩,矩生方,方生正,正生历,历生四时,四时生万物。圣人因而理之,道遍矣。

以冬日至始,数四十六日,冬尽而春始。天子东出其国四十六里而坛,服青而絻青,搢玉总,带玉监,朝诸侯卿大夫列士,循千百姓,号曰祭日,牺牲以鱼。发出令曰:``生而勿杀,赏而勿罚,罪狱勿断,以待期年。''教民樵室钻鐩,墐灶泄井,所以寿民也。耟、耒、耨、怀、鉊、鈶、叉、橿、权渠、繉紲,所以御春夏之事也,必具。教民为酒食,所以为孝敬也。民生而无父母谓之孤子;无妻无子,谓之老鳏;无夫无子,谓之老寡。此三人者,皆就官而众,可事者不可事者,食如言而勿遗。多者为功,寡者为罪,是以路无行乞者也。路有行乞者,则相之罪也。天子之春令也。

以冬日至始,数九十二日,谓之春至。天子东出其国九十二里而坛,朝诸侯卿大夫列士,循于百姓,号曰祭星,十日之内,室无处女,路无行人。苟不树艺者,谓之贼人;下作之地,上作之天,谓之不服之民;处里为下陈,处师为下通,谓之役夫。三不树而主使之。天子之春令也。

以春日至始,数四十六日,春尽而夏始。天子服黄而静处,朝诸侯卿大夫列士,循于百姓,发号出令曰:``毋聚大众,毋行大火,毋断大木,诛大臣,毋斩大山,毋戮大衍。灭三大而国有害也。''天子之夏禁也。

以春日至始,数九十二日,谓之夏至,而麦熟。大子祀于太宗,其盛以麦。麦者,谷之始也。宗者,族之始也。同族者人,殊族者处。皆齐大材,出祭王母。天子之所以主始而忌讳也。

以夏日至始,数四十六日,夏尽而秋始,而黍熟。天子祀于太祖,其盛以黍。黍者,谷之美者也;祖者,国之重者也。大功者太祖,小功者小祖,无功者无祖。无功者皆称其位而立沃,有功者观于外。祖者所以功祭也,非所以戚祭也。天子之所以异贵贱而赏有功也。

以夏日至始,数九十二日,谓之秋至。秋至而禾熟。天子祀干太惢,西出其国百三十八里而坛,服白而絻白,搢玉总,带锡监,吹埙篪之风,凿动金石之音,朝诸侯卿大夫列士,循于百姓,号曰祭月,牺牲以彘。发号出令:``罚而勿赏,夺而勿予;罪狱诛而勿生,终岁之罪,毋有所赦。作衍牛马之实,在野者王。''天子之秋计也。

以秋日至始,数四十六日,秋尽而冬始。天子服黑絻黑而静处,朝诸侯卿大夫列士,循于百姓,发号出令曰:``毋行大火,毋斩大山,毋塞大水,毋犯天之隆。''天子之冬禁也。

以秋日至始,数九十二日,天子北出九十二里而坛,服黑而絻黑,朝诸侯卿大夫列士,号曰发繇。趣山人断伐,具械器;趣菹人薪雚苇,足蓄积。三月之后,皆以其所有易其所无,谓之大通三月之蓄。

凡在趣耕而不耕,民以不令,不耕之害也。宜芸而不芸,百草皆存,民以仅存,不芸之害也。宜获而不获,风雨将作,五谷以削,士民零落,不获之害也。宜藏而不藏,雾气阳阳,宜死者生,宜蛰者鸣,不藏之害也。张耜当弩,铫耨当剑戟,获渠当胁(革可),蓑笠当栐橹,故耕械具则战械备矣。

\end{document}
