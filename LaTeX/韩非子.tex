\PassOptionsToPackage{unicode=true}{hyperref} % options for packages loaded elsewhere
\PassOptionsToPackage{hyphens}{url}
%
\documentclass[]{article}
\usepackage{lmodern}
\usepackage{amssymb,amsmath}
\usepackage{ifxetex,ifluatex}
\usepackage{fixltx2e} % provides \textsubscript
\ifnum 0\ifxetex 1\fi\ifluatex 1\fi=0 % if pdftex
  \usepackage[T1]{fontenc}
  \usepackage[utf8]{inputenc}
  \usepackage{textcomp} % provides euro and other symbols
\else % if luatex or xelatex
  \usepackage{unicode-math}
  \defaultfontfeatures{Ligatures=TeX,Scale=MatchLowercase}
\fi
% use upquote if available, for straight quotes in verbatim environments
\IfFileExists{upquote.sty}{\usepackage{upquote}}{}
% use microtype if available
\IfFileExists{microtype.sty}{%
\usepackage[]{microtype}
\UseMicrotypeSet[protrusion]{basicmath} % disable protrusion for tt fonts
}{}
\IfFileExists{parskip.sty}{%
\usepackage{parskip}
}{% else
\setlength{\parindent}{0pt}
\setlength{\parskip}{6pt plus 2pt minus 1pt}
}
\usepackage{hyperref}
\hypersetup{
            pdfborder={0 0 0},
            breaklinks=true}
\urlstyle{same}  % don't use monospace font for urls
\setlength{\emergencystretch}{3em}  % prevent overfull lines
\providecommand{\tightlist}{%
  \setlength{\itemsep}{0pt}\setlength{\parskip}{0pt}}
\setcounter{secnumdepth}{0}
% Redefines (sub)paragraphs to behave more like sections
\ifx\paragraph\undefined\else
\let\oldparagraph\paragraph
\renewcommand{\paragraph}[1]{\oldparagraph{#1}\mbox{}}
\fi
\ifx\subparagraph\undefined\else
\let\oldsubparagraph\subparagraph
\renewcommand{\subparagraph}[1]{\oldsubparagraph{#1}\mbox{}}
\fi

% set default figure placement to htbp
\makeatletter
\def\fps@figure{htbp}
\makeatother


\date{}

\begin{document}

\hypertarget{header-n804}{%
\section{韩非子}\label{header-n804}}

\begin{center}\rule{0.5\linewidth}{\linethickness}\end{center}

\tableofcontents

\begin{center}\rule{0.5\linewidth}{\linethickness}\end{center}

\hypertarget{header-n810}{%
\subsection{初见秦}\label{header-n810}}

臣闻:``不知而言,不智;知而不言,不忠。''为人臣不忠,当死;言而不当,亦当死。虽然,臣愿悉言所闻,唯大王裁其罪。

臣闻:天下阴燕阳魏,连荆固齐,收韩而成从,将西面以与秦强为难。臣窃笑之。世有三亡,而天下得之,其此之谓乎!臣闻之曰:``以乱攻治者亡,以邪攻正者亡,以逆攻顺者亡''。今天下之府库不盈,囷仓空虚,悉其士民,张军数十百万,其顿首戴羽为将军断死于前不至千人,皆以言死。白刃在前,斧锧在后,而却走不能死也,非其士民不能死也,上不能故也。言赏则不与,言罚则不行,赏罚不信,故士民不死也。今秦出号令而行赏罚,有功无功相事也。出其父母怀衽之中,生未尝见寇耳。闻战,顿足徒裼,犯白刃,蹈炉炭,断死于前者皆是也。夫断死与断生者不同,而民为之者,是贵奋死也。夫一人奋死可以对十,十可以对百,百可以千,千可以对万,万可以克天下矣。今秦地折长补短,方数千里,名师数十百万。秦之号令赏罚,地形利害,天下莫若也。以此与天下,天下不足兼而有也。是故秦战未尝不克,攻未尝不取,所当未尝不破,开地数千里,此其大功也。然而兵甲顿,士民病,蓄积索,田畴荒,囷仓虚,四邻诸侯不服,霸王之名不成。此无异故,其谋臣皆不尽其忠也。

臣敢言之:往者齐南破荆,东破宋,西服秦,北破燕,中使韩、魏,土地广而兵强,战克攻取,诏令天下。齐之清济浊河,足以为限;长城巨防,足以为塞。齐,五战之国也,一战不克而无齐。由此观之,夫战者,万乘之存亡也。且闻之曰:``削迹无遗根,无与祸邻,祸乃不存。''秦与荆人战,大破荆,袭郢,取洞庭、五湖、江南,荆王君臣亡走,东服于陈。当此时也,随荆以兵,则荆可举;荆可举,则民足贪也,地足利也,东以弱齐、燕,中以凌三晋。然则是一举而霸王之名可成也,四邻诸侯可朝也,而谋臣不为,引军而退,复与荆人为和。令荆人得收亡国,聚散民,立社稷主,置宗庙,令率天下西面以与秦为难。此固以失霸王之道一矣。天下又比周而军华下,大王以诏破之,兵至梁郭下。围梁数旬,则梁可拔;拔梁,则魏可举;举魏,则荆、赵之意绝;荆、赵之意绝,则赵危;赵危而荆狐疑;东以弱齐、燕,中以凌三晋。然则是一举而霸王之名可成也,四邻诸侯可朝也,而谋
臣不为,引军而退,复与魏氏为和。令魏氏反收亡国,聚散民,立社稷主,置宗庙,令率天下西面以与秦为难。此固以失霸王之道二矣。前者穰侯之治秦也,用一国之兵而欲以成两国之功,是故兵终身暴露于外,士民疲病于内,霸王之名不成。此固以失霸王之道三矣。

赵氏,中央之国也,杂民所居也,其民轻而难用也。号令不治,赏罚不信,地形不便,下不能尽其民力。彼固亡国之形也,而不忧民萌,悉其士民军于长平之下,以争韩上党。大王以诏破之,拔武安。当是时也,赵氏上下不相亲也,贵贱不相信也。然则邯郸不守。拔邯郸,管山东河间,引军而去,西攻修武,逾华,绛代、上党。代四十六县,上党七十县,不用一领甲,不苦一士民,此皆秦有也。以代、上党不战而毕为秦矣,东阳、河外不战而毕反为齐矣,中山、呼沲以北不战而毕为燕矣。然则是赵举,赵举则韩亡,韩亡则荆、魏不能独立,荆、魏不能独立,则是一举而坏韩、蠹魏、拔荆,东以弱齐、燕,决白马之口以沃魏氏,是一举而三晋亡,从者败也。大王垂拱以须之,天下编随而服矣,霸王之名可成。而谋臣不为,引军而退,复与赵氏为和。夫以大王之明,秦兵之强,弃霸王之业,地曾不可得,乃取欺于亡国。是谋臣之拙也。且夫赵当亡而不亡,秦当霸而不霸,天下固以量秦之谋臣一矣。乃复悉士卒以攻邯郸,不能拔也,弃甲兵弩,战竦而却,天下固已量秦力二矣。军乃引而复,并于孚下,大王又并军而至,与战不能克之也,又不能反,军罢而去,天下固量秦力三矣。内者量吾谋臣,外者极吾兵力。由是观之,臣以为天下之从,几不能矣。内者,吾甲兵顿,士民病,蓄积索,田畴荒,囷仓虚;外者,天下皆比意甚固。愿大王有以虑之也。

且臣闻之曰:``战战栗栗,日慎一日,苟慎其道,天下可有。''何以知其然也?昔者纣为天子,将率天下甲兵百万,左饮于淇溪,右饮于洹溪,淇水竭而洹水不流,以与周武王为难。武王将素甲三千,战一日,而破纣之国,禽其身,据其地而有其民,天下莫伤。知伯率三国之众以攻赵襄主于晋阳,决水而灌之三月,城且拔矣,襄主钻龟筮占兆,以视利害,何国可降。乃使其臣张孟谈。于是乃潜行而出,反知伯之约,得两国之众,以攻知伯,禽其身,以复襄主之初。今秦地折长补短,方数千里,名师数十百万。秦国之号令赏罚,地形利害,天下莫如也。此与天下,可兼而有也。臣昧死愿望见大王,言所以破天下之从,举赵,亡韩,臣荆、魏,亲齐、燕,以成霸王之名,朝四邻诸侯之道。大王诚听其说,一举而天下之从不破,赵不举,韩不亡,荆、魏不臣,齐、燕不北,霸王之名不成,四邻诸侯不朝,大王斩臣以徇国,以为王谋不忠者也。

\hypertarget{header-n817}{%
\subsection{存韩}\label{header-n817}}

韩事秦三十余年,出则为扞蔽,入则为席荐。秦特出锐师取地而韩随之,怨悬于天下,功归于强秦。且夫韩入贡职,与郡县无异也。今日臣窃闻贵臣之计,举兵将伐韩。夫赵氏聚士卒,养从徒,欲赘天下之兵,明秦不弱则诸\textbackslash{}侯必灭宗庙,欲西面行其意,非一日之计也。今释赵之患,而攘内臣之韩,则天下明赵氏之计矣。

夫韩,小国也,而以应天下四击,主辱臣苦,上下相与同忧久矣。修守备,戎强敌,有蓄积,筑城池以守固。今伐韩,未可一年而灭,拔一城而退,则权轻于天下,天下摧我兵矣。韩叛,则魏应之,赵据齐以为原,如此,则以韩、魏资赵假齐以固其从,而以与争强,赵之福而秦之祸也。夫进而击赵不能取,退而攻韩弗能拔,则陷锐之卒勤于野战,负任之旅罢于内攻,则合群苦弱以敌而共二万乘,非所以亡赵之心也。均如贵臣之计,则秦必为天下兵质矣。陛下虽以金石相弊,则兼天下之日未也。

今贱臣之愚计:使人使荆,重币用事之臣,明赵之所以欺秦者;与魏质以安其心,从韩而伐赵,赵虽与齐为一,不足患也。二国事毕,则韩可以移书定也。是我一举二国有亡形,则荆、魏又必自服矣。故曰:``兵者,凶器也。''不可不审用也。以秦与赵敌衡,加以齐,今又背韩,而未有以坚荆、魏之心。夫一战而不胜,则祸构矣。计者,所以定事也,不可不察也。韩、秦强弱,在今年耳。且赵与诸侯阴谋久矣。夫一动而弱于诸侯,危事也;为计而使诸
侯有意我之心,至殆也。见二疏,非所以强于诸侯也。臣窃愿陛下之幸熟图之!攻伐而使从者闻焉,不可悔也。

诏以韩客之所上书,书言韩子之未可举,下臣斯。甚以为不然。秦之有韩,若人之有腹心之病也,虚处则然,若居湿地,著而不去,以极走,则发矣。夫韩虽臣于秦,未尝不为秦病,今若有卒报之事,韩不可信也。秦与赵为难。荆苏使齐,未知何如。以臣观之,则齐、赵之交未必以荆苏绝也;若不绝,是悉赵而应二万乘也。夫韩不服秦之义而服于强也。今专于齐、赵,则韩必为腹心之病而发矣。韩与荆有谋,诸侯应之,则秦必复见崤塞之患。

非之来也,未必不以其能存韩也为重于韩也。辩说属辞,饰非诈谋,以钓利于秦,而以韩利窥陛下。夫秦、韩之交亲,则非重矣,此自便之计也。

臣视非之言,文其淫说靡辩,才甚。臣恐陛下淫非之辩而听其盗心,因不详察事情。今以臣愚议:秦发兵而未名所伐,则韩之用事者以事秦为计矣。臣斯请往见韩王,使来入见,大王见,因内其身而勿遣,稍召其社稷之臣,以与韩人为市,则韩可深割也。因令象武发东郡之卒,窥兵于境上而未名所之,则齐人惧而从苏之计,是我兵未出而劲韩以威擒,强齐以义从矣。闻于诸侯也,赵氏破胆,荆人狐疑,必有忠计。荆人不动,魏不足患也,则诸侯可蚕食而尽,赵氏可得与敌矣。愿陛下幸察愚臣之计,无忽。

秦遂遣斯使韩也。

李斯往诏韩王,未得见,因上书曰:``昔秦、韩戮力一意,以不相侵,天下莫敢犯,如此者数世矣。前时五诸侯尝相与共伐韩,秦发兵以救之。韩居中国,地不能满千里,而所以得与诸侯班位于天下,君臣相保者,以世世相教事秦之力也。先时五诸侯共伐秦,韩反与诸侯先为雁行以向秦军于阙下矣。诸侯兵困力极,无奈何,诸侯兵罢。杜仓相秦,起兵发将以报天下之怨而先攻荆。荆令尹患之,曰:`夫韩以秦为不义,而与秦兄弟共苦天下。已又背秦,先为雁行以攻关。韩则居中国,展转不可知。'天下共割韩上地十城以谢秦,解其兵。夫韩尝一背秦而国迫地侵,兵弱至今,所以然者,听奸臣之浮说,不权事实,故虽杀戮奸臣,不能使韩复强。

今赵欲聚兵士,卒以秦为事,使人来借道,言欲伐秦,其势必先韩而后秦。且臣闻之:`唇亡则齿寒。'夫秦、韩不得无同忧,其形可见。魏欲发兵以攻韩,秦使人将使者于韩。今秦王使臣斯来而不得见,恐左右袭曩奸臣之计,使韩复有亡地之患。臣斯不得见,请归报,秦韩之交必绝矣。斯之来使,以奉秦王之欢心,愿效便计,岂陛下所以逆贱臣者邪?臣斯愿得一见,前进道愚计,退就葅戮,愿陛下有意焉。今杀臣于韩,则大王不足以强,若不听臣之计,则祸必构矣。秦发兵不留行,而韩之社稷忧矣。臣斯暴身于韩之市,则虽欲察贱臣愚忠之计,不可得已。过鄙残,国固守,鼓铎之声于耳,而乃用臣斯之计,晚矣。且夫韩之兵于天下可知也,今又背强秦。夫弃城而败军,则反掖之寇必袭城矣。城尽则聚散,则无军矣。城固守,则秦必兴兵而围王一都,道不通,则难必谋,其势不救,左右计之者不用,愿陛下熟图之。若臣斯之所言有不应事实者,愿大王幸使得毕辞于前,乃就吏诛不晚也。秦王饮食不甘,游观不乐,意专在图赵,使臣斯来言,愿得身见,因急于陛下有计也。今使臣不通,则韩之信未可知也。夫秦必释赵之患而移兵于韩,愿陛下幸复察图之,而赐臣报决。''

\hypertarget{header-n828}{%
\subsection{难言}\label{header-n828}}

臣非非难言也,所以难言者:言顺比滑泽,洋洋纚纚然,则见以
为华而不实。敦祗恭厚,鲠固慎完,则见以为掘而不伦。多言繁称,
连类比物,则见以为虚而无用。捴微说约,径省而不饰,则见以为刿
而不辩。激急亲近,探知人情,则见以为谮而不让。闳大广博,妙远不测,则见以为夸而无用。家计小谈,以具数言,则见以为陋。言而近世,辞不悖逆,则见以为贪生而谀上。言而远俗,诡躁人间,则见以为诞。捷敏辩给,繁于文采,则见以为史。殊释文学,以质信言,则见以为鄙。时称诗书,道法往古,则见以为诵。此臣非之所以难言而重患也。

故度量虽正,未必听也;义理虽全,未必用也。大王若以此不信
,则小者以为毁訾诽谤,大者患祸灾害死亡及其身。故子胥善谋而吴戮之,仲尼善说而匡围之,管夷吾实贤而鲁囚之。故此三大夫岂不贤哉?而三君不明也。上古有汤至圣也,伊尹至智也;夫至智说至圣,

然且七十说而不受,身执鼎俎为庖宰,昵近习亲,而汤乃仅知其贤而用之。故曰以至智说至圣,未必至而见受,伊尹说汤是也;以智说愚必不听,文王说纣是也。故文王说纣而纣囚之,翼侯炙,鬼侯腊,比干剖心,梅伯醢,夷吾束缚,而曹羁奔陈,伯里子道乞,傅说转鬻,孙子膑脚于魏,吴起收泣于岸门、痛西河之为秦、卒枝解于楚,公叔痤言国器、反为悖,公孙鞅奔秦,关龙逢斩,苌宏分胣,尹子阱于棘,司马子期死而浮于江,田明辜射,宓子贱、西门豹不斗而死人手,董安于死而陈于市,宰予不免于田常,范睢折胁于魏。此十数人者,皆世之仁贤忠良有道术之士也,不幸而遇悖乱闇惑之主而死,然则虽贤圣不能逃死亡避戮辱者何也?则愚者难说也,故君子不少也。且至言忤于耳而倒于心,非贤圣莫能听,愿大王熟察之也。

\hypertarget{header-n833}{%
\subsection{爱臣}\label{header-n833}}

爱臣太亲,必危其身;人臣太贵,必易主位;主妾无等,必危嫡子;兄弟不服,必危社稷;臣闻千乘之君无备,必有百乘之臣在其侧,以徒其民而倾其国;万乘之君无备,必有千乘之家在其侧,以徒其威而倾其国。是以奸臣蕃息,主道衰亡。是故诸候之博大,天子之害也;群臣之太富,君主之败也。将相之管主而隆家,此君人者所外也。万物莫如身之至贵也,位之至尊也,主威之重,主势之隆也。此四美者,不求诸外,不请于人,议之而得之矣。故曰:人主不能用其富,则终于外也。此君人者之所识也。

昔者纣之亡,周之卑,皆从诸候之博大也;晋也分也,齐之夺也,皆以群臣之太富也。夫燕、宋之所以弑其君者,皆此类也。故上比之殷周,中比之燕、宋,莫不从此术也。是故明君之蓄其臣也,尽之以法,质之以备。故不赦死,不宥刑;赦死宥刑,是谓威淫。社稷将危,国家偏威。是故大臣之禄虽大,不得藉威城市;党与虽众,不得臣士卒。故人臣处国无私朝,居军无私交,其府军不得私贷于家。此明君之所以禁其邪。是故不得四从,不载奇兵,非传非遽,载奇兵革,罪死不赦。此明君之所以备不虞者也。

\hypertarget{header-n836}{%
\subsection{主道}\label{header-n836}}

道者,万物之始,是非之纪也。是以明君守始以知万物之源,治纪以知善败之端。故虚静以待,令名自命也,令事自定也。虚则知实之情,静则知动者正。有言者自为名,有事者自为形,形名参同,君乃无事焉,归之其情。故曰:君无见其所欲,君见其所欲,臣自将雕琢;君无见其意,君见其意,臣将自表异。故曰:去好去恶,臣乃见素;去旧去智,臣乃自备。故有智而不以虑,使万物知其处;有贤而不以行,观臣下之所因;有勇而不以怒,使群臣尽其武。是故去智而有明,去贤而有功,去勇而有强。君臣守职,百官有常,因能而使之,是谓习常。故曰:寂乎其无位而处,漻乎莫得其所。明君无为于上,君臣竦惧乎下。明君之道,使智者尽其虑,而君因以断事,故君不躬于智;贤者勑其材,君因而任之,故君不躬于能;有功则君有其贤,有过则臣任其罪,故君不躬于名。是故不贤而为贤者师,不智而为智者正。臣有其劳,君有其成功,此之谓贤主之经也。

道在不可见,用在不可知君;虚静无事,以暗见疵。见而不见,闻而不闻,知而不知。知其言以往,勿变勿更,以参合阅焉。官有一人,勿令通言,则万物皆尽。函掩其迹,匿有端,下不能原;去其智,绝其能,下不能意。保吾所以往而稽同之,谨执其柄而固握之。绝其望,破其意,毋使人欲之,不谨其闭,不固其门,虎乃将在。不慎其事,不掩其情,贼乃将生。弑其主,代其所,人莫不与,故谓之虎。处其主之侧为奸臣,闻其主之忒,故谓之贼。散其党,收其余,闭其门,夺其辅,国乃无虎。大不可量,深不可测,同合刑名,审验法式,擅为者诛,国乃无贼。是故人主有五壅:臣闭其主曰壅,臣制财利曰壅,臣擅行令曰壅,臣得行义曰壅,臣得树人曰壅。臣闭其主,则主失位;臣制财利,则主失德;行令,则主失制;臣得行义,则主失明;臣得树人,则主失党。此人主之所以独擅也,非人臣之所以得操也。

人主之道,静退以为宝。不自操事而知拙与巧,不自计虑而知福与咎。是以不言而善应,不约而善增。言已应,则执其契;事已增,则操其符。符契之所合,赏罚之所生也。故群臣陈其言,君以其主授其事,事以责其功。功当其事,事当其言,则赏;功不当其事,事不当其言,则诛。明君之道,臣不得陈言而不当。是故明君之行赏也,暖乎如时雨,百姓利其泽;其行罚也,畏乎如雷霆,神圣不能解也。故明君无偷赏,无赦罚。赏偷,则功臣墯其业;赦罚,则奸臣易为非。是故诚有功,则虽疏贱必赏;诚有过,则虽近爱必诛。疏贱必赏,近爱必诛,则疏贱者不怠,而近爱者不骄也。

\hypertarget{header-n840}{%
\subsection{有度}\label{header-n840}}

国无常强,无常弱。奉法者强,则国强;奉法者弱,则国弱。荆庄王并国二十六,开地三千里;庄王之氓社稷也,而荆以亡。齐桓公并国三十,启地三千里;桓公之氓社稷也,而齐以亡。燕襄王以河为境,以蓟为国,袭涿、方城,残齐,平中山,有燕者重,无燕者轻;襄王之氓社稷也,而燕以亡。魏安釐王攻燕救赵,取地河东;攻尽陶、魏之地;加兵于齐,私平陆之都;攻韩拔管,胜于淇下;睢阳之事,荆军老而走;蔡、召陵之事,荆军破;兵四布于天下,威行于冠带之国;安釐王死而魏以亡。故有荆庄、齐桓公,则荆、齐可以霸;有燕襄、魏安釐,则燕、魏可以强。今皆亡国者,其群臣官吏皆务所以乱而不务所以治也。其国乱弱矣,又皆释国法而私其外,则是负薪而救火也,乱弱甚矣!

故当今之时,能去私曲就公法者,民安而国治;能去私行行公法者,则兵强而敌弱。故审得失有法度之制者,加以群臣之上,则主不可欺以诈伪;审得失有权衡之称者,以听远事,则主不可欺以天下之轻重。今若以誉进能,则臣离上而下比周;若以党举官,则民务交而不求用于法。故官之失能者其国乱。以誉为赏,以毁为罚也,则好赏恶罚之人,释公行,行私术,比周以相为也。忘主外交,以进其与,则其下所以为上者薄也。交众、与多,外内朋党,虽有大过,其蔽多矣。故忠臣危死于非罪,奸邪之臣安利于无功。忠臣之所以危死而不以其罪,则良臣伏矣;奸邪之臣安利不以功,则奸臣进矣。此亡之本也。若是,则群臣废庆法而行私重,轻公法矣。数至能人之门,不一至主之廷;百虑私家之便,不一图主之国。属数虽多,非所尊君也;百官虽具,非所以任国也。然则主有人主之名,而实托于群臣之家也。故臣曰:亡国之廷无人焉。廷无人者,非朝廷之衰也;家务相益,不务厚国;大臣务相尊,而不务尊君;小臣奉禄养交,不以官为事。此其所以然者,由主之不上断于法,而信下为之也。故明主使法择人,不自举也;使法量功,不自度也。能者不可弊,败者不可饰,誉者不能进,非者弗能退,则君臣之间明辩而易治,故主仇法则可也。

贤者之为人臣,北面委质,无有二心。朝廷不敢辞贱,军旅不敢辞难;顺上之为,从主之法,虚心以待令,而无是非也。故有口不以私言,有目不以私视,而上尽制之。为人臣者,譬之若手,上以修头,下以修足;清暖寒热,不得不救;镆铘传体,不敢弗搏慼,无私贤哲之臣,无私事能之士。故民不越乡而交,无百里之感。贵贱不相逾,愚智提衡而立,治之至也。今夫轻爵禄,易去亡,以择其主,臣不谓廉。诈说逆法,倍主强谏,臣不谓忠。行惠施利,收下为名,臣不谓仁。离俗隐居,而以诈非上,臣不谓义。外使诸候,内耗其国,伺其危险之陂,以恐其主曰;"交非我不亲,怨非我不解"。而主乃信之,以国听之。卑主之名以显其身,毁国之厚以利其家,臣不谓智。此数物者,险世之说也,而先王之法所简也。先王之法曰:"臣毋或作威,毋或作利,从王之指;无或作恶,从王之路。"古者世治之民,奉公法,废私术,专意一行,具以待任。"

夫为人主而身察百官,则日不足,力不给。且上用目,则下饰观;上用耳,则下饰声;上用虑,则下繁辞。先王以三者为不足,故舍己能而因法数,审赏罚。先王之所守要,故法省而不侵。独制四海之内,聪智不得用其诈,险躁不得关其佞,奸邪无所依。远在千里外,不敢易其辞;势在郎中,不敢蔽善饰非;朝廷群下,直凑单微,不敢相逾越。故治不足而日有馀,上之任势使然之。

夫人臣之侵其主也,如地形焉,即渐以往,使人主失端,东西易面而不自知。故先王立司南以端朝夕。故明主使其群臣不游意于法之外,不为惠于法之内,动无非法。峻法,所以凌过游外私也;严刑,所以遂令惩下也。威不贰错,制不共门。威、制共,则众邪彰矣;法不信,则君行危矣;刑不断,则邪不胜矣。故曰:巧匠目意中绳,然必先以规矩为度;上智捷举中事,必以先王之法为比。故绳直而枉木断,准夷而高科削,权衡县而重益轻,斗石设而多益少。故以法治国,举措而已矣。法不阿贵,绳不挠曲。法之所加,智者弗能辞,勇者弗敢争。刑过不辟大臣,赏善不遗匹夫。故矫上之失,诘下之邪,治乱决缪,绌羡齐非,一民之轨,莫如法。厉官威名,退淫殆,止诈伪,莫如刑。刑重,则不敢以贵易贱;法审,则上尊而不侵。上尊而不侵,则主强而守要,故先王贵之而传之。人主释法用私,则上下不别矣。

\hypertarget{header-n846}{%
\subsection{二柄}\label{header-n846}}

明主之所导制其臣者,二柄而已矣。二柄者,刑德也。何谓刑德?曰:杀戮之谓刑,庆赏之谓德。为人臣者畏诛罚而利庆赏,故人主自用其刑德,则群臣畏其威而归其利矣。故世之奸臣则不然,所恶,则能得之其主而罪之;所爱,则能得之其主而赏之;今人主非使赏罚之威利出于已也,听其臣而行其赏罚,则一国之人皆畏其臣而易其君,归其臣而去其君矣。此人主失刑德之患也。夫虎之所以能服狗者,爪牙也。使虎释其爪牙而使狗用之,则虎反服于狗矣。人主者,以刑德制臣者也。今君人者释其刑德而使臣用之,则君反制于臣矣。故田常上请爵禄而行之群臣,下大斗斛而施于百姓,此简公失德而田常用之也,故简公见弑。子罕谓宋君曰:"夫庆赏赐予者,民之所喜也,君自行之;杀戮刑罚者,民之所恶也,臣请当之。"于是宋君失刑百子罕用之,故宋君见劫。田常徒用德而简公弑,子罕徒用刑而宋君劫。故今世为人臣者兼刑德而用之,则是世主之危甚于简公、宋君也。故劫杀拥蔽之,主非失刑德而使臣用之,而不危亡者,则未尝有也。

人主将欲禁奸,则审合刑名者,言异事也。为人臣者陈而言,君以其言授之事,专以其事责其功。功当其事,事当其言,则赏;功不当其事,事不当其言,则罚。故群臣其言大而功小者则罚,非罚小功也,罚功不当名也;群臣其言小而功大者亦罚,非不说于大功也,以为不当名也害甚于有大功,故罚。昔者韩昭候醉而寝,典冠者见君之寒也,故加衣于君之上,觉寝而说,问左右曰:"谁加衣者?"左右对曰:"典冠。"君因兼罪典衣与典冠。其罪典衣,以为失其事也;其罪典冠,以为越其职也。非不恶寒也,以为侵官之害甚于寒。故明主之畜臣,臣不得越官而有功,不得陈言而不当。越官则死,不当则罪。守业其官,所言者贞也,则群臣不得朋党相为矣。

人主有二患:任贤,则臣将乘于贤以劫其君;妄举,则事沮不胜。故人主好贤,则群臣饰行以要群欲,则是群臣之情不效;群臣之情不效,则人主无以异其臣矣。故越王好勇而民多轻死;楚灵王好细腰而国中多饿人;齐桓公妒外而好内,故竖刁自宫以治内;桓公好味,易牙蒸其子首而进之;燕子哙好贤,故子之明不受国。故君见恶,则群臣匿端;君见好,则群臣诬能。人主欲见,则群臣之情态得其资矣。故子之托于贤以夺其君者也,竖刁、易牙,因君之欲以侵其君者也。其卒,子哙以乱死,桓公虫流出户而不葬。此其故何也?人君以情借臣之患也。人臣之情非必能爱其君也,为重利之故也。今人主不掩其情,不匿其端,而使人臣有缘以侵其主,则群臣为子之、田常不难矣。故曰:"去好去恶,群臣见素。"群臣见素,则大君大蔽矣。

\hypertarget{header-n850}{%
\subsection{扬权}\label{header-n850}}

天有大命,人有大命。夫香美脆味,厚酒肥肉,甘口而疾形;曼理皓齿,说情而捐精。故去甚去泰,身乃无害。权不欲见,素无为也。事在四方,要在中央。圣人执要,四方来效。虚而待之,彼自以之。四海既藏,道阴见阳。左右既立,开门而当。勿变勿易,与二俱行。行之不已,是谓履理也。

夫物者有所宜,材者有所施,各处其宜,故上下无为。使鸡司夜,令狸执鼠,皆用其能,上乃无事。上有所长,事乃不方。矜而好能,下之所欺:辩惠好生,下因其材。上下易用,国故不治。

用一之道,以名为首,名正物定,名倚物徒。故圣人执一以静,使名自命,令事自定。不见其采,下故素正。因而任之,使自事之;因而予之,彼将自举之;正与处之,使皆自定之。上以名举之,不知其名,复修其形。形名参同,用其所生。二者诚信,下乃贡情。

谨修所事,待命于天,毋失其要,乃为圣人。圣人之道,去智与巧。智巧不去,难以为常。民人用之,其身多殃;主上用之,其国危亡。因天之道,反形之理,督参鞠之,终则有始。虚以静后,未尝用己。凡上之患,必同其端;信而勿同,万民一从。

夫道者,弘大而无形;德者,核理而普至。至于群生,斟酌用之,万物皆盛,而不与其宁。道者,下周于事,因稽而命,与时生死。参名异事,通一同情。故曰:道不同于万物,德不同于阴阳,衡不同于轻重,绳不同于出入,和不同于燥湿,君不同于群臣。--凡此六者,道之出也。道无双,故曰一。是故明君贵独道之容。君臣不同道,下以名祷。君操其名,臣效其形,形名参同,上下和调也。

凡听之道,以其所出,反以为之入。故审名以定位,明分以辩类。听言之道,溶若甚醉。脣乎齿乎,吾不为始乎;齿乎脣乎,愈惛々乎。彼自离之,吾因以知之;是非辐凑,上不与构。虚静无为,道之情也;叁伍比物,事之形也。叁之以比物,伍之以合虚。根干不革,则动泄不失矣。动之溶之,无为而攻之。喜之,则多事;恶之,则生怨。故去喜去恶,虚心以为道舍。上不与共之,民乃宠之;上不与义之,使独为之。上固闭内扃,从室视庭,咫尺已具,皆之其处。以赏者赏,以刑者刑,因其所为,各以自成。善恶必及,孰敢不信?规矩既设,三隅乃列。

主上不神,下将有因;其事不当,下考其常。若天若地,是谓累解;若地若天,孰疏孰亲?能象天地,是谓圣人。欲治其内,置而勿亲;欲治其外,宫置一人;不使自恣,安得移并?大臣之门,唯恐多人。凡治之极,下不能得。周合刑名,民乃守职;去此更求,是谓大惑。猾民愈众,奸邪满侧。故曰:毋富人而贷焉,毋贵人而逼焉;毋专信一人而失其都国焉;腓大于股,难以趣走。主失其神,虎随其后。主上不知,虎将为狗。主不蚤止,狗益无已。虎成其群,以弑其母。为主而无臣,奚国之有?主施其法,大虎将怯;主施其刑,大虎自宁。法制苟信,虎化为人,复反其真。

欲为其国,必伐其聚;不伐其聚,彼将聚众。欲为其地,必适其赐;不适其赐,乱人求益。彼求我予,假仇人斧;假之不可,彼将用之以伐我。黄帝有言曰:"上下一日百战。"下匿其私,用试其上;上操度量,以割其下。故度量之立,主之宝也;党与之具,臣之宝也。臣之所不弑其君者,党与不具也。故上失扶寸,下得寻常。有国君,不大其都;有道之臣,不贵其家。有道之君,不贵其臣;贵之富之,彼将代之。备危恐殆,急置太子,祸乃无从起。内索出圉,必身自执其度量。厚者亏之,薄者靡之。亏靡有量,毋使民比周,同欺其上。亏之若月,靡之若热。简令谨诛,必尽其罚。

毋弛而弓,一栖两雄,其斗颜(?左加口字旁)颜,豺狼在牢,其羊不繁。一家二贵,事乃无功。夫妻持政,子无适从。为人君者,数披其木,毋使木技扶疏;木枝扶疏,将塞公闾,私门将实,公庭将虚,主将壅围。数披其木,无使木枝外拒;木枝外拒,将逼主处。数披其木,毋使枝大本小;枝大本小,将不胜春风;不胜春风,枝将害心。公子既众,宗室忧唫。止之之道,数披其木,毋使枝茂。木数披,党与乃离。掘其根本,木乃不神。填其汹渊,毋使水清。探其怀,夺之威。主上用之,若电若雷。

\hypertarget{header-n861}{%
\subsection{八奸}\label{header-n861}}

凡人臣之所道成奸者有八术:一曰同床,二曰在旁,三曰父兄,四曰养殃,五曰民萌,六曰流行,七曰威强,八曰四方。

何谓同床?曰:贵夫人,爱孺子,便僻好色,此人主之所惑也。托于燕处之虞,乘醉饱之时,而求其所欲,此必听之术也。为人臣者内事之以金玉,使惑其主,此之谓"同床"。二曰在旁。何谓在谤?曰:优笑侏儒,左右近习,此人主未命而唯唯,未使而诺诺,先意承旨,观貌察色以先主心者也。此皆俱进俱退,皆应皆对,一辞同轨以移主心者也。为人臣者内事之以金玉玩好,外为之行不法,使之化其主,此之谓"在旁"。三曰父兄。何谓父兄?曰:侧室公子,人主之所亲爱也;大臣廷吏,人主之所与度计也。此皆尽力毕议,人主之所必听也。为人臣者事公子侧室以音声子女,收大臣延吏以辞言,处约言事,事成则进爵益禄,以劝其心,犯其主,此之谓"父兄"。四曰养殃。何谓养殃?曰:人主乐美宫室台池,好饰子女狗马以娱其心,此人主之殃也。为人臣者尽民力以美宫室台池,重赋敛以饰子女狗马,以娱其主而乱其心,从其所欲,而树私利其间,此谓"养殃"。五曰民萌。何谓民萌?曰:为人臣者散公财以说民人,行小惠以取百姓,使朝廷市井皆劝权誉己,以塞其主而成其所欲,此之谓"民萌"。六曰流行。何谓流行?曰:人主者,固壅其言谈,希于听论议,易移以辩说。为人臣者求诸候之辩士,养国中之能说者,使之以语其私。为巧文之言,流行之辞,示之以利势,惧之以患害,施属虚辞以坏其主,此之谓"流行"。七曰威强。何谓威强?曰:君人者,以群臣百姓为威强者也。群臣百姓之所善,则君善之;非群臣百姓之所善,则君不善之。为人臣者,聚带剑之客,养必死之士,以彰其威,明焉己者必利,不为己者必死,以恐其群臣百姓而行其私,此之谓"威强"。八曰四方。何谓四方?曰:君人者,国小,则事大国;兵弱,则畏强兵。大国之所索,小国必听;强兵之所加,弱兵必服。为人臣者,重赋敛,尽府库,虚其国以事大国,而用其威求诱其君;甚者举兵以聚边境而制敛于内,薄者数内大使以震其君,使之恐惧,此之谓"四方"。凡此八者,人臣之所以道成奸,世主所以壅劫,失其所有也,不可不察焉。

明君之于内也,娱其色而不行其谒,不使私请。其于左右也,使其身必责其言,不使益辞。其于父兄大臣也,听其言也必使以罚任于后,不令妄举。其于观东玩好也,必令之有所出,不使擅进擅退,不使群臣虞其意。其于德施也,纵禁财,发坟仓,利于民者,必出于君,不使人臣私其德。其于说议也,称誉者所善,毁疵者所恶,必实其能,察其过,不使群臣相为语。其于勇力之士也,军旅之功无逾赏,邑斗之勇无赦罪,不使群臣行私财。其于诸候之求索也,法则听之,不法则距之。则谓亡君者,非莫有其国也,而有之者,皆非己有也。令臣以外为制于内,则是君人者亡也。听大国为救亡也,而亡亟于不听,故不听。群臣知不听,则不外诸候,诸候知不听,则不受臣之诬其君矣。

明主之为官职爵禄也,所以进贤材劝有功也。故曰:贤材者处厚禄任大官;功大者有尊爵受重赏。官贤者量其能,赋禄者称其功。是以贤者不诬能以事其主,有功者乐进其业,故事成功立,今则不然,不课贤不肖,不论有功劳,用诸候之重,听左右之谒,父兄大臣上请爵禄于上,而下卖之以收财利及以树私党。故财利多者买官以为贵,有左右之交者请谒以成重。功劳之臣不论,官职之迁失谬。是以吏偷官而外交,弃事而亲财。是以贤者懈怠而不劝,有功者隳而简其业,此亡国之风也。

\hypertarget{header-n867}{%
\subsection{十过}\label{header-n867}}

十过:一曰行小忠,则大忠之贼也。二曰顾小利,则大利之残也。三曰行僻自用,无礼诸候,则亡身之至也。四曰不务听治而好五音,则穷身之事也。五曰贪愎喜利,则灭国杀身之本也。六曰耽于女乐,不顾国政,则亡国之祸也。七曰离内远游而忽于谏士,则危身之道也。八曰过而不听于忠臣,而独行其意,则灭高名为人笑之始也。九曰内不量力,外恃诸候,则削国之患也。十曰国小无礼,不用谏臣,则绝世之势也。

奚谓小忠?昔者楚共王与晋厉公战于鄢陵,楚师败,而共王伤其目。酣战之时,司马之反渴而求饮,竖谷阳操觞酒而进之。子反曰:"嘻!,退,酒也。"阳曰:"非酒也。"子反受而饮之。子反之为人也,嗜酒,而甘之,弗能绝于口,而醉。战既罢,共王欲战,令人召司马子反,司马子反辞以心疾。共王驾而自往,入其幄中,闻酒臭而还,曰:"今日之战,不谷亲伤。所恃者,司马也,而司马又醉如此,是亡楚国之社稷而不恤吾众也。不谷无复战矣。"于是还师而去,斩司马子反以为大戮。故竖阳之进酒,不以仇子反也,其心忠爱之而适足以杀之。故曰:行小忠,则大忠之贼也。

奚谓顾小利?昔者晋献公欲假道于虞以伐虢。荀息曰:"君其以垂棘之璧与屈产之乘,赂虞公,求假道焉,必假我道。"君曰:"垂棘之璧,吾先君之宝也;屈产之乘,寡人之骏马也。若受吾币不假之道,将奈何?"荀息曰:"彼不假我道,必不敢受我。若受我,而假我道,则是宝犹取之内府而藏之外府也,马犹取之内厩而著之外厩也。君勿尤。"君曰:"诺。"乃使荀息以垂棘之璧与屈产之乘赂虞公而求假道焉。虞公贪利其璧与马而欲许之。宫之奇谏曰:"不可许。夫虞之有虢也,如车之有辅。辅依车,车亦依辅,虞、虢之势正是也。若假之道,则虢朝亡而虞夕从之矣。不可,愿勿许。"虞公弗听,逐假之道。荀息伐虢克之,还反处三年,与兵伐虞,又克之。荀息牵马操璧而报献公,献公说曰:"璧则犹是也。虽然,马齿亦益长矣。"故虞公之兵殆而地削者,何也?爱小利而不虑其害。故曰:顾小利,则大利之残也。

奚谓行僻?昔者楚灵王为申之会,宋太子后至,执而囚之;狎徐君;拘齐庆封。中射士谏曰:"合诸候不可无礼,此存亡之机也。昔者桀为有戎之会而有纟昏叛之,纣为黎丘之蒐而戎狄叛之,由无礼也。君其图之。"君不听,遂行其其意。居未期年,灵王南游,群臣从而劫之。灵王饿而死乾溪之上。故曰:行僻自用,无礼诸候,则亡身之至也。

奚谓好音?昔者卫灵公将之晋,至濮水之上,税车而放马,设舍以宿。夜分,而闻鼓新声者而说之。他人问左右,尽报弗闻。乃召师涓而告之,曰:"有鼓新声者,使人问左右,尽报弗闻。其状似鬼神,子为我听而写之。"师涓曰:"诺。因静坐抚琴而写之。师涓明日报曰:"臣得之矣,而未习也,请复一宿习之。"灵公曰:"诺。"因复留宿。明日而习之,遂去之晋。30晋平公觞之于施夷之台。酒酣,灵公起"。公曰:"有新声,愿请以示。"平公曰:"善"。"乃召师涓,令坐师旷之旁,援琴鼓之。未终,师旷抚止之,曰:"此亡国之声,不可遂也。"平公曰:"此道奚出?"师旷曰:"此师延之所作,与纣为靡靡之也。及武王伐纣,师延东走,至于濮水而自投。故闻此声者,必于水之上。先闻此声者,其国必削,不可遂。"平公曰:"寡人所好者,音也,子其使遂之。"师涓鼓动究之。平公问师旷曰:"此所谓何声也?"师旷曰:"此所谓清商也。"公曰:"清商固最悲乎?"师旷曰:"不如清徵。"公曰:"清徵可得而闻乎?"师旷曰:"不可。古之听清徵者,皆有德义之君也。今吾君德薄,不足以听。"平公曰:"寡人之所好者,音也,愿试听之。"师旷不得已,援琴而鼓。一奏之,有玄鹤二八,道南方来,集于郎门之垝;再奏之,而列。三奏之,延颈而鸣,舒翼而舞,音中宫商之声,声闻于天。平公大说,坐者皆喜。平公提觞而起为师旷寿,反坐而问曰:"音莫悲于清徵乎?"师旷曰:"不如清角。"平公曰:"清角可得而闻乎?"师旷曰:"不可。昔者黄帝合鬼神于泰山之上,驾象车而六蛟龙,毕方并鎋,蚩尤居前,风伯进扫,雨师洒道,虎狼在前,鬼神在后,腾蛇伏地,凤皇覆上,大合鬼神,作为清角。今吾君德薄,不足听之。听之,将恐有败。"平公曰:"寡人老矣,所好者音也,愿遂听之。"师旷不得已而鼓之。一奏之,有玄云从西北方起;再奏之,大风至,大雨随之,裂帷幕,破俎豆,隳廊瓦。坐者散走,平公恐惧伏于廊室之间。晋国大旱,赤地三年。平公之身遂癃病。故曰:不务听治,而好五音不已,则穷身之事也。

奚谓贪愎?昔者智伯瑶率赵、韩、魏而伐范、中行,灭之。反归,休兵数年。因令人请地于韩。韩康子欲勿与,段规谏曰:"不可不与也。夫知伯之为人也,好利而骜愎。彼来请地而弗与,则移兵于韩必矣。君其与之。与之彼狃,又将请地他国。他国且有不听,不听,则知伯必加之兵。如是,韩可以免于患而待其事之变。"康子曰:"诺。"因令使者致万家之县一于知铁。知伯说,又令人请地于魏。宣子欲勿与,赵葭谏曰:"彼请地于韩,韩与之。今请地于魏,魏弗与,则是魏内自强,而外怒知伯也。如弗予,其措兵于魏必矣。不如予之。"宣子曰:"诺。"因令人致万家之县一于知伯。知伯又令人之赵请蔡,皋狼之地,赵襄子弗与。知伯因阴约韩、魏将以伐赵。襄子召张孟谈而告之曰:"夫知伯之为人也,阳亲而阴疏。三使韩、魏而寡人不与焉,其措兵于寡人必矣。今吾安居而可?"张孟谈曰:"夫董阏于,简主之才臣也,其治晋阳,而尹铎循之,其余教犹存,君其定居晋阳而已矣。"君是曰:"诺。"乃召延陵生,令将车骑先至晋阳,君因从之。君至,而行其城郭及五官之藏。城郭不治,仓无积粟,府无储钱,库无甲兵,邑无守具。襄子惧,乃召张孟谈曰:"寡人行城郭及五官之藏,皆不备具,吾将何以应敌。?"张孟谈曰:"臣闻圣人之治,藏于民,不藏于府库,务修其教,不治城郭。君其出令,令民自遗三年之食,有馀粟者入之仓;遗三年之用,有馀钱者入之府;遗有奇人者使治城郭之缮。"君夕出令,明日,仓不容粟,府无积钱。库不受甲兵。居五日而城郭已治,守备已具。君召张孟谈而问之曰:"吾城郭已治,守备已具。钱粟已足,甲兵有馀。吾奈无箭何?"张孟谈曰:"臣闻董子之治晋阳也,公宫之垣皆以荻蒿楛楚墙之,其楛高至于丈,君发而用之。"于是发而试之,其坚则虽簵之劲弗能过也。君曰:"箭已足矣,奈无金何?"张孟谈曰:"臣闻董子之治晋阳也,公宫令舍之堂,皆以炼铜为柱质。君发而用之。"于是发而用之,有余金矣。号令已定,守备已具。三国之兵果至。至则乘晋阳之城,遂战。三月弗能拔。因舒军而围之,决晋阳之水以灌之。围晋阳三年。城中巢居而处,悬釜而炊,财食将尽,士大夫羸病。襄子谓张孟谈曰:"粮食匮,财力尽,士大夫羸病,吾恐不能守矣!欲以城下,何国之可下?"张孟谈曰:"臣闻之:'亡弗能存,危弗能安,则无为贵智矣。'君释此计者。臣请试潜行而出,见韩、魏之君。"张孟谈见韩、魏之君曰:"臣闻:'亡齿寒。'今知伯率二君而伐赵,赵将亡矣。赵亡,则二君为之次。"二君曰:"我知其然也。虽然,知伯之为人也中,粗而少亲。我谋而觉,则其祸必至矣。为之奈何?"张孟谈曰:"谋出二君之口而入臣之耳,人莫之知也。"二君因与张孟谈约三军之反,与之期日。夜遣孟谈入晋阳,以报二君之反。襄子迎孟谈而再拜之,且恐且喜。二君以约遣张孟谈,因朝知伯而出,遇智过于辕门之外。智过怪其色,因入见知伯曰:"二君貌将有变。"君曰:"何如?"曰:"其行矜而意高,非他时节也,君不如先之。"君曰:"吾与二主约谨矣,破赵而三分其地,寡人所以亲之,必不侵欺。兵之著于晋阳三年,今旦暮将拔之而飨其利,何乃将有他心?必不然。子释勿忧,勿出于口。"明旦,二主又朝而出,复见智过于辕门。智过入见曰:"君以臣之言告二主乎?"君曰:"何以知之?"曰:"今日二主朝而出,见臣而其色动,而视属臣。此必有变,君不如杀之。"君曰:"子置勿复言。"智过曰:"不可,必杀之。若不能杀,遂亲之。"君曰;"亲之奈何?"智过曰:"魏宣子谋臣曰赵葭,韩康子之谋臣曰段规,此皆能移其君之计。君与其二君约:破赵国,因封二子者各万家之县一。如是,则二主之心可以无变矣。"知伯曰:"破赵而三分其地,又封二子者各万家之县一,则吾所得者少。不可。"智过见其言之不听也,出,因更其族为辅氏。至于期日之夜,赵氏杀其守堤之吏而决其水灌知伯军。知伯军救水而乱,韩、魏翼而击之,襄子将卒犯其前,大败知伯之军而擒知伯。知伯身死军破,国分为三,为天下笑。故曰:贪愎好利,则灭国杀身之本也。

奚谓耽于女乐?昔者戎王使由余聘于秦,穆公问之曰:"寡人尝闻道而未得目见之也,原闻古之明主得国失国常何以?"由余对曰:"臣尝得闻之矣,常以俭得之,以奢失之。"穆公曰:"寡人不辱而问道于子,子以俭对寡人何也?"由余对曰:"臣闻昔者尧有天下,饭于土簋,饮于土铏。其地南至交趾,北至"幽都,东西至日月所出入者,莫不实服。尧禅天下,虞舜受之,作为食器,斩山木而财子,削锯修其迹,流漆墨其上,输之于宫以为食器。诸候以为益侈,国之不服者十三。舜禅天下而传之于禹,禹作为祭器,墨染其外,而硃画书其内,缦帛为茵,将席颇缘,触酌有采,而樽俎有饰。此弥侈矣,而国之不服者三十三。夏后氏没,殷人受之,作为大路,而建旒九,食器雕琢,觞酌刻镂,白壁垩墀,茵席雕文。此弥侈矣,而国之不服者五十三。君子皆知文章矣,而欲服者弥少。臣故曰:俭其道也。"由余出,公乃召内史廖而告之,曰:"寡人:'闻邻国有圣人,敌国之忧也。'今由余,圣人也,寡人患之,吾将余何?"内史廖曰:"臣闻戎王之居,僻陋而道远,未闻中国之声。君其遣之女乐,以乱其政,而后为由余请期,以疏其谏。彼君臣有间而后可图也。"君曰:"诺。"乃使内史廖以女乐二八遣戎王,因为由余请期。戎王许诺,见其女乐而说之,设酒张饮,日以听乐,终几不迁,牛马半死。由余归,因谏戎王,戎王弗听,由余遂去之秦。秦穆公迎而拜之上卿,问其兵势与其地形。既以得之,举兵而伐之,兼国十二,开地千里。故曰:耽于女乐,不顾国政,则亡国之祸也。

奚谓离内远游?昔者齐景公游于海而乐之。号令诸大夫曰:"言归者死。"颜涿聚曰:"君游海而乐之,奈臣有图国者何?君虽乐之,将安得。"齐景公曰:"寡人布令曰'言归者死',今子犯寡人之令。"援戈将击之。颜涿聚曰:"昔桀杀关龙逢而纣杀王子比干,今君虽杀臣之身以三之可也。臣言为国,非为身也。"延颈而前曰:"君击之矣!"君乃释戈趣驾而归。至三日,而闻国人有谋不内齐景公者矣。齐景公所以遂有齐国者,颜涿聚之力地。故曰:离内远游,则危身之道也。

奚谓过而不听于忠臣?昔者齐桓公九合诸候,一匡天下,为五伯长,管仲佐之。管仲老,不能用事,休居于家。桓公从而问之曰:"仲父家居有病,即不幸而不起此病,政安迁之?"管仲曰:"臣老矣,不可问也。虽然,臣闻之,知臣莫若君,知子莫若父。君其试以心决之。"君曰:"鲍叔牙何如?"管仲曰:"不可。鲍叔牙为人,刚愎而上悍。刚则犯民以暴,愎则不得民心,悍则下不为用。其心不惧,非霸者之佐也。"公曰:"然则竖刁何如?"管仲曰:"不可。夫人之情莫不爱其身。公妒而好内,竖刁自獖以为治内。其身不爱,又安能爱君?"公曰:"然,则术公子开方何如?"管仲曰:"不可。齐、卫之间不过十日之行,开方为事君,欲适君之故,十五年不归见其父母,此非人情也。其父母之不亲也,又能亲君乎?"公曰:"然则易牙何?"管仲曰:"不可。夫易牙为君主味。君之所未尝食唯人肉耳,易牙蒸其子首而进之,君所知也。人之情莫不爱其子,今蒸其子以为膳于君,其子弗爱,又安能爱君乎?"公曰:"然则孰可?"管仲曰:"隰朋可。其为人也,坚中而廉外,少欲而多信。夫坚中,则足以为表;廉外,则可以大任;少欲,则能临其众;多信,则能亲邻国。此霸者之佐也,君其用之。"君曰:"诺。"居一年馀,管钟死,君遂不用隰朋而与竖刁。刁莅事三年,桓公南游堂阜,竖刁率易牙、卫公子开方及大臣为乱。桓公渴馁而死南门之寝、公守之室,身死三月不收,虫出于户。故桓公之兵横行天下,为五伯长,卒见弑于其臣,而灭高名,为天下笑者,何也?不用管仲之过也。故曰:过而不听于忠臣,独行其意,则灭其高名为人笑之始也。

奚谓内不量力?昔者秦之攻宜阳,韩氏急。公仲朋谓韩君曰:"与国不可恃也,岂如因张仪为和于秦哉!因赂以名都而南与伐楚,是患解于秦而害交于楚也。"公曰:"善。"乃警公仲之行,将西和秦。楚王闻之,惧,召陈轸而告之曰:"韩朋将西和秦,今将奈何?"陈轸曰:"秦得韩之都一,驱其练甲,秦、韩为一以南乡楚,此秦王之所以庙祠而求也,其为楚害必矣。王其趣发信臣,多其车,重其币,以奉韩曰:\textbackslash{}'不谷之国虽小,卒已悉起,愿大国之信意于秦也。因愿大国令使者入境视楚之起卒也。\textbackslash{}'"韩使人之楚,楚王因发车骑,陈之下路,谓韩使者曰:"报韩君,言弊邑之兵今将入境矣。"使者还报韩君,韩君大大悦,止公仲。公仲曰:"不可。夫以实害我者,秦也;以名救我者,楚也。听楚之虚言而轻强秦之实祸,则危国之本也。"韩君弗听。公仲怒而归,十日不朝。宜阳益急,韩君令使者趣卒于楚,冠盖相望而卒无至者。宜阳果拔,为诸候笑。故曰:内不量力,外恃诸候者,则国削之患也。

奚谓国小无礼?昔者晋公子重耳出亡,过于曹,曹君袒裼而观之。釐负羁与叔瞻侍于前。

叔瞻谓曹君曰:"臣观晋公子,非常人也。君遇之无礼,彼若有时反国而起兵,即恐为曹伤,君不如杀之。"曹君弗听。釐负羁归而不乐,其妻问之曰:"公从外来而有不乐之色,何也?"负羁曰:"吾闻之,有福不及,祸来连我。今日吾君召晋公子,其遇之无礼。我与在前,吾是以不乐。"其妻曰:"吾观晋公子,万乘之主也;其左右从者,万乘之相也。今穷而出亡过于曹,曹遇之无礼。此若反国,必诛无礼,则曹其首也。子奚不先自贰焉。"负羁曰:"诺。"盛黄金于壶,充之以餐,加璧其上,夜令人遗公子。公子见使者,再拜,受其餐而辞其璧。

公子自曹入楚,自楚入秦。入秦三年,秦穆公召群臣而谋曰:"昔者晋献公与寡人交,诸候莫弗闻。献公不幸离群臣,出入十年矣。嗣子不善,吾恐此将仿令其宗庙不祓阴而社稷不血食也。如是弗定,则非与人交之道。吾欲辅重耳而入之晋,何如?"群臣皆曰:"善。"公因起卒,革车五百乘,畴骑二千,步卒五万,辅重耳入之于晋,立为晋君。重耳即位三年,举兵而伐曹矣。因令人告曹君曰:"悬叔瞻而出之,我且杀而以为大戮。"又令人告釐负羁曰:"军旅薄城,吾知子不违也。其表子之闾,寡人将以为令,令军勿敢犯。"曹人闻之,率其亲戚而保釐负羁之闾者七百馀家。此礼之所用也。故曹,小国也,而迫于晋、楚之间,其君之危犹累卵也,而以无礼莅之,此所以绝世也。故曰:国小无礼,不用谏臣,则绝世之势也。

\hypertarget{header-n882}{%
\subsection{孤愤}\label{header-n882}}

智术之士,必远见而明察,不明察,不能烛私;能法之士,必强毅而劲直,不劲直,不能矫奸。人臣循令而从事,案法而治官,非谓重人也。重人也者,无令而擅为,亏法以利私,耗国以便家,力能得其君,此所为重人也。智术之士明察,听用,且烛重人之阴情;能法之直到劲直,听用,矫重人之奸行。故智术能法之士用,则贵重之臣必在绳之外矣。是智法之士与当涂之人,不可两存之仇也。

当涂之人擅事要,则外内为之用矣。是以诸候不因,则事不应,故敌国为之讼;百官不因,则业不进,故群臣为之用;郎中不因,则不得近主,故左右为之匿;学士不因,则养禄薄礼卑,故学士为之谈也。此四助者,邪臣之所以自饰也。重人不能忠主而进其仇,人主不能越四助而烛察其臣,故人主愈弊而大臣愈重。

凡当涂者之于人主也,希不信爱也,又且习故。若夫即主心,同乎好恶,因其所自进也。官爵贵重,朋党又众,而一国为之讼。则法术之士欲干上者,非有所信爱之亲,习故之泽也,又将以法术之言矫人主阿辟之心,是与人主相反也。处势卑贱,无党孤特。夫以疏远与近爱信争,其数不胜也;以新旅与习故争,其数不胜也;以反主意与同好恶争,其数不胜也;以轻贱与贵重争,其数不胜也;以一口与一国争,其数不胜也。法术之士操五不胜之势,以发数而又不得见;当涂之人乘五胜之资,而旦暮独说于前。故法术之士奚道得进,而人主奚时得悟乎?故资必不胜而势不两存,法术之士焉得不危?其可以罪过诬者,以公法而诛之;其不可被以罪过者,以私剑而穷之。是明法术而逆主上者,不戮于吏诛,必死于私剑矣。朋党比周以弊主,言曲以使私者,必信于重人矣。故其可以攻伐借者,以官爵贵之;其不可借以美名者,以外权重之之。是以弊主上而趋于私门者,不显于官爵,必重于外权矣。今人主不合参验而行诛,不待见功而爵禄,故法术之士安能蒙死亡而进其说?奸邪之臣安肯乘利而退其身?故主上愈卑,私门益尊。

夫越虽国富兵强,中国之主皆知无益于己也,曰:"非吾所得制也。"今有国者虽地广人众,然而人主壅蔽,大臣专权,是国为越也。智不类越,而不智不类其国,不察其类者也。人之所以谓齐亡者,非地与城亡也,吕氏弗制而田氏用之;所以谓晋亡者,亦非地与城亡也,姬氏不制而六卿专之也。今大臣执柄独断,而上弗知收,是人主不明也。与死人同病者,不可生也;与亡国同事者,不可存也。今袭迹于齐、晋,欲国安存,不可得也。

凡法术之难行也,不独万乘,千乘亦然。人主之左右不必智也,人主于人有所智而听之,因与左右论其言,是与愚人论智也;人主之左右不必贤也,人主于人有所贤而礼之,因与左右论其行,是与不肖论贤也。智者决策于愚人,贤士程行于不肖,则贤智之士羞而人主之论悖矣。人臣之欲得官者,其修士且以精洁固身,其智士且以治辩进业。其修士不能以货赂事人,恃其精洁而更不能以枉法为治,则修智之士不事左右、不听请谒矣。人主之左右,行非伯夷也,求索不得,货赂不至,则精辩之功息,而毁诬之言起矣。治辩之功制于近习,精洁之行决于毁誉,则修智之吏废,则人主之明塞矣。不以功伐决智行,不以叁伍审罪过,而听左右近习之言,则无能之士在廷,而愚污之吏处官矣。

万乘之患,大臣太重;千乘之患,左右太信;此人主之所公患也。且人臣有大罪,人主有大失,臣主之利与相异者也。何以明之哉?曰:主利在有能而任官,臣利在无能而得事;主利在有劳而爵禄,臣利在无功而富贵;主利在豪杰使能,臣利在朋党用私。是以国地削而私家富,主上卑而大臣重。故主失势而臣得国,主更称蕃臣,而相室剖符。此人臣之所以谲主便私也。故当也之重臣,主变势而得固宠者,十无二三。是其故何也?人臣之罪大也。臣有大罪者,其行欺主也,其罪当死亡也。智士者远见而畏于死亡,必不从重人矣;贤士者修廉而羞与奸臣欺其主,必不从重臣矣,是当涂者徒属,非愚而不知患者,必污而不避奸者也。大臣挟愚污之人,上与之欺主,下与之收利侵渔,朋党比周,相与一口,惑主败法,以乱士民,使国家危削,主上劳辱,此大罪也。臣有大罪而主弗禁,此大失也。使其主有大失于上,臣有大罪于下,索国之不亡者,不可得也。

\hypertarget{header-n890}{%
\subsection{说难}\label{header-n890}}

凡说之难:非吾知之有以说之之难也,又非吾辩之能明吾意之难也,又非吾敢横失而能尽之难也。凡说之难:在知所说之心,可以吾说当之。所说出于为名高者也,而说之以厚利,则见下节而遇卑贱,必弃远矣。所说出于厚利者也,而说之以名高,则见无心而远事情,必不收矣。所说阴为厚利而显为名高者也,而说之以名高,则阳收其身而实疏之;说之以厚利,则阴用其言显弃其身矣。此不可不察也。

夫事以密成,语以泄败。未必其身泄之也,而语及所匿之事,如此者身危。彼显有所出事,而乃以成他故,说者不徒知所出而已矣,又知其所以为,如此者身危。夫异事而当,知者揣之外而得之,事泄于外,必以为己也,如此者身危。周泽未渥也,而语极知,说行而有功,则德忘;说不行而有败,则见疑,如此者身危。贵人有过端,而说者明言礼义以挑其恶,如此者身危。贵人或得计而欲自以为功,说者与知焉,如此者身危。强以其所不能为,止以其所不能已,如此者身危。故与之论大人,则以为间己矣;与之论细人,则以为卖重。论其所爱,则以为借资;论其所憎,则以为尝己也,径省其说,则以为不智而拙之;米盐博辩,则以为多而交之。略事陈意,则曰怯懦而不尽;虑事广肆,则曰草野而倨侮。此说之难,不可不知也。

凡说之务,在知饰所说之所矜而灭其所耻。彼有私急也,必以公义示而强之。其意有下也,然而不能已,说者因为之饰其美而少其不为也。其心有高也,而实不能及,说者为之举其过而见其恶,而多其不行也。有欲矜以智能,则为之举异事之同类者,多为之地,使之资说于我,而佯不知也以资其智。欲内相存之言,则必以美名明之,而微见其合于私利也。欲陈危害之事,则显其毁诽而微见其合于私患也。誉异人与同行者,规异事与同计者。有与同污者,则必以大饰其无伤也;有与同败者,则必以明饰其无失也。彼自多其力,则毋以其难概之也;自勇其断,则无以其谪怒之;自智其计,则毋以其败躬之。大意无所拂悟,辞言无所击摩,然后极骋智辩焉。此道所得,亲近不疑而得尽辞也。伊尹为宰,百里奚为虏,皆所以干其上也。此二人者,皆圣人也;然犹不能无役身以进,如此其污也!今以吾言为宰虏,而可以听用而振世,此非能仕之所耻也。夫旷日离久,而周泽既渥,深计而不疑,引争而不罪,则明割利害以致其功,直指是非以饰其身,以此相持,此说之成也。

昔者郑武公欲伐胡,故先以其女妻胡君以娱其意。因问于群臣:"吾欲用兵,谁可伐者?"大夫关其思对曰:"胡可伐。"武公怒而戮之,曰:"胡,兄弟之国也。子言伐之,何也?"胡君闻之,以郑为亲己,遂不备郑。郑人袭胡,取之。宋有富人,天雨墙坏。其子曰:"不筑,必将有盗。"其邻人之父亦云。暮而果大亡其财。其家甚智其子,而疑邻人之父。此二人说者皆当矣,厚者为戮,薄者见疑,则非知之难也,处知则难也。故绕朝之言当矣,其为圣人于晋,而为戮于秦也,此不可不察。

昔者弥子瑕有宠于卫君。卫国之法:窃驾君车者刖。弥子瑕母病,人间往夜告弥子,弥子矫驾君车以出。君闻而贤之,曰:"教哉!为母之故,亡其刖罪。"异日,与君游于果围,食桃而甘,不尽,以其半啖君。君曰:"爱我哉!亡其口味以啖寡人。"及弥子色衰爱弛,得罪于君,君曰:"是固尝矫驾吾车,又尝啖我以馀桃。"故弥子之行未变于初也,而以前之所以见贤而后获罪者,爱憎之变也。故有爱于主,则智当而加亲;有赠于主,则智不当见罪而加疏。故谏说谈论之士,不可不察爱憎之主而后说焉。

夫龙之为虫也,柔可狎而骑也;然其喉下有逆鳞径尺,若人有婴之者,则必杀人。人主亦有逆鳞,说者能无婴人主之逆鳞,则几矣。

\hypertarget{header-n897}{%
\subsection{和氏}\label{header-n897}}

楚人和氏得玉璞楚山中,奉而献之厉王。厉王使玉人相之。玉人曰:"石也。"王以和为诳,而刖其左足。及厉王薨,武王即位。和又奉其璞而献之武王。武王使玉人相之。又曰:"石也。"王又以和为诳,而刖其右足。武王薨,文王即位。和乃抱其璞而哭于楚山之下,三日三夜,泪尽而继之以血。王闻之,使人问其故,曰:"天下之刖者多矣,子奚哭之悲也?"和曰:"吾非悲刖也,悲夫宝玉而题之以石,贞士而名之以诳,此吾所以悲也。"王乃使玉人理其璞而得宝焉,遂命曰:"和氏之璧。"

夫珠玉,人主之所急也。和虽献璞而未美,未为主之害也,然犹两足斩而宝乃论,论宝若此其难也!今人主之于法术也,未必和璧之急也;而禁群臣士民之私邪。然则有道者之不戮也,特帝王之璞未献耳。主用术,则大臣不得擅断,近习不敢卖重;官行法,则浮萌趋于耕农,而游士危于战陈;则法术者乃群臣士民之所祸也。人主非能倍大臣之议,越民萌之诽,独周乎道言也,则法术之士虽至死亡,道必不论矣。

昔者吴起教楚悼王以楚国之俗,
曰:``大臣太重,封君太众;若此,则上主而下虐民,
此贫国弱兵之道也。不如使封君之子孙三世而收爵禄,绝减百吏之禄秩,
损不急之枝官,以奉选练之士。''悼王行之期年而薨矣,吴起枝解于楚。
商君教秦孝公以连什伍,设告坐之过,燔诗书而明法令,塞私门之请而遂公家之劳,
禁游宦之民而显耕战之士。孝公行之,主以尊安,国以富强。八年而薨,商君车裂于秦。楚不用吴起而削乱,秦行商君法而富强,二子之言也已当矣,然而枝解吴起而车裂商君者何也?大臣苦法而细民恶治也。当今之世,大臣贪重,细民安乱,甚于秦、楚之俗,而人主无悼王、孝公之听,则法术之士安能蒙二子之危也而明己之法术哉!此世所以乱无霸王也。

\hypertarget{header-n902}{%
\subsection{奸劫弑臣}\label{header-n902}}

凡奸臣皆欲顺人主之心以取亲幸之势者也。是以主有所善,臣从而誉之;主有所憎,臣因而毁之。凡人之大体,取舍同者则相是也,取舍异者则相非也。今人臣之所誉者,人主之所是也,此之谓同取;人臣之所毁者,人主之所非也,此之谓同舍。夫取舍合而相与逆者,未尝闻也。此人臣之所以取信幸之道也。夫奸臣得乘信幸之势以毁誉进退群臣者,人主非有术数以御之也,非参验以审之也,必将以曩之合己信今之言,此幸臣之所以得欺主成私者也。故主必蔽于上,而臣必重于下矣,此之谓擅主之臣。

国有擅主之臣,则群下不得尽其智力以陈其忠,百官之吏不得奉法以致其功矣。何以明之?夫安利者就之,危害者去之,此人之情也。今为臣尽力以致功,竭智以陈忠者,其身困而家贫,父子罹其害;为奸利以弊人主,行财货以事贵重之臣者,身尊家富,父子被其泽:人焉能去安利之道而就危害之处哉?治国若此其过也,而上欲下之无奸,吏之奉法,其不可得亦明矣。故左右知贞信之不可以得安利也,必曰:"我以忠信事上,积功劳而求安,是犹盲而欲知黑白之情,必不几矣。若以道化行正理,不趋富贵,事上而求安,是犹聋而欲审清浊之声也,愈不几矣。二者不可以得安,我安能无相比周,蔽主上,为奸私以适重人哉?"此必不顾人主之义矣。其百官之吏亦知方正之不可以得安也,必曰:"我以清廉事上而求安,若无规矩而欲为方圆也,必不几矣;若以守法不朋党治官而求安,是犹以足搔顶也,愈不几也!二者不可以得安,能无废法行私以适重人哉?"此必不顾君上之法矣。故以私为重人者众,而以法事君者少矣。是以主孤于上而臣成党于下,此田成之所以杀简公者也。

夫有术者之为人臣也,得效度数之言,上明主法,下困奸臣,以尊主安国者也。是以度数之言得效于前,则赏罚必用于后矣。人主诚明于圣人之术,而不苟于世欲之言,循名实而定是非,因参验而审言辞。是以左右近习之臣,知伪诈之不可以得安也,必曰:"我不去奸私之行,尽力竭智以事主,而乃以相与比周,妄毁誉以求安,是犹负千钧之重,陷于不测之渊而求生也,必不几矣。"百官之吏,亦知为奸利之不可以得安也,必曰:"我不以清廉方正奉法,乃以贪污之心枉法以取私利,是犹上高陵之颠堕峻裕谷之下而求生,必不几矣。"安危之道若此其明也,左右安能以虚言惑主,而百官安敢以贪渔下?是以臣得陈其忠而不弊,下得守其职而不怨。此管仲之所以治齐,而商君之所以强秦也。

从是观之,则圣人之治国也,固有使人不得不爱我之道,而不恃人之以爱为我也。恃人之以爱为我者危矣,恃吾不可不为者安矣。夫君臣非有骨肉之亲,正直之道可以得利,则臣尽力以事主;正直之道不可以得安,则臣行私以干上。明主知之,故设利害之道以示天下而已矣。夫是以人主虽不口教百官,不目索奸邪,而国已治矣。人主者,非目若离娄乃为明也,非耳若师旷乃为聪也。不任其数,而待目以为明,所见都少矣,非不弊之术也。不因其势,而待耳以为聪,所闻者寡矣,非不欺之道也。明主者,使天下不得不为己视,天下不得不为己听。故身在深宫之中而明照四海之内,而天下弗能蔽弗能欺者,何也?暗乱之道废而聪明之势兴也。故善任势者国安,不知因其势者国危。古秦之俗,君臣废法而服私,是以国乱兵弱而主卑。商君说秦孝公以变法易俗而明公道,赏告奸、困末作而利本事。当此之时,秦民习故俗之有罪可以得免,无功可以得尊显也,故轻犯新法。于是犯之者其诛重而必,告之者其赏厚而信,故奸莫不得而被刑者众,民疾怨而众过日闻。孝公不听,遂行商君之法。民后知有罪之必诛,而告私奸者众也,故民莫犯,其刑无所加。是以国治而兵强,地广而主尊。此其所以然者,匿罪之罚重,而告奸之赏厚也。此亦使天下必为己视听之道也。至治之法术已明矣,而世学者弗知也。

且夫世之愚学,皆不知乱之情,讘讠夹多诵先古之书,以乱当世之治;智虑不足以避阱井之陷,又妄非有术之士。听其言者危,用其计者乱,此亦愚之至大而患之至甚者也。俱与有术之士,有谈说之名,而实相去千万也。此夫名同而实有异者也。夫世愚学之人比有术之士也,犹蚁垤之比大陵也,其相去远矣。而圣人者,审于是非之实,察于治乱之情也。故其治国也,正明法,陈严刑,将以救群生之乱,去天下之祸,使强不陵弱,众不暴寡,耆老得遂,幼孤得长,边境不侵,群臣相关,父子相保,而无死亡系虏之患,此亦功之至厚者也。愚人不知,顾以为暴。愚者固欲治而恶其所以治,皆恶危而喜其所以危者。何以知之?夫严刑重罚者,民之所恶也,而国之所以治也;哀怜百姓轻刑罚者,民之所喜,而国之所以危也。圣人为法国者,必逆于世,而顺于道德。知之者同于义而异于俗;弗知这者,异于义而同于俗。天下知之者少,则义非矣。

处非道之位,被众口之谮,溺于当世之言,而欲当严天子而求安,几不亦难哉!此夫智士所以至死而不显于世者也。楚庄王之弟春申君,有爱妾曰余,春申君之正妻子曰甲。余欲君之弃其妻也,因自伤其身以视君而泣,曰:"得为君之妾,甚幸。虽然,适夫人非所以事君也,适君非所以事夫人也。身故不肖,力不足以适二主,其势不俱适,与其死夫人所者,不若赐死君前。妾以赐死,若复幸于左右,愿君必察之,无为人笑。"君因信妾余之诈,为弃正妻。余又欲杀甲而以其子为后,因自裂其呆衣之里,以示君而泣,曰:"余之得幸君之日久矣,甲非弗知也,今乃欲强戏余。余与争之,至裂余之衣,而此子之不孝,莫大于此矣!"君怒,而杀甲也。故妻以妾余之诈弃,而子以之死。从是观之,父子爱子也,犹可以毁而害也;君臣之相与也,非有父子之亲也,而群臣之毁言,非特一妾之口也,何怪夫贤圣之戮死哉!此商君之所以车裂于秦,而吴起之所以枝解于楚者也。凡人臣者,有罪固不欲诛,无功者皆欲尊显。而圣人之治国也,赏不加于无功,而诛必行于有罪者也。然则有术数者之为人也,固左右奸臣之所害,非明主弗能听也。

世之学者说人主,不曰:"乘威严之势以困奸邪之臣",而皆曰:"仁义惠爱而已矣!"世主美仁义之名而不察其实,是以大者国亡身死,小者地削主卑。何以明之?夫施与贫困者,此世之所谓仁义;哀怜百姓,不忍诛罚者,此世之所谓惠爱也。夫有施与贫困,则无功者得赏;不忍诛罚,则暴乱者不止。国有无功得赏者,则民不外务当敌斩首,内不急力田疾作,皆欲行货财,事富贵,为私善,立名誉,以取尊官厚俸。故奸私之臣愈众,而暴乱之徒愈胜,不亡何时!夫严刑者,民之所畏也;重罚者,民之所恶也。故圣人陈其所畏以禁其邪,设其所恶以防其奸,是以国安而暴乱不起。吾以是明仁义爱惠之不足用,而严刑重罚之可以治国也。无棰策之威,衔橛之备,虽造父不能以服马;无规矩之法,绳墨之端,虽王尔不能以成方圆;无威严之势,赏罚之法,虽舜不能以为治。今世主皆轻释重罚严诛,行爱惠,而欲霸王之功,亦不可几也。故善为主者,明赏设利以劝之,使民以功赏而不以仁义赐;严刑重罚以禁之,使民以罪诛而不以爱惠免。是以无功者不望,而有罪者不幸矣。讬于犀车良马之上,则可以陆犯阪阻之患;乘舟之安,持楫之利,则可以水绝江河之难;操法术之数,行重罚严诛,则可以致霸王之功。治国之有法术赏罚,犹若陆行之有犀车良马也,水行之有轻舟便楫也,乘之者遂得其成。伊尹得之,汤以王;管仲得之,齐以霸;商君得之,秦以强。此三人者,皆明于霸王之术,察于治强之数,而不以牵于世俗之言;适当世明主之意,则有直任布衣之士,立为卿相之处;处位治国,则有尊主广地之实:此之谓足贵之臣。汤得伊尹,以百里之地立为天子;桓公得管仲,立为五霸主,九合诸候,一匡天下;孝公得商君,地以广,兵以强。故有忠臣者,外无敌国之患,内无乱臣之忧,长安于天下,而名垂后世,所谓忠臣也。若夫豫让为智伯臣也,上不能说人主使之明法术度数之理以避祸难之患,下不能领御其众以安其国;及襄子之杀智伯也,豫让乃自黔劓,败其形容,以为智伯报襄子之仇。是虽有残刑杀身以为人主之名,而实无益于智伯若秋毫之末。此吾之所下也,而世主以为忠而高之。古有伯夷叔齐者,武王让以天下而弗受,二人饿死首阳之陵。若此臣,不畏重诛,不利重赏,不可以罚禁也,不可以赏使也,此之谓无益之臣也。吾所少而去也,而世主之所多而求也。

谚曰:"厉怜王。"此不恭之言也。虽然,古无虚谚,不可不察也。此谓劫杀死亡之主言也。人主无法术以御其臣,虽长年而美材,大臣犹将得势,擅事主断,而各为其私急。而恐父兄毫杰之士,借人主之力,以禁诛于己也,故杀贤长而立幼弱,废正的而立不义。故《春秋》记之曰:"楚王子围将聘于郑,未出境,闻王病而反。因入问病,以其冠缨绞王而杀之,遂自立也。齐崔杼,其妻美,而庄公通之,数如崔氏之室。及公往,崔子之徒贾举率崔子之徒而攻公。公入室,请与之分国,崔子不许;公请自刃于庙,崔子又不听;公乃走,逾于北墙。贾举射公,中其股,公坠,崔子之徒以戈斫公而死之,而立其弟景公。"近之所见:李兑之用赵也,饿主父百日而死,卓齿之用齐也,擢湣王之筋,悬之庙梁,宿昔而死。故厉虽癕肿疕疡,上比于《春秋》,未至于绞颈射股也;下比于近世,未至饿死擢筋也。故劫杀死亡之君,此其心之忧惧,形之苦痛也,必甚于厉矣。由此观之,虽"厉怜王"可也。

\hypertarget{header-n911}{%
\subsection{亡征}\label{header-n911}}

凡人主之国小而家大,权轻而臣重者,可亡也。简法禁而务谋虑,荒封内而恃交援者,可亡也。群臣为学,门子好辩,商贾外积,小民内困者,可亡也。好宫室台榭陂池,事车服器玩,好罢露百姓,煎靡货财者,可亡也。用时日,事鬼神,信卜筮而好祭祀者,可亡也。听以爵不以众言参验,用一人为门户者,可亡也。官职可以重求,爵禄可以货得者,可亡也。缓心而无成,柔茹而寡断,好恶无决而无所定立者,可亡也。饕贪而无厌,近利而好得者,可亡也。喜淫辞而不周于法,好辩说而不求其用,滥于文丽而不顾其功者,可亡也。浅薄而易见,漏泄而无藏,不能周密而通群臣之语者,可亡也。很刚而不和,愎谏而好胜,不顾社稷而轻为自信者,可亡也。恃交援而简近邻,怙强大之救而侮所迫之国者,可亡也。羁旅侨士,重帑在外,上间谋计,下与民事者,可亡也。民信其相,下不能其上,主爱信之而弗能废者,可亡也。境内之杰不事,而求封外之士,不以功伐课试,而好以各问举错,羁旅起贵以陵故常者,可亡也。轻其适正,庶子称衡,太子未定而主即世者,可亡也。大心而无悔,国乱而自多,不料境内之资而易其邻敌者,可亡也。国小而不处卑,力少而不畏强,无礼而侮大邻,贪愎而拙交者,可亡也。太子已置,而娶于强敌以为后妻,则太子危,如是,则群臣易虑者,可亡也。怯慑而弱守,蚤见而心柔懦,知有谓可,断而弗敢行者,可亡也。出君在外而国更置,质太子未反而君易子,如是则国摧;国摧者,可亡也。挫辱大臣而狎其身,刑戮小民而逆其使,怀怒思耻而专习则贼生,贼生者,可亡也。大臣两重,父兄众强,内党外援以争事势者,可亡也。婢妾之言听,爱玩之智用,外内悲惋而数行不法者,可亡也。简侮大臣,无礼父兄,劳苦百姓,杀戮不辜者,可亡也。好以智矫法,时以行杂公,法禁变易,号令数下者,可亡也。无地固,城郭恶,无畜积,财物寡,无守战之备而轻攻伐者,可亡也。种类不寿,主数即世,婴兒为君,大臣专制,树羁旅以为党,数割地以待交者,可亡也。太子尊显,徒属众强,多大国之交,而威势蚤具者,可亡也。变褊而心急,轻疾而易动发,心悁忿而不訾前后者,可亡也。主多怒而好用兵,简本教而轻战攻者,可亡也。贵臣相妒,大臣隆盛,外藉敌国,内困百姓,以攻怨雠,而人主弗诛者,可亡也。君不肖而侧室贤,太子轻而庶子伉,官吏弱而人民桀,如此则国躁;国躁者,可亡也。藏恕而弗发,悬罪而弗诛,使群臣阴赠而愈忧惧,而久未可知者,可亡也。出军命将太重,边地任守太尊,专制擅命,径为而无所请者,可亡也。后妻淫乱,主母畜秽,外内混通,男女无别,是谓两主;两主者,可亡也,后妻贱而婢妾贵,太子卑而庶子尊,相室轻而典谒重,如此则内外乖;内外乖者,可亡也。大臣甚贵,偏党众强,壅塞主断而重擅国者,可亡也。私门之官用,马府之世绌,乡曲之善举者,可亡也。官职之劳废,贵私行而贱公功者,可亡也。公家虚而大臣实,正户贫而寄寓富,耕战之士困,末作之民利者,可亡也。见大利而不趋,闻祸端而不备,浅薄于争守之事,而务以仁义自饰者,可亡也。不为人主之孝,而慕瓜夫之孝,不顾社稷之利,而听主母之令,女子用国,刑馀用事者,可亡也。辞辩而不法,心智而无术,主多能而不以法度从事者,可亡也。亲臣进而故人退,不肖用事而贤良伏,无功贵而劳苦贱,如是则下怨;下怨者,可亡也。父兄大臣禄秩过功,章服侵等,宫室供养大侈,而人主弗禁,则臣心无穷,臣心无穷者,可亡也。公胥公孙与民同门,暴慠其邻者,可亡也。

亡征者,非曰必亡,言其可亡也。夫两尧不能相王,两桀不能相亡;亡王之机,必其治乱,其强弱相踦者也。木之折也必通蠹,墙之坏也必通隙。然木虽蠹,无疾风不折;墙虽隙,无大雨不坏。万乘之主,有能服术行法以为亡征之君风雨者,其兼天下不难矣。

\hypertarget{header-n914}{%
\subsection{三守}\label{header-n914}}

人主有三守。三守完,则国安身荣;三守不完,则国危身殆。何谓三守?人臣有议当途之失,用事之过,举臣之情,人主不心藏而漏之近习能人,使人臣之欲有言者,不敢不下适近习能人之心,而乃上以闻人主,然则端言直道之人不得见,而忠直日疏。爱人,不独利也,待誉而后利之;憎人不独害也,待非而后害之。然则人主无威而重在左右矣。恶自治之劳惮,使群臣辐凑之变,因传柄移藉,使杀生之机,夺予之要在大臣,如是者侵。此谓三守不完。三守不完,则劫杀之征也。

凡劫有三:有明劫,有刑劫,人臣有大臣之尊,外操国要以资群臣,使外内之事非已不得行。虽有贤良,逆者必有祸,而顺者必有福。然则群臣直莫敢忠主忧国以争社稷之利害。人主虽贤,不能独计,而人臣有不敢忠主,则国为亡国矣。此谓国无臣。国无臣者,岂郎中虚而朝臣少哉?群臣持禄养交,行私道而不效公忠,此谓明劫。鬻宠擅权,矫外以胜内,险言祸福得失之形,以阿主之好恶。人主听之,卑身轻国以资之,事败与主分其祸,而功成则臣独专之。诸用事之人,壹心同辞以语其美,则主言恶者必不信矣。此谓事劫。至于守司囹圄,禁制刑罚,人臣擅之,此谓刑劫。三守不完,则三劫者起;三守完,则三劫者止。三劫止塞,则王矣。

\hypertarget{header-n917}{%
\subsection{备内}\label{header-n917}}

人主之患在于信人,信人,则制于人。人臣之于其君,非有骨肉之亲也,缚于势而不得不事也。故为人臣者,窥觇其君心也,无须臾之休,而人主怠傲处上,此世所以有劫君杀主也。为人主而大信其子,则奸臣得乘于子以成其私,故李兑传赵王而饿主父。为人主而大信其妻,则奸臣得乘于妻以成其私,故优施传丽姬杀申生而立奚齐。夫以妻之近与子之亲而犹不可信,则其余无可信者矣。

且万乘之主,千乘之君,后妃夫人、适子为太子者,或有欲其君之蚤死者。何以知其然,夫妻者,非有骨肉之恩也,爱则亲,不爱则疏。语曰:"其母好者其子抱。"然则其为之反也,其母恶者其子释。丈夫年五十而好色未解也,妇人年三十而美色衰矣。以衰美之妇人事好色之丈夫,则身见疏贱,而子疑不为后,此后妃夫人之所以冀其君之死者也。唯母为后而子为主,则令无不行,禁无不止,男女之乐不减于先君,而擅万乘不疑,此鸩毒扼昧之所以用也。故《桃左春秋》曰:"人主这疾死者不能处半。",人主弗知,则乱多资。故曰:利君死者众,则人主危。故王良爱马,越王勾践爱人,为战与驰。医善吮人之伤,含人之血,非骨肉之亲也,利所加也。故与人成舆,则欲人之富贵;匠人成棺,则欲人之夭死也。非舆人仁而匠人贼也,人不贵,则舆不售;人不死,则棺不买。情非憎人也,利在人之死也,故后妃、夫人太子之党成而欲君之死也,君不死,则势不重。情非憎君也,利在君之死也。故人主不可以不加心于利己死者。故日月晕围于外,其贼在内,备其所憎,祸在所爱。是故明王不举不参之事,不食非常之食;远听而近视,以审内外之失,省同异之言以知朋党之分,偶参伍之验以责陈言之实;执后以应前,按法以治众,众端以参观。士无幸赏,无逾行,杀必当,罪不赦,则奸邪无所容其私。

徭役多则民苦,民苦则权势起,权势起则复除重,复除重则贵人富。苦民以富贵人,起势以藉人臣,非天下长利也。故曰:徭役少则民安,民安则下无重权,下无重权则权势灭,权势灭则德在上矣。今夫水之胜火亦明矣,然而釜鬵间之,水煎沸竭尽其上,而火得炽盛焚其下,水失其所以胜者矣。今夫治之禁奸又明于此,然法守之臣为釜鬵之行,则法独明于胸中,而已失其所以禁奸者矣。上古之传言,《春秋》所记,犯法为逆以成大奸者,未尝不从尊贵之臣也。然而法令之所以备,刑罚之所以诛,常于卑赋,是以其民绝望,无所告诉。大臣比周,蔽上为一,阴相善而阳相恶,以示无私,相为耳目,以候主隙,人主掩蔽,无道得闻,有主名而无实,臣专法而行之,周天子是也。偏借其权势,则上下易位矣,此言人臣之不可借权势。

\hypertarget{header-n921}{%
\subsection{南面}\label{header-n921}}

人主之过,在己任臣矣,又必反与其所不任者备之,此其说必与其所任者为仇,而主反制于其所不任者。今所与备人者,且曩之所备也.人主不能明法而以制大臣之威,无道得小人之信矣。人主释法而以臣备臣,则相爱者比周而相誉,相憎者朋党而相非。非誉交争,则主惑乱矣。人臣者,非名誉请谒无以进取,非背法专制无以为威,非假于忠信无以不禁,三者,愍主坏法之资也。人主使人臣虽有智能,不得背法而专制;虽有贤行,不得逾功而先劳,虽有忠信,不得释法而不禁:此之谓明法。

人主有诱于事者,有壅于言者,二者不可不察也。人臣易言事者,少索资,以事诬主。主诱而不察,因而多之,则是臣反以事制主也。如是者谓之诱,诱于事者困于患。共进言少,其退费多,虽有功,其进言不信。不信者有罪,事有功者必赏,则群臣莫敢饰言以愍主。主道者,使人臣前言不复于后,复言不复于前,事虽有功,必伏其罪,谓之任下。

人臣为主设事而恐其非也,则先出说设言曰:"议是事者,妒事者也。"人主藏是言,不更听群臣;群臣畏是言,不敢议事。二势者用,则忠臣不听而誉臣独任。如是者谓之壅于言,壅于言者制于臣矣。主道者,使人臣必有言之责,又有不言之责。言无端末辩无所验者,此言之责也;以不言避责持重位者,此不言之责也。人主使人臣言者必知其端以责其实,不言者必问其取舍以为之责。则人臣莫敢妄言矣,又不敢默然矣,言、默则皆有责也。

人主欲为事,不通其端末,而以明其欲,有为之者,其为不得利,必以害反。知此者,任理去欲。举事有道,计其入多,其出少者,可为也。惑主不然,计其入,不计其出,出虽倍其入,不知其害,则是名得而实亡。如是者功小而害大矣。凡功者,其入多,其出少,乃可谓功。今大费无罪而少得为功,则人臣出大费而成小功,小功成而主亦有害。

不知治者,必曰:"无变古,毋易常。"变与不变,圣人不听,正治而已。则古之无变,常之毋易,在常古之可与不可。伊尹毋变殷,太公毋变周,则汤、武不王矣。管仲毋易齐,郭偃毋更晋,则桓、文不霸矣。凡人难变古者,惮易民之安也。夫不变古者,袭乱之迹;适民心者,恣奸之行也。民愚而不知乱,上懦而不能更,是治之失也。人主者,明能知治,严必行之,故虽拂于民,必立其治。说在商君之内外而铁殳,重盾而豫戒也。故郭偃之始治也,文公有官卒;管仲始治也,桓公有武车:戒民之备也。是以愚戆窳堕之民,苦小费而忘大利也,故夤虎受阿谤而振小变而失长便,故邹贾非载旅。狎习于乱而容于治,故郑人不能归。

\hypertarget{header-n927}{%
\subsection{饰邪}\label{header-n927}}

凿龟数策,兆曰"大吉",而以攻燕者,赵也。凿龟数筴,兆曰"大吉",而以攻赵者,燕也。剧辛之事燕,无功而社稷危;邹衍之事燕,无功而国道绝。赵代先得意于燕,后得意于齐,国乱节高。自以为与秦提衡,非赵龟神而燕龟欺也。赵又尝凿龟数筴而北伐燕,将劫燕以逆秦,兆曰"大吉"。始攻大梁而秦出上党矣,兵至厘而六城拔矣;至阳城,秦拔鄴矣;庞援揄兵而南,则鄣尽矣。臣故曰:赵龟虽无远见于燕,且宜近见于秦。秦以其"大吉",辟地有实,救燕有有名。赵以其"大吉",地削兵辱,主不得意而死。又非秦龟神而赵龟欺也。初时者,魏数年东乡攻尽陶、卫,数年西乡以失其国,此非丰隆、五行、太一、王相、摄提、六神、五括、天河、殷抢、岁星非数年在西也,又非天缺、弧逆、刑星、荧惑、奎台非数年在东也。故曰:龟筴鬼神不足举胜,左右背乡不足以专战。然而恃之,愚莫大焉。

古者先王尽力于亲民,加事于明法。彼法明,则忠臣劝;罚必,则邪臣止。忠劝邪止而地广主尊者,秦是也;群臣朋党比周以隐正道行私曲而地削主卑者,山东是也。乱弱者亡,人之性也;治强者王,古之道也。越王勾践恃大朋之龟与吴战而不胜,身臣入宦于吴;反国弃龟,明法亲民以报吴,则夫差为擒。故恃鬼神者慢于法,恃诸侯者危其国。曹恃齐而不听宋,齐攻荆而宋灭曹。邢恃吴而不听齐,越伐吴而齐灭邢。许恃荆而不听魏,荆攻宋而魏灭许。郑恃魏而不听韩,魏攻荆而韩灭郑。今者韩国小而恃大国,主慢而听秦、魏,恃齐、荆为用,而小国愈亡。故恃人不足以广壤,而韩不见也。荆为攻魏而加兵许、鄢,齐攻任、扈而削魏,不足以存郑,而韩弗知也。此皆不明其法禁以治其国,恃外以灭其社稷者也。

臣故曰:明于治之数,则国虽小,富;赏罚敬信,民虽寡,强。赏罚无度,国虽大,兵弱者,地非其地,民非其民也。无地无民,尧、舜不能以王,三代不能以强。人主又以过予,人臣又以徒取。舍法律而言先王以明古之功者,上任之以国。臣故曰:是原古之功,以古之赏赏今之人也。主以是过予,而臣以此徒取矣。主过予,则臣偷幸;臣徒取,则功不尊。无功者受赏,则财匮而民望;财匮而民望,则民不尽力矣。故用赏过者失民,用刑过者民不畏。有赏不足以劝,有刑不足以禁,则国虽大,必危。

故曰:小知不可使谋事,小忠不可使主法。荆恭王与晋厉公战于鄢陵,荆师败,恭王伤。酣战,而司马子反渴而求饮,其友竖谷阳奉卮酒而进之。子反曰:"去之,此酒也。"竖谷阳曰:"非也。"子反受而饮之。子反为人嗜酒,甘之,不能绝之于口,醉而卧。恭王欲复战而谋事,使人召子反,子反辞以心疾。恭王驾而往视之,入幄中,闻酒臭而还,曰:"今日之战,寡人目亲伤。所恃者司马,司马又如此,是亡荆国之社稷而不恤吾众也。寡人无与复战矣。"罢师而去之,斩子反以为大戮。故曰:竖谷阳之进酒也,非以端恶子反也,实心以忠爱之,而适足以杀之而已矣。此行小忠而贼大忠者也。故曰:小忠,大忠之贼也。若使小忠主法,则必将赦罪,赦罪以相爱,是与下安矣,然而妨害于治民者也。

当魏之方明《立辟》、从宪令行之时,有功者必赏,有罪者必诛,强匡天下,威行四邻;及法慢,妄予,而国日削矣。当赵之方明《国律》、从大军之时,人众兵强,辟地齐、燕;及《国律》满,用者弱,而国日削矣。当燕之方明《奉法》、审官断之时,东县齐国,南尽中山之地;及《奉法》已亡,官断不用,左右交争,论从其下,则兵弱而地削,国制于邻敌矣。故曰:明法者强,慢法者弱。强弱如是其明矣,而世主弗为,国亡宜矣。语曰:"家有常业,虽饥不饿;国有常法,虽危不亡。"夫舍常法而从私意,则臣下饰于智能;臣下饰于智能,则法禁不立矣。是亡意之道行,治国之道废也。治国之道,去害法者,则不惑于智能,不矫于名誉矣。昔者舜使吏决鸿水,先令有功而舜杀之;禹朝诸候之君会稽之上,防风之君后至而禹斩之。以此观之,先令者杀,后令者斩,则古者先贵如令矣。故镜执清而无事,美恶从而比焉;衡执正而无事,轻重从而载焉。夫摇镜,则不得为明;摇衡,则不得为正,法之谓也。故先王以道为常,以法为本。本治者名尊,本乱者名绝。凡智能明通,有以则行,无以则止。故智能单道,不可传于人。而道法万全,智能多失。夫悬衡而知平,设规而知圆,万全之道也。明主使民饰于道之故,故佚而有功。释规而任巧,释法而任智,惑乱之道也。乱主使民饰于智,不知道之故,故劳而无功。释法禁而听请谒群臣卖官于上,取赏于下,是以利在私家而威在群臣。故民无尽力事主之心,而务为交于上。民好上交,则货财上流,而巧说者用。若是,则有功者愈少。奸臣愈进而材臣退,则主惑而不知所行,民聚而不知所道。此废法禁、后功劳、举名誉、听请谒之失也。凡败法之人,必设诈托物以来亲,又好言天下之所希有。此暴君乱主之所以惑也,人臣贤佐之所以侵也。故人臣称伊尹、管仲之功,则背法饰智有资;称比干、子胥之忠而见杀,则疾强谏有辞。夫上称贤明,不称暴乱,不可以取类,若是者禁。君子立法以为是也,今人臣多立其私智以法为非者,是邪以智,过法立智。如是者禁,主之道也。

明主之道,必明于公私之分,明法制,去私恩。夫令必行,禁必止,人主之公义也;必行其私,信于朋友,不可为赏劝,不可为罚沮,人臣之私义也。私义行则乱,公义行则治,故公私有分。人臣有私心,有公义。修身洁白而行公行正,居官无私,人臣之公义也;污行从欲,安身利家,人臣之私心也。明主在上,则人臣去私心行公义;乱主在上,则人臣去公义行私心。故君臣异心,君以计畜臣,臣以计事君,君臣之交,计也。害身而利国,臣弗为也;害国而利臣,君不为也。臣之情,害身无利;君之情,害国无亲。君臣也者,以计合者也。至夫临难必死,尽智竭力,为法为之。故先王明赏以劝之,严刑以威之。赏刑明,则民尽死;民尽死,则兵强主尊。刑赏不察,则民无功而求得,有罪而幸免,则兵弱主卑。故先王贤佐尽力竭智。故曰:公私不可不明,法禁不可不审,先王知之矣。

\hypertarget{header-n934}{%
\subsection{解老}\label{header-n934}}

德者,内也。得者,外也。"上德不德",言其神不淫于外也。神不淫于外,则身全。身全之谓德。德者,得身也。凡德者,以无为集,以无欲成,以不思安,以不用固。为之欲之,则德无舍;德无舍,则不全。用之思之,则不固;不固,则无功;无功,则生于德。德则无德,不德则有德。故曰:"上德不德,是以有德。"

所以贵无为无思为虚者,谓其意无所制也。夫无术者,故以无为无思为虚也。夫故以无为无思为虚者,其意常不忘虚,是制于为虚也。虚者,谓其意无所制也。今制于为虚,是不虚也。虚者之无为也,不以无为为有常。不以无为为有常,则虚;虚,则德盛;德盛之为上德。故曰:"上德无为而无不为也。"仁者,谓其中心欣然爱人也;其喜人之有福,而恶人之有祸也;生心之所不能已也,非求其报也。故曰:"上仁为之而无以为也。"

义者,君臣上下之事,父子贵贱之差也,知交朋友之接也,亲疏内外之分也。臣事君宜,下怀上宜,子事父宜,贱敬贵宜,知交朋友之相助也宜,亲者内而疏者外宜。义者,谓其宜也,宜而为之。故曰:"上义为之而有以为也。"

礼者,所以貌情也,群义之文章也,君臣父子之交也,贵贱贤不肖之所以别也。中心怀而不谕,故疾趋卑拜而明之;实心爱而不知,故好言繁辞以信之。礼者,外饰之所以谕内也。故曰:礼以貌情也。凡人之为外物动也,不知其为身之礼也。众人之为礼也,以尊他人也,故时劝时衰。君子之为礼,以为其身;以为其身,故神之为上礼;上礼神而众人贰,故不能相应;不能相应,故曰:"上礼为之而莫之应。"众人虽贰,圣人之复恭敬尽手足之礼也不衰。故曰:"攘臂而仍之。"

道有积而积有功;德者,道之功。功有实而实有光;仁者,德之光。光有泽而泽有事;义者,仁之事也。事有礼而礼有文;礼者,义之文也。故曰:"失道而后失德,失德而后失仁,失仁而后失义,失义而后失礼。"

礼为情貌者也,文为质饰者也。夫君子取情而去貌,好质而恶饰。夫恃貌而论情者,其情恶也;须饰而论质者,其质衰也。何以论之?和氏之璧,不饰以五采;隋侯之珠,不饰以银黄。其质至美,物不足以饰之。夫物之待饰而后行者,其质不美也。是以父子之间,其礼朴而不明,故曰:"理薄也。"凡物不并盛,阴阳是也;理相夺予,威德是也;实厚者貌薄,父子之礼是也。由是观之,礼繁者,实心衰也。然则为礼者,事通人之朴心者也。众人之为礼也,人应则轻欢,不应则责怨。今为礼者事通人之朴心而资之以相责之分,能毋争乎?有争则乱,故曰:"夫礼者,忠信之薄也,而乱之首乎。"

先物行先理动之谓前识。前识者,无缘而妄意度也。何以论之?詹何坐,弟子侍,牛鸣于门外。弟子曰:"是黑牛也在而白其题。"詹何曰:"然,是黑牛也,而白在其角。"使人视之,果黑牛而以布裹其角。以詹子之术,婴众人之心,华焉殆矣!故曰:"道之华也。"尝试释詹子之察,而使五尺之愚童子视之,亦知其黑牛而以布裹其角也。故以詹子之察,苦心伤神,而后与五尺之愚童子同功,是以曰:"愚之首也。"故曰:"前识者,道之华也,而愚之首也。"

所谓"大丈夫"者,谓其智之大也。所谓"处其厚而不处其薄"者,行情实而去礼貌也。所谓"处其实不处其华"者,必缘理,不径绝也。所谓"去彼取此"者,去貌、径绝而取缘理、好情实也。故曰:"去彼取此。"

人有祸,则心畏恐;心畏恐,则行端直;行端直,则思虑熟;思虑熟,则得事理。行端直,则无祸害;无祸害,则尽天年。得事理,则必成功。尽天年,则全而寿。必成功,则富与贵。全寿富贵之谓福。而福本于有祸。故曰:"祸兮福之所倚。"以成其功也。

人有福,则富贵至;富贵至,则衣食美;衣食美,则骄心生;骄心生,则行邪僻而动弃理。行邪僻,则身夭死;动弃理,则无成功。夫内有死夭之难而外无成功之名者,大祸也。而祸本生于有福。故曰:"福兮祸之所伏。"

夫缘道理以从事者,无不能成。无不能成者,大能成天子之势尊,而小易得卿相将军之赏禄。夫弃道理而妄举动者,虽上有天子诸侯之势尊,而下有猗顿、陶硃、卜祝之富,犹失其民人而亡其财资也。众人之轻弃道理而易妄举动者,不知其祸福之深大而道阔远若是也,故谕人曰:"孰知其极。"

人莫不欲富贵全寿,而未有能免于贫贱死夭之祸也。心欲富贵全寿,而今贫贱死夭,是不能至于其所欲至也。凡失其所欲之路而妄行者之谓迷,迷则不能至于其所欲至矣。今众人之不能至于其所欲至,故曰:"迷。"众人之所不能至于其所欲至也,自天地之剖判以至于今。故曰:"人之迷也,其日故以久矣。"

所谓方者,内外相应也,言行相称也。所谓廉者,必生死之命也,轻恬资财也。所谓直者,义必公正,公心不偏党也。所谓光者,官爵尊贵,衣裘壮丽也。今有道之士,虽中外信顺,不以诽谤穷堕;虽死节轻财,不以侮罢羞贪;虽义端不党,不以去邪罪私;虽势尊衣美,不以夸贱欺贫。其故何也?使失路者而肯听习问知,即不成迷也。今众人之所以欲成功而反为败者,生于不知道理,而不肯问知而听能。众人不肯问知听能,而圣人强以其祸败适之,则怨。众人多而圣人寡,寡之不胜众,数也。今举动而与天下之为仇,非全身长生之道也,是以行轨节而举之也。故曰:"方而不割,廉而不刿,直而不肆,光而不耀。"

聪明睿智,天也;动静思虑,人也。人也者,乘于天明以视,寄于天聪以听,托于天智以思虑。故视强,则目不明;听甚,则耳不聪;思虑过度,则智识乱。目不明,则不能决黑白之分;耳不聪,则不能别清浊之声;智识乱,则不能审得失之地。目不能决黑白之色则谓之盲;耳不能别清浊之声则谓之聋;心不能审得失之地则谓之狂。盲则不能避昼日之险,聋则不能知雷霆之害,狂则不能免人间法令之祸。书之所谓"治人"者,适动静之节,省思虑之费也。所谓"事天"者,不极聪明之力,不尽智识之任。苟极尽,则费神多;费神多,则盲聋悖狂之祸至,是以啬之。啬之者,爱其精神,啬其智识也。故曰:"治人事天莫如啬。"

众人之用神也躁,躁则多费,多费之谓侈。圣人之用神也静,静则少费,少费之谓啬。啬之谓术也,生于道理。夫能啬也,是从于道而服于理者也。众人离于患,陷于祸,犹未知退,而不服从道理。圣人虽未见祸患之形,虚无服从于道理,以称蚤服。故曰:"夫谓啬,是以蚤服。"知治人者,其思虑静;知事天者,其孔窍虚。思虑静,故德不去;孔窍虚,则和气日入。故曰:"重积德。"夫能令故德不去,新和气日至者,蚤服者也。故曰:"蚤服,是谓重积德。"积德而后神静,神静而后和多,和多而后计得,计得而后能御万物,能御万物则战易胜敌,战易胜敌而论必盖世,论必盖世,故曰"无不克。"无不克本于重积德,故曰"重积德,则无不克。"战易胜敌,则兼有天下;论必盖世,则民人从。进兼有天下而退从民人,其术远,则众人莫见其端末。莫见其端末,是以莫知其极。故曰:"无不克,则莫知其极。"

凡有国而后亡之,有身而后殃之,不可谓能有其国、能保其身。夫能有其国,必能安其社稷;能保其身,必能终其天年;而后可谓能有其国、能保其身矣。夫能有其国、保其身者,必且体道。体道,则其智深;其智深,则其会远;其会远,众人莫能见其所极。唯夫能令人不见其事极,不见其事极者为保其身、有其国。故曰:"莫知其极。莫知其极,则可以有国。"

所谓"有国之母":母者,道也;道也者,生于所以有国之术;所以有国之术,故谓之"有国之母。"夫道以与世周旋者,其建生也长,持禄也久。故曰:"有国之母,可以长久。"树木有曼根,有直根。直根者,书之所谓"柢"也。柢也者,木之所以建生也;曼根者,木之所以持生也。德也者,人之所以建生也;禄也者,人之所以持生也。今建于理者,其持禄也久,故曰:"深其根。"体其道者,其生日长,故曰:"固其柢。"柢固,则生长;根深,则视久,故曰:"深其根,固其柢,长生久视之道也。"

工人数变业则失其功,作者数摇徙则亡其功。一人之作,日亡半日,十日则亡五人之功矣;万人之作,日亡半日,十日则亡五万人之功矣。然则数变业者,其人弥众,其亏弥大矣。凡法令更则利害易,利害易则民务变,民务变谓之变业。故以理观之,事大众而数摇之,则少成功;藏大器而数徙之,则多败伤;烹小鲜而数挠之,则贼其宰;治大国而数变法,则民苦之。是以有道之君贵静,不重变法。故曰:"治大国者若烹小鲜。"

人处疾则贵医,有祸则畏鬼。圣人在上,则民少欲;民少欲,则血气治而举动理;举动理则少祸害。夫内无痤疽瘅痔之害,而外无刑罚法诛之祸者,其轻恬鬼也甚。故曰:"以道莅天下,其鬼不神。"治世之民,不与鬼神相害也。故曰:"非其鬼不神也,其神不伤人也。"鬼祟也疾人之谓鬼伤人,人逐除之之谓人伤鬼也。民犯法令之谓民伤上,上刑戮民之谓上伤民。民不犯法,则上亦不行刑;上不行刑之谓上不伤人,故曰:"圣人亦不伤民。"上不与民相害,而人不与鬼相伤,故曰:"两不相伤。"民不敢犯法,则上内不用刑罚,而外不事利其产业。上内不用刑罚,而外不事利其产业,则民蕃息。民蕃息而畜积盛。民蕃息而畜积盛之谓有德。凡所谓祟者,魂魄去而精神乱,精神乱则无德。鬼不祟人则魂魄不去,魂魄不去而精神不乱,精神不乱之谓有德。上盛畜积而鬼不乱其精神,则德尽在于民矣。故曰:"两不相伤,则德交归焉。"言其德上下交盛而俱归于民也。

有道之君,外无怨仇于邻敌,而内有德泽于人民。夫外无怨仇于邻敌者,其遇诸侯也外有礼义。内有德泽于人民者,其治人事也务本。遇诸侯有礼义,则役希起;治民事务本,则淫奢止。凡马之所以大用者,外供甲兵而内给淫奢也。今有道之君,外希用甲兵,而内禁淫奢。上不事马于战斗逐北,而民不以马远通淫物,所积力唯田畴。积力于田畴,必且粪灌。故曰:"天下有道,却走马以粪也。"

人君无道,则内暴虐其民而外侵欺其邻国。内暴虐,则民产绝;外侵欺,则兵数起。民产绝,则畜生少;兵数起,则士卒尽。畜生少,则戎马乏;士卒尽,则军危殆。戎马乏则将马出;军危殆,则近臣役。马者,军之大用;郊者,言其近也。今所以给军之具于谞马近臣。故曰:"天下无道,戎马生于郊矣。"

人有欲,则计会乱;计会乱,而有欲甚;有欲甚,则邪心胜;邪心胜,则事经绝;事经绝,则祸难生。由是观之,祸难生于邪心,邪心诱于可欲。可欲之类,进则教良民为奸,退则令善人有祸。奸起,则上侵弱君;祸至,则民人多伤。然则可欲之类,上侵弱君而下伤人民。夫上侵弱君而下伤人民者,大罪也。故曰:"祸莫大于可欲。"是以圣人不引五色,不淫于声乐;明君贱玩好而去淫丽。

人无毛羽,不衣则不犯寒;上不属天而下不著地,以肠胃为根本,不食则不能活;是以不免于欲利之心。欲利之心不除,其身之忧也。故圣人衣足以犯寒,食足以充虚,则不忧矣。众人则不然,大为诸侯,小余千金之资,其欲得之忧不除也。胥靡有免,死罪时活,今不知足者之忧终身不解。故曰:"祸莫大于不知足。"

故欲利甚于忧,忧则疾生;疾生而智慧衰;智慧衰,则失度量;失度量,则妄举动;妄举动,则祸害至;祸害至而疾婴内;疾婴内,则痛,祸薄外;则苦。苦痛杂于肠胃之间;苦痛杂于肠胃之间,则伤人也惨。惨则退而自咎,退而自咎也生于欲利。故曰:"咎莫惨于欲利。"

道者,万物之所然也,万理之所稽也。理者,成物之文也;道者,万物之所以成也。故曰:"道,理之者也。"物有理,不可以相薄;物有理不可以相薄,故理之为物之制。万物各异理,万物各异理而道尽。稽万物之理,故不得不化;不得不化,故无常操。无常操,是以死生气禀焉,万智斟酌焉,万事废兴焉。天得之以高,地得之以藏,维斗得之以成其威,日月得之以恆其光,五常得之以常其位,列星得之以端其行,四时得之以御其变气,轩辕得之以擅四方,赤松得之与天地统,圣人得之以成文章。道,与尧、舜俱智,与接舆俱狂,与桀、纣俱灭,与汤、武俱昌。以为近乎,游于四极;以为远乎,常在吾侧;以为暗乎,其光昭昭;以为明乎,其物冥冥;而功成天地,和化雷霆,宇内之物,恃之以成。凡道之情,不制不形,柔弱随时,与理相应。万物得之以死,得之以生;万事得之以败,得之以成。道譬诸若水,溺者多饮之即死,渴者适饮之即生;譬之若剑戟,愚人以行忿则祸生,圣人以诛暴则福成。故得之以死,得之以生,得之以败,得之以成。

人希见生象也,而得死象之骨,案其图以想其生也,故诸人之所以意想者皆谓之"象"也。今道虽不可得闻见,圣人执其见功以处见其形,故曰:"无状之状,无物之象"。

凡理者,方圆、短长、粗靡、坚脆之分也,故理定而后可得道也。故定理有存亡,有死生,有盛衰。夫物之一存一亡,乍死乍死,初盛而后衰者,不可谓常。唯夫与天地之剖判也俱生,至天地之消散也不死不衰者谓"常"。而常者,无攸易,无定理。无定理,非在于常所,是以不可道也。圣人观其玄虚,用其周行,强字之曰"道",然而可论。故曰:"道之可道,非常道也"。

人始于生而卒于死。始之谓出,卒之谓入。故曰:"出生入死"。人之身三百六十节,四肢、九窍其大具也。四肢九窍十有三者,十有三者之动静尽属于生焉。属之谓徒也,故曰:生之徒也十有三者。至死也,十有三具者皆还而属之于死,死之徒亦有十三。"故曰:"生之徒十有三,死之徒十有三。"凡民之生生而和固动,动尽则损也;而动不止,是损而不止也。损而不止则生尽,生尽之谓死,则十有三具者皆为死死地也。故曰:"民之生,生而动,动皆之死地,亦十有三。"

是以圣人爱精神而贵处静。不爱精神不贵处静,此甚大于兕虎之害。夫兕虎有域,动静有时。避其域,省其时,则免其兕虎之害矣。民独知兕虎之有爪角也,而莫知万物之尽有爪角也,不免于万物之害。何以论之?时雨降集,旷野闲静,而以昏晨犯山川,则风露之爪角害之。事上不忠,轻犯禁令,则刑法之爪角害之。处乡不节,憎爱无度,则争斗之爪角害之。嗜欲无限,动静不节,则痤疽之爪角害之。好用其私智而弃道理,则纲罗之爪角害之。兕虎有域,而万害有原,避其域,塞其原,则免于诸害矣。凡兵革者,所以备害也。重生者,虽入军无忿争之心;无忿争之心,则无所用救害之备。此非独谓野处之军也。圣人之游世也,无害人之心,无害人之心,则必无人害,无人害,则不备人。故曰:"陆行不遇兕虎。"入山不特备以救害,故曰:"入军不备甲兵。"远诸害,故曰"兕无所投其角,虎匏砥渥Γ匏萜淙小?quot;不设备而必无害,天地之道理也。体天地之道,故曰:"无死地焉。"动无死地,而谓之"善摄生"矣。

爱子者慈于子,重生者慈于身,贵功者慈于事。慈母之于弱子也,务致其福;务致其福,则事除其祸;事除其祸,则思虑熟;思虑熟,则得事理;得事理,则必成功;必成;工,则其行之也不疑;不疑之谓勇。圣人之于万事也,尽如慈母之为弱子虑也,故见必行之道。见必行之道则其从事亦不疑;不疑之谓勇。不疑生于慈,故曰:"慈,故能勇。"

周公曰:"冬日之闭冻也不固,则春夏之长草木也不茂。"天地不能常侈常费,而况于人乎?故万物必有盛衰,万事必有弛张,国家必有文武,官治必有赏罚。是以智士俭用其财则家富,圣人爱宝其神则精盛,人君重战其卒则民众,民众则国广。是以举之曰:"俭,故能广。"

凡物之有形者易裁也,易割也。何以论之?有形,则有短长;有短长,则有小大;有小大,则有方圆;有方圆,则有坚脆;有坚脆,则有轻重;有轻重,则有白黑。短长、大小、方圆、坚脆、轻重、白黑之谓理。理定而物易割也。故议于大庭而后言则立,权议之士知之矣。故欲成方圆而随其规矩,则万事之功形矣。而万物莫不有规矩,议言之士,计会规矩也。圣人尽随于万物之规矩,故曰:"不敢为天下先。"不敢为天下先,则事无不事,功无不功,而议必盖世,欲无处大官,其可得乎?处大官之谓为成事长。是以故曰:"不敢为天下先,故能为成事长。"

慈于子者不敢绝衣食,慈于身者不敢离法度,慈于方圆者不敢舍规矩。故临兵而慈于士吏则战胜敌,慈于器械则城坚固。故曰:"慈,于战则胜,以守则固。"夫能自全也而尽随于万物之理者,必且有天生。天生也者,生心也,故天下之道尽之生也。若以慈卫之也,事必万全,而举无不当,则谓之宝矣。故曰:"吾有三宝,持而宝之。"

书之所谓"大道"也者,端道也。所谓"貌施"也者,邪道也。所谓"径大"也者,佳丽也。佳丽也者,邪道之分也。"朝甚除"也者,狱讼繁也。狱讼繁,则田荒;田荒,则府仓虚;府仓虚,则国贫;国贫,而民俗淫侈;民俗淫侈,则衣食之业绝;衣食之业绝,则民不得无饰巧诈;饰巧诈,则知采文;知采文之谓"服文采"。狱讼繁仓廪虚,而有以淫侈为俗,则国之伤也,若以利剑刺之。故曰:"带利剑。"诸夫饰智故以至于伤国者,其私家必富;私家必富,故曰:"资货有馀。"国有若是者,则愚民不得无术而效之;效之,则小盗生。由是观之,大奸作则小盗随,大奸唱则小盗和。竽也者,五声之长者也,故竽先则钟瑟皆随,竽唱则诸乐皆和。今大奸作则俗之民唱,俗之民唱则小盗必和。故"服文采,带利剑,厌饮食,而货资有馀者,是之谓盗竽矣。"

人无愚智,莫不有趋舍。恬淡平安,莫不知祸福之所由来。得于好恶,怵于淫物,而后变乱。所以然者,引于外物,乱于玩好也。恬淡有趋舍之义,平安知祸福之计。而今也玩好变之,外物引之;引之而往,故曰"拔"。至圣人不然:一建其趋舍,虽见所好之物,能引,不能引之谓"不拔";一于其情,虽有可欲之类,神不为动,神不为动之谓"不脱"。为人子孙者,体此道以守宗庙,宗庙不灭之谓"祭祀不绝"。身以积精为德,家以资财为德,乡国天下皆以民为德。今治身而外物不能乱其精神,故曰:"修之身,其德乃真。"真者,慎之固也。治家者,无用之物不能动其计,则资有馀,故曰:"修之家,其德有馀。"治乡者行此节,则家之有馀者益众,故曰:"修之乡,其德乃长。"治邦者行此节,则乡之有德者益众,故曰:"修之邦,其德乃丰。"莅天下者行此节,则民之生莫不受其泽,故曰:"修之天下,其德乃普。"修身者以此别君子小人,治乡治邦莅天下者名以此科适观息耗,则万不失一。故曰:"以身观身,以家观家,以乡观乡,以邦观邦,以天下观天下。吾奚以知天下之然也?以此。"

\hypertarget{header-n970}{%
\subsection{喻老}\label{header-n970}}

天下有道,无急患,则曰静,遽传不用。故曰:"却走马以粪。"天下无道,攻击不休,相守数年不已,甲胄生虮虱,燕雀处帷幄,而兵不归。故曰:"戎马生于郊。"

翟人有献丰狐、玄豹之皮于晋文公。文公受客皮而叹曰:"此以皮之美自为罪。"夫治国者以名号为罪,徐偃王是也;以城与地为罪,虞、虢是也。故曰:"罪莫大于可欲。"

智伯兼范、中行而攻赵不已,韩、魏反之,军败晋阳,身死高梁之东,遂卒被分,漆其首以为溲器。故曰:"祸莫大于不知足。"

虞君欲屈产之乘与垂棘之璧,不听宫之奇,故邦亡身死。故曰:"咎莫惨于欲得。"

邦以存为常,霸王其可也;身以生为常,富贵其可也。不以欲自害,则邦不亡,身不死。故曰:"知足之为足矣。"

楚庄王既胜,狩于河雍,归而赏孙叔敖。孙叔敖请汉间之地,沙石之处。楚邦之法,禄臣再世而收地,唯孙叔敖独在。此不以其邦为收者,瘠也,故九世而祀不绝。故曰:"善建不拔,善抱不脱,子孙以其祭祀,世世不辍。"孙叔敖之谓也。

制在己曰重,不离位曰静。重则能使轻,静则能使躁。故曰:"重为轻根,静为躁君。"故曰:"君子终日行,不离辎重也"。邦者,人君之辎重也。主父生传其邦,此离其辎重者也,故虽有代、云中之乐,超然已无赵矣。主父,万乘之主,而以身轻于天下。无势之谓轻,离位之谓躁,是以生幽而死。故曰:"轻则失臣,躁则失君。"主父之谓也。

势重者,人君之渊也。君人者,势重于人臣之间,失则不可复得矣。简公失之于田成,晋公失之于六卿,而上亡身死。故曰:"鱼不可脱于深渊。"赏罚者,邦之利器也,在君则制臣,在臣则胜君。君见赏,臣则损之以为德;君见罚,臣则益之以为威。人君见赏,则人臣用其势;人君见罚,而人臣乘其威。故曰:"邦之利器,不可以示人。"

越王入宦于吴,而观之伐齐以弊吴。吴兵既胜齐人于艾陵,张之于江、济,强之于黄池,故可制于五湖。故曰:"将欲翕之,必固张之;将欲弱之,必固强之。"晋献公将欲袭虞,遗之以璧马;知伯将袭仇由,遗之以广车。故曰:"将欲取之,必固与之。"起事于无形,而要大功于天下,"是谓微明"。处小弱而重自卑,谓"损弱胜强也。"

有形之类,大必起于小;行久之物,族必起于少。故曰:"天下之难事必作于易,天下之大事必作于细。"是以欲制物者于其细也。故曰:"图难于其易也,为大于其细也。"千丈之堤,以蝼蚁之穴溃;百步之室,以突隙之烟焚。故曰:白圭之行堤也塞其穴,丈人之慎火也涂其隙,是以白圭无水难,丈人无火患。此皆慎易以避难,敬细以远大者也。扁鹊见蔡桓公,立有间。扁鹊曰:"君有疾在腠理,不治将恐深。"桓侯曰:"寡人无疾。"扁鹊出。桓侯曰:"医之好治不病以为功。"居十日,扁鹊复见曰:"君之病在肌肤,不治将益深。"桓侯不应。扁鹊出。桓侯又不悦。居十日,扁鹊复见曰:"君之病在肠胃,不治将益深。"桓侯又不应。扁鹊出。桓侯又不悦。居十日,扁鹊望桓侯而还走,桓侯故使人问之。扁鹊曰:"病在腠理,汤熨之所及也;在肌肤,针石之所及也;在肠胃,火齐之所及也;在骨髓,司命之所属,无奈何也。今在骨髓,臣是以无请也。"居五日,桓侯体痛,使人索扁鹊,已逃秦矣。桓侯遂死。故良医之治病也,攻之于腠理。此皆争之于小者也。夫事之祸福亦有腠理之地,故圣人蚤从事焉。

昔晋公子重耳出亡,过郑,郑君不礼。叔瞻谏曰:"此贤公子也,君厚待之,可以积德。"郑君不听。叔瞻又谏曰:"不厚待之,不若杀之,无令有后患。"郑君又不听。及公子返晋邦,举兵伐郑,大破之,取八城焉。晋献公以垂棘之璧假道于虞而伐虢,大夫宫之奇谏曰:"不可。脣亡而齿寒,虞、虢相救,非相德也。今日晋灭虢,明日虞必随之亡。"虞君不听,受其璧而假之道。晋已取虢,还,反灭虞。此二臣者皆争于腠理者也,而二君不用也。然则叔瞻、宫之奇亦虞、虢之扁鹊也,而二君不听,故郑以破,虞以亡。故曰:"其安易持也,其未兆易谋也。"

昔者纣为象箸而箕子怖,以为象箸必不加于土铏,必将犀玉之杯;象箸玉杯必不羹菽藿,必旄、象、豹胎;旄、象、豹胎必不衣短褐而食于茅屋之下,则锦衣九重,广室高台。吾畏其卒,故怖其始。居五年,纣为肉圃,设砲烙,登糟丘,临酒池,纣遂以亡。故箕子见象箸以知天下之祸。故曰:"见小曰明。"

勾践入宦于吴,身执干戈为吴王洗马,故能杀夫差于姑苏。文王见詈于王门,颜色不变,而武王擒纣于牧野。故曰:"守柔曰强。"越王之霸也不病宦,武王之王也不病詈。故曰:"圣人之不病也,以其不病,是以无病也。"

宋之鄙人得璞玉而献之子罕,子罕不受。鄙人曰:"此宝也,宜为君子器,不宜为细人用。"子罕曰:"尔以玉为宝,我以不受子玉为宝。"是以鄙人欲玉,而子罕不欲玉。故曰:"欲不欲,而不贵难得之货。"

王寿负书而行,见徐冯于周涂。冯曰:"事者,为也;为生于时,知者无常事。书者,言也;言生于知,知者不藏书。今子何独负之而行?"于是王寿因焚其书而舞之。故知者不以言谈教,而慧者不以藏书箧。此世之所过也,而王寿复之,是学不学也。故曰:"学不学,复归众人之所过也。"

夫物有常容,因乘以导之。因随物之容,故静则建乎德,动则顺乎道。宋人有为其君以象为楮叶者,三年而成。丰杀茎柯,毫芒繁泽,乱之楮叶之中而不可别也。此人遂以功食禄于宋邦。列子闻之曰:"使天地三年而成一叶,则物之有叶者寡矣。"故不乘天地之资而载一人之身,不随道理之数而学一人之智,此皆一叶之行也。故冬耕之稼,后稷不能羡也;丰年大禾,臧获不能恶也。以一人之力,则后稷不足;随自然,则臧获有余。故曰:"恃万物之自然而不敢为也。"

空窍者,神明之户牖也。耳目竭于声色,精神竭于外貌,故中无主。中无主,则祸福虽如丘山,无从识之。故曰:"不出于户,可以知天下;不窥于牖,可以知天道。"此言神明之不离其实也。

赵襄主学御于王子于期,俄而与于期逐,三易马而三后。襄主曰:"子之教我御,术未尽也?"对曰:"术已尽,用之则过也。凡御之所贵:马体安于车,人心调于马,而后可以进速致远。今君后则欲逮臣,先则恐逮于臣。夫诱道争远,非先则后也,而先后心皆在于臣,上何以调于马?此君之所以后也。"

白公胜虑乱,罢朝,倒杖而策锐贯颐,血流至于地而不知。郑人闻之曰:"颐之忘,将何不忘哉!"故曰:"其出弥远者,其智弥少。"此言智周乎远,则所遗在近也。是以圣人无常行也。能并智,故曰:"不行而知。"能并视,故曰:"不见而明。"随时以举事,因资而立功,用万物之能而获利其上,故曰:"不为而成。"

楚庄王莅政三年,无令发,无政为也。右司马御座而与王隐曰:"有鸟止南方之阜,三年不翅,不飞不鸣,嘿然无声,此为何名?"王曰:"三年不翅,将以长羽翼;不飞不鸣,将以观民则。虽无飞,飞必冲天;虽无鸣,鸣必惊人。子释之,不谷知之矣。"处半年,乃自听政。所废者十,所起者九,诛大臣五,举处士六,而邦大治。举兵诛齐,败之徐州,胜晋于河雍,合诸侯于宋,遂霸天下。庄王不为小害善,故有大名;不蚤见示,故有大功。故曰:"大器晚成,大音希声。"

楚庄王欲伐越,杜子谏曰:"王之伐越,何也?"曰:"政乱兵弱。"庄子曰:"臣患智之如目也,能见百步之外而不能自见其睫。王之兵自败于秦、晋,丧地数百里,此兵之弱也。庄跷为盗于境内而吏不能禁,此政之乱也。王之弱乱,非越之下也,而欲伐越,此智之如目也。"王乃止。故知之难,不在见人,在自见。故曰:"自见之谓明。"子夏见曾子。曾子曰:"何肥也?"对曰:"战胜,故肥也。"曾子曰:"何谓也?"子夏曰:"吾入见先王之义则荣之,出见富贵之乐又荣之,两者战于胸中,未知胜负,故癯。今先王之义胜,故肥。"是以志之难也,不在胜人,在自胜也。故曰:"自胜之谓强。"

周有玉版,纣令胶鬲索之,文王不予;费仲来求,因予之。是胶鬲贤而费仲无道也。周恶贤者之得志也,故予费仲。文王举太公于渭滨者,贵之也;而资费仲玉版者,是爱之也。故曰:"不贵其师,不爱其资,虽知大迷,是谓要妙。"

\hypertarget{header-n993}{%
\subsection{说林上}\label{header-n993}}

汤以伐桀,而恐天下言己为贪也,因乃让天下于务光。而恐务光之受之也,乃使人说务光曰:"汤杀君,而欲传恶声于子,故让天下于子。"务光因自投于河。

秦武王令甘茂择所欲为于仆与行事,孟卯曰:"公不如为仆。公所长者使也。公虽为仆,王犹使之于公也。公佩仆玺而为行事,是兼官也。"

子圉见孔子于商太宰。孔子出,子圉入,请问客。太宰曰:"吾已见孔子,则视子犹蚤虱之细者也。吾今见之于君。"子圉恐孔子贵于君也,因谓太宰曰:"君已见孔子,亦将视子犹蚤虱也。"太宰因弗复见也。

魏惠王为臼里之盟,将复立于天子。彭喜谓郑君曰:"君勿听。大国恶有天子,小国利之。若君与大不听,魏焉能与小立之?"

晋人伐邢,齐桓公将救之。鲍叔曰:"太蚤。邢不亡,晋不敝;晋不敝,齐不重。且夫持危之功,不如存亡之德大。君不如晚救之以敝晋,齐实利;待邢亡而复存之,其名实美。"桓公乃弗救。

子胥出走,边候得之。子胥曰:"上索我者,以我有美珠也。今我已亡之矣。我且曰子取吞之。"候因释之。

庆封为乱于齐而欲走越。其族人曰:"晋近,奚不之晋?"庆封曰:"越远,利以避难。"族人曰:"变是心也,居晋而可;不变是心也,虽远越,其可以安乎?"

智伯索地于魏宣子,魏宣子弗予。任章曰:"何故不予?"宣子曰:"无故请地,故弗予。"任章曰:"无故索地,邻国必恐。彼重欲无厌,天下必惧。君予之地,智伯必骄而轻敌,邻邦必惧而相亲。以相亲之兵待轻敌之国,则智伯之命不长矣。《周书》曰:\textbackslash{}'将欲败之,必姑辅之;将欲取之,必姑予之。\textbackslash{}'君不如予之以骄智伯。且君何释以天下图智氏,而独以吾国为智氏质乎?"君曰:"善。"乃与之万户之邑。智伯大悦,因索地于赵,弗与,因围晋阳。韩、魏反之外,赵氏应之内,智氏以亡。

秦康公筑台三年。荆人起兵,将欲以兵攻齐。任妄曰:"饥召兵,疾召兵,劳召兵,乱召兵。君筑台三年,今荆人起兵将攻齐,臣恐其攻齐为声,而以袭秦为实也,不如备之。"戍东边,荆人辍行。

齐攻宋,宋使臧孙子南求救于荆。荆大说,许救之,甚欢。臧孙子忧而反。其御曰:"索救而得,今子有忧色,何也?"臧孙子曰:"宋小而齐大。夫救小宋而恶于大齐,此人之所以忧也;而荆王说,必以坚我也。我坚而齐敝,荆之所利也。"臧孙子乃归。齐人拔五城于宋而荆救不至。

魏文侯借道于赵而攻中山,赵肃侯将不许。赵刻曰:"君过矣。魏攻中山而弗能取,则魏必罢。罢则魏轻,魏轻则赵重。魏拔中山,必不能越赵而有中山也。是用兵者魏也,而得地者赵也。君必许之。许之而大欢,彼将知君利之也,必将辍行。君不如借之道,示以不得已也。"

鸱夷子皮事田成子,田成子去齐,走而之燕,鸱夷子皮负传而从。至望邑,子皮曰:"子独不闻涸泽之蛇乎?泽涸,蛇将徙。有小蛇谓大蛇曰:"子行而我随之,人以为蛇之行者耳,必有杀子者。不如相衔负我以行,人以我为神君也。乃相衔负以越公道。人皆避之,曰:'神君也。'今子美而我恶,以子为我上客,千乘之君也;以子为我使者,万乘之卿也。子不如为我舍人。"田成子因负传而随之。至逆旅,逆旅之君待之甚敬,因献酒肉。

温人之周,周不纳客。问之曰:"客耶?"对曰:"主人。"问其巷人而不知也,吏因囚之。君使人问之曰:"子非周人也,而自谓非客,何也?"对曰:"臣少也诵《诗》,曰:'普天之下,莫非王土;率土之滨,莫非王臣。'今君天子,则我天子之臣也。岂有为人之臣而又为之客哉?故曰:主人也。"君使出之。

韩宣王谓樛留曰:"吾欲两用公仲、公叔,其可乎?"对曰:"不可。晋用六卿而国分,简公两用田成、阚止而简公杀魏两用犀首、张仪,而西河之外亡。今王两用之,其多力者树其党,寡力者借外权。群臣有内树党以骄主,有外为交以削地,则王之国危矣。"

绍绩味醉寐而亡其裘。宋君曰:"醉足以亡裘乎?"对曰:"桀以醉亡天下,而《康诰》曰:'毋彝酒。'彝酒者,常酒也。常酒者,天子失天下,匹夫失其身。"

管仲、隰朋从于桓公而伐孤竹,春往冬反,迷惑失道。管仲曰:"老马之智可用也。"乃放老马而随之,遂得道。行山中无水,隰朋曰:"蚁冬居山之阳,夏居山之阴。蚁壤一寸而有水。"乃掘地,遂得水。以管仲之圣而隰朋之智,至其所不知,不难师于老马与蚁。今人不知以其愚心而师圣人之智,不亦过乎?

有献不死之药于荆王者,谒者操之以入。中射之士问曰:"可食乎?"曰:"可。"因夺而食之。王大怒,使人杀中射之士。中射之士使人说王曰:"臣问谒者,曰'可食',臣故食之,是臣无罪,而罪在谒者也。且客献不死之药,臣食之而王杀臣,是死药也,是客欺王也。夫杀无罪之臣,而明人之欺王也,不如释臣。"王乃不杀。

田驷欺邹君,邹君将使人杀之。田驷恐,告惠子。惠子见邹君曰:"今有人见君,则夹其一目,奚如?"君曰:"我必杀之。"惠子曰:"瞽两目夹,君奚为不杀?"君曰:"不能勿夹。"惠子曰:"田驷东欺齐侯,南欺荆王,驷之于欺人,瞽也,君奚怨焉?"邹君乃不杀。

鲁穆公使众公子或宦于晋,或宦于荆。犁鉏曰:"假人于越而救溺子,越人虽善游,子必不生矣。失火而取水于海,海水虽多,火必不灭矣,远水不救近火也。今晋与荆虽强,而齐近,鲁患其不救乎!"

严遂不善周君,患之。冯沮曰:"而韩傀贵于君。不如行贼于韩傀,则君必以为严氏也。"

张谴相韩,病将死。公乘无正怀三十金而问其疾。居一日,君问张谴曰:"若子死,将谁使代子?"答曰:"无正重法而畏上。虽然,不如公子食我之得民也。"张谴死,因相公乘无正。

乐羊为魏将而攻中山,其子在中山,中山之君烹其子而遗之羹。乐羊坐于幕下而啜之,尽一杯。文候谓堵师赞曰:"乐羊以我故而食其子之肉。"答曰:"其子而食之,且谁不食?"乐羊罢中山,文候赏其功而疑其心。孟孙猎得鹿,使秦西巴持之归,其母随之而啼。秦西巴弗忍而与之。孟孙适,至而求鹿。答曰:"余弗忍而与其母。"孟孙大怒,逐之。居三月,复召以为其子传。其御曰:"曩将罪之,今召以为子传,何也?"孟孙曰:"夫不忍鹿,又且忍吾子乎?"故曰:"巧诈不如拙诚。"乐羊以有功见疑,秦古巴以有罪益信。

曾从子,善相剑者也。卫君怨吴王。曾从子曰:"吴王好剑,臣相剑者也。臣请为吴王相剑,拔而示之,因为君刺之。"卫君曰:"子之为是也,非缘义也,为利也。吴强而富,卫弱而贫。子必往,吾恐子为吴王用之于我也。"乃逐之。

纣为象箸而箕子怖,以为象箸必不盛羹于土铏,则必将犀玉之杯;玉杯象箸必不盛菽藿,则必旄象豹胎;旄象豹胎必不衣短褐而舍茅茨之下,则必锦衣九重,高台广室也。称此以求,则天下不足矣。圣人见微以知萌,见端以知末,故见象箸而怖,知天下之不足也。

周公旦已胜殷,将攻商盖。辛公甲曰:"大难攻,小易服。不如服众小以劫大。"乃攻九夷而商盖服矣。

纣为长夜之饮,欢以失日,问其左右,尽不知也。乃使人问箕子。箕子谓其徒曰:"为天下主而一国皆失日,天下其危矣。一国皆不知而我独知之,吾其危矣。"辞以醉而不知。

鲁人身善织屦,妻善织缟,而欲徙于越。或谓之曰:"子必穷矣。"鲁人曰:"何也?"曰:"屦为履之也,而越人跣行;缟为冠之也,而越人被发。以子之所长,游于不用之国,欲使无穷,其可得乎?"

陈轸贵于魏王。惠子曰:"必善事左右。夫杨,横树之即生,倒树之即生,折而树之又生。然使十人树之而一人拔之,则毋生杨。至以十人之众,树易生之物而不胜一人者,何也?树之难而去之易也。子虽工自树于王,而欲去子者从,子必危矣。"

鲁季孙新弑其君,吴起仕焉。或谓起曰:"夫死者始死而血,已血而衄,已衄而灰,已灰而土。及其土也,无可为者矣。今季孙乃始血,其毋乃未可知也。"吴起因去之晋。

隰斯弥见田成子,田成子与登台四望。三面皆暢,南望,隰子家之树蔽之。田成子亦不言。隰子归,使人伐之;斧离数创,隰子止之。其相室曰:"何变之数也?"隰子曰:"古者有谚曰:'知渊中之鱼者不祥。'夫田子将有大事,而我示之知微,我必危矣。不伐树,未有罪也;知人之所不言,其罪大矣。"乃不伐也。

杨子过于宋,东之逆旅,有妾二人,其恶者贵,美者贱。杨子问其故。逆旅之父答曰:"美者自美,吾不知其美也;恶者自恶,吾不知其恶也。"杨子谓弟子曰:"行贤而自贤之心,焉往而不美。"

卫人嫁其子而教之曰:"必私积聚。为人妇而出,常也;其成居,幸也。"其子因私积聚,其姑以为多私而出之。其子所以反者倍其所以嫁。其父不自罪于教子非也,而自知其益富。念人臣之处官者,皆是类也。

鲁丹三说中山之君而不受也,因散五十金事其左右。复见,未语,而君与之食。鲁丹出,而不反舍,遂去中山。其御曰:"及见,乃始善我。何故去之?"鲁丹曰:"夫以人言善我,必以人言罪我。"未出境,而公子恶之曰:"为赵来间中山。"君因索而罪之。

田伯鼎好士而存其君,白公好士而乱荆。其好士则同,其所以为则异。公孙友自刖而尊百里,竖刁自宫而谄桓公。其自刑则同,其所以自刑之为则异。慧子曰:"狂者东走,逐者亦东走。其东走则同,其所以东走之为则异。故曰:同事之人,不可不审察也。"

\hypertarget{header-n1028}{%
\subsection{说林下}\label{header-n1028}}

伯乐教二人相踶马,相与之简子厩观马。一人举踶马。其一人从后而循之,三抚其尻而马不踢。此自以为失相。其一人曰:``子非失相也,此其为马也,踒肩而肿膝。夫踢马也者,举后而任前,肿膝不可任也,故后不举。子巧于相踢马拙于任肿膝。''夫事有所必归,而以有所肿膝而不任,智者之所独知也。惠子曰:``置猿于柙中,则与豚同。''故势不便,非所以逞能也。

卫将军文子见曾子,曾子不起而延于坐席,正身于奥。文子谓其御曰:``曾子,愚人也哉!以我为君子也,君子安可毋敬也?以我为暴人也,暴人安可侮也?曾子不戮,命也。''

鸟有周周者,重首而屈尾,将欲饮于河,则必颠,乃衔其羽而饮之,人之所有饮不足者,不可不索其羽也。鳣似蛇,蚕似蠋,人见蛇则惊核,见蠋,则毛起。渔者持鳣,妇人拾蚕,利之所在,皆为贲、诸。

伯乐教其所憎者相千里之马,教其所爱者相驭马。千里之马时一有,其利缓;驭马日售,其利急。此《周书》所谓``下言而上用者,惑也。''

桓赫曰:``刻削之道,鼻莫如大,目莫如小。鼻大可小,小不可大也;目小可大,大不可小也。''举事亦然:为其后可复者也,则事寡败矣。

崇候、恶来知不适纣之诛也,而不见武王之灭之也。比干、子胥知其君之必亡也,而不知身之死也。故曰:``崇候、恶来知心而不知事,比干、子胥知事而不知心。''圣人其备矣。

宋太宰贵而主断。季子将见宋君,梁子闻之曰:``语必可与太宰三坐乎,不然,将不免。''季子因说以贵主而轻国。

杨朱之弟杨布,衣素衣而出,天雨,解素衣,衣缁衣而反,其狗不知而吠之。杨布怒,将击之。杨硃曰:``子勿击也,子亦犹是。曩者使女狗白而往,墨而来,子岂能毋怪哉?``

惠子曰:羿执鞅持扞,操弓关机,越人争为持的。弱子扞弓,慈母入室闭户。''故曰:``可必,则越人不疑羿;不可必,则慈母逃弱子。''

桓公问管仲:``富有涯乎?``答曰:``水之以涯,其无水者也;富之以涯,其富已足者也。人不能自止于足,而亡其富之涯乎!``

宋之富贾有监止子者,与人争买百金之璞玉,因佯失而毁之,负其百金,而理其毁瑕,得千溢焉。事有举之而有败,而贤其毋举之者,负之时也。

有欲以御见荆王者,众驺妒之。因曰:``臣能撽鹿``见王,王为御,不及鹿;自御,及之。王善其御也,乃言众驺妒之。

荆令公子将伐陈。丈人送之曰:``晋强,不可不慎也。''公子曰:``丈人奚忧!吾为丈人破晋。''丈人曰:``可。吾方庐陈南门之外。''公子曰:``是何也?``曰:``我笑勾践也。为人之如是其易也,己独何为密密十年难乎?``

尧以天下让许由,许由逃之,舍于家人,家人藏其皮冠,夫弃天下而家人藏其皮冠,是不知许由者也。

三虱食彘相与讼,一虱过之,曰:``讼者奚说?``三虱曰:``争肥饶之地。''一虱曰:``若亦不患腊之至而茅之燥耳,其又奚患?``于是乃相与聚嘬其身而食之。彘臞,人乃弗杀。

虫有虺者,一身两口,争食相龁遂相杀也,人臣之争事而亡其国者,皆虺类也。

宫有垩,器有涤,则洁矣。行身亦然,其无垩之地则寡非矣。

公子纠将为乱,桓公使使者视之。使者报曰:``笑不乐,视不见,必为乱。''乃使鲁人杀之。

公孔弘断发而为越王骑,公孔喜使人绝之曰:``吾不与子为昆弟矣。''公孙弘曰:``我断发,子断颈而为人用兵,我将谓之何?``周南之战,公孙喜死焉。

有与悍者邻,欲卖宅而避之。人曰:``是其贯将满矣,子姑待之。''答曰:``吾恐其以我满贯也。''遂去之。故曰:``物之几者,非所靡也。''

孔子谓弟子曰:``孰能道子西之钓名也?``子贡曰:``赐也能。''乃导之,不复疑也。孔子曰:``宽哉,不被于利!洁哉,民性有恆!曲为曲,直为直。孔子曰子西不免。''白公之难,子西死焉。故曰:``直于行者曲于欲。''

晋中行文子出亡,过于县邑。从者曰:``此啬夫,公之故人。公奚不休舍,且待后车?``文子曰:``吾尝好音,此人遗我鸣琴;吾好佩,此人遗我玉环:是振我过者也。以求容于我者,吾恐其以我求容于人也。''乃去之。果收文子后车二乘而献之其君矣。

周趮谓宫他曰:``为我谓齐王曰:'以齐资我于魏,请以魏事王。'``宫他曰:``不可,是示之无魏也,齐王必不资于无魏者,而以怨有魏者。公不如曰:'以王之所欲,臣请以听魏听王。'齐王必以公为有魏也,必因公。是公有齐也,因以有齐、魏矣。''

白圭谓宋大尹曰:``君长,自知政,公无事矣。今君少主也,而务名,不如令荆贺君之孝也,则君不寿公位而大敬重公,则公常用宋矣。''

管仲、鲍叔相谓曰:``不寿君乱甚矣,必失国。齐国之诸公子其可辅者,非公子纠,则小白也。与子人事一人焉,先达者相收。''管仲乃从公子纠,鲍叔从小白。国人果弑君。小白先人为君,鲁人拘管仲而效之,鲍叔言而相之。故谚曰:``巫咸虽善祝,不能自祓也;秦医虽善除,不能自弹也。''以管仲之圣而待鲍叔之助,此鄙谚所谓``虏自卖裘而不售,士自誉辩而不信``者也。

荆王伐吴,吴使沮卫、蹶鬲犒于荆师,而将军曰:``缚之,杀以衅鼓。''问之曰:``汝来,卜乎?``答曰:``卜。''``卜吉乎?``曰:``吉。''荆人曰:``今荆将以汝衅鼓,其何也?``答曰:``是故其所以吉也。吴使臣来也,固视将军怒,将军怒,将深沟高垒;将军不怒,将懈怠。今也将军杀臣,则吴必警守矣。且国之卜,非为一臣卜。夫杀一臣而存一国,其不言吉何也?且死者无知,则以臣衅鼓无益也;死者有知也,臣将当战之时,臣使鼓不鸣。''荆人因不杀也。

知伯将伐仇由,而道难不通,乃铸大钟遗仇由之君。仇由之君大说,除道将内之。赤章曼枝曰:``不可。此小之所以事大也,而今也大以来,卒必随之,不可内也。''仇由之君不听,遂内之。赤章曼枝因断毂而驱,至于齐,七月而仇由亡矣。

越已胜吴,又索卒于荆而攻晋。左史倚相谓荆王曰:``夫越破吴,豪士死,锐卒尽,大甲伤。今又索卒以攻晋,示我不病也。不如起师与分吴。''荆王曰:``善。''因起师而从越。越王怒,将击之。大夫种曰:``不可。吾豪士尽,大甲伤。我与战,必不克。不如赂之。''乃割露山之阴五百里以赂之。

荆伐陈,吴救之,军间三十里,雨十日,夜星。左史倚相谓子期曰:``雨十日,甲辑而兵聚。吴人必至,不如备之。''乃为陈。陈未成也而吴人至,见荆陈而反。左史曰:``吴反覆六十里,其君子必休,小人必食。我行三十里击之,必可败也。''乃从之,遂破吴军。

韩、赵相与为难。韩子索兵于魏,曰:``愿借师以伐赵。''魏文候曰:``寡人与赵兄弟,不可以从。''赵又索兵以攻韩。文候曰:``寡人与韩兄弟,不敢从。''二国不得兵,怒而反。已乃知文候以构于已,乃皆朝魏。

齐伐鲁,索谗鼎,鲁以其雁往。齐人曰:``雁也。''鲁人曰:``真也。''齐曰:``使乐正子春来,吾将听子。''鲁君请乐正子春,乐正子春曰:``胡不以其真往也?``君曰:``我爱之。''答曰:``臣亦爱臣之信。''

韩咎立为君,未定也。弟在周,周欲重之,而恐韩咎不立也。綦毋恢曰:``不若以车百乘送之。得立,因曰'为戒';不立,则曰'来效贼'也。''

靖郭君将城薛,客多以谏者。靖郭君谓谒者曰:``毋为客通。''齐人有请见者曰:``臣请三言而已。过三言,臣请烹。''靖郭君因见之。客趋进曰:``海,大,鱼。''因反走。靖郭君曰:请闻其说。客曰:``臣不敢以死为戏。''靖郭君曰:``原为寡人言之。''答曰:``君闻大鱼乎?网不能止,缴不能絓\textbackslash{}也,荡而失水,蝼蚁得意焉。今夫齐亦君之海也。君长有齐,奚以薛为?君失齐,虽隆薛城至于天,犹无益也。''靖郭君曰:``善。''乃辍,不城薛。

荆王弟在秦,秦不出也。中射之士曰:``资臣百金,臣能出之。''因载百金之晋,见叔向,曰:``荆王弟在秦,秦不出也。请以百金委叔向。''叔向受金而以见之晋平公曰:``可以城壶丘矣。''平公曰:``何也?``对曰:``荆王弟在秦,秦不出也,是秦恶荆也,必不敢禁我城壶丘。若禁之,我曰:'为我出荆王之弟,吾不城也。'彼如出之,可以德荆;彼不出,是卒恶也,必不敢禁我城壶丘矣。''公曰:``善。''乃城壶丘。谓秦公曰:``为我出荆王之弟,吾不城也。''秦因出之。荆王大说,以链金百镒遗晋。

阖庐攻郢,战三胜,问子胥曰:``可以退乎?``子胥曰:``溺人者一饮而止,则无溺者,以其不休也。不如乘之以沉之。''

郑人有一子,将宦,谓其家曰:``必筑坏墙,是不善人将窃。''其巷人亦云。不时筑,而人果窃之。以其子为智,以巷人告者为盗。

\hypertarget{header-n1065}{%
\subsection{观行}\label{header-n1065}}

古之人目短于自见,故以镜观面;智短于自知,故以道正已。故镜无见疵之罪,道无明过之恶。目失镜,则无以正须眉;身失道,则无以知迷惑。西门豹之性急,故佩韦以缓已;董安于之心缓,故弦统以自急。故以有余补不足,以长绩短,之谓明主。

天下有信数三:一曰智有所有不能立,二曰力有所不能举,三曰强有所有不能胜。故虽有尧之智而无众人之助,大功不立;有乌获之劲而不得人助,不能自举;有贲、育之强而无法术,不得长胜。故势有不可得,事有不可成。故乌获轻千钧而重其身,非其重于千钧也,势不便也。离硃易百步而难眉睫,非百步近而眉睫远也,道不可也。故明主不穷乌获以其不能自举,不困离硃以其不能自见。因可势,求易道,故用力寡而功名立。时有满虚,事有利害,物有生死,人主为三者发喜怒之色,则金石之士离心焉。圣贤之朴深矣。古明主观人,不使人观己。明于尧不能独成,乌获之不能自举,贲育之不能自胜,以法术则观行之道毕矣。

\hypertarget{header-n1068}{%
\subsection{安危}\label{header-n1068}}

安术有七,危道有六。

安术:一曰,赏罚随是非;二曰,祸福随善恶;三曰,死生随法度;四曰,有贤不肖而无爱恶;五曰,有愚智而无非誉;六曰,有尺寸而无意度;七曰,有信而无诈。

危道:一曰,断削于绳之内;二曰,断割于法之外;三曰,利人之所害;四曰,乐人之所祸;五曰,危人于所安;六曰,所爱不亲,所恶不疏。如此,则人失其所以乐生,而忘其所以重死。人不乐生,则人主不尊:不重死,则令不行也。

使天下皆极智能于仪表,尽力于权衡,以动则胜,以静则安。治世使人乐生于为是,爱身于为非,小人少而君子多。故社稷常立,国家久安。左奔车之上无仲尼,覆舟之下无伯夷。故号令者,国之舟车也。安则智廉生,危则争鄙起。故安国之法,若饥而食,寒而衣,不令而自然也。先王寄理于竹帛.其道顺,故后世服。今使人饥寒去衣食,虽贲、育不能行;废自然,虽顺道而不立。强勇之所不能行,则上不能安。上以无厌责已尽。则下对``无有``;无有,则轻法。法所以为国也,而轻之,则功不立,名不成。

闻古扁鹊之治其病也,以刀刺骨;圣人之救危国也,以忠拂耳。刺骨,故小痛在体而长利在身;拂耳,故小逆在心而久福在国。故甚病之人利在忍痛,猛毅之君以福拂耳。忍痛,故扁鹊尽巧;拂耳,则子胥不失。寿安之术也。病而不忍痛,则失扁鹊之巧;危而不拂耳,则失圣人之意。如此,长利不远垂,功名不久立。

人主不自刻以尧而责人臣以子胥,是幸殷人之尽如比干;尽如此干,则上不失,下不亡。不权其力而有田成,而幸其身尽如比干,故国不得一安。废尧、舜而立桀、纣,则人不得乐所长而忧所短。失所长,则国家无功;守所短,则民不乐生。以无功御不乐生,不可行于齐民。如此,则上无以使下,下无以事上。

安危在是非,不在于强弱。存亡在虚实,不在于众寡。故齐万乘也,而名实不称,上空虚于国,内不充满于名实,故臣得夺主。杀,天子也,而无是非;赏于无功,使谗谀以诈伪为贵;诛于无罪,使伛以天性剖背。以诈伪为是,天性为非,小得胜大。

明主坚内,故不外失。失之近而不亡于远者无有。故周之夺殷也,拾遗于庭,使殷不遗于朝,则周不敢望秋毫于境。而况敢易位乎?

明主之道忠法,其法忠心,故临之而法,去之而思。尧无胶漆之约于当世而道行,舜无置锥之地于后世而德结。能立道于往古而重德于万世者之谓明主。

\hypertarget{header-n1078}{%
\subsection{守道}\label{header-n1078}}

圣王之立法也,其赏足以劝善,其威足以胜暴,其备足以必完。治世之臣,功多者位尊,力极者赏厚,情尽者名立。善之生如春,恶之死如秋,故民劝极力而乐尽情,此之谓上下相得。上下相得,故能使用力者自极于权衡,而务至于任鄙;战士出死,而愿为贲、育;守道者皆怀金石之心,以死子胥之节。用力者为任鄙,战如贲、育,中为金石,则君人者高枕而守己完矣。

古之善守者,以其所重禁其所轻,以其所难止其所易。故君子与小人俱正,盗跖与曾、史俱廉。何以知之?夫贪盗不赴溪而掇金,赴溪而掇金则身不全;贲、育不量敌则无勇名,盗跖不计可则利不成。明主之守禁也,贲、育见侵于其所不能胜,盗跖见害于其所不能取,故能禁贲、育之所不能犯,守盗跖之所不能取,则暴者守愿,邪者反正。大勇愿,巨盗贞,则天下公平,而齐民之情正矣。

人主离法失人,则危于伯夷不妄取,而不免于田成、盗跖之祸。何也?今天下无一伯夷,而奸人不绝世,故立法度量。度量信则伯夷不失是,而盗跖不得非;法分明则贤不得夺不肖,强不得侵弱,众不得暴寡。托天下于尧之法,则贞士不失分,奸人不侥幸。寄千金于羿之矢,则伯夷不得亡,而盗跖不敢取。尧明于不失奸,故天下无邪;羿巧于不失发,故千金不亡。邪人不寿而盗跖止。如此,故图不载宰予,不举六卿;书不著子胥,不明夫差。孙、吴之略废,盗跖之心伏。人主甘服于玉堂之中,而无瞋目切齿倾取之患;人臣垂拱手金城之内,而无扼腕聚脣嗟唶之祸。服虎而不以柙,禁奸而不以法,塞伪而不以符,此贲、育之所患,尧、舜之所难也。故设柙非所以备鼠也,所以使怯弱能服虎也;立法非所以备曾、史也,所以使庸主能止盗跖也;为符非所以豫尾生也,所以使众人不相谩也。不恃比干之死节,不幸乱臣之无诈也;恃怯之所能服,握庸主之所易守。当今之世,为人主忠计,为天下结德者,利莫长于此。故君人者无亡国之图,而忠臣无失身之画。明于尊位必赏,故能使人尽力于权衡,死节于官职。通贲、育之情,不以死易生;惑于盗跖之贪,不以财易身;则守国之道毕备矣。

\hypertarget{header-n1082}{%
\subsection{用人}\label{header-n1082}}

闻古之善用人者,必循天顺人而明赏罚。循天,则用力寡而功立;顺人,则刑罚省而令行;明赏罚,则伯夷、盗跖不乱。如此,则白黑分矣。治国之臣,效功于国以履位,见能于官以受职,尽力于权衡以任事。人臣皆宜其能,胜其官,轻其任,而莫怀余力于心,莫负兼官之责于君。故内无伏怨之乱,外无马服之患。明君使事不相干,故莫讼;使士不兼官,故技长;使人不同功,故莫争。争讼止,技长立,则强弱不觳力,冰炭不合形,天下莫得相伤,治之至也。

释法术而心治,尧不能正一国,去规矩而妄意度,奚仲不能成一轮;废尺寸而差短长,王尔不能半中。使中主守法术,拙匠守规矩尺寸,则万不失矣。君人者能去贤巧之所不能,守中拙之所万不失,则人力尽而功名立。

明主立可为之赏,设可避之罚。故贤者劝赏而不见子胥之祸,不肖者少罪而不见伛剖背,肓者处平而不遇深谷,愚者守静而不陷险危。如此,则上下之恩结矣。古之人曰:``其心难知,喜怒难中也。''故以表示目,以鼓语耳,以法教心。君人者释三易之数而行一难知之心,如此,则怒积于上而怨积于下。以积怒而御积怨,则两危矣。明主之表易见,故约立;其教易知,故言用;其法易为,故令行。三者立而上无私心,则下得循法而治,望表而动,随绳而断,因攒而缝。如此,则上无私威之毒,而下无愚拙之诛。故上居明而少怒,下尽忠而少罪。

闻之曰:``举事无患者,尧不得也。''而世未尝无事也。君人者不轻爵禄,不易富贵,不可与救危国。故明主厉廉耻,招仁义。昔者介子推无爵禄而义随文公,不忍口腹而仁割其肌,故人主结其德,书图著其名。人主乐乎使人以公尽力,而苦乎以私夺威;人臣安乎以能受职,而苦乎以一负二。故明主除人臣之所苦,而立人主之所乐。上下之利,莫长于此。不察私门之内,轻虑重事,厚诛薄罪,久怨细过,长侮偷快,数以德追祸,是断手而续以玉也,故世有易身之患。

人主立难为而罪不及,则私怨生;人臣失所长而奉难给,则伏怨结。劳苦不抚循,忧悲不哀怜,喜则誉小人,贤不肖俱赏,怒则毁君子,使伯夷与盗跖俱辱,故臣有叛主。

使燕王内憎其民而外爱鲁人,而燕不用而鲁不附。民见憎,不能尽力而务功;鲁见说,而不能离死命而亲他主。如此,则人臣为隙穴,而人主独立。以隙穴之臣而事独立之主,此之谓危殆。

释仪的而妄发,虽中小不巧;释法制而妄怒,虽杀戮而奸人不恐。罪生甲,祸归乙,伏怨乃结。故至治之国,有赏罚而无喜怒。故圣人极有刑法,而死无螫毒,故奸人服。发矢中的,赏罚当符,故尧复生,羿复立。如此,则上无殷、夏之患,下无比干之祸,君高枕而臣乐业,道蔽天地,德极万世矣。

夫人主不寒隙穴而劳力于赭垩,暴雨疾风必坏。不去眉睫之祸而慕贲、育之死,不谨萧墙之患而固金城于远境,不用近贤之课而外结万乘之交于千里,飘风一旦起,则贲、育不及救,而外交不及至,祸莫大于此。当今之世,为人主忠计者,必无使燕王说鲁人,无使近世慕贤于古,无思越人以救中国溺者。如此,则上下亲,内功立,外名成。

\hypertarget{header-n1091}{%
\subsection{功名}\label{header-n1091}}

明君之所以立功成名者四:一曰天时,二曰人心,三曰技能,四曰势位。非天时,虽十尧不能冬生一穗;逆人心,虽贲、育不能尽人力。故得天时则不务而自生,得人心,则不趣而自劝;因技能则不急而自疾;得势位则不推进而名成。若水之流,若船之浮。守自然之道,行毋穷之令,故曰明主。

夫有材而无势,虽贤不能制不肖。故立尺材于高山之上,下则临千仞之谷,材非长也,位高也。桀为天子,能制天下,非贤也,势重也;尧为匹夫,不能正三家,非不肖也,位卑也。千钧得船则浮,锱铢失船则沉,非千钧轻锱铢重也,有势之与无势也。故短之临高也以位,不肖之制贤也以势。人主者,天下一力以共载之,故安;众同心以共立之,故尊。人臣守所长,尽所能,故忠。以尊主御忠臣,则长乐生而功名成。名实相持而成,形影相应而立,故臣主同欲而异使。人主之患在莫之应,故曰,一手独拍,虽疾无声。人臣之忧在不得一,故曰,右手画圆,左手画方,不能两成。故曰,至治之国,君若桴,臣若鼓,技若车,事若马。故人有余力易于应,而技有余巧便于事。立功者不足于力,亲近者不足于信,成名者不足于势。近者不亲,而远者不结,则名不称实者也。圣人德若尧、舜,行若伯夷,而位不载于世,则功不立,名不遂。故古之能致功名者,众人助之以力,近者结之以成,远者誉之以名,尊者载之以势。如此,故太山之功长立于国家,而日月之名久著于天地。此尧之所以南面而守名,舜之所以北面而效功也。

\hypertarget{header-n1094}{%
\subsection{大体}\label{header-n1094}}

古之全大体者:望天地,观江海,因山谷,日月所照,四时所行,云布风动;不以智累心,不以私累己;寄治乱于法术,托是非于赏罚,属轻重于权衡;不逆天理,不伤情性;不吹毛而求小疵,不洗垢而察难知;不引绳之外,不推绳之内;不急法之外,不缓法之内;守成理,因自然;祸福生乎道法,而不出乎爱恶;荣辱之责在乎己,而不在乎人。故至安之世,法如朝露,纯朴不散,心无结怨,口无烦言。故车马不疲弊于远路,旌旗不乱乎大泽,万民不失命于寇戎,雄骏不创寿于旗幢;豪杰不著名于图书,不录功于盘盂,记年之牒空虚。故曰:利莫长乎简,福莫久于安。使匠石以千岁之寿,操钩,视规矩,举绳墨,而正太山;使贲、育带干将而齐万民;虽尽力于巧,极盛于寿,太山不正,民不能齐。故曰:古之牧天下者,不使匠石极巧以败太山之体,不使贲、育尽威以伤万民之性。因道全法,君子乐而大奸止。澹然闲静,因天命,持大体。故使人无离法之罪,鱼无失水之祸。如此,故天下少不可。

上不天则下不遍覆,心不地则物不毕载。太山不立好恶,故能成其高;江海不择小助,故能成其富。故大人寄形于天地而万物备,历心于山海而国家富。上无忿怒之毒,下无伏怨之患,上下交顺,以道为舍。故长利积,大功立,名成于前,德垂于后,治之至也。

\hypertarget{header-n1097}{%
\subsection{内储说上七术}\label{header-n1097}}

主之所用也七术,所察也六微。七术:一曰众端参观,二曰必罚明威,三曰信赏尽能,四曰一听责下,五曰疑诏诡使,六曰挟知而问,七曰倒言反事。此七者,主之所用也。

△经一参观

观听不参则诚不闻,听有门户则臣壅塞。其说在侏儒之梦见灶,哀公之称``莫众而迷''。故齐人见河伯,与惠子之言``亡其半''也。其患在竖牛之饿叔孙,而江乙之说荆俗也。嗣公欲治不知,故使有敌。是以明主推积铁之类而察一市之患。

△经二必罚

爱多者则法不立,威寡者则下侵上。是以刑罚不必则禁令不行。其说在董子之行石邑,与子产之教游吉也。故仲尼说陨霜,而殷法刑弃灰;将行去乐池,而公孙鞅重轻罪。是以丽水之金不守,而积泽之火不救。成欢以太仁弱齐国,卜皮以慈惠亡魏王。管仲知之,故断死人;嗣公知之,故买胥靡。

△经三赏誉

赏誉薄而谩者下不用也,赏誉厚而信者下轻死。其说在文子称``若兽鹿``。故越王焚宫室,而吴起倚车辕,李悝断讼以射,宋崇门以毁死。勾践知之,故式怒蛙;昭侯知之,故藏弊裤。厚赏之使人为贲、诸也,妇人之拾蚕,渔者之握鳣,是以效之。

△经四一听

一听则愚智不纷,责下则人臣不参。其说在``索郑``与``吹竽``。其患在申子之以赵绍、韩沓为尝试。故公子汜议割河东,而应侯谋弛上党。

△经五诡使

数见久待而不任,奸则鹿散。使人问他则并鬻私。是以庞敬还公大夫,而戴让诏视辒车;周主亡玉簪,商太宰论牛矢。

△经六挟智

挟智而问,则不智者至;深智一物,众隐皆变。其说在昭侯之握一爪也。故必审南门而三乡得。周主索曲杖而群臣惧,卜皮使庶子,西门豹详遗辖。

△经七

倒言反事以尝所疑,则奸情得。故阳山谩樛竖,淖齿为秦使,齐人欲为乱,子之以白马,子产离讼者,嗣公过关市。

△说一

卫灵公之时,弥子瑕有宠,专于卫国。侏儒有见公者曰:``臣之梦践矣。''公曰:``何梦?``对曰:``梦见灶,为见公也。''公怒曰:``吾闻见人主者梦见日,奚为见寡人而梦见灶?``对曰:``夫日兼烛天下,一物不能当也;人君兼烛一国,一人不能拥也。故将见人主者梦见日。夫灶,一人炀焉,则后人无从见矣。今或者一人有炀君者乎?则臣虽梦见灶,不亦可乎!``

鲁哀公问于孔子曰:``鄙谚曰:'莫众而迷。'今寡人举事与群臣虑之,而国愈乱,其故何也?``孔子对曰:``明主之问臣,一人知之,一人不知也。如是者,明主在上,群臣直议于下。今群臣无不一辞同轨乎季孙者,举鲁国尽化为一,君虽问境内之人,犹不免于乱也。''

一曰:晏婴子聘鲁,哀公问曰:``语曰:'莫三人而迷。'今寡人与一国虑之,鲁不免于乱,何也?``晏子曰:``古之所谓'莫三人而迷'者,一人失之,二人得之,三人足以为众矣,故曰'莫三人而迷。'今鲁国之群臣以千百数,一言于季氏之私,人数非不众,所言者一人也,安得三哉?``

齐人有谓齐王曰:``河伯,大神也。王何不试与之遇乎?臣请使王遇之。''乃为坛场大水之上,而与王立之焉。有间,大鱼动,因曰:``此河伯。''

张仪欲以秦、韩与魏之势伐齐、荆,而惠施欲以齐、荆偃兵。二人争之。群臣左右皆为张子言,而以攻齐、荆为利,而莫为惠子言。王果听张子,而以惠子言为不可。攻齐、荆事已定,惠子入见。王言曰:``先生毋言矣。攻齐、荆之事果利矣,一国尽以为然。''惠子因说:``不可不察也。夫齐、荆之事也诚利,一国尽以为利,是何智者之众也?攻齐、荆之事诚不可利,一国尽以为利,何愚者之众也?凡谋者,疑也。疑也者,诚疑以为可者半,以为不可者半。今一国尽以为可,是王亡半也。劫主者,固亡其半者一也。''

叔孙相鲁,贵而主断。其所爱者曰竖牛,亦擅用叔孙之令。叔孙有子曰壬,竖牛妒而欲杀之,因与壬游于鲁君所。鲁君赐之玉环,壬拜受之而不敢佩,使竖牛请之叔孙。竖牛欺之曰:``吾已为尔请之矣,使尔佩之。''壬因佩之。竖牛因谓叔孙:``何不见壬于君乎?``叔孙曰:``孺子何足见也。''竖牛曰:``壬固已数见于君矣。君赐之玉环,壬已佩之矣。''叔孙召壬见之,而果佩之,叔孙怒而杀壬。壬兄曰丙,竖牛又妒而欲杀之。叔孙为丙铸钟,钟成,丙不敢击,使竖牛请之叔孙。竖牛不为请,又欺之曰:``吾已为尔请之矣,使尔击之。''丙因击之。叔孙闻之曰:``丙不请而擅击钟。''怒而逐之。丙出走齐,居一年,竖牛为谢叔孙,叔孙使竖牛召之,又不召而报之曰:``吾已召之矣,丙怒甚,不肯来。''叔孙大怒,使人杀之。二子已死,叔孙有病,竖牛因独养之而去左右,不内人,曰:``叔孙不欲闻人声。''因不食而饿杀。叔孙已死,竖牛因不发丧也,徙其府库重宝空之而奔齐。夫听所信之言而子父为人僇,此不参之患也。

江乙为魏王使荆,谓荆王曰:``臣入王之境内,闻王之国俗曰:'君子不蔽人之美,不言人之恶。'诚有之乎?``王曰:``有之。''``然则若白公之乱,得庶无危乎?诚得如此,臣免死罪矣。''

卫嗣君重如耳,爱世姬,而恐其皆因其爱重以壅己也,乃贵薄疑以敌如耳,尊魏姬以耦世姬,曰:``以是相参也。''嗣君知欲无壅,而未得其术也。夫不使贱议贵,下必坐上,而必待势重之均也,而后敢相议,则是益树壅塞之臣也。嗣君之壅乃始。

夫矢来有乡,则积铁以备一乡;矢来无乡,则为铁室以尽备之。备之则体不伤。故彼以尽备之不伤,此以尽敌之无奸也。

庞恭与太子质于邯郸,谓魏王曰:``今一人言市有虎,王信之乎?``曰:``不信。''``二人言市有虎,王信之乎?``曰:``不信。''``三人言市有虎,王信之乎?``王曰:``寡人信之。''庞恭曰:``夫市之无虎也明矣,然而三人言而成虎。今邯郸之去魏也远于市,议臣者过于三人,愿王察之。''庞恭从邯郸反,竟不得见。

△说二

董阏于为赵上地守,行石邑山中,见深涧,峭如墙,深百仞,因问其旁乡左右曰:``人尝有入此者乎?``对曰:``无有。''曰:``婴兒、盲聋、狂悖之人尝有入此者乎?``对曰:``无有。''``牛马犬彘尝有入此者乎?``对曰:``无有。''董阏于喟然太息曰:``吾能治矣。使吾治之无赦,犹入涧之必死也,则人莫之敢犯也,何为不治?``

子产相郑,病将死,谓游吉曰:``我死后,子必用郑,必以严莅人。夫火形严,故人献灼;水形懦,人多溺。子必严子之形,无令溺子之懦。''子产死。游吉不肯严形,郑少年相率为盗,处于萑泽,将遂以为郑祸。游吉率车骑与战,一日一夜仅能克之。游吉喟然叹曰:``吾蚤行夫子之教,必不悔至于此矣。''

鲁哀公问于仲尼曰:``《春秋》之记曰:'冬十二月陨霜不杀菽。'何为记此?``仲尼对曰:``此言可以杀而不杀也。夫宜杀而不杀,桃李冬实。天失道,草木犹犯干之,而况于人君乎?``

殷之法,刑弃灰于街者。子贡以为重,问之仲尼。仲尼曰:``知治之道也。夫弃灰于街必掩人,掩人,人必怒,怒则斗,斗必三族相残也。此残三族之道也,虽刑之可也。且夫重罚者,人之所恶也;而无弃灰,人之所易也。使人行之所易,而无离所恶,此治之道也。''

一曰:殷之法,刑弃灰于公道者断其手。子贡曰:``弃灰之罪轻,断手之罚重,古人何太毅也?``曰:``无弃灰,所易也;断手,所恶也。行所易,不关所恶,古人以为易,故行之。''

中山之相乐池,以车百乘使赵,选其客之有智能者以为将行,中道而乱。乐池曰:``吾以公为有智,而使公为将行,今中道而乱,何也?``客因辞而去,曰:``公不知治。有威足以服之人,而利足以劝之,故能治之。今臣,君之少客也。夫从少正长,从贱治贵,而不得操其利害之柄以制之,此所以乱也。尝试使臣,彼之善者我能以为卿相,彼不善者我得以斩其首,何故而不治!``

公孙鞅之法也重轻罪。重罪者,人之所难犯也;而小过者,人之所易去也。使人去其所易,无离其所难,此治之道。夫小过不生,大罪不至,是人无罪而乱不生也。

一曰:公孙鞅曰:``行刑重其轻者,轻者不至,重者不来,是谓以刑去刑也。''

荆南之地,丽水之中生金,人多窃采金。采金之禁:得而辄辜磔于市。甚众,壅离其水也,而人窃金不止。大罪莫重辜磔于市,犹不止者,不必得也。故今有于此,曰:``予汝天下而杀汝身。''庸人不为也。夫有天下,大利也,犹不为者,知必死。故不必得也,则虽辜磔,窃金不止;知必死,则有天下不为也。

鲁人烧积泽。天北风,火南倚,恐烧国。哀公惧,自将众趣救火。左右无人,尽逐兽而火不救,乃召问仲尼。仲尼曰:``夫逐兽者乐而无罚,救火者苦而无赏,此火之所以无救也。''哀公曰:``善。''仲尼曰:``事急不及以赏。救火者尽赏之,则国不足以赏于人。请徒行罚。''哀公曰:``善。''于是仲尼乃下令曰:``不救火者比降北之罪,逐兽者比入禁之罪。''令下未遍而火已救矣。

成驩谓齐王曰:``王太仁,太不忍人。''王曰:``太仁,太不忍人,非善名邪?``对曰:``此人臣之善也,非人主之所行也。夫人臣必仁而后可与谋,不忍人而后可近也;不仁则不可与谋,忍人则不可近也。''王曰:``然则寡人安所太仁,安不忍人?``对曰:``王太仁于薛公,而太不忍于诸田。太仁薛公,则大臣无重;太不忍诸田,则父兄犯法。大臣无重,则兵弱于外;父兄犯法,则政乱于内。兵弱于外,政乱于内,此亡国之本也。''

魏惠王谓卜皮曰:``子闻寡人之声闻亦何如焉?``对曰:``臣闻王之慈惠也。''王欣然喜曰:``然则功且安至?``对曰:``王之功至于亡。''王曰:``慈惠,行善也。行之而亡,何也?``卜皮对曰:``夫慈者不忍,而惠者好与也。不忍则不诛有过,好予则不待有功而赏。有过不罪,无功受赏,虽亡,不亦可乎?``

齐国好厚葬,布帛尽于衣衾,材木尽于棺椁。桓公患之,以告管仲曰:``布帛尽则无以为蔽,材木尽则无以为守备,而人厚葬之不休,禁之奈何?``管仲对曰:``凡人之有为也,非名之则利之也。''于是乃下令曰:``棺椁过度者戮其尸,罪夫当丧者。''夫戮死无名,罪当丧者无利,人何故为之也?

卫嗣君之时,有胥靡逃之魏,因为襄王之后治病。卫嗣君闻之,使人请以五十金买之,五反而魏王不予,乃以左氏易之。群臣左右谏曰:``夫以一都买胥靡,可乎?``王曰:``非子之所知也。夫治无小而乱无大。法不立而诛不必,虽有十左氏无益也;法立而诛必,虽失十左氏无害也。''魏王闻之,曰:``主欲治而不听之,不祥。''因载而往,徒献之。

△说三

齐王问于文子曰:``治国何如?``对曰:``夫赏罚之为道,利器也。君固握之,不可以示人。若如臣者,犹兽鹿也,唯荐草而就。''

越王问于大夫文种曰:``吾欲伐吴,可乎?``对曰:``可矣。吾赏厚而信,罚严而必。君欲知之,何不试焚宫室?``于是遂焚宫室,人莫救之。乃下令曰:``人之救火者死,比死敌之赏;救火而不死者,比胜敌之赏;不救火者,比降北之罪。''人之涂其体,被濡衣而走火者,左三千人,右三千人。此知必胜之势也。

吴起为魏武侯西河之守。秦有小亭临境,吴起欲攻之。不去,则甚害田者;去之,则不足以征甲兵。于是乃倚一车辕于北门之外而令之曰:``有能徙此南门之外者,赐之上田、上宅。''人莫之徙也。及有徙之者,遂赐之如令。俄又置一石赤菽东门之外而令之曰:``有能徙此于西门之外者,赐之如初。''人争徙之。乃下令曰:``明日且攻亭,有能先登者,仕之国大夫,赐之上田宅。''人争趋之,于是攻亭一朝而拔之。
\\
李悝为魏文侯上地之守,而欲人之善射也,乃下令曰:``人之有狐疑之讼者,令之射的,中之者胜,不中者负。''令下而人皆疾习射,日夜不休。及与秦人战,大败之,以人之善战射也。

宋崇门之巷人,服丧而毁,甚瘠,上以为慈爱于亲,举以为官师。明年,人之所以毁死者岁十余人。子之服亲丧者,为爱之也,而尚可以赏劝也,况君上之于民乎?

越王虑伐吴,欲人之轻死也,出见怒蛙,乃为之式。从者曰:``奚敬于此?``王曰:``为其有气故也。''明年之请以头献王者岁十余人。由此观之,誉之足以杀人矣。

一曰:越王勾践见怒蛙而式之。御者曰:``何为式?``王曰:``蛙有气如此,可无为式乎?``士人闻之曰:``蛙有气,王犹为式,况士人有勇者乎!``是岁人有自刭死,以其头献者。故越王将复吴而试其教,燔台而鼓之,使民赴火者,赏在火者;临江而鼓之,使人赴水者,赏在水也;临战而使人绝头刳腹而无顾心者,赏在兵也。又况据法而进贤,其助甚此矣。

韩昭侯使人藏弊裤,侍者曰:``君亦不仁矣,弊裤不以赐左右而藏之。''昭侯曰:``非子之所知也。吾闻明主之爱一嚬一笑,嚬有为嚬,而笑有为笑。今夫裤,岂特嚬笑哉!裤之与嚬笑相去远矣。吾必待有功者,故收藏之未有予也。''鳣似蛇,蚕似鳣。人见蛇则惊骇,见蠋则毛起。然而妇人拾蚕,渔者握鳣,利之所在,则忘其所恶,皆为贲诸。

△说四

魏王谓郑王曰:``始郑、梁一国也,已而别,今愿复得郑而合之梁。''郑君患之,召群臣而与之谋所以对魏。公子谓郑君曰:``此甚易应也。君对魏曰:'以郑为故魏而可合也,则弊邑亦愿得梁而合之郑。''魏王乃止。

齐宣王使人吹竽,必三百人。南国处士请为王吹竽,宣王说之,廪食以数百人。宣王死,氵昬王立,好一一听之,处士逃。一曰:韩昭侯曰:``吹竽者众,无以知其善者。''田严对曰:``一一而听之。''

赵令人因申子于韩请兵,将以攻魏。申子欲言之君,而恐君之疑己外市也,不则恐恶于赵,乃令赵绍、韩沓尝试君之动貌而后言之。内则知昭侯之意,外则有得赵之功。

三国兵至,韩王谓楼缓曰:``三国之兵深矣!寡人欲割河东而讲,何如?``对曰:``夫割河东,大费也;免国于患,大功也。此父兄之任也,王何不召公子汜而问焉?``王召公子汜而告之,对曰:``讲亦悔,不讲亦悔。王今割河东而讲,三国归,王必曰:'三国固且去矣,吾特以三城送之。'不讲,三国也入韩,则国必大举矣,王必大悔。王曰:'不献三城也。'臣故曰:讲亦悔,不讲亦悔。''王曰:``为我悔也,宁亡三城而悔,无危乃悔。寡人断讲矣。''

应侯谓秦王曰:``王得宛、叶、兰田、阳夏,断河内,困梁、郑,所以未王者,赵未服也。弛上党在一而已,以临东阳,则邯郸口中虱也。王拱而朝天下,后者以兵中之。然上党之安乐,其处甚剧,臣恐弛之而不听,奈何?``王曰:``必弛易之矣。''

△说五

庞敬,县令也。遣市者行,而召公大夫而还之。立有间,无以诏之,卒遣行。市者以为令与公大夫有言,不相信,以至无奸。

戴驩,宋太宰,夜使人曰:``吾闻数夜有乘辒车至李史门者,谨为我伺之。''使人报曰:``不见辒车,见有奉笥而与李史语者,有间,李史受笥。''

周主亡玉簪,令吏求之,三日不能得也。周主令人求,而得之家人之屋间。周主曰:``吾之吏之不事事也。求簪三日不得之,吾令人求之,不移日而得之。''于是吏皆耸惧,以为君神明也。

商太宰使少庶子之市,顾反而问之曰:``何见于市?``对曰:``无见也。''太宰曰:``虽然,何见也?``对曰:``市南门之外甚众牛车,仅可以行耳。''太宰因诫使者:``无敢告人吾所问于女。''因召市吏而诮之曰:``市门之外何多牛屎?``市吏甚怪太宰知之疾也,乃悚惧其所也

△说六

韩昭侯握爪,而佯亡一爪,求之甚急。左右因割其爪而效之。昭侯以此察左右之诚不。

韩昭侯使骑于县,使者报,昭侯问曰:``何见也?``对曰:``无所见也。''昭侯曰:``虽然,何见?``曰:``南门之外,有黄犊食苗道左者。''昭侯谓使者:``毋敢泄吾所问于女。''乃下令曰:``当苗时,禁牛马入人田中,固有令,而吏不以为事,牛马甚多入人田中。亟举其数上之;不得,将重其罪。''于是三乡举而上之。昭侯曰:``未尽也。''复往审之,乃得南门之外黄犊。吏以昭侯为明察,皆悚惧其所而不敢为非。

周主下令索曲杖,吏求之数日不能得。周主私使人求之,不移日而得之。乃谓吏曰:``吾知吏不事事也。曲杖甚易也,而吏不能得,我令人求之,不移日而得之,岂可谓忠哉!``吏乃皆悚惧其所,以君为神明。

卜皮为县令,其御史污秽而有爱妾,卜皮乃使少庶子佯爱之,以知御史阴情。

西门豹为鄴令,佯亡其车辖,令吏求之不能得,使人求之而得之家人屋间。

△说七

阳山君相谓,闻王之疑己也,乃伪谤勷竖以知之。

淖齿闻齐王之恶己也,乃矫为秦使以知之。

齐人有欲为乱者,恐王知之,因诈逐所爱者,令走王知之。

子之相燕,坐而佯言:``走出门者何,白马也?``左右皆言不见。有一人走追之,报曰:``有。''子之以此知左右之不诚信。

有相与讼者,子产离之,而无使得通辞,倒其言以告而知之。

卫嗣公使人为客过关市,关市苛难之,因事关市以金,关吏乃舍之。嗣公为关吏曰:``某时有客过而所,与汝金,而汝因遣之。''关吏乃大恐,而以嗣公为明察。

\hypertarget{header-n1172}{%
\subsection{内储说下六微}\label{header-n1172}}

六微:一曰权借在下,二曰利异外借,三曰托于似类,四曰利害有反,五曰参疑内争,六曰敌国废置。此六者,主之所察也。

△经一权借

权势不可以借人,上失其一,臣以为百。故臣得借则力多,力多则内外为用,内外为用则人主壅。其说在老聃之言失鱼也。是以人主久语而左右鬻怀刷,其患在胥僮之谏厉公,与州侯之一言而燕人浴矢也。

△经二利异

君臣之利异,故人臣莫忠,故臣利立而主利灭。是以奸臣者召敌兵以内除,举外事以眩主,苟成其私利,不顾国患。其说在卫人之夫妻祷祝也。故戴歇议子弟,而三桓攻昭公;公叔内齐军,而翟黄召韩兵;太宰嚭说大夫种,大成牛教申不害;司马喜告赵王,吕仓规秦、楚;宋石遗卫君书,白圭教暴谴。

△经三似类

似类之事,人主之所以失诛,而大臣之所以成私也。是以门人捐水而夷射诛,济阳自矫而二人罪,司马喜杀爰骞而季辛诛,郑袖言恶臭而新人劓,费无忌教郄宛而令尹诛,陈需杀张寿而犀首走。故烧刍\textless{}广会\textgreater{}郄而中山罪,杀老儒而济阳赏也。

△经四有反

事起而有所利,其尸主之;有所害,必反察之。是以明主之论也,国害则省其利者,臣害则察其反者。其说在楚兵至而陈需相,黍种贵而廪吏覆。是以昭奚恤执贩茅,而不僖侯谯其次;文公发绕炙,而穰侯请立帝。

△经五参疑

参疑之势,乱之所由生也,故明主慎之。是以晋骊姬杀太子申生,而郑夫人用毒药,卫州吁杀其君完,公子根取东周,王子职甚有宠而商臣果作乱,严遂、韩傀争而哀侯果遇贼,田常、阚止、戴驩、皇喜敌而宋君、简公杀。其说在狐突之称``二好``,与郑昭之对``未生``也。

△经六废置

敌之所务,在淫察而就靡,人主不察,则敌废置矣。故文王资费仲,而秦王患楚使;黎且去仲尼,而干象沮甘茂。是以子胥宣言而子常用,内美人而虞、虢亡,佯遗书而苌弘死,用鸡猳而郐桀尽。

△庙攻

``参疑````废置``之事,明主绝之于内而施之于外。资其轻者,辅其弱者,此谓``庙攻``。参伍既用于内,观听又行于外,则敌伪得。其说在秦侏儒之告惠文君也。故襄疵言袭鄴,而嗣公赐令席。

△说一

势重者,人主之渊也;臣者,势重之鱼也。鱼失于渊而不可复得也,人主失其势重于臣而不可复收也。古之人难正言,故托之于鱼。

赏罚者,利器也,君操之以制臣,臣得之以拥主。故君先见所赏,则臣鬻之以为德;君先见所罚,则臣鬻之以为威。故曰:``国之利器,不可以示人。''

靖郭君相齐,与故人久语,则故人富,怀左右刷,则左右重。久语怀刷,小资也,犹以成富,况于吏势乎?

晋厉公之时,六卿贵,胥僮、长鱼矫谏曰:``大臣贵重,敌主争事,外市树党,下乱国法,上以劫主,而国不危者,未尝有也。''公曰:``善。''乃诛三卿。胥僮、长鱼矫又谏曰:``夫同罪之人偏诛而不尽,是怀怨而借之间也。''公曰:``吾一朝而夷三卿,予不忍尽也。''长鱼矫对曰:``公不忍之,彼将忍公。''公不听。居三月,诸卿作难,遂杀厉公而分其地。

州侯相荆,贵而主断。荆王疑之,因问左右,左右对曰:``无有。''如出一口也。

燕人无惑,故浴狗矢。燕人其妻有私通于士,其夫早自外而来,士适出。夫曰:``何客也?``其妻曰:``无客。''问左右,左右言``无有``,如出一口。其妻曰:``公惑易也。''因浴之以狗矢。

一曰:燕人李季好远出,其妻私有通于士,季突至,士在内中,妻患之。其室妇曰:``令公子裸而解发,直出门,吾属佯不见也。''于是公子从其计,疾走出门。季曰:``是何人也?``家室皆曰:``无有。''季曰:``吾见鬼乎?``妇人曰:``然。''``为之奈何?``曰:``取五牲之矢浴之。''季曰:``诺。''乃浴以矢。一曰浴以兰汤。

△说二

卫人有夫妻祷者而祝曰:``使我无故,得百束布。''其夫曰:``何少也?``对曰:``益是,子将以买妾。''

荆王欲宦诸公子于四邻,戴歇曰:``不可。''``宦公子于四邻,四邻必重之。''曰:``子出者重,重则必为所重之国党,则是教子于外市也,不便。''

鲁孟孙、叔孙、季孙相戮力劫昭公,遂夺其国而擅其制。鲁三桓逼公,昭公攻季孙氏,而孟孙氏、叔孙氏相与谋曰:``救之乎?``叔孙氏之御者曰:``我家臣也,安知公家?````凡有季孙与无季孙于我孰利?``皆曰:``无季孙必无叔孙。''``然则救之。''于是撞西北隅而入。孟孙见叔孙之旗入,亦救之。三桓为一,昭公不胜。逐之,死于乾侯。

公叔相韩而有攻齐,公仲甚重于王,公叔恐王之相公仲也,使齐、韩约而攻魏。公叔因内齐军于郑以劫其君,以固其位而信两国之约。

翟璜,魏王之臣也,而善于韩。乃召韩兵令之攻魏,因请为魏王构之以自重也。

越王攻吴王,吴王谢而告服,越王欲许之。范蠡、大夫种曰:``不可。昔天以越与吴,吴不受,今天反夫差,亦天祸也。以吴予越,再拜受之,不可许也。''太宰嚭遗大夫种书曰:``狡兔尽则良犬烹,敌国灭则谋臣亡。大夫何不释吴而患越乎?``大夫种受书读之,太息而叹曰:``杀之,越与吴同命。''

大成牛从赵谓申不害于韩曰:``以韩重我于赵,请以赵重子于韩,是子有两韩,我有两赵。''

司马喜,中山君之臣也,而善于赵,尝以中山之谋微告赵王。

吕仓,魏王之臣也,而善于秦、荆。微讽秦、荆令之攻魏,因请行和以自重也。

宋石,魏将也;卫君,荆将也。两国构难,二子皆将。宋石遗卫君书曰:``二军相当,两旗相望,唯毋一战,战必不两存。此乃两主之事也,与子无有私怨,善者相避也。''

白圭相魏,暴谴相韩。白圭谓暴谴曰:``子以韩辅我于魏,我以魏待子于韩,臣长用魏,子长用韩。''

△说三

齐中大夫有夷射者,御饮于王,醉甚而出,倚于郎门。门者刖跪请曰:``足下无意赐之余沥乎?``夷射叱曰:``去!刑余之人,何事乃敢乞饮长者!``刖跪走退。及夷射去,刖跪因捐水郎门霤下,类溺者之状。明日,王出而呵之,曰:``谁溺于是?``刖跪对曰:``臣不见也。虽然,昨日中大夫夷射立于此。''王因诛夷射而杀之。

魏王臣二人不善济阳君,济阳君因伪令人矫王命而谋攻己。王使人问济阳君曰:``谁与恨?``对曰:``无敢与恨。虽然,尝与二人不善,不足以至于此。''王问左右,左右曰:``固然。''王因诛二人者。

季辛与爰骞相怨,司马喜新与季辛恶,因微令人杀爰骞,中山之君以为季辛也,因诛之。

荆王所爱妾有郑袖者。荆王新得美女,郑袖因教之曰:``王甚喜人之掩口也,为近王,必掩口。''美女入见,近王,因掩口。王问其故,郑袖曰:``此固言恶王之臭。''及王与郑袖、美女三人坐,袖因先诫御者曰:``王适有言,必亟听从王言。''美女前,近王甚,数掩口。王悖然怒曰:``劓之。''御因揄刀而劓美人。

一曰:魏王遗荆王美人,荆王甚悦之。夫人郑袖知王悦爱之也,亦悦爱之,甚于王,衣服玩好择其所欲为之。王曰:``夫人知我爱新人也,其悦爱之甚于寡人,此孝子所以养亲,忠臣之所以事君也。''夫人知王之不以己为妒也,因为新人曰:``王甚悦爱子,然恶子之鼻,子见王,常掩鼻,则王长幸子矣。''于是新人从之,每见王,常掩鼻。王谓夫人曰:``新人见寡人常掩鼻,何也?``对曰:``不知也。''王强问之,对曰:``顷尝言恶闻王臭。''王怒曰:``劓之。''夫人先诫御者曰:``王适有言,必可从命。''御者因揄刀而劓美人。

费无极,荆令尹之近者也。郄宛新事令尹,令尹甚爱之。无极因谓令尹曰:``君爱宛甚,何不一为酒其家?``令尹曰:``善。''因令之为具于郄宛之家。无极教宛曰:``令尹甚傲而好兵,子必谨敬,先亟陈兵堂下及门庭。''宛因为之。令尹往而大惊,曰:``此何也?``无极曰:``君殆,去之!事未可知也。''令尹大怒,举兵而诛郄宛,遂杀之。

犀首与张寿为怨,陈需新入,不善犀首,因使人微杀张寿。魏王以为犀首也,乃诛之。

中山有贱公子,马甚瘦,车甚弊。左右有私不善者,乃为之请王曰:``公子甚贫,马甚瘦,王何不益之马食?``王不许。左右因微令夜烧刍厩。王以为贱公子也,乃诛之。

魏有老儒而不善济阳君。客有与老儒私怨者,因攻老儒杀之,以德于济阳君,曰:``臣为其不善君也,故为君杀之。''济阳君因不察而赏之。

一曰:济阳君有少庶子者,不见知,欲入爱于君者。齐使老儒掘药于马梨之山。济阳少庶子欲以为功,入见于君曰:``齐使老儒掘药于马梨之山,名掘药也,实间君之国。君杀之,是将以济阳君抵罪于齐矣。臣请刺之。''君曰:``可。''于是明日得之城阴而刺之,济阳君还,益亲之。

△说四

陈需,魏王之臣也,善于荆王,而令荆攻魏。荆攻魏。陈需因请为魏王行解之,因以荆势相魏。

韩昭侯之时,黍种尝贵甚。昭侯令人覆廪,吏果窃黍种而粜之甚多。

昭奚恤之用荆也,有烧仓\textless{}广会\textgreater{}\{穴卯\}者而不知其人。昭奚恤令吏执贩茅者而问之,果烧也。

昭僖侯之时,宰人上食,而羹中有生肝焉。昭侯召宰人之次而谯之曰:``若何为置生肝寡人羹中?``宰人顿首服死罪,曰:``窃欲去尚宰人也。''

一曰:僖侯浴,汤中有砾。僖侯曰:``尚浴免,则有当代者乎?``左右对曰:``有。''僖侯曰:``召而来。''谯之曰:``何为置砾汤中?``对曰:``尚浴免,则臣得代之,是以置砾汤中。''

文公之时,宰臣上炙而发绕之。文公召宰人而谯之曰:``女欲寡人之哽耶,奚以发绕炙?``宰人顿首再拜,请曰:``臣有死罪三:援砺砥刀,利犹干将也,切肉肉断而发不断,臣之罪一也;援锥贯脔而不见发,臣之罪二也;奉炽炉炭,肉尽赤红,炙熟而发不焦,臣之罪三也。堂下得微有疾臣者乎?``公曰:``善。''乃召其堂下而谯之,果然,乃诛之。

一曰:晋平公觞客,少庶子进炙而发绕之。平公趣杀砲人,毋有反令。砲人呼天曰:``嗟乎!臣有三罪,死而不自知乎!``平公曰:``何谓也?``对曰:``臣刀之利,风靡骨断,而发不断,是臣之一死也;桑炭炙之,肉红白而发不焦,是臣之二死也;炙熟,又重睫而视之,发绕炙而目不见,是臣之三死也。意者堂下其有翳憎臣者乎?杀臣不亦蚤乎!``

穰侯相秦,而齐强。穰侯欲立秦为帝而齐不听,因请立齐为东帝,而不能成也。

△说五

晋献公之时,骊姬贵,拟于后妻,而欲以其子奚齐代太子申生,因患申生于君而杀之,遂立奚齐为太子。

郑君已立太子矣,而有所爱美女欲以其子为后。夫人恐,因用毒药贼君杀之。

卫州吁重于卫,拟于君,群臣百姓尽畏其势重。州吁果杀其君而夺之政。

公子朝,周太子也,弟公子根甚有宠于君。君死,遂以东周叛,分为两国。

楚成王以商臣为太子,既而又欲置公子职。商臣作乱,遂攻杀成王。

一曰:楚成王以商臣为太子,既欲置公子职。商臣闻之,未察也,乃为其傅潘崇曰:``奈何察之也?``潘崇曰:``飨江羋而勿敬也。''太子听之,江羋曰:``呼,役夫!宜君王之欲废女而立职也。''商臣曰:``信矣。''潘崇曰:``能事之乎?``曰:``不能。''``能为之诸侯乎?``曰:``不能。''``能举大事乎?``曰:``能。''于是乃起宿营之甲而攻成王。成王请食熊膰而死,不许,遂自杀。

韩傀相韩哀侯,严遂重于君,二人甚相害也。严遂乃令人刺韩傀于朝,韩傀走君而抱之,遂刺韩傀而兼哀侯。

田恆相齐,阚止重于简公,二人相憎而欲相贼也。田恆因行私惠以取其国,遂杀简公而夺之政。

戴驩为宋太宰,皇喜重于君,二人争事而相害也。皇喜遂杀宋君而夺其政。

狐突曰:``国君好内则太子危,好外则相室危。''

郑君问郑昭曰:``太子亦何如?``对曰:``太子未生也。''君曰:``太子已置,而曰'未生',何也?``对曰:``太子虽置,然而君之好色不已,所爱有子,君必爱之,爱之则必欲以为后,臣故曰'太子未生'也。''

△说六

文王资费仲而游于纣之旁,令之谏纣而乱其心。

荆王使人之秦,秦王甚礼之。王曰:``敌国有贤者,国之忧也。今荆王之使者甚贤,寡人患之。''群臣谏曰:``以王之贤圣与国之资厚,愿荆王之贤人,王何不深知之而阴有之。荆以为外用也,则必诛之。''

仲尼为政于鲁,道不拾遗,齐景公患之。黎且谓景公曰:``去仲尼,犹吹毛耳。君何不迎之以重禄高位,遗哀公女乐以骄荣其意。哀公新乐之,必怠于政,仲尼必谏,谏必轻绝于鲁。''景公曰:``善。''乃令黎且以女乐二八遗哀公,哀公乐之,果怠于政。仲尼谏不听,去而之楚。

楚王谓干象曰:``吾欲以楚扶甘茂而相之秦,可乎?``干相对曰:``不可也。''王曰:``何也?``曰:``甘茂少而事史举先生。史举,上蔡之监门也,大不事君,小不事家,以苛刻闻天下。茂事之,顺焉。惠王之明,张仪之辨也,茂事之,取十官而免于罪,是茂贤也。''王曰:``相人敌国而相贤,其不可何也?``干象曰:``前时王使邵滑之越,五年而能亡越。所以然者,越乱而楚治也。日者知用之越,今忘之秦,不亦太亟忘乎?``王曰:``然则为之奈何?``干象对曰:``不如相共立。''王曰:``共立可相,何也?``对曰:``共立少见爱幸,长为贵卿,被王衣,含杜若,握玉环,以听于朝,且利以乱秦矣。''

吴政荆,子胥使人宣言于荆曰:``子期用,将击之;子常用,将去之。''荆人闻之,因用子常而退子期也,吴人击之,遂胜之。

晋献公伐虞、虢,乃遗之屈产之乘,垂棘之璧,女乐二八,以荣其意而乱其政。

叔向之谗苌弘也,为书苌弘,谓叔向曰:``子为我谓晋君,所与君期者,时可矣。何不亟以兵来?``因佯遗其书周君之庭而急去行。周以苌弘为卖周也,乃诛苌弘而杀之。

郑桓公将欲袭郐,先问郐之豪杰、良臣、辩智、果敢之士,尽与姓名,择郐之良田赂之,为官爵之名而书之,因为设坛场郭门之外而埋之,衅之以鸡豭,若盟状。郐君以为内难也而尽杀其良臣。桓公袭郐,遂取之。

秦侏儒善于荆王,而阴有善荆王左右而内重于惠文君。荆适有谋,侏儒常先闻之以告惠文君。

鄴令襄疵阴善赵王左右。赵王谋袭鄴,襄疵常辄闻而先言之魏王。魏王备之,赵乃辄还。

卫嗣君之时,有人于县令之左右。县令发蓐而席弊甚,嗣公还令人遗之席,曰:``吾闻汝今者发蓐而席弊甚,赐汝席。''县令大惊,以君为神也。'

\hypertarget{header-n1252}{%
\subsection{外储说左上}\label{header-n1252}}

△经一

明主之道,如有若之应密子也。明主之听言也,美其辩;其观行也,贤其远。故群臣士民之道言者迂弘,其行身也离世。其说在田鸠对荆王也。故墨子为木鸢,讴癸筑武宫。夫药酒忠言,明君圣主之以独知也。

△经二

人主之听言也,不以功用为的,则说者多``棘刺``、``白马``之说;不以仪的为关,则射者皆如羿也。人主于说也,皆如燕王学道也;而长说者,皆如郑人争年也。是以言有纤察微难而非务也。故季、惠、宋、墨皆画策也;论有迂深闳大,非用也。故畏震胆车言而拂难坚确,非功也,故务、卞、鲍、介、田仲皆坚瓠也。且虞庆诎匠也而屋坏,范且穷工而弓折。是故求其诚者,非归饷也不可。

△经三

挟夫相为则责望,自为则事行。故父子或怨谯,取庸作者进美羹。说在文公之先宣言,与勾践之称如皇也。故桓公藏蔡怒而攻楚,吴起怀瘳实而吮伤。且先王之赋颂,钟鼎之铭,皆播吾之迹,华山之博也。然先王所期者利也,所用者力也。筑社之谚,自辞说也。请许学者而行宛曼于先王,或者不宜今乎?如是,不能更也。郑县人得车厄也,卫人佐弋也,卜子妻写弊裤也,而其少者也。先王之言,有其所为小而世意之大者,有其所为大而世意之小者,未可必知也。说在宋人之解书,与梁人之读记也。故先王有郢书,而后世多燕说。夫不适国事而谋先王,皆归取度者也。

△经四

利之所在民归之,名之所彰士死之。是以功外于法而赏加焉,则上不信得所利于下;名外于法而誉加焉,则士劝名而不畜之于君。故中章、胥己仕,而中牟之民弃田圃而随文学者邑之半;平公腓痛足痹而不敢坏坐,晋国之辞仕托者国之锤。此三士者,言袭法则官府之籍也,行中事则如令之民也,二君之礼太甚。若言离法而行远功,则绳外民也,二君有何礼之?礼之当亡。且居学之士,国无事不用力,有难不被甲,礼之则惰修耕战之功;不礼则周主上之法。国安则尊显,危则为屈公之威,人主奚得于居学之士哉?故明主论李疵视中山也。

△经五

《诗》曰:``不躬不亲,庶民不信。''傅说之以``无衣紫``,缓之以郑简、宋襄,责之以尊厚耕战。夫不明分,不责诚,而以躬亲位下,且为``下走睡卧``,与去``掩弊微服``。孔丘不知,故称犹盂;邹君不知,故先自脩。明主之道,如叔向赋猎,与昭侯之奚听也。

△经六

小信成则大信立,故明主积于信。赏罚不信,则禁令不行,说在文公之攻原与箕郑救饿也。是以吴起须故人而食,文侯会虞人而猎。故明主表信,如曾子杀彘也。患在厉王击警鼓,与李悝谩两和也。

△说一

宓子贱治单父。有若见之曰:``子何臞也?''宓子曰:``君不知齐不肖,使治单父,官事急,心忧之,故#也。''有若曰:``昔者舜鼓五弦、歌《南风》之诗而天下治。今以单父之细也,治之而忧,治天下将奈何乎?故有术而御之,身坐于庙堂之上,有处女子之色,无害于治;无术而御之,身虽瘁臞,犹未益也。''

楚王谓田鸠曰:``墨子者,显学也。其身体则可,其言多而不辩,何也?''曰:``昔秦伯嫁其女于晋公子,令晋为之饰装,从衣文之媵七十人。至晋,晋人爱其妾而贱公女。此可谓善嫁妾而未可谓善嫁女也。楚人有卖其珠于郑者,为木兰之椟,薰以桂椒,缀以珠玉,饰以玫瑰,辑以翡翠。郑人买其椟而还其珠。此可谓善卖椟矣,未可谓善鬻珠也。今世之谈也,皆道辩说文辞之言,人主览其文而忘有用。墨子之说,传先王之道,论圣人之言,以宣告人。若辩其辞,则恐人怀其文,忘其直,以文害用也。此与楚人鬻珠、秦伯嫁女同类,故其言多不辩。''

墨子为木鸢,三年而成,蜚一日而败。弟子曰:``先生之巧,至能使木鸢飞。''墨子曰:``吾不如为车輗者巧也。用咫尺之木,不费一朝之事,而引三十石之任,致远力多,久于岁数。今我为鸢,三年成,蜚一日而败。''惠子闻之曰:``墨子大巧,巧为輗,拙为鸢。''

宋王与齐仇也,筑武宫,讴癸倡,行者止观,筑者不倦。王闻,召而赐之。对曰:``臣师射稽之讴又贤于癸。''王召射稽使之讴,行者不止,筑者知倦。王曰:``行者不止,筑者知倦,其讴不胜如癸美,何也?''对曰:``王试度其功。''癸四板,射稽八板;擿其坚,癸五寸,射稽二寸。

夫良药苦于口,而智者劝而饮之,知其入而已己疾也。忠言拂于耳,而明主听之,知其可以致功也。

△说二

宋人有请为燕王以棘刺之端为母猴者,必三月斋,然后能观之。燕王因以三乘养之。右御冶工言王曰:``臣闻人主无十日不燕之斋。今知王不能久斋以观无用之器也,故以三月为期。凡刻削者,以其所以削必小。今臣冶人也,无以为之削,此不然物也。王必察之。''王因囚而问之,果妄,乃杀之。冶人又谓王曰:``计无度量,言谈之士多棘刺之说也。''

一曰:燕王征巧术之人,卫人诸以棘刺之端为母猴。燕王说之,养之以五乘之奉。王曰:``吾试观客为棘刺之母猴。''客曰:``人主欲观之,必半岁不入宫,不饮酒食肉,雨霁日出,视之晏阴之间,而棘刺之母猴乃可见也。''燕王因养卫人,不能观其母猴。郑有台下之冶者谓燕王曰:``臣为削者也。诸微物必以削削之,而所削必大于削。今棘刺之端不容削锋,难以治棘刺之端。王试观客之削,能与不能可知也。''王曰:``善。''谓卫人曰:``客为棘刺之?''曰:``以削。''王曰:``吾欲观见之。''客曰:``臣请之舍取之。''因逃。

兒说,宋人,善辩者也,持``白马非马也``服齐稷下之辩者。乘白马而过关,则顾白马之赋。故籍之虚辞则能胜一国,考实按形不能谩于一人。

夫新砥砺杀矢,彀弩而射,虽冥而妄发,其端未尝不中秋毫也,然而莫能复其处,不可谓善射,无常仪的也;设五寸之的,引十步之远,非羿、逄蒙不能必全者,有常仪的也;有度难而无度易也。有常仪的,则羿、逄蒙以五寸为巧;无常仪的,则以妄发而中秋毫为拙。故无度而应之,则辩士繁说;设度而持之,虽知者犹畏失也,不敢妄言。今人主听说不应之以度,而说其辩,不度以功,誉其行而不入关。此人主所以长欺,而说者所以长养也。

客有教燕王为不死之道者,王使人学之,所使学者未及学而客死。王大怒,诛之。王不知客之欺己,而诛学者之晚也。夫信不然之物而诛无罪之臣,不察之患也。且人所急无如其身,不能自使其无死,安能使王长生哉?

郑人有相与争年者。一人曰:``吾与尧同年。''其一人曰:``我与黄帝之兄同年。''讼此而不决,以后息者为胜耳。

客有为周君画筴者,三年而成。君观之,与髹筴者同状。周君大怒。画筴者曰:``筑十版之墙,凿八尺之牖,而以日始出时加之其上而观。''周君为之,望见其状,尽成龙蛇禽兽车马,万物状备具。周君大悦。此筴之功非不微难也,然其用与素髹筴同。

客有为齐王画者,齐王问曰:``画孰最难者?''曰:``犬马最难。''``孰易者?''曰:``鬼魅最易。''夫犬马,人所知也,旦暮罄于前,不可类之,故难。鬼魅无形者,不罄于前,故易之也。

齐有居士田仲者,宋人屈谷见之,曰:``谷闻先生之义,不恃仰人而食,今谷有巨瓠之道,坚如石,厚而无窍,献之。''仲曰:``夫瓠所贵者,谓其可以盛也。今厚而无窍,则不可以剖以盛物;而任重如坚石,则不可以剖而以斟。吾无以瓠为也。''曰:``然,谷将弃之。''今田仲不恃仰人而食,亦无益人之国,亦坚瓠之类也。

虞庆为屋,谓匠人曰:``屋太尊。''匠人对曰:``此新屋也,涂濡而椽生。''虞庆曰:``不然。夫濡涂重而生椽挠,以挠椽任重涂,此宜卑。更日久,则涂干而椽燥。涂干则轻,椽燥则直,以直椽任轻涂,此益尊。''匠人诎,为之而屋坏。

一曰:虞庆将为屋,匠人曰:``材生而涂濡。夫材生则桡,涂濡则重,以桡任重,今虽成,久必坏。''虞庆曰:``材干则直,涂干则轻。今诚得干,日以轻直,虽久必不坏。''匠人诎,作之成,有间,屋果坏。

范且曰:``弓之折,必于其尽也,不于其始也。夫工人张弓也,伏檠三旬而蹈弦,一日犯机,是节之其始而暴之其尽也,焉得无折?且张弓不然:伏檠一日而蹈弦,三旬而犯机,是暴之其始而节之其尽也。''工人穷也,为之,弓折。

范且、虞庆之言,皆文辩辞胜而反事之情。人主说而不禁,此所以败也。夫不谋治强之功,而艳乎辩说文丽之声,是却有术之士而任``坏屋``、``折弓``也。故人主之于国事也,皆不达乎工匠之构屋张弓也。然而士穷乎范且、虞庆者,为虚辞,其无用而胜,实事,其无易而穷也。人主多无用之辩,而少无易之言,此所以乱也。今世之为范且、虞庆者不辍,而人主说之不止,是贵``败``、``折``之类,而以知术之人为工匠也。工匠不得施其技巧,故坏屋折弓;知治之人不得行其方术,故国乱而主危。

夫婴兒相与戏也,以尘为饭,以涂为羹,以木为胾,然至日晚必归饷者,尘饭涂羹可以戏而不可食也。夫称上古之传颂,辩而不悫,道先王仁义而不能正国者,此亦可以戏而不可以为治也。夫慕仁义而弱乱者,三晋也;不慕而治强者,秦也,然而未帝者,治未毕也。
\\
△说三

人为婴兒也,父母养之简,子长人怨。子盛壮成人,其供养薄,父母怒而诮之。子父至亲也,而或谯或怨者,皆挟相为而不周于为己也。夫卖庸而播耕者,主人费家而美食,调布而求易钱者,非爱庸客也,曰:如是,耕者且深,耨者熟耘也。庸客致力而疾耘耕者,尽巧而正畦陌畦畤者,非爱主人也,曰:如是,羹且美,钱布且易云也。此其养功力,有父子之泽矣,而心调于用者,皆挟自为心也。故人行事施予,以利之为心,则越人易和;以害之为心,则父子离且怨。

文公伐宋,乃先宣言曰:``吾闻宋君无道,蔑侮长老,分财不中,教令不信,余来为民诛之。''

越伐吴,乃先宣言曰:``我闻吴王筑如皇之台,掘渊泉之池,罢苦百姓,煎靡财货,以尽民力,余来为民诛之。''

蔡女为桓公妻,桓公与之乘舟,夫人荡舟,桓公大惧,禁之不止,怒而出之。乃且复召之,因复更嫁之。桓公大怒,将伐蔡。仲父谏曰:``夫以寝席之戏,不足以伐人之国,功业不可冀也,请无以此为稽也。''桓公不听。仲父曰:``必不得已,楚之菁茅不贡于天子三年矣,君不如举兵为天子伐楚。楚服,因还袭蔡,曰:'余为天子伐楚,而蔡不以兵听从',遂灭之。此义于名而利于实,故必有为天子诛之名,而有报仇之实。''

吴起为魏将而攻中山,军人有病疽者,吴起跪而自吮其脓。伤者之母立而泣,人问曰:``将军于若子如是,尚何为而泣?''对曰:``吴起吮其父之创而父死,今是子又将死也,今吾是以泣。''

赵主父令工施钩梯而缘播吾,刻疏人迹其上,广三尺,长五尺,而勒之曰:``主父常游于此。''

秦昭王令工施钩梯而上华山,以松柏之心为博,箭长八尺,棋长八寸,而勒之曰:``昭王尝与天神博于此矣。''

文公反国至河,令笾豆捐之,席蓐捐之,手足胼胝,面目黧黑者后之。咎犯闻之而夜哭。公曰:``寡人出亡二十年,乃今得反国。咎犯闻之,不喜而哭,意不欲寡人反国耶?''犯对曰:``笾豆所以食也,而君捐之;席蓐所以卧也,而君弃之。手足胼胝,面目黧黑,劳有功者也,而君后之。今臣有与在后,中不胜其哀。故哭。且臣为君行诈伪以反国者众矣。臣尚自恶也,而况于君。''再拜而辞。文公止之曰:``谚曰:'筑社者攐撅而置之,端冕而祀之。'今子与我取之,而不与我治之,与我置之,而不与我祀之焉。''乃解左骖而盟于河。

郑县人卜子使其妻为裤,其妻问曰:``今裤何如?''夫曰:``象吾故裤。''妻子因毁新令如故裤。

郑县人有得车轭者,而不知其名,问人曰:``此何种也?''对曰:``此车轭也。''俄又复得一,问人曰:``此何种也?''对曰:``此车轭也。''问者大怒曰:``曩者曰车轭,今又曰车轭,是何众也?此女欺我也!``遂与之斗。

卫人有佐弋者,鸟至,因先以其裷濬麾之,鸟惊而不射也。

郑县人卜子妻之市,买鳖以归。过颍水,以为渴也,因纵而饮之,遂亡其鳖。

夫少者侍长者饮,长者饮,亦自饮也。

一曰:鲁人有自喜者,见长年饮酒不能釂则唾之,亦效唾之。一曰:宋人有少者亦欲效善,见长者饮无余,非堪酒饮也,而欲尽之。

书曰:``绅之束之。''宋人有治者,因重带自绅束也。人曰:``是何也?''对曰:``书言之,固然。''

书曰:``既雕既琢,还归其朴。''梁人有治者,动作言学,举事于文,曰:``难之。''顾失其实。人曰:``是何也?''对曰:``书言之,固然。''

郢人有遗燕相国书者,夜书,火不明,因谓持烛者曰:``举烛``而误书``举烛``。举烛,非书意也。燕相国受书而说之,曰:``举烛者,尚明也;尚明也者,举贤而任之。''燕相白王,王大说,国以治。治则治矣,非书意也。今世学者,多似此类。

郑人有且置履者,先自度其足而置之其坐,至之市而忘操之。已得履,乃曰:``吾忘持度,反归取之。''及反,市罢,遂不得履。人曰:``何不试之以足?''曰:``宁信度,无自信也。''

△说四

王登为中牟令,上言于襄主曰:``中牟有士曰中章、胥己者,其身甚修,其学甚博,君何不举之?''主曰:``子见之,我将为中大夫。''相室谏曰:``中大夫,晋重列也。今无功而受,非晋臣之意。君其耳而未之邪!``襄主曰:``我取登,既耳而目之矣,登之所取又耳而目之,是耳目人绝无已也。''王登一日而见二中大夫,予之田宅。中牟之人弃其田耘、卖宅圃而随文学者,邑之半。

叔向御坐平公请事,公腓痛足痹转筋而不敢坏坐。晋国闻之,皆曰:``叔向贤者,平公礼之,转筋而不敢坏坐。''晋国之辞仕托,慕叔向者国之锤矣。

郑县人有屈公者,闻敌,恐,因死;恐已,因生。

赵主父使李疵视中山可攻不也。还报曰:``中山可伐也。君不亟伐,将后齐、燕。''主父曰:``何故可攻?''李疵对曰:``其君见好岩穴之士,所倾盖与车以见穷闾陋巷之士以十数,伉礼下布衣之士以百数矣。''君曰:``以子言论,是贤君也,安可攻?''疵曰:``不然。夫好显岩穴之士而朝之,则战士怠于行阵;上尊学者,下士居朝,则农夫惰于田。战士怠于行阵者,则兵弱也;农夫惰于田者,则国贫也。兵弱于敌,国贫于内,而不亡者,未之有也。伐之不亦可乎?''主父曰:``善。''举兵而伐中山,遂灭也。

△说五

齐桓公好服紫,一国尽服紫。当是时也,五素不得一紫。桓公患之,谓管仲曰:``寡人好服紫,紫贵甚,一国百姓好服紫不已,寡人奈何?''管仲曰:``君欲止之,何不试勿衣紫也?谓左右曰:'吾甚恶紫之臭。'于是左右适有衣紫而进者,公必曰:'少却,吾恶紫臭。'``公曰:``诺。''于是日,郎中莫衣紫,其明日,国中莫衣紫;三日,境内莫衣紫也。

一曰:齐王好衣紫,齐人皆好也。齐国五素不得一紫。齐王患紫贵,傅说王曰:``《诗》云:'不躬不亲,庶民不信。'今王欲民无衣紫者,王请自解紫衣而朝,群臣有紫衣进者,曰:'益远!寡人恶臭。'``是日也,郎中莫衣紫;是月也,国中莫衣紫;是岁也,境内莫衣紫。

郑简公谓子产曰:``国小,迫于荆、晋之间。今城郭不完,兵甲不备,不可以待不虞。''子产曰:``臣闭其外也已远矣,而守其内也已固矣,虽国小,犹不危之也。君其勿忧。''是以没简公身无患。

一曰:子产相郑,简公谓子产曰:``饮酒不乐也。俎豆不大,钟鼓竽瑟不鸣,寡人之事不一,国家不定,百姓不治,耕战不辑睦,亦子之罪。子有职,寡人亦有职,各守其职。''子产退而为政五年,国无盗贼,道不拾遗,桃枣廕于街者莫有援也,锥刀遗道三日可反。三年不变,民无饥也。

宋襄公与楚人战于涿谷上,宋人既成列矣,楚人未及济,右司马购强趋而谏曰:``楚人众而宋人寡,请使楚人半涉,未成列而击之,必败。''襄公曰:``寡人闻君子曰:'不重伤,不擒二毛,不推人于险,不迫人于厄。不鼓不成列。'今楚未济而击之,害义。请使楚人毕涉成阵而后鼓士进之。''右司马曰:``君不爱宋民,腹心不完,特为义耳。''公曰:``不反列,且行法。''右司马反列,楚人已成列撰阵矣,公乃鼓之。宋人大败,公伤股,三日而死。此乃慕自亲仁义之祸。夫必恃人主之自躬亲而后民听从,是则将令人主耕以为上,服战雁行也民乃肯耕战,则人主不泰危乎!而人臣不泰安乎!

齐景公游少海,传骑从中来谒曰:``婴疾甚,且死,恐公后之。''景公遽起,传骑又至。景公曰:``趋驾烦且之乘,使驺子韩枢御之。''行数百步,以驺为不疾,夺辔代之御;可数百步,以马为不进,尽释车而走。以且烦之良而驺子韩枢之巧,而以为不如下走也。

魏昭王欲与官事,谓孟尝君曰:``寡人欲与官事。''君曰:``王欲与官事,则何不试习读法?''昭王读法十余简而睡卧矣。王曰:``寡人不能读此法。''夫不躬亲其势柄,而欲为人臣所宜为者也,睡不亦宜乎。

孔子曰:``为人君者犹盂也,民犹水也。盂方水方,盂圜水圜。''

邹君好服长缨,左右皆服长缨,缨甚贵。邹君患之,问左右,左右曰:``君好服,百姓亦多服,是以贵。''君因先自断其缨而出,国中皆不服缨。君不能下令为百姓服度以禁之,乃断缨出以示先民,是先戮以莅民也。

叔向赋猎,功多者受多,功少者受少。

韩昭侯谓申子曰:``法度甚不易行也。''申子曰:``法者,见功而与赏,因能而受官。今君设法度而听左右之请,此所以难行也。''昭侯曰:``吾自今以来,知行法矣,寡人奚听矣。''一日,申子请仕其从兄官。昭侯曰:``非所学于子也。听子之谒,败子之道乎?亡其用子之谒。''申子辟舍请罪。

△说六

晋文公攻原,裹十日粮,遂与大夫期十日。至原十日,而原不下,击金而退,罢兵而去。士有从原中出者,曰:``原三日即下矣。''群臣左右谏曰:``夫原之食竭力尽矣,君姑待之。''公曰:``吾与士期十日,不去,是亡吾信也。得原失信,吾不为也。''遂罢兵而去。原人闻曰:``有君如彼其信也,可无归乎?''乃降公。卫人闻曰:``有君如彼其信也,可无从乎?''乃降公。孔子闻而记之曰:``攻原得卫者,信也。''

文公问箕郑曰:``救饿奈何?''对曰:``信。''公曰:``安信?''曰:``信名,信事,信义。信名则群臣守职,善恶不逾,百事不怠;信事则不失天时,百姓不逾;信义则近亲劝勉,而远者归之矣。''

吴起出,遇故人而止之食。故人曰:``诺``期返而食。''吴子曰:``待公而食。''故人\textless{}至暮\textgreater{}不来,起不食而待之。明日早,令人求故人。故人来,方与之食。

魏文侯与虞人期猎。明日,会天疾风,左右止文侯,不听,曰:``不可以风疾之故而失信,吾不为也。''遂自驱车往,犯风而罢虞人。

曾子之妻之市,其子随之而泣,其母曰:``女还,顾反为女杀彘。''妻适市来,曾子欲捕彘杀之。妻止之曰:``特与婴兒戏耳。''曾子曰:``婴兒非与戏也。婴兒非有知也,待父母而学者也,听父母之教。今子欺之,是教子欺也。母欺子,子而不信其母,非以成教也。''遂烹彘也。

楚厉王有警鼓与百姓为戒,饮酒醉,过而击之也。民大惊。使人止之曰:``吾醉而与左右戏击之也。''民皆罢。居数月,有警,击鼓而民不赴,乃更令明号而民信之。

李悝警其两和,曰:``谨警敌人,旦暮且至击汝。''如是者再三而敌不至,两和懈怠,不信李悝。居数月,秦人来袭之,至几夺其军。此不信患也。

一曰:李悝与秦人战,谓左和曰:``速上!右和已上矣。''又驰而至右和曰:``左和已上矣。''左右和曰:``上矣。''于是皆争上。其明年,与秦人战。秦人袭之,至几夺其军。此不信之患。

有相与讼者。子产离之,而毋使通辞,到至其言以告而知也。

惠嗣公使人伪关市,关市呵难之,因事关市以金,关市乃合之。嗣公谓关市曰:``某时有客过而予汝金,因谴之。''关市大恐,以嗣公为明察

\hypertarget{header-n1332}{%
\subsection{外储说左下}\label{header-n1332}}

△经一

以罪受诛,人不怨上,跀危坐子皋。以功受赏,臣不德君,翟璜操右契而乘轩。襄王不知,故昭卯五乘而履\textless{}尸桥\textgreater{}。上不过任,臣不诬能,即臣将为夫少室周。

△经二

恃势而不恃信,故东郭牙议管仲。恃术而不恃信,故浑轩非文公。故有术之主,信赏以尽能,必罚以禁邪,虽有驳行,必得所利。简主之相阳虎,哀公问``一足``。

△经三

失臣主之理,则文王自履而矜。不易朝燕之处,则季孙终身庄而遇贼。

△经四

利所禁,禁所利,虽神不行。誉所罪,毁所赏,虽尧不治。夫为门而不使入委利而不使进,乱之所以产也。齐侯不听左右,魏主不听誉者,而明察群臣,则鉅不费金钱,孱不用璧。西门豹请复治邺,足以知之。犹盗婴儿之矜裘与跀危子荣衣。子绰左右画,去蚁驱蝇。安得无桓公之忧索官与宣主之患臞马也?

△经五

臣以卑俭为行,则爵不足以观赏;宠光无节,则臣下侵逼。说在苗贲皇非献伯,孔子议晏婴。故仲尼论管仲与孙叔敖。而出入之容变,阳虎之言见其臣。而简主之应人臣也失主术。朋党相和,臣下得欲,则人主孤;群臣功成名就举,下不相和,则人主明。阳虎将为赵武之贤、解狐之公,而简主以为枳棘,非所以教国也。

△经六

公室卑则忌直言,私行胜则少公功。说在文子之直言,武子之用杖;子产忠谏,子国谯怒;梁车用法而成侯收玺;管仲以公而国人谤怨。

△说一

孔子相卫,弟子子皋为狱吏,刖人足,所跀者守门。人有恶孔子于卫君者,曰:``尼欲作乱。''卫君欲执孔子。孔子走,弟子皆逃。子皋从出门,跀危引之而逃之门下室中,吏追不得。夜半,子皋问跀危曰:``吾不能亏主之法令而亲跀子之足,是子报仇之时,而子何故乃肯逃我?我何以得此于子?''跀危曰:``吾断足也,固吾罪当之,不可奈何。然方公之狱治臣也,公倾侧法令,先后臣以言,欲臣之免也甚,而臣知之。及狱决罪定,公憱然不悦,形于颜色,臣见又知之。非私臣而然也,夫天性仁心固然也。此臣之所以悦而德公也。''

孔子曰:``善为利者树德,不能为吏者树怨。概者,平量者也;吏者,平法者也。治国者,不可失平也。''

田子方从齐之魏,望翟黄乘轩骑驾出,方以为文侯也,移车异路避之,则徒翟黄也。方问曰:``子奚乘是车也?''曰:``君谋欲伐中山,臣荐翟角而谋果;且伐之,臣荐乐羊而中山拔;得中山,忧欲治之,臣荐李克而中山治;是以君赐此车。''方曰:``宠之称功尚薄。''

秦、韩攻魏,昭卯西说而秦、韩罢;齐、荆攻魏,卯东说而齐、荆罢。魏襄王养之以五乘。卯曰:``伯夷以将军葬于首阳山之下,而天下曰:`夫以伯夷之贤与其称仁,而以将军葬,是手足不掩也。'今臣罢四国之兵,而王乃与臣五乘,此其称功,犹嬴胜而履蹻。''

少室周者,古之贞廉洁悫者也,为赵襄主力士。与中牟徐子角力,不若也,入言之襄主以自代也。襄主曰:``子之处,人之所欲也,何为言徐子以自代?''曰:``臣以力事君者也。今徐子力多臣,臣不以自代也。恐他人言之而为罪也。''

一曰:少室周为襄主骖乘,至晋阳,有力士牛子耕,与角力而不胜。周言于主曰:``主之所以使臣骖乘者,以臣多力也。今有多力于臣者,愿进之。``

△说二

齐桓公将立管仲,令群臣曰:``寡人才将立管仲为仲父。善者入门而左,不善者入门而右。''东郭牙中门而立。公曰:寡人立管仲为仲父,令曰`善者左,不善者右。'今子何为中门而立?''牙曰:``以管仲之智,为能谋天下乎?''公曰:``能。''``以断,为敢行大事乎?''公曰:``敢。''牙曰:``若知能谋天下,断敢行大事,君因专属之国柄焉。以管仲之能,乘公之势以治齐国,得无危乎?''公曰:``善。''乃令隰朋治内、管仲治外以相参。

晋文公出亡,箕郑挈壶餐而从,迷而失道,与公相失,饥而道泣,寝饿而不敢食。及文公反国,举兵攻原,克而拔之。文公曰:``夫轻忍饥馁之患而必全壶餐,是将不以原叛。''乃举以为原令。大夫浑轩闻而非之,曰:``以不动壶餐之故,怙其不以原叛也,不亦无术乎?''故明主者,不恃其不我叛也,恃吾不可叛也;不恃其不我欺也,恃吾不可欺也。

阳虎议曰:``主贤明,则悉心以事之;不肖,则饰奸而试之。''逐于鲁,疑于齐,走而之赵,赵简主迎而相之。左右曰:``虎善窃人国政,何故相也?''简主曰:``阳虎务取之,我务守之。''遂执术而御之。阳虎不敢为非,以善事简主,兴主之强,几至于霸也。

鲁哀公问于孔子曰:``吾闻古者有夔一足,其果信有一足乎?''孔子对曰:``不也,夔非一足也。夔者忿戾恶心,人多不说喜也。虽然,其所以得免于人害者,以其信也。人皆曰:`独此一,足矣。'夔非一足也,一而足也。''哀公曰:``审而是,固足矣。''

一曰:哀公问于孔子曰:``吾闻夔一足,信乎?:''曰:``夔,人也,何故也足?彼其无他异,而独通于声。尧曰:`夔一而足矣。'使为乐正。故君子曰:`夔有一,足。非一足也。''

△说三

文王伐崇,至凤黄虚,袜系解,因自结。太公望曰:``何为也?''王曰:``上,君与处皆其师;中,皆其友;下,尽其使也。今皆先君之臣,故无可使也。''

一曰:晋文公与楚战,至黄凤之陵,履系解,因自结之。左右曰:``不可以使人乎?''公曰:``吾闻:上,君所与居,皆其所畏也;中,君之所与居,皆其所爱也;下,君之所与居,皆其所侮也。寡人虽不肖,先君之人皆在,是以难之也。''

季孙好士,终身庄,居处衣服常如朝迁。而季孙适懈,有过失,而不能长为也。故客以为厌易已,相与怨之,遂杀季孙。故君子去泰去甚。

一曰:南宫敬子问颜涿聚曰:``季孙养孔子之徒,所朝服与坐者以十数,而遇贼,何也?''曰:``昔周成王近优侏儒以逞其意,而与君子断事,是能成其欲于天下。今季孙养孔子之徒,所朝服而与坐者以十数,而与优侏儒断事,是以遇贼。故曰:不在所与居,在所与谋也。''

孔子侍坐于鲁哀公,哀公赐之桃与黍。哀公曰:``请用。''仲尼先饭黍而后啖桃,左右皆掩口而笑。哀公曰:``黍者,非饭之也,以雪桃也。''仲尼对曰:``丘知之矣。夫黍者,五谷之长也,祭先王为盛。果蓏有六,而桃为下,祭先王不得入庙。丘之闻也,君子贱雪贵,不闻以贵雪贱。今以五谷之长雪果蓏之下,是以上雪下也。丘以为妨义,故不敢以先于宗庙之盛也。''

简主谓左右:``车席泰美。夫冠虽贱,头必戴之;屡虽贵,足必履之。今车席如此,太美,吾将何\textless{}尸桥\textgreater{}以履之?夫美下而耗上,妨义之本也。''费仲说纣曰:``西伯昌贤,百姓悦之,诸候附焉,不可不诛;不诛,必为殷祸。''纣曰:``子言,义主,何可诛?''费仲曰:``冠虽穿弊,必戴于头;履虽五采,必践之于地。今西伯昌,人臣也,修义而人向之,卒为天下患,其必昌乎!人人不以其贤为其主,非可不诛也。且主而诛臣,焉有过?''纣曰:``夫仁义者,上所以劝下也。今昌好仁义,诛之不可。''三说不用,故亡。

齐宣王问匡倩曰:``儒者博乎?''曰:``不也。''王曰:``何也?''匡倩对曰:``博者贵枭,胜者必杀枭。杀枭者,是杀所贵也。儒者以为害义,故不博也。''又问曰:``儒者弋乎?''曰:``不也。弋者,从下害于上者也,是从下伤君也,儒者以为害义,故不弋。''又问:``儒者鼓瑟乎:``曰:``不也。夫瑟以小弦为大声,以大弦为小声,是大小易序,贵贱易位,儒者以为害义,故不鼓也。''宣王曰:``善。''仲尼曰:``与其使民谄下也,宁使民谄上。''

△说四

钜者,齐之居士;孱者,魏之居士。齐、魏之君不明,不能亲照境内,而听左右之言,故二子费金璧而求入仕也。

西门豹为鄴令,清克洁欲,秋毫之端无私利也,而甚简左右。左右因相与比周而恶之。居期年,上计,君收其玺。豹自请曰:``臣昔者不知所以治鄴,今臣得矣,原请玺复以治鄴。不当,请伏斧锧之罪。''文候不忍而复与之。豹因重敛百姓,急事左右。期年,上计,文候迎而拜之。豹对曰:``往年臣为君治鄴,而君夺臣玺;今臣为左右治鄴,而君拜臣。臣不能治矣。''遂纳玺而去。文候不受,曰:``寡人曩不知子,今知矣。愿子勉为寡人治之。''遂不受。

齐有狗盗之子,与刖危子戏而相夸。盗子曰:``吾父之裘独有尾。''危子曰:``吾父独冬不失裤。''

子绰曰:``人莫能左画方而右画圆也。以肉去蚁蚁愈多,以鱼驱蝇蝇愈至。''

桓公谓管仲曰:``官少而索者众,寡人忧之。''管仲曰:``君无听左右之请,因能而受禄,禄功而与官,则莫敢索官。君何患焉?''

韩宣子曰:``吾马菽粟多矣,甚臞,何也?寡人患之。''周市对曰:``使驺尽粟以食,虽无肥,不可得也。名为多与之,其实少,虽无臞,亦不可得也。主不审其情实,坐而患之,马犹不肥也。''

桓公问置吏于管仲,管仲曰:``辩察于辞,清洁于货,习人情,夷吾不如弦商,请立以为大理。登降肃,以明礼待宾,臣不如隰朋,请立以为大行。垦草创邑,辟地生粟,臣不如宁戚,请以为大田。三军既成阵,使士视死如归,臣不如公子成父,请以为大司马。犯颜极谏,臣不如东郭牙,请立以为谏臣。治齐,此五子足矣;将欲霸王,夷吾在此。''

△说五

孟献伯相晋,堂下生藿藜,门外长荆棘,食不二味,坐不重席,晋无衣帛之妾,居不粟马,出不从车。叔向闻之,以告苗贲皇。贲皇非之曰:``是出主之爵禄以付下也。''

一曰:盂献伯拜上卿,叔向往贺,门有御马不食禾。向曰:``子无二马二与,何也?''献伯曰:``吾观国人尚有饥色,是以不秣马;班白者多以徒行,故不二舆。''向曰:``吾始贺子之拜卿,今贺子之俭也。''向出,语苗贲皇曰:``助吾贺献伯之俭也。''苗子曰:``何贺焉?夫爵禄旗章,所以异功伐,别贤不肖也。故晋国之法,上大夫二舆二乘,中大夫二舆一乘,下大夫专乘,此明等级也。且夫卿必有军事,是故修车马,比卒乘,以备戎事。有难,则以备不虞;平夷,则以给朝事。今乱晋国之政,乏不虞之备,以成节,以洁私名,献伯之俭也可与?又何贺?''

管仲相齐,曰:``臣贵矣,然而臣贫。''桓公曰:``使子有三归之家。''曰:``臣富矣,然而臣卑。''桓公使立于高、国之上。曰:``臣尊矣,然而臣疏。''乃立为仲父。孔子闻而非之曰:``泰侈逼上。''

一曰:管仲父出,硃盖青衣,置鼓而归,庭有陈鼎,家有三鼎。孔子曰:``良大夫也,其侈逼上。''

孙叔敖相楚,栈车牝马,粝饼菜羹,枯鱼之膳,冬羔裘,夏葛衣,面有饥色,则良大夫也。其俭逼下。

阳虎去齐走赵,简主问曰:``吾闻子善树人。''虎曰:``臣居鲁,树三人,皆为令尹;及虎抵罪于鲁,皆搜索于虎也。臣居齐,荐三人,一人得近王,一人为县令,一人为候吏;及臣得罪,近王者不见臣,县令者迎臣执缚,候吏者追臣至境上,不及而止。虎不善树人。,``主俯而笑曰:``夫树柤梨橘柚者,食之则甘,嗅之则香;树枳棘者,成而刺人。故君子慎所树。''

中牟无令,晋平公问赵武曰:``中牟,三国之股肱,邯郸之肩髀,寡人欲得其良令也,谁使而可?''武曰:``邢伯子可。''公曰:``非子之仇也?''曰:``私仇不入公门。''公又问曰:``中府之令,谁使而可?''曰:``臣子可。''故曰:``外举不避仇,内举不避子。''赵武所荐四十六人于其君,及武死,各就宾位,其无私德若此也。

平公问叔向曰:``群臣孰贤?''曰:``赵武。''公曰:``子党于师人。''向曰:``武立如不胜衣,言如不出口,然所举士也数十人,皆得其意,而公家甚赖之。况武子之生也不利于家,死不托于孤,臣敢以为贤也。''

解狐荐其仇于简主以为相。其仇以为且幸释己也,乃因住拜谢。狐乃引弓迎而射之,曰:``夫荐汝,公也,以汝能当之也。夫仇汝,吾私怨也,不以私怨汝之故拥汝于吾君。''故私怨不入公门。

一曰:解狐举邢伯柳为上党守,柳往谢之,曰:``子释罪,敢不再拜?''曰:``举子,公也;怨子,私也。子往矣,怨子如初也。''

郑县人卖豚,人问其价。曰:``道远日暮,安暇语汝。''

△说六

范文子喜直言,武子击之以杖:``夫直议者不为人所容,无所容则危身,非徒危身,又将危父。''

子产者,子国之子也。子产忠于郑君,子国谯怒之曰:``夫介异于人臣,而独忠于主。主贤明,能听汝;不明,将不汝听。听与不听,未可必知,而汝已离于群臣;离于群臣,则必危汝身矣。非徒危己也,又且危父也。''

梁车新为邺令,其姊往看之,暮而后,门闭;因逾郭而入。车遂刖其足。赵成侯以为不兹,夺之玺而免之令。

管仲束缚,自鲁之齐,道而饥渴,过绮乌封人而乞食。鸟封人跪而食之,甚敬。封人因窃谓仲曰:``适幸,及齐不死而用齐,将何报我?''曰:``如子之言,我且贤之用,能之使,劳之论。我何以报子?''封人怨之。

\hypertarget{header-n1391}{%
\subsection{外储说右上}\label{header-n1391}}

君所以治臣者有三:

△经一

势不足以化则除之。师旷之对,晏子之说,皆合势之易也,而道行之难,是与兽逐走也,未知除患。患之可除,在子夏之说《春秋》也:``善持势者,蚤绝其奸萌。''故季孙让仲尼以遇势,而况错之于君乎。是以太公望杀狂矞,而臧获不乘骥。嗣公知之,故不驾鹿。薛公知之,故与二孪博。此皆知同异之反也。故明主之牧臣也,说在畜鸟。

△经二

人主者,利害之轺毂也,射者众,故人主共矣。是以好恶见则下有因,而人主惑矣;辞言通则臣难言,而主不神矣。说在申子之言``六慎``,与唐易之言弋也。患在国羊之请变,与宣王之太息也。明之以靖郭氏之献十珥也,与犀首、甘茂之道穴闻也。堂谷公知术,故问玉卮;昭候能术,故以听独寝。明主之道,在申子之劝独断也。

△经三

术之不行,有故。不杀其狗则酒酸。夫国也有狗,且左右皆社鼠也。人主无尧之再诛,与庄王之应太子,而皆有薄媪之决蔡妪也。知贵不能以教歌之法先揆之。吴起之出爱妻,文公之斩颠颉,皆违其情者也。故能使人弹疽者,秘其忍痛者也。

△说一

赏之誉不劝,罚之毁之不畏.四者加焉不变,则除之。

齐景公之晋,从平公饮,师旷侍坐。景公问政于师旷曰:``太师将奚以教寡人?''师旷曰:``君必惠民而已。''中坐,酒酣,将出,又复问政于师旷曰:``太师奚以教寡人?''。师旷曰:``君必惠民而已矣。''景公出之舍,师旷送之,又问政于师旷。师旷曰:``君必惠民而已矣。''景公归思,未醒,而得师旷之所谓公子尾、公子夏者,景公之二弟也,甚得齐民,家富贵而说之,拟于公室,此危吾位者也,今谓我惠民,使我与二弟争民邪?于是反国,发禀粟以赋众贫,散府馀财以赐孤寡,仓无陈粟,府无馀财,宫妇不御者出嫁之,七十受禄米,鬻德惠施于民也,已与二弟争民。居二年,二弟出走,公子夏逃楚,公子尾走晋。

景公与晏了子游于少海,登柏寝之台而还望其国曰:``美哉!泱泱乎,堂堂乎!后世将孰有此?''晏子对曰:``其田成氏乎!``景公曰:``寡人有此国也,而曰田成氏有之,何也?''晏子对曰:``夫田氏甚得齐民,其于民也,上之请爵禄行诸大臣,下之私大斗斛区釜以出贷,小斗斛区釜以收之。杀一牛,取一豆肉,馀以食士。终岁,布帛取二制焉,馀以衣士。故市木之价不加贵于山,泽之鱼监龟鳖赢蚌不贵于海。君重敛,而田成氏厚施。齐尝大饥,道旁饿死者不可胜数也,父子相牵而趋田成氏者,不闻不生。故秦周之民相与歌之曰:'讴乎,其已乎!苞乎,其往归田成子乎!'《诗》曰:'虽无德与女,式歌且舞。'今田成氏之德而民之歌舞,民德归之矣。故曰:'其田成氏乎!'``公泫然出涕曰:``不亦悲乎!寡人有国而田成氏有之,今为之奈何?''晏子对曰:``君何患焉?若君欲夺之,则近贤而远不肖,治其烦乱,缓其刑罚,振贫穷而恤孤寡,行恩惠而给不足,民将归君,则虽有十田成氏,其如君何?''

或曰:景公不知用势,而师旷、晏子不知患。夫猎者托车舆之安,用六马之足,使王良佐辔,则身不劳而易及轻兽矣。今释车舆之利,捐六马之足与王良之御,而下走逐兽,则虽楼季之足无时及兽矣。托良马固车,则臧获有馀。国者,君之车也;势者,君之马也。夫不处势以禁诛擅爱之臣,而必德厚以与天下齐行以争民,是皆不乘君之车,为因马之利,释车而下走者也。故曰:景公不知用势之主也,而师旷、晏子不知除患之臣也。

子夏曰:``《春秋》之记臣杀君、子杀父者,以十数矣,皆非一日之积也,有渐而以至矣。凡奸者,行久而成积,积成而力多,力多而能杀,故明主蚤绝之。''今田常之为乱,有渐见矣,而君不诛。晏子不使其君禁侵陵之臣,而使其主行惠,故简公受其祸。故子夏曰:``善持势者,蚤绝奸之萌。''

季孙相鲁,子路为郈令。鲁以五月起众为长沟,当此之为,子路以其私秧粟为浆饭,要作沟者于五父之衢而飡之。孔子闻之,使子贡往覆其饭,击毁其器,曰:``鲁君有民,子奚为乃餐之?''子路怫然怒,攘肱而入,请曰:``夫子疾由之为仁义乎?所学于夫子者,仁义也;仁义者,与天下共其所有而同其利其也。今以由之伯粟而餐民,其不可何也?''孔子曰:``由之野也!吾以女知之,女徒未及也。女故如是之不知礼也!女之餐之,为受之也。夫礼,天子爱天下,诸候爱境内,大夫爱官职,士家其家,过其所受曰侵。今鲁君有民而子擅爱之,是子侵也,不亦诬乎!``言未卒,而季孙使者至,让曰:``肥也起民而使之,先生使弟子止徒役而餐之,将夺肥之民耶?''孔子驾而去鲁。以孔子之贤,而季孙非鲁君也,以人臣之资,假人主之术,蚤禁于未形,而子路不得行其私惠,而害不得生,况人主乎!以景公之势而禁田常之侵也,则必无劫弑之患矣。

太公望东封于齐,齐东海上有居士曰狂矞、华士昆弟二人者立议曰:``吾不臣天子,不友诸侯,耕作而食之,掘井而饮之,吾无求于人也。无上之名,无君之禄,不事仕而事力。''太公望至于营丘,使吏执而杀之,以为首诛。周公旦从鲁闻之,发急传而问之曰:``夫二子,贤者也。今日飨国而杀贤者,何也?''太公望曰:``是昆弟二人立议曰:'吾不臣天子,不友诸侯,耕作而食之,掘井而饮之,吾无求于人也。无上之名,无君之禄,不事仕而事力。'彼不臣天子者,是望不得而臣也;不友诸侯者,是望不得而使也;耕作而食之,掘井而饮之,无求于人者,是望不得以赏罚劝禁也。且无上名,虽知,不为望用;不仰君禄,虽贤,不为望功。不仕,则不治;不任,则不忠。且先王之所以使其臣民者,非爵禄则刑罚也。今四者不足以使之,则望当谁为君乎?不服兵革而显,不亲耕耨而名,又非所以教于国也。今有马于此,如骥之状者,天下之至良也。然而驱之不前,却之不止,左之不左,右之不右,则臧获虽贱,不托其足。臧获之所愿托其足于骥者,以骥之可以追利辟害也。今不为人用,臧获虽贱,不托其足焉。已自谓以为世之贤士,而不为主用,行极贤而不用于君,此非明主之所以臣也,亦骥之不可左右矣,是以诛之。''

一曰:太公望东封于齐。海上有贤者狂矞,太公望闻之,往请焉,三却马于门而狂矞不报见也,太公望诛之。当是时也,周公旦在鲁,驰往止之;比至,已诛之矣。周公旦曰:狂矞,天下贤者也,夫子何为诛之?''太公望曰:``狂矞也议不臣天子,不友诸候,吾恐其乱法易教也,故以为首诛。今有马于此,形容似骥也,然驱之不往,引之不前,虽臧获不托足于其轸也。''

如耳说卫嗣公,卫嗣公说而太息。左右曰:``公何为不相也?''公曰:``夫马似鹿者,而题之千金。然而有百金之马而无千金之鹿者,何也?马为人用而鹿不为人用也。今如耳万乘之相也,外有大国之意,其心不在卫,虽辩知,亦不为寡人用,吾是以不相也。''

薛公子相魏昭候也,左右有栾子者曰阳胡,潘,其于王甚重,而不为薛公。薛公患之,于是乃召与之博,予之人百金,令之昆弟博;俄又益之人二百金。方博有问,谒者言客张季之子在门,公怫然怒,抚兵而授谒者曰:``杀之!吾闻季之不为文也。''立有间,时季羽在侧,曰:``不然。窃闻季为公甚,顾其人阴未闻耳。''乃辍不杀客大礼之,曰:``曩者闻季之不为文也,故欲杀之;今诚为文也,岂忘季哉!``告廪献千石之粟,告府献五百金,告驺私厩献良马固车二乘,因令奄将宫人之美妾二十人并遗季也。栾子因相谓曰:``为公者必利,不为公者必害,吾曹何爱不为公?''因私竞劝而遂为之。薛公以人臣之势,假人主之术也,而害不得生,况错之人主乎!夫驯鸟者断其下翎,则必恃人而食,焉得不驯乎?夫明主畜臣亦然,令臣不得不利君之禄,不得无服上之名。夫利君之禄,服上之名,焉得不服?

△说二

申子曰:``上明见,人备之;其不明见,人惑之。其知见,人饰之;不知见,人匿之。其无欲见,人司之;其有欲见,人饵之。故曰:吾无从知之,惟无为可以规之。''

一曰:申子曰:``慎而言也,人且知女;慎而行也,人且随女。而有知见也,人且匿女;而无知见也,人且意女。女有知也,人且臧女;女无知也,人且行女。故曰:惟无为可以规之。''

田子方问唐易鞠曰:``弋者何慎?''对曰``鸟以数百目视子,子以二目御之,子谨周子禀。''田子方曰:``善。子加之弋,我加之国。''郑长者闻之曰:``田子方知欲为禀,而未得所以为禀。夫虚无无见者,禀见。''

一曰:齐宣王问弋于唐易子曰:``弋者奚贵?''唐易子曰:``在于谨禀。''王曰:``何谓谨禀?''对曰:``鸟以数十目视人,人以二目视鸟,奈何不谨禀也?故曰'在于谨禀'也。''王曰:``然则为天下何以为此禀?今人主以二目视一国,一国以万目视人主,将何以自为禀乎?''对曰:``郑长者有言曰:'夫虚静无为而无见也。'其可以为此禀乎!``

国羊重于郑君,闻君之恶己也,侍饮,因先谓君曰:``臣适不幸而有过,愿君幸而告之。臣请变更,则臣免死罪矣。''

客有说韩宣王,宣王说而太息。左右引王之说之,以先告客以为德。

靖郭君之相齐也,王后死,未知所置,乃献玉珥以知之。

一曰:薛公相齐,齐威王夫人死,中有十孺子,皆贵于王,薛公欲知王所欲立,而请置一人以为夫人。王听之,则是说行于王而重于置夫人也;王不听,是说不行而轻于置夫人也。欲先知王之所欲置以劝王置之,于是为十玉耳而美其一而献之。王以赋十孺子,明日坐,视美珥之所在而劝王以为夫人。

甘茂相秦惠王,惠王爱公孙衍,与之间有所言,曰:``寡人将相子。''甘茂之吏道穴闻之,以告甘茂。甘茂入见王,曰:``王得贤相,臣敢再拜贺。''``寡人托国于子,安更得贤相?''对曰:``将相犀首。''王曰:``子安闻之?''对曰:``犀首告臣。''王怒犀道之泄,乃逐之。

一曰:犀首,天下之善将也,梁王之臣也。秦王欲得之与治天下,犀首曰:``衍人臣也,不敢离主之国。''居期年,犀首抵罪于梁王,逃而入秦,秦王甚善之。樗里疾,秦之将也,恐犀首之代之将也,凿穴于王之所常隐语者。俄而王果与犀首计,曰:吾欲攻韩,奚如?''犀首曰:``秋可矣。''王曰:``吾欲以国累子,子必勿泄也。''犀首反走再拜曰:``受命。''于是樗是疾已道穴听之矣。郎中皆曰:``兵秋起攻韩,犀首为将。''于是日也,郎中尽知之;于是月也,境内尽知之。王召樗里疾曰:``是何匈匈也,何道出?''樗里疾曰:``似犀首也。''王曰:``吾无与犀首言也,其犀首何哉?''樗里疾曰:``犀首也羁旅新抵罪,其孤,是言自嫁于众。''王曰:``然。''使人召犀首,已逃诸候矣。

堂谷公谓昭候曰:``今有千金之玉卮而无当,可以盛水乎?''昭候曰:``不可。''``有瓦器而不漏,可以盛酒乎?''昭候曰:``可。''对曰:``夫瓦器,至贱也,不漏可以盛酒。虽有千金之玉卮,至贵而无当,漏不可盛水,则人孰注浆哉?今为人之主而漏其君臣之语,是犹无当之玉卮也,虽有圣智,莫尽其术,为其漏也。''昭候曰:``然。''昭侯闻堂谷公之言,自此之后,欲发天下之大事,未尝不独寝,恐梦言而使人知其谋也。

一曰:堂谷公见昭候曰:``今有白玉之卮而无当,有瓦卮而无当。君渴,将何以饮?''君曰:``以瓦卮。''堂鸡公曰:``白玉之卮美,而君不以饮者,以其无当耶?''君曰:``然。''堂谷公曰:``为人主而漏泄其君臣之语,譬犹玉卮之无当。''堂谷公每见而出,昭候必独卧,惟恐梦言泄于妻妾。

申子曰:``独视者谓明,独听者为聪。能独断者,故可以为天下主。''

说三

宋人有酤酒者,升概甚平,遇客甚谨,为酒甚美,县帜甚高,然而不售,酒酸。怪其故,问其所知闾长者杨倩,倩曰:``汝狗猛耶?''曰:``狗猛则酒何故而不售?''曰:``人畏焉。或令孺子怀钱挈壶雍而往酤,而狗迓而龁之,此酒所以酸而不售也。''夫国亦有狗铬,有道之士怀其术而欲以明万乘之主,大臣为猛狗迎而龁之,此人主之所以蔽肋,而有道这士所以不用也。故桓公问管仲:``治国最奚患?''对曰:``最患社鼠矣。''公曰:``何患社鼠哉?''对曰:``君亦见夫为社者乎?树木而涂之,鼠穿其间,掘穴托其中。熏之则恐焚木,灌之则恐涂阤,此社鼠之所以不得也。今人君之左右,出则为势重而收利于民,入则比周而蔽恶于君。内间主之情以告外,外内为重,诸臣百吏以为富。吏不诛则乱法,诛之则君不安。据而有之,此亦国之社鼠也。''故人臣执柄而擅禁,明为己者必利,而不为己者必害,此亦猛狗也。夫大臣为猛狗而龁有道之士矣,左右又为社鼠而间主之情,人主不觉。如此,主焉得无壅,国焉得无亡乎?

一曰:宋之酤酒者有庄氏者,其酒常美。或使仆往酤庄氏之酒,其狗龁人,使者不敢往,乃酤他家之酒。问曰:``何为不酤庄氏之酒?''对曰:``今日庄氏之酒酸。''故曰:``不杀其狗则酒酸。''一曰:桓公问管仲曰:``治国何患?''对曰:``最苦社鼠。夫社,木而涂之,鼠因自托也。熏之则木焚,灌之则涂阤,此所以苦于社鼠也。今人君左右,出则为势重以收利于民,入则比周谩侮蔽恶以欺于君,不诛则乱法,诛之则人主危。据而有之,此亦社鼠也。''故人臣执柄擅禁,明为己者必利,不为己者必害,亦猛狗也。故左右为社鼠,用事者为猛狗,则术不行矣。

尧欲传天下于舜,鲧谏曰:``不祥哉!孰以天下而传之天匹夫乎?''尧不听,举兵而诛杀鲧于羽山之郊。共工又谏曰:``孰以天下而传之于匹夫乎?''尧不听,又举兵而诛共工于幽州之都。于是天下莫敢言无传天下于舜。仲尼闻之曰:``尧之知舜之贤,非其难者也。夫至乎诛谏者,必传之舜,乃其难也。''一曰:``不以其所疑败其所察则难也。''

荆庄王有茅门之法,曰:``群臣大夫诸公子入朝,马蹄践霤者,廷理斩其辀戮其御。''于是太子入朝,马蹄践霤,廷理斩其辀,戮其御。太子怒,入为王泣曰:``为我诛戮廷理。''王曰:``法者,所以敬宗庙,尊社稷。故能立法从令,尊敬社稷者,社稷之臣也,焉可诛也?夫犯法废令,不尊敬社稷者,是臣乘君而下尚校也。臣乘君,则主失威;下尚校则上位危。威失位危,社稷不守,吾将何以遗子孙?''于是太子乃还走,避舍露宿三曰,北面再拜请死罪。

一曰:楚王急召太子。楚国之法,车不得至于茆门。天雨,廷中有潦,太子遂驱车至于茆门。廷理曰:``车不得至茆门。至茆门,非法也。''太子曰:``王召急,不得须无潦。''遂驱之。廷理举殳而击其马,败其驾。太子入为王泣曰:``廷中多潦,驱车至茆门,廷理曰'非法也',举殳击臣马,败臣驾。王必诛之。''王曰:``前有老主而不逾,后有储主而不属,矜矣!是真吾守法之牙也。''乃益爵二级,而开后门出太子,勿复过。

卫嗣君谓薄疑曰:``子小寡人之国以为不足仕,则寡人力能仕子,请进爵以子为上卿。''乃进田万顷。薄子曰:``疑之母亲疑,以疑为能相万乘所不窕也。然疑家巫有蔡妪者,疑母甚爱信之,属之家事焉。疑智足以信言家事,疑母尽以听疑也。然已与肄言者,亦必复决之于蔡妪也。故论疑之智能,以疑为能相万乘而不窕也;论其亲,则子母之间也;然犹不免议之于蔡妪也。今疑之于人主也,非子母之亲也,而人主皆有蔡妪。人主之蔡妪,必其重人也。重人者,能行私者也。夫行私者,绳之外也;而疑之所言,法之内也。绳之外与法之内,仇也,不相受也。''

一曰:卫君之晋,谓薄疑曰:``吾欲与子皆行。''薄疑曰:``媪也在中,请归与媪计之。卫君自请薄媪。曰:``疑,君之臣也,君有意从之,甚善。''卫君曰:``吾以请之媪,媪许我矣。''薄疑归,言之媪也,曰:``卫君之爱疑奚与媪?''媪曰:``不如吾爱子也。''``卫君之贤疑奚与媪也?''曰:``不如吾贤子也。''``媪与疑计家事已决矣,乃更请决之于卜者蔡妪。今卫君从疑而行,虽与疑决计,必与他蔡妪败之。如是,则疑不得则长为臣矣。''

夫教歌者,使先呼而诎之,其声反清徵者,乃教之。

一曰:教歌者先揆以法,疾呼中宫,徐呼中徵。疾不中宫,徐不中徵,不可谓教。

吴起,卫左氏中人也,使其妻织组,而幅狭于度。吴子使更之。其妻曰:``诺。''及成,复度之,果不中度,吴子大怒。其妻对曰:``吾始经之而不可更也。''吴子出之,其妻请其兄而索入,其兄曰:``吴子,为法者也。其为法也,且欲以与万乘致功,必先践之妻妾,然后行之,子毋几索入矣。''其妻之弟又重于卫君,乃因以卫君之重请吴子。吴子不听,遂去卫而入``

一曰:吴起示其妻以组,曰:``子为我织组,令之如是。''组已就而效之,其组异善。起曰:``使子为组,令之如是,而今也异善,何也?''其妻曰:``用财若一也,加务善之。''吴起曰:``非语也。''使之衣而归。其父往请之,吴起曰:``起家无虚言。''

晋文公问于狐偃曰:``寡人甘肥周于堂,卮酒豆肉集于宫,壶酒不清,生肉不布,杀一牛遍于国中,一岁之功尽以衣士卒,其足以战民乎?''狐子曰:``不足。''文公曰:``吾弛关市之征而缓刑罚,其足以战民乎?''狐子对曰:``不足。''文公曰:``吾民之有丧资者,寡人亲使郎中视事,有罪者赦之,贫穷不足者与之,其足以战民乎?''狐子对曰:``不足。此皆所以慎产也;而战之者,杀之也。民之从公也,为慎产也,公因而迎杀之,失所以为从公矣。''曰:然则何如足以战民乎?''狐子对曰:``令无得不战。''公曰:``无得不战奈何?''狐子对曰:``信赏必罚,其足以战。''公曰:``刑罚之极安至?''对曰:``不辟亲贵,法行所爱。''文公曰:``善。''明日,令田于圃陆,期以日中为期,后期者行军法焉。于是公有所爱者日颠颉,后期,吏请其罪,文公陨涕而忧。吏曰:``请用事焉。''遂斩颠颉之脊以徇百姓,以明法之信也。而后百姓皆惧曰:``君于颠颉之贵重如彼甚也,而君犹行法焉,况于我则何有矣。''文公见民之可战也,于是遂兴兵伐原,克之;伐卫,东其亩,取五鹿;攻阳胜虢;伐曹;南围郑,反之陴;罢宋围。还与荆人战城濮,大败荆人;返为践土之盟,遂成衡雍之义:一举而八有功。所以然者,无他故异物,从狐偃之谋,假颠颉之脊也。

夫痤疽之痛也,非刺骨髓,则烦心不可支也;非如是,不能使人以半寸砥石弹之。今人主之于治亦然:非人不知有若则安;欲治其国,非如是不能听圣知则诛乱臣。乱臣者必重人,重人者,必人主所甚亲爱也。人主所甚亲爱也者,是同坚白也。夫以布衣之资,欲以离人主之坚白所爱,是犹以解左髀说右髀者,是身必死而说不行者也。

\hypertarget{header-n1438}{%
\subsection{外储说右下}\label{header-n1438}}

△经一

赏罚共则禁令不行。何以明之?明之以造父、于期。子罕为出彘,田恒为圃池,故宋君、简公弑。患在王良、造父之共车,田连、成房之共琴也。

△经二

治强生于法,弱乱生于阿,君明于此,则正赏罚而非仁下也。爵禄生于功,诛罚生于罪,臣明于此,则尽死力而非忠君也。君通于不仁,臣通于不忠,则可以王矣。昭襄知主情而不发五苑,田鲔知臣情故教田章,而公仪辞鱼。

△经三

明主者,鉴于外也,而外事不得不成,故苏代非齐王。人主鉴于上也,而居者不适不显,故潘寿言禹情。人主无所觉悟,方吾知之,故恐同衣同族,而况借于权乎!吴章知之,故说以佯,而况借于诚乎!赵王恶虎目而壅。明主之道,如周行人之却卫侯也。

△经四

人主者,守法责成以立功者也。闻有吏虽乱而有独善之民,不闻有乱民而有独治之吏,故明主治吏不治民。说在摇木之本与引网之纲。故失火之啬夫,不可不论也。救火者,吏操壶走火,则一人之用也;操鞭使人,则役万夫。故所遇术者,如造父之遇惊马,牵马推车则不能进,代御执辔持策则马咸骛矣。是以说在椎锻平夷,榜檠矫直。不然,败在淖齿用齐戮闵王,李兑用赵饿主父也。

△经五

国事之理,则不劳而成。故兹郑之踞辕而歌以上高梁也。其患在赵简主税吏清轻重;薄疑之言``国中饱'',简主喜而府库虚,百姓饿而奸吏富也。故桓公巡民而管仲省腐财怨女。不然,则在延陵乘马不得进,造父过之而为之泣也。

△说一

造父御四马,驰骤周旋而恣欲于马。恣欲于马者,擅辔策之制也。然马惊于出彘而造父不能禁制者,非辔策之严不足也,威分于出彘也。王子于期为驸驾,辔策不用而择欲于马,擅刍水之利也。然马过于圃池而驸驾败者,非刍水之利不足也,德分子圃池也。故王良、造父,天下之善御者也,然而使王良操左革而叱咤之,使造父操右革而鞭笞之,马不能行十里,共故也。田连、成窍,天下善鼓琴者也,然而田连鼓上、成窍擑(音叶,用手指按)下而不能成曲,亦共故也。夫以王良、造父之巧,共辔而御,不能使马,人主安能与其臣共权以为治?以田连、成窍之巧,共琴而不能成曲,人主又安能与其臣共势以成功乎?

一曰:造父为齐王驸驾,渴马服成,效驾圃中。渴马见圃他,去车走池,驾败。王子于期为赵简主取道争千里之表,其始发也,彘伏沟中,王子于期齐辔策而进之,彘突出于沟中,马惊驾败。

司城子罕谓宋君曰:``庆赏赐与,展之所喜也,君自行之;杀戮诛罚,民之所恶也,臣访当之。''宋君曰:``诺。''于是出威令,诛大臣。君曰``问子罕''也。于是大臣畏之,细民归之。处期年,子罕杀宋君而夺政。故子罕为出彘以夺其君国。

简公在上位,罚重而诛严,厚赋敛而杀戮民。田成恒设慈爱,明宽厚。简公以齐民为渴马,不以恩加民,而田成恒以仁厚为圃地也。

一曰:造父为齐王驸驾,以渴服马,百日而服成。服成,请效驾齐王,王曰;``效驾于圃中。''造父驱车入圃,马见圃池而走,造父不能禁。造父以渴服马久矣,今马见池,駻而走,虽造父不能治。今简公之以法禁其众久矣,而田成恒利之,是田成恒倾圃池而示渴民也。

一曰:王子于期为宋君为千里之逐。已驾,察手吻文。且发矣,驱而前之,轮中绳;引而却之,马掩迹。拊而发之。彘逸出于窦中。马退而却,策不能进前也;马駻而走,辔不能正也。

一曰:司城子罕谓宋君曰:``庆赏赐予者,民之所好也,君自行之;诛罚杀戮者,民之所恶也,臣访当之。''于是戮细民而诛大臣,君曰:``与子罕议之。''居期年,民知杀生之命制于子罕也,故一国归焉。故子罕劫宋君而夺其政,法不能禁也。故曰:``子罕为出彘,而田成常为圃池也。''令王良、造父共车,人操一边辔而出门闾,驾必败而道不至也。令田连、成窍共琴,人抚一弦而挥,则音必败、曲不遂突。

△说二

秦昭王有病,百姓里买牛而家为王祷。公孙述出见之,人贺王曰:``百姓乃皆里买牛为王祷。''王使人问之,果有之。王曰:``訾之人二甲。夫非令而擅祷,是爱寡人也。夫爱寡人,寡人亦且改法而心与之相循者,是法不立;法不立,乱亡之道也。不如人罚二甲而复与为治。''

一曰:秦襄王病,百姓为之祷;病愈,杀牛塞祷。郎中阎遏、公孙衍出见之,曰:``非社腊之时也,奚自杀牛而祠社?''怪而问之。百姓曰:``人主病,为之祷;今病愈,杀牛塞祷。''阎遏、公孙衍说,见王,拜贺曰:``过尧、舜矣。''王惊曰:``何谓也?''对曰:``尧、舜,其民未至为之祷也。今王病而民以牛祷,病愈,杀牛塞祷,故臣窃以王为过治、舜也。''王因使人问之,何里为之,訾其里正与伍老屯二甲。阎遏、公孙衍愧不敢言。居数月,王饮酒酣乐,阎遏、公孙衍谓王曰:``前时臣窃以王为过尧、舜,非直敢谀也。尧、舜病,且其民未至为之祷也;分王病,而民以牛祷,病愈,杀牛塞涛。今乃訾其里正与伍老屯二甲,臣窃怪之。''王曰:``于何故不知于此?彼民之所以为我用者,非以吾爱之为我用者也,以吾势之为我用者也。吾释势与民相收,若是,吾适不爱而民因不为我用也,故遂绝爱道也。''

秦大饥,应侯请曰:``五苑之草著:蔬菜、、橡果、枣栗,足以活民,清发之。``昭襄王曰:``吾秦法,使民有功而受赏,有罪而受诛。今发五苑之蔬草者,使民有功与无功俱赏也。夫使民有功与无功俱赏者,此乱之道也。夫发五苑而乱,不如弃枣蔬而治。''一曰:``令发五苑之蓏、蔬、枣、栗,足以活民,是用民有功与无功争取也。夫生乱,不如死而治,大夫其释之。''

田鲸教其子田章曰:``欲利而身,先利而君;欲富而家,先富而国。''

一曰:田鲔数其子田章曰:``主卖官爵,臣卖智力,故自恃无恃人。''

公仪休相鲁而嗜鱼,一国尽争买鱼而献之,公议子不受。其弟谏曰:``夫子嗜鱼而不受者,何也?''对日:``夫唯嗜鱼,故不受也。夫即受鱼,必有下人之色;有下人之色,将枉于法;枉于法,则免于相。虽嗜鱼,此不必致我鱼,我又不能自给鱼。即无受鱼而不免于相,虽嗜鱼,我能长自给鱼。''此明夫恃人不如自恃也,明于人之为己者不如己之自为也。

△说三

子之相燕,贵而主断。苏代为齐使燕,王问之曰:``齐王亦何如主也?''对曰:``必不霸矣。''燕王曰:``何也?''对曰:``昔桓公之霸也,内事属鲍叔,外事属管仲,桓公被发而御妇人,日游于市。今齐王不信其大臣。''于是燕王因益大信子之。子之闻之,使人遗苏代金百镒,而听其所使。

一曰:苏代为齐使燕,见无益子之,则必不得事而还,贡赐又不出,于是见燕王,乃誉齐王。燕王曰:``齐王何若是之贤也?则将必王乎?''苏代曰:``救亡不暇,安得王哉?''燕王曰:``何也?''曰:``其任所爱不均。''燕王曰:``其亡何也?''曰:``昔者齐桓公爱管仲,置以为仲父,内事理焉,外事断焉,举国而归之,故一匡天下,九合诸侯。今齐任所爱不均,是以知其亡也。''燕王曰:``今吾任子之,天下未之闻也?''于是明日张朝而听子之。

潘寿谓燕王曰:``王不如以国让子之。人所以谓尧贤者,以其让天下于许由,许由必不受也,则是尧有让许由之名而实不失天下也。今王以国让子之,子之必不受也,则是王有让子之之名而与尧同行也。''于是燕王因举国而属之,子之大重。

一曰:潘寿,隐者。燕使人聘之。潘寿见燕王曰:``臣恐子之之如益也。''王曰:``何益哉?对曰:``古者禹死,将传天下于益,启之人因相与攻益而立启。今王信爱子之,将传国子之,太子之人尽怀印,为子之之人无一人在朝廷者。王不幸弃群臣,则子之亦益也。''王因收吏玺,自三百石以上皆效之子之,子之大重。夫人主之所以镜照者,诸侯之士徒也,今诸侯之士徒皆私门之党也。人主之所以自浅娋者,岩穴之士徒也,今岩穴之士徒皆私门之舍人也。是何也?夺褫之资在子之也。故吴章曰:``人主不佯僧爱人。佯爱人,不得复憎也;佯憎人,不得复爱也。''

一曰:燕王欲传国于子之也,问之潘寿,对曰:``禹爱益而任天下于益,已而以启人为吏。及老,而以启为不足任天下,故传天下于益,而势重尽在启也。已而启与友党攻益而夺之天下,是禹名传天下子益,而实令启自取之也。此禹之不及尧、舜明矣。今王欲传之子之,而吏无非太子之人者也,是名传之而实令太于自取之也。''燕王乃收玺,自三百石以上皆效之子之,子之遂重。

方吾子曰:``吾闻之古礼:行不与同服者同车,不与同族者共家,而况君人者乃借其权而外其势乎!``

吴章谓韩宣王曰:``人主不可佯爱人,一日不可复憎万;不可以佯憎人,一日不可复爱也。故佯憎佯爱之征见,则谀者因资而毁誉之。虽有明主,不能复收,而况于以诚借人也!''

赵天游于圃中,左右以兔与虎而辍,盻然环其眼。王曰:``可恶哉,虎目也!''左右曰:``平阳君之目可恶过此。见此未有害也,见平阳君之目如此者,则必死矣。''其明日,平阳君闻之,使人杀言者,而王不诛也。

卫君入朝于周,周行人问其号,对曰:``诸侯辟疆。''周行人却之曰:``诸侯不得与天子同号。''卫君乃自更曰:``诸侯燬。''而后内之。什尼闻之曰:``远哉禁逼!虚名不以借人,况实事乎?''

△说四

摇木者一一摄其叶,则劳而不遍;左右拊其本,而叶遍摇矣。临渊而摇木,鸟惊而高,鱼恐而下。善张网者引其纲,若一一摄万目而后得,则是劳而难;引其纲,而鱼已囊矣。故吏者,民之本、纲者也,故圣人治吏不治民。

救火者,令吏挈壶瓮而走火,则一人之用也;操鞭箠指麾而趣使人,则制万夫。是以圣人不亲细民,明主不躬小事。造父方耨,时有子父乘车过者,马惊而不行,其子下车牵马,父子推车,请造助我推车。造父因收器,辍而寄载之,援其子之乘。乃始检辔持策,未之用也,而马咸骛矣。使造父而不能御,虽尽力劳身助之推车,马犹不肯行也。今身使佚,且寄载,有德于人者,有术而御之也。故国者,君之车也;势者,君之马也。无术以御之,身虽劳,犹不免乱;有术以御之,身处佚乐之地,又致帝王之功也。

椎锻者,所以平不夷也;榜檠者,所以矫不宜也。圣人之为法也,所以平不夷、矫不直也。淖齿之用齐也,擢闵王之筋;李兑之用赵也,饿杀主父。此二君者,皆不能用其椎锻榜檠,故身死为戮而为天下笑。

一曰:入齐,则独闻淖齿而不闻齐王;人赵,则独闻李兑而不闻赵王。故曰:人主者不操术,则威势轻而臣擅名。

一曰:武灵王使惠文王莅政,李兑为相,武灵王不以身躬亲杀生之柄,故劫于李兑。

一曰:田婴相齐,人有说王者曰:``终岁之计,王不一以数日之间自听之,则无以知吏之奸邪得失也。''王曰:``善。''田婴闻之,即送请于王而听其计。王将听之矣,田婴令官具押券斗石参升之计。王自听计,计不胜听,罢食后,复坐,不复暮食美。田婴复谓曰:``群臣所终岁日夜不敢偷怠之事也,王以一夕听之,则群臣有为劝勉矣。''王曰:``诺。''俄而王已睡矣,吏尽揄刀削其押券升石之计。王听之,乱乃始生。

△说五

兹郑子引辇上高梁而不能支。兹郑踞辕而歌,前者止,后者趋,辇乃上。使兹郑无术以致人,则身虽绝力至死,辇犹不上也。今身不至劳苦而辇以上者,有术以致人之故也。

赵简主出税者,吏请轻重。简主曰:``勿轻勿重。重,则利入于上;若轻,则利归于民。束无私利而正矣。''薄疑调赵简主曰:``君之国中饱。''简主欣然而喜曰:``何如焉?''对曰:``府库空虚于上,百姓贫饿于下,然而奸吏富矣。''

齐桓公微服以巡民家,人有年老而自养者,桓公问其故。对日:``臣有子三人,家贫无以妻之,佣未反。''桓公归,以告管仲。管仲曰:``畜积有腐弃之财,则人饥饿;宫中有怨女,则民无妻。''桓公曰:``善。''乃论宫中有妇人而嫁之。下令于民日:``丈夫二十而室,妇人十五而嫁。''

一曰:桓公微服而行于民间,有鹿门稷者,行年七十而无妻。桓公问管仲曰:``有民老而无妻者平?''管仲曰:``有鹿门稷者,行年七十矣而无妻。''桓公曰:``何以令之有妻?''管仲曰:``臣闻之:上有积财,则民臣必匾乏于下;宫中有怨女,则有老而无妻者。''桓公曰:``善。''令于宫中``女子未尝御出嫁之''。乃令男子年二十而室,女年十五而嫁。则内无怨女,外无旷夫。

延陵卓子乘苍龙挑文之乘,钩饰在前,错錣在后。马欲进则钩饰禁之,欲退则错錣贯之,马因旁出。造父过而为之泣涕,曰:``古之治人亦然矣。夫赏所以劝之,而毁存焉;罚所以禁之,而誉加焉。民中立而不知所由,此亦圣人之所为泣也。''

一曰:延陵卓子乘苍龙与翟文之乘,前则有错饰,后则有利鎚,进则引之,退则策之。马前不得进,后不得退,遂避而逸,因下抽刀而刎其脚。造父见之,泣,终日不食,因仰天而叹曰:``策,所以进之也,错饰在前;引,所以退之也,利錣在后。今人主以其清洁也进之,以其不适左右也退之;以其公正也誉之,以其不听从也废之。民惧,中立而不知所由,此圣人之所为泣也。''

\hypertarget{header-n1488}{%
\subsection{难一}\label{header-n1488}}

一

晋文公将与楚人战,召舅犯问之,曰:``吾将与楚人战,彼众我寡,为之奈何?''舅犯曰:``臣闻之,繁礼君子,不厌忠信;战阵之间,不厌诈伪。君其诈之而已矣。''文公辞舅犯,因召雍季而问之,曰:``我将与楚人战,彼众我寡,为之奈何?''雍季对曰:``焚林而田,偷取多兽,后必无兽;以诈遇民,偷取一时,后必无复。''文公曰:``善。''辞雍季,以舅犯之谋与楚人战以败之。归而行爵,先雍季而后舅犯。群臣曰:``城濮之事,舅犯谋也。夫用其言而后其身,可乎?''文公曰:``此非君所知也。夫舅犯言,一时之权也;雍季言,万世之利也。''仲尼闻之,曰:``文公之霸也,宜哉!既知一时之权,又知万世之利。''

或曰:雍季之对,不当文公之问。凡对问者,有因问小大缓急而对也。所问高大,而对以卑狭,则明主弗受也。今文公问``以少遇众'',而对曰``后必无复'',此非所以应也。且文公不不知一时之权,又不知万世之利。战而胜,则国安而身定,兵强而威立,虽有后复,莫大于此,万世之利奚患不至?战而不胜,则国亡兵弱,身死名息,拔拂今日之死不及,安暇待万世之利?待万世之利,在今日之胜;今日之胜,在诈于敌;诈敌,万世之利而已。故曰:雍季之对,不当文公之问。且文公不知舅犯之言。舅犯所谓``不厌诈伪''者,不谓诈其民,谓诈其敌也。敌者,所伐之国也,后虽无复,何伤哉?文公之所以先雍季者,以其功耶?则所以胜楚破军者,舅犯之谋也;以其善言耶?则雍季乃道其``后之无复''也,此未有善言也。舅犯则以兼之矣。舅犯曰``繁礼君子,不厌忠信''者:忠,所以爱其下也;信,所以不欺其民也。夫既以爱而不欺矣,言孰善于此?然必曰``出于诈伪''者,军旅之计也。舅犯前有善言,后有战胜,故舅犯有二功而后论,雍季无一焉而先赏。``文公之霸,不亦宜乎?''仲尼不知善赏也。

二

历山之农者侵畔,舜往耕焉,期年。甽亩正。河滨之渔者争坻,舜往渔焉,期年而让长。东夷之陶者器苦窳,舜往陶焉,期年而器牢。仲尼叹曰:``耕、渔与陶,非舜官也,而舜往为之者,所以救败也。舜其信仁乎!乃躬藉处苦而民从之。故曰:``圣人之德化乎!''

或问儒者曰:``方此时也,尧安在?''其人曰:``尧为天子。''``然则仲尼之圣尧奈何?圣人明察在上位,将使天下无奸也。今耕渔不争,陶器不窳,舜又何德而化?舜之救败也,则是尧有失也。贤舜,则去尧之明察;圣尧,则去舜之德化:不可两得也。楚人有鬻盾与矛者,誉之曰:`盾之坚,莫能陷也。'又誉其矛曰:`吾矛之利,于物无不陷也。'或曰:`以子之矛陷子之盾,何如?'其人弗能应也。夫不可陷之盾与无不陷之矛,不可同世而立。今尧、舜之不可两誉,矛盾之说也。且舜救败,期年已一过,三年已三过。舜有尽,寿有尽,天下过无已者,有尽逐无已,所止者寡矣。赏罚使天下必行之,令曰:`中程者赏,弗中程者诛。'令朝至暮变,暮至朝变,十日而海内毕矣,奚待期年?舜犹不以此说尧令从己,乃躬亲,不亦无术乎?且夫以身为苦而后化民者,尧、舜之所难也;处势而骄下者,庸主之所易也。将治天下,释庸主之所易,道尧、舜之所难,未可与为政也。''

三

管仲有病,桓公往问之,曰:``仲父病,不幸卒于大命,将奚以告寡人?''管仲曰:``微君言,臣故将谒之。愿君去竖刁,除易牙,远卫公子开方。易牙为君主,惟人肉未尝,易牙烝其子首而进之。夫人唯情莫不爱其子,今弗爱其子,安能爱君?君妒而好内,竖刁自宫以治内。人情莫不爱其身,身且不爱,安能爱君?闻开方事君十五年,齐、卫之间不容数日行,弃其母,久宦不归。其母不爱,安能爱君?臣闻之:`矜伪不长,盖虚不久。'愿君久去此三子者也。''管仲卒死,桓公弗行。及桓公死,虫出尸不葬。

或曰:管仲所以见告桓公者,非有度者之言也。所以去竖刁、易牙者,以不爱其身,适君之欲也。曰:``不爱其身,安能爱君?''然则臣有尽死力以为其主者,管仲将弗用也。曰``不爱其死力,安能爱君?''是君去忠臣也。且以不爱其身度其不爱其君,是将以管仲之不能死公子纠度其不死桓公也,是管仲亦在所去之域矣。明主之道不然,设民所欲以求其功,故为爵禄以劝之;设民所恶以禁其奸,故为刑罚以威之。庆赏信而刑罚必,故君举功于臣而奸不用于上,虽有竖刁,其奈君何?且臣尽死力以与君市,君垂爵禄以与臣市。君臣之际,非父子之亲也,计数之所出也。君有道,则臣尽力而奸不生;无道,则臣上塞主明而下成私。管仲非明此度数于桓公也,使去竖刁,一竖刁又至,非绝奸之道也。且桓公所以身死虫流出尸不葬者,是臣重也。臣重之实,擅主也。有擅主之臣,则君令不下究,臣情不上通。一人之力能隔君臣之间,使善败不闻,祸福不通,故有不葬之患也。明主之道:一人不兼官,一官不兼事;卑贱不待尊贵而进论,大臣不因左右而见;百官修通,群臣辐凑;有赏者君见其功,有罚者君知其罪。见知不悖于前,赏罚不弊于后,安有不葬之患?管仲非明此言于桓公也,使去三子,故曰:管仲无度矣。

四

襄子围于晋阳中,出围,赏有功者五人,高赫为赏首。张孟谈曰:``晋阳之事,赫无大功,今为赏首,何也?''襄子曰:``晋阳之事,寡人国家危,社稷殆矣。吾群臣无有不骄侮之意者,惟赫子不失君臣之礼,是以先之。仲尼闻之曰:``善赏哉!襄子赏一人而天下为人臣者莫敢失礼矣。''

或曰:仲尼不知善赏矣。夫善赏罚者,百官不敢侵职,群臣不敢失礼。上设其法,而下无奸诈之心。如此,则可谓善赏罚矣。使襄子于晋阳也,令不行,禁不止,是襄子无国,晋阳无君也,尚谁与守哉?今襄子于晋阳也,知氏灌之,曰灶生龟,而民无反心,是君臣亲也。襄子有君臣亲之泽,操令行禁止之法,而犹有骄侮之臣,是襄子失罚也。为人臣者,乘事而有功则赏。今赫仅不骄侮,而襄子赏之,是失赏也。明主赏不加于无功,罚不加于无罪。今襄子不诛骄侮之臣,而赏无功之赫,安在襄子之善赏也?故曰:``仲尼不知善赏。

五

晋平公与群臣饮,饮酣,乃喟然叹曰:``莫乐为人君,惟其言而莫之违。''师旷侍坐于前,援琴撞之。公披衽而避,琴坏于壁。公曰:``太师谁撞?''师旷曰:``今者有小人言于侧者,故撞之。''公曰:``寡人也。''师旷曰:``哑!是非君人者之言也。''左右请除之,公曰:``释之,以为寡人戒。''

或曰:平公失君道,师旷失臣礼。夫非其行而诛其身,君子于臣也;非其行则陈其言,善谏不听则远其身者,臣之于君也。今师旷非平公之行,不陈人臣之谏,而行人主之诛,举琴而亲其体,是逆上下之位,而失人臣之礼也。夫为人臣者,君有过则谏,谏不听则轻爵禄以待之,此人臣之礼义也。今师旷非平公之过,举琴而亲其体,虽严父不加于子,而师旷行之于君,此大逆之术也。臣行大逆,平公喜而听之,是失君道也。故平公之迹不可明也,使人主过于听而不悟其失;师旷之行亦不可明也,使奸臣袭极谏而饰弑君之道。不可谓两明,此为两过。故曰:平公失君道,师旷亦失臣礼矣。

六

齐桓公时,有处士曰小臣稷,桓公三往而弗得见。桓公曰:``吾闻布衣之士不轻爵禄,无以易万乘之主;万乘之主不好仁义,亦无以下布衣之士。''于是五往乃得见之。

或曰:桓公不知仁义。夫仁义者,忧天下之害,趋一国之患,不避卑辱谓之仁义。故伊尹以中国为乱,道为宰于汤;百里奚以秦为乱,道虏于穆公。皆忧天下之害,趋一国之患,不辞卑辱,故谓之仁义。今桓公以万乘之势,下匹夫之士,将欲忧齐国,而小臣不行,见小臣之忘民也。忘民不可谓仁义。仁义者,不失人臣之礼,不败君臣之位者也。是故四封之内,执会而朝名曰臣,臣吏分职受事名曰萌。今小臣在民萌之众,而逆君上之欲,故不可谓仁义。仁义不在焉,桓公又从而礼之。使小臣有智能而遁桓公,是隐也,宜刑;若无智能而虚骄矜桓公,是诬也,宜戮。小臣之行,非刑则戮。桓公不能领臣主之理而礼刑戮之人,是桓公以轻上侮君之俗教于齐国也,非所以为治也。故曰:桓公不知仁义。

七

靡笄之役,韩献子将斩人。郄献子闻之,驾往救之。比至,则已斩之矣。郄子因曰:``胡不以徇?''其仆曰:``曩不将救之乎?''郄子曰:``吾敢不分谤乎?''

或曰:``郄子言,不可不察也,非分谤也。韩子之所斩也,若罪人,不可救,救罪人,法之所以败也,法败则国乱;若非罪人,则劝之以徇,劝之以徇,是重不辜也,重不辜,民所以起怨者也,民怨则国危郄子之言,非危则乱,不可不察也。且韩子之所斩若罪人,郄子奚分焉?斩若非罪人,则已斩之矣,而郄子乃至,是韩子之谤已成而郄子且后至也。夫郄子曰``以徇'',不足以分斩人之谤,而又生徇之谤。是子言分谤也?昔者纣为炮烙,崇侯、恶来又曰斩涉者之胫也,奚分于纣之谤?且民之望于上也甚矣,韩子弗得,且望郄子之得也;今郄子俱弗得,则民绝望于上矣。故曰:郄子之言非分谤也,益谤也。且郄子之往救罪也,以韩子为非也;不道其所以为非,而劝之``以徇'',是使韩子不知其过也。夫下使民望绝于上,又使韩子不知其失,吾未得郄子之所以分谤者也。

八

桓公解管仲之束缚而相之。管仲曰:``臣有宠矣,然而臣卑。''公曰:``使子立高、国之上。''管仲曰:``臣贵矣,然而臣贫。''公曰:``使子有三归之家。''管仲曰:``臣富矣,然而臣疏。''于是立以为仲父。霄略曰:``管仲以贱为不可以治国,故请高、国之上;以贫为不可以治富,故请三归;以疏为不可以治亲,故处仲父。管仲非贪。以便治也。''

或曰:今使臧获奉君令诏卿相,莫敢不听,非卿相卑而臧获尊也,主令所加,莫敢不从也。今使管仲之治不缘桓公,是无君也,国无君不可以为治。若负桓公之威,下桓公之令,是臧获之所以信也,奚待高、国、仲父之尊而后行哉?当世之行事、都丞之下征令者,不辟尊贵,不就卑贱。故行之而法者,虽巷伯信乎卿相;行之而非法者,虽大吏诎乎民萌。今管仲不务尊主明法,而事增宠益爵,是非管仲贪欲富贵,必暗而不知术也。故曰:管仲有失行,霄略有过誉。

九

韩宣王问于樛留:``吾欲两用公仲、公叔,其可乎?''樛留对曰:``昔魏两用楼、翟而亡西河,楚两用昭、景而亡鄢、郢。今君两用公仲、公叔,此必将争事而外市,则国必忧矣。''

或曰:``昔者齐桓公两用管仲、鲍叔,成汤两用伊尹、仲虺。夫两用臣者国之忧,则是桓公不霸,成汤不王也。湣王一用淖齿,而手死乎东庙;主父一用李兑,减食而死。主有术,两用不为患;无术,两用则争事而外市,一则专制而劫弑。今留无术以规上,使其主去两用一,是不有西河、鄢、郢之忧,则必有身死减食之患,是樛留未有善以知之知言也。

\hypertarget{header-n1516}{%
\subsection{难二}\label{header-n1516}}

景公过晏子,曰:``子宫小,近市,请徙子家豫章之圃。''晏子再拜而辞曰:``且婴家贫,待市食,而朝暮趋之,不可以远。''景公笑曰:``子家习市,识贵贱乎?''是时景公繁于刑。晏子对曰:``踊贵而屦贱。''景公曰:``何故?''对曰:``刑多也。''景公造然变色曰:``寡人其暴乎!''于是损刑五。

或曰:晏子之贵踊,非其诚也,欲便辞以止多刑也。此不察治之患也。夫刑当无多,不当无少。无以不当闻,而以太多说,无术之患。败军之诛以千百数,犹且不止;即治乱之刑如恐不胜,而奸尚不尽。今晏子不察其当否,而以太多为说,不亦妄乎?夫惜草茅者耗禾穗,惠盗贼者伤良民。今缓刑罚,行宽惠,是利奸邪而害善人也,此非所以为治也。

二

齐桓公饮酒醉,遗其冠,耻之,三日不朝。管仲曰:``此非有国之耻也,公胡其不雪之以政?''公曰:``胡其善!''因发仓囷赐贫穷,论囹圄出薄罪。外三日而民歌之曰:``公胡不复遗冠乎!''

或曰:管仲雪桓公之耻天小人,而生桓公之耻于君子矣。使桓公发仓囷而赐贫穷,讼囹圄而出薄罪,非义也,不可以雪耻;使之而义也,桓公宿义,须遗冠而后行之,则是桓公行义非为遗冠也?是虽雪遗冠之耻于小人,而亦遗义之耻于君子矣。且夫发囷仓而赐贫穷者,是赏无功也;论囹圄而出薄罪者,是不诛过也。夫赏无功,则民偷幸而望于上;不诛过,则民不惩而易为非。此乱之本也,安可以雪耻哉?

三

昔者文王侵孟、克莒、举酆,三举事而纣恶之。文王乃惧,请入洛西立地、赤壤之国方千里,以请解炮烙之刑。天下皆说。仲尼闻之,曰:``仁哉,文王!轻千里之国而请解炮烙之刑。智哉,文王!出千里之地而得天下之心。''

或曰:仲尼以文王为智也,不亦过乎?夫智者,知祸难之地而辟之者也,是以身不及于患也。使文王所以见恶于纣者,以其不得人心耶,则虽索人心以解恶可也。纣以其大得人心而恶之,己又轻地以收人心,是重见疑也,固其所以桎梏、囚于姜里也。郑长者有言:``体道,无为无见也。''此最宜于文王矣,不使人疑之也。仲尼以文王为智,未及此论也。

四

晋平公问叔向曰:``昔者齐桓公九合诸侯,一匡天下,不识臣之力也?''叔向对曰:``管仲善制割,宾胥无善削缝,隰朋善纯缘,衣成,君举而服之。亦臣之力也,君何力之有?''师旷伏琴而笑之。公曰:``太师奚笑也?''师旷对曰:``臣笑叔向之对君也。凡为人臣者,犹炮宰和五味而进之君。君弗食,孰敢强之也?臣请譬之:君者,壤地也;臣者,草木也。必壤地美,然后草木硕大。亦君之力,臣何力之有?''

或曰:叔向、师旷之对,皆偏辞也。夫一匡天下,九合诸侯,美之大者也,非专君之力也,又非专臣之力也。昔者宫之奇在虞,僖负羁在曹,二臣之智,言中事,发中功,虞、曹俱亡者,何也?此有其臣而无其君者也。且蹇叔处干而干亡,处秦而秦霸,非蹇叔愚于干而智于秦也,此有臣与无臣也。向曰``臣之力也,''不然矣。昔者桓公宫中二市,妇闾二百,被发而御妇人。得管仲,为五伯长,失管仲、得竖刁而身死,虫流出尸不葬。以为非臣之力也,且不以管仲为霸;以为君之力也,且不以竖刁为乱。昔者晋文公慕于齐女而亡归,咎犯极谏,故使反晋国。故桓公以管仲合,文公以舅犯霸,而师旷曰``君之力也,''又不然矣。凡五霸所以能成功名于天下者,必君臣俱有力焉。故曰:叔向、师旷之对,皆偏辞也。

五

齐桓公之时,晋客至,有司请礼。桓公曰:``告仲父''者三。而优笑曰:``易哉,为君!一曰仲父,二曰仲父。''桓公曰:``吾闻君人者劳于索人,佚于使从。吾得仲父已难矣,得仲父之后,何为不易乎哉?''

或曰:桓公之所应优,非君人者之言也。桓公以君人为劳于索人,何索人为劳哉?伊尹自以为宰干汤,百里奚自以为虏干穆公。虏,所辱也;宰,所羞也。蒙羞辱而接君上,贤者之忧世急也。然则君人者无逆贤而已矣,索贤不为人主难。且官职,所以任贤也;爵禄,所以赏功也。设官职,陈爵禄,而士自至,君人者奚其劳哉?使人又非所佚也。人主虽使人,必度量准之,以刑名参之;以事遇于法则行,不遇于法则止;功当其言则赏,不当则诛。以刑名收臣,以度量准下,此不可释也,君人者焉佚哉?

索人不劳,使人不佚,而桓公曰:``劳于索人,佚于使人''者,不然。且桓公得管仲又不难。管仲不死其君而归桓公,鲍叔轻官让能而任之,桓公得管仲又不难,明矣。已得管仲之后,奚遽易哉?管仲非周公旦。周公旦假为天子七年,成王壮,授之以政,非为天下计也,为其职也。夫不夺子而行天下者,必不背死君而事其仇;背死君而事其仇者,必不难夺子而行天下;不难夺子而行天下者,必不难夺其君国矣。管仲,公子纠之臣也,谋杀桓公而不能,其君死而臣桓公,管仲之取舍非周公旦,未可知也。若使管仲大贤也,且为汤武,桀、纣之臣也;桀、纣作乱,汤、武夺之。今桓公以易居其上,是以桀、纣之行居汤、武之上,桓公危矣。若使管仲不肖人也,且为田常。田常,简公之臣也,而弑其君。今桓公以易居其上,是以简公之易居田常之上也,桓公又危矣。管仲非周公旦以明矣,然为汤、武与田常,未可知也。为汤、武,有桀、纣之危;为田常,有简公之乱也。已得仲父之后,桓公奚遽易哉?若使桓公之任管仲,必知不欺己也,是知不欺主之臣也。然虽知不欺主之臣,今桓公以任管仲之专借竖刁、易牙,虫流出尸而不葬,桓公不知臣欺主与不欺主已明矣,而任臣如彼其专也,故曰:桓公暗主。

六

李兑治中山,苦陉令上计而入多。李兑曰:``语言辨,听之说,不度于义,谓之窕言。无山林泽谷之利而入多者,谓之窕货。君子不听窕言,不受窕货。之姑免矣。''

或曰:李子设辞曰:``夫言语辩,听之说,不度于义者,谓之窕言。''辩,在言者;说,在听者:言非听者也。所谓不度于义,非谓听者,必谓所听也。听者,非小人,则君子也。小人无义,必不能度之义也;君子度之义,必不肯说也。夫曰:``言语辩,听之说,不度于义''者,必不诚之言也。入多之为窕货也,未可远行也。李子之奸弗蚤禁,使至于计,则遂过也。无术以知而入多,入多者,穰也,虽倍入,将奈何?举事慎阴阳之和,种树节四时之适,无早晚之失、寒温之灾,则入多。不以小功妨大务,不以私欲害人事,丈夫尽于耕农,妇人力于织纴,则入多。务于畜养之理,察于土地之宜,六畜遂,五谷殖,则入多。明于权计,审于地形、舟车、机械之利,用力少,致功大,则入多。利商市关梁之行,能以所有致所无,客商归之,外货留之,俭于财用,节于衣食,宫室器械周于资用,不事玩好,则入多。入多,皆人为也。若天事,风雨时,寒温适,土地不加大,而有丰年之功,则入多。人事、天功二物者皆入多,非山林泽谷之利也。夫无山林泽谷之利入多,因谓之窕货者,无术之害也。

七

赵简子围卫之郛郭,犀盾、犀橹,立于矢石之所及,鼓之而士不起。简子投枹曰:``乌乎!吾之士数弊也。''行人烛过免胄而对曰:``臣闻之:亦有君之不能士耳,士无弊者。昔者吾先君献公并国十七,服国三十八,战十有二胜,是民之用也。献公没,惠公即位,淫衍暴乱,身好玉女,秦人恣侵,去绛十七里,亦是人之用也。惠公没,文公授之,围卫,取邺,城濮之战,五败荆人,取尊名于天下,亦此人之用也。亦有君不能士耳,士无弊也。''简子乃去盾、橹,立矢石之所及,鼓之而士乘之,战大胜。简子曰:``与吾得革车千乘,不如闻行人烛过之一言也。''

或曰:行人未有以说也,乃道惠公以此人是败,文公以此人是霸,未见所以用人也。简子未可以速去盾、橹也。严亲在围,轻犯矢石,孝子之所爱亲也。孝子爱亲,百数之一也。今以为身处危而人尚可战,是以百族之子于上皆若孝子之爱亲也,是行人之诬也。好利恶害,夫人之所有也。赏厚而信,人轻敌矣;刑重而必,失人不比矣。长行徇上,数百不一失;喜利畏罪,人莫不然。将众者不出乎莫不然之数,而道乎百无失人之行,人未知众之道也。

\hypertarget{header-n1538}{%
\subsection{难三}\label{header-n1538}}

一

鲁穆公问于子思曰:``吾闻庞氏之子不孝,其行奚如?''子思对曰:``君子尊贤以崇德,举善以观民。若夫过行,是细人之所识也,臣不知也。''子思出。子服厉伯入见,问庞氏子,子服厉伯对曰:``其过三。''皆君之所未尝闻。自是这后,君贵子思而贱子服厉伯也。

或曰:鲁之公室,三世劫于季氏,不亦宜乎?明君求善而赏之,求奸而诛之,其得之一也。故以善闻之者,以说善同于上者也;以奸闻之者,以恶奸同于上者也:此宜赏誉之所及也。不以奸闻,是异于上而下比周于奸者也,此宜毁罚之所及也。今子思不以过闻而穆公贵之,厉伯以奸闻而穆公贱之。人情皆喜贵而恶贱,故季氏之乱成而不上闻,此鲁君之所以劫也。且此亡王之俗,取、鲁之民所以自美,而穆公独贵之,不亦倒乎?

二

文公出亡,献公使寺人披攻之蒲城,披斩其祛,文公奔翟。惠公即位,又使攻之惠窦,不得也。及文公反国,披求见。公曰:``蒲城之役,君令一宿,而汝即至;惠窦之难,君令三宿,而汝一宿,何其速也?''披对曰:``君令不二。除君之恶,恐不堪。蒲人、翟人,余何有焉?今公即位,其无蒲、翟乎?且桓公置射钩而相管仲。''君乃见之。

或曰:齐、晋绝祀,不亦宜乎?桓公能用管仲之功而忘射钩之怨,文公能听寺人之言而弃斩祛之罪,桓公、文公能容二子者也。后世之君,明不及二公;后世之臣,贤不如二子。不忠之臣以事不明之君,君不知,则有燕操、子罕、田常之贼;知之,则以管仲、寺人自解。君必不诛而自以为有桓、文之德,是臣仇而明不能烛,多假之资,自以为贤而不戒,则虽无后嗣,不亦可乎?且寺人之言也,直饰君令而不贰者,则是贞于君也。死君后生,臣不愧,而复为贞。今惠公朝卒而暮事文公,寺人之不贰何如?

三

人有设桓公隐者曰:``一难,二难,三难,何也?''桓公不能对,以告管仲。管仲对曰:``一难也,近优而远士。二难也,去其国而数之海。三难也,君老而晚置太子。''桓公曰:``善。''不择日而庙礼太子。

或曰:管仲之射隐,不得也。士之用不在近远,而优俳侏儒固人主之所与燕也,则近优而远士而以为治,非其难者也。夫处世而不能用其有,而悖不去国,是以一人之力禁一国。以一人之力禁一国者,少能胜之。明能照远奸而见隐微,必行之令,虽远于海,内必无变。然则去国之海而不劫杀,非其难者也。楚成王置商臣以为太子,又欲置公子职,商臣作难,遂弑成王。公子宰,周太子也,公子根有宠,遂以东州反,分而为两国。此皆非晚置太子之患也。夫分势不二,庶孽卑,宠无藉,虽处大臣,晚置太子可也。然则晚置太子,庶孽不乱,又非其难也。物之所谓难者,必借人成势而勿侵害己,可谓一难也,贵妾不使二后,二难也。爱孽不使危正适,专听一臣而不敢隅君,此则可谓三难也。

四

叶公子高问政于仲尼,仲尼曰:``政在悦近而来远。''哀公问政于仲尼,仲尼曰:``政在选贤。''齐景公问政于仲尼,仲尼曰:``政在节财。''三公出,子贡问曰:``三公问夫子政一也。夫子对之不同,何也?''仲尼曰:``叶都大而国小,民有背心,故曰`政在悦近而来远'。鲁哀公有大臣三人,外障距诸侯四邻之士,内比周而以愚其君,使宗庙不扫除,社稷不血食者,必是三臣也,故曰`政在选贤'。齐景公筑雍门,为路寝,一朝而以三百乘之家赐者三,故曰`政在节财'。''

或曰:仲尼之对,亡国之言也。恐民有倍心,而诚说之``悦近而来远'',则是教民怀惠。惠之为政,无功者受赏,而有罪者免,此法之所以败也。法败而政乱,以乱政治败民,未见其可也。且民有倍心者,君上之明有所不及也。不绍叶公之明,而使之悦近而来远,是舍吾势之所能禁而使与不行惠以争民,非能持势者也。夫尧之贤,六王之冠也。舜一从而咸包,而尧无天下矣。有人无术以禁下,恃为舜而不失其民,不亦无术乎?明君见小奸于微,故民无大谋;行小诛于细,故民无大乱。此谓``图难于其所易也,为大者于其所细也。''今有功者必赏,赏者不得君,力之所致也;有罪者必诛,诛者不怨上,罪之所生也。民知诛罚之皆起于身也,故疾功利于业,而不受赐于君。``太上,下智有之。''此言太上之下民无说也,安取怀惠之民?上君之民无利害,说以``悦近来远'',亦可舍已。

哀公有臣外障距内比周以愚其君,而说之以``选贤'',此非功伐之论也,选其心之所谓贤者也。使哀公知三子外障距内比周也,则三子不一日立矣。哀公不知选贤,选其心之所谓贤,故三子得任事。燕子哙贤子之而非孙卿,故身死为僇;夫差智太宰嚭而愚子胥,故灭于越。鲁君不必知贤,而说以选贤,是使哀公有夫差、燕哙之患也。明君不自举臣,臣相进也;不自贤,功自徇也。论之于任,试之于事,课之于功,故群臣公政而无私,不隐贤,不进不肖。然则人主奚劳于选贤?

景公以百乘之家赐,而说以``节财'',是使景公无术使智富之侈,而独俭于上,未免于贫也。有君以千里养其口腹,则虽桀、纣不侈焉。齐国方三千里而桓公以其半自养,是侈于桀、纣也;然而能为五霸冠者,知侈俭之地也。为君不能禁下而自禁者谓之劫,不能饰下而自饰者谓之乱,不节下而自节者谓之贫。明君使人无私,以诈而食者禁;力尽于事、归利于上者必闻,闻者必赏;污秽为私者必知,知者必诛。然,故忠臣尽忠于公,民士竭力于家,百官精克于上,侈倍景公,非国之患也。然则说之以节财,非其急者也。

夫对三公一言而三公可以无患,知下之谓也。知下明,则禁于微;禁于微,则奸无积;奸无积,则无比周;无比周,则公私分;分私分,则朋党散;朋党散,则无外障距内比周之患。知下明,则见精沐;见精沐,则诛赏明,诛赏明,则国不贫。故曰:一对而三公无患,知下之谓也。

五

郑子产晨出,过东匠之闾,闻妇人之哭,抚其御之手而听之。有间,遣吏执而问之,则手绞其夫者也。异日,其御问曰:``夫子何以知之?''子产曰:``其声惧。凡人于其亲爱也,始病而忧,临死而惧,已死而哀。今哭已死,不哀而惧,是以知其有奸也。''

或曰:子产之治,不亦多事乎?奸必待耳目之所及而后知之,则郑国之得奸者寡矣。不任典成之吏,不察参伍之政,不明度量,恃尽聪明劳智虑而以知奸,不亦无术乎?且夫物众而智寡,寡不胜众,智不足以遍知物,故则因物以治物。下众而上寡,寡不胜众者,言君不足以遍知臣也,故因人以知人。是以形体不劳而事治,智虑不用而奸得。故宋人语曰:``一雀过羿,必得之,则羿诬矣。以天下为之罗,则雀不失矣。''夫知奸亦有大罗,不失其一而已矣。不修其理,而以己之胸察为之弓矢,则子产诬矣。老子曰:``以智治国,国之贼也。''其子产之谓矣。

六

秦昭王问于左右曰:``今时韩、魏孰与始强?''右左对曰:``弱于始也。''。``今之如耳、魏齐孰与曩之孟常、芒卯?''对曰:``不及也。''王曰:``孟常、芒卯率强韩、魏,犹无奈寡人何也。''左右对曰:``甚然。''中期推琴而对曰:``王之料天下过矣。夫六晋之时,知氏最强,灭范、中行而从韩、魏之兵以伐赵,灌以晋水,城之未沈者三板。知伯出,魏宣子御,韩康子为骖乘。知伯曰:`始吾不知水可以灭人之国,吾乃今知之。汾水可以灌安邑,绛水可以灌平阳。'魏宣子肘韩康子,康子践宣子之足,肘足乎车上,而知氏分于晋阳之下。今足下虽强,未若知氏;韩、魏虽弱,未至如其晋阳之下也。此天下方用肘足之时,愿王勿易之也。''

或曰:昭王之问也有失,左右中期之对也有过。凡明主之治国也,任其势。势不可害,则虽强天下无奈何也,而况孟常、芒卯、韩、魏能奈我何?其势可害也,则不肖如耳、魏齐及韩、魏犹能害之。然则害与不侵,在自恃而已矣,奚问乎?自恃其不可侵,强与弱奚其择焉?失在不自恃,而问其奈何也,其不侵也幸矣。申子曰:``失之数而求之信,则疑矣。''其昭王之谓也。知伯无度,从韩康、魏宣而图以水灌灭其国,此知伯之所以国亡而身死,头为饮杯之故也。今昭王乃问孰与始强,其畏有水人之患乎?虽有左右,非韩、魏之二子也,安有肘足之事?而中期曰``勿易'',此虚言也。且中期之所官,琴瑟也。弦不调,弄不明,中期之任也,此中期所以事昭王者也。中期善承其任,未慊昭王也,而为所不知,岂不妄哉?左右对之曰:``弱于始''与``不及''则可矣,其曰``甚然''则谀也。申子曰:``治不逾官,虽知不言。''今中期不知而尚言之。故曰:昭王之问有失,左右中期之对皆有过也。

七

管子曰:``见其可,说之有证;见其不可,恶之有形。赏罚信于所见,虽所不见,其敢为之乎?见其可,说之无证;见其不可,恶之无形。赏罚不信于所见,而求所不见之外,不可得也。''

或曰:广廷严居,众人之所肃也;宴室独处,曾、史之所僈也。观人之所肃,非行情也。且君上者,臣下之所为饰也。好恶在所见,臣下之饰奸物以愚其君,必也。明不能烛远奸,见隐微,而待之以观饰行,定赏罚,不亦弊乎?

八

管子曰:``言于室,满于室;言于堂,满于堂:是谓天下王。''

或曰:管仲之所谓言室满室、言堂满堂者,非特谓游戏饮食之言也,必谓大物也。人主之大物,非法则术也。法者,编著之图籍,设之于官府,而布之于百姓者也。术者,藏之于胸中,以偶众端而潜御群臣者也。故法莫如显,而术不欲见。是以明主言法,则境内卑贱莫不闻知也,不独满于堂;用术,则亲爱近习莫之得闻也,不得满室。而管子犹曰``言于室,满室,言于堂满堂'',非法术之言也。

\hypertarget{header-n1566}{%
\subsection{难四}\label{header-n1566}}

一

卫孙文子聘于鲁,公登亦登。叔孙穆子趋进曰:``诸侯之会,寡君未尝后卫君也。今子不后寡君一等,寡君未知所过也。子其少安。''孙子无辞,亦无悛容。穆子退而告人曰:``孙子必亡。亡臣而不后君,过而不悛,亡之本也。''

或曰:天子失道,诸侯伐之,故有汤、武。诸侯失道,大夫伐之,故有齐、晋。臣而伐君者必亡,则是汤、武不王,晋、齐不立也。孙子君于卫,而后不臣于鲁,臣之君也。君有失也,故臣有得也。不命亡于有失之君,而命亡于有得之臣,不察。鲁不得诛卫大夫,而卫君之明不知不悛之臣。孙子虽有是二也,臣以亡?其所以亡其失,所以得君也。

或曰:臣主之施,分也。臣能夺君者,以得相也。故非其分而取者,众之所夺也;辞其分而取者,民之所予也。是以桀索岷山之女,纣求比干之心,而天下离;汤身易名,武身受詈,而海内服;赵咺走山,田氏外仆,而齐、晋从。则汤、武之所以王,齐晋之所以立,非必以其君也,彼得之而后以君处之也。今未有其所以得,而行其所以处,是倒义而逆德也。倒义,则事之所以败也;逆德,则怨之所以聚也。败亡之不察,何也?

二

鲁阳虎欲攻三桓,不克而奔齐,景公礼之。鲍文子谏曰:``不可。阳虎有宠于季氏而欲伐于季孙,贪其富也。今君富于季孙,而齐大于鲁,阳虎所以尽诈也。景公乃囚阳虎。

或曰:千金之家,其子不仁,人之急利甚也。桓公,五伯之上也,争国而杀其兄,其利大也。臣主之间,非兄弟之亲也。劫杀之功,制万乘而享大利,则群臣孰非阳虎也?事以微巧成,以疏拙败。群臣之未起难也,其备未具也。群臣皆有阳虎之心,而君上不知,是微而巧也。阳虎贪于天下,以欲攻上,是疏而拙也。不使景公加诛于拙虎,是鲍文子之说反也。臣之忠诈,在君所行也。君明而严,则群臣忠;君懦而暗,则群臣诈。知微之谓明,无救赦之谓严。不知齐之巧臣而诛鲁之成乱,不亦妄乎?

或曰:仁贪不同心。故公子目夷辞宋,而楚商臣弑父;郑去疾予弟,而鲁桓弑兄。五伯兼并,而以桓律人,则是皆无贞廉也。且君明而严,则群臣忠。阳虎为乱于鲁,不成而走,入齐而不诛,是承为乱也。君明则诛,知阳虎之可济乱也,此见微之情也。语曰:``诸侯以国为亲。''君严则阳虎之罪不可失,此无救赦之实也,则诛阳虎,所以使群臣忠也。未知齐之巧臣而废明乱之罚,责于未然而不诛昭昭之罪,此则妄矣。今诛鲁之罪乱以威群臣之有奸心者,而可以得季、孟、叔孙之亲,鲍文之说,何以为反?

三

郑伯将以高渠弥为卿,昭公恶之,固谏不听。及昭公即位,惧其杀己也,辛卯,弑昭公而立子亶也。君子曰:``昭公知所恶矣。''公子圉曰:``高伯其为戮乎,报恶已甚矣。''

或曰:公子圉之言也,不亦反乎?昭公之及于难者,报恶晚也。然则高伯之晚于死者,报恶甚也。明君不悬怒,悬怒,则臣罪轻举以行计,则人主危。故灵台之饮,卫侯怒而不诛,故褚师作难;食鼋之羹,郑君怒而不诛,故子公杀君。君子之举``知所恶'',非甚之也,曰:知之若是其明也,而不行诛焉,以及于死。故``故所恶'',以见其无权也。人君非独不足于见难而已,或不足于断制,今昭公见恶,稽罪而不诛,使渠弥含憎惧死以侥幸,故不免于杀,是昭公之报恶不甚也。

或曰:报恶甚者,大诛报小罪。大诛报小罪也者,狱之至也。狱之患,故非在所以诛也,以仇之众也。是以晋厉公灭三郄而栾、中行作难,郑子都杀伯咺而食鼎起祸,吴王诛子胥而越句践成霸。则卫侯之逐,郑灵之弑,不以褚师之不死而公父之不诛也,以未可以怒而有怒之色,未可诛而有诛之心。怒其当罪,而诛不逆人心,虽悬奚害?夫未立有罪,即位之后,宿罪而诛,齐胡之所以灭也。君行之臣,犹有后患,况为臣而行之君乎?诛既不当,而以尽为心,是与天下有仇也。则虽为戮,不亦可乎!

四

卫灵之时,弥子瑕有宠于卫国。侏儒有见公者曰:``臣之梦浅矣。''公曰:``奚梦?''``梦见灶者,为见公也。''公怒曰:``吾闻人主者梦见日,奚为见寡人而梦见灶乎?''侏儒曰:``夫日兼照天下,一物不能当也。人君兼照一国,一人不能壅也。故将见人主而梦日也。夫灶,一人炀焉,则后人无从见矣。或者一人炀君邪?则臣虽梦灶,不亦可乎?''公曰:``善。''遂去雍鉏,退弥子瑕,而用司空狗。

或曰:侏儒善假于梦以见主道矣,然灵公不知侏儒之言也。去雍鉏,退弥子瑕,而用司空狗者,是去所爱而用所贤也。郑子都贤庆建而壅焉,燕子哙贤子之而壅焉。夫去所爱而用所贤,未免使一人炀己也。不肖者炀主,不足以害明;今不加知而使贤者炀主己,则贤矣。

或曰:屈到嗜芰,文王嗜菖蒲菹,非正味也,而二贤尚之,所味不必美。晋灵侯说参无恤,燕哙贤子之,非正士也,而二君尊之,所贤不必贤也。非贤而贤用之,与爱而用之同。贤诚贤而举之,与用所爱异状。故楚庄举叔孙而霸,商辛用费仲而灭,此皆用所贤而事相反也。燕哙虽举所贤,而同于用所爱,卫奚距然哉?则侏儒之未可见也。君壅而不知其壅也,已见之后而知其壅也,故退壅臣,是加知之也。曰``不加知而使贤者炀己则必危'',而今以加知矣,则虽炀己,必不危矣。

\hypertarget{header-n1583}{%
\subsection{难势}\label{header-n1583}}

慎子曰:飞龙乘云,腾蛇游雾,云罢雾霁,而龙蛇与蚓蚁同矣,则失其所乘也。贤人而诎于不肖者,则权轻位卑也;不肖而能服于贤者,则权重位尊也。尧为匹夫,不能治三人;而桀为天子,能乱天下:吾以此知势位之足恃而贤智之不足慕也。夫弩弱而矢高者,激于风也;身不肖而令行者,得助于众也。尧教于隶属而民不听,至于南面而王天下,令则行,禁则止。则此观之,贤智未足以服众,而势位足以缶贤者也。

应慎子曰:飞龙乘云,腾蛇游雾,吾不以龙蛇为不托于云雾之势也。虽然,夫择贤而专任势,足以为治乎?则吾未得见也。夫有云雾之势而能乘游之者,龙蛇之材美之也;今云盛而蚓弗能乘也,雾而蚁不能游也,夫有盛云雾之势而不能乘游者,蚓蚁之材薄也。今桀、纣南面而王天下,以天子之威为之云雾,而天下不免乎大乱者,桀、纣之材薄也。

且其人以尧之势以治天下也,其势何以异桀之势也,乱天下者也。夫势者,非能必使贤者用已,而不肖者不用已也。贤者用之则天下治,不肖者用之则天下乱。人之情性,贤者寡而不肖者众,而以威势之利济乱世之不肖人,则是以势乱天下者多矣,以势治天下者寡矣。夫势者,便治而利乱者也。故《周书》曰:``毋为虎傅翼,飞入邑,择人而食之。''夫乘不肖人于势,是为虎傅翼也。桀、纣为高台深池以尽民力,为炮烙以伤民性,桀、纣得成肆行者,南面之威为之翼也。使桀、纣为匹夫,未始行一而身在刑戮矣。势者,养虎狼之心而成暴风乱之事者也,此天下之大患也。势之于治乱,本末有位也,而语专言势之足以治天下者,则其智之所至者浅矣。

夫良马固车,使臧获御之则为人笑,王良御之而日取千里。车马非异也,或至乎千里,或为人笑,则巧拙相去远矣。今以国位为车,以势为马,以号令为辔,以刑罚为鞭策,使尧、舜御之则天下治,桀、纣御之则天下乱,则贤不肖相去远矣。夫欲追速致远,不知任王良;欲进利除害,不知任贤能:此则不知类之患也。夫尧舜亦治民之王良也。

复应之曰:其人以势为足恃以治官;客曰``必待贤乃治'',则不然矣。夫势者,名一而变无数者也。势必于自然,则无为言于势矣。吾所为言势者,言人之所设也。夫尧、舜生而在上位,虽有十桀、纣不能乱者,则势治也;桀、纣亦生而在上位,虽有十尧、舜而亦不能治者,则势乱也。故曰:``势治者则不可乱,而势乱者则不可治也。''此自然之势也,非人之所得设也。若吾所言,谓人之所得势也而已矣,贤何事焉?何以明其然也?客曰:``人有鬻矛与盾者,誉其盾之坚,`物莫能陷也',俄而又誉其矛曰:`吾矛之利,物无不陷也。'人应之曰:`以子之矛,陷子之盾,何如?'其人弗能应也。''以为不可陷之盾,与无不陷之矛,为名不可两立也。夫贤之为势不可禁,而势之为道也无不禁,以不可禁之势,此矛盾之说也。夫贤势之不相容亦明矣。

且夫尧、舜、桀、纣千世而一出,是比肩随踵而生也。世之治者不绝于中,吾所以为言势者,中也。中者,上不及尧、舜,而下亦不为桀、纣。抱法处势则治,背法去势则乱。今废势背法而待尧、舜,尧、舜至乃治,是千世乱而一治也。抱法处势而待桀、纣,桀、纣至乃乱,是千世治而一乱也。且夫治千而乱一,与治一而乱千也,是犹乘骥、而分驰也,相去亦远矣。夫弃隐栝之法,去度量之数,使奚仲为车,不能成一轮。无庆赏之劝,刑罚之威,释势委法,尧、舜户说而人辨之,不能治三家。夫势之足用亦明矣,而曰``必待贤'',则亦不然矣。

且夫百日不食以待粱肉,饿者不活;今待尧、舜之贤乃治当世之民,是犹待粱肉而救饿之说也。夫曰:``良马固车,臧获御之则为人笑,王良御之则日取乎千里'',吾不以为然。夫待越人之善海游者以救中国之溺人,越人善游矣,而溺者不济矣。夫待古之王良以驭今之马,亦犹越人救溺之说也,不可亦明矣。夫良马固车,五十里而一置,使中手御之,追速致远,可以及也,而千里可日致也,何必待古之王良乎?且御,非使王良也,则必使臧获败之;治,非使尧、舜也,则必使桀、纣乱之。此味非饴蜜也,必苦莱、亭历也。此则积辩累辞,离理失术,两未之议也,奚可以难夫道理之言乎哉?客议未及此论也。

\hypertarget{header-n1584}{%
\subsection{问辩}\label{header-n1584}}

或问曰:``辩安生乎?''

对曰:``生于上之不明也。''

问者曰:``上之不明因生辩也,何哉?''

对曰:``明主之国,令者,言最贵者也;法者,事最适者也。言无二贵,法不两适,故言行而不轨于法令者必禁。若其无法令而可以接诈、应变、生利、揣事者,上必采其言而责其实。言当,则有大利;不当,则有重罪。是以愚者畏罪而不敢言,智者无以讼。此所以无辩之故也。乱世则不然:主有令,而民以文学非之;官府有法,民以私行矫之。人主顾渐其法令而尊学者之智行,此世之所以多文学也。夫言行者,以功用为之的彀者也。夫砥砺杀矢而以妄发,其端未尝不中秋毫也,然而不可谓善射者,无常仪的也。设五寸之的,引十步之远,非羿、逢蒙不能必中者,有常仪的也。故有常,则羿、逢蒙以五寸的为巧;无常,则以妄发之中秋毫为拙。今听言观行,不以功用为之的彀,言虽至察,行虽至坚,则妄发之说也。是以乱世之听言也,以难知为察,以博文为辩;其观行也,以离群为贤,以犯上为抗。人主者说辩察之言,尊贤抗之行,故夫作法术之人,立取舍之行,别辞争之论,而莫为之正。是以儒服、带剑者众,而耕战之士寡;坚白、无厚之词章,而宪令之法息。故曰:上不明,则辩生焉。

\hypertarget{header-n1585}{%
\subsection{问田}\label{header-n1585}}

徐渠问田鸠曰:``臣闻智士不袭下而遇君,圣人不见功而接上。令阳城义渠,明将也,而措于毛伯;公孙亶回,圣相也,而关于州部,何哉?''田鸠曰:``此无他故异物,主有度、上有术之故也。且足下独不闻楚将宋觚而失其政,魏相冯离而亡其国?二君者驱于声词,眩乎辩说,不试于毛伯,不关乎州部,故有失政亡国之患。由是观之,夫无毛伯之试,州部之关,岂明主之备哉!''

堂谿公谓韩子曰:``臣闻服礼辞让,全之术也;修行退智,遂之道也。今先生立法术,设度数,臣窃以为危于身而殆于躯。何以效之?所闻先生术曰:`楚不用吴起而削乱,秦行商君而富强。二子之言已当矣,然而吴起支解而商君车裂者,不逢世遇主之患也。'逢遇不可必也,患祸不可斥也。夫舍乎全遂之道而肆乎危殆之行,窃为先生无取焉。''韩子曰:``明先生之言矣。夫治天下之柄,齐民萌之度,甚未易处也。然所以废先王之教,而行贱臣之所取者,窍以为立法术,设度数,所以利民萌便众庶之道也。故不惮乱主暗上之患祸,而必思以齐民萌之资利者,仁智之行也。惮乱主暗上之患祸,而避乎死亡之害,知明而不见民萌之资夫科身者,贪鄙之为也。臣不忍向贪鄙之为,不敢伤仁智之行。先王有幸臣之意,然有大伤臣之实。''

\hypertarget{header-n1586}{%
\subsection{定法}\label{header-n1586}}

问者曰:``申不害、公孙鞅,此二家之言孰急于国?''

应之曰:``是不可程也。人不食,十日则死;大寒之隆,不衣亦死。谓之衣食孰急于人,则是不可一无也,皆养生之具也。今申不害言术而公孙鞅为法。术者,因任而授官,循名而责实,操杀生之柄,课群臣之能者也。此人主之所执也。法者,宪令著于官府,刑罚必于民心,赏存乎慎法,而罚加乎奸令者也。此臣之所师也。君无术则弊于上,臣无法则乱于下,此不可一无,皆帝王之具也。''

问者曰:``徒术而无法,徒法而无术,其不可何哉?''

对曰:``申不害,韩昭侯之佐也。韩者,晋之别国也。晋之故法未息,而韩之新法又生;先君之令未收,而后君之令又下。申不害不擅其法,不一其宪令,则奸多。故利在故法前令则道之,利在新法后令则道之,利在故新相反,前后相勃,则申不害虽十使昭侯用术,而奸臣犹有所谲其辞矣。故托万乘之劲韩,七十年而不至于霸王者,虽用术于上,法不勤饰于官之患也。公孙鞅之治秦也,设告相坐而责其实,连什伍而同其罪,赏厚而信,刑重而必。是以其民用力劳而不休,逐敌危而不却,故其国富而兵强;然而无术以知奸,则以其富强也资人臣而已矣。及孝公、商君死,惠王即位,秦法未败也,而张仪以秦殉韩、魏。惠王死,武王即位,甘茂以秦殉周。武王死,昭襄王即位,穰侯越韩、魏而东攻齐,五年而秦不益尺土之地,乃城其陶邑之封。应侯攻韩八年,成其汝南之封。自是以来,诸用秦者,皆应、穰之类也。故战胜,则大臣尊;益地,则私封立:主无术以知奸也。商君虽十饰其法,人臣反用其资。故乘强秦之资数十年而不至于帝王者,法不勤饰于官,主无术于上之患也。''

问者曰:``主用申子之术,而官行商君之法,可乎?''

对曰:``申子未尽于法也。申子言:`治不逾官,虽知弗言'。治不逾官,谓之守职也可;知而弗言,是不谓过也。人主以一国目视,故视莫明焉;以一国耳听,故听莫聪焉。今知而弗言,则人主尚安假借矣?商君之法曰:`斩一首者爵一级,欲为官者为五十石之官;斩二首者爵二级,欲为官者为百石之官。'官爵之迁与斩首之功相称也。今有法曰:`斩首者令为医、匠。'则屋不成而病不已。夫匠者手巧也,而医者齐药也,而以斩首之功为之,则不当其能。今治官者,智能也;今斩首者,勇力之所加也。以勇力之所加而治者智能之官,是以斩首之功为医、匠也。故曰:二子之于法术,皆未尽善也。''

\hypertarget{header-n1587}{%
\subsection{说疑}\label{header-n1587}}

凡治之大者,非谓其赏罚之当也。赏无功之人,罚不辜之民,非谓明也。赏有功,罚有罪,而不失其人,方在于人者也,非能生功止过者也。是故禁奸之法,太上禁其心,其次禁其言,其次禁其事。今世皆曰:``尊主安国者,必以仁义智能'',而不知卑主危国者之必以仁义智能也。故有道之主,远仁义,去智能,服之以法。是以誉广而名威,民治而国安,知用民之法也。凡术也者,主之所以执也;法也者,官之所以师也。然使郎中日闻道于郎门之外,以至于境内日见法,又非其难者也。

昔者有扈氏有失度,讙兜氏有孤男,三苗有成驹,桀有侯侈,纣有崇侯虎,晋有优施,此六人者,亡国之臣也。言是如非,言非如是,内险以贼,其外小谨,以征其善;称道往古,使良事沮;善禅其主,以集精微,乱之以其所好:此夫郎中左右之类者也。往世之主,有得人而身安国存者,

有得人而身危国亡者。得人之名一也,而利害相千万也,故人主左右不可不慎也。为人主者诚明于臣之所言,则别贤不肖如黑白矣。

若夫许由、续牙、晋伯阳、秦颠颉、卫侨如、狐不稽、重明、董不识、卞随、务光、伯夷、叔齐,此十二者,皆上见利不喜,下临难不恐,或与之天下而不取,有萃辱之名,则不乐食谷之利。夫见利不喜,上虽厚赏,无以劝之;临难不恐,上虽严刑,无以威之:此之谓不令之民也。此十二人者,或伏死于窟穴,或槁死于草木,或饥饿于山谷,或沉溺于水泉。有如此,先古圣王皆不能臣,当今之世,将安用之?

若夫关龙逄、王子比干、随季梁、陈泄冶、楚申胥、吴子胥,此六人者,皆疾争强谏以胜其君。言听事行,则如师徒之势;一言而不听,一事则不行,则陵其主以语,待之以其身,虽死家破,要领不属,手足异处,不难为也。如此臣者,先古圣王皆不能忍也,当今之时,将安用之?

若夫齐田恒、宋子罕、鲁季孙意如、晋侨如、卫子南劲、郑太宰欣、楚白公、周单荼、燕子之,此九人者之为其臣也,皆朋党比周以事其君,隐正道而行私曲,上逼君,下乱治,援外以挠内,亲下以谋上,不难为也。如此臣者,唯圣王智主能禁之,若夫昏乱之君,能见之乎?

若夫后稷、皋陶、伊尹、周公旦、太公望、管仲、隰朋、百里奚、蹇叔、舅犯、赵襄、范蠡、大夫种、逢同、华登,此十五人者为其臣也,皆夙兴夜寐,单身贱体,竦心白意;明刑辟、治官职以事其君,进善言、通道法而不敢矜其善,有成功立事而不敢伐其劳;不难破家以便国,杀身以安主,以其主为高天泰山之尊,而以其身为壑谷洧之卑;主有明名广誉于国,而身不难受壑谷洧之卑。如此臣者,虽当昏乱之主尚可致功,况于显明之主乎?此谓霸王之佐也。

若夫周滑之、郑王孙申、陈公孙宁、仪行父、荆芋尹申亥、随少师、越种干、吴王孙、晋阳成泄、齐竖刁、易牙,此十二人者之为其臣也,皆思小利而忘法义,进则掩蔽贤良以阴暗其主,退则挠乱百官而为祸难;皆辅其君,共其欲,苟得一说于主,虽破国杀众,不难为也。有臣如此,虽当圣王尚恐夺之,而况昏乱之君,其能无失乎?有臣如此者,皆身死国亡,为天下笑。故周威公身杀,国分为二;郑子阳身杀,国分为三;陈灵身死于夏征舒氏;荆灵王死于乾谿之上;随亡于荆;吴并于越;知伯灭于晋阳之下;桓公身死七日不收。故曰:謟谀之臣,唯圣王知之,而乱主近之,故至身死国亡。

圣王明君则不然,内举不避亲,外举不避仇。是在焉,从而举之;非在焉,从而罚之。是以贤良遂进而奸邪并退,故一举而能服诸侯。其在记曰:尧有丹朱,而舜有商均,启有五观,商有太甲,武王有管、蔡。五王之所诛者,皆父兄子弟之亲也,而所杀亡其身残破其家者何也?以其害国伤民败法类也。观其所举,或在山林薮泽岩穴之间,或在囹圄緤绁缠索之中,或在割烹刍牧饭牛之事。然明主不羞其卑贱也,以其能,为可以明法,便国利民,从而举之,身安名尊。

乱主则不然,不知其臣之意行,而任之以国,故小之名卑地削,大之国亡身死。不明于用臣也。无数以度其臣者,必以其众人之口断之。众人所誉,从而悦之;众之所非,从而憎之。故为人臣者破家残賥,内构党与、外接巷族以为誉,从阴约结以相固也,虚相与爵禄以相劝也。曰:``与我者将利之,不与我者将害之。''众贪其利,劫其威:``彼诚喜,则能利己;忌怒,则能害己。''众归而民留之,以誉盈于国,发闻于主。主不能理其情,因以为贤。彼又使谲诈之士,外假为诸侯之宠使,假之以舆马,信之以瑞节,镇之以辞令,资之以币帛,使诸侯淫说其主,微挟私而公议。所为使者,异国之主也;所为谈者,左右之人也。主说其言而辩其辞,以此人者天下之贤士也。内外之于左右,其讽一而语同。大者不难卑身尊位以下之,小者高爵重禄以利之。夫奸人之爵禄重而党与弥众,又有奸邪之意,则奸臣愈反而说之,曰:``古之所谓圣君明王者,非长幼弱也,及以次序也;以其构党与,聚巷族,逼上弑君而求其利也。''彼曰:``何知其然也?''因曰:``舜逼尧,禹逼舜,汤放桀,武王伐纣。此四王者,人臣弑其君者也,而天下誉之。察四王之情,贪得人之意也;度其行,暴乱之兵也。然四王自广措也,而天下称大焉;自显名也,而天下称明焉。则威足以临天下,利足以盖世,天下从之。''又曰:``以今时之所闻,田成子取齐,司城子罕取宋,太宰欣取郑,单氏取周,易牙之取卫,韩、魏、赵三子分晋,此六人者,臣之弑其君者也。''奸臣闻此,然举耳以为是也。故内构党与,外摅巷族,观时发事,一举而取国家。且夫内以党与劫弑其君,外以诸侯之讙骄易其国,隐敦适,持私曲,上禁君,下挠治者,不可胜数也。是何也?则不明于择臣也。记曰:``周宣王以来,亡国数十,其臣弑其君取国者众矣。''然则难之从内起与从外作者相半也。能一尽其民力,破国杀身者,尚皆贤主也。若夫转身法易位,全众傅国,最其病也。

为人主者,诚明于臣之所言,则虽弋驰骋,撞钟舞女,国犹且存也;不明臣之所言,虽节俭勤劳,布衣恶食,国犹自亡也。赵之先君敬侯,不修德行,而好纵欲,适身体之所安,耳目之所乐,冬日弋,夏浮淫,为长夜,数日不废御觞,不能饮者以筒灌其口,进退不肃、应对不恭者斩于前。故居处饮食如此其不节也。制刑杀戮如此其无度也,然敬侯享国数十年,兵不顿于敌国,地不亏于四邻,内无君臣百官之乱,外无诸侯邻国之患,明于所以任臣也。燕君子哙,邵公之后也,地方数千里,持戟数十万,不安子女之乐,不听钟石之声,内不堙污池台榭,外不弋田猎,又亲操耒耨以修畎亩。子哙之苦身以忧民如此其甚也,虽古之所谓圣王明君者,其勤身而忧世不甚于此矣。然而子哙身死国亡,夺于子之,而天下笑之。此其何故也?不明乎所以任臣也。

故曰:人臣有五奸,而主不知也。为人主者,有侈用财货赂以取誉者,有务庆赏赐予以移众者,有务朋党徇智尊士以擅逞者,有务解免赦罪狱以事威者,有务奉下直曲、怪言、伟服、瑰称以眩民耳目者。此五者,明君之所疑也,而圣主之所禁也。去此五者,则譟诈之人不敢北面谈立;文言多、实行寡而不当法者,不敢诬情以谈说。是以群臣居则修身,动则任力,非上之令不敢擅作疾言诬事,此圣王之所以牧臣下也。彼圣主明君,不适疑物以窥其臣也。见疑物而无反者,天下鲜矣。故曰:孽有拟适之子,配有拟妻之妾,廷有拟相之臣,臣有拟主之宠,此四者,国之所危也。故曰:内宠并后,外宠贰政,枝子配适,大臣拟主,乱之道也。故《周记》曰:``无尊妾而卑妻,无孽适子而尊小枝,无尊嬖臣而匹上卿,无尊大臣以拟其主也。''四拟者破,则上无意、下无怪也;四拟不破,则陨身灭国矣。

\hypertarget{header-n1588}{%
\subsection{诡使}\label{header-n1588}}

圣人之所以为治道者三:一曰``利'',二曰``威'',三曰``名''。夫利者,所以得民也;威者,所以行令也;名者,上下之所同道也。非此三者,虽有不急矣。今利非无有也,而民不化上;威非不存也,而下不听从;官非无法也,而治不当名。三者非不存也,而世一治一乱者,何也?夫上之所贵与其所以为治相反也。

夫立名号,所以为尊也;今有贱名轻实者,世谓``高''。设爵位,所以为贱贵基也;而简上不求见者,谓之``贤''。威利,所以行令也;而无利轻威者,世谓之``重''。法令,所以为治也;而不从法令为私善者,世谓之``忠''。官爵,所以劝民也;而好名义不进仕者,世谓之``烈士''。刑罚,所以擅威也;而轻法不避刑戮死亡之罪者,世谓之``勇夫''。民之急名也,甚其求利也;如此,则士之饥饿乏绝者,焉得无岩居苦身以争名于天下哉?故世之所以不治者,非下之罪,上失其道也。常贵其所以乱,而贱其所以治,是故下之所欲,常与上之所以为治相诡也。

今下而听其上,上之所争也。而惇悫纯信,用心怯言,则谓之``窭''。守法固,听令审,则谓之``愚''。敬上畏罪,则谓之``怯''。言时节,行中适,则谓之``不肖''。无二心私学吏,听吏从教者,则谓之``陋''。

难致,谓之``正''。难予,谓之``廉''。难禁,谓之``齐''。有令不听从,谓之``勇''。无利于上,谓之``愿''。少欲、宽惠、行德,谓之``仁''。重厚自尊,谓之``长者''。私学成群,谓之``师徒''。闲静安居,谓之``有思''。损仁逐利,谓之``疾''。险躁佻反覆,谓之``智''。先为人而后自为,类名号,言泛爱天下,谓之``圣''。言大本,称而不可用,行而乘于世者,谓之``大人''。贱爵禄,不挠上者,谓之``杰''。下渐行如此,入则乱民,出则不便也。上宜禁其欲,灭其迹,而不止也,又从而尊之,是教下乱上以为治也。

凡所治者,刑罚也;今有私行义者尊。社稷之所以立者,安静也;而躁险谗谀者任。四封之内所以听从者,信与德也;而陂知倾覆者使。令之所以行,威之所以立者,恭俭听上;而岩居非世者显。仓廪之所以实者,耕农之本务也;而綦组、锦绣、刻画为末作者富。名之所以成,城池之所以广者,战士也;今死士之孤饥饿乞于道,而优笑酒徒之属乘车衣丝。赏禄,所以尽民力易下死也;今战胜攻取之士劳而赏不霑,而卜筮、视手理、狐虫为顺辞于前者日赐。上握度量,所以擅生杀之柄也;今守度奉量之士欲以忠婴上而不得见,巧言利辞行奸轨以幸偷世者数御。据法直言,名刑相当,循绳墨,诛奸人,所以为上治也,而愈疏远;謟施顺意从欲以危世者近习。悉租税,专民力,所以备难充仓府也,而士卒之逃事伏匿、附托有威之门以避徭赋而上不得者万数。夫陈善田利宅,所以战士卒也,而断头裂腹、播骨乎平原野者,无宅容身,身死田亩;而女妹有色,大臣左右无功者,择宅而受,择田而食。赏利一从上出,所善制下也;而战介之士不得职,而闲官之士尊显。上以此为教,名安得无卑,位安得无危?夫卑名位者,必下之不从法令、有二心无私学反逆世者也;而不禁其行、不破其群以散其党,又从而尊之,用事者过矣。上世之所以立廉耻者,所以属下也;今士大夫不羞污泥丑辱而宦,女妹私义之门不待次而宦。赏赐之,所以为重也;而战斗有功之士贫贱,而便辟优徒超级。名号诚信,所以通威也;而主掩障,近习女谒并行,百官主爵迁人,用事者过矣。大臣官人,与下先谋比周,虽不法行,威利在下,则主卑而大臣重矣。

夫立法令者,以废私也。法令行而私道废矣。私者,所以乱法也。而士有二心私学、岩居
路、托伏深虑,大者非世,细者惑下;上不禁,又从而尊之以名,化之以实,是无功而显,无劳而富也。如此,则士之有二心私学者,焉得无深虑、勉知诈与诽谤法令,以求索与世相反者也?凡乱上反世者,常士有二心私学者也。故《本言》曰:``所以治者,法也;所以乱者,私也。法立,则莫得为私矣。''故曰:道私者乱,道法者治。上无其道,则智者有私词,贤者有私意。上有私惠,下有私欲,圣智成群,造言作辞,以非法措于上。上不禁塞,又从而尊之,是教下不听上、不从法也。是以贤者显名而居,奸人赖赏而富。贤者显名而居,奸人赖赏而富,是以上不胜下也。

\hypertarget{header-n1589}{%
\subsection{六反}\label{header-n1589}}

畏死远难,降北之民也,而世尊之曰``贵生之士''。学道立方,离法之民也,而世尊之曰``文学之士''游居厚养,牟食之民也,而世尊之曰``有能之士''。语曲牟知,伪诈之民也。而世尊之曰``辩智之士''。行剑攻杀,暴憿之民也,而世尊之曰``磏勇之士''。活贼匿奸,当死之民也,而世尊之曰``任誉之士''。此六民者,世之所誉也。赴险殉诚,死节之民,而世少之曰``失计之民''也。寡闻从令,全法之民也,而世少之曰``朴陋之民''也。力作而食,生利之民也,而世少之曰``寡能之民''也,嘉厚纯粹,整谷之民也,而世少之曰``愚戆之民''也。重命畏事,尊上之民也,而世少之曰``怯慑之民''也。挫贼遏奸,明上之民也,而世少之曰``謟谗之民''也。此六民者,世之所毁也。奸伪无益之民六,而世誉之如彼;耕战有益之民六,而世毁之如此:此之谓``六反''。布衣循私利而誉之,世主听虚声而礼之,礼之所在,利必加焉。百姓循私害而訾之,世主壅于俗而贱之,贱之所在,害必加焉。故名赏在乎私恶当罪之民,而毁害在乎公善宜赏之士,索国之富强,不可得也。

古者有谚曰:``为政犹沐也,虽有弃发,必为之。''爱弃发之费而忘长发之利,不知权者也。夫弹痤者痛,饮药者苦,为苦惫之故不弹痤饮药,则身不活,病不已矣。今上下之接,无子父之泽,而欲以行义禁下,则交必有郄矣。且父母之于子也,产男则相贺,产女则杀之。此俱出父母之怀衽,然男子受贺,女子杀之者,虑其后便,计之长利也。故父母之于子也,犹用计算之心以相待也,而况无父子之泽乎?今学者之说人主也,皆去求利之心,出相爱之道,是求人主之过父母之亲也,此不熟于论恩,诈而诬也,故明主不受也。圣人之治也,审于法禁,法禁明著,则官法;必于赏罚,赏罚不阿,则民用。官治则国富,国富则兵强,而霸王之业成矣。霸王者,人主之大利也。人主挟大利以听治,故其任官者当能,其赏罚无私。使士民明焉,尽力致死,则功伐可立而爵禄可致,爵禄致而富贵之业成矣。富贵者,人臣之大利也。人臣挟大利以从事,故其行危至死,其力尽而不望。此谓君不仁,臣不忠,则不可以霸王矣。

夫奸必知则备,必诛则止;不知则肆,不诛则行。夫陈轻货于幽隐,虽曾、史可疑也;悬百金于市,虽大盗不取也。不知,则曾、史可疑于幽隐;必知,则大盗不取悬金于市。故明主之治国也,众其守而重其罪,使民以法禁而不以廉止。母之爱子也倍父,父令之行于子者十母;吏之于民无爱,令之行于民也万父。母积爱而令穷,吏威严而民听从,严爱之策亦可决矣。且父母之所以求于子也,动作则欲其安利也,行身则欲其远罪也。君上之于民也,有难则用其死,安平则尽其力。亲以厚爱关子于安利而不听,君以无爱利求民之死力而令行。明主知之,故不养恩爱之心而增威严之势。故母厚爱处,子多败,推爱也;父薄爱教笞,子多善,用严也。

今家人之治产也,相忍以饥寒,相强以劳苦,虽犯军旅之难,饥馑之患,温衣美食者,必是家也;相怜以衣食,相惠以佚乐,天饥岁荒,嫁妻卖子者,必是家也。故法之为道,前苦而长利;仁之为道,偷乐而后穷。圣人权其轻重,出其大利,故用法之相忍,而弃仁人之相怜也。学者之言皆曰``轻刑'',此乱亡之术也。凡赏罚之必者,劝禁也。赏厚,则所欲之得也疾;罚重,则所恶之禁也急。夫欲利者必恶害,害者,利之反也。反于所欲,焉得无恶?欲治者必恶乱,乱者,治之反也。是故欲治甚者,其赏必厚矣;其恶乱甚者,其罚必重矣。今取于轻刑者,其恶乱不甚也,其欲治又不甚也。此非特无术也,又乃无行。是故决贤、不肖、愚、知之美,在赏罚之轻重。且夫重刑者,非为罪人也。明主之法,揆也。治贼,非治所揆也;所揆也者,是治死人也。刑盗,非治所刑也;治所刑也者,是治胥靡也。故曰:重一奸之罪而止境内之邪,此所以为治也。重罚者,盗贼也;而悼惧者,良民也。欲治者奚疑于重刑名!若夫厚赏者,非独赏功也,又劝一国。受赏者甘利,未赏者慕业,是报一人之功而劝境内之众也,欲治者何疑于厚赏!今不知治者皆曰:``重刑伤民,轻刑可以止奸,何必于重哉?''此不察于治者也。夫以重止者,未必以轻止也;以轻止者,必以重止矣。是以上设重刑者而奸尽止,奸尽止,则此奚伤于民也?所谓重刑者,奸之所利者细,而上之所加焉者大也。民不以小利加大罪,故奸必止者也。所谓轻刑者,奸之所利者大,上之所加焉者小也。民慕其利而傲其罪,故奸不止也。故先圣有谚曰:``不踬于山,而踬于垤。''山者大,故人顺之;垤微小,故人易之也。今轻刑罚,民必易之。犯而不诛,是驱国而弃之也;犯而诛之,是为民设陷也。是故轻罪者,民之垤也。是以轻罪之为民道也,非乱国也,则设民陷也,此则可谓伤民矣!

今学者皆道书策之颂语,不察当世之实事,曰:``上不爱民,赋敛常重,则用不足而下恐上,故天下大乱。''此以为足其财用以加爱焉,虽轻刑罚,可以治也。此言不然矣。凡人之取重赏罚,固已足之之后也;虽财用足而后厚爱之,然而轻刑,犹之乱也。夫当家之爱子,财货足用,货财足用则轻用,轻用则侈泰。亲爱之则不忍,不忍则骄恣。侈泰则家贫,骄恣则行暴。此虽财用足而爱厚,轻利之患也。凡人之生也,财用足则隳于用力,上懦则肆于为非。财用足而力作者,神农也;上治懦而行修者,曾、史也,夫民之不及神农、曾、史亦明矣。老聃有言曰:``知足不辱,知止不殆。''夫以殆辱之故而不求于足之外者,老聃也。今以为足民而可以治,是以民为皆如老聃。故桀贵在天子而不足于尊,富有四海之内而不足于宝。君人者虽足民,不能足使为君天子,而桀未必为天子为足也,则虽足民,何可以为治也?故明主之治国也,适其时事以致财物,论其税赋以均贫富,厚其爵禄以尽贤能,重其刑罚以禁奸邪,使民以力得富,以事致贵,以过受罪,以功致赏,而不念慈惠之赐,此帝王之政也。

人皆寐,则盲者不知;皆嘿,则喑者不知。觉而使之视,问而使之对,则喑盲者穷矣。不听其言也,则无术者不知;不任其身也,则不肖者不知。听其言而求其当,任其身而责其功,则无术不肖者穷矣。夫欲得力士而听其自言,虽庸人与乌获不可别也;授之以鼎俎,则罢健效矣。故官职者,能士之鼎俎也,任之以事而愚智分矣。故无术者得于不用,不肖者得于不任。言不用而自文以为辩,身不任者而自饰以为高。世主眩其辩、滥其高而尊贵之,是不须视而定明也,不待对而定辩也,喑盲者不得矣。明主听其言必责其用,观其行必求其功,然则虚旧之学不谈,矜诬之行不饰矣。

\hypertarget{header-n1590}{%
\subsection{八说}\label{header-n1590}}

为故人行私谓之``不弃'',以公财分施谓之``仁人'',轻禄重身谓之``君子'',枉法曲亲谓之``有行'',弃官宠交谓之``有侠'',离世遁上谓之``高傲'',交争逆令谓之``刚材'',行惠取众谓之``得民''。不弃者,吏有奸也;仁人者,公财损也;君子者,民难使也;有行者,法制毁也;有侠者,官职旷也;高傲者,民不事也;刚材者,令不行也;得民者,君上孤也。此八者,匹夫之私誉,人主之大败也。反此八者,匹夫之私毁,人主之公利也。人主不察社稷之利害,而用匹夫之私毁,索国之无危乱,不可得矣。

任人以事,存亡治乱之机也,无术以任人,无所任而不败。人君之所任,非辩智则修洁也。任人者,使有势也。智士者未必信也,为多其智,因惑其信也。以智士之计,处乘势之资而为其私急,则君必欺焉。为智者之不可信也,故任修士者,使断事也。修士者未必智,为洁其身、因惑其智。以愚人之所惽,处治事之官而为所然,则事必乱矣。故无术以用人,任智则君欺,任修则君事乱,此无术之患也。明君之道,贱德义贵,下必坐上,决诚以参,听无门户,故智者不得诈欺。计功而行赏,程能而授事,察端而观失,有过者罪,有能者得,故愚者不任事。智者不敢欺,愚者不得断,则事无失矣。

察士然后能知之,不可以为令,夫民不尽察。贤者然后行之,不可以为法,夫民不尽贤。杨朱、墨崔,天下之所察也,干世乱而卒不决,虽察而不可以为官职之令。鲍焦、华角,天下之所贤也,鲍焦木枯,华角赴河,虽贤不可以为耕战之士。故人主之察,智士尽其辩焉;人主之所尊,能士能尽其行焉。今世主察无用之辩,尊远功之行,索国之富强,不可得也。博习辩智如孔、墨,孔、墨不耕耨,则国何得焉?修孝寡欲如曾、史,曾、史不战攻,则国何利焉?匹夫有私便,人主有公利。不作而养足,不仕而名显,此私便也;息文学而明法度,塞私便而一功劳,此公利也。错法以道民也,而又贵文学,则民之所师法也疑;赏功以劝民也,而又尊行修,则民之产利也惰。夫贵文学以疑法,尊行修以贰功,索国之富强,不可得也。

搢
笏干戚,不适有方铁铦;登降周旋,不逮日中奏百;《狸首》射侯,不当强弩趋发;干城距衡冲,不若堙穴伏橐。古人亟于德,中世逐于智,当今争于力。古者寡事而备简,朴陋而不尽,故有珧铫而推车者。古者人寡而相亲,物多而轻利易让,故有揖让而传天下者。然则行揖让,高慈惠,而道仁厚,皆推政也。处多事之时,用寡事之器,非智者之备也;当大争之世,而循揖让之轨,非圣人之治也。故智者不乘推车,圣人不行推政也。

法所以制事,事所以名功也。法有立而有难,权其难而事成,则立之;事成而有害,权其害而功多,则为之。无难之法,无害之功,天下无有也。是以拔千丈之都,败十万之众,死伤者军之乘,甲兵折挫,士卒死伤,而贺战胜得地者,出其小害计其大利也。夫沐者有弃发,除者伤血肉。为人见其难,因释其业,是无术之事也。先圣有言曰:``规有摩而水有波,我欲更之,无奈之何!''此通权之言也。是以说有必立而旷于实者,言有辞拙而急于用者。故圣人不求无害之言,而务无易之事。人之不事衡石者,非贞廉而远利也,石不能为人多少,衡不能为人轻重,求索不能得,故人不事也。明主之国,官不敢枉法,吏不敢为私利,货赂不行,是境内之事尽如衡石也。此其臣有奸者必知,知者必诛。是以有道之主,不求清洁之吏,而务必知之术也。

慈母之于弱子也,爱不可为前。然而弱子有僻行,使之随师;有恶病,使之事医。不随师则陷于刑,不事医则疑于死。慈母虽爱,无益于振刑救死,则存子者非爱也。子母之性,爱也;臣主之权,策也。母不能以爱存家,君安能以爱持国?明主者通于富强,则可以得欲矣。故谨于听治,富强之法也。明其法禁,察其谋计。法明则内无变乱之患,计得于外无死虏之祸。故存国者,非仁义也。仁者,慈惠而轻财者也;暴者,心毅而易诛者也。慈惠,则不忍;轻财,则好与。心毅,则憎心见于下;易诛,则妄杀加于人。不忍,则罚多宥赦;好与,则赏多无功。憎心见,则下怨其上;妄诛,则民将背叛。故仁人在位,下肆而轻犯禁法,偷幸而望于上;暴人在位,则法令妄而臣主乖,民怨而乱心生。故曰:仁暴者,皆亡国者也。

不能具美食而劝饿人饭,不为能活饿者也;不能辟草生粟而劝贷施赏赐,不能为富民者也。今学者之言也,不务本作而好末事,知道虚圣以说民,此劝饭之说。劝饭之说,明主不受也。

书约而弟子辩,法省而民讼简,是以圣人之书必著论,明主之法必详尽事。尽思虑,揣得失,智者之所难也;无思无虑,挈前言而责后功,愚者之所易也。明主虑愚者之所易,以责智者之所难,故智虑力劳不用而国治也。

酸甘咸淡,不以口断而决于宰尹,则厨人轻君而重于宰尹矣。上下清浊,不以耳断而决于乐正,则瞽工轻君而重于乐正矣。治国是非,不以术断而决于宠人,则臣下轻君而重于宠人矣。人主不亲观听,而制断在下,托食于国者也。

使人不衣不食而不饥不寒,又不恶死,则无事上之意。意欲不宰于君,则不可使也。今生杀之柄在大臣,而主令得行者,未尝有也。虎豹必不用其爪牙而与鼷鼠同威,万金之家必不用其富厚而与监门同资。有土之君,说人不能利,恶人不能害,索人欲畏重己,不可得也。

人臣肆意陈欲曰``侠'',人主肆意陈欲曰``乱'';人臣轻上曰``骄'',人主轻下曰``暴''。行理同实,下以受誉,上以得非。人臣大得,人主大亡。

明主之国,有贵臣,无重臣。贵臣者,爵尊而官大也;重臣者,言听而力多者也。明主之国,迁官袭级,官爵受功,故有贵臣。言不度行而有伪,必诛,故无重臣也。

\hypertarget{header-n1591}{%
\subsection{八经}\label{header-n1591}}

一、因情

凡治天下,必因人情。人情者,有好恶,故赏罚可用;赏罚可用,则禁令可立而治道具矣。君执柄以处势,故令行禁止。柄者,杀生之制也;势者,胜众之资也。废置无度则权渎,赏罚下共则威分。是以明主不怀爱而听,不留说而计。故听言不参,则权分乎奸;智力不用,则君穷乎臣。故明主之行制也天,其用人也鬼。天则不非,鬼则不困。势行教严,逆而不违,毁誉一行而不议。故赏贤罚暴,誉善之至者也;赏暴罚贤,举恶之至者也:是谓赏同罚异。赏莫如厚,使民利之;誉莫如美,使民荣之;诛莫如重,使民畏之;毁莫如恶,使民耻之。然后一行其法,禁诛于私家,不害功罪。赏罚必知之,知之,道尽矣。

二、主道

力不敌众,智不尽物。与其用一人,不如用一国,故智力敌而群物胜。揣中则私劳,不中则任过。下君尽己之能,中君尽人之力,上君尽人之智。是以事至而结智,一听而公会。听不一则后悖于前,后悖于前则愚智不分;不公会则犹豫而不断,不断则事留。自取一,则毋道堕壑之累。故使之讽,讽定而怒。是以言陈之曰,必有策籍。结智者事发而验,结能者功见而谋成败。成败有征,赏罚随之。事成则君收其功,规败则臣任其罪。君人者合符犹不亲,而况于力乎?事智犹不亲,而况于悬乎?故非用人也不取同,同则君怒。使人相用则君神,则下尽。下尽下,则臣上不因君,而主道毕矣。

三、起乱

知臣主之异利者王,以为同者劫,与共事者杀。故明主审公私之分,审利害之地,奸乃无所乘。乱之所生六也:主母,后姬,子姓,弟兄,大臣,显贤。任吏责臣,主母不放;礼施异等,后姬不疑;分势不贰,庶适不争;权籍不失,史弟不侵;下不一门,大臣有拥;禁赏必行,显贤不乱。臣有二因,谓外内也。外曰畏,内曰爱。所畏之求得,所爱之言听,此乱臣之所因也。外国之置诸吏者,结诛亲暱重帑,则外不籍矣;爵禄循功,请者俱罪,则内不因矣。外不籍,内不因,则奸充塞矣。官袭节而进,以至大任,智也。其位至而任大者,以三节持之:曰质,曰镇,曰固。亲戚妻子,质也;爵禄厚而必,镇也;参伍责怒,固也。贤者止于质,贪饕化于镇,奸邪穷于固。忍不制则下上,小不除则大诛,而名实当则径之。生害事,死伤名,则行饮食;不然,而与其仇:此谓除阴奸也。医曰诡,诡曰易。易功而赏,见罪而罚,而诡乃止。是非不泄,说谏不通,而易乃不用。父兄贤良播出曰游祸,其患邻敌多资。僇辱之人近习曰狎贼,其患发忿疑辱之心生。藏怒持罪而不发曰增乱,其患侥幸妄举之人起。大臣两重提衡而不踦曰卷祸,其患家隆劫杀之难作。脱易不自神曰弹威,其患贼夫酖毒之乱起。此五患者,人主之不知,是有劫杀之事。废置之事,生于内则治,生于外则乱。是以明主以功论之内,而以利资之外,其故国治而敌乱。即乱之道:臣憎,则起外若眩;臣爱,则起内若药。

四、立道

参伍之道:行参以谋多,揆伍以责失。行参必拆,揆伍必怒。不拆则渎上,不怒则相和。拆之征足以知多寡,怒之前不及其众。观听之势,其征在比周而赏异也,诛毋谒而罪同。言会众端,必揆之以地,谋之以天,验之以物,参之以人。四征者符,乃可以观矣。参言以知其诚,易视以改其泽,执见以得非常。一用以务近习,重言以惧远使。举往以悉其前,即迩以知其内,疏置以知其外。握明以问所暗,诡使以绝黩泄。倒言以尝所疑,论反以得阴奸。设谏以纲独为,举错以观奸动。明说以诱避过,卑适以观直謟。宣闻以通未见,作斗以散朋党。深一以警众心,泄异以易其虑。似类则合其参,陈过则明其固。知辟罪以止威,阴使时循以省衷。渐更以离通比。下约以侵其上:相室,约其廷臣;廷臣,约其官属;兵士,约其军吏;遣使,约其行介;县令,约其辟吏;郎中,约其左右;后姬,约其宫媛。此之谓条达之道。言通事泄,则术不行。

五、类柄

明主,其务在周密。是以喜见则德偿,怒见则威分。故明主之言隔塞而不通,周密而不见。故以一得十者,下道也;以十得一者,上道也。明主兼行上下,故奸无所失。伍、官、连、县而邻,谒过赏,失过诛。上之于下,下之于上,亦然。是故上下贵贱相畏以法,相诲以和。民之性,有生之实,有生之名。为君者有贤知之名,有赏罚之实。名实俱至,故福善必闻矣。

六、参言

听不参,则无以责下;言不督乎用,则邪说当上。言之为物也以多信,不然之物,十从云疑,百人然乎,千人不可解也。呐者言之疑,辩者言之信。奸之食上也,取资乎众,籍信乎辩,而以类饰其私。人主不餍忿而待合参,其势资下也。有道之主听言,督其用,课其功,功课而赏罚生焉,故无用之辩不留朝。任事者知不足以治职,则放官收。说大而夸则穷端,故奸得而怒。无故而不当为诬,诬而罪臣。言必有报,说必责用也,故朋党之言不上闻。凡听之道,人臣忠论以闻奸,博论以内一,人主不智则奸得资。明主之道,己喜,则求其所纳;己怒,则察其所构;论于已变之后,以得毁誉公私之征。众谏以效智故,使君自取一以避罪,故众之谏也败。君之取也,无副言于上以设将然,今符言于后以知谩诚语。明主之道,臣不得两谏,必任其一语;不得擅行,必合其参,故奸无道进矣。

七、听法

官之重也,毋法也;法之息也,上暗也。上暗无度,则官擅为;官擅为,故奉重无前;则征多;征多故富。官之富重也,乱功之所生也。明主之道取于任,贤于官,赏于功。言程,主喜,俱必利;不当,主怒,俱必害;则人不私父兄而进其仇雠。势足以行法,奉足以给事,而私无所生,故民劳苦而轻官。任事者毋重,使其宠必在爵;外官者毋私,使其利必在禄;故民尊爵而重禄。爵禄,所以赏也;民重所以赏也,则国治。刑之烦也,名之缪也,赏誉不当则民疑,民之重名与其重赏也均。赏者有诽焉,不足以劝;罚者有誉焉,不足以禁。明主之道,赏必出乎公利,名必在乎为上。赏誉同轨,非诛俱行。然则民无荣于赏之内。有重罚者必有恶名,故民畏。罚,所以禁也;民畏所以禁,则国治矣。

八、主威

行义示则主威分,慈仁听则法制毁。民以制畏上,而上以势卑下,故下肆很触而荣于轻君之俗,则主威分。民以法难犯上,而上以法挠慈仁,故下明爱施而务赇纹之政,是以法令隳。尊私行以贰主威,行赇纹以疑法,听之则乱治,不听则谤主,故君轻乎位而法乱乎官,此之谓无常之国。明主之道,臣不得以行义成荣,不得以家利为功,功名所生,必出于官法。法之年外,虽有难行,不以显焉,故民无以私名。设法度以齐民,信赏罚以尽民能,明诽誉以劝沮。名号、赏罚、法令三隅。故大臣有行则尊君,百姓有功则利上,此之谓有道之国也。

\hypertarget{header-n1592}{%
\subsection{五蠹}\label{header-n1592}}

上古之世,人民少而禽兽众,人民不胜禽兽虫蛇。有圣人作,构木为巢以避群害,而民悦之,使王天下,号曰有巢氏。民食果蓏蚌蛤,腥臊恶臭而伤害腹胃,民多疾病。有圣人作,钻燧取火以化腥臊,而民说之,使王天下,号之曰燧人氏。中古之世,天下大水,而鲧、禹决渎。近古之世,桀、纣暴乱,而汤、武征伐。今有构木钻燧于夏后氏之世者,必为鲧、禹笑矣;有决渎于殷、周之世者,必为汤、武笑矣。然则今有美尧、舜、汤、武、禹之道于当今之世者,必为新圣笑矣。是以圣人不期修古,不法常可,论世之事,因为之备。宋有人耕田者,田中有株,兔走触株,折颈而死,因释其耒而守株,冀复得兔,兔不可复得,而身为宋国笑。今欲以先王之政,治当世之民,皆守株之类也。

古者丈夫不耕,草木之实足食也;妇人不织,禽兽之皮足衣也。不事力而养足,人民少而财有余,故民不争。是以厚赏不行,重罚不用,而民自治。今人有五子不为多,子又有五子,大父未死而有二十五孙。是以人民众而货财寡,事力劳而供养薄,故民争,虽倍赏累罚而不免于乱。

尧之王天下也,茅茨不翦,采椽不斫;粝粢之食,藿之羹;冬日麂裘,夏日葛衣;虽监门之服养,不亏于此矣。禹之王天下也,身执耒歃以为民先,股无肢,胫不生毛,虽臣虏之劳,不苦于此矣。以是言之,夫古之让天子者,是去监门之养,而离臣虏之劳也,古传天下而不足多也。今之县令,一日身死,子孙累世絜驾,故人重之。是以人之于让也,轻辞古之天子,难去今之县令者,薄厚之实异也。夫山居而谷汲者,腊而相遗以水;泽居苦水者,买庸而决窦。故饥岁之春,幼弟不饷;穰岁之秋,疏客必食。非疏骨肉爱过客也,多少之实异也。是以古之易财,非仁也,财多也;今之争夺,非鄙也,财寡也。轻辞天子,非高也,势薄也;争士橐,非下也,权重也。故圣人议多少、论薄厚为之政。故罚薄不为慈,诛严不为戾,称俗而行也。故事因于世,而备适于事。

古者大王处丰、镐之间,地方百里,行仁义而怀西戎,遂王天下。徐偃王处汉东,地方五百里,行仁义,割地而朝者三十有六国。荆文王恐其害己也,举兵伐徐,遂灭之。故文王行仁义而王天下,偃王行仁义而丧其国,是仁义用于古不用于今也。故曰:世异则事异。当舜之时,有苗不服,禹将伐之。舜曰:``不可。上德不厚而行武,非道也。''乃修教三年,执干戚舞,有苗乃服。共工之战,铁铦矩者及乎敌,铠甲不坚者伤乎体。是干戚用于古不用于今也。故曰:事异则备变。上古竞于道德,中世逐于智谋,当今争于气力。齐将攻鲁,鲁使子贡说之。齐人曰:``子言非不辩也,吾所欲者土地也,非斯言所谓也。''遂举兵伐鲁,去门十里以为界。故偃王仁义而徐亡,子贡辩智而鲁削。以是言之,夫仁义辩智,非所以持国也。去偃王之仁,息子贡之智,循徐、鲁之力使敌万乘,则齐、荆之欲不得行于二国矣。

夫古今异俗,新故异备。如欲以宽缓之政,治急世之民,犹无辔策而御马,此不知之患也。今儒、墨皆称先王兼爱天下,则视民如父母。何以明其然也?曰:``司寇行刑,君为之不举乐;闻死刑之报,君为流涕。''此所举先王也。夫以君臣为如父子则必治,推是言之,是无乱父子也。人之情性莫先于父母,皆见爱而未必治也,虽厚爱矣,奚遽不乱?今先王之爱民,不过父母之爱子,子未必不乱也,则民奚遽治哉?且夫以法行刑,而君为之流涕,此以效仁,非以为治也。夫垂泣不欲刑者,仁也;然而不可不刑者,法也。先王胜其法,不听其泣,则仁之不可以为治亦明矣。

且民者固服于势,寡能怀于义。仲尼,天下圣人也,修行明道以游海内,海内说其仁、美其义而为服役者七十人。盖贵仁者寡,能义者难也。故以天下之大,而为服役者七十人,而仁义者一人。鲁哀公,下主也,南面君国,境内之民莫敢不臣。民者固服于势,诚易以服人,故仲尼反为臣而哀公顾为君。仲尼非怀其义,服其势也。故以义则仲尼不服于哀公,乘势则哀公臣仲尼。今学者之说人主也,不乘必胜之势,而务行仁义则可以王,是求人主之必及仲尼,而以世之凡民皆如列徒,此必不得之数也。

今有不才之子,父母怒之弗为改,乡人谯之弗为动,师长教之弗为变。夫以父母之爱、乡人之行、师长之智,三美加焉,而终不动,其胫毛不改。州部之吏,操官兵,推公法,而求索奸人,然后恐惧,变其节,易其行矣。故父母之爱不足以教子,必待州部之严刑者,民固骄于爱、听于威矣。故十仞之城,楼季弗能逾者,峭也;千仞之山,跛牂易牧者,夷也。故明王峭其法而严其刑也。布帛寻常,庸人不释;铄金百溢,盗跖不掇。不必害,则不释寻常;必害手,则不掇百溢。故明主必其诛也。是以赏莫如厚而信,使民利之;罚莫如重而必,使民畏之;法莫如一而固,使民知之。故主施赏不迁,行诛无赦,誉辅其赏,毁随其罚,则贤、不肖俱尽其力矣。

今则不然。其有功也爵之,而卑其士官也;以其耕作也赏之,而少其家业也;以其不收也外之,而高其轻世也;以其犯禁罪之,而多其有勇也。毁誉、赏罚之所加者,相与悖缪也,故法禁坏而民愈乱。今兄弟被侵,必攻者,廉也;知友辱,随仇者,贞也。廉贞之行成,而君上之法犯矣。人主尊贞廉之行,而忘犯禁之罪,故民程于勇,而吏不能胜也。不事力而衣食,谓之能;不战功而尊,则谓之贤。贤能之行成,而兵弱而地荒矣。人主说贤能之行,而忘兵弱地荒之祸,则私行立而公利灭矣。

儒以文乱法,侠以武犯禁,而人主兼礼之,此所以乱也。夫离法者罪,而诸先王以文学取;犯禁者诛,而群侠以私剑养。故法之所非,君之所取;吏之所诛,上之所养也。法、趣、上、下,四相反也,而无所定,虽有十黄帝不能治也。故行仁义者非所誉,誉之则害功;文学者非所用,用之则乱法。楚之有直躬,其父窃羊,而谒之吏。令尹曰:``杀之!''以为直于君而曲于父,报而罪之。以是观之,夫君之直臣,父之暴子也。鲁人从君战,三战三北。仲尼问其故,对曰:``吾有老父,身死莫之养也。''仲尼以为孝,举而上之。以是观之,夫父之孝子,君之背臣也。故令尹诛而楚奸不上闻,仲尼赏而鲁民易降北。上下之利,若是其异也,而人主兼举匹夫之行,而求致社稷之福,必不几矣。

古者苍颉之作书也,自环者谓之私,背私谓之公,公私之相背也,乃苍颉固以知之矣。今以为同利者,不察之患也,然则为匹夫计者,莫如修行义而习文学。行义修则见信,见信则受事;文学习则为明师,为明师则显荣:此匹夫之美也。然则无功而受事,无爵而显荣,为有政如此,则国必乱,主必危矣。故不相容之事,不两立也。斩敌者受赏,而高慈惠之行;拔城者受爵禄,而信廉爱之说;坚甲厉兵以备难,而美荐绅之饰;富国以农,距敌恃卒,而贵文学之士;废敬上畏法之民,而养游侠私剑之属。举行如此,治强不可得也。国平养儒侠,难至用介士,所利非所用,所用非所利。是故服事者简其业,而于游学者日众,是世之所以乱也。

且世之所谓贤者,贞信之行也;所谓智者,微妙之言也。微妙之言,上智之所难知也。今为众人法,而以上智之所难知,则民无从识之矣。故糟糠不饱者不务粱肉,短褐不完者不待文绣。夫治世之事,急者不得,则缓者非所务也。今所治之政,民间之事,夫妇所明知者不用,而慕上知之论,则其于治反矣。故微妙之言,非民务也。若夫贤良贞信之行者,必将贵不欺之士;不欺之士者,亦无不欺之术也。布衣相与交,无富厚以相利,无威势以相惧也,故求不欺之士。今人主处制人之势,有一国之厚,重赏严诛,得操其柄,以修明术之所烛,虽有田常、子罕之臣,不敢欺也,奚待于不欺之士?今贞信之士不盈于十,而境内之官以百数,必任贞信之士,则人不足官。人不足官,则治者寡而乱者众矣。故明主之道,一法而不求智,固术而不慕信,故法不败,而群官无奸诈矣。

今人主之于言也,说其辩而不求其当焉;其用于行也,美其声而不责其功。是以天下之众,其谈言者务为辨而不周于用,故举先王言仁义者盈廷,而政不免于乱;行身者竞于为高而不合于功,故智士退处岩穴,归禄不受,而兵不免于弱,政不免于乱,此其故何也?民之所誉,上之所礼,乱国之术也。今境内之民皆言治,藏商、管之法者家有之,而国贫,言耕者众,执耒者寡也;境内皆言兵,藏孙、吴之书者家有之,而兵愈弱,言战者多,被甲者少也。故明主用其力,不听其言;赏其功,伐禁无用。故民尽死力以从其上。夫耕之用力也劳,而民为之者,曰:可得以富也。战之事也危,而民为之者,曰:可得以贵也。今修文学,习言谈,则无耕之劳而有富之实,无战之危而有贵之尊,则人孰不为也?是以百人事智而一人用力。事智者众,则法败;用力者寡,则国贫:此世之所以乱也。\textgreater{}\textgreater{}\textgreater{}\textgreater{}故明主之国,无书简之文,以法为教;无先王之语,以吏为师;无私剑之捍,以斩首为勇。是境内之民,其言谈者必轨于法,动作者归之于功,为勇者尽之于军。是故无事则国富,有事则兵强,此之谓王资。既畜王资而承敌国之儥超五帝侔三王者,必此法也。

今则不然,士民纵恣于内,言谈者为势于外,外内称恶,以待强敌,不亦殆乎!故群臣之言外事者,非有分于从衡之党,则有仇雠之忠,而借力于国也。从者,合众强以攻一弱也;而衡者,事一强以攻众弱也:皆非所以持国也。今人臣之言衡者,皆曰:``不事大,则遇敌受祸矣。''事大未必有实,则举图而委,效玺而请兵矣。献图则地削,效玺则名卑,地削则国削,名卑则政乱矣。事大为衡,未见其利也,而亡地乱政矣。人臣之言从者,皆曰:``不救小而伐大,则失天下,失天下则国危,国危而主卑。''救小未必有实,则起兵而敌大矣。救小未必能存,而交大未必不有疏,有疏则为强国制矣。出兵则军败,退守则城拔。救小为从,未见其利,而亡地败军矣。是故事强,则以外权士官于内;求小,则以内重求利于外。国利未立,封土厚禄至矣;主上虽卑,人臣尊矣;国地虽削,私家富矣。事成,则以权长重;事败,则以富退处。人主之于其听说也于其臣,事未成则爵禄已尊矣;事败而弗诛,则游说之士孰不为用缴之说而侥幸其后?故破国亡主以听言谈者之浮说。此其故何也?是人君不明乎公私之利,不察当否之言,而诛罚不必其后也。皆曰:``外事,大可以王,小可以安。''夫王者,能攻人者也;而安,则不可攻也。强,则能攻人者也;治,则不可攻也。治强不可责于外,内政之有也。今不行法术于内,而事智于外,则不至于治强矣。

鄙谚曰:``长袖善舞,多钱善贾。''此言多资之易为工也。故治强易为谋,弱乱难为计。故用于秦者,十变而谋希失;用于燕者,一变而计希得。非用于秦者必智,用于燕者必愚也,盖治乱之资异也。故周去秦为从,期年而举;卫离魏为衡,半岁而亡。是周灭于从,卫亡于衡也。使周、卫缓其从衡之计,而严其境内之治,明其法禁,必其赏罚,尽其地力以多其积,致其民死以坚其城守,天下得其地则其利少,攻其国则其伤大,万乘之国莫敢自顿于坚城之下,而使强敌裁其弊也,此必不亡之术也。舍必不亡之术而道必灭之事,治国者之过也。智困于内而政乱于外,则亡不可振也。

民之政计,皆就安利如辟危穷。今为之攻战,进则死于敌,退则死于诛,则危矣。弃私家之事而必汗马之劳,家困而上弗论,则穷矣。穷危之所在也,民安得勿避?故事私门而完解舍,解舍完则远战,远战则安。行货赂而袭当涂者则求得,求得则私安,私安则利之所在,安得勿就?是以公民少而私人众矣。

夫明王治国之政,使其商工游食之民少而名卑,以寡趣本务而趋末作。今世近习之请行,则官爵可买;官爵可买,则商工不卑也矣。奸财货贾得用于市,则商人不少矣。聚敛倍农而致尊过耕战之士,则耿介之士寡而高价之民多矣。

是故乱国之俗:其学者,则称先王之道以籍仁义,盛容服而饰辩说,以疑当世之法,而贰人主之心。其言古者,为设诈称,借于外力,以成其私,而遗社稷之利。其带剑者,聚徒属,立节操,以显其名,而犯五官之禁。其患御者,积于私门,尽货赂,而用重人之谒,退汗马之劳。其商工之民,修治苦之器,聚弗靡之财,蓄积待时,而侔农夫之利。此五者,邦之蠹也。人主不除此五蠹之民,不养耿介之士,则海内虽有破亡之国,削灭之朝,亦勿怪矣。

\hypertarget{header-n1593}{%
\subsection{显学}\label{header-n1593}}

上古之世,人民少而禽兽众,人民不胜禽兽虫蛇。有圣人作,构木为巢以避群害,而民悦之,使王天下,号曰有巢氏。民食果蓏蚌蛤,腥臊恶臭而伤害腹胃,民多疾病。有圣人作,钻燧取火以化腥臊,而民说之,使王天下,号之曰燧人氏。中古之世,天下大水,而鲧、禹决渎。近古之世,桀、纣暴乱,而汤、武征伐。今有构木钻燧于夏后氏之世者,必为鲧、禹笑矣;有决渎于殷、周之世者,必为汤、武笑矣。然则今有美尧、舜、汤、武、禹之道于当今之世者,必为新圣笑矣。是以圣人不期修古,不法常可,论世之事,因为之备。宋有人耕田者,田中有株,兔走触株,折颈而死,因释其耒而守株,冀复得兔,兔不可复得,而身为宋国笑。今欲以先王之政,治当世之民,皆守株之类也。

古者丈夫不耕,草木之实足食也;妇人不织,禽兽之皮足衣也。不事力而养足,人民少而财有余,故民不争。是以厚赏不行,重罚不用,而民自治。今人有五子不为多,子又有五子,大父未死而有二十五孙。是以人民众而货财寡,事力劳而供养薄,故民争,虽倍赏累罚而不免于乱。

尧之王天下也,茅茨不翦,采椽不斫;粝粢之食,藿之羹;冬日麂裘,夏日葛衣;虽监门之服养,不亏于此矣。禹之王天下也,身执耒歃以为民先,股无肢,胫不生毛,虽臣虏之劳,不苦于此矣。以是言之,夫古之让天子者,是去监门之养,而离臣虏之劳也,古传天下而不足多也。今之县令,一日身死,子孙累世絜驾,故人重之。是以人之于让也,轻辞古之天子,难去今之县令者,薄厚之实异也。夫山居而谷汲者,腊而相遗以水;泽居苦水者,买庸而决窦。故饥岁之春,幼弟不饷;穰岁之秋,疏客必食。非疏骨肉爱过客也,多少之实异也。是以古之易财,非仁也,财多也;今之争夺,非鄙也,财寡也。轻辞天子,非高也,势薄也;争士橐,非下也,权重也。故圣人议多少、论薄厚为之政。故罚薄不为慈,诛严不为戾,称俗而行也。故事因于世,而备适于事。

古者大王处丰、镐之间,地方百里,行仁义而怀西戎,遂王天下。徐偃王处汉东,地方五百里,行仁义,割地而朝者三十有六国。荆文王恐其害己也,举兵伐徐,遂灭之。故文王行仁义而王天下,偃王行仁义而丧其国,是仁义用于古不用于今也。故曰:世异则事异。当舜之时,有苗不服,禹将伐之。舜曰:``不可。上德不厚而行武,非道也。''乃修教三年,执干戚舞,有苗乃服。共工之战,铁铦矩者及乎敌,铠甲不坚者伤乎体。是干戚用于古不用于今也。故曰:事异则备变。上古竞于道德,中世逐于智谋,当今争于气力。齐将攻鲁,鲁使子贡说之。齐人曰:``子言非不辩也,吾所欲者土地也,非斯言所谓也。''遂举兵伐鲁,去门十里以为界。故偃王仁义而徐亡,子贡辩智而鲁削。以是言之,夫仁义辩智,非所以持国也。去偃王之仁,息子贡之智,循徐、鲁之力使敌万乘,则齐、荆之欲不得行于二国矣。

夫古今异俗,新故异备。如欲以宽缓之政,治急世之民,犹无辔策而御马,此不知之患也。今儒、墨皆称先王兼爱天下,则视民如父母。何以明其然也?曰:``司寇行刑,君为之不举乐;闻死刑之报,君为流涕。''此所举先王也。夫以君臣为如父子则必治,推是言之,是无乱父子也。人之情性莫先于父母,皆见爱而未必治也,虽厚爱矣,奚遽不乱?今先王之爱民,不过父母之爱子,子未必不乱也,则民奚遽治哉?且夫以法行刑,而君为之流涕,此以效仁,非以为治也。夫垂泣不欲刑者,仁也;然而不可不刑者,法也。先王胜其法,不听其泣,则仁之不可以为治亦明矣。

且民者固服于势,寡能怀于义。仲尼,天下圣人也,修行明道以游海内,海内说其仁、美其义而为服役者七十人。盖贵仁者寡,能义者难也。故以天下之大,而为服役者七十人,而仁义者一人。鲁哀公,下主也,南面君国,境内之民莫敢不臣。民者固服于势,诚易以服人,故仲尼反为臣而哀公顾为君。仲尼非怀其义,服其势也。故以义则仲尼不服于哀公,乘势则哀公臣仲尼。今学者之说人主也,不乘必胜之势,而务行仁义则可以王,是求人主之必及仲尼,而以世之凡民皆如列徒,此必不得之数也。

今有不才之子,父母怒之弗为改,乡人谯之弗为动,师长教之弗为变。夫以父母之爱、乡人之行、师长之智,三美加焉,而终不动,其胫毛不改。州部之吏,操官兵,推公法,而求索奸人,然后恐惧,变其节,易其行矣。故父母之爱不足以教子,必待州部之严刑者,民固骄于爱、听于威矣。故十仞之城,楼季弗能逾者,峭也;千仞之山,跛牂易牧者,夷也。故明王峭其法而严其刑也。布帛寻常,庸人不释;铄金百溢,盗跖不掇。不必害,则不释寻常;必害手,则不掇百溢。故明主必其诛也。是以赏莫如厚而信,使民利之;罚莫如重而必,使民畏之;法莫如一而固,使民知之。故主施赏不迁,行诛无赦,誉辅其赏,毁随其罚,则贤、不肖俱尽其力矣。

今则不然。其有功也爵之,而卑其士官也;以其耕作也赏之,而少其家业也;以其不收也外之,而高其轻世也;以其犯禁罪之,而多其有勇也。毁誉、赏罚之所加者,相与悖缪也,故法禁坏而民愈乱。今兄弟被侵,必攻者,廉也;知友辱,随仇者,贞也。廉贞之行成,而君上之法犯矣。人主尊贞廉之行,而忘犯禁之罪,故民程于勇,而吏不能胜也。不事力而衣食,谓之能;不战功而尊,则谓之贤。贤能之行成,而兵弱而地荒矣。人主说贤能之行,而忘兵弱地荒之祸,则私行立而公利灭矣。

儒以文乱法,侠以武犯禁,而人主兼礼之,此所以乱也。夫离法者罪,而诸先王以文学取;犯禁者诛,而群侠以私剑养。故法之所非,君之所取;吏之所诛,上之所养也。法、趣、上、下,四相反也,而无所定,虽有十黄帝不能治也。故行仁义者非所誉,誉之则害功;文学者非所用,用之则乱法。楚之有直躬,其父窃羊,而谒之吏。令尹曰:``杀之!''以为直于君而曲于父,报而罪之。以是观之,夫君之直臣,父之暴子也。鲁人从君战,三战三北。仲尼问其故,对曰:``吾有老父,身死莫之养也。''仲尼以为孝,举而上之。以是观之,夫父之孝子,君之背臣也。故令尹诛而楚奸不上闻,仲尼赏而鲁民易降北。上下之利,若是其异也,而人主兼举匹夫之行,而求致社稷之福,必不几矣。

古者苍颉之作书也,自环者谓之私,背私谓之公,公私之相背也,乃苍颉固以知之矣。今以为同利者,不察之患也,然则为匹夫计者,莫如修行义而习文学。行义修则见信,见信则受事;文学习则为明师,为明师则显荣:此匹夫之美也。然则无功而受事,无爵而显荣,为有政如此,则国必乱,主必危矣。故不相容之事,不两立也。斩敌者受赏,而高慈惠之行;拔城者受爵禄,而信廉爱之说;坚甲厉兵以备难,而美荐绅之饰;富国以农,距敌恃卒,而贵文学之士;废敬上畏法之民,而养游侠私剑之属。举行如此,治强不可得也。国平养儒侠,难至用介士,所利非所用,所用非所利。是故服事者简其业,而于游学者日众,是世之所以乱也。

且世之所谓贤者,贞信之行也;所谓智者,微妙之言也。微妙之言,上智之所难知也。今为众人法,而以上智之所难知,则民无从识之矣。故糟糠不饱者不务粱肉,短褐不完者不待文绣。夫治世之事,急者不得,则缓者非所务也。今所治之政,民间之事,夫妇所明知者不用,而慕上知之论,则其于治反矣。故微妙之言,非民务也。若夫贤良贞信之行者,必将贵不欺之士;不欺之士者,亦无不欺之术也。布衣相与交,无富厚以相利,无威势以相惧也,故求不欺之士。今人主处制人之势,有一国之厚,重赏严诛,得操其柄,以修明术之所烛,虽有田常、子罕之臣,不敢欺也,奚待于不欺之士?今贞信之士不盈于十,而境内之官以百数,必任贞信之士,则人不足官。人不足官,则治者寡而乱者众矣。故明主之道,一法而不求智,固术而不慕信,故法不败,而群官无奸诈矣。

今人主之于言也,说其辩而不求其当焉;其用于行也,美其声而不责其功。是以天下之众,其谈言者务为辨而不周于用,故举先王言仁义者盈廷,而政不免于乱;行身者竞于为高而不合于功,故智士退处岩穴,归禄不受,而兵不免于弱,政不免于乱,此其故何也?民之所誉,上之所礼,乱国之术也。今境内之民皆言治,藏商、管之法者家有之,而国贫,言耕者众,执耒者寡也;境内皆言兵,藏孙、吴之书者家有之,而兵愈弱,言战者多,被甲者少也。故明主用其力,不听其言;赏其功,伐禁无用。故民尽死力以从其上。夫耕之用力也劳,而民为之者,曰:可得以富也。战之事也危,而民为之者,曰:可得以贵也。今修文学,习言谈,则无耕之劳而有富之实,无战之危而有贵之尊,则人孰不为也?是以百人事智而一人用力。事智者众,则法败;用力者寡,则国贫:此世之所以乱也。\textgreater{}\textgreater{}\textgreater{}\textgreater{}故明主之国,无书简之文,以法为教;无先王之语,以吏为师;无私剑之捍,以斩首为勇。是境内之民,其言谈者必轨于法,动作者归之于功,为勇者尽之于军。是故无事则国富,有事则兵强,此之谓王资。既畜王资而承敌国之儥超五帝侔三王者,必此法也。

今则不然,士民纵恣于内,言谈者为势于外,外内称恶,以待强敌,不亦殆乎!故群臣之言外事者,非有分于从衡之党,则有仇雠之忠,而借力于国也。从者,合众强以攻一弱也;而衡者,事一强以攻众弱也:皆非所以持国也。今人臣之言衡者,皆曰:``不事大,则遇敌受祸矣。''事大未必有实,则举图而委,效玺而请兵矣。献图则地削,效玺则名卑,地削则国削,名卑则政乱矣。事大为衡,未见其利也,而亡地乱政矣。人臣之言从者,皆曰:``不救小而伐大,则失天下,失天下则国危,国危而主卑。''救小未必有实,则起兵而敌大矣。救小未必能存,而交大未必不有疏,有疏则为强国制矣。出兵则军败,退守则城拔。救小为从,未见其利,而亡地败军矣。是故事强,则以外权士官于内;求小,则以内重求利于外。国利未立,封土厚禄至矣;主上虽卑,人臣尊矣;国地虽削,私家富矣。事成,则以权长重;事败,则以富退处。人主之于其听说也于其臣,事未成则爵禄已尊矣;事败而弗诛,则游说之士孰不为用缴之说而侥幸其后?故破国亡主以听言谈者之浮说。此其故何也?是人君不明乎公私之利,不察当否之言,而诛罚不必其后也。皆曰:``外事,大可以王,小可以安。''夫王者,能攻人者也;而安,则不可攻也。强,则能攻人者也;治,则不可攻也。治强不可责于外,内政之有也。今不行法术于内,而事智于外,则不至于治强矣。

鄙谚曰:``长袖善舞,多钱善贾。''此言多资之易为工也。故治强易为谋,弱乱难为计。故用于秦者,十变而谋希失;用于燕者,一变而计希得。非用于秦者必智,用于燕者必愚也,盖治乱之资异也。故周去秦为从,期年而举;卫离魏为衡,半岁而亡。是周灭于从,卫亡于衡也。使周、卫缓其从衡之计,而严其境内之治,明其法禁,必其赏罚,尽其地力以多其积,致其民死以坚其城守,天下得其地则其利少,攻其国则其伤大,万乘之国莫敢自顿于坚城之下,而使强敌裁其弊也,此必不亡之术也。舍必不亡之术而道必灭之事,治国者之过也。智困于内而政乱于外,则亡不可振也。

民之政计,皆就安利如辟危穷。今为之攻战,进则死于敌,退则死于诛,则危矣。弃私家之事而必汗马之劳,家困而上弗论,则穷矣。穷危之所在也,民安得勿避?故事私门而完解舍,解舍完则远战,远战则安。行货赂而袭当涂者则求得,求得则私安,私安则利之所在,安得勿就?是以公民少而私人众矣。

夫明王治国之政,使其商工游食之民少而名卑,以寡趣本务而趋末作。今世近习之请行,则官爵可买;官爵可买,则商工不卑也矣。奸财货贾得用于市,则商人不少矣。聚敛倍农而致尊过耕战之士,则耿介之士寡而高价之民多矣。

是故乱国之俗:其学者,则称先王之道以籍仁义,盛容服而饰辩说,以疑当世之法,而贰人主之心。其言古者,为设诈称,借于外力,以成其私,而遗社稷之利。其带剑者,聚徒属,立节操,以显其名,而犯五官之禁。其患御者,积于私门,尽货赂,而用重人之谒,退汗马之劳。其商工之民,修治苦之器,聚弗靡之财,蓄积待时,而侔农夫之利。此五者,邦之蠹也。人主不除此五蠹之民,不养耿介之士,则海内虽有破亡之国,削灭之朝,亦勿怪矣。

\hypertarget{header-n1594}{%
\subsection{忠孝}\label{header-n1594}}

世之显学,儒、墨也。儒之所至,孔丘也。墨之所至,墨翟也。自孔子之死也,有子张之儒,有子思之儒,有颜氏之儒,有孟氏之儒,有漆雕氏之儒,有仲良氏之儒,有孙氏之儒,有乐正乐之儒。自墨子之死也,有相里氏之墨,有相夫氏之墨,有邓陵氏之墨。故孔、墨之后,儒分为八,墨离为三,取舍相反不同,而皆自谓真孔、墨,孔、墨不可复生,将谁使定世之学乎?孔子、墨子俱道尧、舜,而取舍不同,皆自谓真尧、舜,尧、舜不复生,将谁使定儒、墨之诚乎?殷、周七百余岁,虞、夏二千余岁,而不能定儒、墨之真;今乃欲审尧、舜之道于三千岁之前,意者其不可必乎!无参验而必之者,愚也;弗能必而据之者,诬也。故明据先王,必定尧、舜者,非愚则诬也。愚诬之学,杂反之行,明主弗受也。

墨者之葬也,冬日冬服,夏日夏服,桐棺三寸,服丧三月,世以为俭而礼之。儒者破家而葬,服丧三年,大毁扶杖,世主以为孝而礼之。夫是墨子之俭,将非孔子之侈也;是孔子之孝,将非墨子之戾也。今孝、戾、侈、俭俱在儒、墨,而上兼礼之。漆雕之议,不色挠,不目逃,行曲则违于臧获,行直则怒于诸侯,世主以为廉而礼之。宋荣子之议,设不斗争,取不随仇,不羞囹圄,见侮不辱,世主以为宽而礼之。夫是漆雕之廉,将非宋荣之恕也;是宋荣之宽,将非漆雕之暴也。今宽、廉、恕、暴俱在二子,人主兼而礼之。自愚诬之学、杂反之辞争,而人主俱听之,故海内之士,言无定术,行无常议。夫冰炭不同器而久,寒暑不兼时而至,杂反之学不两立而治。今兼听杂学缪行同异之辞,安得无乱乎?听行如此,其于治人又必然矣。

今世之学士语治者,多曰:``与贫穷地以实无资。''今夫与人相善也,无丰年旁入之利而独以完给者,非力则俭也。与人相善也,无饥馑、疾疚、祸罪之殃独以贫穷者,非侈则堕也。侈而堕者贫,而力而俭者富。今上征敛于富人以布施于贫家,是夺力俭而与侈堕也,而欲索民之疾作而节用,不可得也。

今有人于此,义不入危城,不处军旅,不以天下大利易其胫一毛,世主必从而礼之,贵其智而高其行,以为轻物重生之士也。夫上所以陈良田大宅,设爵禄,所以易民死命也。今上尊贵轻物重生之士,而索民之出死而重殉上事,不可得也。藏书策,习谈论,聚徒役,服文学而议说,世主必从而礼之,曰:``敬贤士,先王之道也。''夫吏之所税,耕者也;而上之所养,学士也。耕者则重税,学士则多赏,而索民之疾作而少言谈,不可得也。立节参明,执操不侵,怨言过于耳,必随之以剑,世主必从而礼之,以为自好之士。夫斩首之劳不赏,而家斗之勇尊显,而索民之疾战距敌而无私斗,不可得也。国平则养儒侠,难至则用介士。所养者非所用,所用者非所养,此所以乱也。且夫人主于听学也,若是其言,宜布之官而用其身;若非其言,宜去其身而息其端。今以为是也,而弗布于官;以为非也,而不息其端。是而不用,非而不息,乱亡之道也。

澹台子羽,君子之容也,仲尼几而取之,与处久而行不称其貌。宰予之辞,雅而文也,仲尼几而取之,与处久而智不充其辩。故孔子曰:``以容取人乎,失之子羽;以言取人乎,失之宰予。''故以仲尼之智而有失实之声。今之新辩滥乎宰予,而世主之听眩乎仲尼,为悦其言,因任其身,则焉得无失乎?是以魏任孟卯之辩,而有华下之患;赵任马服之辩,而有长平之祸。此二者,任辩之失也。夫视锻锡而察青黄,区冶不能以必剑;水击鹄雁,陆断驹马,则臧获不疑钝利。发齿吻形容,伯乐不能以必马;授车就驾,而观其末涂,则臧获不疑驽良。观容服,听辞言,仲尼不能以必士;试之官职,课其功伐,则庸人不疑于愚智。故明主之吏,宰相必起于州部,猛将必发于卒伍。夫有功者必赏,则爵禄厚而愈劝;迁官袭级,则官职大而愈治。夫爵禄大而官职治,王之道也。

磐石千里,不可谓富;象人百万,不可谓强。石非不大,数非不众也,而不可谓富强者,磐不生粟,象人不可使距敌也。今商官技艺之士亦不垦而食,是地不垦,与磐石一贯也。儒侠毋军劳,显而荣者,则民不使,与象人同事也。夫祸知磐石象人,而不知祸商官儒侠为不垦之地、不使之民,不知事类者也。

故敌国之君王虽说吾义,吾弗入贡而臣;关内之侯虽非吾行,吾必使执禽而朝。是故力多则人朝,力寡则朝于人,故明君务力。夫严家无悍虏,而慈母有败子。吾以此知威势之可以禁暴,而德厚之不足以止乱也。

夫圣人之治国,不恃人之为吾善也,而用其不得为非也。恃人之为吾善也,境内不什数;用人不得为非,一国可使齐。为治者用众而舍寡,故不务德而务法。夫必恃自直之箭,百世无矢;恃自圜之木,千世无轮矣。自直之箭,自圜之木,百世无有一,然而世皆乘车射禽者何也?隐栝之道用也。虽有不恃隐栝而有自直之箭、自圜之术,良工弗贵也。何则?乘者非一人,射者非一发也。不恃赏罚而恃自善之民,明主弗贵也。何则?国法不可失,而所治非一人也。故有术之君,不随适然之善,而行必然之道。

今或谓人曰:``使子必智而寿'',则世必以为狂。夫智,性也;寿,命也。性命者,非所学于人也,而以人之所不能为说人,此世之所以谓之为狂也。谓之不能然,则是谕也,夫谕性也。以仁义教人,是以智与寿说也,有度之主弗受也。故善毛啬、西施之美,无益吾面;用脂泽粉黛,则倍其初。言先王之仁义,无益于治;明吾法度,必吾赏罚者,亦国之脂泽粉黛也。故明主急其助而缓其颂,故不道仁义。

今巫祝之祝人曰:``使若千秋万岁。''千秋万岁之声括耳,而一日之寿无征于人,此人所以简巫祝也。今世儒者之说人主,不善今之所以为治,而语已治之功;不审官法之事,不察奸邪之情,而皆道上古之传誉、先王之成功。儒者饰辞曰:``听吾言,则可以霸王。''此说者之巫祝,有度之主不受也。故明主举实事,去无用,不道仁义者故,不听学者之言。

今不知治者必曰:``得民之心。''欲得民之心而可以为治,则是伊尹、管仲无所用也,将听民而已矣。民智之不可用,犹婴儿之心也。夫婴儿不剔首则腹痛,不
 痤则寖益。剔首、 
痤,必一人抱之,慈母治之,然犹啼呼不止,婴儿子不知犯其所小苦致其所大利也。今上急耕田垦草以厚民产也,而以上为酷;修刑重罚以为禁邪也,而以上为严;征赋钱粟以实仓库,且以救饥馑、备军旅也,而以上为贪;境内必知介而无私解,并力疾斗,所以禽虏也,而以上为暴。此四者,所以治安也,而民不知悦也。夫求圣通之。

\hypertarget{header-n1595}{%
\subsection{人主}\label{header-n1595}}

天下皆以孝悌忠顺之道为是也,而莫知察孝悌忠顺之道而审行之,是以天下乱。皆以尧舜之道为是而法之,是以有弑君,有曲于父。尧、舜、汤、武或反群臣之义,乱后世之教者也。尧为人君而君其臣,舜为人臣而臣其君,汤、武为人臣而弑其主、刑其尸,而天下誉之,此天下所以至今不治者也。夫所谓明君者,能畜其臣者也;所谓贤臣者,能明法辟、治官职以戴其君者也。今尧自以为明而不能以畜舜,舜自以为贤而不能以戴尧;汤、武自以为义而弑其君长,此明君且常与而贤臣且常取也。故至今为人子者有取其父之家,为人臣者有取其君之国者矣。父而让子,君而让臣,此非所以定位一教之道也。臣之所闻曰:``臣事君,子事父,妻事夫。三者顺则天下治,三者逆则天下乱,此天下之常道也。''明王贤臣而弗易也,则人主虽不肖,臣不敢侵也。今夫上贤任智无常,逆道也,而天下常以为治。是故田氏夺吕氏于齐,戴氏夺子氏于宋。此皆贤且智也,岂愚且不肖乎?是废常上贤则乱,舍法任智则危。故曰:上法而不上贤。

记曰:``舜见瞽瞍,其容造焉。孔子曰:当是时也,危哉,天下岌岌!有道者,父固不得而子,君固不得而臣也。'''臣曰:孔子本未知教悌忠顺之道也。然则有道者,进不为臣主,退不为父子耶?父之所以欲有贤子者,家贫则富之,父苦则乐之;君之所以欲有贤臣者,国乱则治之,主卑则尊之。今有贤子而不为父,则父之处家也苦;有贤臣而不为君,则君之处位也危。然则父有贤子,君有贤臣,适足以为害耳,岂得利焉哉?所谓忠臣,不危其君;孝子,不非其亲。今舜以贤取君之国,而汤、武以义放弑其君,此皆以贤而危主者也,而天下贤之。古之烈士,进不臣君,退不为家,是进则非其君,退则非其亲者也。且夫进不臣君,退不为家,乱世绝嗣之道也。是故贤尧、舜、汤、武而是烈士,天下之乱术也。瞽瞍为舜父而舜放之,象为舜弟而杀之。放父杀弟,不可谓仁;妻帝二女而取天下,不可谓义。仁义无有,不可谓明。《诗》云:``普天之下,莫非王土;率土之滨,莫非王臣。''信若《诗》之言也,是舜出则臣其君,入则臣其父,妾其母,妻其主女也。故烈士内不为家,乱世绝嗣;而外矫于君,朽骨烂肉,施于土地,流于川谷,不避蹈水火。使天下从而效之,是天下遍死而愿夭也。此皆释世而不治是也。世之所为烈士者,虽众独行,取异于人,为恬淡之学而理恍惚之言。臣以为恬淡,无用之教也;恍惚,无法之言也。言出于无法,数出于无用者,天下谓之察。臣以为人生必事君养亲,事君养亲不可以恬淡;之人必以言论忠信法术,言论忠信法术不可以恍惚。恍惚之言,恬淡之学,天下之惑术也。孝子之事父也,非竞取父之家也;忠臣之事君也,非竞取君之国也。夫为人子而常誉他人之亲曰:``某子之亲,夜寝早起,强力生财以养子孙臣妾。''是诽谤其亲者也。为人臣常誉先王之德厚而愿之,诽谤其君者也。非其亲者知谓不孝,而非其君者天下此贤之,此所以乱也。故人臣毋称尧舜之贤,毋誉汤、武之伐,毋言烈士之高,尽力守法,专心于事主者为忠臣。

古者黔首悗密春惷愚,故可以虚名取也。今民儇诇智慧,欲自用,不听上。上必且劝之以赏,然后可进;又且畏之以罚,然后不敢退。而世皆曰:``许由让天下,赏不足以劝;盗跖犯刑赴难,罚不足以禁。''臣曰:未有天下而无以天下为者,许由是也;已有天下而无以天下为者,尧、舜是也。毁廉求财,犯刑趋利,忘身之死者,盗跖是也。此二者,殆物也。治国用民之道也,不以此二者为量。治也者,治常者也;道也者,道常者也。殆物妙言,治之害也。天下太平之士,不可以赏劝也;天下太下之士,不可以刑禁也。然为太上士不设赏,为太下士不设刑,则治国用民之道失矣。

故世人多不言国法而言从横。诸侯言从者曰:``从成必霸'';而言横者曰:``横成必王''。山东之言从横未尝一日而止也,然而功名不成,霸王不立者,虚言非所以成治也。王者独行谓之王,是以三王不务离合而正,五霸不待从横而察,治内以裁外而已矣。

\hypertarget{header-n1596}{%
\subsection{饬令}\label{header-n1596}}

人主之所以身危国亡者,大臣太贵,左右太威也。所谓贵者,无法而擅行,操国柄而便私者也。所谓威者,擅权势而轻重者也。此二者,不可不察也。夫马之所以能任重引车致远道者,以筋力也。万乘之主、千乘之君所以制天下而征诸侯者,以其威势也。威势者,人主之筋力也。今大臣得威,左右擅势,是人主失力;人主失力而能有国者,千无一人。虎豹之所以能胜人执百兽者,以其爪牙也,当使虎豹失其爪牙,则人必制之矣。今势重者,人主之爪牙也,君人而失其爪牙,虎豹之类也。宋君失其爪牙于子罕,简公失其爪牙于田常,而不蚤夺之,故身死国亡。今无术之主皆明知宋、简之过也,而不悟其失,不察其事类者也。

且法术之士与当涂之臣,不相容也。何以明之?主有术士,则大臣不得制断,近习不敢卖重;大臣、左右权势息,则人主之道明矣。今则不然,其当涂之臣得势擅事以环其私,左右近习朋党比周以制疏远,则法述之士奚时得进用,人主奚时得论裁?故有术不必用,而势不两立。法述之士焉得无危?故君人者非能退大臣之议,而背左右之讼,独合乎道言也,则法术之士安能蒙死亡之危而进说乎?此世之所以不治也。明主者,推功而爵禄,称能而官事,所举者必有贤,所用者必有能,贤能之士进,则私门之请止矣。夫有功者受重禄,有能者处大官,则私剑之士安得无离于私勇而疾距敌,游宦之士焉得无挠于私门而务于清洁矣?此所以聚贤能之士,而散私门之属也。今近习者不必智,人主之于人也或有所知而听之,入因与近习论其言,听近习而不计其智,是与愚论智也。其当涂者不必贤,人主之于人或有所贤而礼之,入因与当途者论其行,听其言而不用贤,是与不肖论贤也。故智者决策于愚人,贤士程行于不肖,则贤智之士奚时得用,而人主之明塞矣。昔关龙逄说桀而伤其四肢,王子比干谏纣而剖其心,子胥忠直夫差而诛于属镂。此三子者,为人臣非不忠,而说非不当也,然不免于死亡之患者,主不察贤智之言,而蔽于愚不肖之患也。今人主非肯用法术之士,听愚不肖之臣,则贤智之士孰敢当三子之危而进其智能者乎?此世之所以乱也。

\hypertarget{header-n1597}{%
\subsection{心度}\label{header-n1597}}

饬令,则法不迁;法平,则吏无奸。法已定矣,不以善言售法。任功,则民少言;任善,则民多言。行法曲断,以五里断者王,以九里断者强,宿治者削。

以刑治,以赏战、厚禄,以用术。行都之过,则都无奸市。物多者众,农弛奸胜,则国必削。民有余食,使以粟出爵,必以其力,则震不怠。三寸之管毋当,不可满也。授官爵出利禄不以功,是无当也。国以功授官与爵,此谓以成智谋,以威勇战,其国无敌。国以功授官与爵,则治见者省,言有塞,此谓以治去治,以言去言,以功与爵者也。故国多力,而天下莫之能侵也。兵出必取,取必能有之;案兵不攻必当。朝廷之事,小者不毁,效功取官爵,廷虽有辟言,不得以相干也,是谓以数治。以力攻者,出一取十;以言攻者,出十丧百。国好力,此谓以难攻;国好言,此谓以易攻。其能胜其害,轻其任,而道坏余力于心,莫负乘宫之责于君。内无伏怨,使明者不相干,故莫讼;使士不兼官,故技长;使人不同功,故莫争。言此谓易攻。

重刑少赏,上爱民,民死赏;多赏轻刑,上不爱民,民不死赏。利出一空者,其国无敌;利出二空者,其兵半用;利出十空者,民不守。重刑明民,大制使人,则上利。行刑,重其轻者,轻者不至,重者不来,此谓以刑去刑。罪重而刑轻。刑轻则事生,此谓以刑致刑,其国必削。

\hypertarget{header-n1598}{%
\subsection{制分}\label{header-n1598}}

夫凡国博君尊者,未尝非法重而可以至乎令行禁止于天下者也。是以君人者分爵制禄,则法必严以重之。夫国治则民安,事乱则邦危。法重者得人情,禁轻者失事实。且夫死力者,民之所有者也,情莫不出其死力以致其所欲;而好恶者,上之所制也,民者好利禄而恶刑罚。上掌好恶以御民力,事实不宜失矣,然而禁轻事失者,刑赏失也。其治民不秉法为善也,如是,则是无法也。

故治乱之理,宜务分刑赏为急。治国者莫不有法,然而有存有亡;亡者,其制刑赏不分也。治国者,其刑赏莫不有分:有持以异为分,不可谓分;至于察君之分,独分也。是以其民重法而畏禁,愿毋抵罪而不敢胥赏。故曰:不待刑赏而民从事矣。

是故夫至治之国,善以止奸为务。是何也?其法通乎人情,关乎治理也。然则去微奸之道奈何?其务令之相规其情者也。则使相窥奈何?曰:盖里相坐而已。禁尚有连于己者,理不得相窥,唯恐不得免。有奸心者不令得忘,窥者多也。如此,则慎己而窥彼,发奸之密。告过者免罪受赏,失奸者必诛连刑。如此,则奸类发矣。奸不容细,私告任坐使然也。

夫治法之至明者,任数不任人。是以有术之国,不用誉则毋适,境内必治,任数也。亡国使兵公行乎其地,而弗能圉禁者,任人而无数也。自攻者人也,攻人者数也。故有术之国,去言而任法。

凡畸功之循约者虽知,过刑之于言者难见也,是以刑赏惑乎贰。所谓循约难知者,奸功也。臣过之难见者,失根也。循理不见虚功,度情诡乎奸根,则二者安得无两失也?是以虚士立名于内,而谈者为略于外,故愚、怯、勇、慧相连而以虚道属俗而容乎世。故其法不用,而刑罚不加乎僇人。如此,则刑赏安得不容其二?实故有所至,而理失其量,量之失,非法使然也,法定而任慧也。释法而任慧者,则受事者安得其务?务不与事相得,则法安得无失,而刑安得无烦?是以赏罚扰乱,邦道差误,刑赏之不分白也。

\end{document}
