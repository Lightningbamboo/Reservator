\PassOptionsToPackage{unicode=true}{hyperref} % options for packages loaded elsewhere
\PassOptionsToPackage{hyphens}{url}
%
\documentclass[]{article}
\usepackage{lmodern}
\usepackage{amssymb,amsmath}
\usepackage{ifxetex,ifluatex}
\usepackage{fixltx2e} % provides \textsubscript
\ifnum 0\ifxetex 1\fi\ifluatex 1\fi=0 % if pdftex
  \usepackage[T1]{fontenc}
  \usepackage[utf8]{inputenc}
  \usepackage{textcomp} % provides euro and other symbols
\else % if luatex or xelatex
  \usepackage{unicode-math}
  \defaultfontfeatures{Ligatures=TeX,Scale=MatchLowercase}
\fi
% use upquote if available, for straight quotes in verbatim environments
\IfFileExists{upquote.sty}{\usepackage{upquote}}{}
% use microtype if available
\IfFileExists{microtype.sty}{%
\usepackage[]{microtype}
\UseMicrotypeSet[protrusion]{basicmath} % disable protrusion for tt fonts
}{}
\IfFileExists{parskip.sty}{%
\usepackage{parskip}
}{% else
\setlength{\parindent}{0pt}
\setlength{\parskip}{6pt plus 2pt minus 1pt}
}
\usepackage{hyperref}
\hypersetup{
            pdfborder={0 0 0},
            breaklinks=true}
\urlstyle{same}  % don't use monospace font for urls
\setlength{\emergencystretch}{3em}  % prevent overfull lines
\providecommand{\tightlist}{%
  \setlength{\itemsep}{0pt}\setlength{\parskip}{0pt}}
\setcounter{secnumdepth}{0}
% Redefines (sub)paragraphs to behave more like sections
\ifx\paragraph\undefined\else
\let\oldparagraph\paragraph
\renewcommand{\paragraph}[1]{\oldparagraph{#1}\mbox{}}
\fi
\ifx\subparagraph\undefined\else
\let\oldsubparagraph\subparagraph
\renewcommand{\subparagraph}[1]{\oldsubparagraph{#1}\mbox{}}
\fi

% set default figure placement to htbp
\makeatletter
\def\fps@figure{htbp}
\makeatother


\date{}

\begin{document}

\hypertarget{header-n0}{%
\section{离骚}\label{header-n0}}

\begin{center}\rule{0.5\linewidth}{\linethickness}\end{center}

\tableofcontents

\begin{center}\rule{0.5\linewidth}{\linethickness}\end{center}

\hypertarget{header-n6}{%
\subsection{离骚}\label{header-n6}}

\emph{屈原}

帝高阳之苗裔兮,朕皇考曰伯庸。\\
摄提贞于孟陬兮,惟庚寅吾以降。\\
皇览揆余初度兮,肇锡余以嘉名:\\
名余曰正则兮,字余曰灵均。\\
纷吾既有此内美兮,又重之以修能。\\
扈江离与辟芷兮,纫秋兰以为佩。\\
汩余若将不及兮,恐年岁之不吾与。\\
朝搴阰之木兰兮,夕揽洲之宿莽。\\
日月忽其不淹兮,春与秋其代序。\\
惟草木之零落兮,恐美人之迟暮。(惟 通:唯)\\
不抚壮而弃秽兮,何不改乎此度?\\
乘骐骥以驰骋兮,来吾道夫先路!\\
昔三后之纯粹兮,固众芳之所在。\\
杂申椒与菌桂兮,岂惟纫夫蕙茝!\\
彼尧、舜之耿介兮,既遵道而得路。\\
何桀纣之猖披兮,夫惟捷径以窘步。\\
惟夫党人之偷乐兮,路幽昧以险隘。\\
岂余身之殚殃兮,恐皇舆之败绩!\\
忽奔走以先后兮,及前王之踵武。\\
荃不查余之中情兮,反信谗而齌怒。\\
余固知謇謇之为患兮,忍而不能舍也。\\
指九天以为正兮,夫惟灵修之故也。\\
曰黄昏以为期兮,羌中道而改路!\\
初既与余成言兮,后悔遁而有他。\\
余既不难夫离别兮,伤灵修之数化。\\
余既滋兰之九畹兮,又树蕙之百亩。\\
畦留夷与揭车兮,杂杜衡与芳芷。\\
冀枝叶之峻茂兮,愿俟时乎吾将刈。\\
虽萎绝其亦何伤兮,哀众芳之芜秽。\\
众皆竞进以贪婪兮,凭不厌乎求索。\\
羌内恕己以量人兮,各兴心而嫉妒。\\
忽驰骛以追逐兮,非余心之所急。\\
老冉冉其将至兮,恐修名之不立。\\
朝饮木兰之坠露兮,夕餐秋菊之落英。\\
苟余情其信姱以练要兮,长顑颔亦何伤。\\
掔木根以结茝兮,贯薜荔之落蕊。\\
矫菌桂以纫蕙兮,索胡绳之纚纚。\\
謇吾法夫前修兮,非世俗之所服。\\
虽不周于今之人兮,愿依彭咸之遗则。\\
长太息以掩涕兮,哀民生之多艰。\\
余虽好修姱以鞿羁兮,謇朝谇而夕替。\\
既替余以蕙纕兮,又申之以揽茝。\\
亦余心之所善兮,虽九死其犹未悔。\\
怨灵修之浩荡兮,终不察夫民心。\\
众女嫉余之蛾眉兮,谣诼谓余以善淫。\\
固时俗之工巧兮,偭规矩而改错。\\
背绳墨以追曲兮,竞周容以为度。\\
忳郁邑余侘傺兮,吾独穷困乎此时也。\\
宁溘死以流亡兮,余不忍为此态也。\\
鸷鸟之不群兮,自前世而固然。\\
何方圜之能周兮,夫孰异道而相安?\\
屈心而抑志兮,忍尤而攘诟。\\
伏清白以死直兮,固前圣之所厚。\\
悔相道之不察兮,延伫乎吾将反。\\
回朕车以复路兮,及行迷之未远。\\
步余马于兰皋兮,驰椒丘且焉止息。\\
进不入以离尤兮,退将复修吾初服。\\
制芰荷以为衣兮,集芙蓉以为裳。\\
不吾知其亦已兮,苟余情其信芳。\\
高余冠之岌岌兮,长余佩之陆离。\\
芳与泽其杂糅兮,唯昭质其犹未亏。\\
忽反顾以游目兮,将往观乎四荒。\\
佩缤纷其繁饰兮,芳菲菲其弥章。\\
民生各有所乐兮,余独好修以为常。\\
虽体解吾犹未变兮,岂余心之可惩。\\
女嬃之婵媛兮,申申其詈予,曰:\\
「鲧婞直以亡身兮,终然夭乎羽之野。\\
汝何博謇而好修兮,纷独有此姱节?\\
薋菉葹以盈室兮,判独离而不服。」\\
众不可户说兮,孰云察余之中情?\\
世并举而好朋兮,夫何茕独而不予听?\\
依前圣以节中兮,喟凭心而历兹。\\
济沅、湘以南征兮,就重华而敶词:\\
启《九辩》与《九歌》兮,夏康娱以自纵。\\
不顾难以图后兮,五子用失乎家衖。\\
羿淫游以佚畋兮,又好射夫封狐。\\
固乱流其鲜终兮,浞又贪夫厥家。\\
浇身被服强圉兮,纵欲而不忍。\\
日康娱而自忘兮,厥首用夫颠陨。\\
夏桀之常违兮,乃遂焉而逢殃。\\
后辛之菹醢兮,殷宗用而不长。\\
汤、禹俨而祗敬兮,周论道而莫差。\\
举贤才而授能兮,循绳墨而不颇。\\
皇天无私阿兮,览民德焉错辅。\\
夫维圣哲以茂行兮,苟得用此下土。\\
瞻前而顾后兮,相观民之计极。\\
夫孰非义而可用兮?孰非善而可服?\\
阽余身而危死兮,览余初其犹未悔。\\
不量凿而正枘兮,固前修以菹醢。\\
曾歔欷余郁邑兮,哀朕时之不当。\\
揽茹蕙以掩涕兮,沾余襟之浪浪。\\
跪敷衽以陈辞兮,耿吾既得此中正。\\
驷玉虬以桀鹥兮,溘埃风余上征。\\
朝发轫于苍梧兮,夕余至乎县圃。\\
欲少留此灵琐兮,日忽忽其将暮。\\
吾令羲和弭节兮,望崦嵫而勿迫。\\
路漫漫其修远兮,吾将上下而求索。\\
饮余马于咸池兮,总余辔乎扶桑。\\
折若木以拂日兮,聊逍遥以相羊。\\
前望舒使先驱兮,后飞廉使奔属。\\
鸾皇为余先戒兮,雷师告余以未具。\\
吾令凤鸟飞腾兮,继之以日夜。\\
飘风屯其相离兮,帅云霓而来御。\\
纷总总其离合兮,斑陆离其上下。\\
吾令帝阍开关兮,倚阊阖而望予。\\
时暧暧其将罢兮,结幽兰而延伫。\\
世溷浊而不分兮,好蔽美而嫉妒。\\
朝吾将济于白水兮,登阆风而绁马。\\
忽反顾以流涕兮,哀高丘之无女。\\
溘吾游此春宫兮,折琼枝以继佩。\\
及荣华之未落兮,相下女之可诒。\\
吾令丰隆乘云兮,求宓妃之所在。\\
解佩纕以结言兮,吾令謇修以为理。\\
纷总总其离合兮,忽纬繣其难迁。\\
夕归次于穷石兮,朝濯发乎洧盘。\\
保厥美以骄傲兮,日康娱以淫游。\\
虽信美而无礼兮,来违弃而改求。\\
览相观于四极兮,周流乎天余乃下。\\
望瑶台之偃蹇兮,见有娀之佚女。\\
吾令鸩为媒兮,鸩告余以不好。\\
雄鸠之鸣逝兮,余犹恶其佻巧。\\
心犹豫而狐疑兮,欲自适而不可。\\
凤皇既受诒兮,恐高辛之先我。\\
欲远集而无所止兮,聊浮游以逍遥。\\
及少康之未家兮,留有虞之二姚。\\
理弱而媒拙兮,恐导言之不固。\\
世溷浊而嫉贤兮,好蔽美而称恶。\\
闺中既以邃远兮,哲王又不寤。\\
怀朕情而不发兮,余焉能忍而与此终古?\\
索琼茅以筳篿兮,命灵氛为余占之。\\
曰:「两美其必合兮,孰信修而慕之?\\
思九州之博大兮,岂惟是其有女?」\\
曰:「勉远逝而无狐疑兮,孰求美而释女?\\
何所独无芳草兮,尔何怀乎故宇?」\\
世幽昧以昡曜兮,孰云察余之善恶?\\
民好恶其不同兮,惟此党人其独异!\\
户服艾以盈要兮,谓幽兰其不可佩。\\
览察草木其犹未得兮,岂珵美之能当?\\
苏粪壤以充祎兮,谓申椒其不芳。\\
欲从灵氛之吉占兮,心犹豫而狐疑。\\
巫咸将夕降兮,怀椒糈而要之。\\
百神翳其备降兮,九疑缤其并迎。\\
皇剡剡其扬灵兮,告余以吉故。\\
曰:「勉升降以上下兮,求矩矱之所同。\\
汤、禹俨而求合兮,挚、咎繇而能调。\\
苟中情其好修兮,又何必用夫行媒?\\
说操筑于傅岩兮,武丁用而不疑。\\
吕望之鼓刀兮,遭周文而得举。\\
宁戚之讴歌兮,齐桓闻以该辅。\\
及年岁之未晏兮,时亦犹其未央。\\
恐鹈鴃之先鸣兮,使夫百草为之不芳。」\\
何琼佩之偃蹇兮,众薆然而蔽之。\\
惟此党人之不谅兮,恐嫉妒而折之。\\
时缤纷其变易兮,又何可以淹留?\\
兰芷变而不芳兮,荃蕙化而为茅。\\
何昔日之芳草兮,今直为此萧艾也?\\
岂其有他故兮,莫好修之害也!\\
余以兰为可恃兮,羌无实而容长。\\
委厥美以从俗兮,苟得列乎众芳。\\
椒专佞以慢慆兮,樧又欲充夫佩帏。\\
既干进而务入兮,又何芳之能祗?\\
固时俗之流从兮,又孰能无变化?\\
览椒兰其若兹兮,又况揭车与江离?\\
惟兹佩之可贵兮,委厥美而历兹。\\
芳菲菲而难亏兮,芬至今犹未沬。\\
和调度以自娱兮,聊浮游而求女。\\
及余饰之方壮兮,周流观乎上下。\\
灵氛既告余以吉占兮,历吉日乎吾将行。\\
折琼枝以为羞兮,精琼爢以为粻。\\
为余驾飞龙兮,杂瑶象以为车。\\
何离心之可同兮?吾将远逝以自疏。\\
邅吾道夫昆仑兮,路修远以周流。\\
扬云霓之晻蔼兮,鸣玉鸾之啾啾。\\
朝发轫于天津兮,夕余至乎西极。\\
凤皇翼其承旗兮,高翱翔之翼翼。\\
忽吾行此流沙兮,遵赤水而容与。\\
麾蛟龙使梁津兮,诏西皇使涉予。\\
路修远以多艰兮,腾众车使径待。\\
路不周以左转兮,指西海以为期。\\
屯余车其千乘兮,齐玉轪而并驰。\\
驾八龙之婉婉兮,载云旗之委蛇。\\
抑志而弭节兮,神高驰之邈邈。\\
奏《九歌》而舞《韶》兮,聊假日以媮乐。\\
陟升皇之赫戏兮,忽临睨夫旧乡。\\
仆夫悲余马怀兮,蜷局顾而不行。\\
乱曰:已矣哉!\\
国无人莫我知兮,又何怀乎故都!\\
既莫足与为美政兮,吾将从彭咸之所居!

\hypertarget{header-n11}{%
\subsection{九歌}\label{header-n11}}

\emph{屈原}

\hypertarget{header-n16}{%
\subsubsection{东皇太一}\label{header-n16}}

吉日兮辰良,穆将愉兮上皇;\\
抚长剑兮玉珥,璆锵鸣兮琳琅;\\
瑶席兮玉瑱,盍将把兮琼芳;\\
蕙肴蒸兮兰藉,奠桂酒兮椒浆;\\
扬枹兮拊鼓,疏缓节兮安歌;\\
陈竽瑟兮浩倡;\\
灵偃蹇兮姣服,芳菲菲兮满堂;\\
五音纷兮繁会,君欣欣兮乐康。

\hypertarget{header-n21}{%
\subsubsection{云中君}\label{header-n21}}

浴兰汤兮沐芳,华采衣兮若英;\\
灵连蜷兮既留,烂昭昭兮未央;\\
謇将憺兮寿宫,与日月兮齐光;\\
龙驾兮帝服,聊翱游兮周章;\\
灵皇皇兮既降,猋远举兮云中;\\
览冀洲兮有余,横四海兮焉穷;\\
思夫君兮太息,极劳心兮忡忡;

\hypertarget{header-n26}{%
\subsubsection{湘君}\label{header-n26}}

君不行兮夷犹,蹇谁留兮中洲;\\
美要眇兮宜修,沛吾乘兮桂舟;\\
令沅湘兮无波,使江水兮安流;\\
望夫君兮未来,吹参差兮谁思;\\
驾飞龙兮北征,邅吾道兮洞庭;\\
薜荔柏兮蕙绸,荪桡兮兰旌;\\
望涔阳兮极浦,横大江兮扬灵;\\
扬灵兮未极,女婵媛兮为余太息;\\
横流涕兮潺湲,隐思君兮陫侧;\\
桂棹兮兰枻,斵冰兮积雪;\\
采薜荔兮水中,搴芙蓉兮木末;\\
心不同兮媒劳,恩不甚兮轻绝;\\
石濑兮浅浅,飞龙兮翩翩;\\
交不忠兮怨长,期不信兮告余以不闲;\\
朝骋骛兮江皋,夕弭节兮北渚;\\
鸟次兮屋上,水周兮堂下;\\
捐余玦兮江中,遗余佩兮醴浦;\\
采芳洲兮杜若,将以遗兮下女;\\
时不可兮再得,聊逍遥兮容与。

\hypertarget{header-n31}{%
\subsubsection{湘夫人}\label{header-n31}}

帝子降兮北渚,目眇眇兮愁予;\\
袅袅兮秋风,洞庭波兮木叶下;\\
登白薠兮骋望,与佳期兮夕张;\\
鸟何萃兮苹中,罾何为兮木上?(苹 通:蘋)\\
沅有茝兮醴有兰,思公子兮未敢言;\\
荒忽兮远望,观流水兮潺湲;\\
麋何食兮庭中,蛟何为兮水裔;\\
朝驰余马兮江皋,夕济兮西澨;\\
闻佳人兮召余,将腾驾兮偕逝;\\
筑室兮水中,葺之兮荷盖;\\
荪壁兮紫坛,播芳椒兮成堂;\\
桂栋兮兰橑,辛夷楣兮药房;\\
罔薜荔兮为帷,擗蕙櫋兮既张;\\
白玉兮为镇,疏石兰兮为芳;\\
芷葺兮荷屋,缭之兮杜衡;\\
合百草兮实庭,建芳馨兮庑门;\\
九嶷缤兮并迎,灵之来兮如云;\\
捐余袂兮江中,遗余褋兮醴浦;\\
搴汀洲兮杜若,将以遗兮远者;\\
时不可兮骤得,聊逍遥兮容与!

\hypertarget{header-n36}{%
\subsubsection{大司命}\label{header-n36}}

广开兮天门,纷吾乘兮玄云;\\
令飘风兮先驱,使涷雨兮洒尘;\\
君回翔兮以下,逾空桑兮从女;\\
纷总总兮九州,何寿夭兮在予;\\
高飞兮安翔,乘清气兮御阴阳;\\
吾与君兮齐速,导帝之兮九坑;\\
灵衣兮被被,玉佩兮陆离;\\
一阴兮一阳,众莫知兮余所为;\\
折疏麻兮瑶华,将以遗兮离居;\\
老冉冉兮既极,不寖近兮愈疏;\\
乘龙兮辚辚,高驰兮冲天;\\
结桂枝兮延伫,羌愈思兮愁人;\\
愁人兮奈何,愿若今兮无亏;\\
固人命兮有当,孰离合兮何为?

\hypertarget{header-n41}{%
\subsubsection{少司命}\label{header-n41}}

秋兰兮麋芜,罗生兮堂下;\\
绿叶兮素华,芳菲菲兮袭予;\\
夫人兮自有美子,荪何以兮愁苦;\\
秋兰兮青青,绿叶兮紫茎;\\
满堂兮美人,忽独与余兮目成;\\
入不言兮出不辞,乘回风兮载云旗;\\
悲莫悲兮生别离,乐莫乐兮新相知;\\
荷衣兮蕙带,儵而来兮忽而逝;\\
夕宿兮帝郊,君谁须兮云之际;\\
与女沐兮咸池,曦女发兮阳之阿;\\
望美人兮未来,临风怳兮浩歌;\\
孔盖兮翠旌,登九天兮抚彗星;\\
竦长剑兮拥幼艾,荪独宜兮为民正。

\hypertarget{header-n46}{%
\subsubsection{东君}\label{header-n46}}

暾将出兮东方,照吾槛兮扶桑;\\
抚余马兮安驱,夜皎皎兮既明;\\
驾龙輈兮乘雷,载云旗兮委蛇;\\
长太息兮将上,心低徊兮顾怀;\\
羌声色兮娱人,观者儋兮忘归;\\
縆瑟兮交鼓,萧钟兮瑶簴;\\
鸣篪兮吹竽,思灵保兮贤姱;\\
翾飞兮翠曾,展诗兮会舞;\\
应律兮合节,灵之来兮敝日;\\
青云衣兮白霓裳,举长矢兮射天狼;\\
操余弧兮反沦降,援北斗兮酌桂浆;\\
撰余辔兮高驰翔,杳冥冥兮以东行。

\hypertarget{header-n51}{%
\subsubsection{河伯}\label{header-n51}}

与女游兮九河,冲风起兮水扬波;\\
乘水车兮荷盖,驾两龙兮骖螭;\\
登昆仑兮四望,心飞扬兮浩荡;\\
日将暮兮怅忘归,惟极浦兮寤怀;\\
鱼鳞屋兮龙堂,紫贝阙兮珠宫;\\
灵何惟兮水中;\\
乘白鼋兮逐文鱼,与女游兮河之渚;\\
流澌纷兮将来下;\\
子交手兮东行,送美人兮南浦;\\
波滔滔兮来迎,鱼鳞鳞兮媵予。

\hypertarget{header-n56}{%
\subsubsection{山鬼}\label{header-n56}}

若有人兮山之阿,被薜荔兮带女萝;\\
既含睇兮又宜笑,子慕予兮善窈窕;\\
乘赤豹兮从文狸,辛夷车兮结桂旗;\\
被石兰兮带杜衡,折芳馨兮遗所思;\\
余处幽篁兮终不见天,路险难兮独后来;\\
表独立兮山之上,云容容兮而在下;\\
杳冥冥兮羌昼晦,东风飘兮神灵雨;\\
留灵修兮憺忘归,岁既晏兮孰华予;\\
采三秀兮于山间,石磊磊兮葛蔓蔓;\\
怨公子兮怅忘归,君思我兮不得闲;\\
山中人兮芳杜若,饮石泉兮荫松柏;\\
君思我兮然疑作;\\
雷填填兮雨冥冥,猿啾啾兮狖夜鸣;\\
风飒飒兮木萧萧,思公子兮徒离忧。

\hypertarget{header-n61}{%
\subsubsection{国殇}\label{header-n61}}

操吴戈兮被犀甲,车错毂兮短兵接;\\
旌蔽日兮敌若云,矢交坠兮士争先;\\
凌余阵兮躐余行,左骖殪兮右刃伤;\\
霾两轮兮絷四马,援玉枹兮击鸣鼓;\\
天时怼兮威灵怒,严杀尽兮弃原野;\\
出不入兮往不反,平原忽兮路超远;\\
带长剑兮挟秦弓,首身离兮心不惩;\\
诚既勇兮又以武,终刚强兮不可凌;\\
身既死兮神以灵,魂魄毅兮为鬼雄。

\hypertarget{header-n66}{%
\subsubsection{礼魂}\label{header-n66}}

成礼兮会鼓,传芭兮代舞;\\
姱女倡兮容与;\\
春兰兮秋菊,长无绝兮终古。

\hypertarget{header-n70}{%
\subsection{天问}\label{header-n70}}

\emph{屈原}

曰:遂古之初,谁传道之?\\
上下未形,何由考之?\\
冥昭瞢暗,谁能极之?\\
冯翼惟象,何以识之?\\
明明暗暗,惟时何为?\\
阴阳三合,何本何化?\\
圜则九重,孰营度之?\\
惟兹何功,孰初作之?\\
斡维焉系,天极焉加?\\
八柱何当,东南何亏?\\
九天之际,安放安属?\\
隅隈多有,谁知其数?\\
天何所沓?十二焉分?\\
日月安属?列星安陈?\\
出自汤谷,次于蒙汜。\\
自明及晦,所行几里?\\
夜光何德,死则又育?\\
厥利维何,而顾菟在腹?\\
女岐无合,夫焉取九子?\\
伯强何处?惠气安在?\\
何阖而晦?何开而明?\\
角宿未旦,曜灵安藏?\\
不任汩鸿,师何以尚之?\\
佥曰``何忧,何不课而行之?''\\
鸱龟曳衔,鲧何听焉?\\
顺欲成功,帝何刑焉?\\
永遏在羽山,夫何三年不施?\\
伯禹愎鲧,夫何以变化?\\
纂就前绪,遂成考功。\\
何续初继业,而厥谋不同?\\
洪泉极深,何以窴之?\\
地方九则,何以坟之?\\
河海应龙?何尽何历?\\
鲧何所营?禹何所成?\\
康回冯怒,墬何故以东南倾?\\
九州安错?川谷何洿?\\
东流不溢,孰知其故?\\
东西南北,其修孰多?\\
南北顺椭,其衍几何?\\
昆仑悬圃,其尻安在?\\
增城九重,其高几里?\\
四方之门,其谁从焉?\\
西北辟启,何气通焉?\\
日安不到?烛龙何照?\\
羲和之未扬,若华何光?\\
何所冬暖?何所夏寒?\\
焉有石林?何兽能言?\\
焉有虬龙,负熊以游?\\
雄虺九首,鯈忽焉在?\\
何所不死?长人何守?\\
靡蓱九衢,枲华安居?\\
灵蛇吞象,厥大何如?\\
黑水玄趾,三危安在?\\
延年不死,寿何所止?\\
鲮鱼何所?鬿堆焉处?\\
羿焉彃日?乌焉解羽?\\
禹之力献功,降省下土四方。\\
焉得彼嵞山女,而通之於台桑?\\
闵妃匹合,厥身是继。\\
胡维嗜不同味,而快鼌饱?\\
启代益作后,卒然离蠥。\\
何启惟忧,而能拘是达?\\
皆归射鞫,而无害厥躬。\\
何后益作革,而禹播降?\\
启棘宾商,《九辨》《九歌》。\\
何勤子屠母,而死分竟地?\\
帝降夷羿,革孽夏民。\\
胡射夫河伯,而妻彼雒嫔?\\
冯珧利决,封豨是射。\\
何献蒸肉之膏,而后帝不若?\\
浞娶纯狐,眩妻爰谋。\\
何羿之射革,而交吞揆之?\\
阻穷西征,岩何越焉?\\
化而为黄熊,巫何活焉?\\
咸播秬黍,莆雚是营。\\
何由并投,而鲧疾修盈?\\
白蜺婴茀,胡为此堂?\\
安得夫良药,不能固臧?\\
天式从横,阳离爰死。\\
大鸟何鸣,夫焉丧厥体?\\
蓱号起雨,何以兴之?\\
撰体协胁,鹿何膺之?\\
鳌戴山抃,何以安之?\\
释舟陵行,何之迁之?\\
惟浇在户,何求于嫂?\\
何少康逐犬,而颠陨厥首?\\
女歧缝裳,而馆同爰止。\\
何颠易厥首,而亲以逢殆?\\
汤谋易旅,何以厚之?\\
覆舟斟寻,何道取之?\\
桀伐蒙山,何所得焉?\\
妺嬉何肆,汤何殛焉?\\
舜闵在家,父何以鳏?\\
尧不姚告,二女何亲?\\
厥萌在初,何所亿焉?\\
璜台十成,谁所极焉?\\
登立为帝,孰道尚之?\\
女娲有体,孰制匠之?\\
舜服厥弟,终然为害。\\
何肆犬豕,而厥身不危败?\\
吴获迄古,南岳是止。\\
孰期去斯,得两男子?\\
缘鹄饰玉,后帝是飨。\\
何承谋夏桀,终以灭丧?\\
帝乃降观,下逢伊挚。\\
何条放致罚,而黎服大说?\\
简狄在台,喾何宜?\\
玄鸟致贻,女何喜?\\
该秉季德,厥父是臧。\\
胡终弊于有扈,牧夫牛羊?\\
干协时舞,何以怀之?\\
平胁曼肤,何以肥之?\\
有扈牧竖,云何而逢?\\
击床先出,其命何从?\\
恒秉季德,焉得夫朴牛?\\
何往营班禄,不但还来?\\
昏微循迹,有狄不宁。\\
何繁鸟萃棘,负子肆情?\\
眩弟并淫,危害厥兄。\\
何变化以作诈,而后嗣逢长?\\
成汤东巡,有莘爰极。\\
何乞彼小臣,而吉妃是得?\\
水滨之木,得彼小子。\\
夫何恶之,媵有莘之妇?\\
汤出重泉,夫何辠尤?\\
不胜心伐帝,夫谁使挑之?\\
会朝争盟,何践吾期?\\
苍鸟群飞,孰使萃之?\\
列击纣躬,叔旦不嘉。\\
何亲揆发足,周之命以咨嗟?\\
授殷天下,其位安施?\\
反成乃亡,其罪伊何?\\
争遣伐器,何以行之?\\
并驱击翼,何以将之?\\
昭后成游,南土爰底。\\
厥利惟何,逢彼白雉?\\
穆王巧梅,夫何为周流?\\
环理天下,夫何索求?\\
妖夫曳炫,何号于市?\\
周幽谁诛?焉得夫褒姒?\\
天命反侧,何罚何佑?\\
齐桓九会,卒然身杀。\\
彼王纣之躬,孰使乱惑?\\
何恶辅弼,谗谄是服?\\
比干何逆,而抑沈之?\\
雷开阿顺,而赐封之?\\
何圣人之一德,卒其异方?\\
梅伯受醢,箕子详狂?\\
稷维元子,帝何竺之?\\
投之于冰上,鸟何燠之?\\
何冯弓挟矢,殊能将之?\\
既惊帝切激,何逢长之?\\
伯昌号衰,秉鞭作牧。\\
何令彻彼岐社,命有殷国?\\
迁藏就岐,何能依?\\
殷有惑妇,何所讥?\\
受赐兹醢,西伯上告。\\
何亲就上帝罚,殷之命以不救?\\
师望在肆,昌何识?\\
鼓刀扬声,后何喜?\\
武发杀殷,何所悒?\\
载尸集战,何所急?\\
伯林雉经,维其何故?\\
何感天抑墬,夫谁畏惧?\\
皇天集命,惟何戒之?\\
受礼天下,又使至代之?\\
初汤臣挚,后兹承辅。\\
何卒官汤,尊食宗绪?\\
勋阖梦生,少离散亡。\\
何壮武历,能流厥严?\\
彭铿斟雉,帝何飨?\\
受寿永多,夫何久长?\\
中央共牧,后何怒?\\
蜂蛾微命,力何固?\\
惊女采薇,鹿何佑?\\
北至回水,萃何喜?\\
兄有噬犬,弟何欲?\\
易之以百两,卒无禄?\\
薄暮雷电,归何忧?\\
厥严不奉,帝何求?\\
伏匿穴处,爰何云?\\
荆勋作师,夫何长?\\
悟过改更,我又何言?\\
吴光争国,久余是胜。\\
何环穿自闾社丘陵,爰出子文?\\
吾告堵敖以不长。\\
何试上自予,忠名弥彰?

\hypertarget{header-n75}{%
\subsection{九章}\label{header-n75}}

\emph{屈原}

\hypertarget{header-n80}{%
\subsubsection{惜诵}\label{header-n80}}

惜诵以致愍兮,发愤以抒情。\\
所作忠而言之兮,指苍天以为正。\\
令五帝使折中兮,戒六神与向服。\\
俾山川以备御兮,命咎繇使听直。\\
竭忠诚而事君兮,反离群而赘肬。\\
忘儇媚以背众兮,待明君其知之。\\
言与行其可迹兮,情与貌其不变。\\
故相臣莫若君兮,所以证之不远。\\
吾谊先君而后身兮,羌众人之所仇也。\\
专惟君而无他兮,又众兆之所雠也。\\
壹心而不豫兮,羌无可保也。\\
疾亲君而无他兮,有招祸之道也。\\
思君其莫我忠兮,忽忘身之贱贫。\\
事君而不贰兮,迷不知宠之门。\\
患何罪以遇罚兮,亦非余之所志也。\\
行不群以巅越兮,又众兆之所咍也。\\
纷逢尤以离谤兮,謇不可释也。\\
情沉抑而不达兮,又蔽而莫之白也。\\
心郁邑余侘傺兮,又莫察余之中情。\\
固烦言不可结而诒兮,愿陈志而无路。\\
退静默而莫余知兮,进号呼又莫吾闻。\\
申侘傺之烦惑兮,中闷瞀之忳忳。\\
昔余梦登天兮,魂中道而无杭。\\
吾使厉神占之兮,曰有志极而无旁。\\
终危独以离异兮,曰君可思而不可恃。\\
故众口其铄金兮,初若是而逢殆。\\
惩于羹者而吹齑兮,何不变此志也?\\
欲释阶而登天兮,犹有曩之态也。\\
众骇遽以离心兮,又何以为此伴也?\\
同极而异路兮,又何以为此援也?\\
晋申生之孝子兮,父信谗而不好。\\
行婞直而不豫兮,鲧功用而不就。\\
吾闻作忠以造怨兮,忽谓之过言。\\
九折臂而成医兮,吾至今而知其信然。\\
矰弋机而在上兮,罻罗张而在下。\\
设张辟以娱君兮,愿侧身而无所。\\
欲儃徊以干傺兮,恐重患而离尤。\\
欲高飞而远集兮,君罔谓汝何之?\\
欲横奔而失路兮,盖志坚而不忍。\\
背膺牉以交痛兮,心郁结而纡轸。\\
擣木兰以矫蕙兮,糳申椒以为粮。\\
播江离与滋菊兮,愿春日以为糗芳。\\
恐情质之不信兮,故重著以自明。\\
矫兹媚以私处兮,愿曾思而远身。

\hypertarget{header-n85}{%
\subsubsection{涉江}\label{header-n85}}

余幼好此奇服兮,年既老而不衰。\\
带长铗之陆离兮,冠切云之崔嵬,\\
被明月兮佩宝璐。\\
世混浊而莫余知兮,吾方高驰而不顾。\\
驾青虬兮骖白螭,吾与重华游兮瑶之圃。\\
登昆仑兮食玉英,与天地兮同寿,\\
与日月兮同光。\\
哀南夷之莫吾知兮,旦余济乎江湘。\\
乘鄂渚而反顾兮,欸秋冬之绪风。\\
步余马兮山皋,邸余车兮方林。\\
乘舲船余上沅兮,齐吴榜以击汰。\\
船容与而不进兮,淹回水而疑滞。\\
朝发枉渚兮,夕宿辰阳。\\
苟余心其端直兮,虽僻远之何伤。\\
入溆浦余儃徊兮,迷不知吾所如。\\
深林杳以冥冥兮,乃猿狖之所居。\\
山峻高以蔽日兮,下幽晦以多雨。\\
霰雪纷其无垠兮,云霏霏而承宇。\\
哀吾生之无乐兮,幽独处乎山中。\\
吾不能变心而从俗兮,固将愁苦而终穷。\\
接舆髡首兮,桑扈臝行。\\
忠不必用兮,贤不必以。\\
伍子逢殃兮,比干菹醢。\\
与前世而皆然兮,吾又何怨乎今之人!\\
余将董道而不豫兮,固将重昏而终身!\\
乱曰:鸾鸟凤皇,日以远兮。\\
燕雀乌鹊,巢堂坛兮。\\
露申辛夷,死林薄兮。\\
腥臊并御,芳不得薄兮。\\
阴阳易位,时不当兮。\\
怀信佗傺,忽乎吾将行兮!

\hypertarget{header-n90}{%
\subsubsection{哀郢}\label{header-n90}}

皇天之不纯命兮,何百姓之震愆?\\
民离散而相失兮,方仲春而东迁。\\
去故乡而就远兮,遵江夏以流亡。\\
出国门而轸怀兮,甲之鼂吾以行。\\
发郢都而去闾兮,怊荒忽其焉极?\\
楫齐扬以容与兮,哀见君而不再得。\\
望长楸而太息兮,涕淫淫其若霰。\\
过夏首而西浮兮,顾龙门而不见。\\
心婵媛而伤怀兮,眇不知其所蹠。\\
顺风波以从流兮,焉洋洋而为客。\\
凌阳侯之汜滥兮,忽翱翔之焉薄。\\
心絓结而不解兮,思蹇产而不释。\\
将运舟而下浮兮,上洞庭而下江。\\
去终古之所居兮,今逍遥而来东。\\
羌灵魂之欲归兮,何须臾而忘反。\\
背夏浦而西思兮,哀故都之日远。\\
登大坟以远望兮,聊以舒吾忧心。\\
哀州土之平乐兮,悲江介之遗风。\\
当陵阳之焉至兮,淼南渡之焉如?\\
曾不知夏之为丘兮,孰两东门之可芜?\\
心不怡之长久兮,忧与愁其相接。\\
惟郢路之辽远兮,江与夏之不可涉。\\
忽若不信兮,至今九年而不复。\\
惨郁郁而不通兮,蹇侘傺而含慼。\\
外承欢之汋约兮,谌荏弱而难持。\\
忠湛湛而愿进兮,妒被离而鄣之。\\
尧舜之抗行兮,瞭杳杳而薄天。\\
众谗人之嫉妒兮,被以不慈之伪名。\\
憎愠惀之修美兮,好夫人之慷慨。\\
众踥蹀而日进兮,美超远而逾迈。\\
乱曰:\\
曼余目以流观兮,冀一反之何时?\\
鸟飞反故乡兮,狐死必首丘。\\
信非吾罪而弃逐兮,何日夜而忘之?

\hypertarget{header-n95}{%
\subsubsection{抽思}\label{header-n95}}

心郁郁之忧思兮,独永叹乎增伤。\\
思蹇产之不释兮,曼遭夜之方长。\\
悲秋风之动容兮,何回极之浮浮。\\
数惟荪之多怒兮,伤余心之忧忧。\\
愿摇起而横奔兮,览民尤以自镇。\\
结微情以陈词兮,矫以遗夫美人。\\
昔君与我诚言兮,曰黄昏以为期。\\
羌中道而回畔兮,反既有此他志。\\
憍吾以其美好兮,览余以其修姱。\\
与余言而不信兮,盖为余而造怒。\\
愿承閒而自察兮,心震悼而不敢。\\
悲夷犹而冀进兮,心怛伤之憺憺。\\
兹历情以陈辞兮,荪详聋而不闻。\\
固切人之不媚兮,众果以我为患。\\
初吾所陈之耿著兮,岂至今其庸亡?\\
何独乐斯之謇謇兮?愿荪美之可光。\\
望三王以为像兮,指彭咸以为仪。\\
夫何极而不至兮,故远闻而难亏。\\
善不由外来兮,名不可以虚作。\\
孰无施而有报兮,孰不实而有获?\\
少歌曰:\\
与美人抽思兮,并日夜而无正。\\
憍吾以其美好兮,敖朕辞而不听。\\
倡曰:有鸟自南兮,来集汉北。\\
好姱佳丽兮,牉独处此异域。\\
惸茕独而不群兮,又无良媒在其侧。\\
道卓远而日忘兮,愿自申而不得。\\
望北山而流涕兮,临流水而太息。\\
望孟夏之短夜兮,何晦明之若岁?\\
惟郢路之辽远兮,魂一夕而九逝。\\
曾不知路之曲直兮,南指月与列星。\\
愿径逝而未得兮,魂识路之营营。\\
何灵魂之信直兮,人之心不与吾心同!\\
理弱而媒不通兮,尚不知余之从容。\\
乱曰:\\
长濑湍流,溯江潭兮。\\
狂顾南行,聊以娱心兮。\\
轸石崴嵬,蹇吾愿兮。\\
超回志度,行隐进兮。\\
低徊夷犹,宿北姑兮。\\
烦冤瞀容,实沛徂兮。\\
愁叹苦神,灵遥思兮。\\
路远处幽,又无行媒兮。\\
道思作颂,聊以自救兮。\\
忧心不遂,斯言谁告兮。

\hypertarget{header-n100}{%
\subsubsection{怀沙}\label{header-n100}}

滔滔孟夏兮,草木莽莽。\\
伤怀永哀兮,汩徂南土。\\
眴兮杳杳,孔静幽默。\\
郁结纡轸兮,离慜而长鞠。\\
抚情效志兮,冤屈而自抑。\\
刓方以为圜兮,常度未替。\\
易初本迪兮,君子所鄙。\\
章画志墨兮,前图未改。\\
内厚质正兮,大人所盛。\\
巧倕不斲兮,孰察其拨正。\\
玄文处幽兮,矇瞍谓之不章;\\
离娄微睇兮,瞽以为无明。\\
变白以为黑兮,倒上以为下。\\
凤皇在笯兮,鸡鹜翔舞。\\
同糅玉石兮,一概而相量。\\
夫惟党人之鄙固兮,羌不知余之所臧。\\
任重载盛兮,陷滞而不济。\\
怀瑾握瑜兮,穷不知所示。\\
邑犬之群吠兮,吠所怪也。\\
非俊疑杰兮,固庸态也。\\
文质疏内兮,众不知余之异采。\\
材朴委积兮,莫知余之所有。\\
重仁袭义兮,谨厚以为丰。\\
重华不可遌兮,孰知余之从容!\\
古固有不并兮,岂知其何故也?\\
汤禹久远兮,邈而不可慕也?\\
惩违改忿兮,抑心而自强。\\
离慜而不迁兮,愿志之有像。\\
进路北次兮,日昧昧其将暮。\\
舒忧娱哀兮,限之以大故。\\
乱曰:\\
浩浩沅湘,分流汩兮。\\
修路幽蔽,道远忽兮。\\
怀质抱情,独无匹兮。\\
伯乐既没,骥焉程兮。\\
民生禀命,各有所错兮。\\
定心广志,余何所畏惧兮?\\
曾伤爰哀,永叹喟兮。\\
世浑浊莫吾知,人心不可谓兮。\\
知死不可让,愿勿爱兮。\\
明告君子,吾将以为类兮。

\hypertarget{header-n105}{%
\subsubsection{思美人}\label{header-n105}}

思美人兮,揽涕而竚眙。\\
媒绝路阻兮,言不可结而诒。\\
蹇蹇之烦冤兮,陷滞而不发。\\
申旦以舒中情兮,志沉菀而莫达。\\
愿寄言于浮云兮,遇丰隆而不将。\\
因归鸟而致辞兮,羌迅高而难当。\\
高辛之灵盛兮,遭玄鸟而致诒。\\
欲变节以从俗兮,媿易初而屈志。\\
独历年而离愍兮,羌凭心犹未化。\\
宁隐闵而寿考兮,何变易之可为!\\
知前辙之不遂兮,未改此度。\\
车既覆而马颠兮,蹇独怀此异路。\\
勒骐骥而更驾兮,造父为我操之,\\
迁逡次而勿驱兮,聊假日以须是时。\\
指嶓冢之西隈兮,与纁黄以为期。\\
开春发岁兮,白日出之悠悠。\\
吾将荡志而愉乐兮,遵江夏以娱忧。\\
揽大薄之芳茝兮,搴长洲之宿莽。\\
惜吾不及古人兮,吾谁与玩此芳草?\\
解萹薄与杂菜兮,备以为交佩。\\
佩缤纷以缭转兮,遂萎绝而离异。\\
吾且儃徊以娱忧兮,观南人之变态。\\
窃快在中心兮,扬厥凭而不竢。\\
芳与泽其杂糅兮,羌芳华自中出。\\
纷郁郁其远蒸兮,满内而外扬。\\
情与质信可保兮,羌居蔽而闻章。\\
令薜荔以为理兮,惮举趾而缘木。\\
因芙蓉而为媒兮,惮褰裳而濡足。\\
登高吾不说兮,入下吾不能。\\
固朕形之不服兮,然容与而狐疑。\\
广遂前画兮,未改此度也。\\
命则处幽吾将罢兮,愿及白日之未暮也。\\
独茕茕而南行兮,思彭咸之故也。

\hypertarget{header-n110}{%
\subsubsection{惜往日}\label{header-n110}}

惜往日之曾信兮,受命诏以昭时。\\
奉先功以照下兮,明法度之嫌疑。\\
国富强而法立兮,属贞臣而日竢。\\
秘密事之载心兮,虽过失犹弗治。\\
心纯庞而不泄兮,遭谗人而嫉之。\\
君含怒而待臣兮,不清澈其然否。\\
蔽晦君之聪明兮,虚惑误又以欺。\\
弗参验以考实兮,远迁臣而弗思。\\
信谗谀之浑浊兮,盛气志而过之。\\
何贞臣之无罪兮,被离谤而见尤。\\
惭光景之诚信兮,身幽隐而备之。\\
临沅湘之玄渊兮,遂自忍而沉流。\\
卒没身而绝名兮,惜壅君之不昭。\\
君无度而弗察兮,使芳草为薮幽。\\
焉舒情而抽信兮,恬死亡而不聊。\\
独障壅而弊隐兮,使贞臣为无由。\\
闻百里之为虏兮,伊尹烹于庖厨。\\
吕望屠于朝歌兮,宁戚歌而饭牛。\\
不逢汤武与桓缪兮,世孰云而知之。\\
吴信谗而弗味兮,子胥死而后忧。\\
介子忠而立枯兮,文君寤而追求。\\
封介山而为之禁兮,报大德之优游。\\
思久故之亲身兮,因缟素而哭之。\\
或忠信而死节兮,或訑谩而不疑。\\
弗省察而按实兮,听谗人之虚辞。\\
芳与泽其杂糅兮,孰申旦而别之?\\
何芳草之早殀兮,微霜降而下戒。\\
谅聪不明而蔽壅兮,使谗谀而日得。\\
自前世之嫉贤兮,谓蕙若其不可佩。\\
妒佳冶之芬芳兮,嫫母姣而自好。\\
虽有西施之美容兮,谗妒入以自代。\\
愿陈情以白行兮,得罪过之不意。\\
情冤见之日明兮,如列宿之错置。\\
乘骐骥而驰骋兮,无辔衔而自载;\\
乘泛泭以下流兮,无舟楫而自备。\\
背法度而心治兮,辟与此其无异。\\
宁溘死而流亡兮,恐祸殃之有再。\\
不毕辞而赴渊兮,惜壅君之不识。

\hypertarget{header-n115}{%
\subsubsection{橘颂}\label{header-n115}}

后皇嘉树,橘徕服兮。\\
受命不迁,生南国兮。\\
深固难徙,更壹志兮。\\
绿叶素荣,纷其可喜兮。\\
曾枝剡棘,圆果抟兮。\\
青黄杂糅,文章烂兮。\\
精色内白,类任道兮。\\
纷緼宜修,姱而不丑兮。\\
嗟尔幼志,有以异兮。\\
独立不迁,岂不可喜兮?\\
深固难徙,廓其无求兮。\\
苏世独立,横而不流兮。\\
闭心自慎,终不失过兮。\\
秉德无私,参天地兮。\\
愿岁并谢,与长友兮。\\
淑离不淫,梗其有理兮。\\
年岁虽少,可师长兮。\\
行比伯夷,置以为像兮。

\hypertarget{header-n120}{%
\subsubsection{悲回风}\label{header-n120}}

悲回风之摇蕙兮,心冤结而内伤。\\
物有微而陨性兮,声有隐而先倡。\\
夫何彭咸之造思兮,暨志介而不忘!\\
万变其情岂可盖兮,孰虚伪之可长?\\
鸟兽鸣以号群兮,草苴比而不芳。\\
鱼葺鳞以自别兮,蛟龙隐其文章。\\
故荼荠不同亩兮,兰茝幽而独芳。\\
惟佳人之永都兮,更统世以自贶。\\
眇远志之所及兮,怜浮云之相羊。\\
介眇志之所惑兮,窃赋诗之所明。\\
惟佳人之独怀兮,折若椒以自处。\\
曾歔欷之嗟嗟兮,独隐伏而思虑。\\
涕泣交而凄凄兮,思不眠以至曙。\\
终长夜之曼曼兮,掩此哀而不去。\\
寤从容以周流兮,聊逍遥以自恃。\\
伤太息之愍怜兮,气于邑而不可止。\\
糺思心以为纕兮,编愁苦以为膺。\\
折若木以弊光兮,随飘风之所仍。\\
存彷佛而不见兮,心踊跃其若汤。\\
抚珮衽以案志兮,超惘惘而遂行。\\
岁曶曶其若颓兮,时亦冉冉而将至。\\
薠蘅槁而节离兮,芳以歇而不比。\\
怜思心之不可惩兮,证此言之不可聊。\\
宁溘死而流亡兮,不忍此心之常愁。\\
孤子吟而抆泪兮,放子出而不还。\\
孰能思而不隐兮,照彭咸之所闻。\\
登石峦以远望兮,路眇眇之默默。\\
入景响之无应兮,闻省想而不可得。\\
愁郁郁之无快兮,居戚戚而不可解。\\
心鞿羁而不开兮,气缭转而自缔。\\
穆眇眇之无垠兮,莽芒芒之无仪。\\
声有隐而相感兮,物有纯而不可为。\\
邈漫漫之不可量兮,缥绵绵之不可纡。\\
愁悄悄之常悲兮,翩冥冥之不可娱。\\
凌大波而流风兮,讬彭咸之所居。\\
上高岩之峭岸兮,处雌蜺之标颠。\\
据青冥而摅虹兮,遂儵忽而扪天。\\
吸湛露之浮源兮,漱凝霜之雰雰。\\
依风穴以自息兮,忽倾寤以婵媛。\\
冯昆仑以澂雾兮,隐渂山以清江。\\
惮涌湍之礚礚兮,听波声之汹汹。\\
纷容容之无经兮,罔芒芒之无纪。\\
轧洋洋之无从兮,驰委移之焉止?\\
漂翻翻其上下兮,翼遥遥其左右。\\
氾潏潏其前后兮,伴张驰之信期。\\
观炎气之相仍兮,窥烟液之所积。\\
悲霜雪之俱下兮,听潮水之相击。\\
借光景以往来兮,施黄棘之枉策。\\
求介子之所存兮,见伯夷之放迹。\\
心调度而弗去兮,刻著志之无适。\\
曰吾怨往昔之所冀兮,悼来者之悐悐。\\
浮江淮而入海兮,从子胥而自适。\\
望大河之洲渚兮,悲申徒之抗迹。\\
骤谏君而不听兮,重任石之何益?\\
心絓结而不解兮,思蹇产而不释。

\hypertarget{header-n124}{%
\subsection{远游}\label{header-n124}}

\emph{屈原}

悲时俗之迫阨兮,愿轻举而远游。\\
质菲薄而无因兮,焉讬乘而上浮?\\
遭沈浊而污秽兮,独郁结其谁语!\\
夜耿耿而不寐兮,魂营营而至曙。\\
惟天地之无穷兮,哀人生之长勤。\\
往者余弗及兮,来者吾不闻。\\
步徙倚而遥思兮,怊惝怳而乖怀。\\
意荒忽而流荡兮,心愁悽而增悲。\\
神倏忽而不反兮,形枯槁而独留。\\
内惟省以操端兮,求正气之所由。\\
漠虚静以恬愉兮,澹无为而自得。

闻赤松之清尘兮,愿承风乎遗则。\\
贵真人之休德兮,美往世之登仙;\\
与化去而不见兮,名声著而日延。\\
奇傅说之讬辰星兮,羡韩众之得一。\\
形穆穆以浸远兮,离人群而遁逸。\\
因气变而遂曾举兮,忽神奔而鬼怪。\\
时仿佛以遥见兮,精晈晈以往来。\\
超氛埃而淑邮兮,终不反其故都。\\
免众患而不惧兮,世莫知其所如。

恐天时之代序兮,耀灵晔而西征。\\
微霜降而下沦兮,悼芳草之先蘦。\\
聊仿佯而逍遥兮,永历年而无成。\\
谁可与玩斯遗芳兮?长向风而舒情。\\
高阳邈以远兮,余将焉所程?

重曰:\\
春秋忽其不淹兮,奚久留此故居。\\
轩辕不可攀援兮,吾将从王乔而娱戏。\\
餐六气而饮沆瀣兮,漱正阳而含朝霞。\\
保神明之清澄兮,精气入而麤秽除。\\
顺凯风以从游兮,至南巢而壹息。\\
见王子而宿之兮,审壹气之和德。

曰``道可受兮,不可传;\\
其小无内兮,其大无垠。\\
毋滑而魂兮,彼将自然;\\
壹气孔神兮,于中夜存。\\
虚以待之存,无为之先;\\
庶类以成兮,此德之门。''

闻至贵而遂徂兮,忽乎吾将行。\\
仍羽人于丹丘,留不死之旧乡。\\
朝濯发于汤谷兮,夕晞余身兮九阳。\\
吸飞泉之微液兮,怀琬琰之华英。\\
玉色頩以脕颜兮,精醇粹而始壮。\\
质销铄以汋约兮,神要眇以淫放。\\
嘉南州之炎德兮,丽桂树之冬荣;\\
山萧条而无兽兮,野寂漠其无人。\\
载营魄而登霞兮,掩浮云而上征。\\
命天阍其开关兮,排阊阖而望予。\\
召丰隆使先导兮,问太微之所居。\\
集重阳入帝宫兮,造旬始而观清都。

朝发轫于太仪兮,夕始临乎于微闾。\\
屯余车之万乘兮,纷容与而并驰。\\
驾八龙之婉婉兮,载云旗之逶蛇。\\
建雄虹之采旄兮,五色杂而炫耀。\\
服偃蹇以低昂兮,骖连蜷以骄骜。\\
骑胶葛以杂乱兮,斑漫衍而方行。\\
撰余辔而正策兮,吾将过乎句芒。\\
历太皓以右转兮,前飞廉以启路。\\
阳杲杲其未光兮,凌天地以径度。\\
风伯为余先驱兮,氛埃辟而清凉。\\
凤凰翼其承旂兮,遇蓐收乎西皇。\\
揽慧星以为旍兮,举斗柄以为麾。\\
叛陆离其上下兮,游惊雾之流波。\\
时暧曃其曭莽兮,召玄武而奔属。\\
后文昌使掌行兮,选署众神以并轂。\\
路漫漫其修远兮,徐弭节而高厉。

左雨师使径侍兮,右雷公以为卫。\\
欲度世以忘归兮,意姿睢以抯挢。\\
内欣欣而自美兮,聊媮娱以淫乐。\\
涉青云以汎滥游兮,忽临睨夫旧乡。\\
仆夫怀余心悲兮,边马顾而不行。\\
思旧故以想象兮,长太息而掩涕。\\
汜容与而遐举兮,聊抑志而自弭。\\
指炎神而直驰兮,吾将往乎南疑。

览方外之荒忽兮,沛罔瀁而自浮。\\
祝融戒而跸御兮,腾告鸾鸟迎宓妃。\\
张咸池奏承云兮,二女御九韶歌。\\
使湘灵鼓瑟兮,令海若舞冯夷。\\
玄螭虫象并出进兮,形蟉虯而逶蛇。\\
雌蜺便娟以增挠兮,鸾鸟轩翥而翔飞。\\
音乐博衍无终极兮,焉乃逝以徘徊。\\
舒并节以驰骛兮,逴绝垠乎寒门。\\
轶迅风于清源兮,从颛顼乎增冰。\\
历玄冥以邪径兮,乘间维以反顾。\\
召黔赢而见之兮,为余先乎平路。\\
经营四方兮,周流六漠。\\
上至列缺兮,降望大壑。\\
下峥嵘而无地兮,上寥廓而无天。\\
视倏忽而无见兮,听惝恍而无闻。\\
超无为以至清兮,与泰初而为邻。

\hypertarget{header-n137}{%
\subsection{卜居}\label{header-n137}}

\emph{屈原}

屈原既放,三年不得复见。竭知尽忠而蔽障于谗。心烦虑乱,不知所从。乃往见太卜郑詹尹曰:``余有所疑,愿因先生决之。''詹尹乃端策拂龟,曰:``君将何以教之?''

屈原曰:``吾宁悃悃款款,朴以忠乎,将送往劳来,斯无穷乎?

``宁诛锄草茅以力耕乎,将游大人以成名乎?宁正言不讳以危身乎,将从俗富贵以偷生乎?宁超然高举以保真乎,将哫訾栗斯,喔咿儒儿,以事妇人乎?宁廉洁正直以自清乎,将突梯滑稽,如脂如韦,以洁楹乎?

``宁昂昂若千里之驹乎,将泛泛若水中之凫,与波上下,偷以全吾躯乎?宁与骐骥亢轭乎,将随驽马之迹乎?宁与黄鹄比翼乎,将与鸡鹜争食乎?

``此孰吉孰凶?何去何从?

``世溷浊而不清:蝉翼为重,千钧为轻;黄钟毁弃,瓦釜雷鸣;谗人高张,贤士无名。吁嗟默默兮,谁知吾之廉贞!''

詹尹乃释策而谢曰:``夫尺有所短,寸有所长;物有所不足,智有所不明;数有所不逮,神有所不通。用君之心,行君之意。龟策诚不能知此事。''

\hypertarget{header-n148}{%
\subsection{渔父}\label{header-n148}}

\emph{屈原}

屈原既放,游于江潭,行吟泽畔,颜色憔悴,形容枯槁。渔父见而问之曰:``子非三闾大夫与?何故至于斯?''屈原曰:``举世皆浊我独清,众人皆醉我独醒,是以见放。''

渔父曰:``圣人不凝滞于物,而能与世推移。世人皆浊,何不淈其泥而扬其波?众人皆醉,何不餔其糟而歠其醨?何故深思高举,自令放为?''

屈原曰:``吾闻之,新沐者必弹冠,新浴者必振衣;安能以身之察察,受物之汶汶者乎?宁赴湘流,葬于江鱼之腹中。安能以皓皓之白,而蒙世俗之尘埃乎?''

渔父莞尔而笑,鼓枻而去,乃歌曰:``沧浪之水清兮,可以濯吾缨;沧浪之水浊兮,可以濯吾足。''遂去,不复与言。

\hypertarget{header-n156}{%
\subsection{九辩}\label{header-n156}}

\emph{宋玉}

悲哉,秋之为气也!\\
萧瑟兮草木摇落而变衰。\\
憭栗兮若在远行,登山临水兮送将归。\\
泬漻兮天高而气清,寂寥兮收潦而水清。\\
憯悽增欷兮,薄寒之中人,\\
怆怳懭悢兮,去故而就新。\\
坎廪兮贫士失职而志不平,\\
廓落兮羁旅而无友生,\\
惆怅兮而私自怜!\\
燕翩翩其辞归兮,蝉寂漠而无声。\\
雁廱廱而南游兮,鹍鸡啁哳而悲鸣。\\
独申旦而不寐兮,哀蟋蟀之宵征。\\
时亹亹而过中兮,蹇淹留而无成。\\
悲忧穷戚兮独处廓,有美一人兮心不绎。\\
去乡离家兮来远客,超逍遥兮今焉薄!\\
专思君兮不可化,君不知兮可奈何!\\
蓄怨兮积思,心烦憺兮忘食事。\\
原一见兮道余意,君之心兮与余异。\\
车既驾兮朅而归,不得见兮心伤悲。\\
倚结軨兮长太息,涕潺湲兮下霑轼。\\
忼慨绝兮不得,中瞀乱兮迷惑。\\
私自怜兮何极?心怦怦兮谅直。\\
皇天平分四时兮,窃独悲此凛秋。\\
白露既下百草兮,奄离披此梧楸。\\
去白日之昭昭兮,袭长夜之悠悠。\\
离芳蔼之方壮兮,余萎约而悲愁。\\
秋既先戒以白露兮,冬又申之以严霜。\\
收恢台之孟夏兮,然欿傺而沉藏。\\
叶菸邑而无色兮,枝烦挐而交横。\\
颜淫溢而将罢兮,柯仿佛而萎黄。\\
萷櫹椮之可哀兮,形销铄而瘀伤。\\
惟其纷糅而将落兮,恨其失时而无当。\\
揽騑辔而下节兮,聊逍遥以相佯。\\
岁忽忽而遒尽兮,恐余寿之弗将。\\
悼余生之不时兮,逢此世之俇攘。\\
澹容与而独倚兮,蟋蟀鸣此西堂。\\
心怵惕而震荡兮,何所忧之多方。\\
卬明月而太息兮,步列星而极明。\\
窃悲夫蕙华之曾敷兮,纷旖旎乎都房。\\
何曾华之无实兮,从风雨而飞飏!\\
以为君独服此蕙兮,羌无以异于众芳。\\
闵奇思之不通兮,将去君而高翔。\\
心闵怜之惨悽兮,愿一见而有明。\\
重无怨而生离兮,中结轸而增伤。\\
岂不郁陶而思君兮?君之门以九重!\\
猛犬狺狺而迎吠兮,关梁闭而不通。\\
皇天淫溢而秋霖兮,后土何时而得漧?\\
塊独守此无泽兮,仰浮云而永叹!\\
何时俗之工巧兮?背绳墨而改错!

郤骐骥而不乘兮,策驽骀而取路。\\
当世岂无骐骥兮,诚莫之能善御。\\
见执辔者非其人兮,故駶跳而远去。\\
凫雁皆唼夫梁藻兮,凤愈飘翔而高举。\\
圜凿而方枘兮,吾固知其鉏铻而难入。\\
众鸟皆有所登棲兮,凤独遑遑而无所集。\\
原衔枚而无言兮,尝被君之渥洽。\\
太公九十乃显荣兮,诚未遇其匹合。\\
谓骐骥兮安归?谓凤皇兮安棲?\\
变古易俗兮世衰,今之相者兮举肥。\\
骐骥伏匿而不见兮,凤皇高飞而不下。\\
鸟兽犹知怀德兮,何云贤士之不处?\\
骥不骤进而求服兮,凤亦不贪餧而妄食。\\
君弃远而不察兮,虽原忠其焉得?\\
欲寂漠而绝端兮,窃不敢忘初之厚德。\\
独悲愁其伤人兮,冯郁郁其何极?\\
霜露惨悽而交下兮,心尚幸其弗济。\\
霰雪雰糅其增加兮,乃知遭命之将至。\\
原徼幸而有待兮,泊莽莽与野草同死。\\
原自往而径游兮,路壅绝而不通。\\
欲循道而平驱兮,又未知其所从。\\
然中路而迷惑兮,自压桉而学诵。\\
性愚陋以褊浅兮,信未达乎从容。\\
窃美申包胥之气盛兮,恐时世之不固。\\
何时俗之工巧兮?灭规矩而改凿!\\
独耿介而不随兮,原慕先圣之遗教。\\
处浊世而显荣兮,非余心之所乐。\\
与其无义而有名兮,宁穷处而守高。\\
食不媮而为饱兮,衣不苟而为温。\\
窃慕诗人之遗风兮,原讬志乎素餐。\\
蹇充倔而无端兮,泊莽莽而无垠。\\
无衣裘以御冬兮,恐溘死不得见乎阳春。\\
靓杪秋之遥夜兮,心缭悷而有哀。\\
春秋逴逴而日高兮,然惆怅而自悲。\\
四时递来而卒岁兮,阴阳不可与俪偕。\\
白日晼晚其将入兮,明月销铄而减毁。\\
岁忽忽而遒尽兮,老冉冉而愈弛。\\
心摇悦而日幸兮,然怊怅而无冀。\\
中憯恻之悽怆兮,长太息而增欷。\\
年洋洋以日往兮,老嵺廓而无处。\\
事亹亹而觊进兮,蹇淹留而踌躇。\\
何氾滥之浮云兮?猋壅蔽此明月。\\
忠昭昭而原见兮,然霠曀而莫达。\\
原皓日之显行兮,云蒙蒙而蔽之。\\
窃不自聊而原忠兮,或黕点而汙之。\\
尧舜之抗行兮,瞭冥冥而薄天。\\
何险巇之嫉妒兮?被以不慈之伪名。\\
彼日月之照明兮,尚黯黮而有瑕。\\
何况一国之事兮,亦多端而胶加。\\
被荷裯之晏晏兮,然潢洋而不可带。

既骄美而伐武兮,负左右之耿介。\\
憎愠惀之修美兮,好夫人之慷慨。\\
众踥蹀而日进兮,美超远而逾迈。\\
农夫辍耕而容与兮,恐田野之芜秽。\\
事緜緜而多私兮,窃悼後之危败。\\
世雷同而炫曜兮,何毁誉之昧昧!\\
今修饰而窥镜兮,後尚可以竄藏。\\
愿寄言夫流星兮,羌倏忽而难当。\\
卒壅蔽此浮云,下暗漠而无光。\\
尧舜皆有所举任兮,故高枕而自适。\\
谅无怨于天下兮,心焉取此怵惕?\\
乘骐骥之浏浏兮,驭安用夫强策?\\
谅城郭之不足恃兮,虽重介之何益?\\
邅翼翼而无终兮,忳惛惛而愁约。\\
生天地之若过兮,功不成而无嶜。\\
原沉滞而不见兮,尚欲布名乎天下。\\
然潢洋而不遇兮,直怐愗而自苦。\\
莽洋洋而无极兮,忽翱翔之焉薄?\\
国有骥而不知乘兮,焉皇皇而更索?\\
宁戚讴于车下兮,桓公闻而知之。\\
无伯乐之相善兮,今谁使乎誉之?\\
罔流涕以聊虑兮,惟著意而得之。\\
纷纯纯之愿忠兮,妒被离而鄣之。\\
原赐不肖之躯而别离兮,放游志乎云中。\\
乘精气之抟抟兮,骛诸神之湛湛。\\
骖白霓之習習兮,历群灵之丰丰。\\
左硃雀之茇茇兮,右苍龙之躣躣。\\
属雷师之阗阗兮,通飞廉之衙衙。\\
前轻辌之锵锵兮,后辎乘之从从。\\
载云旗之委蛇兮,扈屯骑之容容。\\
计专专之不可化兮,原遂推而为臧。\\
赖皇天之厚德兮,还及君之无恙!

\hypertarget{header-n163}{%
\subsection{招魂}\label{header-n163}}

\emph{屈原}

朕幼清以廉洁兮,身服义而未沫。\\
主此盛德兮,牵于俗而芜秽。\\
上无所考此盛德兮,长离殃而愁苦。\\
帝告巫阳曰:``有人在下,我欲辅之。\\
魂魄离散,汝筮予之。''\\
巫阳对曰:``掌梦!\\
上帝其难从;若必筮予之,\\
恐后之谢,不能复用。''\\
巫阳焉乃下招曰:

魂兮归来!去君之恒干,\\
何为四方些?舍君之乐处,\\
而离彼不祥些!

魂兮归来!东方不可以讬些。\\
长人千仞,惟魂是索些。\\
十日代出,流金铄石些。\\
彼皆习之,魂往必释些。\\
归来兮!不可以讬些。

魂兮归来!南方不可以止些。\\
雕题黑齿,得人肉以祀,以其骨为醢些。\\
蝮蛇蓁蓁,封狐千里些。\\
雄虺九首,往来倏忽,吞人以益其心些。\\
归来兮!不可久淫些。

魂兮归来!西方之害,流沙千里些。\\
旋入雷渊,爢散而不可止些。\\
幸而得脱,其外旷宇些。\\
赤蚁若象,玄蜂若壶些。\\
五谷不生,丛菅是食些。\\
其土烂人,求水无所得些。\\
彷徉无所倚,广大无所极些。\\
归来兮!恐自遗贼些。

魂兮归来!北方不可以止些。\\
增冰峨峨,飞雪千里些。\\
归来兮!不可以久些。

魂兮归来!君无上天些。\\
虎豹九关,啄害下人些。\\
一夫九首,拔木九千些。\\
豺狼从目,往来侁侁些。\\
悬人以嬉,投之深渊些。\\
致命于帝,然后得瞑些。\\
归来!往恐危身些。

魂兮归来!君无下此幽都些。\\
土伯九约,其角觺觺些。\\
敦脄血拇,逐人伂駓駓些。\\
参目虎首,其身若牛些。\\
此皆甘人,归来!恐自遗灾些。

魂兮归来!入修门些。\\
工祝招君,背行先些。\\
秦篝齐缕,郑绵络些。\\
招具该备,永啸呼些。

魂兮归来!反故居些。\\
天地四方,多贼奸些。\\
像设君室,静闲安些。\\
高堂邃宇,槛层轩些。\\
层台累榭,临高山些。\\
网户朱缀,刻方连些。\\
冬有穾厦,夏室寒些。\\
川谷径复,流潺湲些。\\
光风转蕙,氾崇兰些。\\
经堂入奥,朱尘筵些。\\
砥室翠翘,挂曲琼些。\\
翡翠珠被,烂齐光些。\\
蒻阿拂壁,罗帱张些。\\
纂组绮缟,结琦璜些。\\
室中之观,多珍怪些。\\
兰膏明烛,华容备些。\\
二八侍宿,射递代些。\\
九侯淑女,多迅众些。\\
盛鬋不同制,实满宫些。\\
容态好比,顺弥代些。\\
弱颜固植,謇其有意些。\\
姱容修态,絚洞房些。\\
蛾眉曼睩,目腾光些。\\
靡颜腻理,遗视矊些。\\
离榭修幕,侍君之闲些。\\
悲帷翠帐,饰高堂些。\\
红壁沙版,玄玉梁些。\\
仰观刻桷,画龙蛇些。\\
坐堂伏槛,临曲池些。\\
芙蓉始发,杂芰荷些。\\
紫茎屏风,文缘波些。\\
文异豹饰,侍陂陁些。\\
轩辌既低,步骑罗些。\\
兰薄户树,琼木篱些。\\
魂兮归来!何远为些?

室家遂宗,食多方些。\\
稻粢穱麦,挐黄梁些。\\
大苦醎酸,辛甘行些。\\
肥牛之腱,臑若芳些。\\
和酸若苦,陈吴羹些。\\
胹鳖炮羔,有柘浆些。\\
鹄酸臇凫,煎鸿鸧些。\\
露鸡臛蠵,厉而不爽些。\\
粔籹蜜饵,有餦餭些。\\
瑶浆蜜勺,实羽觞些。\\
挫糟冻饮,酎清凉些。\\
华酌既陈,有琼浆些。\\
归来反故室,敬而无妨些。\\
肴羞未通,女乐罗些。\\
敶钟按鼓,造新歌些。\\
《涉江》《采菱》,发《扬荷》些。\\
美人既醉,朱颜酡些。\\
嬉光眇视,目曾波些。\\
被文服纤,丽而不奇些。\\
长发曼鬋,艳陆离些。\\
二八齐容,起郑舞些。\\
衽若交竿,抚案下些。\\
竽瑟狂会,搷鸣鼓些。\\
宫庭震惊,发\textless{}激楚\textgreater{}些。\\
吴歈蔡讴,奏大吕些。\\
士女杂坐,乱而不分些。\\
放敶组缨,班其相纷些。\\
郑卫妖玩,来杂陈些。\\
《激楚》之结,独秀先些。\\
菎蔽象棋,有六簙些。\\
分曹并进,遒相迫些。\\
成枭而牟,呼五白些。\\
晋制犀比,费白日些。\\
铿钟摇簴,揳梓瑟些。\\
娱酒不废,沈日夜些。\\
兰膏明烛,华灯错些。\\
结撰至思,兰芳假些。\\
人有所极,同心赋些。\\
酎饮尽欢,乐先故些。\\
魂兮归来!反故居些。

乱曰:\\
献岁发春兮,汨吾南征。\\
菉蘋齐叶兮,白芷生。\\
路贯庐江兮,左长薄。\\
倚沼畦瀛兮,遥望博。\\
青骊结驷兮,齐千乘。\\
悬火延起兮,玄颜烝。\\
步及骤处兮,诱骋先。\\
抑骛若通兮,引车右还。\\
与王趋梦兮,课后先。\\
君王亲发兮,惮青兕。\\
朱明承夜兮,时不可以淹。\\
皋兰被径兮,斯路渐。\\
湛湛江水兮,上有枫。\\
目极千里兮,伤春心。\\
魂兮归来,哀江南。

\hypertarget{header-n179}{%
\subsection{大招}\label{header-n179}}

\emph{屈原}

青春受谢,白日昭只。\\
春气奋发,万物遽只。\\
冥凌浃行,魂无逃只。\\
魂魄归来!无远遥只。

魂乎归来!无东无西,无南无北只。\\
东有大海,溺水浟浟只。\\
螭龙并流,上下悠悠只。\\
雾雨淫淫,白皓胶只。

魂乎无东!汤谷寂寥只。\\
魂乎无南!南有炎火千里,蝮蛇蜒只。\\
山林险隘,虎豹蜿只。\\
鰅鳙短狐,王虺骞只。\\
魂乎无南!蜮伤躬只;

魂乎无西!西方流沙,漭洋洋只。\\
豕首纵目,被发鬤只。\\
长爪踞牙,诶笑狂只。\\
魂乎无西!多害伤只。

魂乎无北!北有寒山,趠龙赩只。\\
代水不可涉,深不可测只。\\
天白颢颢,寒凝凝只。\\
魂乎无往!盈北极只。

魂魄归来!闲以静只。\\
自恣荆楚,安以定只。\\
逞志究欲,心意安只。\\
穷身永乐,年寿延只。\\
魂乎归来!乐不可言只。

五谷六仞,设菰梁只。\\
鼎臑盈望,和致芳只。\\
内鸧鸽鹄,味豺羹只。\\
魂乎归来!恣所尝只。

鲜蠵甘鸡,和楚酪只。\\
醢豚苦狗,脍苴蒪只。\\
吴酸蒿蒌,不沾薄只。\\
魂兮归来!恣所择只。

炙鸹烝凫,煔鹑敶只。\\
煎鰿膗雀,遽爽存只。\\
魂乎归来!丽以先只。

四酎并孰,不涩嗌只。\\
清馨冻饮,不歠役只。\\
吴醴白蘖,和楚沥只。\\
魂乎归来!不遽惕只。

代秦郑卫,鸣竽张只。\\
伏戏《驾辩》,楚《劳商》只。\\
讴和《扬阿》,赵萧倡只。\\
魂乎归来!定空桑只。

二八接舞,投诗赋只。\\
叩钟调磬,娱人乱只。\\
四上竞气,极声变只。\\
魂乎归来!听歌譔只。

朱唇皓齿,嫭以姱只。\\
比德好闲,习以都只。\\
丰肉微骨,调以娱只。\\
魂乎归来!安以舒只。

嫮目宜笑,娥眉曼只。\\
容则秀雅,稚朱颜只。\\
魂乎归来!静以安只。

姱修滂浩,丽以佳只。\\
曾颊倚耳,曲眉规只。\\
滂心绰态,姣丽施只。\\
小腰秀颈,若鲜卑只。\\
魂乎归来!思怨移只。

易中利心,以动作只。\\
粉白黛黑,施芳泽只。\\
长袂拂面,善留客只。\\
魂乎归来!以娱昔只。

青色直眉,美目媔只。\\
靥辅奇牙,宜笑嘕只。\\
丰肉微骨,体便娟只。\\
魂乎归来!恣所便只。

夏屋广大,沙堂秀只。\\
南房小坛,观绝霤只。\\
曲屋步壛,宜扰畜只。\\
腾驾步游,猎春囿只。\\
琼轂错衡,英华假只。\\
茝兰桂树,郁弥路只。\\
魂乎归来!恣志虑只。

孔雀盈园,畜鸾皇只!\\
鵾鸿群晨,杂鶖鸧只。\\
鸿鹄代游,曼骕驦只。\\
魂乎归来!凤凰翔只。

曼泽怡面,血气盛只。\\
永宜厥身,保寿命只。\\
室家盈廷,爵禄盛只。\\
魂乎归来!居室定只。

接径千里,出若云只。\\
三圭重侯,听类神只。\\
察笃夭隐,孤寡存只。\\
魂兮归来!正始昆只。

田邑千畛,人阜昌只。\\
美冒众流,德泽章只。\\
先威后文,善美明只。\\
魂乎归来!赏罚当只。

名声若日,照四海只。\\
德誉配天,万民理只。\\
北至幽陵,南交阯只。\\
西薄羊肠,东穷海只。\\
魂乎归来!尚贤士只。

发政献行,禁苛暴只。\\
举杰压陛,诛讥罢只。\\
直赢在位,近禹麾只。\\
豪杰执政,流泽施只。\\
魂乎来归!国家为只。

雄雄赫赫,天德明只。\\
三公穆穆,登降堂只。\\
诸侯毕极,立九卿只。\\
昭质既设,大侯张只。\\
执弓挟矢,揖辞让只。\\
魂乎来归!尚三王只。

\hypertarget{header-n208}{%
\subsection{惜誓}\label{header-n208}}

\emph{贾谊}

惜余年老而日衰兮,岁忽忽而不反。\\
登苍天而高举兮,历众山而日远。\\
观江河之纡曲兮,离四海之霑濡。\\
攀北极而一息兮,吸沆瀣以充虚。\\
飞朱鸟使先驱兮,驾太一之象舆。\\
苍龙蚴虯于左骖兮,白虎骋而为右騑。\\
建日月以为盖兮,载玉女于後车。\\
驰骛于杳冥之中兮,休息虖昆仑之墟。\\
乐穷极而不厌兮,愿从容虖神明。\\
涉丹水而驰骋兮,右大夏之遗风。\\
黄鹄之一举兮,知山川之纡曲。\\
再举兮,睹天地之圜方。\\
临中国之众人兮,讬回飙乎尚羊。\\
乃至少原之野兮,赤松、王乔皆在旁。\\
二子拥瑟而调均兮,余因称乎清商。\\
澹然而自乐兮,吸众气而翱翔。\\
念我长生而久仙兮,不如反余之故乡。

黄鹄後时而寄处兮,鸱枭群而制之。\\
神龙失水而陆居兮,为蝼蚁之所裁。\\
夫黄鹄神龙犹如此兮,况贤者之逢乱世哉。\\
寿冉冉而日衰兮,固儃回而不息。\\
俗流从而不止兮,众枉聚而矫直。\\
或偷合而苟进兮,或隐居而深藏。\\
苦称量之不审兮,同权概而就衡。\\
或推迻而苟容兮,或直言之谔謣。\\
伤诚是之不察兮,并纫茅丝以为索。\\
方世俗之幽昏兮,眩白黑之美恶。\\
放山渊之龟玉兮,相与贵夫砾石。\\
梅伯数谏而至醢兮,来革顺志而用国。\\
悲仁人之尽节兮,反为小人之所贼。\\
比干忠谏而剖心兮,箕子被发而佯狂。\\
水背流而源竭兮,木去根而不长。\\
非重躯以虑难兮,惜伤身之无功。

已矣哉!\\
独不见夫鸾凤之高翔兮,乃集大皇之野。\\
循四极而回周兮,见盛德而後下。\\
彼圣人之神德兮,远浊世而自藏。\\
使麒麟可得羁而係兮,又何以异虖犬羊?

\hypertarget{header-n215}{%
\subsection{招隐士}\label{header-n215}}

\emph{淮南小山}

桂树丛生兮山之幽,偃蹇连蜷兮枝相缭。\\
山气巄嵷兮石嵯峨,溪谷崭岩兮水曾波。\\
猿狖群啸兮虎豹嗥,攀援桂枝兮聊淹留。\\
王孙游兮不归,春草生兮萋萋。\\
岁暮兮不自聊,蟪蛄鸣兮啾啾。\\
坱兮轧,山曲岪,心淹留兮恫慌忽。\\
罔兮沕,憭兮栗,虎豹穴。\\
丛薄深林兮,人上栗。\\
嵚岑碕礒兮,碅磳磈硊;\\
树轮相纠兮,林木茷骫。\\
青莎杂树兮,薠草靃靡;\\
白鹿麏麚兮,或腾或倚。\\
状貌崟崟兮峨峨,凄凄兮漇漇。\\
猕猴兮熊罴,慕类兮以悲;\\
攀援桂枝兮聊淹留。\\
虎豹斗兮熊罴咆,禽兽骇兮亡其曹。\\
王孙兮归来,山中兮不可以久留。

\hypertarget{header-n220}{%
\subsection{七谏}\label{header-n220}}

\emph{东方朔}

\hypertarget{header-n225}{%
\subsubsection{初放}\label{header-n225}}

平生于国兮,长于原野。\\
言语讷譅兮,又无彊辅。\\
浅智褊能兮,闻见又寡。\\
数言便事兮,见怨门下。\\
王不察其长利兮,卒见弃乎原野。\\
伏念思过兮,无可改者。\\
群众成朋兮,上浸以惑。\\
巧佞在前兮,贤者灭息。\\
尧、舜圣已没兮,孰为忠直?\\
高山崔巍兮,水流汤汤。\\
死日将至兮,与麋鹿同坑。\\
塊兮鞠,当道宿,\\
举世皆然兮,余将谁告?\\
斥逐鸿鹄兮,近习鸱枭,\\
斩伐橘柚兮,列树苦桃。\\
便娟之修竹兮,寄生乎江潭。\\
上葳蕤而防露兮,下泠泠而来风。\\
孰知其不合兮,若竹柏之异心。\\
往者不可及兮,来者不可待。\\
悠悠苍天兮,莫我振理。\\
窃怨君之不寤兮,吾独死而後已。

\hypertarget{header-n230}{%
\subsubsection{沉江}\label{header-n230}}

惟往古之得失兮,览私微之所伤。\\
尧舜圣而慈仁兮,後世称而弗忘。\\
齐桓失于专任兮,夷吾忠而名彰。\\
晋献惑于孋姬兮,申生孝而被殃。\\
偃王行其仁义兮,荆文寤而徐亡。\\
纣暴虐以失位兮,周得佐乎吕望。\\
修往古以行恩兮,封比干之丘垄。\\
贤俊慕而自附兮,日浸淫而合同。\\
明法令而修理兮,兰芷幽而有芳。\\
苦众人之妒予兮,箕子寤而佯狂。\\
不顾地以贪名兮,心怫郁而内伤。\\
联蕙芷以为佩兮,过鲍肆而失香。\\
正臣端其操行兮,反离谤而见攘。\\
世俗更而变化兮,伯夷饿于首阳。\\
独廉洁而不容兮,叔齐久而逾明。\\
浮云陈而蔽晦兮,使日月乎无光。\\
忠臣贞而欲谏兮,谗谀毁而在旁。\\
秋草荣其将实兮,微霜下而夜降。\\
商风肃而害生兮,百草育而不长。\\
众并谐以妒贤兮,孤圣特而易伤。\\
怀计谋而不见用兮,岩穴处而隐藏。\\
成功隳而不卒兮,子胥死而不葬。\\
世从俗而变化兮,随风靡而成行。\\
信直退而毁败兮,虚伪进而得当。\\
追悔过之无及兮,岂尽忠而有功。\\
废制度而不用兮,务行私而去公。\\
终不变而死节兮,惜年齿之未央。\\
将方舟而下流兮,冀幸君之发矇。\\
痛忠言之逆耳兮,恨申子之沉江。\\
愿悉心之所闻兮,遭值君之不聪。\\
不开寤而难道兮,不别横之与纵。\\
听奸臣之浮说兮,绝国家之久长。\\
灭规矩而不用兮,背绳墨之正方。\\
离忧患而乃寤兮,若纵火于秋蓬。\\
业失之而不救兮,尚何论乎祸凶。\\
彼离畔而朋党兮,独行之士其何望?\\
日渐染而不自知兮,秋毫微哉而变容。\\
众轻积而折轴兮,原咎杂而累重。\\
赴湘沅之流澌兮,恐逐波而复东。\\
怀沙砾而自沉兮,不忍见君之蔽壅。

\hypertarget{header-n235}{%
\subsubsection{怨世}\label{header-n235}}

世沉淖而难论兮,俗岒峨而嵾嵯。\\
清泠泠而歼灭兮,溷湛湛而日多。\\
枭鸮既以成群兮,玄鹤弭翼而屏移。\\
蓬艾亲入御于床笫兮,马兰踸踔而日加。\\
弃捐药芷与杜衡兮,余柰世之不知芳何?\\
何周道之平易兮,然芜秽而险戏。\\
高阳无故而委尘兮,唐虞点灼而毁议。\\
谁使正其真是兮,虽有八师而不可为。\\
皇天保其高兮,后土持其久。\\
服清白以逍遥兮,偏与乎玄英异色。\\
西施媞媞而不得见兮,嫫母勃屑而日侍。\\
桂蠹不知所淹留兮,蓼虫不知徙乎葵菜。\\
处湣湣之浊世兮,今安所达乎吾志。\\
意有所载而远逝兮,固非众人之所识。\\
骥踌躇于弊輂兮,遇孙阳而得代。\\
吕望穷困而不聊生兮,遭周文而舒志。\\
宁戚饭牛而商歌兮,桓公闻而弗置。\\
路室女之方桑兮,孔子过之以自侍。\\
吾独乖剌而无当兮,心悼怵而耄思。\\
思比干之恲恲兮,哀子胥之慎事。\\
悲楚人之和氏兮,献宝玉以为石。\\
遇厉武之不察兮,羌两足以毕斮。\\
小人之居势兮,视忠正之何若?\\
改前圣之法度兮,喜嗫嚅而妄作。\\
亲谗谀而疏贤圣兮,讼谓闾娵为丑恶。\\
愉近习而蔽远兮,孰知察其黑白?\\
卒不得效其心容兮,安眇眇而无所归薄。\\
专精爽以自明兮,晦冥冥而壅蔽。\\
年既已过太半兮,然埳轲而留滞。\\
欲高飞而远集兮,恐离罔而灭败。\\
独冤抑而无极兮,伤精神而寿夭。\\
皇天既不纯命兮,余生终无所依。\\
愿自沉于江流兮,绝横流而径逝。\\
宁为江海之泥涂兮,安能久见此浊世?

\hypertarget{header-n240}{%
\subsubsection{怨思}\label{header-n240}}

贤士穷而隐处兮,廉方正而不容。\\
子胥谏而靡躯兮,比干忠而剖心。\\
子推自割而飤君兮,德日忘而怨深。\\
行明白而曰黑兮,荆棘聚而成林。\\
江离弃于穷巷兮,蒺藜蔓乎东厢。\\
贤者蔽而不见兮,谗谀进而相朋。\\
枭鸮并进而俱鸣兮,凤皇飞而高翔。\\
原壹往而径逝兮,道壅绝而不通。

\hypertarget{header-n245}{%
\subsubsection{自悲}\label{header-n245}}

居愁懃其谁告兮,独永思而忧悲。\\
内自省而不惭兮,操愈坚而不衰。\\
隐三年而无决兮,岁忽忽其若颓。\\
怜余身不足以卒意兮,冀一见而复归。\\
哀人事之不幸兮,属天命而委之咸池。\\
身被疾而不闲兮,心沸热其若汤。\\
冰炭不可以相并兮,吾固知乎命之不长。\\
哀独苦死之无乐兮,惜予年之未央。\\
悲不反余之所居兮,恨离予之故乡。\\
鸟兽惊而失群兮,犹高飞而哀鸣。\\
狐死必首丘兮,夫人孰能不反其真情?\\
故人疏而日忘兮,新人近而俞好。\\
莫能行于杳冥兮,孰能施于无报?\\
苦众人之皆然兮,乘回风而远游。\\
凌恆山其若陋兮,聊愉娱以忘忧。\\
悲虚言之无实兮,苦众口之铄金。\\
过故乡而一顾兮,泣歔欷而霑衿。\\
厌白玉以为面兮,怀琬琰以为心。\\
邪气入而感内兮,施玉色而外淫。\\
何青云之流澜兮,微霜降之蒙蒙。\\
徐风至而徘徊兮,疾风过之汤汤。\\
闻南籓乐而欲往兮,至会稽而且止。\\
见韩众而宿之兮,问天道之所在?\\
借浮云以送予兮,载雌霓而为旌。\\
驾青龙以驰骛兮,班衍衍之冥冥。\\
忽容容其安之兮,超慌忽其焉如?\\
苦众人之难信兮,愿离群而远举。\\
登峦山而远望兮,好桂树之冬荣。\\
观天火之炎炀兮,听大壑之波声。\\
引八维以自道兮,含沆瀣以长生。\\
居不乐以时思兮,食草木之秋实。\\
饮菌若之朝露兮,构桂木而为室。\\
杂橘柚以为囿兮,列新夷与椒桢。\\
鹍鹤孤而夜号兮,哀居者之诚贞。

\hypertarget{header-n250}{%
\subsubsection{哀命}\label{header-n250}}

哀时命之不合兮,伤楚国之多忧。\\
内怀情之洁白兮,遭乱世而离尤。\\
恶耿介之直行兮,世溷浊而不知。\\
何君臣之相失兮,上沅湘而分离。\\
测汨罗之湘水兮,知时固而不反。\\
伤离散之交乱兮,遂侧身而既远。\\
处玄舍之幽门兮,穴岩石而窟伏。\\
从水蛟而为徙兮,与神龙乎休息。\\
何山石之崭岩兮,灵魂屈而偃蹇。\\
含素水而蒙深兮,日眇眇而既远。\\
哀形体之离解兮,神罔两而无舍。\\
惟椒兰之不反兮,魂迷惑而不知路。\\
愿无过之设行兮,虽灭没之自乐。\\
痛楚国之流亡兮,哀灵修之过到。\\
固时俗之溷浊兮,志瞀迷而不知路。\\
念私门之正匠兮,遥涉江而远去。\\
念女嬃之婵媛兮,涕泣流乎于悒。\\
我决死而不生兮,虽重追吾何及。\\
戏疾濑之素水兮,望高山之蹇产。\\
哀高丘之赤岸兮,遂没身而不反。

\hypertarget{header-n255}{%
\subsubsection{谬谏}\label{header-n255}}

怨灵修之浩荡兮,夫何执操之不固?\\
悲太山之为隍兮,孰江河之可涸?\\
愿承闲而效志兮,恐犯忌而干讳。\\
卒抚情以寂寞兮,然怊怅而自悲。\\
玉与石其同匮兮,贯鱼眼与珠玑。\\
驽骏杂而不分兮,服罢牛而骖骥。\\
年滔滔而自远兮,寿冉冉而愈衰。\\
心悇憛而烦冤兮,蹇超摇而无冀。\\
固时俗之工巧兮,灭规矩而改错。\\
郤骐骥而不乘兮,策驽骀而取路。\\
当世岂无骐骥兮,诚无王良之善驭。\\
见执辔者非其人兮,故驹跳而远去。\\
不量凿而正枘兮,恐矩矱之不同。\\
不论世而高举兮,恐操行之不调。\\
弧弓弛而不张兮,孰云知其所至?\\
无倾危之患难兮,焉知贤士之所死?\\
俗推佞而进富兮,节行张而不著。\\
贤良蔽而不群兮,朋曹比而党誉。\\
邪说饰而多曲兮,正法弧而不公。\\
直士隐而避匿兮,谗谀登乎明堂。\\
弃彭咸之娱乐兮,灭巧倕之绳墨。\\
菎蕗杂于黀蒸兮,机蓬矢以射革。\\
驾蹇驴而无策兮,又何路之能极?\\
以直鍼而为钓兮,又何鱼之能得?\\
伯牙之绝弦兮,无锺子期而听之。\\
和抱璞而泣血兮,安得良工而剖之?\\
同音者相和兮,同类者相似。\\
飞鸟号其群兮,鹿鸣求其友。\\
故叩宫而宫应兮,弹角而角动。\\
虎啸而谷风至兮,龙举而景云往。\\
音声之相和兮,言物类之相感也。\\
夫方圜之异形兮,势不可以相错。\\
列子隐身而穷处兮,世莫可以寄讬。\\
众鸟皆有行列兮,凤独翔翔而无所薄。\\
经浊世而不得志兮,愿侧身岩穴而自讬。\\
欲阖口而无言兮,尝被君之厚德。\\
独便悁而怀毒兮,愁郁郁之焉极?\\
念三年之积思兮,愿壹见而陈辞。\\
不及君而骋说兮,世孰可为明之?\\
身寝疾而日愁兮,情沉抑而不扬。\\
众人莫可与论道兮,悲精神之不通。\\
乱曰:\\
鸾皇孔凤日以远兮,畜凫驾鹅。\\
鸡鹜满堂坛兮,鼉黽游乎华池。\\
要褭奔亡兮,腾驾橐驼。\\
铅刀进御兮,遥弃太阿。\\
拔搴玄芝兮,列树芋荷。\\
橘柚萎枯兮,苦李旖旎。\\
甂瓯登于明堂兮,周鼎潜潜乎深渊。\\
自古而固然兮,吾又何怨乎今之人。

\hypertarget{header-n259}{%
\subsection{哀时命}\label{header-n259}}

\emph{庄忌}

哀时命之不及古人兮,夫何予生之不遘时!\\
往者不可扳援兮,徠者不可与期。\\
志憾恨而不逞兮,杼中情而属诗。\\
夜炯炯而不寐兮,怀隐忧而历兹。\\
心郁郁而无告兮,众孰可与深谋!\\
欿愁悴而委惰兮,老冉冉而逮之。\\
居处愁以隐约兮,志沉抑而不扬。\\
道壅塞而不通兮,江河广而无梁。\\
愿至昆仑之悬圃兮,采锺山之玉英。\\
揽瑶木之橝枝兮,望阆风之板桐。\\
弱水汩其为难兮,路中断而不通。\\
势不能凌波以径度兮,又无羽翼而高翔。\\
然隐悯而不达兮,独徙倚而彷徉。\\
怅惝罔以永思兮,心纡轸而增伤。\\
倚踌躇以淹留兮,日饥馑而绝粮。\\
廓抱景而独倚兮,超永思乎故乡。\\
廓落寂而无友兮,谁可与玩此遗芳?\\
白日晼晼其將入兮,哀余寿之弗将。\\
车既弊而马罢兮,蹇邅徊而不能行。\\
身既不容于浊世兮,不知进退之宜当。

冠崔嵬而切云兮,剑淋离而从横。\\
衣摄叶以储与兮,左袪挂于榑桑;\\
右衽拂于不周兮,六合不足以肆行。\\
上同凿枘于伏戏兮,下合矩矱于虞唐。\\
原尊节而式高兮,志犹卑夫禹汤。\\
虽知困其不改操兮,终不以邪枉害方。\\
世并举而好朋兮,壹斗斛而相量。\\
众比周以肩迫兮,贤者远而隐藏。\\
为凤皇作鹑笼兮,虽翕翅其不容。\\
灵皇其不寤知兮,焉陈词而效忠。\\
俗嫉妒而蔽贤兮,孰知余之从容?\\
愿舒志而抽冯兮,庸讵知其吉凶?\\
璋珪杂于甑窐兮,陇廉与孟娵同宫。\\
举世以为恆俗兮,固将愁苦而终穷。\\
幽独转而不寐兮,惟烦懑而盈匈。\\
魂眇眇而驰骋兮,心烦冤之忡忡。\\
志欿憾而不憺兮,路幽昧而甚难。

塊独守此曲隅兮,然欿切而永叹。\\
愁修夜而宛转兮,气涫沸其若波。\\
握剞劂而不用兮,操规矩而无所施。\\
骋骐骥于中庭兮,焉能极夫远道?\\
置援狖于棂槛兮,夫何以责其捷巧?\\
驷跛鳖而上山兮,吾固知其不能陞。\\
释管晏而任臧获兮,何权衡之能称?\\
箟簬杂于黀蒸兮,机蓬矢以射革。\\
负檐荷以丈尺兮,欲伸要而不可得。\\
外迫胁于机臂兮,上牵联于矰隿。\\
肩倾侧而不容兮,固陿腹而不得息。\\
务光自投于深渊兮,不获世之尘垢。\\
孰魁摧之可久兮,愿退身而穷处。\\
凿山楹而为室兮,下被衣于水渚。\\
雾露濛濛其晨降兮,云依斐而承宇。\\
虹霓纷其朝霞兮,夕淫淫而淋雨。\\
怊茫茫而无归兮,怅远望此旷野。\\
下垂钓于溪谷兮,上要求于仙者。\\
与赤松而结友兮,比王侨而为耦。\\
使枭杨先导兮,白虎为之前後。\\
浮云雾而入冥兮,骑白鹿而容与。

魂眐眐以寄独兮,汨徂往而不归。\\
处卓卓而日远兮,志浩荡而伤怀。\\
鸾凤翔于苍云兮,故矰缴而不能加。\\
蛟龙潜于旋渊兮,身不挂于罔罗。\\
知贪饵而近死兮,不如下游乎清波。\\
宁幽隐以远祸兮,孰侵辱之可为。\\
子胥死而成义兮,屈原沉于汨罗。\\
虽体解其不变兮,岂忠信之可化。\\
志怦怦而内直兮,履绳墨而不颇。\\
执权衡而无私兮,称轻重而不差。\\
摡尘垢之枉攘兮,除秽累而反真。\\
形体白而质素兮,中皎洁而淑清。\\
时猒饫而不用兮,且隐伏而远身。\\
聊窜端而匿迹兮,嗼寂默而无声。\\
独便悁而烦毒兮,焉发愤而筊抒。\\
时暧暧其将罢兮,遂闷叹而无名。\\
伯夷死于首阳兮,卒夭隐而不荣。\\
太公不遇文王兮,身至死而不得逞。\\
怀瑶象而佩琼兮,愿陈列而无正。\\
生天坠之若过兮,忽烂漫而无成。\\
邪气袭余之形体兮,疾憯怛而萌生。\\
原壹见阳春之白日兮,恐不终乎永年。

\hypertarget{header-n267}{%
\subsection{九怀}\label{header-n267}}

\emph{王褒}

\hypertarget{header-n272}{%
\subsubsection{匡机}\label{header-n272}}

极运兮不中,来将屈兮困穷。\\
余深愍兮惨怛,愿一列兮无从。\\
乘日月兮上征,顾游心兮鄗酆。\\
弥览兮九隅,彷徨兮兰宫。\\
芷闾兮药房,奋摇兮众芳。\\
菌阁兮蕙楼,观道兮从横。\\
宝金兮委积,美玉兮盈堂。\\
桂水兮潺湲,扬流兮洋洋。\\
蓍蔡兮踊跃,孔鹤兮回翔。\\
抚槛兮远望,念君兮不忘。\\
怫郁兮莫陈,永怀兮内伤。

\hypertarget{header-n277}{%
\subsubsection{通路}\label{header-n277}}

天门兮墬户,孰由兮贤者?\\
无正兮溷厕,怀德兮何睹?\\
假寐兮愍斯,谁可与兮寤语?\\
痛凤兮远逝,畜鴳兮近处。\\
鲸鱏兮幽潜,从虾兮游陼。\\
乘虬兮登阳,载象兮上行。\\
朝发兮葱岭,夕至兮明光。\\
北饮兮飞泉,南采兮芝英。\\
宣游兮列宿,顺极兮彷徉。\\
红采兮骍衣,翠缥兮为裳。\\
舒佩兮綝纚,竦余剑兮干将。\\
腾蛇兮后从,飞駏兮步旁。\\
微观兮玄圃,览察兮瑶光。\\
启匮兮探筴,悲命兮相当。\\
纫蕙兮永辞,将离兮所思。\\
浮云兮容与,道余兮何之?\\
远望兮仟眠,闻雷兮阗阗。\\
阴忧兮感余,惆怅兮自怜。

\hypertarget{header-n282}{%
\subsubsection{危俊}\label{header-n282}}

林不容兮鸣蜩,余何留兮中州?\\
陶嘉月兮总驾,搴玉英兮自修。\\
结荣茝兮逶逝,将去烝兮远游。\\
径岱土兮魏阙,历九曲兮牵牛。\\
聊假日兮相佯,遗光燿兮周流。\\
望太一兮淹息,纡余辔兮自休。\\
晞白日兮皎皎,弥远路兮悠悠。\\
顾列孛兮缥缥,观幽云兮陈浮。\\
钜宝迁兮砏磤,雉咸雊兮相求。\\
泱莽莽兮究志,惧吾心兮懤懤。\\
步余马兮飞柱,览可与兮匹俦。\\
卒莫有兮纤介,永余思兮怞怞。

\hypertarget{header-n287}{%
\subsubsection{昭世}\label{header-n287}}

世溷兮冥昏,违君兮归真。\\
乘龙兮偃蹇,高回翔兮上臻。\\
袭英衣兮缇{[}糹習{]},披华裳兮芳芬。\\
登羊角兮扶舆,浮云漠兮自娱。\\
握神精兮雍容,与神人兮相胥。\\
流星坠兮成雨,进瞵盼兮上丘墟。\\
览旧邦兮滃郁,余安能兮久居。\\
志怀逝兮心懰栗,纡余辔兮踌躇。\\
闻素女兮微歌,听王后兮吹竽。\\
魂悽怆兮感哀,肠回回兮盘纡。\\
抚余佩兮缤纷,高太息兮自怜。\\
使祝融兮先行,令昭明兮开门。\\
驰六蛟兮上征,竦余驾兮入冥。\\
历九州兮索合,谁可与兮终生。\\
忽反顾兮西囿,睹轸丘兮崎倾。\\
横垂涕兮泫流,悲余后兮失灵。

\hypertarget{header-n292}{%
\subsubsection{尊嘉}\label{header-n292}}

季春兮阳阳,列草兮成行。\\
余悲兮兰生,委积兮从横。\\
江离兮遗捐,辛夷兮挤臧。\\
伊思兮往古,亦多兮遭殃。\\
伍胥兮浮江,屈子兮沉湘。\\
运余兮念兹,心内兮怀伤。\\
望淮兮沛沛,滨流兮则逝。\\
榜舫兮下流,东注兮磕磕。\\
蛟龙兮导引,文鱼兮上濑。\\
抽蒲兮陈坐,援芙蕖兮为盖。\\
水跃兮余旌,继以兮微蔡。\\
云旗兮电骛,倏忽兮容裔。\\
河伯兮开门,迎余兮欢欣。\\
顾念兮旧都,怀恨兮艰难。\\
窃哀兮浮萍,汎淫兮无根。

\hypertarget{header-n297}{%
\subsubsection{蓄英}\label{header-n297}}

秋风兮萧萧,舒芳兮振条。\\
微霜兮眇眇,病殀兮鸣蜩。\\
玄鸟兮辞归,飞翔兮灵丘。\\
望谿谷兮滃郁,熊罴兮呴嗥。\\
唐虞兮不存,何故兮久留?\\
临渊兮汪洋,顾林兮忽荒。\\
修余兮袿衣,骑霓兮南上。\\
乘云兮回回,亹亹兮自强。\\
将息兮兰皋,失志兮悠悠。\\
蒶蕴兮霉黧,思君兮无聊。\\
身去兮意存,怆恨兮怀愁。

\hypertarget{header-n302}{%
\subsubsection{思忠}\label{header-n302}}

登九灵兮游神,静女歌兮微晨。\\
悲皇丘兮积葛,众体错兮交纷。\\
贞枝抑兮枯槁,枉车登兮庆云。\\
感余志兮惨栗,心怆怆兮自怜。\\
驾玄螭兮北征,曏吾路兮葱岭。\\
连五宿兮建旄,扬氛气兮为旌。\\
历广漠兮驰骛,览中国兮冥冥。\\
玄武步兮水母,与吾期兮南荣。\\
登华盖兮乘阳,聊逍遥兮播光。\\
抽库娄兮酌醴,援瓟瓜兮接粮。\\
毕休息兮远逝,发玉軔兮西行。\\
惟时俗兮疾正,弗可久兮此方。\\
寤辟摽兮永思,心怫郁兮内伤。

\hypertarget{header-n307}{%
\subsubsection{陶壅}\label{header-n307}}

览杳杳兮世惟,余惆怅兮何归。\\
伤时俗兮溷乱,将奋翼兮高飞。\\
驾八龙兮连蜷,建虹旌兮威夷。\\
观中宇兮浩浩,纷翼翼兮上跻。\\
浮溺水兮舒光,淹低佪兮京沶。\\
屯余车兮索友,睹皇公兮问师。\\
道莫贵兮归真,羡余术兮可夷。\\
吾乃逝兮南娭,道幽路兮九疑。\\
越炎火兮万里,过万首兮嶷嶷。\\
济江海兮蝉蜕,绝北梁兮永辞。\\
浮云郁兮昼昏,霾土忽兮塺々。\\
息阳城兮广夏,衰色罔兮中怠。\\
意晓阳兮燎寤,乃自诊兮在兹。\\
思尧舜兮袭兴,幸咎繇兮获谋。\\
悲九州兮靡君,抚轼叹兮作诗。

\hypertarget{header-n312}{%
\subsubsection{株昭}\label{header-n312}}

悲哉于嗟兮,心内切磋。\\
款冬而生兮,凋彼叶柯。\\
瓦砾进宝兮,捐弃随和。\\
铅刀厉御兮,顿弃太阿。\\
骥垂两耳兮,中坂蹉跎。\\
蹇驴服驾兮,无用日多。\\
修洁处幽兮,贵宠沙劘。\\
凤皇不翔兮,鹑鴳飞扬。\\
乘虹骖蜺兮,载云变化。\\
焦明开路兮,后属青蛇。\\
步骤桂林兮,超骧卷阿。\\
丘陵翔儛兮,谿谷悲歌。\\
神章灵篇兮,赴曲相和。\\
余私娱兹兮,孰哉复加。\\
还顾世俗兮,坏败罔罗。\\
卷佩将逝兮,涕流滂沲。\\
乱曰:\\
皇门开兮照下土,株秽除兮兰芷睹。\\
四佞放兮後得禹,圣舜摄兮昭尧绪,\\
孰能若兮原为辅。

\hypertarget{header-n316}{%
\subsection{九叹}\label{header-n316}}

\emph{刘向}

\hypertarget{header-n321}{%
\subsubsection{逢纷}\label{header-n321}}

伊伯庸之末胄兮,谅皇直之屈原。\\
云余肇祖于高阳兮,惟楚怀之婵连。\\
原生受命于贞节兮,鸿永路有嘉名。\\
齐名字于天地兮,并光明于列星。\\
吸精粹而吐氛浊兮,横邪世而不取容。\\
行叩诚而不阿兮,遂见排而逢谗。\\
后听虚而黜实兮,不吾理而顺情。\\
肠愤悁而含怒兮,志迁蹇而左倾。\\
心戃慌其不我与兮,躬速速其不吾亲。\\
辞灵修而陨志兮,吟泽畔之江滨。\\
椒桂罗以颠覆兮,有竭信而归诚。\\
谗夫蔼蔼而漫著兮,曷其不舒予情?\\
始结言于庙堂兮,信中涂而叛之。\\
怀兰蕙与衡芷兮,行中野而散之。\\
声哀哀而怀高丘兮,心愁愁而思旧邦。\\
愿承闲而自恃兮,径淫曀而道壅。\\
颜霉黧以沮败兮,精越裂而衰耄。\\
裳襜襜而含风兮,衣纳纳而掩露。\\
赴江湘之湍流兮,顺波凑而下降。\\
徐徘徊于山阿兮,飘风来之洶洶。\\
驰余车兮玄石,步余马兮洞庭。\\
平明发兮苍梧,夕投宿兮石城。\\
芙蓉盖而菱华车兮,紫贝阙而玉堂。\\
薜荔饰而陆离荐兮,鱼鳞衣而白蜺裳。\\
登逢龙而下陨兮,违故都之漫漫。\\
思南郢之旧俗兮,肠一夕而九运。\\
扬流波之潢潢兮,体溶溶而东回。\\
心怊怅以永思兮,意晻晻而日颓。\\
白露纷以涂涂兮,秋风浏以萧萧。\\
身永流而不还兮,魂长逝而常愁。\\
叹曰:\\
譬彼流水纷扬磕兮,波逢汹涌濆壅滂兮。\\
揄扬涤荡飘流陨往触崟石兮,\\
龙卬脟圈缭戾宛转阻相薄兮,\\
遭纷逢凶蹇离尤兮,垂文扬采遗将来兮。

\hypertarget{header-n326}{%
\subsubsection{离世}\label{header-n326}}

灵怀其不吾知兮,灵怀其不吾闻。\\
就灵怀之皇祖兮,愬灵怀之鬼神。\\
灵怀曾不吾与兮,即听夫人之谀辞。\\
余辞上参于天坠兮,旁引之于四时。\\
指日月使延照兮,抚招摇以质正。\\
立师旷俾端辞兮,命咎繇使并听。\\
兆出名曰正则兮,卦发字曰灵均。\\
余幼既有此鸿节兮,长愈固而弥纯。\\
不从俗而诐行兮,直躬指而信志。\\
不枉绳以追曲兮,屈情素以从事。\\
端余行其如玉兮,述皇舆之踵迹。\\
群阿容以晦光兮,皇舆覆以幽辟。\\
舆中涂以回畔兮,驷马惊而横奔。\\
执组者不能制兮,必折轭而摧辕。\\
断镳衔以驰骛兮,暮去次而敢止。\\
路荡荡其无人兮,遂不禦乎千里。\\
身衡陷而下沉兮,不可获而复登。\\
不顾身之卑贱兮,惜皇舆之不兴。\\
出国门而端指兮,冀壹寤而锡还。\\
哀仆夫之坎毒兮,屡离忧而逢患。\\
九年之中不吾反兮,思彭咸之水游。\\
惜师延之浮渚兮,赴汨罗之长流。\\
遵江曲之逶移兮,触石碕而衡游。\\
波澧澧而扬浇兮,顺长濑之浊流。\\
凌黄沱而下低兮,思还流而复反。\\
玄舆驰而并集兮,身容与而日远。\\
棹舟杭以横濿兮,济湘流而南极。\\
立江界而长吟兮,愁哀哀而累息。\\
情慌忽以忘归兮,神浮游以高历。\\
心蛩蛩而怀顾兮,魂眷眷而独逝。\\
叹曰:\\
余思旧邦心依违兮,\\
日暮黄昏羌幽悲兮,\\
去郢东迁余谁慕兮,\\
谗夫党旅其以兹故兮,\\
河水淫淫情所愿兮,\\
顾瞻郢路终不返兮。

\hypertarget{header-n331}{%
\subsubsection{怨思}\label{header-n331}}

惟郁郁之忧毒兮,志坎壈而不违。\\
身憔悴而考旦兮,日黄昏而长悲。\\
闵空宇之孤子兮,哀枯杨之冤雏。\\
孤雌吟于高墉兮,鸣鸠栖于桑榆。\\
玄蝯失于潜林兮,独偏弃而远放。\\
征夫劳于周行兮,处妇愤而长望。\\
申诚信而罔违兮,情素洁于纽帛。\\
光明齐于日月兮,文采耀燿于玉石。\\
伤压次而不发兮,思沉抑而不扬。\\
芳懿懿而终败兮,名靡散而不彰。\\
背玉门以奔骛兮,蹇离尤而干诟。\\
若龙逢之沉首兮,王子比干之逢醢。\\
念社稷之几危兮,反为雠而见怨。\\
思国家之离沮兮,躬获愆而结难。\\
若青蝇之伪质兮,晋骊姬之反情。\\
恐登阶之逢殆兮,故退伏于末庭。\\
孽臣之号咷兮,本朝芜而不治。\\
犯颜色而触谏兮,反蒙辜而被疑。\\
菀蘼芜与菌若兮,渐藁本于洿渎。\\
淹芳芷于腐井兮,弃鸡骇于筐簏。\\
执棠谿以刜蓬兮,秉干将以割肉。\\
筐泽泻以豹鞟兮,破荆和以继筑。\\
时溷浊犹未清兮,世殽乱犹未察。\\
欲容与以俟时兮,惧年岁之既晏。\\
顾屈节以从流兮,心巩巩而不夷。\\
宁浮沅而驰骋兮,下江湘以邅回。\\
叹曰:\\
山中槛槛余伤怀兮,征夫皇皇其孰依兮,\\
经营原野杳冥冥兮,乘骐骋骥舒吾情兮,\\
归骸旧邦莫谁语兮,长辞远逝乘湘去兮。

\hypertarget{header-n336}{%
\subsubsection{远逝}\label{header-n336}}

志隐隐而郁怫兮,愁独哀而冤结。\\
肠纷纭以缭转兮,涕渐渐其若屑。\\
情慨慨而长怀兮,信上皇而质正。\\
合五岳与八灵兮,讯九鬿与六神。\\
指列宿以白情兮,诉五帝以置辞。\\
北斗为我折中兮,太一为余听之。\\
云服阴阳之正道兮,御后土之中和。\\
佩苍龙之蚴虬兮,带隐虹之逶蛇。\\
曳彗星之皓旰兮,抚朱爵与鵔鸃。\\
游清灵之飒戾兮,服云衣之披披。\\
杖玉策与朱旗兮,垂明月之玄珠。\\
举霓旌之墆翳兮,建黄纁之总旄。\\
躬纯粹而罔愆兮,承皇考之妙仪。\\
惜往事之不合兮,横汨罗而下沥。\\
乘隆波而南渡兮,逐江湘之顺流。\\
赴阳侯之潢洋兮,下石濑而登洲。\\
陆魁堆以蔽视兮,云冥冥而闇前。\\
山峻高以无垠兮,遂曾闳而迫身。\\
雪雰雰而薄木兮,云霏霏而陨集。\\
阜隘狭而幽险兮,石嵾嵯以翳日。\\
悲故乡而发忿兮,去余邦之弥久。\\
背龙门而入河兮,登大坟而望夏首。\\
横舟航而济湘兮,耳聊啾而戃慌。\\
波淫淫而周流兮,鸿溶溢而滔荡。\\
路曼曼其无端兮,周容容而无识。\\
引日月以指极兮,少须臾而释思。\\
水波远以冥冥兮,眇不睹其东西。\\
顺风波以南北兮,雾宵晦以纷纷。\\
日杳杳以西颓兮,路长远而窘迫。\\
欲酌醴以娱忧兮,蹇骚骚而不释。\\
叹曰:\\
飘风蓬龙埃坲々兮,草木摇落时槁悴兮,\\
遭倾遇祸不可救兮,长吟永欷涕究究兮,\\
舒情陈诗冀以自免兮,颓流下陨身日远兮。

\hypertarget{header-n341}{%
\subsubsection{惜贤}\label{header-n341}}

览屈氏之离骚兮,心哀哀而怫郁。\\
声嗷嗷以寂寥兮,顾仆夫之憔悴。\\
拨谄谀而匡邪兮,切淟涊之流俗。\\
荡渨涹之奸咎兮,夷蠢蠢之溷浊。\\
怀芬香而挟蕙兮,佩江蓠之婓婓。\\
握申椒与杜若兮,冠浮云之峨峨。\\
登长陵而四望兮,览芷圃之蠡蠡。\\
游兰皋与蕙林兮,睨玉石之嵾嵯。\\
扬精华以炫燿兮,芳郁渥而纯美。\\
结桂树之旖旎兮,纫荃蕙与辛夷。\\
芳若兹而不御兮,捐林薄而菀死。\\
驱子侨之奔走兮,申徒狄之赴渊。\\
若由夷之纯美兮,介子推之隐山。\\
晋申生之离殃兮,荆和氏之泣血。\\
吴申胥之抉眼兮,王子比干之横废。\\
欲卑身而下体兮,心隐恻而不置。\\
方圜殊而不合兮,钩绳用而异态。\\
欲俟时于须臾兮,日阴曀其将暮。\\
时迟迟其日进兮,年忽忽而日度。\\
妄周容而入世兮,内距闭而不开。\\
俟时风之清激兮,愈氛雾其如塺。\\
进雄鸠之耿耿兮,谗介介而蔽之。\\
默顺风以偃仰兮,尚由由而进之。\\
心懭悢以冤结兮,情舛错以曼忧。\\
搴薜荔于山野兮,采撚支于中洲。\\
望高丘而叹涕兮,悲吸吸而长怀。\\
孰契契而委栋兮,日晻晻而下颓。\\
叹曰:\\
江湘油油长流汩兮,挑揄扬汰荡迅疾兮。\\
忧心展转愁怫郁兮,冤结未舒长隐忿兮,\\
丁时逢殃可奈何兮,劳心悁悁涕滂沱兮。

\hypertarget{header-n346}{%
\subsubsection{忧苦}\label{header-n346}}

悲余心之悁悁兮,哀故邦之逢殃。\\
辞九年而不复兮,独茕茕而南行。\\
思余俗之流风兮,心纷错而不受。\\
遵野莽以呼风兮,步从容于山廋。\\
巡陆夷之曲衍兮,幽空虚以寂寞。\\
倚石岩以流涕兮,忧憔悴而无乐。\\
登巑岏以长企兮,望南郢而闚之。\\
山修远其辽辽兮,涂漫漫其无时。\\
听玄鹤之晨鸣兮,于高冈之峨峨。\\
独愤积而哀娱兮,翔江洲而安歌。\\
三鸟飞以自南兮,览其志而欲北。\\
原寄言于三鸟兮,去飘疾而不可得。\\
欲迁志而改操兮,心纷结其未离。\\
外彷徨而游览兮,内恻隐而含哀。\\
聊须臾以时忘兮,心渐渐其烦错。\\
原假簧以舒忧兮,志纡郁其难释。\\
叹《离骚》以扬意兮,犹未殫于《九章》。\\
长嘘吸以于悒兮,涕横集而成行。\\
伤明珠之赴泥兮,鱼眼玑之坚藏。\\
同驽骡与乘駔兮,杂斑駮与阘茸。\\
葛藟虆于桂树兮,鸱鸮集于木兰。\\
偓促谈于廊庙兮,律魁放乎山间。\\
恶虞氏之箫《韶》兮,好遗风之《激楚》。\\
潜周鼎于江淮兮,爨土鬵于中宇。\\
且人心之持旧兮,而不可保长。\\
邅彼南道兮,征夫宵行。\\
思念郢路兮,还顾睠睠。\\
涕流交集兮,泣下涟涟。\\
叹曰:\\
登山长望中心悲兮,菀彼青青泣如颓兮,\\
留思北顾涕渐渐兮,折锐摧矜凝氾滥兮,\\
念我茕茕魂谁求兮,仆夫慌悴散若流兮。

\hypertarget{header-n351}{%
\subsubsection{愍命}\label{header-n351}}

昔皇考之嘉志兮,喜登能而亮贤。\\
情纯洁而罔薉兮,姿盛质而无愆。\\
放佞人与谄谀兮,斥谗夫与便嬖。\\
亲忠正之悃诚兮,招贞良与明智。\\
心溶溶其不可量兮,情澹澹其若渊。\\
回邪辟而不能入兮,诚原藏而不可迁。\\
逐下袟于後堂兮,迎虙妃于伊雒。\\
刜谗贼于中廇兮,选吕管于榛薄。\\
丛林之下无怨士兮,江河之畔无隐夫。\\
三苗之徒以放逐兮,伊皋之伦以充庐。\\
今反表以为里兮,颠裳以为衣。\\
戚宋万于两楹兮,废周邵于遐夷。\\
却骐骥以转运兮,腾驴骡以驰逐。\\
蔡女黜而出帷兮,戎妇入而綵绣服。\\
庆忌囚于阱室兮,陈不占战而赴围。\\
破伯牙之号钟兮,挟人筝而弹纬。\\
藏瑉石于金匮兮,捐赤瑾于中庭。\\
韩信蒙于介胄兮,行夫将而攻城。\\
莞芎弃于泽洲兮,瓟瓥蠹于筐簏。\\
麒麟奔于九皋兮,熊罴群而逸囿。\\
折芳枝与琼华兮,树枳棘与薪柴。\\
掘荃蕙与射干兮,耘藜藿与蘘荷。\\
惜今世其何殊兮,远近思而不同。\\
或沉沦其无所达兮,或清激其无所通。\\
哀余生之不当兮,独蒙毒而逢尤。\\
虽謇謇以申志兮,君乖差而屏之。\\
诚惜芳之菲菲兮,反以兹为腐也。\\
怀椒聊之蔎蔎兮,乃逢纷以罹诟也。\\
叹曰:\\
嘉皇既殁终不返兮,山中幽险郢路远兮。\\
谗人諓諓孰可愬兮,征夫罔极谁可语兮。\\
行吟累欷声喟喟兮,怀忧含戚何侘傺兮。

\hypertarget{header-n356}{%
\subsubsection{思古}\label{header-n356}}

冥冥深林兮,树木郁郁。\\
山参差以崭岩兮,阜杳杳以蔽日。\\
悲余心之悁悁兮,目眇眇而遗泣。\\
风骚屑以摇木兮,云吸吸以湫戾。\\
悲余生之无欢兮,愁倥傯于山陆。\\
旦徘徊于长阪兮,夕彷徨而独宿。\\
发披披以鬤鬤兮,躬劬劳而瘏悴。\\
魂俇俇而南行兮,泣霑襟而濡袂。\\
心婵媛而无告兮,口噤闭而不言。\\
违郢都之旧闾兮,回湘、沅而远迁。\\
念余邦之横陷兮,宗鬼神之无次。\\
闵先嗣之中绝兮,心惶惑而自悲。\\
聊浮游于山陿兮,步周流于江畔。\\
临深水而长啸兮,且倘佯而氾观。\\
兴离骚之微文兮,冀灵修之壹悟。\\
还余车于南郢兮,复往轨于初古。\\
道修远其难迁兮,伤余心之不能已。\\
背三五之典刑兮,绝洪范之辟纪。\\
播规矩以背度兮,错权衡而任意。\\
操绳墨而放弃兮,倾容幸而侍侧。\\
甘棠枯于丰草兮,藜棘树于中庭。\\
西施斥于北宫兮,仳倠倚于弥楹。\\
乌获戚而骖乘兮,燕公操于马圉。\\
蒯聩登于清府兮,咎繇弃而在野。\\
盖见兹以永叹兮,欲登阶而狐疑。\\
乘白水而高骛兮,因徙弛而长词。\\
叹曰:\\
倘佯垆阪沼水深兮,容与汉渚涕淫淫兮,\\
锺牙已死谁为声兮?纤阿不御焉舒情兮,\\
曾哀悽欷心离离兮,还顾高丘泣如洒兮。

\hypertarget{header-n361}{%
\subsubsection{远游}\label{header-n361}}

悲余性之不可改兮,屡惩艾而不迻。\\
服觉皓以殊俗兮,貌揭揭以巍巍。\\
譬若王侨之乘云兮,载赤霄而凌太清。\\
欲与天地参寿兮,与日月而比荣。\\
登昆仑而北首兮,悉灵圉而来谒。\\
选鬼神于太阴兮,登阊阖于玄阙。\\
回朕车俾西引兮,褰虹旗于玉门。\\
驰六龙于三危兮,朝西灵于九滨。\\
结余轸于西山兮,横飞谷以南征。\\
绝都广以直指兮,历祝融于硃冥。\\
枉玉衡于炎火兮,委两馆于咸唐。\\
贯澒濛以东朅兮,维六龙于扶桑。\\
周流览于四海兮,志升降以高驰。\\
徵九神于回极兮,建虹采以招指。\\
驾鸾凤以上游兮,从玄鹤与鹪明。\\
孔鸟飞而送迎兮,腾群鹤于瑶光。\\
排帝宫与罗囿兮,升县圃以眩灭。\\
结琼枝以杂佩兮,立长庚以继日。\\
凌惊雷以轶骇电兮,缀鬼谷于北辰。\\
鞭风伯使先驱兮,囚灵玄于虞渊。\\
遡高风以低佪兮,览周流于朔方。\\
就颛顼而敶辞兮,考玄冥于空桑。\\
旋车逝于崇山兮,奏虞舜于苍梧。\\
济杨舟于会稽兮,就申胥于五湖。\\
见南郢之流风兮,殒余躬于沅湘。\\
望旧邦之黯黮兮,时溷浊其犹未央。\\
怀兰茝之芬芳兮,妒被离而折之。\\
张绛帷以襜襜兮,风邑邑而蔽之。\\
日暾暾其西舍兮,阳焱焱而复顾。\\
聊假日以须臾兮,何骚骚而自故。\\
叹曰:\\
譬彼蛟龙乘云浮兮,\\
汎淫澒溶纷若雾兮。\\
潺湲轇轕雷动电发馺高举兮。\\
升虚凌冥沛浊浮清入帝宫兮,\\
摇翘奋羽驰风骋雨游无穷兮。

\hypertarget{header-n365}{%
\subsection{九思}\label{header-n365}}

*王逸

\hypertarget{header-n370}{%
\subsubsection{逢尤}\label{header-n370}}

悲兮愁,哀兮忧!\\
天生我兮当闇时,被诼谮兮虚获尤。\\
心烦憒兮意无聊,严载驾兮出戏游。\\
周八极兮历九州,求轩辕兮索重华。\\
世既卓兮远眇眇,握佩玖兮中路躇。\\
羡咎繇兮建典谟,懿风后兮受瑞图。\\
愍余命兮遭六极,委玉质兮於泥涂。\\
遽傽遑兮驱林泽,步屏营兮行丘阿。\\
车軏折兮马虺颓,惷怅立兮涕滂沱。\\
思丁文兮圣明哲,哀平差兮迷谬愚。\\
吕傅举兮殷周兴,忌嚭专兮郢吴虚。\\
仰长叹兮气噎结,悒殟绝兮咶复苏。\\
虎兕争兮於廷中,豺狼斗兮我之隅。\\
云雾会兮日冥晦,飘风起兮扬尘埃。\\
走鬯罔兮乍东西,欲窜伏兮其焉如?\\
念灵闺兮隩重深,原竭节兮隔无由。\\
望旧邦兮路逶随,忧心悄兮志勤劬。\\
魂茕茕兮不遑寐,目眽眽兮寤终朝。

\hypertarget{header-n375}{%
\subsubsection{怨上}\label{header-n375}}

令尹兮謷謷,群司兮譨譨。\\
哀哉兮淈淈,上下兮同流。\\
菽藟兮蔓衍,芳虈兮挫枯。\\
硃紫兮杂乱,曾莫兮别诸。\\
倚此兮岩穴,永思兮窈悠。\\
嗟怀兮眩惑,用志兮不昭。\\
将丧兮玉斗,遗失兮钮枢。\\
我心兮煎熬,惟是兮用忧。\\
进恶兮九旬,复顾兮彭务。\\
拟斯兮二踪,未知兮所投。\\
谣吟兮中野,上察兮璇玑。\\
大火兮西睨,摄提兮运低。\\
雷霆兮硠磕,雹霰兮霏霏。\\
奔电兮光晃,凉风兮怆悽。\\
鸟兽兮惊骇,相从兮宿栖。\\
鸳鸯兮噰噰,狐狸兮徾徾。\\
哀吾兮介特,独处兮罔依。\\
蝼蛄兮鸣东,蟊蠽兮号西。\\
蛓缘兮我裳,蠋入兮我怀。\\
虫豸兮夹余,惆怅兮自悲。\\
伫立兮忉怛,心结縎兮折摧。

\hypertarget{header-n380}{%
\subsubsection{疾世}\label{header-n380}}

周徘徊兮汉渚,求水神兮灵女。\\
嗟此国兮无良,媒女诎兮謰謱。\\
鴳雀列兮譁讙,鸲鹆鸣兮聒余。\\
抱昭华兮宝璋,欲衒鬻兮莫取。\\
言旋迈兮北徂,叫我友兮配耦。\\
日阴曀兮未光,阒睄窕兮靡睹。\\
纷载驱兮高驰,将谘询兮皇羲。\\
遵河皋兮周流,路变易兮时乖。\\
濿沧海兮东游,沐盥浴兮天池。\\
访太昊兮道要,云靡贵兮仁义。\\
志欣乐兮反征,就周文兮邠歧。\\
秉玉英兮结誓,日欲暮兮心悲。\\
惟天禄兮不再,背我信兮自违。\\
逾陇堆兮渡漠,过桂车兮合黎。\\
赴昆山兮馽騄,从邛遨兮栖迟。\\
吮玉液兮止渴,齧芝华兮疗饥。\\
居嵺廓兮尠畴,远梁昌兮几迷。\\
望江汉兮濩渃,心紧絭兮伤怀。\\
时昢昢兮且旦,尘莫莫兮未晞。\\
忧不暇兮寝食,吒增叹兮如雷。

\hypertarget{header-n385}{%
\subsubsection{悯上}\label{header-n385}}

哀世兮睩睩,諓諓兮嗌喔。\\
众多兮阿媚,骪靡兮成俗。\\
贪枉兮党比,贞良兮茕独。\\
鹄窜兮枳棘,鹈集兮帷幄。\\
罽蕠兮青葱,槁本兮萎落。\\
睹斯兮伪惑,心为兮隔错。\\
逡巡兮圃薮,率彼兮畛陌。\\
川谷兮渊渊,山阜兮峉峉。\\
丛林兮崟崟,株榛兮岳岳。\\
霜雪兮漼溰,冰冻兮洛泽。\\
东西兮南北,罔所兮归薄。\\
庇廕兮枯树,匍匐兮岩石。\\
蜷跼兮寒局数,独处兮志不申。\\
年齿尽兮命迫促,魁垒挤摧兮常困辱。\\
含忧强老兮愁无乐,须发苎悴兮顠鬓白。\\
思灵泽兮一膏沐,怀兰英兮把琼若,\\
待天明兮立踯躅。\\
云蒙蒙兮电鯈烁,孤雌惊兮鸣呴呴。\\
思怫郁兮肝切剥,忿悁悒兮孰诉告。

\hypertarget{header-n390}{%
\subsubsection{遭厄}\label{header-n390}}

悼屈子兮遭厄,沉王躬兮湘汨。\\
何楚国兮难化,迄于今兮不易。\\
士莫志兮羔裘,竞佞谀兮谗阋。\\
指正义兮为曲,訿玉璧兮为石。\\
殦雕游兮华屋,鵔鸃栖兮柴蔟。\\
起奋迅兮奔走,违群小兮謑訽。\\
载青云兮上昇,適昭明兮所处。\\
蹑天衢兮长驱,踵九阳兮戏荡。\\
越云汉兮南济,秣余马兮河鼓。\\
云霓纷兮晻翳,参辰回兮颠倒。\\
逢流星兮问路,顾我指兮从左。\\
俓娵觜兮直驰,御者迷兮失轨。\\
遂踢达兮邪造,与日月兮殊道。\\
志阏绝兮安如,哀所求兮不耦。\\
攀天阶兮下视,见鄢郢兮旧宇。\\
意逍遥兮欲归,众秽盛兮沓沓。\\
思哽饐兮诘诎,涕流澜兮如雨。

\hypertarget{header-n395}{%
\subsubsection{悼乱}\label{header-n395}}

嗟嗟兮悲夫,殽乱兮纷挐。\\
茅丝兮同综,冠屦兮共絇。\\
督万兮侍宴,周邵兮负刍。\\
白龙兮见射,灵龟兮执拘。\\
仲尼兮困厄,邹衍兮幽囚。\\
伊余兮念兹,奔遁兮隐居。\\
将升兮高山,上有兮猴猿。\\
欲入兮深谷,下有兮虺蛇。\\
左见兮鸣鵙,右睹兮呼枭。\\
惶悸兮失气,踊跃兮距跳。\\
便旋兮中原,仰天兮增叹。\\
菅蒯兮楙莽,雚苇兮仟眠。\\
鹿蹊兮躖躖,貒貉兮蟫蟫。\\
鹯鹞兮轩轩,鹑鹌兮甄甄。\\
哀我兮寡独,靡有兮齐伦。\\
意欲兮沉吟,迫日兮黄昏。\\
玄鹤兮高飞,曾逝兮青冥。\\
鶬鶊兮喈喈,山鹊兮嚶嚶。\\
鸿鸬兮振翅,归雁兮于征。\\
吾志兮觉悟,怀我兮圣京。\\
垂屣兮将起,跓俟兮硕明。

\hypertarget{header-n400}{%
\subsubsection{伤时}\label{header-n400}}

惟昊天兮昭灵,阳气发兮清明。\\
风習習兮和暖,百草萌兮华荣。\\
堇荼茂兮扶疏,蘅芷彫兮莹嫇。\\
愍贞良兮遇害,将夭折兮碎糜。\\
时混混兮浇饡,哀当世兮莫知。\\
览往昔兮俊彦,亦诎辱兮系纍。\\
管束缚兮桎梏,百贸易兮传卖。\\
遭桓缪兮识举,才德用兮列施。\\
且从容兮自慰,玩琴书兮游戏。\\
迫中国兮迮陿,吾欲之兮九夷。\\
超五岭兮嵯峨,观浮石兮崔嵬。\\
陟丹山兮炎野,屯余车兮黄支。\\
就祝融兮稽疑,嘉己行兮无为。\\
乃回朅兮北逝,遇神孈兮宴娭。\\
欲静居兮自娱,心愁慼兮不能。\\
放余辔兮策驷,忽飙腾兮浮云。\\
蹠飞杭兮越海,从安期兮蓬莱。\\
缘天梯兮北上,登太一兮玉台。\\
使素女兮鼓簧,乘戈和兮讴谣。\\
声噭誂兮清和,音晏衍兮要婬。\\
咸欣欣兮酣乐,余眷眷兮独悲。\\
顾章华兮太息,志恋恋兮依依。

\hypertarget{header-n405}{%
\subsubsection{哀岁}\label{header-n405}}

旻天兮清凉,玄气兮高朗。\\
北风兮潦洌,草木兮苍唐。\\
蛜蚗兮噍噍,蝍蛆兮穰穰。\\
岁忽忽兮惟暮,余感时兮悽怆。\\
伤俗兮泥浊,矇蔽兮不章。\\
宝彼兮沙砾,捐此兮夜光。\\
椒瑛兮湟汙,葈耳兮充房。\\
摄衣兮缓带,操我兮墨阳。\\
昇车兮命仆,将驰兮四荒。\\
下堂兮见虿,出门兮触螽。\\
巷有兮蚰蜓,邑多兮螳螂。\\
睹斯兮嫉贼,心为兮切伤。\\
俛念兮子胥,仰怜兮比干。\\
投剑兮脱冕,龙屈兮蜿蟤。\\
潜藏兮山泽,匍匐兮丛攒。\\
窥见兮溪涧,流水兮沄沄。\\
鼋鼍兮欣欣,鱣鲇兮延延。\\
群行兮上下,骈罗兮列陈。\\
自恨兮无友,特处兮茕茕。\\
冬夜兮陶陶,雨雪兮冥冥。\\
神光兮颎颎,鬼火兮荧荧。\\
修德兮困控,愁不聊兮遑生。\\
忧纡兮郁郁,恶所兮写情。

\hypertarget{header-n410}{%
\subsubsection{守志}\label{header-n410}}

陟玉峦兮逍遥,览高冈兮峣峣。\\
桂树列兮纷敷,吐紫华兮布条。\\
实孔鸾兮所居,今其集兮惟鸮。\\
乌鹊惊兮哑哑,余顾盼兮怊怊。\\
彼日月兮闇昧,障覆天兮祲氛。\\
伊我后兮不聪,焉陈诚兮效忠。\\
摅羽翮兮超俗,游陶遨兮养神。\\
乘六蛟兮蜿蝉,遂驰骋兮升云。\\
扬彗光兮为旗,秉电策兮为鞭。\\
朝晨发兮鄢郢,食时至兮增泉。\\
绕曲阿兮北次,造我车兮南端。\\
谒玄黄兮纳贽,崇忠贞兮弥坚。\\
历九宫兮遍观,睹秘藏兮宝珍。\\
就传说兮骑龙,与织女兮合婚。\\
举天罼兮掩邪,彀天弧兮射奸。\\
随真人兮翱翔,食元气兮长存。\\
望太微兮穆穆,睨三阶兮炳分。\\
相辅政兮成化,建烈业兮垂勋。\\
目瞥瞥兮西没,道遐回兮阻叹。\\
志稸积兮未通,怅敞罔兮自怜。\\
乱曰:\\
天庭明兮云霓藏,三光朗兮镜万方。\\
斥蜥蜴兮进龟龙,策谋从兮翼机衡。\\
配稷契兮恢唐功,嗟英俊兮未为双。

\end{document}
