\PassOptionsToPackage{unicode=true}{hyperref} % options for packages loaded elsewhere
\PassOptionsToPackage{hyphens}{url}
%
\documentclass[]{article}
\usepackage{lmodern}
\usepackage{amssymb,amsmath}
\usepackage{ifxetex,ifluatex}
\usepackage{fixltx2e} % provides \textsubscript
\ifnum 0\ifxetex 1\fi\ifluatex 1\fi=0 % if pdftex
  \usepackage[T1]{fontenc}
  \usepackage[utf8]{inputenc}
  \usepackage{textcomp} % provides euro and other symbols
\else % if luatex or xelatex
  \usepackage{unicode-math}
  \defaultfontfeatures{Ligatures=TeX,Scale=MatchLowercase}
\fi
% use upquote if available, for straight quotes in verbatim environments
\IfFileExists{upquote.sty}{\usepackage{upquote}}{}
% use microtype if available
\IfFileExists{microtype.sty}{%
\usepackage[]{microtype}
\UseMicrotypeSet[protrusion]{basicmath} % disable protrusion for tt fonts
}{}
\IfFileExists{parskip.sty}{%
\usepackage{parskip}
}{% else
\setlength{\parindent}{0pt}
\setlength{\parskip}{6pt plus 2pt minus 1pt}
}
\usepackage{hyperref}
\hypersetup{
            pdfborder={0 0 0},
            breaklinks=true}
\urlstyle{same}  % don't use monospace font for urls
\setlength{\emergencystretch}{3em}  % prevent overfull lines
\providecommand{\tightlist}{%
  \setlength{\itemsep}{0pt}\setlength{\parskip}{0pt}}
\setcounter{secnumdepth}{0}
% Redefines (sub)paragraphs to behave more like sections
\ifx\paragraph\undefined\else
\let\oldparagraph\paragraph
\renewcommand{\paragraph}[1]{\oldparagraph{#1}\mbox{}}
\fi
\ifx\subparagraph\undefined\else
\let\oldsubparagraph\subparagraph
\renewcommand{\subparagraph}[1]{\oldsubparagraph{#1}\mbox{}}
\fi

% set default figure placement to htbp
\makeatletter
\def\fps@figure{htbp}
\makeatother


\date{}

\begin{document}

\hypertarget{header-n1933}{%
\section{庄子}\label{header-n1933}}

\begin{center}\rule{0.5\linewidth}{\linethickness}\end{center}

\tableofcontents

\begin{center}\rule{0.5\linewidth}{\linethickness}\end{center}

\hypertarget{header-n1941}{%
\subsection{外篇}\label{header-n1941}}

\hypertarget{header-n1942}{%
\subsubsection{逍遥游}\label{header-n1942}}

北冥有鱼,其名为鲲。鲲之大,不知其几千里也。化而为鸟,其名
为鹏。鹏之背,不知其几千里也。怒而飞,其翼若垂天之云。是鸟也
,海运则将徙于南冥。南冥者,天池也。

《齐谐》者,志怪者也。《谐》之言曰:``鹏之徙于南冥也,水击
三千里,抟扶摇而上者九万里,去以六月息者也。''野马也,尘埃也
,生物之以息相吹也。天之苍苍,其正色邪?其远而无所至极邪?其
视下也,亦若是则已矣。

且夫水之积也不厚,则其负大舟也无力。覆杯水于坳堂之上,则芥
为之舟。置杯焉则胶,水浅而舟大也。风之积也不厚,则其负大翼也
无力。故九万里则风斯在下矣,而后乃今培风;背负青天而莫之夭阏
者,而后乃今将图南。

蜩与学鸠笑之曰:``我决起而飞,枪榆枋而止,时则不至,而控于地而已矣,奚以之九万里而南为?''适莽苍者,三餐而反,腹犹果然;适百
里者,宿舂粮;适千里者,三月聚粮。之二虫又何知!

小知不及大知,小年不及大年。奚以知其然也?朝菌不知晦朔,蟪
蛄不知春秋,此小年也。楚之南有冥灵者,以五百岁为春,五百岁为
秋;上古有大椿者,以八千岁为春,八千岁为秋。而彭祖乃今以久特
闻,众人匹之,不亦悲乎!

汤之问棘也是已:穷发之北,有冥海者,天池也。有鱼焉,其广数
千里,未有知其修者,其名为鲲。有鸟焉,其名为鹏,背若泰山,翼
若垂天之云,抟扶摇羊角而上者九万里,绝云气,负青天,然后图南
,且适南冥也。

斥鴳笑之曰:``彼且奚适也?我腾跃而上,不过数仞而下,翱翔蓬
蒿之间,此亦飞之至也,而彼且奚适也?''此小大之辩也。

故夫知效一官,行比一乡,德合一君,而徵一国者,其自视也,亦
若此矣。而宋荣子犹然笑之。且举世而誉之而不加劝,举世而非之而
不加沮,定乎内外之分,辩乎荣辱之境,斯已矣。彼其于世,未数数
然也。虽然,犹有未树也。

夫列子御风而行,泠然善也,旬有五日而后反。彼于致福者,未数
数然也。此虽免乎行,犹有所待者也。

若夫乘天地之正,而御六气之辩,以游无穷者,彼且恶乎待哉!故
曰:至人无己,神人无功,圣人无名。

尧让天下于许由,曰:``日月出矣,而爝火不息,其于光也,不亦
难乎!时雨降矣,而犹浸灌,其于泽也,不亦劳乎!夫子立而天下治
,而我犹尸之,吾自视缺然。请致天下。''许由曰:``子治天下,天
下既已治也,而我犹代子,吾将为名乎?名者,实之宾也,吾将为宾
乎?鹪鹩巢于深林,不过一枝;偃鼠饮河,不过满腹。归休乎君,予
无所用天下为!庖人虽不治庖,尸祝不越樽俎而代之矣。''

肩吾问于连叔曰:``吾闻言于接舆,大而无当,往而不返。吾惊怖
其言犹河汉而无极也,大有径庭,不近人情焉。''连叔曰:``其言谓
何哉?''``曰`藐姑射之山,有神人居焉。肌肤若冰雪,淖约若处子;不食五谷,吸风饮露;乘云气,御飞龙,而游乎四海之外;其神凝
,使物不疵疠而年谷熟。'吾以是狂而不信也。''连叔曰:``然,瞽
者无以与乎文章之观,聋者无以与乎钟鼓之声。岂唯形骸有聋盲哉?
夫知亦有之。是其言也,犹时女也。之人也,之德也,将旁礴万物以
为一,世蕲乎乱,孰弊弊焉以天下为事!之人也,物莫之伤,大浸稽
天而不溺,大旱金石流、土山焦而不热。是其尘垢粃糠,将犹陶铸尧舜
者也,孰肯以物为事!''

宋人次章甫而适越,越人断发文身,无所用之。

尧治天下之民,平海内之政。往见四子藐姑射之山,汾水之阳,杳
然丧其天下焉。

惠子谓庄子曰:``魏王贻我大瓠之种,我树之成而实五石。以盛水
浆,其坚不能自举也。剖之以为瓢,则瓠落无所容。非不呺然大也,
吾为其无用而掊之。''庄子曰:``夫子固拙于用大矣。宋人有善为不
龟手之药者,世世以洴澼絖为事。客闻之,请买其方百金。聚族而谋
之曰:`我世世为澼絖,不过数金。今一朝而鬻技百金,请与之。'
客得之,以说吴王。越有难,吴王使之将。冬,与越人水战,大败越
人,裂地而封之。能不龟手一也,或以封,或不免于澼絖,则所用之
异也。今子有五石之瓠,何不虑以为大樽而浮乎江湖,而忧其瓠落无
所容?则夫子犹有蓬之心也夫!''

惠子谓庄子曰:``吾有大树,人谓之樗。其大本臃肿而不中绳墨,
其小枝卷曲而不中规矩。立之涂,匠者不顾。今子之言,大而无用,
众所同去也。''庄子曰:``子独不见狸狌乎?卑身而伏,以候敖者;
东西跳梁,不避高下;中于机辟,死于罔罟。今夫嫠牛,其大若垂天
之云。此能为大矣,而不能执鼠。今子有大树,患其无用,何不树之
于无何有之乡,广莫之野,彷徨乎无为其侧,逍遥乎寝卧其下。不夭
斤斧,物无害者,无所可用,安所困苦哉!

齐物论

南郭子綦隐机而坐,仰天而嘘,荅焉似丧其耦。颜成子游立侍乎前,曰:``何居乎?形固可使如槁木,而心固可使如死灰乎?今之隐机者,非昔之隐机者也?''子綦曰:``偃,不亦善乎,而问之也!今者吾丧我,汝知之乎?女闻人籁而未闻地籁,女闻地籁而不闻天籁夫!''

子游曰:``敢问其方。''子綦曰:``夫大块噫气,其名为风。是唯无作,作则万窍怒呺。而独不闻之翏翏乎?山林之畏佳,大木百围之窍穴,似鼻,似口,似耳,似枅,似圈,似臼,似洼者,似污者。激者、謞者、叱者、吸者、叫者、譹者、宎者,咬者,前者唱于而随者唱喁,泠风则小和,飘风则大和,厉风济则众窍为虚。而独不见之调调之刁刁乎?''

子游曰:``地籁则众窍是已,人籁则比竹是已,敢问天籁。''子綦曰:``夫吹万不同,而使其自己也。咸其自取,怒者其谁邪?''

大知闲闲,小知间间。大言炎炎,小言詹詹。其寐也魂交,其觉也形开。与接为构,日以心斗。缦者、窖者、密者。小恐惴惴,大恐缦缦。其发若机栝,其司是非之谓也;其留如诅盟,其守胜之谓也;其杀如秋冬,以言其日消也;其溺之所为之,不可使复之也;其厌也如缄,以言其老洫也;近死之心,莫使复阳也。喜怒哀乐,虑叹变蜇,姚佚启态------乐出虚,蒸成菌。日夜相代乎前而莫知其所萌。已乎,已乎!旦暮得此,其所由以生乎!

非彼无我,非我无所取。是亦近矣,而不知其所为使。若有真宰,而特不得其眹。可行己信,而不见其形,有情而无形。百骸、九窍、六藏、赅而存焉,吾谁与为亲?汝皆说之乎?其有私焉?如是皆有为臣妾乎?其臣妾不足以相治乎?其递相为君臣乎?其有真君存焉!如求得其情与不得,无益损乎其真。一受其成形,不亡以待尽。与物相刃相靡,其行尽如驰而莫之能止,不亦悲乎!终身役役而不见其成功,苶然疲役而不知其所归,可不哀邪!人谓之不死,奚益!其形化,其心与之然,可不谓大哀乎?人之生也,固若是芒乎?其我独芒,而人亦有不芒者乎?

夫随其成心而师之,谁独且无师乎?奚必知代而自取者有之?愚者与有焉!未成乎心而有是非,是今日适越而昔至也。是以无有为有。无有为有,虽有神禹且不能知,吾独且奈何哉!

夫言非吹也,言者有言。其所言者特未定也。果有言邪?其未尝有言邪?其以为异于鷇音,亦有辩乎?其无辩乎?道恶乎隐而有真伪?言恶乎隐而有是非?道恶乎往而不存?言恶乎存而不可?道隐于小成,言隐于荣华。故有儒墨之是非,以是其所非而非其所是。欲是其所非而非其所是,则莫若以明。

物无非彼,物无非是。自彼则不见,自知则知之。故曰:彼出于是,是亦因彼。彼是方生之说也。虽然,方生方死,方死方生;方可方不可,方不可方可;因是因非,因非因是。是以圣人不由而照之于天,亦因是也。是亦彼也,彼亦是也。彼亦一是非,此亦一是非,果且有彼是乎哉?果且无彼是乎哉?彼是莫得其偶,谓之道枢。枢始得其环中,以应无穷。是亦一无穷,非亦一无穷也。故曰:莫若以明。

以指喻指之非指,不若以非指喻指之非指也;以马喻马之非马,不若以非马喻马之非马也。天地一指也,万物一马也。

可乎可,不可乎不可。道行之而成,物谓之而然。有自也而可,有自也而不可;有自也而然,有自也而不然。恶乎然?然于然。恶乎不然?不然于不然。物固有所然,物固有所可。无物不然,无物不可。故为是举莛与楹,厉与西施,恢诡谲怪,道通为一。

其分也,成也;其成也,毁也。凡物无成与毁,复通为一。唯达者知通为一,为是不用而寓诸庸。庸也者,用也;用也者,通也;通也者,得也;适得而几矣。因是已。已而不知其然,谓之道。劳神明为一而不知其同也,谓之``朝三''。何谓``朝三''?狙公赋芧,曰:``朝三而暮四。''众狙皆怒。曰:``然则朝四而暮三。''众狙皆悦。名实未亏而喜怒为用,亦因是也。是以圣人和之以是非而休乎天钧,是之谓两行。

古之人,其知有所至矣。恶乎至?有以为未始有物者,至矣,尽矣,不可以加矣!其次以为有物矣,而未始有封也。其次以为有封焉,而未始有是非也。是非之彰也,道之所以亏也。道之所以亏,爱之所以成。果且有成与亏乎哉?果且无成与亏乎哉?有成与亏,故昭氏之鼓琴也;无成与亏,故昭氏之不鼓琴也。昭文之鼓琴也,师旷之枝策也,惠子之据梧也,三子之知几乎皆其盛者也,故载之末年。唯其好之也,以异于彼,其好之也,欲以明之。彼非所明而明之,故以坚白之昧终。而其子又以文之纶终,终身无成。若是而可谓成乎,虽我亦成也;若是而不可谓成乎,物与我无成也。是故滑疑之耀,圣人之所图也。为是不用而寓诸庸,此之谓``以明''。

今且有言于此,不知其与是类乎?其与是不类乎?类与不类,相与为类,则与彼无以异矣。虽然,请尝言之:有始也者,有未始有始也者,有未始有夫未始有始也者;有有也者,有无也者,有未始有无也者,有未始有夫未始有无也者。俄而有无矣,而未知有无之果孰有孰无也。今我则已有有谓矣,而未知吾所谓之其果有谓乎?其果无谓乎?

夫天下莫大于秋豪之末,而太山为小;莫寿乎殇子,而彭祖为夭。天地与我并生,而万物与我为一。既已为一矣,且得有言乎?既已谓之一矣,且得无言乎?一与言为二,二与一为三。自此以往,巧历不能得,而况其凡乎!故自无适有,以至于三,而况自有适有乎!无适焉,因是已!

夫道未始有封,言未始有常,为是而有畛也。请言其畛:有左有右,有伦有义,有分有辩,有竞有争,此之谓八德。六合之外,圣人存而不论;六合之内,圣人论而不议;春秋经世先王之志,圣人议而不辩。

故分也者,有不分也;辩也者,有不辩也。曰:``何也?''``圣人怀之,众人辩之以相示也。故曰:辩也者,有不见也。''夫大道不称,大辩不言,大仁不仁,大廉不嗛,大勇不忮。道昭而不道,言辩而不及,仁常而不成,廉清而不信,勇忮而不成。五者圆而几向方矣!故知止其所不知,至矣。孰知不言之辩,不道之道?若有能知,此之谓天府。注焉而不满,酌焉而不竭,而不知其所由来,此之谓葆光。

故昔者尧问于舜曰:``我欲伐宗、脍、胥敖,南面而不释然。其故何也?''舜曰:``夫三子者,犹存乎蓬艾之间。若不释然何哉!昔者十日并出,万物皆照,而况德之进乎日者乎!''

啮缺问乎王倪曰:``子知物之所同是乎?''曰:``吾恶乎知之!''``子知子之所不知邪?''曰:``吾恶乎知之!''``然则物无知邪?''曰:``吾恶乎知之!虽然,尝试言之:庸讵知吾所谓知之非不知邪?庸讵知吾所谓不知之非知邪?且吾尝试问乎女:民湿寝则腰疾偏死,鳅然乎哉?木处则惴栗恂惧,猨猴然乎哉?三者孰知正处?民食刍豢,麋鹿食荐,蝍蛆甘带,鸱鸦耆鼠,四者孰知正味?猨猵狙以为雌,麋与鹿交,鳅与鱼游。毛嫱丽姬,人之所美也;鱼见之深入,鸟见之高飞,麋鹿见之决骤,四者孰知天下之正色哉?自我观之,仁义之端,是非之涂,樊然淆乱,吾恶能知其辩!''啮缺曰:``子不利害,则至人固不知利害乎?''王倪曰:``至人神矣!大泽焚而不能热,河汉冱而不能寒,疾雷破山、飘风振海而不能惊。若然者,乘云气,骑日月,而游乎四海之外,死生无变于己,而况利害之端乎!''

瞿鹊子问乎长梧子曰:``吾闻诸夫子:圣人不从事于务,不就利,不违害,不喜求,不缘道,无谓有谓,有谓无谓,而游乎尘垢之外。夫子以为孟浪之言,而我以为妙道之行也。吾子以为奚若?''

长梧子曰:``是皇帝之所听荧也,而丘也何足以知之!且女亦大早计,见卵而求时夜,见弹而求鸮炙。予尝为女妄言之,女以妄听之。奚旁日月,挟宇宙,为其脗合,置其滑涽,以隶相尊?众人役役,圣人愚芚,参万岁而一成纯。万物尽然,而以是相蕴。予恶乎知说生之非惑邪!予恶乎知恶死之非弱丧而不知归者邪!

丽之姬,艾封人之子也。晋国之始得之也,涕泣沾襟。及其至于王所,与王同筐床,食刍豢,而后悔其泣也。予恶乎知夫死者不悔其始之蕲生乎?梦饮酒者,旦而哭泣;梦哭泣者,旦而田猎。方其梦也,不知其梦也。梦之中又占其梦焉,觉而后知其梦也。且有大觉而后知此其大梦也,而愚者自以为觉,窃窃然知之。``君乎!牧乎!''固哉!丘也与女皆梦也,予谓女梦亦梦也。是其言也,其名为吊诡。万世之后而一遇大圣知其解者,是旦暮遇之也。

既使我与若辩矣,若胜我,我不若胜,若果是也?我果非也邪?我胜若,若不吾胜,我果是也?而果非也邪?其或是也?其或非也邪?其俱是也?其俱非也邪?我与若不能相知也。则人固受其黮暗,吾谁使正之?使同乎若者正之,既与若同矣,恶能正之?使同乎我者正之,既同乎我矣,恶能正之?使异乎我与若者正之,既异乎我与若矣,恶能正之?使同乎我与若者正之,既同乎我与若矣,恶能正之?然则我与若与人俱不能相知也,而待彼也邪?''

``何谓和之以天倪?''曰:``是不是,然不然。是若果是也,则是之异乎不是也亦无辩;然若果然也,则然之异乎不然也亦无辩。化声之相待,若其不相待。和之以天倪,因之以曼衍,所以穷年也。忘年忘义,振于无竟,故寓诸无竟。''

罔两问景曰:``曩子行,今子止;曩子坐,今子起。何其无特操与?''景曰:``吾有待而然者邪?吾所待又有待而然者邪?吾待蛇蚹蜩翼邪?恶识所以然?恶识所以不然?''

昔者庄周梦为胡蝶,栩栩然胡蝶也。自喻适志与!不知周也。俄然觉,则蘧蘧然周也。不知周之梦为胡蝶与?胡蝶之梦为周与?周与胡蝶则必有分矣。此之谓物化。

\hypertarget{header-n1989}{%
\subsubsection{养生主}\label{header-n1989}}

吾生也有涯,而知也无涯。以有涯随无涯,殆已!已而为知者,殆
而已矣!为善无近名,为恶无近刑,缘督以为经,可以保身,可以全
生,可以养亲,可以尽年。

庖丁为文惠君解牛,手之所触,肩之所倚,足之所履,膝之所倚,
砉然响然,奏刀騞然,莫不中音。合于《桑林》之舞,乃中《经首》之会。

文惠君曰:``嘻,善哉!技盍至此乎?''庖丁释刀对曰:``臣之所
好者道也,进乎技矣。始臣之解牛之时,所见无非全牛者;三年之后
,未尝见全牛也;方今之时,臣以神遇而不以目视,官知止而神欲行
。依乎天理,批大郤,导大髋,因其固然。技经肯綮之未尝,而况大
軱乎!良庖岁更刀,割也;族庖月更刀,折也;今臣之刀十九年矣,
所解数千牛矣,而刀刃若新发于硎。彼节者有间而刀刃者无厚,以无
厚入有间,恢恢乎其于游刃必有余地矣。是以十九年而刀刃若新发于
硎。虽然,每至于族,吾见其难为,怵然为戒,视为止,行为迟,动
刀甚微,謋然已解,如土委地。提刀而立,为之而四顾,为之踌躇满
志,善刀而藏之。''文惠君曰:``善哉!吾闻庖丁之言,得养生焉。''

公文轩见右师而惊曰:``是何人也?恶乎介也?天与?其人与?''
曰:``天也,非人也。天之生是使独也,人之貌有与也。以是知其天
也,非人也。''

泽雉十步一啄,百步一饮,不蕲畜乎樊中。神虽王,不善也。

老聃死,秦失吊之,三号而出。弟子曰:``非夫子之友邪?''曰:
``然。''``然则吊焉若此可乎?''曰:``然。始也吾以为其人也,而
今非也。向吾入而吊焉,有老者哭之,如哭其子;少者哭之,如哭其
母。彼其所以会之,必有不蕲言而言,不蕲哭而哭者。是遁天倍情,
忘其所受,古者谓之遁天之刑。适来,夫子时也;适去,夫子顺也。
安时而处顺,哀乐不能入也,古者谓是帝之县解。''

指穷于为薪,火传也,不知其尽也。

\#\#\#

\hypertarget{header-n2000}{%
\subsubsection{人间世}\label{header-n2000}}

颜回见仲尼,请行。曰:``奚之?''曰:``将之卫。''曰:``奚为
焉?''曰:``回闻卫君,其年壮,其行独。轻用其国而不见其过。轻
用民死,死者以国量,乎泽若蕉,民其无如矣!回尝闻之夫子曰:`
治国去之,乱国就之。医门多疾。'愿以所闻思其则,庶几其国有瘳 乎!''

仲尼曰:``嘻,若殆往而刑耳!夫道不欲杂,杂则多,多则扰,扰
则忧,忧而不救。古之至人,先存诸己而后存诸人。所存于己者未定
,何暇至于暴人之所行!且若亦知夫德之所荡而知之所为出乎哉?德
荡乎名,知出乎争。名也者,相札也;知也者争之器也。二者凶器,
非所以尽行也。

且德厚信矼,未达人气;名闻不争,未达人心。而强以仁义绳墨之
言術暴人之前者,是以人恶有其美也,命之曰灾人。灾人者,人必反
灾之。若殆为人灾夫。

且苟为人悦贤而恶不肖,恶用而求有以异?若唯无诏,王公必将乘
人而斗其捷。而目将荧之,而色将平之,口将营之,容将形之,心且
成之。是以火救火,以水救水,名之曰益多。顺始无穷,若殆以不信
厚言,必死于暴人之前矣!

且昔者桀杀关龙逢,纣杀王子比干,是皆修其身以下伛拊人之民,
以下拂其上者也,故其君因其修以挤之。是好名者也。

昔者尧攻丛枝、胥、敖,禹攻有扈。国为虚厉,身为刑戮。其用兵
不止,其求实无已,是皆求名实者也,而独不闻之乎?名实者,圣人
之所不能胜也,而况若乎!虽然,若必有以也,尝以语我来。''

颜回曰:``端而虚,勉而一,则可乎?''曰:``恶!恶可!夫以阳
为充孔扬,采色不定,常人之所不违,因案人之所感,以求容与其心
,名之曰日渐之德不成,而况大德乎!将执而不化,外合而内不訾,
其庸讵可乎!''

``然则我内直而外曲,成而上比。内直者,与天为徒。与天为徒者
,知天子之与己,皆天之所子,而独以己言蕲乎而人善之,蕲乎而人
不善之邪?若然者,人谓之童子,是之谓与天为徒。外曲者,与人之
为徒也。擎跽曲拳,人臣之礼也。人皆为之,吾敢不为邪?为人之所
为者,人亦无疵焉,是之谓与人为徒。成而上比者,与古为徒。其言
虽教,谪之实也,古之有也,非吾有也。若然者,虽直而不病,是之
谓与古为徒。若是则可乎?''仲尼曰:``恶!恶可!大多政法而不谍
。虽固,亦无罪。虽然,止是耳矣,夫胡可以及化!犹师心者也。''

颜回曰:``吾无以进矣,敢问其方。''仲尼曰:``斋,吾将语若。
有心而为之,其易邪?易之者,白囗(左``白''右上``白''右下``本''
音hao4)天不宜。''颜回曰:``回之家贫,唯不饮酒不茹荤者数
月矣。如此则可以为斋乎?''曰:``是祭祀之斋,非心斋也。''

回曰:``敢问心斋。''仲尼曰:``若一志,无听之以耳而听之以心
;无听之以心而听之以气。听止于耳,心止于符。气也者,虚而待物
者也。唯道集虚。虚者,心斋也''

颜回曰:``回之未始得使,实自回也;得使之也,未始有回也,可
谓虚乎?''夫子曰:``尽矣!吾语若:若能入游其樊而无感其名,入
则鸣,不入则止。无门无毒,一宅而寓于不得已则几矣。绝迹易,无
行地难。为人使易以伪,为天使难以伪。闻以有翼飞者矣,未闻以无
翼飞者也;闻以有知知者矣,未闻以无知知者也。瞻彼阕者,虚室生
白,吉祥止止。夫且不止,是之谓坐驰。夫徇耳目内通而外于心知,
鬼神将来舍,而况人乎!是万物之化也,禹、舜之所纽也,伏戏、几
蘧之所行终,而况散焉者乎!''

叶公子高将使于齐,问于仲尼曰:``王使诸梁也甚重。齐之待使者
,盖将甚敬而不急。匹夫犹未可动也,而况诸侯乎!吾甚栗之。子常
语诸梁也曰:`凡事若小若大,寡不道以欢成。事若不成,则必有人
道之患;事若成,则必有阴阳之患。若成若不成而后无患者,唯有德
者能之。'吾食也执粗而不臧,爨无欲清之人。今吾朝受命而夕饮冰
,我其内热与!吾未至乎事之情而既有阴阳之患矣!事若不成,必有
人道之患,是两也。为人臣者不足以任之,子其有以语我来!''

仲尼曰:``天下有大戒二:其一命也,其一义也。子之爱亲,命也
,不可解于心;臣之事君,义也,无适而非君也,无所逃于天地之间
。是之谓大戒。是以夫事其亲者,不择地而安之,孝之至也;夫事其
君者,不择事而安之,忠之盛也;自事其心者,哀乐不易施乎前,知
其不可奈何而安之若命,德之至也。为人臣子者,固有所不得已。行
事之情而忘其身,何暇至于悦生而恶死!夫子其行可矣!

丘请复以所闻:凡交近则必相靡以信,远则必忠之以言。言必或传
之。夫传两喜两怒之言,天下之难者也。夫两喜必多溢美之言,两怒
必多溢恶之言。凡溢之类妄,妄则其信之也莫,莫则传言者殃。故法
言曰:`传其常情,无传其溢言,则几乎全。'

且以巧斗力者,始乎阳,常卒乎阴,泰至则多奇巧;以礼饮酒者,
始乎治,常卒乎乱,泰至则多奇乐。凡事亦然,始乎谅,常卒乎鄙;
其作始也简,其将毕也必巨。言者,风波也;行者,实丧也。夫风波
易以动,实丧易以危。故忿设无由,巧言偏辞。兽死不择音,气息勃
然于是并生心厉。剋核太至,则必有不肖之心应之而不知其然也。苟
为不知其然也,孰知其所终!故法言曰:`无迁令,无劝成。过度益
也。'迁令劝成殆事。美成在久,恶成不及改,可不慎与!且夫乘物
以游心,托不得已以养中,至矣。何作为报也!莫若为致命,此其难 者?''
颜阖将傅卫灵公大子,而问于蘧伯玉曰;``有人于此,其德天杀。
与之为无方则危吾国,与之为有方则危吾身。其知适足以知人之过,
而不知其所以过。若然者,吾奈之何?''蘧伯玉曰:``善哉问乎!戒
之,慎之,正女身哉!形莫若就,心莫若和。虽然,之二者有患。就
不欲入,和不欲出。形就而入,且为颠为灭,为崩为蹶;心和而出,
且为声为名,为妖为孽。彼且为婴儿,亦与之为婴儿;彼且为无町畦
,亦与之为无町畦;彼且为无崖,亦与之为无崖;达之,入于无疵。

汝不知夫螳螂乎?怒其臂以当车辙,不知其不胜任也,是其才之美
者也。戒之,慎之,积伐而美者以犯之,几矣!

汝不知夫养虎者乎?不敢以生物与之,为其杀之之怒也;不敢以全
物与之,为其决之之怒也。时其饥饱,达其怒心。虎之与人异类,而
媚养己者,顺也;故其杀者,逆也。

夫爱马者,以筐盛矢,以蜃盛溺。适有蚊虻仆缘,而拊之不时,则
缺衔毁首碎胸。意有所至而爱有所亡。可不慎邪?''

匠石之齐,至于曲辕,见栎社树。其大蔽牛,絜之百围,其高临山
十仞而后有枝,其可以舟者旁十数。观者如市,匠伯不顾,遂行不辍。
弟子厌观之,走及匠石,曰:`自吾执斧斤以随夫子,未尝见材如此
其美也。先生不肯视,行不辍,何邪?''曰:``已矣,勿言之矣!散
木也。以为舟则沉,以为棺椁则速腐,以为器则速毁,以为门户则液
瞒,以为柱则蠹,是不材之木也。无所可用,故能若是之寿。''

匠石归,栎社见梦曰:``女将恶乎比予哉?若将比予于文木邪?夫
楂梨橘柚果蓏之属,实熟则剥,剥则辱。大枝折,小枝泄。此以其能
苦其生者也。故不终其天年而中道夭,自掊击于世俗者也。物莫不若
是。且予求无所可用久矣!几死,乃今得之,为予大用。使予也而有
用,且得有此大也邪?且也若与予也皆物也,奈何哉其相物也?而几
死之散人,又恶知散木!''匠石觉而诊其梦。弟子曰:``趣取无用,
则为社何邪?''曰:``密!若无言!彼亦直寄焉!以为不知己者诟厉
也。不为社者,且几有翦乎!且也彼其所保与众异,而以义喻之,不 亦远乎!''

南伯子綦游乎商之丘,见大木焉,有异:结驷千乘,隐,将芘其所
藾。子綦曰:``此何木也哉!此必有异材夫!''仰而视其细枝,则拳
曲而不可以为栋梁;俯而视其大根,则轴解而不可以为棺椁;舐其叶,
则口烂而为伤;嗅之,则使人狂醒三日而不已。子綦曰``此果不材之
木也,以至于此其大也。嗟乎,醒三日而不已。子綦曰:``此果不材
之木也,以至于此其大也。嗟乎,神人以此不材。''

宋有荆氏者,宜楸柏桑。其拱把而上者,求狙猴之杙斩之;三围四
围,求高名之丽者斩之;七围八围,贵人富商之家求禅傍者斩之。
故未终其天年而中道之夭于斧斤,此材之患也。故解之以牛之白颡
者,与豚之亢鼻者,与人有痔病者,不可以适河。此皆巫祝以知之
矣,所以为不祥也。此乃神人之所以为大祥也。

支离疏者,颐隐于齐,肩高于顶,会撮指天,五管在上,两髀为胁
。挫针治獬,足以囗(左``饣''右``胡'')口;鼓荚播精,足以食十人。上征武士,则支离攘臂于其间
;上有大役,则支离以有常疾不受功;上与病者粟,则受三锺与十束
薪。夫支离者其形者,犹足以养其身,终其天年,又况支离其德者乎 !

孔子适楚,楚狂接舆游其门曰:``凤兮凤兮,何如德之衰也。来世
不可待,往世不可追也。天下有道,圣人成焉;天下无道,圣人生焉
。方今之时,仅免刑焉!福轻乎羽,莫之知载;祸重乎地,莫之知避
。已乎,已乎!临人以德。殆乎,殆乎!画地而趋。迷阳迷阳,无伤
吾行。吾行郤曲,无伤吾足。''

山木,自寇也;膏火,自煎也。桂可食,故伐之;漆可用,故割之
。人皆知有用之用,而莫知无用之用也。

\hypertarget{header-n2029}{%
\subsubsection{德充符}\label{header-n2029}}

鲁有兀者王骀,从之游者与仲尼相若。常季问于仲尼曰:``王骀,
兀者也,从之游者与夫子中分鲁。立不教,坐不议。虚而往,实而归
。固有不言之教,无形而心成者邪?是何人也?''仲尼曰:``夫子,
圣人也,丘也直后而未往耳!丘将以为师,而况不若丘者乎!奚假鲁
国,丘将引天下而与从之。''

常季曰:``彼兀者也,而王先生,其与庸亦远矣。若然者,其用心
也,独若之何?''仲尼曰:``死生亦大矣,而不得与之变;虽天地覆
坠,亦将不与之遗;审乎无假而不与物迁,命物之化而守其宗也。''

常季曰:``何谓也?''仲尼曰:``自其异者视之,肝胆楚越也;自
其同者视之,万物皆一也。夫若然者,且不知耳目之所宜,而游心乎
德之和。物视其所一而不见其所丧,视丧其足犹遗土也。''

常季曰:``彼为己,以其知得其心,以其心得其常心。物何为最之
哉?''仲尼曰:``人莫鉴于流水而鉴于止水。唯止能止众止。受命于
地,唯松柏独也正,在冬夏青青;受命于天,唯尧、舜独也正,在万
物之首。幸能正生,以正众生。夫保始之徵,不惧之实,勇士一人,
雄入于九军。将求名而能自要者而犹若是,而况官天地、府万物、直
寓六骸、象耳目、一知之所知而心未尝死者乎!彼且择日而登假,人
则从是也。彼且何肯以物为事乎!''

申徒嘉,兀者也,而与郑子产同师于伯昏无人。子产谓申徒嘉曰:
``我先出则子止,子先出则我止。''其明日,又与合堂同席而坐。子
产谓申徒嘉曰:``我先出则子止,子先出则我止。今我将出,子可以
止乎?其未邪?且子见执政而不违,子齐执政乎?''申徒嘉曰:``先
生之门固有执政焉如此哉?子而说子之执政而后人者也。闻之曰:`
鉴明则尘垢不止,止则不明也。久与贤人处则无过。'今子之所取大
者,先生也,而犹出言若是,不亦过乎!''

子产曰:``子既若是矣,犹与尧争善。计子之德,不足以自反邪?
''申徒嘉曰:``自状其过以不当亡者众;不状其过以不当存者寡。知
不可奈何而安之若命,唯有德者能之。游于羿之彀中。中央者,中地
也;然而不中者,命也。人以其全足笑吾不全足者众矣,我怫然而怒
,而适先生之所,则废然而反。不知先生之洗我以善邪?吾之自寐邪
?吾与夫子游十九年,而未尝知吾兀者也。今子与我游于形骸之内,
而子索我于形骸之外,不亦过乎!''子产蹴然改容更貌曰:``子无乃 称!''

鲁有兀者叔山无趾,踵见仲尼。仲尼曰:``子不谨,前既犯患若是
矣。虽今来,何及矣!''无趾曰:``吾唯不知务而轻用吾身,吾是以
亡足。今吾来也,犹有尊足者存,吾是以务全之也。夫天无不覆,地
无不载,吾以夫子为天地,安知夫子之犹若是也!''孔子曰:``丘则
陋矣!夫子胡不入乎?请讲以所闻。''无趾出。孔子曰:``弟子勉之
!夫无趾,兀者也,犹务学以复补前行之恶,而况全德之人乎!''

无趾语老聃曰:``孔丘之于至人,其未邪?彼何宾宾以学子为?彼
且以蕲以諔诡幻怪之名闻,不知至人之以是为己桎梏邪?''老聃曰:
``胡不直使彼以死生为一条,以可不可为一贯者,解其桎梏,其可乎?
''无趾曰:``天刑之,安可解!''

鲁哀公问于仲尼曰:``卫有恶人焉,曰哀骀它。丈夫与之处者,思
而不能去也;妇人见之,请于父母曰:`与为人妻,宁为夫子妾'者
,数十而未止也。未尝有闻其唱者也,常和人而已矣。无君人之位以
济乎人之死,无聚禄以望人之腹,又以恶骇天下,和而不唱,知不出
乎四域,且而雌雄合乎前,是必有异乎人者也。寡人召而观之,果以
恶骇天下。与寡人处,不至以月数,而寡人有意乎其为人也;不至乎
期年,而寡人信之。国无宰,而寡人传国焉。闷然而后应,氾而若辞。
寡人丑乎,卒授之国。无几何也,去寡人而行。寡人恤焉若有亡也,
若无与乐是国也。是何人者也!''

仲尼曰:``丘也尝使于楚矣,适见豚子食于其死母者。少焉眴若,
皆弃之而走。不见己焉尔,不得其类焉尔。所爱其母者,非爱其形
也,爱使其形者也。战而死者,其人之葬也不以翣资;刖者之屡,无
为爱之。皆无其本矣。为天子之诸御:不爪翦,不穿耳;取妻者止于
外,不得复使。形全犹足以为尔,而况全德之人乎!今哀骀它未言而
信,无功而亲,使人授己国,唯恐其不受也,是必才全而德不形者也 。''

哀公曰:``何谓才全?''仲尼曰:``死生、存亡、穷达、贫富、贤
与不肖、毁誉、饥渴、寒暑,是事之变、命之行也。日夜相代乎前,
而知不能规乎其始者也。故不足以滑和,不可入于灵府。使之和豫,
通而不失于兑。使日夜无隙,而与物为春,是接而生时于心者也。是
之谓才全。''``何谓德不形?''曰:``平者,水停之盛也。其可以为
法也,内保之而外不荡也。德者,成和之修也。德不形者,物不能离 也。''

哀公异日以告闵子曰:``始也吾以南面而君天下,执民之纪而忧其
死,吾自以为至通矣。今吾闻至人之言,恐吾无其实,轻用吾身而亡
吾国。吾与孔丘非君臣也,德友而已矣!''

闉跂支离无脣说卫灵公,灵公说之,而视全人:其脰肩肩。甕
囗(上``央''下``瓦''音ang4)大瘿说齐桓公,桓公说之,而视
全人:其脰肩肩。故德有所长而形有所忘。人不忘其所忘而忘其所不
忘,此谓诚忘。

故圣人有所游,而知为孽,约为胶,德为接,工为商。圣人不谋,
恶用知?不斵,恶用胶?无丧,恶用德?不货,恶用商?四者,天鬻也
。天鬻者,天食也。既受食于天,又恶用人!

有人之形,无人之情。有人之形,故群于人;无人之情,故是非不
得于身。眇乎小哉,所以属于人也;謷乎大哉,独成其天。

惠子谓庄子曰:``人故无情乎?''庄子曰:``然。''惠子曰:``人
而无情,何以谓之人?''庄子曰:``道与之貌,天与之形,恶得不谓
之人?''惠子曰:``既谓之人,恶得无情?''庄子曰:``是非吾所谓
情也。吾所谓无情者,言人之不以好恶内伤其身,常因自然而不益生
也。''惠子曰:``不益生,何以有其身?''庄子曰:``道与之貌,天
与之形,无以好恶内伤其身。今子外乎子之神,劳乎子之精,倚树而
吟,据槁梧而瞑。天选子之形,子以坚白鸣。''

\hypertarget{header-n2049}{%
\subsubsection{大宗师}\label{header-n2049}}

知天之所为,知人之所为者,至矣!知天之所为者,天而生也;知
人之所为者,以其知之所知以养其知之所不知,终其天年而不中道夭
者,是知之盛也。虽然,有患:夫知有所待而后当,其所待者特未定
也。庸讵知吾所谓天之非人乎?所谓人之非天乎?且有真人而后有真 知。

何谓真人?古之真人,不逆寡,不雄成,不谟士。若然者,过而弗
悔,当而不自得也。若然者,登高不栗,入水不濡,入火不热,是知
之能登假于道者也若此。

古之真人,其寝不梦,其觉无忧,其食不甘,其息深深。真人之息
以踵,众人之息以喉。屈服者,其嗌言若哇。其耆欲深者,其天机浅 。

古之真人,不知说生,不知恶死。其出不欣,其入不距。翛然而往,
翛然而来而已矣。不忘其所始,不求其所终。受而喜之,忘而复之。
是之谓不以心捐道,不以人助天,是之谓真人。若然者,其心志,其
容寂,其颡頯。凄然似秋,暖然似春,喜怒通四时,与物有宜而莫知
其极。故圣人之用兵也,亡国而不失人心。利泽施乎万世,不为爱人。
故乐通物,非圣人也;有亲,非仁也;天时,非贤也;利害不通,非
君子也;行名失己,非士也;亡身不真,非役人也。若狐不偕、务光、
伯夷、叔齐、箕子、胥余、纪他、申徒狄,是役人之役,适人之适,
而不自适其适者也。

古之真人,其状义而不朋,若不足而不承;与乎其觚而不坚也,张
乎其虚而不华也;邴邴乎其似喜也,崔崔乎其不得已也,滀乎进我色
也,与乎止我德也,广乎其似世也,囗(上``敖''下``言'')乎其未
可制也,连乎其似好闭也,悗乎忘其言也。以刑为体,以礼为翼,以
知为时,以德为循。以刑为体者,绰乎其杀也;以礼为翼者,所以行
于世也;以知为时者,不得已于事也;以德为循者,言其与有足者至
于丘也,而人真以为勤行者也。故其好之也一,其弗好之也一。其一
也一,其不一也一。其一与天为徒,其不一与人为徒,天与人不相胜
也,是之谓真人。

死生,命也;其有夜旦之常,天也。人之有所不得与,皆物之情也
。彼特以天为父,而身犹爱之,而况其卓乎!人特以有君为愈乎己,
而身犹死之,而况其真乎!

泉涸,鱼相与处于陆,相呴以湿,相濡以沫,不如相忘于江湖。与
其誉尧而非桀也,不如两忘而化其道。

夫大块载我以形,劳我以生,佚我以老,息我以死。故善吾生者,
乃所以善吾死也。夫藏舟于壑,藏山于泽,谓之固矣!然而夜半有力
者负之而走,昧者不知也。藏小大有宜,犹有所循。若夫藏天下于天
下而不得所循,是恒物之大情也。特犯人之形而犹喜之。若人之形者
,万化而未始有极也,其为乐可胜计邪?故圣人将游于物之所不得循
而皆存。善妖善老,善始善终,人犹效之,而况万物之所系而一化之 所待乎!

夫道有情有信,无为无形;可传而不可受,可得而不可见;自本自
根,未有天地,自古以固存;神鬼神帝,生天生地;在太极之先而不
为高,在六极之下而不为深,先天地生而不为久,长于上古而不为老
。豨韦氏得之,以挈天地;伏戏氏得之,以袭气母;维斗得之,终古
不忒;日月得之,终古不息;勘坏得之,以袭昆仑;冯夷得之,以游
大川;肩吾得之,以处大山;黄帝得之,以登云天;颛顼得之,以处
玄宫;禺强得之,立乎北极;西王母得之,坐乎少广,莫知其始,莫
知其终;彭祖得之,上及有虞,下及及五伯;傅说得之,以相武丁,
奄有天下,乘东维、骑箕尾而比于列星。

南伯子葵问乎女偊曰:``子之年长矣,而色若孺子,何也?''曰:
``吾闻道矣。''南伯子葵曰:``道可得学邪?''曰:``恶!恶可!子
非其人也。夫卜梁倚有圣人之才而无圣人之道,我有圣人之道而无圣
人之才。吾欲以教之,庶几其果为圣人乎?不然,以圣人之道告圣人
之才,亦易矣。吾犹守而告之,参日而后能外天下;已外天下矣,吾
又守之,七日而后能外物;已外物矣,吾又守之,九日而后能外生;
已外生矣,而后能朝彻;朝彻而后能见独;见独而后能无古今;无古
今而后能入于不死不生。杀生者不死,生生者不生。其为物无不将也,
无不迎也,无不毁也,无不成也。其名为撄宁。撄宁也者,撄而后成 者也。''

南伯子葵曰:``子独恶乎闻之?''曰:``闻诸副墨之子,副墨之子
闻诸洛诵之孙,洛诵之孙闻之瞻明,瞻明闻之聂许,聂许闻之需役,
需役闻之于讴,于讴闻之玄冥,玄冥闻之参寥,参寥闻之疑始。''

子祀、子舆、子犁、子来四人相与语曰:``孰能以无为首,以生为
脊,以死为尻;孰知死生存亡之一体者,吾与之友矣!''四人相视而
笑,莫逆于心,遂相与为友。俄而子舆有病,子祀往问之。曰:``伟
哉,夫造物者将以予为此拘拘也。''曲偻发背,上有五管,颐隐于齐
,肩高于顶,句赘指天,阴阳之气有沴,其心闲而无事,胼囗(左``
足''右``鲜''音xian1)而鉴于井,曰:``嗟乎!夫造物者又将
以予为此拘拘也。''

子祀曰:``女恶之乎?''曰:``亡,予何恶!浸假而化予之左臂以
为鸡,予因以求时夜;浸假而化予之右臂以为弹,予因以求鸮炙;浸
假而化予之尻以为轮,以神为马,予因以乘之,岂更驾哉!且夫得者,
时也;失者,顺也。安时而处顺,哀乐不能入也,此古之所谓县解也,
而不能自解者,物有结之。且夫物不胜天久矣,吾又何恶焉!''

俄而子来有病,喘喘然将死。其妻子环而泣之。子犁往问之,曰:
``叱!避!无怛化!''倚其户与之语曰:``伟哉造化!又将奚以汝为
?将奚以汝适?以汝为鼠肝乎?以汝为虫臂乎?''子来曰:``父母于
子,东西南北,唯命之从。阴阳于人,不翅于父母。彼近吾死而我不
听,我则悍矣,彼何罪焉?夫大块以载我以形,劳我以生,佚我以老
,息我以死。故善吾生者,乃所以善吾死也。今大冶铸金,金踊跃曰
:`我且必为镆铘!'大冶必以为不祥之金。今一犯人之形而曰:`
人耳!人耳!'夫造化者必以为不祥之人。今一以天地为大炉,以造
化为大冶,恶乎往而不可哉!''成然寐,蘧然觉。

子桑户、孟子反、子琴张三人相与友曰:``孰能相与于无相与,相
为于无相为;孰能登天游雾,挠挑无极,相忘以生,无所穷终!''三
人相视而笑,莫逆于心,遂相与友。

莫然有间,而子桑户死,未葬。孔子闻之,使子贡往侍事焉。或编
曲,或鼓琴,相和而歌曰:``嗟来桑户乎!嗟来桑户乎!而已反其真
,而我犹为人猗!''子贡趋而进曰:``敢问临尸而歌,礼乎?''二人
相视而笑曰:``是恶知礼意!''子贡反,以告孔子曰:``彼何人者邪
?修行无有而外其形骸,临尸而歌,颜色不变,无以命之。彼何人者
邪?''孔子曰:``彼游方之外者也,而丘游方之内者也。外内不相及
,而丘使女往吊之,丘则陋矣!彼方且与造物者为人,而游乎天地之
一气。彼以生为附赘县疣,以死为决囗(``病''字以``丸''代``丙''
音huan4)溃痈。夫若然者,又恶知死生先后之所在!假于异物
,托于同体;忘其肝胆,遗其耳目;反复终始,不知端倪;芒然仿徨
乎尘垢之外,逍遥乎无为之业。彼又恶能愦愦然为世俗之礼,以观众
人之耳目哉!''

子贡曰:``然则夫子何方之依?''孔子曰:``丘,天之戮民也。虽
然,吾与汝共之。''子贡曰:``敢问其方?''孔子曰:``鱼相造乎水
,人相造乎道。相造乎水者,穿池而养给;相造乎道者,无事而生定
。故曰:鱼相忘乎江湖,人相忘乎道术。''子贡曰:``敢问畸人?''
曰:``畸人者,畸于人而侔于天。故曰:天之小人,人之君子;人之
君子,天之小人也。''

颜回问仲尼曰:``孟孙才,其母死,哭泣无涕,中心不戚,居丧不
哀。无是三者,以善处丧盖鲁国,固有无其实而得其名者乎?回壹怪
之。''仲尼曰:``夫孟孙氏尽之矣,进于知矣,唯简之而不得,夫已
有所简矣。孟孙氏不知所以生,不知所以死。不知就先,不知就后。
若化为物,以待其所不知之化已乎。且方将化,恶知不化哉?方将不
化,恶知已化哉?吾特与汝,其梦未始觉者邪!且彼有骇形而无损心
,有旦宅而无情死。孟孙氏特觉,人哭亦哭,是自其所以乃。且也相
与`吾之'耳矣,庸讵知吾所谓`吾之'乎?且汝梦为鸟而厉乎天,
梦为鱼而没于渊。不识今之言者,其觉者乎?其梦者乎?造适不及笑
,献笑不及排,安排而去化,乃入于寥天一。''

意而子见许由,许由曰:``尧何以资汝?''意而子曰:``尧谓我:
汝必躬服仁义而明言是非。''许由曰:``而奚来为轵?夫尧既已黥汝
以仁义,而劓汝以是非矣。汝将何以游夫遥荡恣睢转徙之涂乎?''

意而子曰:``虽然,吾愿游于其藩。''许由曰:``不然。夫盲者无
以与乎眉目颜色之好,瞽者无以与乎青黄黼黻之观。''意而子曰:``
夫无庄之失其美,据梁之失其力,黄帝之亡其知,皆在炉捶之间耳。
庸讵知夫造物者之不息我黥而补我劓,使我乘成以随先生邪?''许由
曰:``噫!未可知也。我为汝言其大略:吾师乎!吾师乎!赍万物而
不为义,泽及万世而不为仁,长于上古而不为老,覆载天地、刻雕众
形而不为巧。此所游已!

颜回曰:``回益矣。''仲尼曰:``何谓也?''曰:``回忘仁义矣。
''曰:``可矣,犹未也。''他日复见,曰:``回益矣。''曰:``何谓
也?''曰:``回忘礼乐矣!''曰:``可矣,犹未也。''他日复见,曰
:``回益矣!''曰:``何谓也?''曰:``回坐忘矣。''仲尼蹴然曰:
``何谓坐忘?''颜回曰:``堕肢体,黜聪明,离形去知,同于大通,
此谓坐忘。''仲尼曰:``同则无好也,化则无常也。而果其贤乎!丘
也请从而后也。''

子舆与子桑友。而霖雨十日,子舆曰:``子桑殆病矣!''裹饭而往
食之。至子桑之门,则若歌若哭,鼓琴曰:``父邪!母邪!天乎!人
乎!''有不任其声而趋举其诗焉。子舆入,曰:``子之歌诗,何故若
是?''曰:``吾思夫使我至此极者而弗得也。父母岂欲吾贫哉?天无
私覆,地无私载,天地岂私贫我哉?求其为之者而不得也!然而至此
极者,命也夫!''

\hypertarget{header-n2075}{%
\subsubsection{应帝王}\label{header-n2075}}

啮缺问于王倪,四问而四不知。啮缺因跃而大喜,行以告蒲衣子。
蒲衣子曰:``而乃今知之乎?有虞氏不及泰氏。有虞氏其犹藏仁以要
人,亦得人矣,而未始出于非人。泰氏其卧徐徐,其觉于于。一以己
为马,一以己为牛。其知情信,其德甚真,而未始入于非人。''

肩吾见狂接舆。狂接舆曰:``日中始何以语女?''肩吾曰:``告我
:君人者以己出经式义度,人孰敢不听而化诸!''狂接舆曰:``是欺
德也。其于治天下也,犹涉海凿河而使蚊负山也。夫圣人之治也,治
外夫?正而后行,确乎能其事者而已矣。且鸟高飞以避矰弋之害,鼷
鼠深穴乎神丘之下以避熏凿之患,而曾二虫之无知?''

天根游于殷阳,至蓼水之上,适遭无名人而问焉,曰:``请问为天
下。''无名人曰:``去!汝鄙人也,何问之不豫也!予方将与造物者
为人,厌则又乘夫莽眇之鸟,以出六极之外,而游无何有之乡,以处
圹埌之野。汝又何帛以治天下感予之心为?''又复问,无名人曰:
``汝游心于淡,合气于漠,顺物自然而无容私焉,而天下治矣。''

阳子居见老聃,曰:``有人于此,向疾强梁,物彻疏明,学道不倦
,如是者,可比明王乎?''老聃曰:``是於圣人也,胥易技系,劳形
怵心者也。且也虎豹之文来田,猨狙之便执嫠之狗来藉。如是者,可
比明王乎?''阳子居蹴然曰:``敢问明王之治。''老聃曰:``明王之
治:功盖天下而似不自己,化贷万物而民弗恃。有莫举名,使物自喜。
立乎不测,而游于无有者也。''

郑有神巫曰季咸,知人之死生、存亡、祸福、寿夭,期以岁月旬日
若神。郑人见之,皆弃而走。列子见之而心醉,归,以告壶子,曰:
``始吾以夫子之道为至矣,则又有至焉者矣。''壶子曰:``吾与汝既
其文,未既其实。而固得道与?众雌而无雄,而又奚卵焉!而以道与
世亢,必信,夫故使人得而相汝。尝试与来,以予示之。''

明日,列子与之见壶子。出而谓列子曰:``嘻!子之先生死矣!弗
活矣!不以旬数矣!吾见怪焉,见湿灰焉。''列子入,泣涕沾襟以告
壶子。壶子曰:``乡吾示之以地文,萌乎不震不正,是殆见吾杜德机
也。尝又与来。''明日,又与之见壶子。出而谓列子曰:``幸矣!子
之先生遇我也,有瘳矣!全然有生矣!吾见其杜权矣!''列子入,以
告壶子。壶子曰:``乡吾示之以天壤,名实不入,而机发于踵。是殆
见吾善者机也。尝又与来。''明日,又与之见壶子。出而谓列子曰:
``子之先生不齐,吾无得而相焉。试齐,且复相之。''列子入,以告
壶子。壶子曰:``吾乡示之以以太冲莫胜,是殆见吾衡气机也。鲵桓
之审为渊,止水之审为渊,流水之审为渊。渊有九名,此处三焉。尝
又与来。''明日,又与之见壶子。立未定,自失而走。壶子曰:``追
之!''列子追之不及。反,以报壶子曰:``已灭矣,已失矣,吾弗及
已。''壶子曰:``乡吾示之以未始出吾宗。吾与之虚而委蛇,不知其
谁何,因以为弟靡,因以为波流,故逃也。''然后列子自以为未始学
而归。三年不出,为其妻爨,食豕如食人,于事无与亲。雕琢复朴,
块然独以其形立。纷而封哉,一以是终。

无为名尸,无为谋府,无为事任,无为知主。体尽无穷,而游无朕
。尽其所受乎天而无见得,亦虚而已!至人之用心若镜,不将不逆,
应而不藏,故能胜物而不伤。

南海之帝为儵北海之帝为忽,中央之帝为浑沌。儵与忽时相与遇于
浑沌之地,浑沌待之甚善。儵与忽谋报浑沌之德,曰:``人皆有七窍
以视听食息此独无有,尝试凿之。''日凿一窍,七日而浑沌死。

\hypertarget{header-n2087}{%
\subsection{外篇}\label{header-n2087}}

\hypertarget{header-n2088}{%
\subsubsection{骈拇}\label{header-n2088}}

骈拇枝指出乎性哉,而侈于德;附赘县疣出乎形哉,而侈于性;多
方乎仁义而用之者,列于五藏哉,而非道德之正也。是故骈于足者,
连无用之肉也;枝于手者,树无用之指也;多方骈枝于五藏之情者,
淫僻于仁义之行,而多方于聪明之用也。

是故骈于明者,乱五色,淫文章,青黄黼黻之煌煌非乎?而离朱是
已!多于聪者,乱五声,淫六律,金石丝竹黄钟大吕之声非乎?而师
旷是已!枝于仁者,擢德塞性以收名声,使天下簧鼓以奉不及之法非
乎?而曾、史是已!骈于辩者,累瓦结绳窜句,游心于坚白同异之间
,而敝跬誉无用之言非乎?而杨、墨是已!故此皆多骈旁枝之道,非
天下之至正也。

彼正正者,不失其性命之情。故合者不为骈,而枝者不为跂;长者
不为有余,短者不为不足。是故凫胫虽短,续之则忧;鹤胫虽长,断
之则悲。故性长非所断,性短非所续,无所去忧也。

意仁义其非人情乎!彼仁人何其多忧也。且夫骈于拇者,决之则泣
;枝于手者,齕之则啼。二者或有余于数,或不足于数,其于忧一也。
今世之仁人,蒿目而忧世之患;不仁之人,决性命之情而饕贵富。故意
仁义其非人情乎!自三代以下者,天下何其嚣嚣也。

且夫待钩绳规矩而正者,是削其性者也;待绳约胶漆而固者,是侵
其德者也;屈折礼乐,呴俞仁义,以慰天下之心者,此失其常然也。
天下有常然。常然者,曲者不以钩,直者不以绳,圆者不以规,方者
不以矩,附离不以胶漆,约束不以纆索。故天下诱然皆生,而不知其
所以生;同焉皆得,而不知其所以得。故古今不二,不可亏也。则仁
义又奚连连如胶漆纆索而游乎道德之间为哉!使天下惑也!

夫小惑易方,大惑易性。何以知其然邪?自虞氏招仁义以挠天下也
,天下莫不奔命于仁义。是非以仁义易其性与?

故尝试论之:自三代以下者,天下莫不以物易其性矣!小人则以身
殉利;士则以身殉名;大夫则以身殉家;圣人则以身殉天下。故此数
子者,事业不同,名声异号,其于伤性以身为殉,一也。

臧与谷,二人相与牧羊而俱亡其羊。问臧奚事,则挟策读书;问谷
奚事,则博塞以游。二人者,事业不同,其于亡羊均也。

伯夷死名于首阳之下,盗跖死利于东陵之上。二人者,所死不同,
其于残生伤性均也。奚必伯夷之是而盗跖之非乎?

天下尽殉也:彼其所殉仁义也,则俗谓之君子;其所殉货财也,则
俗谓之小人。其殉一也,则有君子焉,有小人焉。若其残生损性,则
盗跖亦伯夷已,又恶取君子小人于其间哉!

且夫属其性乎仁义者,虽通如曾、史,非吾所谓臧也;属其性于五
味,虽通如俞儿,非吾所谓臧也;属其性乎五声,虽通如师旷,非吾
所谓聪也;属其性乎五色,虽通如离朱,非吾所谓明也。吾所谓臧者
,非所谓仁义之谓也,臧于其德而已矣;吾所谓臧者,非所谓仁义之
谓也,任其性命之情而已矣;吾所谓聪者,非谓其闻彼也,自闻而已
矣;吾所谓明者,非谓其见彼也,自见而已矣。夫不自见而见彼,不
自得而得彼者,是得人之得而不自得其得者也,适人之适而不自适其
适者也。夫适人之适而不自适其适,虽盗跖与伯夷,是同为淫僻也。
余愧乎道德,是以上不敢为仁义之操,而下不敢为淫僻之行也。

\hypertarget{header-n2103}{%
\subsubsection{马蹄}\label{header-n2103}}

马,蹄可以践霜雪,毛可以御风寒。齕草饮水,翘足而陆,此马之
真性也。虽有义台路寝,无所用之。及至伯乐,曰:``我善治马。''
烧之,剔之,刻之,雒之。连之以羁絷,编之以皂栈,马之死者十二
三矣!饥之渴之,驰之骤之,整之齐之,前有橛饰之患,而后有鞭
生筴之威,而马之死者已过半矣!陶者曰:``我善治埴。''圆者中规,
方者中矩。匠人曰:``我善治木。''曲者中钩,直者应绳。夫埴木之
性,岂欲中规矩钩绳哉!然且世世称之曰:``伯乐善治马,而陶匠善
治埴木。''此亦治天下者之过也。

吾意善治天下者不然。彼民有常性,织而衣,耕而食,是谓同德。
一而不党,命曰天放。故至德之世,其行填填,其视颠颠。当是时也
,山无蹊隧,泽无舟梁;万物群生,连属其乡;禽兽成群,草木遂长
。是故禽兽可系羁而游,鸟鹊之巢可攀援而窥。夫至德之世,同与禽
兽居,族与万物并。恶乎知君子小人哉!同乎无知,其德不离;同乎
无欲,是谓素朴。素朴而民性得矣。及至圣人,蹩躠为仁,踶跂为义,
而天下始疑矣。澶漫为乐,摘僻为礼,而天下始分矣。故纯朴不残,
孰为牺尊!白玉不毁,孰为珪璋!道德不废,安取仁义!性情不离,
安用礼乐!五色不乱,孰为文采!五声不乱,孰应六律!

夫残朴以为器,工匠之罪也;毁道德以为仁义,圣人之过也。夫马
陆居则食草饮水,喜则交颈相靡,怒则分背相踢。马知已此矣!夫加
之以衡扼,齐之以月题,而马知介倪闉扼鸷曼诡衔窃辔。故马之知而
能至盗者,伯乐之罪也。夫赫胥氏之时,民居不知所为,行不知所之,
含哺而熙,鼓腹而游。民能已此矣!及至圣人,屈折礼乐以匡天下之
形,县跂仁义以慰天下之心,而民乃始踶跂好知,争归于利,不可止
也。此亦圣人之过也。

\hypertarget{header-n2110}{%
\subsubsection{胠箧}\label{header-n2110}}

将为胠箧探囊发匮之盗而为守备,则必摄缄藤,固扃鐍,此世俗之
所谓知也。然而巨盗至,则负匮揭箧担囊而趋,唯恐缄藤扃鐍之不
固也。然则乡之所谓知者,不乃为大盗积者也?

故尝试论之:世俗之所谓知者,有不为大盗积者乎?所谓圣者,有
不为大盗守者乎?何以知其然邪?昔者齐国邻邑相望,鸡狗之音相闻
,罔罟之所布,耒耨之所刺,方二千余里。阖四竟之内,所以立宗庙
社稷,治邑屋州闾乡曲者,曷尝不法圣人哉?然而田成子一旦杀齐君
而盗其国,所盗者岂独其国邪?并与其圣知之法而盗之,故田成子有
乎盗贼之名,而身处尧舜之安。小国不敢非,大国不敢诛,十二世有
齐国,则是不乃窃齐国并与其圣知之法以守其盗贼之身乎?

尝试论之:世俗之所谓至知者,有不为大盗积者乎?所谓至圣者,
有不为大盗守者乎?何以知其然邪?昔者龙逢斩,比干剖,苌弘胣,
子胥靡。故四子之贤而身不免乎戮。故跖之徒问于跖曰:``盗亦有道
乎?''跖曰:``何适而无有道邪?夫妄意室中之藏,圣也;入先,勇
也;出后,义也;知可否,知也;分均,仁也。五者不备而能成大盗
者,天下未之有也。''由是观之,善人不得圣人之道不立,跖不得圣
人之道不行。天下之善人少而不善人多,则圣人之利天下也少而害天
下也多。故曰:唇竭则齿寒,鲁酒薄而邯郸围,圣人生而大盗起。掊
击圣人,纵舍盗贼,而天下始治矣。

夫川竭而谷虚,丘夷而渊实。圣人已死,则大盗不起,天下平而无
故矣!圣人不死,大盗不止。虽重圣人而治天下,则是重利盗跖也。
为之斗斛以量之,则并与斗斛而窃之;为之权衡以称之,则并与权衡
而窃之;为之符玺以信之,则并与符玺而窃之;为之仁义以矫之,则
并与仁义而窃之。何以知其然邪?彼窃钩者诛,窃国者为诸侯,诸侯之门而仁义存焉,则是非窃仁义圣知邪?故逐于大盗,揭诸侯,窃仁
义并斗斛权衡符玺之利者,虽有轩冕之赏弗能劝,斧钺之威弗能禁。
此重利盗跖而使不可禁者,是乃圣人之过也。

故曰:``鱼不可脱于渊,国之利器不可以示人。''彼圣人者,天下
之利器也,非所以明天下也。故绝圣弃知,大盗乃止;掷玉毁珠,小
盗不起;焚符破玺,而民朴鄙;掊斗折衡,而民不争;殚残天下之圣
法,而民始可与论议;擢乱六律,铄绝竽瑟,塞瞽旷之耳,而天下始
人含其聪矣;灭文章,散五采,胶离朱之目,而天下始人含其明矣。
毁绝钩绳而弃规矩,囗(左``扌''右``丽'')工倕之指,而天下始人
有其巧矣。故曰:大巧若拙。削曾、史之行,钳杨、墨之口,攘弃仁
义,而天下之德始玄同矣。彼人含其明,则天下不铄矣;人含其聪,
则天下不累矣;人含其知,则天下不惑矣;人含其德,则天下不僻矣。
彼曾、史、杨、墨、师旷、工倕、离朱者,皆外立其德而爚乱天下者
也,法之所无用也。

子独不知至德之世乎?昔者容成氏、大庭氏、伯皇氏、中央氏、栗
陆氏、骊畜氏、轩辕氏、赫胥氏、尊卢氏、祝融氏、伏戏氏、神农氏
,当是时也,民结绳而用之。甘其食,美其服,乐其俗,安其居,邻
国相望,鸡狗之音相闻,民至老死而不相往来。若此之时,则至治已
。今遂至使民延颈举踵,曰``某所有贤者'',赢粮而趣之,则内弃其
亲而外去其主之事,足迹接乎诸侯之境,车轨结乎千里之外。则是上
好知之过也!

上诚好知而无道,则天下大乱矣!何以知其然邪?夫弓弩毕弋机变
之知多,则鸟乱于上矣;钩饵罔罟罾笱之知多,则鱼乱于水矣;削格
罗落罯罘之知多,则兽乱于泽矣;知诈渐毒、颉滑坚白、解垢同异之
变多,则俗惑于辩矣。故天下每每大乱,罪在于好知。故天下皆知求
其所不知而莫知求其所已知者,皆知非其所不善而莫知非其所已善者,
是以大乱。故上悖日月之明,下烁山川之精,中堕四时之施,惴耎之
虫,肖翘之物,莫不失其性。甚矣,夫好知之乱天下也!自三代以下
者是已!舍夫种种之机而悦夫役役之佞;释夫恬淡无为而悦夫啍啍之
意,啍啍已乱天下矣!

\hypertarget{header-n2121}{%
\subsubsection{在宥}\label{header-n2121}}

闻在宥天下,不闻治天下也。在之也者,恐天下之淫其性也;宥之
也者,恐天下之迁其德也。天下不淫其性,不迁其德,有治天下者哉
?昔尧之治天下也,使天下欣欣焉人乐其性,是不恬也;桀之治天下
也,使天下瘁瘁焉人苦其性,是不愉也。夫不恬不愉。非德也;非德
也而可长久者,天下无之。

人大喜邪,毗于阳;大怒邪,毗于阴。阴阳并毗,四时不至,寒暑
之和不成,其反伤人之形乎!使人喜怒失位,居处无常,思虑不自得
,中道不成章。于是乎天下始乔诘卓鸷,而后有盗跖、曾、史之行。
故举天下以赏其善者不足,举天下以罚其恶者不给。故天下之大不足
以赏罚。自三代以下者,匈匈焉终以赏罚为事,彼何暇安其性命之情 哉!

而且说明邪,是淫于色也;说聪邪,是淫于声也;说仁邪,是乱于
德也;说义邪,是悖于理也;说礼邪,是相于技也;说乐邪,是相于
淫也;说圣邪,是相于艺也;说知邪,是相于疵也。天下将安其性命
之情,之八者,存可也,亡可也。天下将不安其性命之情,之八者,
乃始脔卷囗(左``犭''右``仓'')囊而乱天下也。而天下乃始尊之惜
之。甚矣,天下之惑也!岂直过也而去之邪!乃齐戒以言之,跪坐以
进之,鼓歌以余儛之。吾若是何哉!

故君子不得已而临莅天下,莫若无为。无为也,而后安其性命之情
。故贵以身于为天下,则可以托天下;爱以身于为天下,则可以寄天
下。故君子苟能无解其五藏,无擢其聪明,尸居而龙见,渊默而雷声
,神动而天随,从容无为而万物炊累焉。吾又何暇治天下哉!

崔瞿问于老聃曰:``不治天下,安藏人心?''老聃曰:``女慎,无
撄人心。人心排下而进上,上下囚杀,淖约柔乎刚强,廉刿雕琢,其
热焦火,其寒凝冰,其疾俯仰之间而再抚四海之外。其居也,渊而静
;其动也,县而天。偾骄而不可系者,其唯人心乎!昔者黄帝始以仁
义撄人之心,尧、舜于是乎股无胈,胫无毛,以养天下之形。愁其五
藏以为仁义,矜其血气以规法度。然犹有不胜也。尧于是放囗调节讙
兜于崇山,投三苗于三峗,流共工于幽都,此不胜天下也。夫施及三
王而天下大骇矣。下有桀、跖,上有曾、史,而儒墨毕起。于是乎喜
怒相疑,愚知相欺,善否相非,诞信相讥,而天下衰矣;大德不同,
而性命烂漫矣;天下好知,而百姓求竭矣。于是乎斤锯制焉,绳墨杀
焉,椎凿决焉。天下脊脊大乱,罪在撄人心。故贤者伏处大山嵁岩之
下,而万乘之君忧栗乎庙堂之上。今世殊死者相枕也,桁杨者相推也
,形戮者相望也,而儒墨乃始离跂攘臂乎桎梏之间。意,甚矣哉!其
无愧而不知耻也甚矣!吾未知圣知之不为桁杨椄槢也,仁义之不为桎
梏凿枘也,焉知曾、史之不为桀、跖嚆矢也!故曰:绝圣弃知,而天 下大治。

黄帝立为天子十九年,令行天下,闻广成子在于空同之上,故往见
之,曰:``我闻吾子达于至道,敢问至道之精。吾欲取天地之精,以
佐五谷,以养民人。吾又欲官阴阳以遂群生,为之奈何?''广成子曰
:``而所欲问者,物之质也;而所欲官者,物之残也。自而治天下,
云气不待族而雨,草木不待黄而落,日月之光益以荒矣,而佞人之心
翦翦者,又奚足以语至道!''黄帝退,捐天下,筑特室,席白茅,闲
居三月,复往邀之。广成子南首而卧,黄帝顺下风膝行而进,再拜稽
首而问曰:``闻吾子达于至道,敢问:治身奈何而可以长久?''广成
子蹶然而起,曰:``善哉问乎!来,吾语女至道:至道之精,窈窈冥
冥;至道之极,昏昏默默。无视无听,抱神以静,形将自正。必静必
清,无劳女形,无摇女精,乃可以长生。目无所见,耳无所闻,心无
所知,女神将守形,形乃长生。慎女内,闭女外,多知为败。我为女
遂于大明之上矣,至彼至阳之原也;为女入于窈冥之门矣,至彼至阴
之原也。天地有官,阴阳有藏。慎守女身,物将自壮。我守其一以处
其和。故我修身千二百岁矣,吾形未常衰。''黄帝再拜稽首曰:``广
成子之谓天矣!''广成子曰:``来!余语女:彼其物无穷,而人皆以
为有终;彼其物无测,而人皆以为有极。得吾道者,上为皇而下为王
;失吾道者,上见光而下为土。今夫百昌皆生于土而反于土。故余将
去女,入无穷之门,以游无极之野。吾与日月参光,吾与天地为常。
当我缗乎,远我昏乎!人其尽死,而我独存乎!''

云将东游,过扶摇之枝而适遭鸿蒙。鸿蒙方将拊脾雀跃而游。云将
见之,倘然止,贽然立,曰:``叟何人邪?叟何为此?''鸿蒙拊脾雀
跃不辍,对云将曰:``游!''云将曰:``朕愿有问也。''鸿蒙仰而视
云将曰:``吁!''云将曰:``天气不和,地气郁结,六气不调,四时
不节。今我愿合六气之精以育群生,为之奈何?''鸿蒙拊脾雀跃掉头
曰:``吾弗知!吾弗知!''云将不得问。又三年,东游,过有宋之野
,而适遭鸿蒙。云将大喜,行趋而进曰:``天忘朕邪?天忘朕邪?''
再拜稽首,愿闻于鸿蒙。鸿蒙曰:``浮游不知所求,猖狂不知所往,
游者鞅掌,以观无妄。朕又何知!''云将曰:``朕也自以为猖狂,而
民随予所往;朕也不得已于民,今则民之放也!愿闻一言。''鸿蒙曰
:``乱天之经,逆物之情,玄天弗成,解兽之群而鸟皆夜鸣,灾及草
木,祸及止虫。意!治人之过也。''云将曰:``然则吾奈何?''鸿蒙
曰:``意!毒哉!僊僊乎归矣!''云将曰:``吾遇天难,愿闻一言。''
鸿蒙曰:``意!心养!汝徒处无为,而物自化。堕尔形体,吐尔聪明,
伦与物忘,大同乎涬溟。解心释神,莫然无魂。万物云云,各复其根,
各复其根而不知。浑浑沌沌,终身不离。若彼知之,乃是离之。无问
其名,无窥其情,物固自生。''云将曰:``天降朕以德,示朕以默。
躬身求之,乃今得也。''再拜稽首,起辞而行。

世俗之人,皆喜人之同乎己而恶人之异于己也。同于己而欲之,异
于己而不欲者,以出乎众为心也。夫以出乎众为心者,曷常出乎众哉
?因众以宁所闻,不如众技众矣。而欲为人之国者,此揽乎三王之利
而不见其患者也。此以人之国侥幸也。几何侥幸而不丧人之国乎?其
存人之国也,无万分之一;而丧人之国也,一不成而万有余丧矣!悲
夫,有土者之不知也!夫有土者,有大物也。有大物者,不可以物。
物而不物,故能物物。明乎物物者之非物也,岂独治天下百姓而已哉
!出入六合,游乎九州,独往独来,是谓独有。独有之人,是之谓至 贵。

大人之教,若形之于影,声之于响,有问而应之,尽其所怀,为天
下配。处乎无响。行乎无方。挈汝适复之,挠挠以游无端,出入无旁
,与日无始。颂论形躯,合乎大同。大同而无己。无己,恶乎得有有
。睹有者,昔之君子;睹无者,天地之友。

贱而不可不任者,物也;卑而不可不因者,民也;匿而不可不为者
,事也;粗而不可不陈者,法也;远而不可不居者,义也;亲而不可
不广者,仁也;节而不可不积者,礼也;中而不可不高者,德也;一
而不可不易者,道也;神而不可不为者,天也。故圣人观于天而不助
,成于德而不累,出于道而不谋,会于仁而不恃,薄于义而不积,应
于礼而不讳,接于事而不辞,齐于法而不乱,恃于民而不轻,因于物
而不去。物者莫足为也,而不可不为。不明于天者,不纯于德;不通
于道者,无自而可;不明于道者,悲夫!何谓道?有天道,有人道。
无为而尊者,天道也;有为而累者,人道也。主者,天道也;臣者,
人道也。天道之与人道也,相去远矣,不可不察也。

\hypertarget{header-n2135}{%
\subsubsection{天地}\label{header-n2135}}

天地虽大,其化均也;万物虽多,其治一也;人卒虽众,其主君也
。君原于德而成于天。故曰:玄古之君天下,无为也,天德而已矣。
以道观言而天下之君正;以道观分而君臣之义明;以道观能而天下之
官治;以道泛观而万物之应备。故通于天地者,德也;行于万物者,
道也;上治人者,事也;能有所艺者,技也。技兼于事,事兼于义,
义兼于德,德兼于道,道兼于天。故曰:古之畜天下者,无欲而天下
足,无为而万物化,渊静而百姓定。《记》曰:``通于一而万事毕,
无心得而鬼神服。''

夫子曰:``夫道,覆载万物者也,洋洋乎大哉!君子不可以不刳心
焉。无为为之之谓天,无为言之之谓德,爱人利物之谓仁,不同同之
之谓大,行不崖异之谓宽,有万不同之谓富。故执德之谓纪,德成之
谓立,循于道之谓备,不以物挫志之谓完。君子明于此十者,则韬乎
其事心之大也,沛乎其为万物逝也。若然者,藏金于山,藏珠于渊;
不利货财,不近贵富;不乐寿,不哀夭;不荣通,不丑穷。不拘一世
之利以为己私分,不以王天下为己处显。显则明。万物一府,死生同 状。''

夫子曰:``夫道,渊乎其居也,漻乎其清也。金石不得无以鸣。故
金石有声,不考不鸣。万物孰能定之!夫王德之人,素逝而耻通于
事,立之本原而知通于神,故其德广。其心之出,有物采之。故形
非道不生,生非德不明。存形穷生,立德明道,非王德者邪!荡荡
乎!忽然出,勃然动,而万物从之乎!此谓王德之人。视乎冥冥,
听乎无声。冥冥之中,独见晓焉;无声之中,独闻和焉。故深之又
深而能物焉;神之又神而能精焉。故其与万物接也,至无而供其求,
时骋而要其宿,大小、长短、修远。''

黄帝游乎赤水之北,登乎昆仑之丘而南望。还归,遗其玄珠。使知
索之而不得,使离朱索之而不得,使喫诟索之而不得也。乃使象罔,
象罔得之。黄帝曰:``异哉,象罔乃可以得之乎?''

尧之师曰许由,许由之师曰啮缺,啮缺之师曰王倪,王倪之师曰被
衣。尧问于许由曰:``啮缺可以配天乎?吾藉王倪以要之。''许由曰
:``殆哉,圾乎天下!啮缺之为人也,聪明睿知,给数以敏,其性过
人,而又乃以人受天。彼审乎禁过,而不知过之所由生。与之配天乎
?彼且乘人而无天。方且本身而异形,方且尊知而火驰,方且为绪使
,方且为物絯,方且四顾而物应,方且应众宜,方且与物化而未始有
恒。夫何足以配天乎!虽然,有族有祖,可以为众父而不可以为众父
父。治,乱之率也,北面之祸也,南面之贼也。''

尧观乎华,华封人曰:``嘻,圣人!请祝圣人,使圣人寿。''尧曰
:``辞。''``使圣人富。''尧曰:``辞。''``使圣人多男子。''尧曰
:``辞。''封人曰:``寿,富,多男子,人之所欲也。女独不欲,何
邪?''尧曰:``多男子则多惧,富则多事,寿则多辱。是三者,非所
以养德也,故辞。''封人曰:``始也我以女为圣人邪,今然君子也。
天生万民,必授之职。多男子而授之职,则何惧之有?富而使人分之
,则何事之有?夫圣人,鹑居而彀食,鸟行而无彰。天下有道,则与
物皆昌;天下无道,则修德就闲。千岁厌世,去而上仙,乘彼白云,
至于帝乡。三患莫至,身常无殃,则何辱之有?''封人去之,尧随之
曰:``请问。''封人曰:``退已!''

尧治天下,伯成子高立为诸侯。尧授舜,舜授禹,伯成子高辞为诸
侯而耕。禹往见之,则耕在野。禹趋就下风,立而问焉,曰:``昔尧
治天下,吾子立为诸侯。尧授舜,舜授予,而吾子辞为诸侯而耕。敢
问其故何也?''子高曰:``昔者尧治天下,不赏而民劝,不罚而民畏
。今子赏罚而民且不仁,德自此衰,刑自此立,后世之乱自此始矣!
夫子阖行邪?无落吾事!''俋俋乎耕而不顾。

泰初有无,无有无名。一之所起,有一而未形。物得以生谓之德;
未形者有分,且然无间谓之命;留动而生物,物成生理谓之形;形体
保神,各有仪则谓之性;性修反德,德至同于初。同乃虚,虚乃大。
合喙鸣。喙鸣合,与天地为合。其合缗缗,若愚若昏,是谓玄德,同 乎大顺。

夫子问于老聃曰:``有人治道若相放,可不可,然不然。辩者有言
曰:`离坚白,若县寓。'若是则可谓圣人乎?''老聃曰:``是胥易
技系,劳形怵心者也。执留之狗成思,猿狙之便自山林来。丘,予告
若,而所不能闻与而所不能言:凡有首有趾、无心无耳者众;有形者
与无形无状而皆存者尽无。其动止也,其死生也,其废起也,此又非
其所以也。有治在人。忘乎物,忘乎天,其名为忘己。忘己之人,是
之谓入于天。''

将闾葂见季彻曰:``鲁君谓葂也曰:`请受教。'辞不获命。既已
告矣,未知中否。请尝荐之。吾谓鲁君曰:`必服恭俭,拔出公忠之
属而无阿私,民孰敢不辑!'''季彻局局然笑曰:``若夫子之言,于
帝王之德,犹螳螂之怒臂以当车轶,则必不胜任矣!且若是,则其自
为处危,其观台多物,
将往投迹者众。''将闾葂覷觑然惊曰:``葂也汒若于夫子之所言矣!
虽然,愿先生之言其风也。''季彻曰:``大圣之治天下也,摇荡民心,
使之成教易俗,举灭其贼心而皆进其独志。若性之自为,而民不知其所
由然。若然者,岂兄尧、舜之教民溟涬然弟之哉?欲同乎德而心居矣!''

子贡南游于楚,反于晋,过汉阴,见一丈人方将为圃畦,凿隧而入
井,抱瓮而出灌,搰然用力甚多而见功寡。子贡曰:``有械于此,一
日浸百畦,用力甚寡而见功多,夫子不欲乎?''为圃者仰而视之曰:
``奈何?''曰:``凿木为机,后重前轻,挈水若抽,数如泆汤,其名
为槔。''为圃者忿然作色而笑曰:``吾闻之吾师,有机械者必有机事,
有机事者必有机心。机心存于胸中则纯白不备。纯白不备则神生不定,
神生不定者,道之所不载也。吾非不知,羞而不为也。''子贡瞒然惭,
俯而不对。有间,为圃者曰:``子奚为者邪?曰:``孔丘之徒也。''为
圃者曰:``子非夫博学以拟圣,於于以盖众,独弦哀歌以卖名声于天下
者乎?汝方将忘汝神气,堕汝形骸,而庶几乎!而身之不能治,而何
暇治天下乎!子往矣,无乏吾事。''

子贡卑陬失色,顼顼然不自得,行三十里而后愈。其弟子曰:``向
之人何为者邪?夫子何故见之变容失色,终日不自反邪?''曰:``始
吾以为天下一人耳,不知复有夫人也。吾闻之夫子:事求可,功求成
,用力少,见功多者,圣人之道。今徒不然。执道者德全,德全者形
全,形全者神全。神全者,圣人之道也。托生与民并行而不知其所之
,汒乎淳备哉!功利机巧必忘夫人之心。若夫人者,非其志不之,非
其心不为。虽以天下誉之,得其所谓,囗(上敖''下``言'')然不顾;
以天下非之,失其所谓,傥然不受。天下之非誉无益损焉,是谓全德
之人哉!我之谓风波之民。''反于鲁,以告孔子。孔子曰:``彼假修
浑沌氏之术者也。识其一,不识其二;治其内而不治其外。夫明白入
素,无为复朴,体性抱神,以游世俗之间者,汝将固惊邪?且浑沌氏
之术,予与汝何足以识之哉!''

谆芒将东之大壑,适遇苑风于东海之滨。苑风曰:``子将奚之?''
曰:``将之大壑。''曰:``奚为焉?''曰:``夫大壑之为物也,注焉
而不满,酌焉而不竭。吾将游焉!''苑风曰:``夫子无意于横目之民
乎?愿闻圣治。''谆芒曰:``圣治乎?官施而不失其宜,拔举而不失
其能,毕见其情事而行其所为,行言自为而天下化。手挠顾指,四方
之民莫不俱至,此之谓圣治。''``愿闻德人。''曰:``德人者,居无
思,行无虑,不藏是非美恶。四海之内共利之之谓悦,共给之之谓安
。怊乎若婴儿之失其母也,傥乎若行而失其道也。财用有余而不知其
所自来,饮食取足而不知其所从,此谓德人之容。''``愿闻神人。''
曰:``上神乘光,与形灭亡,是谓照旷。致命尽情,天地乐而万事销
亡,万物复情,此之谓混溟。''

门无鬼与赤张满稽观于武王之师,赤张满稽曰:``不及有虞氏乎!
故离此患也。''门无鬼曰:``天下均治而有虞氏治之邪?其乱而后治
之与?''赤张满稽曰:``天下均治之为愿,而何计以有虞氏为!有虞
氏之药疡也,秃而施髢,病而求医。孝子操药以修慈父,其色燋然,
圣人羞之。至德之世,不尚贤,不使能,上如标枝,民如野鹿。端正
而不知以为义,相爱而不知以为仁,实而不知以为忠,当而不知以为
信,蠢动而相使不以为赐。是故行而无迹,事而无传。

孝子不谀其亲,忠臣不谄其君,臣、子之盛也。亲之所言而然,所
行而善,则世俗谓之不肖子;君之所言而然,所行而善,则世俗谓之
不肖臣。而未知此其必然邪?世俗之所谓然而然之,所谓善而善之,
则不谓之道谀之人也!然则俗故严于亲而尊于君邪?谓己道人,则勃
然作色;谓己谀人,则怫然作色。而终身道人也,终身谀人也,合譬
饰辞聚众也,是终始本末不相坐。垂衣裳,设采色,动容貌,以媚一
世,而不自谓道谀;与夫人之为徒,通是非,而不自谓众人也,愚之
至也。知其愚者,非大愚也;知其惑者,非不惑也。大惑者,终身不
解;大愚者,终身不灵。三人行而一人惑,所适者,犹可致也,惑者
少也;二人惑则劳而不至,惑者胜也。而今也以天下惑,予虽有祈向
,不可得也。不亦悲乎!大声不入于里耳,折杨、皇囗(上``艹''下
夸''音hua1),则嗑然而笑。是故高言不止于众人之心;至言不
出,俗言胜也。以二缶钟惑,而所适不得矣。而今也以天下惑,予虽
有祈向,其庸可得邪!知其不可得也而强之,又一惑也!故莫若释之
而不推。不推,谁其比忧!厉之人,夜半生其子,遽取火而视之,汲
汲然唯恐其似己也。

百年之木,破为牺尊,青黄而文之,其断在沟中。比牺尊于沟中之
断,则美恶有间矣,其于失性一也。跖与曾、史,行义有间矣,然其
失性均也。且夫失性有五:一曰五色乱目,使目不明;二曰五声乱耳
,使耳不聪;三曰五臭熏鼻,困囗(``悛''字以``凶''代``厶''音z
ong1)中颡;四曰五味浊口,使口厉爽;五曰趣舍滑心,使性飞
扬。此五者,皆生之害也。而杨、墨乃始离囗(左``足''右``支'')
自以为得,非吾所谓得也。夫得者困,可以为得乎?则鸠囗(左``号
号''右``鸟'')之在于笼也,亦可以为得矣。且夫趣舍声色以柴其内
,皮弁鹬冠囗(左``扌''右``晋''音jin4)笏绅修以约其外。内
支盈于柴栅,外重囗(左``纟''右``墨'')缴囗囗(左``目''右``完
'')然在囗(左``纟''右``墨'')缴之中,而自以为得,则是罪人交
臂历指而虎豹在于囊槛,亦可以为得矣!

\hypertarget{header-n2155}{%
\subsubsection{天道}\label{header-n2155}}

天道运而无所积,故万物成;帝道运而无所积,故天下归;圣道运
而无所积,故海内服。明于天,通于圣,六通四辟于帝王之德者,其
自为也,昧然无不静者矣!圣人之静也,非曰静也善,故静也。万物
无足以挠心者,故静也。水静则明烛须眉,平中准,大匠取法焉。水
静犹明,而况精神!圣人之心静乎!天地之鉴也,万物之镜也。夫虚
静恬淡寂漠无为者,天地之平而道德之至也。故帝王圣人休焉。休则
虚,虚则实,实则伦矣。虚则静,静则动,动则得矣。静则无为,无
为也,则任事者责矣。无为则俞俞。俞俞者,忧患不能处,年寿长矣
。夫虚静恬淡寂漠无为者,万物之本也。明此以南乡,尧之为君也;
明此以北面,舜之为臣也。以此处上,帝王天子之德也;以此处下,
玄圣素王之道也。以此退居而闲游,江海山林之士服;以此进为而抚
世,则功大名显而天下一也。静而圣,动而王,无为也而尊,朴素而
天下莫能与之争美。夫明白于天地之德者,此之谓大本大宗,与天和
者也。所以均调天下,与人和者也。与人和者,谓之人乐;与天和者
,谓之天乐。庄子曰:``吾师乎,吾师乎!赍万物而不为戾;泽及万
世而不为仁;长于上古而不为寿;覆载天地、刻雕众形而不为巧。''
此之谓天乐。故曰:知天乐者,其生也天行,其死也物化。静而与阴
同德,动而与阳同波。故知天乐者,无天怨,无人非,无物累,无鬼
责。故曰:其动也天,其静也地,一心定而王天下;其鬼不祟,其魂
不疲,一心定而万物服。言以虚静推于天地,通于万物,此之谓天乐
。天乐者,圣人之心以畜天下也。

夫帝王之德,以天地为宗,以道德为主,以无为为常。无为也,则
用天下而有余;有为也,则为天下用而不足。故古之人贵夫无为也。
上无为也,下亦无为也,是下与上同德。下与上同德则不臣。下有为
也,上亦有为也,是上与下同道。上与下同道则不主。上必无为而用
下,下必有为为天下用。此不易之道也。

故古之王天下者,知虽落天地,不自虑也;辩虽雕万物,不自说也
;能虽穷海内,不自为也。天不产而万物化,地不长而万物育,帝王
无为而天下功。故曰:莫神于天,莫富于地,莫大于帝王。故曰:帝
王之德配天地。此乘天地,驰万物,而用人群之道也。

本在于上,末在于下;要在于主,详在于臣。三军五兵之运,德之
末也;赏罚利害,五刑之辟,教之末也;礼法度数,刑名比详,治之
末也;钟鼓之音,羽旄之容,乐之末也;哭泣衰囗(左``纟''右``至
''),隆杀之服,哀之末也。此五末者,须精神之运,心术之动,然
后从之者也。末学者,古人有之,而非所以先也。君先而臣从,父先
而子从,兄先而弟从,长先而少从,男先而女从,夫先而妇从。夫尊
卑先后,天地之行也,故圣人取象焉。天尊地卑,神明之位也;春夏
先,秋冬后,四时之序也;万物化作,萌区有状,盛衰之杀,变化之
流也。夫天地至神矣,而有尊卑先后之序,而况人道乎!宗庙尚亲,
朝廷尚尊,乡党尚齿,行事尚贤,大道之序也。语道而非其序者,非
其道也。语道而非其道者,安取道哉!

是故古之明大道者,先明天而道德次之,道德已明而仁义次之,仁
义已明而分守次之,分守已明而形名次之,形名已明而因任次之,因
任已明而原省次之,原省已明而是非次之,是非已明而赏罚次之,赏
罚已明而愚知处宜,贵贱履位,仁贤不肖袭情。必分其能,必由其名
。以此事上,以此畜下,以此治物,以此修身,知谋不用,必归其天
。此之谓大平,治之至也。故书曰:``有形有名。''形名者,古人有
之,而非所以先也。古之语大道者,五变而形名可举,九变而赏罚可
言也。骤而语形名,不知其本也;骤而语赏罚,不知其始也。倒道而
言,迕道而说者,人之所治也,安能治人!骤而语形名赏罚,此有知
治之具,非知治之道。可用于天下,不足以用天下。此之谓辩士,一
曲之人也。礼法数度,形名比详,古人有之。此下之所以事上,非上
之所以畜下也。

昔者舜问于尧曰:``天王之用心何如?''尧曰:``吾不敖无告,不
废穷民,苦死者,嘉孺子而哀妇人,此吾所以用心已。''舜曰:``美
则美矣,而未大也。''尧曰:``然则何如?''舜曰:``天德而出宁,
日月照而四时行,若昼夜之有经,云行而雨施矣!''尧曰:``胶胶扰
扰乎!子,天之合也;我,人之合也。''夫天地者,古之所大也,而
黄帝、尧、舜之所共美也。故古之王天下者,奚为哉?天地而已矣!

孔子西藏书于周室,子路谋曰:``由闻周之征藏史有老聃者,免而
归居,夫子欲藏书,则试往因焉。''孔子曰:``善。''往见老聃,而
老聃不许,于是囗(左``纟''右``番''音fan2)十二经以说。老
聃中其说,曰:``大谩,愿闻其要。''孔子曰:``要在仁义。''老聃
曰:``请问:仁义,人之性邪?''孔子曰:``然,君子不仁则不成,
不义则不生。仁义,真人之性也,又将奚为矣?''老聃曰:``请问:
何谓仁义?''孔子曰:``中心物恺,兼爱无私,此仁义之情也。''老
聃曰:``意,几乎后言!夫兼爱,不亦迂夫!无私焉,乃私也。夫子
若欲使天下无失其牧乎?则天地固有常矣,日月固有明矣,星辰固有
列矣,禽兽固有群矣,树木固有立矣。夫子亦放德而行,遁遁而趋,
已至矣!又何偈偈乎揭仁义,若击鼓而求亡子焉!意,夫子乱人之性 也。''

士成绮见老子而问曰:``吾闻夫子圣人也。吾固不辞远道而来愿见
,百舍重趼而不敢息。今吾观子非圣人也,鼠壤有余蔬而弃妹,不仁
也!生熟不尽于前,而积敛无崖。''老子漠然不应。士成绮明日复见
,曰:``昔者吾有剌于子,今吾心正囗(左``谷''右``阝'')矣,何
故也?''老子曰:``夫巧知神圣之人,吾自以为脱焉。昔者子呼我牛
也而谓之牛;呼我马也而谓之马。苟有其实,人与之名而弗受,再受
其殃。吾服也恒服,吾非以服有服。''士成绮雁行避影,履行遂进,
而问修身若何。老子曰:``而容崖然,而目冲然,而颡囗(左上``月
''左下``廾''右``页'')然,而口阚然,而状义然。似系马而止也,
动而持,发也机,察而审,知巧而睹于泰,凡以为不信。边竟有人焉
,其名为窃。''

老子曰:``夫道,于大不终,于小不遗,故万物备。广广乎其无不
容也,渊渊乎其不可测也。形德仁义,神之末也,非至人孰能定之!
夫至人有世,不亦大乎,而不足以为之累;天下奋柄而不与之偕;审
乎无假而不与利迁;极物之真,能守其本。故外天地,遗万物,而神
未尝有所困也。通乎道,合乎德,退仁义,宾礼乐,至人之心有所定 矣!''

世之所贵道者,书也。书不过语,语有贵也。语之所贵者,意也,
意有所随。意之所随者,不可以言传也,而世因贵言传书。世虽贵之
,我犹不足贵也,为其贵非其贵也。故视而可见者,形与色也;听而
可闻者,名与声也。悲夫!世人以形色名声为足以得彼之情。夫形色
名声,果不足以得彼之情,则知者不言,言者不知,而世岂识之哉!

桓公读书于堂上,轮扁斫轮于堂下,释椎凿而上,问桓公曰:``敢
问:``公之所读者,何言邪?''公曰:``圣人之言也。''曰:``圣人
在乎?''公曰:``已死矣。''曰:``然则君之所读者,古人之糟粕已
夫!''桓公曰:``寡人读书,轮人安得议乎!有说则可,无说则死!
''轮扁曰:``臣也以臣之事观之。斫轮,徐则甘而不固,疾则苦而不
入,不徐不疾,得之于手而应于心,口不能言,有数存乎其间。臣不
能以喻臣之子,臣之子亦不能受之于臣,是以行年七十而老斫轮。古
之人与其不可传也死矣,然则君之所读者,古人之糟粕已夫!''

\hypertarget{header-n2170}{%
\subsubsection{天运}\label{header-n2170}}

``天其运乎?地其处乎?日月其争于所乎?孰主张是?孰维纲是?
孰居无事推而行是?意者其有机缄而不得已乎?意者其运转而不能自
止邪?云者为雨乎?雨者为云乎?孰隆施是?孰居无事淫乐而劝是?
风起北方,一西一东,有上仿徨。孰嘘吸是?孰居无事而披拂是?敢
问何故?''巫咸囗(左``礻''右``召''音shao4)曰:``来,吾
语女。天有六极五常,帝王顺之则治,逆之则凶。九洛之事,治成德
备,临照下土,天下戴之,此谓上皇。''

商大宰荡问仁于庄子。庄子曰:``虎狼,仁也。''曰:``何谓也?
''庄子曰:``父子相亲,何为不仁!''曰:``请问至仁。''庄子曰:
``至仁无亲。''大宰曰:``荡闻之,无亲则不爱,不爱则不孝。谓至
仁不孝,可乎?''庄子曰:``不然,夫至仁尚矣,孝固不足以言之。
此非过孝之言也,不及孝之言也。夫南行者至于郢,北面而不见冥山
,是何也?则去之远也。故曰:以敬孝易,以爱孝难;以爱孝易,而
忘亲难;忘亲易,使亲忘我难;使亲忘我易,兼忘天下难;兼忘天下
易,使天下兼忘我难。夫德遗尧、舜而不为也,利泽施于万世,天下
莫知也,岂直大息而言仁孝乎哉!夫孝悌仁义,忠信贞廉,此皆自勉
以役其德者也,不足多也。故曰:至贵,国爵并焉;至富,国财并焉
;至愿,名誉并焉。是以道不渝。''

北门成问于黄帝曰:``帝张咸池之乐于洞庭之野,吾始闻之惧,复
闻之怠,卒闻之而惑,荡荡默默,乃不自得。''帝曰:``汝殆其然哉
!吾奏之以人,徵之以天,行之以礼义,建之以大清。夫至乐者,先
应之以人事,顺之以天理,行之以五德,应之以自然。然后调理四时
,太和万物。四时迭起,万物循生。一盛一衰,文武伦经。一清一浊
,阴阳调和,流光其声。蛰虫始作,吾惊之以雷霆。其卒无尾,其始
无首。一死一生,一偾一起,所常无穷,而一不可待。汝故惧也。吾
又奏之以阴阳之和,烛之以日月之明。其声能短能长,能柔能刚,变
化齐一,不主故常。在谷满谷,在坑满坑。涂囗(左``谷''右``阝''
)守神,以物为量。其声挥绰,其名高明。是故鬼神守其幽,日月星
辰行其纪。吾止之于有穷,流之于无止。子欲虑之而不能知也,望之
而不能见也,逐之而不能及也。傥然立于四虚之道,倚于槁梧而吟:
`目知穷乎所欲见,力屈乎所欲逐,吾既不及,已夫!'形充空虚,
乃至委蛇。汝委蛇,故怠。吾又奏之以无怠之声,调之以自然之命。
故若混逐丛生,林乐而无形,布挥而不曳,幽昏而无声。动于无方,
居于窈冥,或谓之死,或谓之生;或谓之实,或谓之荣。行流散徙,
不主常声。世疑之,稽于圣人。圣也者,达于情而遂于命也。天机不
张而五官皆备。此之谓天乐,无言而心说。故有焱氏为之颂曰:`听
之不闻其声,视之不见其形,充满天地,苞裹六极。'汝欲听之而无
接焉,而故惑也。乐也者,始于惧,惧故祟;吾又次之以怠,怠故遁
;卒之于惑,惑故愚;愚故道,道可载而与之俱也。''

孔子西游于卫,颜渊问师金曰:``以夫子之行为奚如?''师金曰:
``惜乎!而夫子其穷哉!''颜渊曰:``何也?''师金曰:``夫刍狗之
未陈也,盛以箧衍,巾以文绣,尸祝齐戒以将之。及其已陈也,行者
践其首脊,苏者取而爨之而已。将复取而盛以箧衍,巾以文绣,游居
寝卧其下,彼不得梦,必且数眯焉。今而夫子亦取先王已陈刍狗,聚
弟子游居寝卧其下。故伐树于宋,削迹于卫,穷于商周,是非其梦邪
?围于陈蔡之间,七日不火食,死生相与邻,是非其眯邪?夫水行莫
如用舟,而陆行莫如用车。以舟之可行于水也,而求推之于陆,则没
世不行寻常。古今非水陆与?周鲁非舟车与?今蕲行周于鲁,是犹推
舟于陆也!劳而无功,身必有殃。彼未知夫无方之传,应物而不穷者
也。且子独不见夫桔槔者乎?引之则俯,舍之则仰。彼,人之所引,
非引人者也。故俯仰而不得罪于人。故夫三皇五帝之礼义法度,不矜
于同而矜于治。故譬三皇五帝之礼义法度,其犹囗(左``木''右``且
'')梨橘柚邪!其味相反而皆可于口。故礼义法度者,应时而变者也
。今取囗(``援''字以``犭''代``扌'')狙而衣以周公之服,彼必囗
(``龄''字以``乞''代``令'')啮挽裂,尽去而后慊。观古今之异,
犹囗狙之异乎周公也。故西施病心而颦其里,其里之丑人见之而美之
,归亦捧心而颦其里。其里之富人见之,坚闭门而不出;贫人见之,
挈妻子而去之走。彼知颦美而不知颦之所以美。惜乎,而夫子其穷哉 !''

孔子行年五十有一而不闻道,乃南之沛见老聃。老聃曰:``子来乎
?吾闻子,北方之贤者也!子亦得道乎?''孔子曰:``未得也。''老
子曰:``子恶乎求之哉?''曰:``吾求之于度数,五年而未得也。''
老子曰:``子又恶乎求之哉?''曰:``吾求之于阴阳,十有二年而未
得也。''老子曰:``然,使道而可献,则人莫不献之于其君;使道而
可进,则人莫不进之于其亲;使道而可以告人,则人莫不告其兄弟;
使道而可以与人,则人莫不与其子孙。然而不可者,无它也,中无主
而不止,外无正而不行。由中出者,不受于外,圣人不出;由外入者
,无主于中,圣人不隐。名,公器也,不可多取。仁义,先王之蘧庐
也,止可以一宿而不可久处。觏而多责。古之至人,假道于仁,托宿
于义,以游逍遥之虚,食于苟简之田,立于不贷之圃。逍遥,无为也
;苟简,易养也;不贷,无出也。古者谓是采真之游。以富为是者,
不能让禄;以显为是者,不能让名。亲权者,不能与人柄,操之则栗
,舍之则悲,而一无所鉴,以窥其所不休者,是天之戮民也。怨、恩
、取、与、谏、教、生杀八者,正之器也,唯循大变无所湮者为能用
之。故曰:正者,正也。其心以为不然者,天门弗开矣。''

孔子见老聃而语仁义。老聃曰:``夫播糠眯目,则天地四方易位矣
;蚊虻囗(左``口''右上``先先''右下``日''音zan4)肤,则通
昔不寐矣。夫仁义惨然,乃愤吾心,乱莫大焉。吾子使天下无失其朴
,吾子亦放风而动,总德而立矣!又奚杰杰然若负建鼓而求亡子者邪
!夫鹄不日浴而白,乌不日黔而黑。黑白之朴,不足以为辩;名誉之
观,不足以为广。泉涸,鱼相与处于陆,相囗(左``口''右``句'')
以湿,相濡以沫,不若相忘于江湖。''

孔子见老聃归,三日不谈。弟子问曰:``夫子见老聃,亦将何规哉
?''孔子曰:``吾乃今于是乎见龙。龙,合而成体,散而成章,乘乎
云气而养乎阴阳。予口张而不能囗(左``口''右上``力''右中``力力
''右下``月''音xie2)。予又何规老聃哉?''子贡曰:``然则人
固有尸居而龙见,雷声而渊默,发动如天地者乎?赐亦可得而观乎?
''遂以孔子声见老聃。老聃方将倨堂而应,微曰:``予年运而往矣,
子将何以戒我乎?''子贡曰:``夫三皇五帝之治天下不同,其系声名
一也。而先生独以为非圣人,如何哉?''老聃曰:``小子少进!子何
以谓不同?''对曰:``尧授舜,舜授禹。禹用力而汤用兵,文王顺纣
而不敢逆,武王逆纣而不肯顺,故曰不同。''老聃曰:``小子少进,
余语汝三皇五帝之治天下:黄帝之治天下,使民心一。民有其亲死不
哭而民不非也。尧之治天下,使民心亲。民有为其亲杀其杀而民不非
也。舜之治天下,使民心竞。民孕妇十月生子,子生五月而能言,不
至乎孩而始谁,则人始有夭矣。禹之治天下,使民心变,人有心而兵
有顺,杀盗非杀人。自为种而`天下'耳。是以天下大骇,儒墨皆起
。其作始有伦,而今乎妇女,何言哉!余语汝:三皇五帝之治天下,
名曰治之,而乱莫甚焉。三皇之知,上悖日月之明,下睽山川之精,
中堕四时之施。其知惨于蛎虿之尾,鲜规之兽,莫得安其性命之情者
,而犹自以为圣人,不可耻乎?其无耻也!''子贡蹴蹴然立不安。

孔子谓老聃曰:``丘治《诗》、《书》、《礼》、《乐》、《易》
、《春秋》六经,自以为久矣,孰知其故矣,以奸者七十二君,论先
王之道而明周、召之迹,一君无所钩用。甚矣!夫人之难说也?道之
难明邪?''老子曰:``幸矣,子之不遇治世之君!夫六经,先王之陈
迹也,岂其所以迹哉!今子之所言,犹迹也。夫迹,履之所出,而迹
岂履哉!夫白囗(左上``臼''左下``儿''右``鸟''音yi4)之相视
,眸子不运而风化;虫,雄鸣于上风,雌应于下风而风化。类自为雌
雄,故风化。性不可易,命不可变,时不可止,道不可壅。苟得于道
,无自而不可;失焉者,无自而可。''孔子不出三月,复见,曰:``
丘得之矣。乌鹊孺,鱼傅沫,细要者化,有弟而兄啼。久矣,夫丘不
与化为人!不与化为人,安能化人。''老子曰:``可,丘得之矣!''

\hypertarget{header-n2182}{%
\subsubsection{刻意}\label{header-n2182}}

刻意尚行,离世异俗,高论怨诽,为亢而已矣。此山谷之士,非世
之人,枯槁赴渊者之所好也。语仁义忠信,恭俭推让,为修而已矣。
此平世之士,教诲之人,游居学者之所好也。语大功,立大名,礼君
臣,正上下,为治而已矣。此朝廷之士,尊主强国之人,致功并兼者
之所好也。就薮泽,处闲旷,钓鱼闲处,无为而已矣。此江海之士,
避世之人,闲暇者之所好也。吹囗(左``口''右``句'')呼吸,吐故
纳新,熊经鸟申,为寿而已矣。此道引之士,养形之人,彭祖寿考者
之所好也。若夫不刻意而高,无仁义而修,无功名而治,无江海而闲
,不道引而寿,无不忘也,无不有也。淡然无极而众美从之。此天地
之道,圣人之德也。

故曰:夫恬淡寂漠,虚无无为,此天地之平而道德之质也。故曰:
圣人休休焉则平易矣。平易则恬淡矣。平易恬淡,则忧患不能入,邪
气不能袭,故其德全而神不亏。故曰:圣人之生也天行,其死也物化
。静而与阴同德,动而与阳同波。不为福先,不为祸始。感而后应,
迫而后动,不得已而后起。去知与故,遁天之理。故无天灾,无物累
,无人非,无鬼责。其生若浮,其死若休。不思虑,不豫谋。光矣而
不耀,信矣而不期。其寝不梦,其觉无忧。其神纯粹,其魂不罢。虚
无恬淡,乃合天德。故曰:悲乐者,德之邪也;喜怒者,道之过也;
好恶者,德之失也。故心不忧乐,德之至也;一而不变,静之至也;
无所于忤,虚之至也;不与物交,淡之至也;无所于逆,粹之至也。
故曰:形劳而不休则弊,精用而不已则劳,劳则竭。水之性,不杂则
清,莫动则平;郁闭而不流,亦不能清;天德之象也。故曰:纯粹而
不杂,静一而不变,淡而无为,动而以天行,此养神之道也。

夫有干越之剑者,柙而藏之,不敢用也,宝之至也。精神四达并流
,无所不极,上际于天,下蟠于地,化育万物,不可为象,其名为同
帝。纯素之道,唯神是守。守而勿失,与神为一。一之精通,合于天
伦。野语有之曰:``众人重利,廉士重名,贤士尚志,圣人贵精。''
故素也者,谓其无所与杂也;纯也者,谓其不亏其神也。能体纯素, 谓之真人。

\hypertarget{header-n2189}{%
\subsubsection{缮性}\label{header-n2189}}

缮性于俗学,以求复其初;滑欲于俗思,以求致其明:谓之蔽蒙之 民。

古之治道者,以恬养知。生而无以知为也,谓之以知养恬。知与恬
交相养,而和理出其性。夫德,和也;道,理也。德无不容,仁也;
道无不理,义也;义明而物亲,忠也;中纯实而反乎情,乐也;信行
容体而顺乎文,礼也。礼乐遍行,则天下乱矣。彼正而蒙己德,德则
不冒。冒则物必失其性也。古之人,在混芒之中,与一世而得淡漠焉
。当是时也,阴阳和静,鬼神不扰,四时得节,万物不伤,群生不夭
,人虽有知,无所用之,此之谓至一。当是时也,莫之为而常自然。

逮德下衰,及燧人、伏羲始为天下,是故顺而不一。德又下衰,及
神农、黄帝始为天下,是故安而不顺。德又下衰,及唐、虞始为天下
,兴治化之流,枭淳散朴,离道以善,险德以行,然后去性而从于心
。心与心识知,而不足以定天下,然后附之以文,益之以博。文灭质
,博溺心,然后民始惑乱,无以反其性情而复其初。由是观之,世丧
道矣,道丧世矣,世与道交相丧也。道之人何由兴乎世,世亦何由兴
乎道哉!道无以兴乎世,世无以兴乎道,虽圣人不在山林之中,其德
隐矣。隐故不自隐。古之所谓隐士者,非伏其身而弗见也,非闭其言
而不出也,非藏其知而不发也,时命大谬也。当时命而大行乎天下,
则反一无迹;不当时命而大穷乎天下,则深根宁极而待:此存身之道
也。古之存身者,不以辩饰知,不以知穷天下,不以知穷德,危然处
其所而反其性,己又何为哉!道固不小行,德固不小识。小识伤德,
小行伤道。故曰:正己而已矣。乐全之谓得志。

古之所谓得志者,非轩冕之谓也,谓其无以益其乐而已矣。今之所
谓得志者,轩冕之谓也。轩冕在身,非性命也,物之傥来,寄也。寄
之,其来不可圉,其去不可止。故不为轩冕肆志,不为穷约趋俗,其
乐彼与此同,故无忧而已矣!今寄去则不乐。由是观之,虽乐,未尝
不荒也。故曰:丧己于物,失性于俗者,谓之倒置之民。

\hypertarget{header-n2197}{%
\subsubsection{秋水}\label{header-n2197}}

秋水时至,百川灌河。泾流之大,两涘渚崖之间,不辩牛马。于是
焉河伯欣然自喜,以天下之美为尽在己。顺流而东行,至于北海,东
面而视,不见水端。于是焉河伯始旋其面目,望洋向若而叹曰:``野
语有之曰:`闻道百,以为莫己若者。'我之谓也。且夫我尝闻少仲
尼之闻而轻伯夷之义者,始吾弗信。今我睹子之难穷也,吾非至于子
之门则殆矣,吾长见笑于大方之家。''北海若曰:``井蛙不可以语于海者,拘于虚也;夏虫不可以语于冰者,笃于时也;曲士不可以语于
道者,束于教也。今尔出于崖涘,观于大海,乃知尔丑,尔将可与语
大理矣。天下之水,莫大于海:万川归之,不知何时止而不盈;尾闾
泄之,不知何时已而不虚;春秋不变,水旱不知。此其过江河之流,
不可为量数。而吾未尝以此自多者,自以比形于天地,而受气于阴阳,
吾在于天地之间,犹小石小木之在大山也。方存乎见少,又奚以自多!
计四海之在天地之间也,不似礨空之在大泽乎?计中国之在海内不似
稊米之在太仓乎?号物之数谓之万,人处一焉;人卒九州,谷食之所
生,舟车之所通,人处一焉。此其比万物也,不似豪末之在于马体乎?
五帝之所连,三王之所争,仁人之所忧,任士之所劳,尽此矣!伯夷
辞之以为名,仲尼语之以为博。此其自多也,不似尔向之自多于水乎?''

河伯曰:``然则吾大天地而小豪末,可乎?''北海若曰:``否。夫
物,量无穷,时无止,分无常,终始无故。是故大知观于远近,故小
而不寡,大而不多:知量无穷。证向今故,故遥而不闷,掇而不跂:
知时无止。察乎盈虚,故得而不喜,失而不忧:知分之无常也。明乎
坦涂,故生而不说,死而不祸:知终始之不可故也。计人之所知,不
若其所不知;其生之时,不若未生之时;以其至小,求穷其至大之域,
是故迷乱而不能自得也。由此观之,又何以知毫末之足以定至细之倪,
又何以知天地之足以穷至大之域!''

河伯曰:``世之议者皆曰:`至精无形,至大不可围。'是信情乎
?''北海若曰:``夫自细视大者不尽,自大视细者不明。夫精,小之
微也;郛,大之殷也:故异便。此势之有也。夫精粗者,期于有形者
也;无形者,数之所不能分也;不可围者,数之所不能穷也。可以言
论者,物之粗也;可以意致者,物之精也;言之所不能论,意之所不
能察致者,不期精粗焉。是故大人之行:不出乎害人,不多仁恩;动
不为利,不贱门隶;货财弗争,不多辞让;事焉不借人,不多食乎力
,不贱贪污;行殊乎俗,不多辟异;为在从众,不贱佞谄;世之爵禄
不足以为劝,戮耻不足以为辱;知是非之不可为分,细大之不可为倪
。闻曰:`道人不闻,至德不得,大人无己。'约分之至也。''

河伯曰:``若物之外,若物之内,恶至而倪贵贱?恶至而倪小大?
''北海若曰:``以道观之,物无贵贱;以物观之,自贵而相贱;以俗
观之,贵贱不在己。以差观之,因其所大而大之,则万物莫不大;因
其所小而小之,则万物莫不小。知天地之为稊米也,知毫末之为
丘山也,则差数睹矣。以功观之,因其所有而有之,则万物莫不有;
因其所无而无之,则万物莫不无。知东西之相反而不可以相无,则功
分定矣。以趣观之,因其所然而然之,则万物莫不然;因其所非而非
之,则万物莫不非。知尧、桀之自然而相非,则趣操睹矣。昔者尧、
舜让而帝,之、哙让而绝;汤、武争而王,白公争而灭。由此观之,
争让之礼,尧、桀之行,贵贱有时,未可以为常也。梁丽可以冲城而
不可以窒穴,言殊器也;骐骥骅骝一日而驰千里,捕鼠不如狸狌,言
殊技也;鸱鸺夜撮蚤,察毫末,昼出瞋目而不见丘山,言殊性也。故
曰:盖师是而无非,师治而无乱乎?是未明天地之理,万物之情也。
是犹师天而无地,师阴而无阳,其不可行明矣!然且语而不舍,非愚
则诬也!帝王殊禅,三代殊继。差其时,逆其俗者,谓之篡夫;当其
时,顺其俗者,谓之义之徒。默默乎河伯,女恶知贵贱之门,小大之 家!''

河伯曰:``然则我何为乎?何不为乎?吾辞受趣舍,吾终奈何?''
北海若曰:``以道观之,何贵何贱,是谓反衍;无拘而志,与道大蹇
。何少何多,是谓谢施;无一而行,与道参差。严乎若国之有君,其
无私德;繇繇乎若祭之有社,其无私福;泛泛乎其若四方之无穷,其
无所畛域。兼怀万物,其孰承翼?是谓无方。万物一齐,孰短孰长?
道无终始,物有死生,不恃其成。一虚一满,不位乎其形。年不可举
,时不可止。消息盈虚,终则有始。是所以语大义之方,论万物之理
也。物之生也,若骤若驰。无动而不变,无时而不移。何为乎,何不
为乎?夫固将自化。''

河伯曰:``然则何贵于道邪?''北海若曰:``知道者必达于理,达
于理者必明于权,明于权者不以物害己。至德者,火弗能热,水弗能
溺,寒暑弗能害,禽兽弗能贼。非谓其薄之也,言察乎安危,宁于祸
福,谨于去就,莫之能害也。故曰:`天在内,人在外,德在乎天。
'知天人之行,本乎天,位乎得,踯躅而屈伸,反要而语极。''曰:
``何谓天?何谓人?''北海若曰:``牛马四足,是谓天;落马首,穿
牛鼻,是谓人。故曰:`无以人灭天,无以故灭命,无以得殉名。谨
守而勿失,是谓反其真。'''

夔怜蚿,蚿怜蛇,蛇怜风,风怜目,目怜心。夔谓蚿曰:``吾以一
足趻踔而不行,予无如矣。今子之使万足,独奈何?''蚿曰:``不然。
子不见夫唾者乎?喷则大者如珠,小者如雾,杂而下者不可胜数也。
今予动吾天机,而不知其所以然。''蚿谓蛇曰:``吾以众足行,而不
及子之无足,何也?''蛇曰:``夫天机之所动,何可易邪?吾安用足
哉!''蛇谓风曰:``予动吾脊胁而行,则有似也。今子蓬蓬然起于北
海,蓬蓬然入于南海,而似无有,何也?''风曰:``然,予蓬蓬然起
于北海而入于南海也,然而指我则胜我,鰌我亦胜我。虽然,夫折大
木,蜚大屋者,唯我能也。''故以众小不胜为大胜也。为大胜者,唯 圣人能之。

孔子游于匡,宋人围之数匝,而弦歌不辍。子路入见,曰:``何夫
子之娱也?''孔子曰:``来,吾语女。我讳穷久矣,而不免,命也;
求通久矣,而不得,时也。当尧、舜而天下无穷人,非知得也;当桀
、纣而天下无通人,非知失也:时势适然。夫水行不避蛟龙者,渔父
之勇也;陆行不避兕虎者,猎夫之勇也;白刃交于前,视死若生者,
烈士之勇也;知穷之有命,知通之有时,临大难而不惧者,圣人之勇
也。由,处矣!吾命有所制矣!''无几何,将甲者进,辞曰:``以为
阳虎也,故围之;今非也,请辞而退。''

公孙龙问于魏牟曰:``龙少学先王之道,长而明仁义之行;合同异
,离坚白;然不然,可不可;困百家之知,穷众口之辩:吾自以为至
达已。今吾闻庄子之言,茫然异之。不知论之不及与?知之弗若与?
今吾无所开吾喙,敢问其方。''公子牟隐机大息,仰天而笑曰:``子
独不闻夫埳井之蛙乎?谓东海之鳖曰:`吾乐与!出跳梁乎井干之上,
入休乎缺甃之崖。赴水则接腋持颐,蹶泥则没足灭跗。还虷蟹与科斗,
莫吾能若也。且夫擅一壑之水,而跨跱埳井之乐,此亦至矣。夫子奚
不时来入观乎?'东海之鳖左足未入,而右膝已絷矣。于是逡巡而却,
告之海曰:`夫千里之远,不足以举其大;千仞之高,不足以极其深。
禹之时,十年九潦,而水弗为加益;汤之时,八年七旱,而崖不为加
损。夫不为顷久推移,不以多少进退者,此亦东海之大乐也。'于是
埳井之蛙闻之,适适然惊,规规然自失也。且夫知不知是非之竟,而
犹欲观于庄子之言,是犹使蚊负山,商蚷驰河也,必不胜任矣。且夫
知不知论极妙之言,而自适一时之利者,是非埳井之蛙与?且彼方跐
黄泉而登大皇,无南无北,爽然四解,沦于不测;无东无西,始于玄
冥,反于大通。子乃规规然而求之以察,索之以辩,是直用管窥天,
用锥指地也,不亦小乎?子往矣!且子独不闻夫寿陵余子之学于邯郸
与?未得国能,又失其故行矣,直匍匐而归耳。今子不去,将忘子之
故,失子之业。''公孙龙口呿而不合,舌举而不下,乃逸而走。

庄子钓于濮水。楚王使大夫二人往先焉,曰:``愿以境内累矣!''
庄子持竿不顾,曰:``吾闻楚有神龟,死已三千岁矣。王巾笥而藏之
庙堂之上。此龟者,宁其死为留骨而贵乎?宁其生而曳尾于涂中乎?
''二大夫曰:``宁生而曳尾涂中。''庄子曰:``往矣!吾将曳尾于涂 中。''

惠子相梁,庄子往见之。或谓惠子曰:``庄子来,欲代子相。''于
是惠子恐,搜于国中三日三夜。庄子往见之,曰:``南方有鸟,其名
为鹓鹐,子知之乎?夫鹓鹐发于南海而飞于北海,非梧桐不止,非练
实不食,非醴泉不饮。于是鸱得腐鼠,鹓鹐过之,仰而视之曰:`吓!'
今子欲以子之梁国而吓我邪?''

庄子与惠子游于濠梁之上。庄子曰:``儵鱼出游从容,是鱼之乐也。''
惠子曰∶``子非鱼,安知鱼之乐?''庄子曰:``子非我,安知我不知
鱼之乐?''惠子曰``我非子,固不知子矣;子固非鱼也,子之不知鱼
之乐,全矣!''庄子曰:``请循其本。子曰`汝安知鱼乐'云者,既
已知吾知之而问我。我知之濠上也。''

\hypertarget{header-n2213}{%
\subsubsection{至乐}\label{header-n2213}}

天下有至乐无有哉?有可以活身者无有哉?今奚为奚据?奚避奚处
?奚就奚去?奚乐奚恶?夫天下之所尊者,富贵寿善也;所乐者,身
安厚味美服好色音声也;所下者,贫贱夭恶也;所苦者,身不得安逸
,口不得厚味,形不得美服,目不得好色,耳不得音声。若不得者,
则大忧以惧,其为形也亦愚哉!夫富者,苦身疾作,多积财而不得尽
用,其为形也亦外矣!夫贵者,夜以继日,思虑善否,其为形也亦疏
矣!人之生也,与忧俱生。寿者惛惛,久忧不死,何之苦也!其为形
也亦远矣!烈士为天下见善矣,未足以活身。吾未知善之诚善邪?诚
不善邪?若以为善矣,不足活身;以为不善矣,足以活人。故曰:``忠
谏不听,蹲循勿争。''故夫子胥争之,以残其形;不争,名亦不成。
诚有善无有哉?今俗之所为与其所乐,吾又
未知乐之果乐邪?果不乐邪?吾观夫俗之所乐,举群趣者,硁硁然如
将不得已,而皆曰乐者,吾未之乐也,亦未之不乐也。果有乐无有哉?
吾以无为诚乐矣,又俗之所大苦也。故曰:``至乐无乐,至誉无誉。''
天下是非果未可定也。虽然,无为可以定是非。至乐活身,唯无为几
存。请尝试言之:天无为以之清,地无为以之宁。故两无为相合,万
物皆化生。芒乎芴乎,而无从出乎!芴乎芒乎,而无有象乎!万物职
职,皆从无为殖。故曰:``天地无为也而无不为也。''人也孰能得无 为哉!

庄子妻死,惠子吊之,庄子则方箕踞鼓盆而歌。惠子曰:``与人居
,长子、老、身死,不哭亦足矣,又鼓盆而歌,不亦甚乎!''庄子曰
:``不然。是其始死也,我独何能无概!然察其始而本无生;非徒无
生也,而本无形;非徒无形也,而本无气。杂乎芒芴之间,变而有气
,气变而有形,形变而有生。今又变而之死。是相与为春秋冬夏四时
行也。人且偃然寝于巨室,而我噭噭然随而哭之,自以为不通乎命, 故止也。''

支离叔与滑介叔观于冥伯之丘,昆仑之虚,黄帝之所休。俄而柳生
其左肘,其意蹶蹶然恶之。支离叔曰:``子恶之乎?''滑介叔曰:``
亡,予何恶!生者,假借也。假之而生生者,尘垢也。死生为昼夜。
且吾与子观化而化及我,我又何恶焉!''

庄子之楚,见空髑髅,髐然有形。撽以马捶,因而问
之,曰:``夫子贪生失理而为此乎?将子有亡国之事、斧铖之诛而为
此乎?将子有不善之行,愧遗父母妻子之丑而为此乎?将子有冻馁之
患而为此乎?将子之春秋故及此乎?''于是语卒,援髑髅,枕而卧。
夜半,髑髅见梦曰:``子之谈者似辩士,诸子所言,皆生人之累也,
死则无此矣。子欲闻死之说乎?''庄子曰:``然。''髑髅曰:``死,
无君于上,无臣于下,亦无四时之事,从然以天地为春秋,虽南面王
乐,不能过也。''庄子不信,曰:``吾使司命复生子形,为子骨肉肌
肤,反子父母、妻子、闾里、知识,子欲之乎?''髑髅深颦蹙额曰:
``吾安能弃南面王乐而复为人间之劳乎!''

颜渊东之齐,孔子有忧色。子贡下席而问曰:``小子敢问:回东之
齐,夫子有忧色,何邪?''孔子曰:``善哉汝问。昔者管子有言,丘
甚善之,曰`褚小者不可以怀大,绠短者不可以汲深。'夫若是者,
以为命有所成而形有所适也,夫不可损益。吾恐回与齐侯言尧、舜、
黄帝之道,而重以燧人、神农之言。彼将内求于己而不得,不得则惑
,人惑则死。且女独不闻邪?昔者海鸟止于鲁郊,鲁侯御而觞之于庙
,奏九韶以为乐,具太牢以为膳。鸟乃眩视忧悲,不敢食一脔,不敢
饮一杯,三日而死。此以己养养鸟也,非以鸟养养鸟也。夫以鸟养养
鸟者,宜栖之深林,游之坛陆,浮之江湖,食之鳅鲦,随行列而止,
逶迤而处。彼唯人言之恶闻,奚以夫譊为乎!咸池九韶之乐,张之洞
庭之野,鸟闻之而飞,兽闻之而走,鱼闻之而下入,人卒闻之,相与
还而观之。鱼处水而生,人处水而死。彼必相与异,其好恶故异也。
故先圣不一其能,不同其事。名止于实,义设于适,是之谓条达而福 持。''

列子行,食于道,从见百岁髑髅,攓蓬而指之曰:``唯予与汝知而
未尝死、未尝生也。若果养乎?予果欢乎?''种有几,得水则为继,
得水土之际则为蛙蠙之衣,生于陵屯则为陵舄,陵舄得郁栖则为乌足,
乌足之根为蛴螬,其叶为胡蝶。胡蝶胥也化而为虫,生于灶下,其状
若脱,其名为鸲掇。鸲掇千日为鸟,其名为干余骨。干余骨之沫为斯
弥,斯弥为食醯。颐辂生乎食醯,黄軦生乎九猷,瞀芮生乎腐蠸,羊奚比乎不箰,久竹
生青宁,青宁生程,程生马,马生人,人又反入于机。万物皆出于机
,皆入于机。''

\hypertarget{header-n2223}{%
\subsubsection{达生}\label{header-n2223}}

达生之情者,不务生之所无以为;达命之情者,不务知之所无奈何
。养形必先之以物,物有余而形不养者有之矣。有生必先无离形,形
不离而生亡者有之矣。生之来不能却,其去不能止。悲夫!世之人以
为养形足以存生,而养形果不足以存生,则世奚足为哉!虽不足为而
不可不为者,其为不免矣!夫欲免为形者,莫如弃世。弃世则无累,
无累则正平,正平则与彼更生,更生则几矣!事奚足遗弃而生奚足遗
?弃事则形不劳,遗生则精不亏。夫形全精复,与天为一。天地者,
万物之父母也。合则成体,散则成始。形精不亏,是谓能移。精而又
精,反以相天。

子列子问关尹曰:``至人潜行不窒,蹈火不热,行乎万物之上而不
栗。请问何以至于此?''关尹曰:``是纯气之守也,非知巧果敢之列
。居,予语女。凡有貌象声色者,皆物也,物与物何以相远!夫奚足
以至乎先!是色而已。则物之造乎不形,而止乎无所化。夫得是而穷
之者,物焉得而止焉!彼将处乎不淫之度,而藏乎无端之纪,游乎万
物之所终始。壹其性,养其气,合其德,以通乎物之所造。夫若是者
,其天守全,其神无隙,物奚自入焉!夫醉者之坠车,虽疾不死。骨
节与人同而犯害与人异,其神全也。乘亦不知也,坠亦不知也,死生
惊惧不入乎其胸中,是故囗(``逆''字的右上加``口口''音e4)物
而不慴。彼得全于酒而犹若是,而况得全于天乎?圣人藏于天,故莫
之能伤也。复仇者,不折镆干;虽有忮心者,不怨飘瓦,是以天下平
均。故无攻战之乱,无杀戮之刑者,由此道也。不开人之天,而开天
之天。开天者德生,开人者贼生。不厌其天,不忽于人,民几乎以其 真。''

仲尼适楚,出于林中,见佝偻者承蜩,犹掇之也。仲尼曰:``子巧
乎,有道邪?''曰:``我有道也。五六月累丸二而不坠,则失者锱铢
;累三而不坠,则失者十一;累五而不坠,犹掇之也。吾处身也,若
蹶株拘;吾执臂也,若槁木之枝。虽天地之大,万物之多,而唯蜩翼
之知。吾不反不侧,不以万物易蜩之翼,何为而不得!''孔子顾谓弟
子曰:``用志不分,乃凝于神。其佝偻丈人之谓乎!''

颜渊问仲尼曰:``吾尝济乎觞深之渊,津人操舟若神。吾问焉曰:
`操舟可学邪?'曰:`可。善游者数能。若乃夫没人,则未尝见舟
而便操之也。'吾问焉而不吾告,敢问何谓也?''仲尼曰:``善游者
数能,忘水也;若乃夫没人之未尝见舟而便操之也,彼视渊若陵,视
舟若履,犹其车却也。覆却万方陈乎前而不得入其舍,恶往而不暇!
以瓦注者巧,以钩注者惮,以黄金注者殙。其巧一也,而有所矜,则
重外也。凡外重者内拙。''

田开之见周威公,威公曰:``吾闻祝肾学生,吾子与祝肾游,亦何
闻焉?''田开之曰:``开之操拔篲以侍门庭,亦何闻于夫子!''威公
曰:``田子无让,寡人愿闻之。''开之曰:``闻之夫子曰:`善养生
者,若牧羊然,视其后者而鞭之。'''威公曰:``何谓也?''田开之
曰:``鲁有单豹者,岩居而水饮,不与民共利,行年七十而犹有婴儿
之色,不幸遇饿虎,饿虎杀而食之。有张毅者,高门县薄,无不走也,
行年四十而有内热之病以死。豹养其内而虎食其外,毅养其外而病攻
其内。此二子者,皆不鞭其后者也。''仲尼曰:``无入而藏,无出而
阳,柴立其中央。三者若得,其名必极。夫畏涂者,十杀一人,则父
子兄弟相戒也,必盛卒徒而后敢出焉,不亦知乎!人之所取畏者,衽
席之上,饮食之间,而不知为之戒者,过也!''

祝宗人玄端以临牢柙说彘,曰:``汝奚恶死!吾将三月囗(``物''
字以``豢''代``勿''音huan4)汝,十日戒,三日齐,藉白茅,
加汝肩尻乎雕俎之上,则汝为之乎?''为彘谋曰:``不如食以糠糟而
错之牢柙之中。''自为谋,则苟生有轩冕之尊,死得于腞楯之上、聚
偻之中则为之。为彘谋则去之,自为谋则取之,所异彘者何也!

桓公田于泽,管仲御,见鬼焉。公抚管仲之手曰:``仲父何见?''
对曰:``臣无所见。''公反,诶诒为病,数日不出。齐士有皇子告敖
者,曰:``公则自伤,鬼恶能伤公!夫忿滀之气,散而不反,则为不
足;上而不下,则使人善怒;下而不上,则使人善忘;不上不下,中
身当心,则为病。''桓公曰:``然则有鬼乎?
''曰:``有。沈有履。灶有髻。户内之烦壤,雷霆处之;东北方之下
者倍阿,鲑囗(上``龙''下``虫''音long2)跃之;西北方之下
者,则泆阳处之。水有罔象,丘有峷,山有夔,野有彷徨,泽有委蛇
。''公曰:``请问委蛇之伏状何如?''皇子曰:``委蛇,其大如毂,
其长如辕,紫衣而朱冠。其为物也恶,闻雷车之声则捧其首而立。见
之者殆乎霸。''桓公囗(左``单''右``辰''音zhen3)然而笑曰
:``此寡人之所见者也。''于是正衣冠与之坐,不终日而不知病之去 也。

纪渻子为王养斗鸡。十日而问:``鸡已乎?''曰:``未也,方虚骄
而恃气。''十日又问,曰:``未也,犹应向景。''十日又问,曰:
``未也,犹疾视而盛气。''十日又问,曰:``几矣,鸡虽有鸣者,已
无变矣,望之似木鸡矣,其德全矣。异鸡无敢应者,反走矣。''

孔子观于吕梁,县水三十仞,流沫四十里,鼋鼍鱼鳖之所不能游也
。见一丈夫游之,以为有苦而欲死也。使弟子并流而拯之。数百步而
出,被发行歌而游于塘下。孔子从而问焉,曰:``吾以子为鬼,察子
则人也。请问:蹈水有道乎?''曰:``亡,吾无道。吾始乎故,长乎
性,成乎命。与齐俱入,与汩偕出,从水之道而不为私焉。此吾所以
蹈之也。''孔子曰:``何谓始乎故,长乎性,成乎命?''曰:``吾生
于陵而安于陵,故也;长于水而安于水,性也;不知吾所以然而然, 命也。''

梓庆削木为鐻,日成,见者惊犹鬼神。鲁侯见而问焉,曰:``子何
术以为焉?''对曰:``臣,工人,何术之有!虽然,有一焉:臣将为
鐻,未尝敢以耗气也,必齐以静心。齐三日,而不敢怀庆赏爵禄;齐
五日,不敢怀非誉巧拙;齐七日,辄然忘吾有四枝形体也。当是时也,
无公朝。其巧专而外骨消,然后入山林,观天性形躯,至矣,然后成鐻,
然后加手焉,不然则已。则以天合天,器之所以疑神者,其是与!''

东野稷以御见庄公,进退中绳,左右旋中规。庄公以为文弗过也。
使之钩百而反。颜阖遇之,入见曰:``稷之马将败。''公密而不应。
少焉,果败而反。公曰:``子何以知之?''曰:``其马力竭矣而犹求
焉,故曰败。''

工倕旋而盖规矩,指与物化而不以心稽,故其灵台一而不桎。忘足,
履之适也;忘要,带之适也;知忘是非,心之适也;不内变,不外从,
事会之适也;始乎适而未尝不适者,忘适之适也。

有孙休者,踵门而诧子扁庆子曰:``休居乡不见谓不修,临难不见
谓不勇。然而田原不遇岁,事君不遇世,宾于乡里,逐于州部,则胡
罪乎天哉?休恶遇此命也?''扁子曰:``子独不闻夫至人之自行邪?
忘其肝胆,遗其耳目,芒然彷徨乎尘垢之外,逍遥乎无事之业,是谓
为而不恃,长而不宰。今汝饰知以惊愚,修身以明汙,昭昭乎若揭日
月而行也。汝得全而形躯,具而九窍,无中道夭于聋盲跛蹇而比于人
数亦幸矣,又何暇乎天之怨哉!子往矣!''孙子出,扁子入。坐有间,
仰天而叹。弟子问曰:``先生何为叹乎?''扁子曰∶``向者休来,吾
告之以至人之德,吾恐其惊而遂至于惑也。''弟子曰:``不然。孙子
之所言是邪,先生之所言非邪,非固不能惑是;孙子所言非邪,先生
所言是邪,彼固惑而来矣,又奚罪焉!''扁子曰:``不然。昔者有鸟
止于鲁郊,鲁君说之,为具太牢以飨之,奏九韶以乐之。鸟乃始忧悲
眩视,不敢饮食。此之谓以己养养鸟也。若夫以鸟养养鸟者,宜栖之
深林,浮之江湖,食之以委蛇,则安平陆而已矣。今休,款启寡闻之
民也,吾告以至人之德,譬之若载鼷以车马,乐鴳以钟鼓也,彼又恶
能无惊乎哉!''

\hypertarget{header-n2240}{%
\subsubsection{山木}\label{header-n2240}}

庄子行于山中,见大木,枝叶盛茂。伐木者止其旁而不取也。问其
故,曰:``无所可用。''庄子曰:``此木以不材得终其天年。''夫子
出于山,舍于故人之家。故人喜,命竖子杀雁而烹之。竖子请曰:``
其一能鸣,其一不能鸣,请奚杀?''主人曰:``杀不能鸣者。''明日
,弟子问于庄子曰:``昨日山中之木,以不材得终其天年;今主人之
雁,以不材死。先生将何处?''庄子笑曰:``周将处乎材与不材之间
。材与不材之间,似之而非也,故未免乎累。若夫乘道德而浮游则不
然,无誉无訾,一龙一蛇,与时俱化,而无肯专为。一上一下,以和
为量,浮游乎万物之祖。物物而不物于物,则胡可得而累邪!此神农
、黄帝之法则也。若夫万物之情,人伦之传则不然:合则离,成则毁
,廉则挫,尊则议,有为则亏,贤则谋,不肖则欺。胡可得而必乎哉
!悲夫,弟子志之,其唯道德之乡乎!''

市南宜僚见鲁侯,鲁侯有忧色。市南子曰:``君有忧色,何也?''
鲁侯曰:``吾学先王之道,修先君之业;吾敬鬼尊贤,亲而行之,无
须臾离居。然不免于患,吾是以忧。''市南子曰:``君之除患之术浅
矣!夫丰狐文豹,栖于山林,伏于岩穴,静也;夜行昼居,戒也;虽
饥渴隐约,犹且胥疏于江湖之上而求食焉,定也。然且不免于罔罗机
辟之患,是何罪之有哉?其皮为之灾也。今鲁国独非君之皮邪?吾愿
君刳形去皮,洒心去欲,而游于无人之野。南越有邑焉,名为建德之
国。其民愚而朴,少私而寡欲;知作而不知藏,与而不求其报;不知
义之所适,不知礼之所将。猖狂妄行,乃蹈乎大方。其生可乐,其死
可葬。吾愿君去国捐俗,与道相辅而行。''君曰:``彼其道远而险,
又有江山,我无舟车,奈何?''市南子曰:``君无形倨,无留居,以
为君车。''君曰:``彼其道幽远而无人,吾谁与为邻?吾无粮,我无
食,安得而至焉?''市南子曰:``少君之费,寡君之欲,虽无粮而乃
足。君其涉于江而浮于海,望之而不见其崖,愈往而不知其所穷。送
君者皆自崖而反。君自此远矣!故有人者累,见有于人者忧。故尧非
有人,非见有于人也。吾愿去君之累,除君之忧,而独与道游于大莫
之国。方舟而济于河,有虚船来触舟,虽有囗(左``忄''右``扁''音
bian3)心之人不怒。有一人在其上,则呼张歙之。一呼而不闻
,再呼而不闻,于是三呼邪,则必以恶声随之。向也不怒而今也怒,
向也虚而今也实。人能虚己以游世,其孰能害之!''

北宫奢为卫灵公赋敛以为钟,为坛乎郭门之外。三月而成上下之县
。王子庆忌见而问焉,曰:``子何术之设?''奢曰:``一之间无敢设
也。奢闻之:`既雕既琢,复归于朴。'侗乎其无识,傥乎其怠疑。
萃乎芒乎,其送往而迎来。来者勿禁,往者勿止。从其强梁,随其曲
傅,因其自穷。故朝夕赋敛而毫毛不挫,而况有大涂者乎!''

孔子围于陈蔡之间,七日不火食。大公任往吊之,曰:``子几死乎
?''曰:``然。''``子恶死乎?''曰:``然。''任曰:``予尝言不死
之道。东海有鸟焉,其名曰意怠。其为鸟也,囗囗(左``羽''右``分
'')囗囗(左``羽''右``失''),而似无能;引援而飞,迫胁而栖;
进不敢为前,退不敢为后;食不敢先尝,必取其绪。是故其行列不斥
,而外人卒不得害,是以免于患。直木先伐,甘井先竭。子其意者饰
知以惊愚,修身以明囗(左``氵''右``于''),昭昭乎如揭日月而行
,故不免也。昔吾闻之大成之人曰:`自伐者无功,功成者堕,名成
者亏。'孰能去功与名而还与众人!道流而不明居,得行而不名处;
纯纯常常,乃比于狂;削迹捐势,不为功名。是故无责于人,人亦无
责焉。至人不闻,子何喜哉!''孔子曰:``善哉!''辞其交游,去其
弟子,逃于大泽,衣裘褐,食杼栗,入兽不乱群,入鸟不乱行。鸟兽
不恶,而况人乎!

孔子问子桑囗(上``雨''下``乎''音hu4)曰:``吾再逐于鲁,
伐树于宋,削迹于卫,穷于商周,围于陈蔡之间。吾犯此数患,亲交
益疏,徒友益散,何与?''子桑hu4曰:``子独不闻假人之亡与?
林回弃千金之璧,负赤子而趋。或曰:`为其布与?赤子之布寡矣;
为其累与?赤子之累多矣。弃千金之璧,负赤子而趋,何也?'林回
曰:`彼以利合,此以天属也。'夫以利合者,迫穷祸患害相弃也;
以天属者,迫穷祸患害相收也。夫相收之与相弃亦远矣,且君子之交淡若水,小人之交甘若醴。君子淡以亲,小人甘以绝,彼无故以合者
,则无故以离。''孔子曰:``敬闻命矣!''徐行翔佯而归,绝学捐书
,弟子无挹于前,其爱益加进。异日,桑hu4又曰:``舜之将死,
真泠禹曰:`汝戒之哉!形莫若缘,情莫若率。'缘则不离,率则不
劳。不离不劳,则不求文以待形。不求文以待形,固不待物。''

庄子衣大布而补之,正囗(``契''字以``糸''代``大''音xie2
)系履而过魏王。魏王曰:``何先生之惫邪?''庄子曰:``贫也,非
惫也。士有道德不能行,惫也;衣弊履穿,贫也,非惫也,此所谓非
遭时也。王独不见夫腾猿乎?其得楠梓豫章也,揽蔓其枝而王长其间
,虽羿、蓬蒙不能眄睨也。及其得柘棘枳枸之间也,危行侧视,振动
悼栗,此筋骨非有加急而不柔也,处势不便,未足以逞其能也。今处
昏上乱相之间而欲无惫,奚可得邪?此比干之见剖心,徵也夫!''

孔子穷于陈蔡之间,七日不火食。左据槁木,右击槁枝,而歌焱氏
之风,有其具而无其数,有其声而无宫角。木声与人声,犁然有当于
人之心。颜回端拱还目而窥之。仲尼恐其广己而造大也,爱己而造哀
也,曰:``回,无受天损易,无受人益难。无始而非卒也,人与天一
也。夫今之歌者其谁乎!''回曰:``敢问无受天损易。''仲尼曰:``
饥渴寒暑,穷桎不行,天地之行也,运物之泄也,言与之偕逝之谓也
。为人臣者,不敢去之。执臣之道犹若是,而况乎所以待天乎?''``
何谓无受人益难?''仲尼曰:``始用四达,爵禄并至而不穷。物之所
利,乃非己也,吾命有在外者也。君子不为盗,贤人不为窃,吾若取
之何哉?故曰:鸟莫知于囗(左``意''右``鸟''音yi4)鸸,目之
所不宜处不给视,虽落其实,弃之而走。其畏人也而袭诸人间。社稷
存焉尔!''``何谓无始而非卒?''仲尼曰:``化其万物而不知其禅之
者,焉知其所终?焉知其所始?正而待之而已耳。''``何谓人与天一
邪?''仲尼曰:``有人,天也;有天,亦天也。人之不能有天,性也
。圣人晏然体逝而终矣!''

庄周游于雕陵之樊,睹一异鹊自南方来者。翼广七尺,目大运寸,
感周之颡,而集于栗林。庄周曰:``此何鸟哉!翼殷不逝,目大不睹
。''蹇裳囗(左``足''右``矍''音jue2)步,执弹而留之。睹一
蝉方得美荫而忘其身。螳螂执翳而搏之,见得而忘形。异鹊从而利之
,见利而忘其真。庄周怵然曰:``噫!物固相累,二类相召也。''捐
弹而反走,虞人逐而谇之。庄周反入,三日不庭。蔺且从而问之,``
夫子何为顷间甚不庭乎?''庄周曰:``吾守形而忘身,观于浊水而迷
于清渊。且吾闻诸夫子曰:`入其俗,从其令。'今吾游于雕陵而忘
吾身,异鹊感吾颡,游于栗林而忘真。栗林虞人以吾为戮,吾所以不 庭也。''

阳子之宋,宿于逆旅。逆旅人有妾二人,其一人美,其一人恶。恶
者贵而美者贱。阳子问其故,逆旅小子对曰:``其美者自美,吾不知
其美也;其恶者自恶,吾不知其恶也。''阳子曰:``弟子记之:行贤
而去自贤之行,安往而不爱哉!''

\hypertarget{header-n2253}{%
\subsubsection{田子方}\label{header-n2253}}

田子方侍坐于魏文侯,数称谿工。文侯曰:``谿工,子之师邪?''
子方曰:``非也,无择之里人也。称道数当故无择称之。''文侯曰:
``然则子无师邪?''子方曰:``有。''曰:``子之师谁邪?''子方曰:
``东郭顺子。''文侯曰:``然则夫子何故未尝称之?''子方曰:``其
为人也真。人貌而天虚,缘而葆真,清而容物。物无道,正容以悟之,
使人之意也消。无择何足以称之!''子方出,文侯傥然,终日不言。
召前立臣而语之曰:``远矣,全德之君子!始吾以圣知之言、仁义之
行为至矣。吾闻子方之师,吾形解而不欲动,口钳而不欲言。吾所学
者,直土埂耳!夫魏真为我累耳!''

温伯雪子适齐,舍于鲁。鲁人有请见之者,温伯雪子曰:``不可。
吾闻中国之君子,明乎礼义而陋于知人心。吾不欲见也。''至于齐,
反舍于鲁,是人也又请见。温伯雪子曰:``往也蕲见我,今也又蕲见
我,是必有以振我也。''出而见客,入而叹。明日见客,又入而叹。
其仆曰:``每见之客也,必入而叹,何耶?''曰:``吾固告子矣:中
国之民,明乎礼义而陋乎知人心。昔之见我者,进退一成规、一成矩
,从容一若龙、一若虎。其谏我也似子,其道我也似父,是以叹也。
''仲尼见之而不言。子路曰:``吾子欲见温伯雪子久矣。见之而不言
,何邪?''仲尼曰:``若夫人者,目击而道存矣,亦不可以容声矣! ''

颜渊问于仲尼曰:``夫子步亦步,夫子趋亦趋,夫子驰亦驰,夫子
奔逸绝尘,而回瞠若乎后矣!''夫子曰:``回,何谓邪?''曰:``夫
子步亦步也,夫子言亦言也;夫子趋亦趋也,夫子辩亦辩也;夫子驰
亦驰也,夫子言道,回亦言道也;及奔逸绝尘而回瞠若乎后者,夫子
不言而信,不比而周,无器而民滔乎前,而不知所以然而已矣。''仲
尼曰:``恶!可不察与!夫哀莫大于心死,而人死亦次之。日出东方
而入于西极,万物莫不比方,有目有趾者,待是而后成功。是出则存
,是入则亡。万物亦然,有待也而死,有待也而生。吾一受其成形,
而不化以待尽。效物而动,日夜无隙,而不知其所终。薰然其成形,
知命不能规乎其前。丘以是日徂。吾终身与汝交一臂而失之,可不哀
与?女殆著乎吾所以著也。彼已尽矣,而女求之以为有,是求马于唐
肆也。吾服,女也甚忘;女服,吾也甚忘。虽然,女奚患焉!虽忘乎
故吾,吾有不忘者存。''

孔子见老聃,老聃新沐,方将被发而干,蛰然似非人。孔子便而待
之。少焉见,曰:``丘也眩与?其信然与?向者先生形体掘若槁木,
似遗物离人而立于独也。''老聃曰:``吾游心于物之初。''孔子曰:
``何谓邪?''曰:``心困焉而不能知,口辟焉而不能言。尝为汝议乎
其将:至阴肃肃,至阳赫赫。肃肃出乎天,赫赫发乎地。两者交通成
和而物生焉,或为之纪而莫见其形。消息满虚,一晦一明,日改月化
,日有所为而莫见其功。生有所乎萌,死有所乎归,始终相反乎无端
,而莫知乎其所穷。非是也,且孰为之宗!''孔子曰:``请问游是。
''老聃曰:``夫得是至美至乐也。得至美而游乎至乐,谓之至人。''
孔子曰:``愿闻其方。''曰:``草食之兽,不疾易薮;水生之虫,不
疾易水。行小变而不失其大常也,喜怒哀乐不入于胸次。夫天下也者
,万物之所一也。得其所一而同焉,则四支百体将为尘垢,而死生终
始将为昼夜,而莫之能滑,而况得丧祸福之所介乎!弃隶者若弃泥涂
,知身贵于隶也。贵在于我而不失于变。且万化而未始有极也,夫孰
足以患心!已为道者解乎此。''孔子曰:``夫子德配天地,而犹假至
言以修心。古之君子,孰能脱焉!''老聃曰:``不然。夫水之于汋也,
无为而才自然矣;至人之于德也,不修而物不能离焉。若天之自高,
地之自厚,日月之自明,夫何修焉!''孔子出,以告颜回曰:``丘之
于道也,其犹醯鸡与!微夫子之发吾覆也,吾不知天地之大全也。''

庄子见鲁哀公,哀公曰:``鲁多儒士,少为先生方者。''庄子曰:
``鲁少儒。''哀公曰:``举鲁国而儒服,何谓少乎?''庄子曰:``周
闻之:儒者冠圜冠者知天时,履句履者知地形,缓佩玦者事至而断。
君子有其道者,未必为其服也;为其服者,未必知其道也。公固以为
不然,何不号于国中曰:`无此道而为此服者,其罪死!'''于是哀
公号之五日,而鲁国无敢儒服者。独有一丈夫,儒服而立乎公门。公
即召而问以国事,千转万变而不穷。庄子曰:``以鲁国而儒者一人耳,
可谓多乎?''

百里奚爵禄不入于心,故饭牛而牛肥,使秦穆公忘其贱,与之政也
。有虞氏死生不入于心,故足以动人。

宋元君将画图,众史皆至,受揖而立,舐笔和墨,在外者半。有一
史后至者,儃儃然不趋,受揖不立,因之舍。公使人视之,则解衣般
礴裸。君曰:``可矣,是真画者也。''

文王观于臧,见一丈夫钓,而其钓莫钓。非持其钓有钓者也,常钓
也。文王欲举而授之政,而恐大臣父兄之弗安也;欲终而释之,而不
忍百姓之无天也。于是旦而属之大夫曰:``昔者寡人梦见良人,黑色
而髯,乘驳马而偏朱蹄,号曰:`寓而政于臧丈人,庶几乎民有瘳乎
!'''诸大夫蹴然曰∶``先君王也。''文王曰:``然则卜之。''诸大
夫曰∶``先君之命,王其无它,又何卜焉。''遂迎臧丈人而授之政。
典法无更,偏令无出。三年,文王观于国,则列士坏植散群,长官者
不成德,斔斛不敢入于四竟。列士坏植散群,则尚同也;长官者不成
德,则同务也,斔斛不敢入于四竟,则诸侯无二心也。文王于是焉以
为大师,北面而问曰:``政可以及天下乎?''臧丈人昧然而不应,泛
然而辞,朝令而夜循,终身无闻。颜渊问于仲尼曰:``文王其犹未邪?
又何以梦为乎?''仲尼曰:``默,汝无言!夫文王尽之也,而又何论
剌焉!彼直以循斯须也。''

列御寇为伯昏无人射,引之盈贯,措杯水其肘上,发之,适矢复沓
,方矢复寓。当是时,犹象人也。伯昏无人曰:``是射之射,非不射
之射也。尝与汝登高山,履危石,临百仞之渊,若能射乎?''于是无
人遂登高山,履危石,临百仞之渊,背逡巡,足二分垂在外,揖御寇
而进之。御寇伏地,汗流至踵。伯昏无人曰:``夫至人者,上窥青天
,下潜黄泉,挥斥八极,神气不变。今汝怵然有恂目之志,尔于中也 殆矣夫!''

肩吾问于孙叔敖曰:``子三为令尹而不荣华,三去之而无忧色。吾
始也疑子,今视子之鼻间栩栩然,子之用心独奈何?''孙叔敖曰:``
吾何以过人哉!吾以其来不可却也,其去不可止也。吾以为得失之非
我也,而无忧色而已矣。我何以过人哉!且不知其在彼乎?其在我乎
?其在彼邪亡乎我,在我邪亡乎彼。方将踌躇,方将四顾,何暇至乎
人贵人贱哉!''仲尼闻之曰:``古之真人,知者不得说,美人不得滥
,盗人不得劫,伏戏、黄帝不得友。死生亦大矣,而无变乎己,况爵
禄乎!若然者,其神经乎大山而无介,入乎渊泉而不濡,处卑细而不
惫,充满天地,既以与人己愈有。''

楚王与凡君坐,少焉,楚王左右曰``凡亡''者三。凡君曰:``凡之
亡也,不足以丧吾存。夫凡之亡不足以丧吾存,则楚之存不足以存存
。由是观之,则凡未始亡而楚未始存也。

\hypertarget{header-n2268}{%
\subsubsection{知北游}\label{header-n2268}}

知北游于玄水之上,登隐弅之丘,而适遭无为谓焉。知谓无为谓曰:
``予欲有问乎若:何思何虑则知道?何处何服则安道?何从何道则得
道?''三问而无为谓不答也。非不答,不知答也。知不得问,反于白
水之南,登狐阕之上,而睹狂屈焉。知以之言也问乎狂屈。狂屈曰:
``唉!予知之,将语若。''中欲言而忘其所欲言。知不得问,反于帝
宫,见黄帝而问焉。黄帝曰:``无思无虑始知道,无处无服始安道,
无从无道始得道。''知问黄帝曰:``我与若知之,彼与彼不知也,其
孰是邪?''黄帝曰:``彼无为谓
真是也,狂屈似之,我与汝终不近也。夫知者不言,言者不知,故圣
人行不言之教。道不可致,德不可至。仁可为也,义可亏也,礼相伪
也。故曰:`失道而后德,失德而后仁,失仁而后义,失义而后礼。
'礼者,道之华而乱之首也。故曰:`为道者日损,损之又损之,以
至于无为。无为而无不为也。'今已为物也,欲复归根,不亦难乎!
其易也其唯大人乎!生也死之徒,死也生之始,孰知其纪!人之生,
气之聚也。聚则为生,散则为死。若死生为徒,吾又何患!故万物一
也。是其所美者为神奇,其所恶者为臭腐。臭腐复化为神奇,神奇复
化为臭腐。故曰:`通天下一气耳。'圣人故贵一。''知谓黄帝曰:
``吾问无为谓,无为谓不应我,非不我应,不知应我也;吾问狂屈,
狂屈中欲告我而不我告,非不我告,中欲告而忘之也;今予问乎若,
若知之,奚故不近?''黄帝曰:``彼其真是也,以其不知也;此其似
之也,以其忘之也;予与若终不近也,以其知之也。''狂屈闻之,以
黄帝为知言。

天地有大美而不言,四时有明法而不议,万物有成理而不说。圣人
者,原天地之美而达万物之理。是故至人无为,大圣不作,观于天地
之谓也。今彼神明至精,与彼百化。物已死生方圆,莫知其根也。扁
然而万物,自古以固存。六合为巨,未离其内;秋豪为小,待之成体
;天下莫不沈浮,终身不故;阴阳四时运行,各得其序;惛然若亡而
存;油然不形而神;万物畜而不知:此之谓本根,可以观于天矣!

啮缺问道乎被衣,被衣曰:``若正汝形,一汝视,天和将至;摄汝
知,一汝度,神将来舍。德将为汝美,道将为汝居。汝瞳焉如新生之
犊而无求其故。''言未卒,啮缺睡寐。被衣大说,行歌而去之,曰:
``形若槁骸,心若死灰,真其实知,不以故自持。媒媒晦晦,无心而
不可与谋。彼何人哉!''

舜问乎丞:``道可得而有乎?''曰:``汝身非汝有也,汝何得有夫
道!''舜曰:``吾身非吾有也,孰有之哉?''曰:``是天地之委形也
;生非汝有,是天地之委和也;性命非汝有,是天地之委顺也;子孙
非汝有,是天地之委蜕也。故行不知所往,处不知所持,食不知所味
。天地之强阳气也,又胡可得而有邪!''

孔子问于老聃曰:``今日晏闲,敢问至道。''老聃曰:``汝齐戒,
疏瀹而心,澡雪而精神,掊击而知。夫道,窨然难言哉!将为汝言其崖
略:夫昭昭生于冥冥,有伦生于无形,精神生于道,形本生于精,而万
物以形相生。故九窍者胎生,
八窍者卵生。其来无迹,其往无崖,无门无房,四达之皇皇也。邀于
此者,四肢强,思虑恂达,耳目聪明。其用心不劳,其应物无方,天
不得不高,地不得不广,日月不得不行,万物不得不昌,此其道与!
且夫博之不必知,辩之不必慧,圣人以断之矣!若夫益之而不加益,
损之而不加损者,圣人之所保也。渊渊乎其若海,魏魏乎其终则复始
也。运量万物而不匮。则君子之道,彼其外与!万物皆往资焉而不匮
。此其道与!

``中国有人焉,非阴非阳,处于天地之间,直且为人,将反于宗。
自本观之,生者,喑噫物也。虽有寿夭,相去几何?须臾之说也,奚
足以为尧、桀之是非!果蓏有理,人伦虽难,所以相齿。圣人遭之而
不违,过之而不守。调而应之,德也;偶而应之,道也。帝之所兴,
王之所起也。

``人生天地之间,若白驹之过隙,忽然而已。注然勃然,莫不出焉
;油然寥然,莫不入焉。已化而生,又化而死。生物哀之,人类悲之
。解其天韬,堕其天帙。纷乎宛乎,魂魄将往,乃身从之。乃大归乎
!不形之形,形之不形,是人之所同知也,非将至之所务也,此众人
之所同论也。彼至则不论,论则不至;明见无值,辩不若默;道不可
闻,闻不若塞:此之谓大得。''

东郭子问于庄子曰:``所谓道,恶乎在?''庄子曰:``无所不在。
''东郭子曰:``期而后可。''庄子曰:``在蝼蚁。''曰:``何其下邪
?''曰:``在稊稗。''曰:``何其愈下邪?''曰:``在瓦甓。''曰:
``何其愈甚邪?''曰:``在屎溺。''东郭子不应。庄子曰:``夫子之
问也,固不及质。正、获之问于监市履狶也,`每下愈况'。汝唯莫必,
无乎逃物。至道若是,大言亦然。周遍咸三者,异名同实,其指一也
。尝相与游乎无有之宫,同合而论,无所终穷乎!尝相与无为乎!澹
澹而静乎!漠而清乎!调而闲乎!寥已吾志,无往焉而不知其所至,
去而来不知其所止。吾往来焉而不知其所终,彷徨乎冯闳,大知入焉
而不知其所穷。物物者与物无际,而物有际者,所谓物际者也。不际
之际,际之不际者也。谓盈虚衰杀,彼为盈虚非盈虚,彼为衰杀非衰
杀,彼为本末非本末,彼为积散非积散也。''

妸荷甘与神农同学于老龙吉。神农隐几,阖户昼瞑。囗荷甘日中奓
户而入,曰:``老龙死矣!''神农隐几拥杖而起,嚗然放杖而笑,曰:
``天知予僻陋谩诞,故弃予而死。已矣,夫子无所发予之狂言而死矣
夫!''弇堈吊闻之,曰:``夫体道者,天下之君子所系焉。今于道,
秋豪之端万分未得处一焉,而犹知藏其狂言而死,又况夫体道者乎!
视之无形,听之无声,于人之论者,谓之冥冥,所以论道而非道也。''

于是泰清问乎无穷,曰:``子知道乎?''无穷曰:``吾不知。''又
问乎无为,无为曰:``吾知道。''曰:``子之知道,亦有数乎?''曰
:``有。''曰:``其数若何?''无为曰:``吾知道之可以贵、可以贱
、可以约、可以散,此吾所以知道之数也。''泰清以之言也问乎无始
,曰:``若是,则无穷之弗知与无为之知,孰是而孰非乎?''无始曰
:``不知深矣,知之浅矣;弗知内矣,知之外矣。''于是泰清仰而叹
曰:``弗知乃知乎,知乃不知乎!孰知不知之知?''无始曰:``道不
可闻,闻而非也;道不可见,见而非也;道不可言,言而非也!知形
形之不形乎!道不当名。''无始曰:``有问道而应之者,不知道也;
虽问道者,亦未闻道。道无问,问无应。无问问之,是问穷也;无应
应之,是无内也。以无内待问穷,若是者,外不观乎宇宙,内不知乎
大初。是以不过乎昆仑,不游乎太虚。''

光曜问乎无有曰:``夫子有乎?其无有乎?''光曜不得问而孰视其
状貌:窨然空然。终日视之而不见,听之而不闻,搏之而不得也。光
曜曰:``至矣,其孰能至此乎!予能有无矣,而未能无无也。及为无
有矣,何从至此哉!''

大马之捶钩者,年八十矣,而不失豪芒。大马曰:``子巧与!有道
与?''曰:``臣有守也。臣之年二十而好捶钩,于物无视也,非钩无
察也。''是用之者假不用者也,以长得其用,而况乎无不用者乎!物 孰不资焉!

冉求问于仲尼曰:``未有天地可知邪?''仲尼曰:``可。古犹今也
。''冉求失问而退。明日复见,曰:``昔者吾问`未有天地可知乎?
'夫子曰:`可。古犹今也。'昔日吾昭然,今日吾昧然。敢问何谓
也?''仲尼曰:``昔之昭然也,神者先受之;今之昧然也,且又为不
神者求邪!无古无今,无始无终。未有子孙而有孙子可乎?''冉求未
对。仲尼曰:``已矣,末应矣!不以生生死,不以死死生。死生有待
邪?皆有所一体。有先天地生者物邪?物物者非物,物出不得先物也
,犹其有物也。犹其有物也无已!圣人之爱人也终无已者,亦乃取于 是者也。''

颜渊问乎仲尼曰:``回尝闻诸夫子曰:`无有所将,无有所迎。'
回敢问其游。''仲尼曰:``古之人外化而内不化,今之人内化而外不
化。与物化者,一不化者也。安化安不化?安与之相靡?必与之莫多
。狶韦氏之囿,黄帝之圃,有虞氏之宫,汤武之室。君子之人,若儒
墨者师,故以是非相赍也,而况今之人乎!圣人处物不伤物。不伤物
者,物亦不能伤也。唯无所伤者,为能与人相将迎。山林与,皋壤与,
使我欣欣然而乐与!乐未毕也,哀又继之。
哀乐之来,吾不能御,其去弗能止。悲夫,世人直为物逆旅耳!夫知
遇而不知所不遇,知能能而不能所不能。无知无能者,固人之所不免
也。夫务免乎人之所不免者,岂不亦悲哉!至言去言,至为去为。齐
知之,所知则浅矣!''

\hypertarget{header-n2286}{%
\subsection{杂篇}\label{header-n2286}}

\hypertarget{header-n2287}{%
\subsubsection{庚桑楚}\label{header-n2287}}

老聃之役有庚桑楚者,偏得老聃之道,以北居畏垒之山。其臣之画
然知者去之,其妾之挈然仁者远之。拥肿之与居,鞅掌之为使。居三
年,畏垒大壤。畏垒之民相与言曰:``庚桑子之始来,吾洒然异之。
今吾日计之而不足,岁计之而有余。庶几其圣人乎!子胡不相与尸而
祝之,社而稷之乎?''庚桑子闻之,南面而不释然。弟子异之。庚桑
子曰:``弟子何异于予?夫春气发而百草生,正得秋而万宝成。夫春
与秋,岂无得而然哉?天道已行矣。吾闻至人,尸居环堵之室,而百
姓猖狂,不知所如往。今以畏垒之细民,而窃窃焉欲俎豆予于贤人之
间,我其杓之人邪?吾是以不释于老聃之言。''弟子曰:``不然。夫
寻常之沟,巨鱼无所还其体,而鲵鳅为之制;步仞之丘陵,巨兽无所
隐其躯,而孽狐为之祥。且夫尊贤授能,先善与利,自古尧、舜以然
,而况畏垒之民乎!夫子亦听矣!''庚桑子曰:``小子来!夫函车之
兽,介而离山,则不免于网罟之患;吞舟之鱼,荡而失水,则蚁能苦
之。故鸟兽不厌高,鱼鳖不厌深。夫全其形生之人,藏其身也,不厌
深眇而已矣!且夫二子者,又何足以称扬哉!是其于辩也,将妄凿垣
墙而殖蓬蒿也,简发而栉,数米而炊,窃窃乎又何足以济世哉!举贤
则民相轧,任知则民相盗。之数物者,不足以厚民。民之于利甚勤,
子有杀父,臣有杀君;正昼为盗,日中穴阫。吾语女:大乱之本,必
生于尧、舜之间,其末存乎千世之后。千世之后,其必有人与人相食 者也。''

南荣趎蹴然正坐曰:``若趎之年者已长矣,将恶乎托业以及此言邪
?''庚桑子曰:``全汝形,抱汝生,无使汝思虑营营。若此三年,则
可以及此言矣!''南荣趎曰:``目之与形,吾不知其异也,而盲者不
能自见;耳之与形,吾不知其异也,而聋者不能自闻;心之与形,吾
不知其异也,而狂者不能自得。形之与形亦辟矣,而物或间之邪?欲
相求而不能相得。今谓趎曰:`全汝形,抱汝生,无使汝思虑营营。'
趎勉闻道达耳矣!''庚桑子曰:``辞尽矣,奔蜂不能化藿蠋,越鸡
不能伏鹄卵,鲁鸡固能矣!鸡之与鸡,其德非不同也。有能与不能者
,其才固有巨小也。今吾才小,小足以化子。子胡不南见老子!''南
荣囗赢粮,七日七夜至老子之所。老子曰:``子自楚之所来乎?''南
荣囗曰:``唯。''老子曰:``子何与人偕来之众也?''南荣囗惧然顾
其后。老子曰:``子不知吾所谓乎?''南荣囗俯而惭,仰而叹,曰:
``今者吾忘吾答,因失吾问。''老子曰:``何谓也?''南荣囗曰:``
不知乎人谓我朱愚,知乎反愁我躯;不仁则害人,仁则反愁我身;不
义则伤彼,义则反愁我己。我安逃此而可?此三言者,趎之所患也。
愿因楚而问之。''老子曰:``向吾见若眉睫之间,吾因以得汝矣。今
汝又言而信之。若规规然若丧父母,揭竿而求诸海也。女亡人哉!惘
惘乎,汝欲反汝情性而无由入,可怜哉!''南荣囗请入就舍,召其所
好,去其所恶。十日自愁,复见老子
。老子曰:``汝自洒濯,孰哉郁郁乎!然而其中津津乎犹有恶也。夫
外韄者不可繁而捉,将内揵;内韄者不可缪而捉,将外揵;外内韄者,
道德不能持,而况放道而行者乎!''南荣囗曰:``里人有病,里人问
之,病者能言其病,然其病病者犹未病也。若趎之闻大道,譬犹饮药
以加病也。趎愿闻卫生之经而已矣。''老子曰:``卫生之经,能抱一
乎!能勿失乎!能无卜筮而知吉凶乎!能止乎!能已乎!能舍诸人而
求诸己乎!能翛然乎!能侗然乎!能儿子乎!儿子终日嗥而嗌不嗄,
和之至也;终日握而手不掜,共其德也;终日视而目不瞬,偏不在外
也。行不知所之,居不知所为,与物委蛇而同其波。是卫生之经已。''
南荣囗曰:``然则是至人之德已乎?''曰:``非也。是乃所谓冰解冻
释者。夫至人者,相与交食乎地而交乐乎天,不以人物利害相撄,不
相与为怪,不相与为谋,不相与为事,翛然而往,侗然而来。是谓卫
生之经已。''曰:``然则是至乎?''曰:``未也。吾固告汝曰:`能
儿子乎!'儿子动不知所为,行不知所之,身若槁木之枝而心若死灰。
若是者,祸亦不至,福亦不来。祸福无有,恶有人灾也!''

宇泰定者,发乎天光。发乎天光者,人见其人,物见其物。人有修
者,乃今有恒。有恒者,人舍之,天助之。人之所舍,谓之天民;天
之所助,谓之天子。

学者,学其所不能学也?行者,行其所不能行也?辩者,辩其所不
能辩也?知止乎其所不能知,至矣!若有不即是者,天钧败之。备物
将以形,藏不虞以生心,敬中以达彼。若是而万恶至者,皆天也,而
非人也,不足以滑成,不可内于灵台。灵台者有持,而不知其所持而
不可持者也。不见其诚己而发,每发而不当;业入而不舍,每更为失
。为不善乎显明之中者,人得而诛之;为不善乎幽间之中者,鬼得而
诛之。明乎人、明乎鬼者,然后能独行。券内者,行乎无名;券外者
,志乎期费。行乎无名者,唯庸有光;志乎期费者,唯贾人也。人见
其跂,犹之魁然。与物穷者,物入焉;与物且者,其身之不能容,焉
能容人!不能容人者无亲,无亲者尽人。兵莫惨于志,镆铘为下;寇
莫大于阴阳,无所逃于天地之间。非阴阳贼之,心则使之也。

道通其分也,其成也毁也。所恶乎分者,其分也以备。所以恶乎备
者?其有以备。故出而不反,见其鬼。出而得,是谓得死。灭而有实
,鬼之一也。以有形者象无形者而定矣!出无本,入无窍,有实而无
乎处,有长而无乎本剽,有所出而无窍者有实。有实而无乎处者,宇
也;有长而无本剽者,宙也。有乎生,有乎死;有乎出,有乎入。入
出而无见其形,是谓天门。天门者,无有也。万物出乎无有。有不能
以有为有,必出乎无有,而无有一无有。圣人藏乎是。

古之人,其知有所至矣。恶乎至?有以为未始有物者,至矣,尽矣
,弗可以加矣!其次以为有物矣,将以生为丧也,以死为反也,是以
分已。其次曰始无有,既而有生,生俄而死。以无有为首,以生为体
,以死为尻。孰知有无死生之一守者,吾与之为友。是三者虽异,公
族也。昭景也,著戴也;甲氏也,著封也:非一也。

有生黬也,披然曰``移是''。尝言``移是'',非所言也。虽然,不
可知者也。腊者之有膍胲,可散而不可散也;观室者周于寝庙,又适
其偃焉!为是举``移是''。请尝言``移是'':是以生为本,以知
为师,因以乘是非。果有名实,因以己为质,使人以为己节,因以死
偿节。若然者,以用为知,以不用为愚;以彻为名,以穷为辱。``移
是'',今之人也,是蜩与学鸠同于同也。

蹍市人之足,则辞以放骜,兄则以妪,大亲则已矣。故曰:至礼有
不人,至义不物,至知不谋,至仁无亲,至信辟金。彻志之勃,解心
之谬,去德之累,达道之塞。贵富显严名利六者,勃志也;容动色理
气意六者,谬心也;恶欲喜怒哀乐六者,累德也;去就取与知能六者,
塞道也。此四六者不荡胸中则正,正则静,静则明,明则虚,虚则无
为而无不为也。

道者,德之钦也;生者,德之光也;性者,生之质也。性之动谓之
为,为之伪谓之失。知者,接也;知者,谟也。知者之所不知,犹睨
也。动以不得已之谓德,动无非我之谓治,名相反而实相顺也。羿工
乎中微而拙乎使人无己誉;圣人工乎天而拙乎人;夫工乎天而俍乎人
者,唯全人能之。虽虫能虫,虽虫能天。全人恶天,恶人之天,而况
吾天乎人乎!一雀适羿,羿必得之,或也。以天下为之笼,则雀无所
逃。是故汤以胞人笼伊尹,秦
穆公以五羊之皮笼百里奚。是故非以其所好笼之而可得者,无有也。
介者拸画,外非誉也。胥靡登高而不惧,遗死生也。夫复谐不馈而忘人
,忘人,因以为天人矣!故敬之而不喜,侮之而不怒者,唯同乎天和
者为然。出怒不怒,则怒出于不怒矣;出为无为,则为出于无为矣!
欲静则平气,欲神则顺心。有为也欲当,则缘于不得已。不得已之类
,圣人之道。

\hypertarget{header-n2300}{%
\subsubsection{徐无鬼}\label{header-n2300}}

徐无鬼因女商见魏武侯,武侯劳之曰:``先生病矣,苦于山林之劳
,故乃肯见于寡人。''徐无鬼曰:``我则劳于君,君有何劳于我!君
将盈耆欲,长好恶,则性命之情病矣;君将黜耆欲,牵好恶,则耳目
病矣。我将劳君,君有何劳于我!''武侯超然不对。少焉,徐无鬼曰
:``尝语君吾相狗也:下之质,执饱而止,是狸德也;中之质,若视
日;上之质,若亡其一。吾相狗又不若吾相马也。吾相马:直者中绳
,曲者中钩,方者中矩,圆者中规。是国马也,而未若天下马也。天
下马有成材,若卹若失,若丧其一。
若是者,超轶绝尘,不知其所。''武侯大悦而笑。徐无鬼出,女商曰
:``先生独何以说吾君乎?吾所以说吾君者,横说之则以《诗》、《
书》、《礼》、《乐》,从说则以《金板》、《六韬》,奉事而大有
功者不可为数,而吾君未尝启齿。今先生何以说吾君?使吾君说若此
乎?''徐无鬼曰:``吾直告之吾相狗马耳。''女商曰:``若是乎?''
曰:``子不闻夫越之流人乎?去国数日,见其所知而喜;去国旬月,
见所尝见于国中者喜;及期年也,见似人者而喜矣。不亦去人滋久,
思人滋深乎?夫逃虚空者,藜藋柱乎鼪鼬之径,良位其空,闻人足音
跫然而喜矣,又况乎昆弟亲戚之謦欬其侧者乎!久矣夫,莫以真人之
言謦kai4吾君之侧乎!''

徐无鬼见武侯,武侯曰:``先生居山林,食芧栗,厌葱韭,以宾寡
人,久矣夫!今老邪?其欲干酒肉之
味邪?其寡人亦有社稷之福邪?''徐无鬼曰:``无鬼生于贫贱,未尝
敢饮食君之酒肉,将来劳君也。''君曰:``何哉!奚劳寡人?''曰:
``劳君之神与形。''武侯曰:``何谓邪?''徐无鬼曰:``天地之养也
一,登高不可以为长,居下不可以为短。君独为万乘之主,以苦一国
之民,以养耳目鼻口,夫神者不自许也。夫神者,好和而恶奸。夫奸
,病也,故劳之。唯君所病之何也?''武侯曰:``欲见先生久矣!吾
欲爱民而为义偃兵,其可乎?''徐无鬼曰:``不可。爱民,害民之始
也;为义偃兵,造兵之本也。君自此为之,则殆不成。凡成美,恶器
也。君虽为仁义,几且伪哉!形固造形,成固有伐,变固外战。君亦
必无盛鹤列于丽谯之间,无徒骥于锱坛之宫,无藏逆于得,无以巧胜
人,无以谋胜人,无以战胜人。夫杀人之士民,兼人之土地,以养吾
私与吾神者,其战不知孰善?胜之恶乎在?君若勿已矣!修胸中之诚
以应天地之情而勿撄。夫民死已脱矣,君将恶乎用夫偃兵哉!

黄帝将见大隗乎具茨之山,方明为御,昌寓骖乘,张若、谐朋前马,
昆阍、滑稽后车。至于襄城之野,七圣皆迷,无所问涂。适遇牧马童
子,问涂焉,曰:``若知具茨之山乎?''曰:``然。''``若知大隗之
所存乎?''曰:``然。''黄帝曰:``异哉小
童!非徒知具茨之山,又知大隗之所存。请问为天下。''小童曰:``
夫为天下者,亦若此而已矣,又奚事焉!予少而自游于六合之内,予
适有瞀病,有长者教予曰:`若乘日之车而游于襄城之野。'今予病
少痊,予又且复游于六合之外。夫为天下亦若此而已。予又奚事焉!
''黄帝曰:``夫为天下者,则诚非吾子之事,虽然,请问为天下。''
小童辞。黄帝又问。小童曰:``夫为天下者,亦奚以异乎牧马者哉!
亦去其害马者而已矣!''黄帝再拜稽首,称天师而退。

知士无思虑之变则不乐;辩士无谈说之序则不乐;察士无凌谇之事
则不乐:皆囿于物者也。招世之士兴朝;中民之士荣官;筋国之士矜
雅;勇敢之士奋患;兵革之士乐战;枯槁之士宿名;法律之士广治;
礼乐之士敬容;仁义之士贵际。农夫无草莱之事则不比;商贾无市井
之事则不比;庶人有旦暮之业则劝;百工有器械之巧则壮。钱财不积
则贪者忧,权势不尤则夸者悲,势物之徒乐变。遭时有所用,不能无
为也,此皆顺比于岁,不物于易者也。驰其形性,潜之万物,终身不 反,悲夫!

庄子曰:``射者非前期而中谓之善射,天下皆羿也,可乎?''惠子
曰:``可。''庄子曰:``天下非有公是也,而各是其所是,天下皆尧
也,可乎?''惠子曰:``可。''庄子曰:``然则儒墨杨秉四,与夫子
为五,果孰是邪?或者若鲁遽者邪?其弟子曰:`我得夫子之道矣!
吾能冬爨鼎而夏造冰矣!'鲁遽曰:`是直以阳召阳,以阴召阴,非
吾所谓道也。吾示子乎吾道。'于是乎为之调瑟,废一于堂,废一于
室,鼓宫宫动,鼓角角动,音律同矣!夫或改调一弦,于五音无当也
,鼓之,二十五弦皆动,未始异于声而音之君已!且若是者邪!''惠
子曰∶``今乎儒墨杨秉,且方与我以辩,相拂以辞,相镇以声,而未
始吾非也,则奚若矣?''庄子曰:``齐人蹢子于宋者,其命阍也不以
完;其求钘钟也以束缚;其求唐子也而未始出域:有遗类矣!夫楚
人寄而蹢阍者;夜半于无人之时而与舟人斗,未始离于岑而足 以造于怨也。''

庄子送葬,过惠子之墓,顾谓从者曰:``郢人垩慢其鼻端若蝇翼,
使匠人斫之。匠石运斤成风,听而斫之,尽垩而鼻不伤,郢人立不失
容。宋元君闻之,召匠石曰:`尝试为寡人为之。'匠石曰:`臣则
尝能斫之。虽然,臣之质死久矣!'自夫子之死也,吾无以为质矣,
吾无与言之矣!''

管仲有病,桓公问之曰:``仲父之病病矣,可不讳云,至于大病,
则寡人恶乎属国而可?''管仲曰:``公谁欲与?''公曰:``鲍叔牙。
''曰:``不可。其为人洁廉,善士也;其于不己若者不比之;又一闻
人之过,终身不忘。使之治国,上且钩乎君,下且逆乎民。其得罪于
君也将弗久矣!''公曰:``然则孰可?''对曰:``勿已则隰朋可。其
为人也,上忘而下畔,愧不若黄帝,而哀不己若者。以德分人谓之圣
;以财分人谓之贤。以贤临人,未有得人者也;以贤下人,未有不得
人者也。其于国有不闻也,其于家有不见也。勿已则隰朋可。''

吴王浮于江,登乎狙之山,众狙见之,恂然弃而走,逃于深蓁。有
一狙焉,委蛇攫囗(``搔''字以``爪''代``虫''音zao3),见巧
乎王。王射之,敏给搏捷矢。王命相者趋射之,狙执死。王顾谓其友
颜不疑曰:``之狙也,伐其巧、恃其便以敖予,以至此殛也。戒之哉
!嗟乎!无以汝色骄人哉?''颜不疑归而师董梧,以锄其色,去乐辞
显,三年而国人称之。

南伯子綦隐几而坐,仰天而嘘。颜成子入见曰:``夫子,物之尤也
。形固可使若槁骸,心固可使若死灰乎?''曰:``吾尝居山穴之中矣
。当是时也,田禾一睹我而齐国之众三贺之。我必先之,彼故知之;
我必卖之,彼故鬻之。若我而不有之,彼恶得而知之?若我而不卖之
,彼恶得而鬻之?嗟乎!我悲人之自丧者;吾又悲夫悲人者;吾又悲

夫悲人之悲者;其后而日远矣!``

仲尼之楚,楚王觞之。孙叔敖执爵而立。市南宜僚受酒而祭,曰:
``古之人乎!于此言已。''曰:``丘也闻不言之言矣,未之尝言,于
此乎言之:市南宜僚弄丸而两家之难解;孙叔敖甘寝秉羽而郢人投兵
;丘愿有喙三尺。''彼之谓不道之道,此之谓不言之辩。故德总乎道
之所一,而言休乎知之所不知,至矣。道之所一者,德不能同也。知
之所不能知者,辩不能举也。名若儒墨而凶矣。故海不辞东流,大之
至也。圣人并包天地,泽及天下,而不知其谁氏。是故生无爵,死无
谥,实不聚,名不立,此之谓大人。狗不以善吠为良,人不以善言为
贤,而况为大乎!夫为大不足以为大,而况为德乎!夫大备矣,莫若
天地。然奚求焉,而大备矣!知大备者,无求,无失,无弃,不以物
易己也。反己而不穷,循古而不摩,大人之诚!

子綦有八子,陈诸前,召九方歅曰:``为我相吾子,孰为祥。''九
方囗曰:``梱也为祥。''子綦瞿然喜曰:``奚若?''曰:``梱也,将
与国君同食以终其身。''子綦索然出涕曰:``吾子何为以至于是极也?''
九方囗曰:``夫与国君同食,泽及三族,而况父母乎!今夫子闻之而
泣,是御福也。子则祥矣,父则不祥。''子綦曰:``歅,汝何足以识
之。而梱祥邪?尽于酒肉,入于鼻口矣,而何足以知其所自来!吾未
尝为牧而牂生于奥,未尝好田而鹑生于宎,若勿怪,何邪?吾所与吾
子游者,游于天地,吾与之邀乐于天,吾与之邀食于地。吾不与之为
事,不与之为谋,不与之为怪。吾与之乘天地之诚而不以物与之相撄,
吾与之一委蛇而不与之为事所宜。今也然有世俗之偿焉?凡有怪征者
必有怪行。殆乎!非我与吾子之罪,几天与之也!吾是以泣也。''无
几何而使梱之于燕,盗得之于道,全而鬻之则难,不若刖之则易。于
是乎刖而鬻之于齐,适当渠公之街,然身食肉而终。

啮缺遇许由曰:``子将奚之?''曰:``将逃尧。''曰:``奚谓邪?
''曰:``夫尧畜畜然仁,吾恐其为天下笑。后世其人与人相食与!夫
民不难聚也,爱之则亲,利之则至,誉之则劝,致其所恶则散。爱利
出乎仁义,捐仁义者寡,利仁义者众。夫仁义之行,唯且无诚,且假
乎禽贪者器。是以一人之断制天下,譬之犹一覕也。夫尧知贤人之利
天下也,而不知其贼天下也。夫唯外乎贤者知之矣。''

有暖姝者,有濡需者,有卷娄者。所谓暖姝者,学一先生之言,则
暖暖姝姝而私自说也,自以为足矣,而未知未始有物也。是以谓暖姝
者也。濡需者,豕虱是也,择疏鬣长毛,自以为广宫大囿。奎蹄曲隈
,乳间股脚,自以为安室利处。不知屠者之一旦鼓臂布草操烟火,而
己与豕俱焦也。此以域进,此以域退,此其所谓濡需者也。卷娄者,
舜也。羊肉不慕蚁,蚁慕羊肉,羊肉羶也。舜有羶行,百姓悦之,故
三徙成都,至邓之虚而十有万家。尧闻舜之贤,举之童土之地,曰:
``冀得其来之泽。''舜举乎童土之地,年齿长矣,聪明衰矣,而不得
休归,所谓卷
娄者也。是以神人恶众至,众至则不比,不比则不利也。故无所甚亲
,无所甚疏,抱德炀和,以顺天下,此谓真人。于蚁弃知,于鱼得计
,于羊弃意。以目视目,以耳听耳,以心复心。若然者,其平也绳,
其变也循。古之真人!以天待之,不以人入天,古之真人!

得之也生,失之也死;得之也死,失之也生:药也。其实堇也,桔
梗也,鸡雍也,豕零也,是时为帝者也,何可胜言!

句践也以甲楯三千栖于会稽,唯种也能知亡
之所以存,唯种也不知其身之所以愁。故曰:鸱目有所适,鹤胫有所
节,解之也悲。故曰:风之过,河也有损焉;日之过,河也有损焉;
请只风与日相与守河,而河以为未始其撄也,恃源而往者也。故水之
守土也审,影之守人也审,物之守物也审。故目之于明也殆,耳之于
聪也殆,心之于殉也殆,凡能其于府也殆,殆之成也不给改。祸之长
也兹萃,其反也缘功,其果也待久。而人以为己宝,不亦悲乎!故有
亡国戮民无已,不知问是也。故足之于地也践,虽践,恃其所不蹍而
后善博也;人之知也少,虽少,恃其所不知而后知天之所谓也。知大
一,知大阴,知大目,知大均,知大方,知大信,知大定,至矣!大
一通之,大阴解之,大目视之,大均缘之,大方体之,大信稽之,大
定持之。尽有天,循有照,冥有枢,始有彼。则其解之也似不解之者,
其知之也似不知之也,不知而后知之。其问之也,不可以有崖,而不
可以无崖。颉滑有实,古今不代,而不可以亏,则可不谓有大扬搉乎!
阖不亦问是已,奚惑然为!以不惑解惑,复于不惑,是尚大不惑。

\hypertarget{header-n2320}{%
\subsubsection{则阳}\label{header-n2320}}

则阳游于楚,夷节言之于王,王未之见。夷节归。彭阳见王果曰:
``夫子何不谭我于王?''王果曰:``我不若公阅休。''彭阳曰:``公
阅休奚为者邪?''曰:``冬则戳鳖于江,夏则休乎山樊。有过而问者
,曰:`此予宅也。'夫夷节已不能,而况我乎!吾又不若夷节。夫
夷节之为人也,无德而有知,不自许,以之神其交,固颠冥乎富贵之
地。非相助以德,相助消也。夫冻者假衣于春,暍者反冬乎冷风。夫
楚王之为人也,形尊而严。其于罪也
,无赦如虎。非夫佞人正德,其孰能桡焉。故圣人其穷也,使家人忘
其贫;其达也,使王公忘爵禄而化卑;其于物也,与之为娱矣;其于
人也,乐物之通而保己焉。故或不言而饮人以和,与人并立而使人化
,父子之宜。彼其乎归居,而一闲其所施。其于人心者,若是其远也
。故曰`待公阅休'。''

圣人达绸缪,周尽一体矣,而不知其然,性也。复命摇作而以天为
师,人则从而命之也。忧乎知,而所行恒无几时,其有止也,若之何
!生而美者,人与之鉴,不告则不知其美于人也。若知之,若不知之
,若闻之,若不闻之,其可喜也终无已,人之好之亦无已,性也。圣
人之爱人也,人与之名,不告则不知其爱人也。若知之,若不知之,
若闻之,若不闻之,其爱人也终无已,人之安之亦无已,性也。旧国
旧都,望之畅然。虽使丘陵草木之缗入之者十九,犹之畅然,况见见
闻闻者也,以十仞之台县众间者也。冉相氏得其环中以随成,与物无
终无始,无几无时。日与物化者,一不化者也。阖尝舍之!夫师天而
不得师天,与物皆殉。其以为事也,若之何!夫圣人未始有天,未始
有人,未始有始,未始有物,与世偕行而不替,所行之备而不洫,其
合之也,若之何!

汤得其司御,门尹登恒为之傅之。从师而不囿,得其随成。为之司
其名之名嬴法得其两见。仲尼之尽虑,为之傅之。容成氏曰:``除日
无岁,无内无外。''

魏莹与田侯牟约,田侯牟背之,魏莹怒,将使人剌之。犀首公孙衍
闻而耻之,曰:``君为万乘之君也,而以匹夫从仇。衍请受甲二十万
,为君攻之,虏其人民,系其牛马,使其君内热发于背,然后拔其国
。忌也出走,然后抶其背,折其脊。''季子闻而耻之,曰:``筑十仞
之城,城者既十仞矣,则又坏之,
此胥靡之所苦也。今兵不起七年矣,此王之基也。衍,乱人也,不可
听也。''华子闻而丑之,曰:``善言伐齐者,乱人也;善言勿伐者,
亦乱人也;谓`伐之与不伐乱人也'者,又乱人也。''君曰:``然则
若何?''曰:``君求其道而已矣。''惠之闻之,而见戴晋人。戴晋人
曰:``有所谓蜗者,君知之乎?''曰:``然。''``有国于蜗之左角者
,曰触氏;有国于蜗之右角者,曰蛮氏。时相与争地而战,伏尸数万
,逐北旬有五日而后反。''君曰:``噫!其虚言与?''曰:``臣请为
君实之。君以意在四方上下有穷乎?''君曰:``无穷。''曰:``知游
心于无穷,而反在通达之国,若存若亡乎?''君曰:``然。''曰:``
通达之中有魏,于魏中有梁,于梁中有王,王与蛮氏有辩乎?''君曰
:``无辩。''客出而君惝然若有亡也。客出,惠子见。君曰:``客,
大人也,圣人不足以当之。''惠子曰:``夫吹管也,犹有嗃也;吹剑
首者,吷而已矣。尧、舜,人之所誉也。道尧、舜于戴晋人之前,譬犹
一吷也。''

孔子之楚,舍于蚁丘之浆。其邻有夫妻臣妾登极者,子路曰:``是
稯稯何为者邪?''仲尼曰:``是圣人仆也。是自埋于民,自藏于畔。
其声销,其志无穷,其口虽言,其心未尝言。方且与世违,而心不屑
与之俱。是陆沉者也,是其市南宜僚邪?''子路请往召之。孔子曰:
``已矣!彼知丘之著于己也,知丘之适楚也,以丘为必使楚王之召己
也。彼且以丘为佞人也。夫若然者,其于佞人也,羞闻其言,而况亲
见其身乎!而何以为存!''子路往视之,其室虚矣。

长梧封人问子牢曰:``君为政焉勿卤莽,治民焉勿灭裂。昔予为禾
,耕而卤莽之,则其实亦卤莽而报予;芸而灭裂之,其实亦灭裂而报
予。予来年变齐,深其耕而熟耰之,其禾蘩以滋,予终年厌飧。''庄
子闻之曰:``今人之治其形,理其心,多有似封人之所谓:遁其天,
离其性,灭其情,亡其神,以众为。故卤莽其性者,欲恶之孽为性,
萑苇蒹葭始萌,以扶吾形,寻擢吾性。并溃漏发,不择所出,漂疽疥
癕,内热溲膏是也。''

柏矩学于老聃,曰:``请之天下游。''老聃曰:``已矣!天下犹是
也。''又请之,老聃曰:``汝将何始?''曰:``始于齐。''至齐,见
辜人焉,推而强之,解朝服而幕之,号天而哭之,曰:``子乎!子乎
!天下有大灾,子独先离之。曰`莫为盗,莫为杀人'。荣辱立然后
睹所病,货财聚然后睹所争。今立人之所病,聚人之所争,穷困人之
身,使无休时。欲无至此得乎?古之君人者,以得为在民,以失为在
己;以正为在民,以枉为在己。故一形有失其形者,退而自责。今则
不然,匿为物而愚不识,大为难而罪不敢,重为任而罚不胜,远其涂
而诛不至。民知力竭,则以伪继之。日出多伪,士民安取不伪。夫力
不足则伪,知不足则欺,财不足则盗。盗窃之行,于谁责而可乎?''

蘧伯玉行年六十而六十化,未尝不始于是之,而卒诎之以非也。未
知今之所谓是之非五十九非也。万物有乎生而莫见其根,有乎出而莫
见其门。人皆尊其知之所知,而莫知恃其知之所不知而后知,可不谓
大疑乎!已乎!已乎!且无所逃。此所谓然与然乎!

仲尼问于大史大弢、伯常骞、狶韦曰:``夫卫灵公饮酒湛乐,不
听国家之政;田猎毕弋,不应诸侯之际:其所以为灵公者何邪?''大
弢曰:``是因是也。''伯常骞曰:``夫灵公有妻三人,同滥而
浴。史鳅奉御而进所,搏币而扶翼。其慢若彼之甚也,见贤人若此其
肃也,是其所以为灵公也。''狶韦曰:``夫灵公也,死,卜葬于故墓,
不吉;卜葬于沙丘而吉。掘之数仞,得石槨焉,洗而视之,有铭焉,
曰:`不冯其子,灵公夺而里之。'夫灵公之为灵也久矣!之二人何 足以识之。''

少知问于大公调曰:``何谓丘里之言?''大公调曰:``丘里者,合
十姓百名而为风俗也,合异以为同,散同以为异。今指马之百体而不
得马,而马系于前者,立其百体而谓之马也。是故丘山积卑而为高,
江河合水而为大,大人合并而为公。是以自外入者,有主而不执;由
中出者,有正而不距。四时殊气,天不赐,故岁成;五官殊职,君不
私,故国治;文武殊材,大人不赐,故德备;万物殊理,道不私,故
无名。无名故无为,无为而无不为。时有终始,世有变化,祸福淳淳
,至有所拂者而有所宜,自殉殊面;有所正者有所差,比于大泽,百
材皆度;观于大山,木石同坛。此之谓丘里之言。''少知曰:``然则
谓之道足乎?''大公调曰:``不然,今计物之数,不止于万,而期曰
万物者,以数之多者号而读之也。是故天地者,形之大者也;阴阳者
,气之大者也;道者为之公。因其大以号而读之则可也,已有之矣,
乃将得比哉!则若以斯辩,譬犹狗马,其不及远矣。''少知曰:``四
方之内,六合之里,万物之所生恶起?''大公调曰:``阴阳相照相盖
相治,四时相代相生相杀。欲恶去就,于是桥起。雌雄片合,于是庸
有。安危相易,祸福相生,缓急相摩,聚散以成。此名实之可纪,精
之可志也。随序之相理,桥运之相使,穷则反,终则始,此物之所有
。言之所尽,知之所至,极物而已。睹道之人,不随其所废,不原其
所起,此议之所止。''少知曰:``季真之莫为,接子之或使。二家之
议,孰正于其情,孰偏于其理?''大公调曰:``鸡鸣狗吠,是人之所
知。虽有大知,不能以言读其所自化,又不能以意其所将为。斯而析
之,精至于无伦,大至于不可围。或之使,莫之为,未免于物而终以
为过。或使则实,莫为则虚。有名有实,是物之居;无名无实,在物
之虚。可言可意,言而愈疏。未生不可忌,已死不可阻。死生非远也
,理不可睹。或之使,莫之为,疑之所假。吾观之本,其往无穷;吾
求之末,其来无止。无穷无止,言之无也,与物同理。或使莫为,言
之本也。与物终始。道不可有,有不可无。道之为名,所假而行。或
使莫为,在物一曲,夫胡为于大方!言而足,则终日言而尽道;言而
不足,则终日言而尽物。道,物之极,言默不足以载。非言非默,议 有所极。''

\hypertarget{header-n2334}{%
\subsubsection{外物}\label{header-n2334}}

外物不可必,故龙逢诛,比干戮,箕子狂,恶来死,桀、纣亡。人
主莫不欲其臣之忠,而忠未必信,故伍员流于江,苌弘死于蜀,藏其
血,三年而化为碧。人亲莫不欲其子之孝,而孝未必爱,故孝己忧而
曾参悲。木与木相摩则然,金与火相守则流,阴阳错行,则天地大骇
,于是乎有雷有霆,水中有火,乃焚大槐。有甚忧两陷而无所逃。螴
蜳不得成,心若县于天地之间,慰暋沈屯,利害相摩,生火甚多,众
人焚和,月固不胜火,于是乎有僓然而道尽。

庄周家贫,故往贷粟于监河侯。监河侯曰:``诺。我将得邑金,将
贷子三百金,可乎?''庄周忿然作色曰:``周昨来,有中道而呼者,
周顾视车辙,中有鲋鱼焉。周问之曰:`鲋鱼来,子何为者耶?'对
曰:`我,东海之波臣也。君岂有斗升之水而活我哉!'周曰:`诺
,我且南游吴越之王,激西江之水而迎子,可乎?'鲋鱼忿然作色曰
:`吾失我常与,我无所处。我得斗升之水然活耳。君乃言此,曾不
如早索我于枯鱼之肆。'''

任公子为大钩巨缁,五十犗以为饵,蹲乎会稽,投竿东海,旦旦而
钓,期年不得鱼。已而大鱼食之,牵巨钩,陷没而下骛,扬而奋鬐,白
波若山,海水震荡,声侔鬼神,惮赫千里。任公子得若鱼,离而腊之
,自制河以东,苍梧已北,莫不厌若鱼者。已而后世辁才讽说之徒,
皆惊而相告也。夫揭竿累,趣灌渎,守鲵鲋,其于得大鱼难矣!饰小
说以干县令,其于大达亦远矣。是以未尝闻任氏之风俗,其不可与经
于世亦远矣!

儒以《诗》、《礼》发冢,大儒胪传曰:``东方作矣,事之何若?
''小儒曰:``未解裙襦,口中有珠。''``《诗》固有之曰:`青青之
麦,生于陵陂。生不布施,死何含珠为?'接其鬓,压其囗(左``岁
''右``页''音hui4),儒以金椎控其颐,徐别其颊,无伤口中珠 。''

老莱子之弟子出薪,遇仲尼,反以告,曰:``有人于彼,修上而趋
下,末偻而后耳,视若营四海,不知其谁氏之子。''老莱子曰:``是
丘也,召而来。''仲尼至。曰:``丘,去汝躬矜与汝容知,斯为君子
矣。''仲尼揖而退,蹙然改容而问曰:``业可得进乎?''老莱子曰:
``夫不忍一世之伤,而骜万世之患。抑固窭邪?亡其略弗及邪?惠以
欢为,骜终身之丑,中民之行易进焉耳!相引以名,相结以隐。与其
誉尧而非桀,不如两忘而闭其所誉。反无非伤也,动无非邪也,圣人
踌躇以兴事,以每成功。奈何哉,其载焉终矜尔!''

宋元君夜半而梦人被发窥阿门,曰:``予自宰路之渊,予为清江使
河伯之所,渔者余且得予。''元君觉,使人占之,曰:``此神龟也。
''君曰:``渔者有余且乎?''左右曰:``有。''君曰:``令余且会朝
。''明日,余且朝。君曰:``渔何得?''对曰:``且之网得白龟焉,
箕圆五尺。''君曰:``献若之龟。''龟至,君再欲杀之,再欲活之。
心疑,卜之。曰:``杀龟以卜吉。''乃刳龟,七十二钻而无遗生筴。
仲尼曰:``神龟能见梦于元君,而不能避余且之
网;知能七十二钻而无遗囗,不能避刳肠之患。如是则知有所困,神
有所不及也。虽有至知,万人谋之。鱼不畏网而畏鹈鹕。去小知而大
知明,去善而自善矣。婴儿生,无硕师而能言,与能言者处也。''

惠子谓庄子曰:``子言无用。''庄子曰:``知无用而始可与言用矣
。夫地非不广且大也,人之所用容足耳,然则厕足而垫之致黄泉,人
尚有用乎?''惠子曰:``无用。''庄子曰:``然则无用之为用也亦明 矣。''

庄子曰:``人有能游,且得不游乎!人而不能游,且得游乎!夫流
遁之志,决绝之行,噫,其非至知厚德之任与!覆坠而不反,火驰而
不顾。虽相与为君臣,时也。易世而无以相贱。故曰:至人不留行焉
。夫尊古而卑今,学者之流也。且以狶韦氏之流观今之世,夫孰能不
波!唯至人乃能游于世而不僻,顺人而不失己。彼教不学,承意不彼。
目彻为明,耳彻为聪,鼻彻为颤,口彻为甘,心彻为知,知彻为德。
凡道不欲壅,壅则哽,哽而不止则跈,跈则众害生。物之有知者恃息。
其不殷,非天之罪。天之穿之,日夜无降,人则顾塞其窦。胞有重阆,
心有天游。室无空虚,则妇姑勃谿;心无天游,则六凿相攘。大林丘
山之善于人也,亦神者不胜。德溢乎名,名溢乎暴,谋稽乎誸,知出
乎争,柴生乎守,官事果乎众宜。春雨日时,草木怒生,铫鎒于是乎
始修,草木之倒植者过半而不知其然。静默可以补病,眦媙可以休老,
宁可以止遽。虽然,若是劳者之务
也,非佚者之所未尝过而问焉;圣人之所以骇天下,神人未尝过而问
焉;贤人所以骇世,圣人未尝过而问焉;君子所以骇国,贤人未尝过
而问焉;小人所以合时,君子未尝过而问焉。

演门有亲死者,以善毁爵为官师,其党人毁而死者半。尧与许由天
下,许由逃之;汤与务光,务光怒之;纪他闻之,帅弟子而蹲于窾水,
诸侯吊之。三年,申徒狄因以踣河。

荃者所以在鱼,得鱼而忘荃;蹄者所以在兔,得兔而忘蹄;言者所以在意,得意而忘言。吾安得夫忘言之人而与之言哉!''

\hypertarget{header-n2348}{%
\subsubsection{寓言}\label{header-n2348}}

寓言十九,重言十七,卮言日出,和以天倪。寓言十九,藉外论之
。亲父不为其子媒。亲父誉之,不若非其父者也。非吾罪也,人之罪
也。与己同则应,不与己同则反。同于己为是之,异于己为非之。重
言十七,所以己言也。是为耆艾,年先矣,而无经纬本末以期年耆者
,是非先也。人而无以先人,无人道也。人而无人道,是之谓陈人。
卮言日出,和以天倪,因以曼衍,所以穷年。不言则齐,齐与言不齐
,言与齐不齐也。故曰:``言无言。''言无言:终身言,未尝言;终
身不言,未尝不言。有自也而可,有自也而不可;有自也而然,有自
也而不然。恶乎然?然于然;恶乎不然?不然于不然。恶乎可?可于
可;恶乎不可?不可于不可。物固有所然,物固有所可。无物不然,
无物不可。非卮言日出,和以天倪,孰得其久!万物皆种也,以不同
形相禅,始卒若环,莫得其伦,是谓天均。天均者,天倪也。

庄子谓惠子曰:``孔子行年六十而六十化。始时所是,卒而非之。
未知今之所谓是之非五十九非也。''惠子曰:``孔子勤志服知也。''
庄子曰:``孔子谢之矣,而其未之尝言也。孔子云:夫受才乎大本,
复灵以生。鸣而当律,言而当法。利义陈乎前,而好恶是非直服人之
口而已矣。使人乃以心服而不敢蘁,立定天下之定。已乎,已乎!吾
且不得及彼乎!''

曾子再仕而心再化,曰:``吾及亲仕,三釜而心乐;后仕,三千锺
而不洎,吾心悲。''弟子问于仲尼曰:``若参者,可谓无所县其罪乎
?''曰:``既已县矣!夫无所县者,可以有哀乎?彼视三釜、三千锺
,如观雀蚊虻相过乎前也。''

颜成子游谓东郭子綦曰:``自吾闻子之言,一年而野,二年而从,
三年而通,四年而物,五年而来,六年而鬼入,七年而天成,八年而
不知死、不知生,九年而大妙。生有为,死也。劝公以其私,死也有
自也,而生阳也,无自也。而果然乎?恶乎其所适,恶乎其所不适?
天有历数,地有人据,吾恶乎求之?莫知其所终,若之何其无命也?
莫知其所始,若之何其有命也?有以相应也,若之何其无鬼邪?无以
相应也,若之何其有鬼邪?''

众罔两问于景曰:``若向也俯而今也仰,向也括撮而今也被发;向
也坐而今也起;向也行而今也止:何也?''景曰:``搜搜也,奚稍问
也!予有而不知其所以。予,蜩甲也,蛇蜕也,似之而非也。火与日
,吾屯也;阴与夜,吾代也。彼,吾所以有待邪,而况乎以无有待者
乎!彼来则我与之来,彼往则我与之往,彼强阳则我与之强阳。强阳
者,又何以有问乎!''

阳子居南之沛,老聃西游于秦。邀于郊,至于梁而遇老子。老子中
道仰天而叹曰:``始以汝为可教,今不可也。''阳子居不答。至舍,
进盥漱巾栉,脱屦户外,膝行而前,曰:``向者弟子欲请夫子,夫子
行不闲,是以不敢;今闲矣,请问其故。''老子曰:``而睢睢盱盱,
而谁与居!大白若辱,盛德若不足。''阳子居蹴然变容曰:``敬闻命
矣!''其往也,舍者迎将其家,公执席,妻执巾栉,舍者避席,炀者
避灶。其反也,舍者与之争席矣!

\hypertarget{header-n2358}{%
\subsubsection{让王}\label{header-n2358}}

尧以天下让许由,许由不受。又让于子州支父,子州之父曰:``以
我为天子,犹之可也。虽然,我适有幽忧之病,方且治之,未暇治天
下也。''夫天下至重也,而不以害其生,又况他物乎!唯无以天下为
者可以托天下也。舜让天下于子州之伯,子州之伯曰:``予适有幽忧
之病,方且治之,未暇治天下也。''故天下大器也,而不以易生。此
有道者之所以异乎俗者也。舜以天下让善卷,善卷曰:``余立于宇宙
之中,冬日衣皮毛,夏日衣葛囗(左``纟''右``希'')。春耕种,形
足以劳动;秋收敛,身足以休食。日出而作,日入而息,逍遥于天地
之间,而心意自得。吾何以天下为哉!悲夫,子之不知余也。''遂不
受。于是去而入深山,莫知其处。舜以天下让其友石户之农。石户之
农曰:``囗囗(左``扌''右``卷'')乎,后之为人,葆力之士也。''
以舜之德为未至也。于是夫负妻戴,携子以入于海,终身不反也。

大王囗(``檀''字去``木''音dan4)父居豳,狄人攻之。事之
以皮帛而不受,事之以犬马而不受,事之以珠玉而不受。狄人之所求
者土地也。大王囗父曰:``与人之兄居而杀其弟,与人之父居而杀其
子,吾不忍也。子皆勉居矣!为吾臣与为狄人臣奚以异。且吾闻之:
不以所用养害所养。''因杖囗(上``竹''下``夹'')而去之。民相连
而从之。遂成国于岐山之下。夫大王囗父可谓能尊生矣。能尊生者,
虽贵富不以养伤身,虽贫贱不以利累形。今世之人居高官尊爵者,皆
重失之。见利轻亡其身,岂不惑哉!

越人三世弑其君,王子搜患之,逃乎丹穴,而越国无君。求王子搜
不得,从之丹穴。王子搜不肯出,越人熏之以艾。乘以王舆。王子搜
援绥登车,仰天而呼曰:``君乎,君乎,独不可以舍我乎!''王子搜
非恶为君也,恶为君之患也。若王子搜者,可谓不以国伤生矣!此固
越人之所欲得为君也。

韩魏相与争侵地,子华子见昭僖侯,昭僖侯有忧色。子华子曰:``
今使天下书铭于君之前,书之言曰:`左手攫之则右手废,右手攫之
则左手废。然而攫之者必有天下。'君能攫之乎?''昭僖侯曰:``寡
人不攫也。''子华子曰:``甚善!自是观之,两臂重于天下也。身亦
重于两臂。韩之轻于天下亦远矣!今之所争者,其轻于韩又远。君固
愁身伤生以忧戚不得也。''僖侯曰:``善哉!教寡人者众矣,未尝得
闻此言也。''子华子可谓知轻重矣!

鲁君闻颜阖得道之人也,使人以币先焉。颜阖守陋闾,苴布之衣,
而自饭牛。鲁君之使者至,颜阖自对之。使者曰:``此颜阖之家与?
''颜阖对曰:``此阖之家也。''使者致币。颜阖对曰:``恐听谬而遗
使者罪,不若审之。''使者还,反审之,复来求之,则不得已!故若
颜阖者,真恶富贵也。

故曰:道之真以治身,其绪余以为国家,其土苴以治天下。由此观
之,帝王之功,圣人之余事也,非所以完身养生也。今世俗之君子,
多危身弃生以殉物,岂不悲哉!凡圣人之动作也,必察其所以之与其
所以为。今且有人于此,以随侯之珠,弹千仞之雀,世必笑之。是何
也?则其所用者重而所要者轻也。夫生者岂特随侯之重哉!

子列子穷,容貌有饥色。客有言之于郑子阳者,曰:``列御寇,盖
有道之士也,居君之国而穷,君无乃为不好士乎?''郑子阳即令官遗
之粟。子列子见使者,再拜而辞。使者去,子列子入,其妻望之而拊
心曰:``妾闻为有道者之妻子,皆得佚乐。今有饥色,君过而遗先生
食,先生不受,岂不命邪?''子列子笑,谓之曰∶``君非自知我也,
以人之言而遗我粟;至其罪我也,又且以人之言,此吾所以不受也。
''其卒,民果作难而杀子阳。

楚昭王失国,屠羊说走而从于昭王。昭王反国,将赏从者。及屠羊
说。屠羊说曰:``大王失国,说失屠羊。大王反国,说亦反屠羊。臣
之爵禄已复矣,又何赏之有。''王曰:``强之。''屠羊说曰:``大王
失国,非臣之罪,故不敢伏其诛;大王反国,非臣之功,故不敢当其
赏。''王曰:``见之。''屠羊说曰:``楚国之法,必有重赏大功而后
得见。今臣之知不足以存国,而勇不足以死寇。吴军入郢,说畏难而
避寇,非故随大王也。今大王欲废法毁约而见说,此非臣之所以闻于
天下也。''王谓司马子綦曰:``屠羊说居处卑贱而陈义甚高,子綦为
我延之以三旌之位。''屠羊说曰:``夫三旌之位,吾知其贵于屠羊之
肆也;万锺之禄,吾知其富于屠羊之利也。然岂可以贪爵禄而使吾君
有妄施之名乎?说不敢当,愿复反吾屠羊之肆。''遂不受也。

原宪居鲁,环堵之室,茨以生草,蓬户不完,桑以为枢而瓮牖,二
室,褐以为塞,上漏下湿,匡坐而弦歌。子贡乘大马,中绀而表素,
轩车不容巷,往见原宪。原宪华冠囗(左``纟''右``徙''音xi1)
履,杖藜而应门。子贡曰:``嘻!先生何病?''原宪应之曰:``宪闻
之,无财谓之贫,学而不能行谓之病。今宪贫也,非病也。''子贡逡
巡而有愧色。原宪笑曰:``夫希世而行,比周而友,学以为人,教以
为己,仁义之慝,舆马之饰,宪不忍为也。''

曾子居卫,囗(``温''字以``纟''代``氵''音yun4)袍无表,
颜色肿哙,手足胼胝,三日不举火,十年不制衣。正冠而缨绝,捉襟
而肘见,纳屦而踵决。曳纵而歌《商颂》,声满天地,若出金石。天
子不得臣,诸侯不得友。故养志者忘形,养形者忘利,致道者忘心矣 。

孔子谓颜回曰:``回,来!家贫居卑,胡不仕乎?''颜回对曰:``
不愿仕。回有郭外之田五十亩,足以给囗(左``饣''右``干''音zh
an1)粥;郭内之田十亩,足以为丝麻;鼓琴足以自娱;所学夫子
之道者足以自乐也。回不愿仕。''孔子愀然变容,曰:``善哉,回之
意!丘闻之:`知足者,不以利自累也;审自得者,失之而不惧;行
修于内者,无位而不怍。'丘诵之久矣,今于回而后见之,是丘之得 也。''

中山公子牟谓瞻子曰:``身在江海之上,心居乎魏阙之下,奈何?
''瞻子曰:``重生。重生则利轻。''中山公子牟曰:``虽知之,未能
自胜也。''瞻子曰:``不能自胜则从,神无恶乎!不能自胜而强不从
者,此之谓重伤。重伤之人,无寿类矣!''魏牟,万乘之公子也,其
隐岩穴也,难为于布衣之士,虽未至乎道,可谓有其意矣!

孔子穷于陈蔡之间,七日不火食,藜羹不糁,颜色甚惫,而弦歌于
室。颜回择菜,子路、子贡相与言曰:``夫子再逐于鲁,削迹于卫,
伐树于宋,穷于商周,围于陈蔡。杀夫子者无罪,藉夫子者无禁。弦
歌鼓琴,未尝绝音,君子之无耻也若此乎?''颜回无以应,入告孔子
。孔子推琴,喟然而叹曰:``由与赐,细人也。召而来,吾语之。''
子路、子贡入。子路曰:``如此者,可谓穷矣!''孔子曰:``是何言
也!君子通于道之谓通,穷于道之谓穷。今丘抱仁义之道以遭乱世之
患,其何穷之为?故内省而不穷于道,临难而不失其德。天寒既至,
霜雪既降,吾是以知松柏之茂也。陈蔡之隘,于丘其幸乎。''孔子削
然反琴而弦歌,子路囗(左``扌''右``乞''音xi4)然执干而舞。
子贡曰:``吾不知天之高也,地之下也。''古之得道者,穷亦乐,通
亦乐,所乐非穷通也。道德于此,则穷通为寒暑风雨之序矣。故许由
娱于颖阳,而共伯得乎丘首。

舜以天下让其友北人无择,北人无择曰:``异哉,后之为人也,居
于畎亩之中,而游尧之门。不若是而已,又欲以其辱行漫我。吾羞见
之。''因自投清泠之渊。

汤将伐桀,因卞随而谋,卞随曰:``非吾事也。''汤曰:``孰可?
''曰∶``吾不知也。''汤又因瞀光而谋,瞀光曰:``非吾事也。''汤
曰∶``孰可?''曰:``吾不知也。''汤曰:``伊尹何如?''曰:``强
力忍垢,吾不知其他也。''汤遂与伊尹谋伐桀,克之。以让卞随,卞
随辞曰:``后之伐桀也谋乎我,必以我为贼也;胜桀而让我,必以我
为贪也。吾生乎乱世,而无道之人再来漫我以其辱行,吾不忍数闻也
!''乃自投囗(左``木''右``周''音zhou1)水而死。汤又让瞀
光,曰:``知者谋之,武者遂之,仁者居之,古之道也。吾子胡不立
乎?''瞀光辞曰:``废上,非义也;杀民,非仁也;人犯其难,我享
其利,非廉也。吾闻之曰:`非其义者,不受其禄;无道之世,不践
其土。'况尊我乎!吾不忍久见也。''乃负石而自沈于庐水。

昔周之兴,有士二人处于孤竹,曰伯夷、叔齐。二人相谓曰:``吾
闻西方有人,似有道者,试往观焉。''至于岐阳,武王闻之,使叔旦
往见之。与盟曰:``加富二等,就官一列。''血牲而埋之。二人相视
而笑,曰:``嘻,异哉!此非吾所谓道也。昔者神农之有天下也,时
祀尽敬而不祈喜;其于人也,忠信尽治而无求焉。乐与政为政,乐与
治为治。不以人之坏自成也,不以人之卑自高也,不以遭时自利也。
今周见殷之乱而遽为政,上谋而下行货,阻兵而保威,割牲而盟以为
信,扬行以说众,杀伐以要利。是推乱以易暴也。吾闻古之士,遭治
世不避其任,遇乱世不为苟存。今天下囗(外``门''内``音''),周
德衰,其并乎周以涂吾身也,不如避之,以洁吾行。''二子北至于首
阳之山,遂饿而死焉。若伯夷、叔齐者,其于富贵也,苟可得已,则
必不赖高节戾行,独乐其志,不事于世。此二士之节也。

\hypertarget{header-n2378}{%
\subsubsection{盗跖}\label{header-n2378}}

孔子与柳下季为友,柳下季之弟名曰盗跖。盗跖从卒九千人,横行
天下,侵暴诸侯。穴室枢户,驱人牛马,取人妇女。贪得忘亲,不顾
父母兄弟,不祭先祖。所过之邑,大国守城,小国入保,万民苦之。
孔子谓柳下季曰:``夫为人父者,必能诏其子;为人兄者,必能教其
弟。若父不能诏其子,兄不能教其弟,则无贵父子兄弟之亲矣。今先
生,世之才士也,弟为盗跖,为天下害,而弗能教也,丘窃为先生羞
之。丘请为先生往说之。''柳下季曰:``先生言为人父者必能诏其子
,为人兄者必能教其弟,若子不听父之诏,弟不受兄之教,虽今先生
之辩,将奈之何哉?且跖之为人也,心如涌泉,意如飘风,强足以距
敌,辩足以饰非。顺其心则喜,逆其心则怒,易辱人以言。先生必无
往。''孔子不听,颜回为驭,子贡为右,往见盗跖。

盗跖乃方休卒徒大山之阳,脍人肝而囗(左``饣''右``甫''音bu
3)之。孔子下车而前,见谒者曰:``鲁人孔丘,闻将军高义,敬再
拜谒者。''谒者入通。盗跖闻之大怒,目如明星,发上指冠,曰:``
此夫鲁国之巧伪人孔丘非邪?为我告之:尔作言造语,妄称文、武,
冠枝木之冠,带死牛之胁,多辞缪说,不耕而食,不织而衣,摇唇鼓
舌,擅生是非,以迷天下之主,使天下学士不反其本,妄作孝弟,而
侥幸于封侯富贵者也。子之罪大极重,疾走归!不然,我将以子肝益
昼囗(左``饣''右``甫'')之膳。''

孔子复通曰:``丘得幸于季,愿望履幕下。''谒者复通。盗跖曰:
使来前!''孔子趋而进,避席反走,再拜盗跖。盗跖大怒,两展其足
,案剑囗(左``目''右``真'')目,声如乳虎,曰:``丘来前!若所
言顺吾意则生,逆吾心则死。''

孔子曰:``丘闻之,凡天下有三德:生而长大,美好无双,少长贵
贱见而皆说之,此上德也;知维天地,能辩诸物,此中德也;勇悍果
敢,聚众率兵,此下德也。凡人有此一德者,足以南面称孤矣。今将
军兼此三者,身长八尺二寸,面目有光,唇如激丹,齿如齐贝,音中
黄钟,而名曰盗跖,丘窃为将军耻不取焉。将军有意听臣,臣请南使
吴越,北使齐鲁,东使宋卫,西使晋楚,使为将军造大城数百里,立
数十万户之邑,尊将军为诸侯,与天下更始,罢兵休卒,收养昆弟,
共祭先祖。此圣人才士之行,而天下之愿也。''

盗跖大怒曰:``丘来前!夫可规以利而可谏以言者,皆愚陋恒民之
谓耳。今长大美好,人见而悦之者,此吾父母之遗德也,丘虽不吾誉
,吾独不自知邪?且吾闻之,好面誉人者,亦好背而毁之。今丘告我
以大城众民,是欲规我以利而恒民畜我也,安可久长也!城之大者,
莫大乎天下矣。尧、舜有天下,子孙无置锥之地;汤、武立为天子,
而后世绝灭。非以其利大故邪?且吾闻之,古者禽兽多而人少,于是
民皆巢居以避之。昼拾橡栗,暮栖木上,故命之曰`有巢氏之民'。
古者民不知衣服,夏多积薪,冬则炀之,故命之曰`知生之民'。神
农之世,卧则居居,起则于于。民知其母,不知其父,与麋鹿共处,
耕而食,织而衣,无有相害之心。此至德之隆也。然而黄帝不能致德
,与蚩由战于涿鹿之野,流血百里。尧、舜作,立群臣,汤放其主,
武王杀纣。自是之后,以强陵弱,以众暴寡。汤、武以来,皆乱人之
徒也。今子修文、武之道,掌天下之辩,以教后世。缝衣浅带,矫言
伪行,以迷惑天下之主,而欲求富贵焉。盗莫大于子,天下何故不谓
子为盗丘,而乃谓我为盗跖?子以甘辞说子路而使从之。使子路去其
危冠,解其长剑,而受教于子。天下皆曰∶`孔丘能止暴禁非。',
其卒之也,子路欲杀卫君而事不成,身菹于卫东门之上,是子教之不
至也。子自谓才士圣人邪,则再逐于鲁,削迹于卫,穷于齐,围于陈
蔡,不容身于天下。子教子路菹。此患,上无以为身,下无以为人。
子之道岂足贵邪?世之所高,莫若黄帝。黄帝尚不能全德,而战于涿
鹿之野,流血百里。尧不慈,舜不孝,禹偏枯,汤放其主,武王伐纣
,文王拘囗(``美''字以``久''代``大''音you3)里。此六子者
,世之所高也。孰论之,皆以利惑其真而强反其情性,其行乃甚可羞
也。世之所谓贤士:伯夷、叔齐。伯夷、叔齐辞孤竹之君,而饿死于
首阳之山,骨肉不葬。鲍焦饰行非世,抱木而死。申徒狄谏而不听,
负石自投于河,为鱼鳖所食。介子推至忠也,自割其股以食文公。文
公后背之,子推怒而去,抱木而燔死。尾生与女子期于梁下,女子不
来,水至不去,抱梁柱而死。此六子者,无异于磔犬流豕、操瓢而乞
者,皆离名轻死,不念本养寿命者也。世之所谓忠臣者,莫若王子比
干、伍子胥。子胥沉江,比干剖心。此二子者,世谓忠臣也,然卒为
天下笑。自上观之,至于子胥、比干,皆不足贵也。丘之所以说我者
,若告我以鬼事,则我不能知也;若告我以人事者,不过此矣,皆吾
所闻知也。今吾告子以人之情:目欲视色,耳欲听声,口欲察味,志
气欲盈。人上寿百岁,中寿八十,下寿六十,除病瘦死丧忧患,其中
开口而笑者,一月之中不过四五日而已矣。天与地无穷,人死者有时
。操有时之具,而托于无穷之间,忽然无异骐骥之驰过隙也。不能说
其志意、养其寿命者,皆非通道者也。丘之所言,皆吾之所弃也。亟
去走归,无复言之!子之道狂狂汲汲,诈巧虚伪事也,非可以全真也
,奚足论哉!''

孔子再拜趋走,出门上车,执辔三失,目芒然无见,色若死灰,据
轼低头,不能出气。

归到鲁东门外,适遇柳下季。柳下季曰:``今者阙然,数日不见,
车马有行色,得微往见跖邪?''孔子仰天而叹曰:``然!''柳下季曰
:``跖得无逆汝意若前乎?''孔子曰:``然。丘所谓无病而自灸也。
疾走料虎头,编虎须,几不免虎口哉!''

子张问于满苟得曰:``盍不为行?无行则不信,不信则不任,不任
则不利。故观之名,计之利,而义真是也。若弃名利,反之于心,则
夫士之为行,不可一日不为乎!''满苟得曰:``无耻者富,多信者显
。夫名利之大者,几在无耻而信。故观之名,计之利,而信真是也。
若弃名利,反之于心,则夫士之为行,抱其天乎!''子张曰:``昔者
桀、纣贵为天子,富有天下。今谓臧聚曰:`汝行如桀、纣。'则有
怍色,有不服之心者,小人所贱也。仲尼、墨翟,穷为匹夫,今谓宰
相曰`子行如仲尼、墨翟。'则变容易色,称不足者,士诚贵也。故
势为天子,未必贵也;穷为匹夫,未必贱也。贵贱之分,在行之美恶
。''满苟得曰:``小盗者拘,大盗者为诸侯。诸侯之门,义士存焉。
昔者桓公小白杀兄入嫂,而管仲为臣;田成子常杀君窃国,而孔子受
币。论则贱之,行则下之,则是言行之情悖战于胸中也,不亦拂乎!
故《书》曰:`孰恶孰美,成者为首,不成者为尾。'''子张曰:``
子不为行,即将疏戚无伦,贵贱无义,长幼无序。五纪六位,将何以
为别乎?''满苟得曰:``尧杀长子,舜流母弟,疏戚有伦乎?汤放桀
,武王杀纣,贵贱有义乎?王季为适,周公杀兄,长幼有序乎?儒者
伪辞,墨子兼爱,五纪六位,将有别乎?且子正为名,我正为利。名
利之实,不顺于理,不监于道。吾日与子讼于无约,曰`小人殉财,
君子殉名,其所以变其情、易其性则异矣;乃至于弃其所为而殉其所
不为则一也。'故曰:无为小人,反殉而天;无为君子,从天之理。
若枉若直,相而天极。面观四方,与时消息。若是若非,执而圆机。
独成而意,与道徘徊。无转而行,无成而义,将失而所为。无赴而富
,无殉而成,将弃而天。比干剖心,子胥抉眼,忠之祸也;直躬证父
,尾生溺死,信之患也;鲍子立干,申子不自理,廉之害也;孔子不
见母,匡子不见父,义之失也。此上世之所传、下世之所语以为士者
,正其言,必其行,故服其殃、离其患也。''

无足问于知和曰:``人卒未有不兴名就利者。彼富则人归之,归则
下之,下则贵之。夫见下贵者,所以长生安体乐意之道也。今子独无
意焉,知不足邪?意知而力不能行邪?故推正不妄邪?''知和曰:``
今夫此人,以为与己同时而生,同乡而处者,以为夫绝俗过世之士焉
,是专无主正,所以览古今之时、是非之分也。与俗化世,去至重,
弃至尊,以为其所为也。此其所以论长生安体乐意之道,不亦远乎!
惨怛之疾,恬愉之安,不监于体;怵惕之恐,欣欣之喜,不监于心。
知为为而不知所以为。是以贵为天子,富有天下,而不免于患也。''
无足曰:``夫富之于人,无所不利。穷美究势,至人之所不得逮,贤
人之所不能及。侠人之勇力而以为威强,秉人之知谋以为明察,因人
之德以为贤良,非享国而严若君父。且夫声色滋味权势之于人,心不
待学而乐之,体不待象而安之。夫欲恶避就,固不待师,此人之性也
。天下虽非我,孰能辞之!''知和曰:``知者之为,故动以百姓,不
违其度,是以足而不争,无以为故不求。不足故求之,争四处而不自
以为贪;有余故辞之,弃天下而不自以为廉。廉贪之实,非以迫外也
,反监之度。势为天子,而不以贵骄人;富有天下,而不以财戏人。
计其患,虑其反,以为害于性,故辞而不受也,非以要名誉也。尧、
舜为帝而雍,非仁天下也,不以美害生;善卷、许由得帝而不受,非
虚辞让也,不以事害己。此皆就其利、辞其害,而天下称贤焉,则可
以有之,彼非以兴名誉也。''无足曰:``必持其名,苦体绝甘,约养
以持生,则亦久病长厄而不死者也。''知和曰:``平为福,有余为害
者,物莫不然,而财其甚者也。今富人,耳营钟鼓管囗(上``竹''下
``龠''音yue4)之声,口惬于刍豢醪醴之味,以感其意,遗忘其
业,可谓乱矣;囗(左``亻''右``亥''音gai1)溺于冯气,若负
重行而上阪,可谓苦矣;贪财而取慰,贪权而取竭,静居则溺,体泽
则冯,可谓疾矣;为欲富就利,故满若堵耳而不知避,且冯而不舍,
可谓辱矣;财积而无用,服膺而不舍,满心戚醮,求益而不止,可谓
忧矣;内则疑劫请之贼,外则畏寇盗之害,内周楼疏,外不敢独行,
可谓畏矣。此六者,天下之至害也,皆遗忘而不知察。及其患至,求
尽性竭财单以反一日之无故而不可得也。故观之名则不见,求之利则
不得。缭意绝体而争此,不亦惑乎!''

\hypertarget{header-n2391}{%
\subsubsection{说剑}\label{header-n2391}}

昔赵文王喜剑,剑士夹门而客三千余人,日夜相击于前,死伤者岁
百余人。好之不厌。如是三年,国衰。诸侯谋之。太子悝患之,募左
右曰:``孰能说王之意止剑士者,赐之千金。''左右曰:``庄子当能
。''太子乃使人以千金奉庄子。庄子弗受,与使者俱往见太子,曰:
``太子何以教周,赐周千金?''太子曰:``闻夫子明圣,谨奉千金以
币从者。夫子弗受,悝尚何敢言。''庄子曰:``闻太子所欲用周者,
欲绝王之喜好也。使臣上说大王而逆王意,下不当太子,则身刑而死
,周尚安所事金乎?使臣上说大王,下当太子,赵国何求而不得也!
''太子曰∶``然。吾王所见,唯剑士也。''庄子曰:``诺。周善为剑
。''太子曰:``然吾王所见剑士,皆蓬头突鬓,垂冠,曼胡之缨,短
后之衣,瞋囗目而语难,王乃说之。今夫子必儒
服而见王,事必大逆。''庄子曰:``请治剑服。''治剑服三日,乃见
太子。太子乃与见王。王脱白刃待之。庄子入殿门不趋,见王不拜。
王曰:``子欲何以教寡人,使太子先。''曰:``臣闻大王喜剑,故以
剑见王。''王曰:``子之剑何能禁制?''曰:``臣之剑十步一人,千
里不留行。''王大悦之,曰:``天下无敌矣。''庄子曰:``夫为剑者
,示之以虚,开之以利,后之以发,先之以至。愿得试之。''王曰:
``夫子休,就舍待命,令设戏请夫子。''王乃校剑士七日,死伤者六
十余人,得五六人,使奉剑于殿下,乃召庄子。王曰:``今日试使士
敦剑。''庄子曰:``望之久矣!''王曰:``夫子所御杖,长短何如?
''曰:``臣之所奉皆可。然臣有三剑,唯王所用。请先言而后试。''
王曰:``愿闻三剑。''曰:``有天子剑,有诸侯剑,有庶人剑。''王
曰:``天子之剑何如?''曰:``天子之剑,以燕谿石城为锋,齐岱为
锷,晋卫为脊,周宋为镡,韩魏为夹,包以四夷,裹以四时,绕以渤
海,带以常山,制以五行,论以刑德,开以阴
阳,持以春夏,行以秋冬。此剑直之无前,举之无上,案之无下,运
之无旁。上决浮云,下绝地纪。此剑一用,匡诸侯,天下服矣。此天
子之剑也。''文王芒然自失,曰:``诸侯之剑何如?''曰:``诸侯之
剑,以知勇士为锋,以清廉士为锷,以贤良士为脊,以忠圣士为镡,
以豪桀士为夹。此剑直之亦无前,举之亦无上,案之亦无下,运之亦
无旁。上法圆天,以顺三光;下法方地,以顺四时;中和民意,以安
四乡。此剑一用,如雷霆之震也,四封之内,无不宾服而听从君命者
矣。此诸侯之剑也。''王曰:``庶人之剑何如?''曰:``庶人之剑,
蓬头突鬓,垂冠,曼胡之缨,短后之衣,瞋目
而语难,相击于前,上斩颈领,下决肝肺。此庶人之剑,无异于斗鸡
,一旦命已绝矣,无所用于国事。今大王有天子之位而好庶人之剑,
臣窃为大王薄之。''王乃牵而上殿,宰人上食,王三环之。庄子曰:
``大王安坐定气,剑事已毕奏矣!''于是文王不出宫三月,剑士皆服 毙其处也。

\hypertarget{header-n2396}{%
\subsubsection{渔父}\label{header-n2396}}

孔子游乎缁帷之林,休坐乎杏坛之上。弟子读书,孔子弦歌鼓琴。
奏曲未半,有渔父者,下船而来,须眉交白,被发揄袂,行原以上,
距陆而止,左手据膝,右手持颐以听。曲终而招子贡、子路二人俱对
。客指孔子曰:``彼何为者也?''子路对曰:``鲁之君子也。''客问
其族。子路对曰:``族孔氏。''客曰:``孔氏者何治也?''子路未应
,子贡对曰:``孔氏者,性服忠信,身行仁义,饰礼乐,选人伦。上
以忠于世主,下以化于齐民,将以利天下。此孔氏之所治也。''又问
曰:``有土之君与?''子贡曰:``非也。''``侯王之佐与?''子贡曰
:``非也。''客乃笑而还行,言曰:``仁则仁矣,恐不免其身。苦心
劳形以危其真。呜呼!远哉,其分于道也。''

子贡还,报孔子。孔子推琴而起,曰:``其圣人与?''乃下求之,
至于泽畔,方将杖拏而引其船,顾见
孔子,还乡而立。孔子反走,再拜而进。客曰:``子将何求?''孔子
曰:``曩者先生有绪言而去,丘不肖,未知所谓,窃待于下风,幸闻
咳唾之音,以卒相丘也。''客曰:``嘻!甚矣,子之好学也!''孔子
再拜而起,曰:``丘少而修学,以至于今,六十九岁矣,无所得闻至
教,敢不虚心!''客曰:``同类相从,同声相应,固天之理也。吾请
释吾之所有而经子之所以。子之所以者,人事也。天子诸侯大夫庶人
,此四者自正,治之美也;四者离位而乱莫大焉。官治其职,人忧其
事,乃无所陵。故田荒室露,衣食不足,征赋不属,妻妾不和,长少
无序,庶人之忧也;能不胜任,官事不治,行不清白,群下荒怠,功
美不有,爵禄不持,大夫之忧也;廷无忠臣,国家昏乱,工技不巧,
贡职不美,春秋后伦,不顺天子,诸侯之忧也;阴阳不和,寒暑不时
,以伤庶物,诸侯暴乱,擅相攘伐,以残民人,礼乐不节,财用穷匮
,人伦不饬,百姓淫乱,天子有司之忧也。今子既上无君侯有司之势
,而下无大臣职事之官,而擅饰礼乐,选人伦,以化齐民,不泰多事
乎?且人有八疵,事有四患,不可不察也。非其事而事之,谓之总;
莫之顾而进之,谓之佞;希意道言,谓之谄;不择是非而言,谓之谀
;好言人之恶,谓之谗;析交离亲,谓之贼;称誉诈伪以败恶人,谓
之慝;不择善否,两容颊适,偷拔其所欲,谓之险。此八疵者,外以
乱人,内以伤身,君子不友,明君不臣。所谓四患者:好经大事,变
更易常,以挂功名,谓之叨;专知擅事,侵人自用,谓之贪;见过不
更,闻谏愈甚,谓之很;人同于己则可,不同于己,虽善不善,谓之
矜。此四患也。能去八疵,无行四患,而始可教已。

孔子愀然而叹,再拜而起,曰:``丘再逐于鲁,削迹于卫,伐树于
宋,围于陈蔡。丘不知所失,而离此四谤者何也?''客凄然变容曰:
``甚矣,子之难悟也!人有畏影恶迹而去之走者,举足愈数而迹愈多
,走愈疾而影不离身,自以为尚迟,疾走不休,绝力而死。不知处阴
以休影,处静以息迹,愚亦甚矣!子审仁义之间,察同异之际,观动
静之变,适受与之度,理好恶之情,和喜怒之节,而几于不免矣。谨
修而身,慎守其真,还以物与人,则无所累矣。今不修之身而求之人
,不亦外乎!''

孔子愀然曰:``请问何谓真?''客曰:``真者,精诚之至也。不精
不诚,不能动人。故强哭者,虽悲不哀,强怒者,虽严不屯,强亲者
,虽笑不和。真悲无声而哀,真怒未发而威,真亲未笑而和。真在内
者,神动于外,是所以贵真也。其用于人理也,事亲则慈孝,事君则
忠贞,饮酒则欢乐,处丧则悲哀。忠贞以功为主,饮酒以乐为主,处
丧以哀为主,事亲以适为主。功成之美,无一其迹矣;事亲以适,不
论所以矣;饮酒以乐,不选其具矣;处丧以哀,无问其礼矣。礼者,
世俗之所为也;真者,所以受于天也,自然不可易也。故圣人法天贵
真,不拘于俗。愚者反此。不能法天而恤于人,不知贵真,禄禄而受
变于俗,故不足。惜哉,子之蚤湛于伪而晚闻大道也!''

孔子再拜而起曰:``今者丘得遇也,若天幸然。先生不羞而比之服
役而身教之。敢问舍所在,请因受业而卒学大道。''客曰:``吾闻之
,可与往者,与之至于妙道;不可与往者,不知其道。慎勿与之,身
乃无咎。子勉之,吾去子矣,吾去子矣!''乃剌船而去,延缘苇间。

颜渊还车,子路授绥,孔子不顾,待水波定,不闻拏音而后敢乘。
子路旁车而问曰:``由得为役久矣,未尝见夫子遇人如此其威也。万
乘之主,千乘之君,见夫子未尝不分庭伉礼,夫子犹有倨傲之容。今
渔父杖拏逆立,而夫子曲要磬折,言拜而应,得无太甚乎!门人皆怪
夫子矣,渔父何以得此乎!''孔子伏轼而叹,曰:``甚矣,由之难化
也!湛于礼义有间矣,而朴鄙之心至今未去。进,吾语汝:夫遇长不
敬,失礼也;见贤不尊,不仁也。彼非至人,不能下人。下人不精,
不得其真,故长伤身。惜哉!不仁之于人也,祸莫大焉,而由独擅之。
且道者,万物之所由也。庶物失之者死,得之者生。为事逆之则败,
顺之则成。故道之所在,圣人尊之。今之渔父之于道,可谓有矣,吾
敢不敬乎!''

\hypertarget{header-n2406}{%
\subsubsection{列御寇}\label{header-n2406}}

列御寇之齐,中道而反,遇伯昏瞀人。伯昏瞀人曰:``奚方而反?
''曰:``吾惊焉。''曰:``恶乎惊?''曰:``吾尝食于十浆而五浆先
馈。''伯昏瞀人曰:``若是则汝何为惊已?''曰:``夫内诚不解,形
谍成光,以外镇人心,使人轻乎贵老,而赍其所患。夫浆人特为食羹
之货,无多余之赢,其为利也薄,其为权也轻,而犹若是,而况于万
乘之主乎!身劳于国而知尽于事。彼将任我以事,而效我以功。吾是
以惊。''伯昏瞀人曰:``善哉观乎!女处已,人将保汝矣!''无几何
而往,则户外之屦满矣。伯昏瞀人北面而立,敦杖蹙之乎颐。立有间
,不言而出。宾者以告列子,列子提屦,跣而走,暨于门,曰:``先
生既来,曾不发药乎?''曰:``
已矣,吾固告汝曰:人将保汝。果保汝矣!非汝能使人保汝,而汝不
能使人无保汝也,而焉用之感豫出异也。必且有感,摇而本性,又无
谓也。与汝游者,又莫汝告也。彼所小言,尽人毒也。莫觉莫悟,何
相孰也。巧者劳而知者忧,无能者无所求,饱食而敖游,汎若不系之
舟,虚而敖游者也!

``郑人缓也,呻吟裘氏之地。祗三年而缓为儒。河润九里,泽及三
族,使其弟墨。儒墨相与辩,其父助翟。十年而缓自杀。其父梦之曰
:`使而子为墨者,予也,阖尝视其良?既为秋柏之实矣。'夫造物
者之报人也,不报其人而报其人之天,彼故使彼。夫人以己为有以异
于人,以贱其亲。齐人之井饮者相捽也。故曰:今之世皆缓也。自是
有德者以不知也,而况有道者乎!古者谓之遁天之刑。圣人安其所安,
不安其所不安;众人安其所不安,不安其所安。

``庄子曰:`知道易,勿言难。知而不言,所以之天也。知而言之
,所以之人也。古之人,天而不人。'朱泙漫学屠龙于支离益,单千
金之家,三年技成而无所用其巧
。圣人以必不必,故无兵;众人以不必必之,故多兵。顺于兵,故行
有求。兵,恃之则亡。小夫之知,不离苞苴竿牍,敝精神乎蹇浅,而
欲兼济道物,太一形虚。若是者,迷惑于宇宙,形累不知太初。彼至
人者,归精神乎无始,而甘冥乎无何有之乡。水流乎无形,发泄乎太
清。悲哉乎!汝为知在毫毛而不知大宁。''

宋人有曹商者,为宋王使秦。其往也,得车数乘。王说之,益车百
乘。反于宋,见庄子,曰:``夫处穷闾厄巷,困窘织屦,槁项黄馘者
,商之所短也;一悟万乘之主而从车百乘者,商之所长也。''庄子曰
:``秦王有病召医。破痈溃痤者得车一乘,舐痔者得车五乘,所治愈
下,得车愈多。子岂治其痔邪?何得车之多也?子行矣!''

鲁哀公问乎颜阖曰:``吾以仲尼为贞幹,国其有瘳乎?''曰:``殆
哉圾乎!仲尼方且饰羽而画,从事华
辞。以支为旨,忍性以视民,而不知不信。受乎心,宰乎神,夫何足
以上民!彼宜女与予颐与,误而可矣!今使民离实学伪,非所以视民
也。为后世虑,不若休之。难治也!''施于人而不忘,非天布也,商
贾不齿。虽以事齿之,神者弗齿。为外刑者,金与木也;为内刑者,
动与过也。宵人之离外刑者,金木讯之;离内刑者,阴阳食之。夫免
乎外内之刑者,唯真人能之。

孔子曰:``凡人心险于山川,难于知天。天犹有春秋冬夏旦暮之期
,人者厚貌深情。故有貌愿而益,有长若不肖,有慎狷而达,有坚而
缦,有缓而悍。故其就义若渴者,其去义若热。故君子远使之而观其
忠,近使之而观其敬,烦使之而观其能,卒然问焉而观其知,急与之
期而观其信,委之以财而观其仁,告之以危而观其节,醉之以酒而观
其侧,杂之以处而观其色。九徵至,不肖人得矣。''

正考父一命而伛,再命而偻,三命而俯,循墙而走,孰敢不轨!如
而夫者,一命而吕钜,再命而于车上舞,三命而名诸父。孰协唐许?
贼莫大乎德有心而心有睫,及其有睫也而内视,内视而败矣!凶德有
五,中德为首。何谓中德?中德也者,有以自好也而吡其所不为者也
。穷有八极,达有三必,形有六府。美、髯、长、大、壮、丽、勇、
、敢,八者俱过人也,因以是穷;缘循、偃仰、困畏,不若人三者俱
通达;知慧外通,勇动多怨,仁义多责,六者所以相刑也。达生之性
者傀,达于知者肖,达大命者随,达小命者遭。

人有见宋王者,锡车十乘。以其十乘骄稚庄子。庄子曰:``河上有
家贫恃纬萧而食者,其子没于渊,得千金之珠。其父谓其子曰:`取
石来锻之!夫千金之珠,必在九重之渊而骊龙颔下。子能得珠者,必
遭其睡也。使骊龙而寤,子尚奚微之有哉!'今宋国之深,非直九重
之渊也;宋王之猛,非直骊龙也。子能得车者,必遭其睡也;使宋王
而寐,子为赍粉夫。''

或聘于庄子,庄子应其使曰:``子见夫牺牛乎?衣以文绣,食以刍
叔。及其牵而入于大庙,虽欲为孤犊,其可得乎!''

庄子将死,弟子欲厚葬之。庄子曰:``吾以天地为棺椁,以日月为
连璧,星辰为珠玑,万物为赍送。吾葬具岂不备邪?何以加此!''弟
子曰:``吾恐乌鸢之食夫子也。''庄子曰:``在上为乌鸢食,在下为
蝼蚁食,夺彼与此,何其偏也。''以不平平,其平也不平;以不徵徵
,其徵也不徵。明者唯为之使,神者徵之。夫明之不胜神也久矣,而
愚者恃其所见入于人,其功外也,不亦悲夫!

\hypertarget{header-n2420}{%
\subsubsection{天下}\label{header-n2420}}

天下之治方术者多矣,皆以其有为不可加矣!古之所谓道术者,果
恶乎在?曰:``无乎不在。''曰∶``神何由降?明何由出?''``圣有
所生,王有所成,皆原于一。''不离于宗,谓之天人;不离于精,谓
之神人;不离于真,谓之至人。以天为宗,以德为本,以道为门,兆
于变化,谓之圣人;以仁为恩,以义为理,以礼为行,以乐为和,熏
然慈仁,谓之君子;以法为分,以名为表,以参为验,以稽为决,其
数一二三四是也,百官以此相齿;以事为常,以衣食为主,蕃息畜藏
,老弱孤寡为意,皆有以养,民之理也。古之人其备乎!配神明,醇
天地,育万物,和天下,泽及百姓,明于本数,系于末度,六通四辟
,小大精粗,其运无乎不在。其明而在数度者,旧法、世传之史尚多
有之;其在于《诗》、《书》、《礼》、《乐》者,邹鲁之士、缙绅
先生多能明之。《诗》以道志,《书》以道事,《礼》以道行,《乐
》以道和,《易》以道阴阳,《春秋》以道名分。其数散于天下而设
于中国者,百家之学时或称而道之。

天下大乱,贤圣不明,道德不一。天下多得一察焉以自好。譬如耳
目鼻口,皆有所明,不能相通。犹百家众技也,皆有所长,时有所用
。虽然,不该不遍,一曲之士也。判天地之美,析万物之理,察古人
之全。寡能备于天地之美,称神明之容。是故内圣外王之道,暗而不
明,郁而不发,天下之人各为其所欲焉以自为方。悲夫!百家往而不
反,必不合矣!后世之学者,不幸不见天地之纯,古人之大体。道术
将为天下裂。

不侈于后世,不靡于万物,不晖于数度,以绳墨自矫,而备世之急
。古之道术有在于是者,墨翟、禽滑厘闻其风而说之。为之大过,已
之大顺。作为《非乐》,命之曰《节用》。生不歌,死无服。墨子泛
爱兼利而非斗,其道不怒。又好学而博,不异,不与先王同,毁古之
礼乐。黄帝有《咸池》,尧有《大章》,舜有《大韶》,禹有《大夏
》,汤有《大濩》,文王有辟雍之乐
,武王、周公作《武》。古之丧礼,贵贱有仪,上下有等。天子棺椁
七重,诸侯五重,大夫三重,士再重。今墨子独生不歌,死不服,桐
棺三寸而无椁,以为法式。以此教人,恐不爱人;以此自行,固不爱
己。未败墨子道。虽然,歌而非歌,哭而非哭,乐而非乐,是果类乎
?其生也勤,其死也薄,其道大觳。使人忧,使人悲,其行难为也。恐其不
可以为圣人之道,反天下之心。天下不堪。墨子虽独能任,奈天下何
!离于天下,其去王也远矣!墨子称道曰:``昔禹之湮洪水,决江河
而通四夷九州也。名山三百,支川三千,小者无数。禹亲自操橐耜而
九杂天下之川。腓无跋,胫无毛,沐甚雨,栉疾风,置万国。禹大圣
也,而形劳天下也如此。''使后世之墨者,多以裘褐为衣,以屐蹻为
服,日夜不休,以自苦为极,曰:``不能如此,非禹之道也,不足谓墨
。''相里勤之弟子,五侯之徒,南方之墨者若获、已齿、邓陵子之属
,俱诵《墨经》,而倍谲不同,相谓别墨。以坚白同异之辩相訾,以
奇偶不仵之辞相应,以巨子为圣人。皆愿为之尸,冀得为其后世,至
今不决。墨翟、禽滑厘之意则是,其行则非也。将使后世之墨者,必
以自苦腓无跋、胫无毛相进而已矣。乱之上也,治之下也。虽然,墨
子真天下之好也,将求之不得也,虽枯槁不舍也,才士也夫!

不累于俗,不饰于物,不苟于人,不忮于众,愿天下之安宁以活民
命,人我之养,毕足而止,以此白心。古之道术有在于是者,宋钘、
尹文闻其风而悦之。作为华山之冠以自表,接万物以别宥为始。语心
之容,命之曰``心之行''。以聏合欢,以调海内。请欲置之以为主。
见侮不辱,救民之斗,禁攻寝兵,救世之战。以此周行天下,上说下
教。虽天下不取,强聒而不舍者也。故曰:上下见厌而强见也。虽然
,其为人太多,其自为太少,曰:``请欲固置五升之饭足矣。''先生
恐不得饱,弟子虽饥,不忘天下,日夜不休。曰:``我必得活哉!''
图傲乎救世之士哉!曰:``君子不为苛察,不以身假物。''以为无益
于天下者,明之不如己也。以禁攻寝兵为外,以情欲寡浅为内。其小
大精粗,其行适至是而止。

公而不党,易而无私,决然无主,趣物而不两,不顾于虑,不谋于
知,于物无择,与之俱往。古之道术有在于是者,彭蒙、田骈、慎到
闻其风而悦之。齐万物以为首,曰:``天能覆之而不能载之,地能载
之而不能覆之,大道能包之而不能辩之。''知万物皆有所可,有所不
可。故曰:``选则不遍,教则不至,道则无遗者矣。''是故慎到弃知
去己,而缘不得已。泠汰于物,以为道理。曰:``知不知,将薄知而
后邻伤之者也。''謑髁无任,而笑天下之尚贤也;纵脱无行,而非天
下之大圣;椎拍輐断,与物宛转;舍是与非,苟可以免。不师知虑,
不知前后,魏
然而已矣。推而后行,曳而后往。若飘风之还,若羽之旋,若磨石之
隧,全而无非,动静无过,未尝有罪。是何故?夫无知之物,无建己
之患,无用知之累,动静不离于理,是以终身无誉。故曰:``至于若
无知之物而已,无用贤圣。夫块不失道。''豪桀相与笑之曰:``慎到
之道,非生人之行,而至死人之理。''适得怪焉。田骈亦然,学于彭
蒙,得不教焉。彭蒙之师曰:``古之道人,至于莫之是、莫之非而已
矣。其风窨然,恶可而言。''常反人,不见观,而不免于魭断。其所
谓道非道,而所言之韪不免于非。彭蒙、田骈、慎到不知道。虽然,
概乎皆尝有闻者也。

以本为精,以物为粗,以有积为不足,澹然独与神明居。古之道术
有在于是者,关尹、老聃闻其风而悦之。建之以常无有,主之以太一
。以濡弱谦下为表,以空虚不毁万物为实。关尹曰:``在己无居,形
物自著。''其动若水,其静若镜,其应若响。芴乎若亡,寂乎若清。
同焉者和,得焉者失。未尝先人而常随人。老聃曰:``知其雄,守其
雌,为天下溪;知其白,守其辱,为天下谷。''人皆取先,己独取后
。曰:``受天下之垢''。人皆取实,己独取虚。``无藏也故有余''。
岿然而有余。其行身也,徐而不费,无为也而笑巧。人皆求福,己独
曲全。曰:``苟免于咎''。以深为根,以约为纪。曰:``坚则毁矣,
锐则挫矣''。常宽容于物,不削于人。虽未至于极,关尹、老聃乎,
古之博大真人哉!

寂漠无形,变化无常,死与?生与?天地并与?神明往与?芒乎何
之?忽乎何适?万物毕罗,莫足以归。古之道术有在于是者,庄周闻
其风而悦之。以谬悠之说,荒唐之言,无端崖之辞,时恣纵而不傥,
不奇见之也。以天下为沈浊,不可与庄语。以卮言为曼衍,以重言为
真,以寓言为广。独与天地精神往来,而不敖倪于万物。不谴是非,
以与世俗处。其书虽环玮,而连犿无伤也。其辞虽参差,而諔诡可观
。彼其充实,不可以已。上与造物者游,而下与外死生、无终始者为
友。其于本也,弘大而辟,深闳而肆;其于宗也,可谓稠适而上遂矣
。虽然,其应于化而解于物也,其理不竭,其来不蜕,芒乎昧乎,未 之尽者。

惠施多方,其书五车,其道舛驳,其言也不中。历物之意,曰:``
至大无外,谓之大一;至小无内,谓之小一。无厚,不可积也,其大
千里。天与地卑,山与泽平。日方中方睨,物方生方死。大同而与小
同异,此之谓`小同异';万物毕同毕异,此之谓`大同异'。南方
无穷而有穷。今日适越而昔来。连环可解也。我知天之中央,燕之北
、越之南是也。泛爱万物,天地一体也。''惠施以此为大,观于天下
而晓辩者,天下之辩者相与乐之。卵有毛。鸡有三足。郢有天下。犬
可以为羊。马有卵。丁子有尾。火不热。山出口。轮不蹍地。目不见。
指不至,至不绝。龟长于蛇。矩不方,规不
可以为圆。凿不围枘。飞鸟之景未尝动也。镞矢之疾,而有不行、不
止之时。狗非犬。黄马骊牛三。白狗黑。孤驹未尝有母。一尺之棰,
日取其半,万世不竭。辩者以此与惠施相应,终身无穷。桓团、公孙
龙辩者之徒,饰人之心,易人之意,能胜人之口,不能服人之心,辩
者之囿也。惠施日以其知与之辩,特与天下之辩者为怪,此其柢也。
然惠施之口谈,自以为最贤,曰:``天地其壮乎,施存雄而无术。''
南方有倚人焉,曰黄缭,问天地所以不坠不陷,风雨雷霆之故。惠施
不辞而应,不虑而对,遍为万物说。说而不休,多而无已,犹以为寡
,益之以怪,以反人为实,而欲以胜人为名,是以与众不适也。弱于
德,强于物,其涂隩矣。由天地之道观惠施之能,其犹一蚊一虻之劳者也。其于物也何庸!夫充一尚可,
曰愈贵,道几矣!惠施不能以此自宁,散于万物而不厌,卒以善辩为
名。惜乎!惠施之才,骀荡而不得,逐万物而不反,是穷响以声,形
与影竞走也,悲夫!

\end{document}
