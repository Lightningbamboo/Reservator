\PassOptionsToPackage{unicode=true}{hyperref} % options for packages loaded elsewhere
\PassOptionsToPackage{hyphens}{url}
%
\documentclass[]{article}
\usepackage{lmodern}
\usepackage{amssymb,amsmath}
\usepackage{ifxetex,ifluatex}
\usepackage{fixltx2e} % provides \textsubscript
\ifnum 0\ifxetex 1\fi\ifluatex 1\fi=0 % if pdftex
  \usepackage[T1]{fontenc}
  \usepackage[utf8]{inputenc}
  \usepackage{textcomp} % provides euro and other symbols
\else % if luatex or xelatex
  \usepackage{unicode-math}
  \defaultfontfeatures{Ligatures=TeX,Scale=MatchLowercase}
\fi
% use upquote if available, for straight quotes in verbatim environments
\IfFileExists{upquote.sty}{\usepackage{upquote}}{}
% use microtype if available
\IfFileExists{microtype.sty}{%
\usepackage[]{microtype}
\UseMicrotypeSet[protrusion]{basicmath} % disable protrusion for tt fonts
}{}
\IfFileExists{parskip.sty}{%
\usepackage{parskip}
}{% else
\setlength{\parindent}{0pt}
\setlength{\parskip}{6pt plus 2pt minus 1pt}
}
\usepackage{hyperref}
\hypersetup{
            pdfborder={0 0 0},
            breaklinks=true}
\urlstyle{same}  % don't use monospace font for urls
\setlength{\emergencystretch}{3em}  % prevent overfull lines
\providecommand{\tightlist}{%
  \setlength{\itemsep}{0pt}\setlength{\parskip}{0pt}}
\setcounter{secnumdepth}{0}
% Redefines (sub)paragraphs to behave more like sections
\ifx\paragraph\undefined\else
\let\oldparagraph\paragraph
\renewcommand{\paragraph}[1]{\oldparagraph{#1}\mbox{}}
\fi
\ifx\subparagraph\undefined\else
\let\oldsubparagraph\subparagraph
\renewcommand{\subparagraph}[1]{\oldsubparagraph{#1}\mbox{}}
\fi

% set default figure placement to htbp
\makeatletter
\def\fps@figure{htbp}
\makeatother


\date{}

\begin{document}

\hypertarget{header-n0}{%
\section{左传}\label{header-n0}}

\begin{center}\rule{0.5\linewidth}{\linethickness}\end{center}

\tableofcontents

\begin{center}\rule{0.5\linewidth}{\linethickness}\end{center}

\hypertarget{header-n7}{%
\subsection{隐公}\label{header-n7}}

\begin{center}\rule{0.5\linewidth}{\linethickness}\end{center}

\hypertarget{header-n9}{%
\subsubsection{隐公元年}\label{header-n9}}

【经】元年春王正月。三月,公及邾仪父盟于蔑。夏五月,郑伯克段于鄢。秋七月,天王使宰咺来归惠公、仲子之賵。九月,及宋人盟于宿。冬十有二月,祭伯来。公子益师卒。

【传】元年春,王周正月。不书即位,摄也。

三月,公及邾仪父盟于蔑,邾子克也。未王命,故不书爵。曰``仪父」,贵之也。公摄位而欲求好于邾,故为蔑之盟。

夏四月,费伯帅师城郎。不书,非公命也。

初,郑武公娶于申,曰武姜,生庄公及共叔段。庄公寤生,惊姜氏,故名曰``寤生」,遂恶之。爱共叔段,欲立之。亟请于武公,公弗许。及庄公即位,为之请制。公曰:``制,岩邑也,虢叔死焉,佗邑唯命。」请京,使居之,谓之京城大叔。祭仲曰:``都,城过百雉,国之害也。先王之制:大都,不过参国之一;中,五之一;小,九之一。今京不度,非制也,君将不堪。」公曰:``姜氏欲之,焉辟害?」对曰:``姜氏何厌之有?不如早为之所,无使滋蔓!蔓,难图也。蔓草犹不可除,况君之宠弟乎?」公曰:``多行不义,必自毙,子姑待之。」
既而大叔命西鄙、北鄙贰于己。公子吕曰:``国不堪贰,君将若之何?欲与大叔,臣请事之;若弗与,则请除之。无生民心。」公曰:``无庸,将自及。」大叔又收贰以为己邑,至于廪延。子封曰:``可矣,厚将得众。」公曰:``不义不昵,厚将崩。」

大叔完、聚,缮甲、兵,具卒,乘,将袭郑,夫人将启之。公闻其期,曰:``可矣!」命子封帅车二百乘以伐京。京叛大叔段,段入于鄢,公伐诸鄢。五月辛丑,大叔出奔共。

书曰:``郑伯克段于鄢。」段不弟,故不言弟;如二君,故曰克;称郑伯,讥失教也:谓之郑志。不言出奔,难之也。

遂置姜氏于城颖,而誓之曰:``不及黄泉,无相见也。」既而悔之。颖考叔为颖谷封人,闻之,有献于公,公赐之食,食舍肉。公问之,对曰:``小人有母,皆尝小人之食矣,未尝君之羹,请以遗之。」公曰:``尔有母遗,繄我独无!」颖考叔曰:``敢问何谓也?」公语之故,且告之悔。对曰:``君何患焉?若阙地及泉,隧而相见,其谁曰不然?」公从之。公入而赋:``大隧之中,其乐也融融!」姜出而赋:``大隧之外,其乐也泄泄!」遂为母子如初。

君子曰:``颖考叔,纯孝也,爱其母,施及庄公。《诗》曰『孝子不匮,永锡尔类。』其是之谓乎!」

秋七月,天王使宰咺来归惠公、仲子之賵。缓,且子氏未薨,故名。天子七月而葬,同轨毕至;诸侯五月,同盟至;大夫三月,同位至;士逾月,外姻至。赠死不及尸,吊生不及哀,豫凶事,非礼也。

八月,纪人伐夷。夷不告,故不书。

有蜚。不为灾,亦不书。

惠公之季年,败宋师于黄。公立而求成焉。九月,及宋人盟于宿,始通也。

冬十月庚申,改葬惠公。公弗临,故不书。惠公之薨也,有宋师,太子少,葬故有阙,是以改葬。卫侯来会葬,不见公,亦不书。郑共叔之乱,公孙滑出奔卫。卫人为之伐郑,取廪延。郑人以王师、虢师伐卫南鄙。请师于邾。邾子使私于公子豫,豫请往,公弗许,遂行。及邾人、郑人盟于翼。不书,非公命也。

新作南门。不书,亦非公命也。

十二月,祭伯来,非王命也。

众父卒。公不与小敛,故不书日。

\hypertarget{header-n29}{%
\subsubsection{隐公二年}\label{header-n29}}

【经】二年春,公会戎于潜。夏五月,莒人入向。无骇帅师入极。秋八月庚辰,公及戎盟于唐。九月,纪裂繻来逆女。冬十月,伯姬归于纪。纪子帛、莒子盟于密。十有二月乙卯,夫人子氏薨。郑人伐卫。

【传】二年春,公会戎于潜,修惠公之好也。戎请盟,公辞。

莒子娶于向,向姜不安莒而归。夏,莒人入向以姜氏还。

司空无骇入极,费庈父胜之。

戎请盟。秋,盟于唐,复修戎好也。

九月,纪裂繻来逆女,卿为君逆也。

冬,纪子帛、莒子盟于密,鲁故也。

郑人伐卫,讨公孙滑之乱也。

\hypertarget{header-n40}{%
\subsubsection{隐公三年}\label{header-n40}}

【经】三年春王二月,己巳,日有食之。三月庚戌,天王崩。夏四月辛卯,君氏卒。秋,武氏子来求赙。八月庚辰,宋公和卒。冬十有二月,齐侯,郑伯盟于石门。癸未,葬宋穆公。

【传】三年春,王三月壬戌,平王崩,赴以庚戌,故书之。

夏,君氏卒。声子也。不赴于诸侯,不反哭于寝,不祔于姑,故不曰薨。不称夫人,故不言葬,不书姓。为公故,曰``君氏」。

郑武公、庄公为平王卿士。王贰于虢,郑伯怨王,王曰``无之」。故周、郑交质。王子狐为质于郑,郑公子忽为质于周。王崩,周人将畀虢公政。四月,郑祭足帅师取温之麦。秋,又取成周之禾。周、郑交恶。

君子曰:``信不由中,质无益也。明恕而行,要之以礼,虽无有质,谁能间之?苟有明信,涧溪沼沚之毛,苹蘩温藻之菜,筐筥錡釜之器,潢污行潦之水,可荐于鬼神,可羞于王公,而况君子结二国之信。行之以礼,又焉用质?《风》有《采繁》、《采苹》,《雅》有《行苇》、《泂酌》,昭忠信也。」

武氏子来求赙,王未葬也。

宋穆公疾,召大司马孔父而属殇公焉,曰:``先君舍与夷而立寡人,寡人弗敢忘。若以大夫之灵,得保首领以没,先君若问与夷,其将何辞以对?请子奉之,以主社稷,寡人虽死,亦无悔焉。」对曰:``群臣愿奉冯也。」公曰:``不可。先君以寡人为贤,使主社稷,若弃德不让,是废先君之举也。岂曰能贤?光昭先君之令德,可不务乎?吾子其无废先君之功。」使公子冯出居于郑。八月庚辰,宋穆公卒。殇公即位。

君子曰:``宋宣公可谓知人矣。立穆公,其子飨之,命以义夫。《商颂》曰:『殷受命咸宜,百禄是荷。』其是之谓乎!」

冬,齐、郑盟于石门,寻卢之盟也。庚戌,郑伯之车偾于济。

卫庄公娶于齐东宫得臣之妹,曰庄姜,美而无子,卫人所为赋《硕人》也。又娶于陈,曰厉妫,生孝伯,早死。其娣戴妫生桓公,庄姜以为己子。公子州吁,嬖人之子也,有宠而好兵,公弗禁,庄姜恶之。石碏谏曰:``臣闻爱子,教之以义方,弗纳于邪。骄、奢、淫、泆,所自邪也。四者之来,宠禄过也。将立州吁,乃定之矣,若犹未也,阶之为祸。夫宠而不骄,骄而能降,降而不憾,憾而能珍者鲜矣。且夫贱妨贵,少陵长,远间亲,新间旧,小加大,淫破义,所谓六逆也。君义,臣行,父慈,子孝,兄爱,弟敬,所谓六顺也。去顺效逆,所以速祸也。君人者将祸是务去,而速之,无乃不可乎?」弗听,其子厚与州吁游,禁之,不可。桓公立,乃老。

\hypertarget{header-n53}{%
\subsubsection{隐公四年 }\label{header-n53}}

【经】四年春王二月,莒人伐杞,取牟娄。戊申,卫州吁弑其君完。夏,公及宋公遇于清。宋公、陈侯、蔡人、卫人伐郑。秋,翬帅师会宋公、陈侯、蔡人、卫人伐郑。九月,卫人杀州吁于濮。冬十有二月,卫人立晋。

【传】四年春,卫州吁弑桓公而立。公与宋公为会,将寻宿之盟。未及期,卫人来告乱。夏,公及宋公遇于清。

宋殇公之即位也,公子冯出奔郑,郑人欲纳之。及卫州吁立,将修先君之怨于郑,而求宠于诸侯以和其民,使告于宋曰:``君若伐郑以除君害,君为主,敝邑以赋与陈、蔡从,则卫国之愿也。」宋人许之。于是,陈、蔡方睦于卫,故宋公、陈侯、蔡人、卫人伐郑,围其东门,五日而还。

公问于众仲曰:``卫州吁其成乎?」对曰:``臣闻以德和民,不闻以乱。以乱,犹治丝而棼之也。夫州吁,阻兵而安忍。阻兵无众,安忍无亲,众叛亲离,难以济矣。夫兵犹火也,弗戢,将自焚也。夫州吁弑其君而虐用其民,于是乎不务令德,而欲以乱成,必不免矣。」

秋,诸侯复伐郑。宋公使来乞师,公辞之。羽父请以师会之,公弗许,固请而行。故书曰``翬帅师」,疾之也。诸侯之师败郑徒兵,取其禾而还。

州吁未能和其民,厚问定君于石子。石子曰:``王觐为可。」曰:``何以得觐?」曰:``陈桓公方有宠于王,陈、卫方睦,若朝陈使请,必可得也。」厚从州吁如陈。石碏使告于陈曰:``卫国褊小,老夫耄矣,无能为也。此二人者,实弑寡君,敢即图之。」陈人执之而请莅于卫。九月,卫人使右宰丑莅杀州吁于濮,石碏使其宰乳羊肩莅杀石厚于陈。

君子曰:``石碏,纯臣也,恶州吁而厚与焉。『大义灭亲』,其是之谓乎!」

卫人逆公子晋于邢。冬十二月,宣公即位。书曰``卫人立晋」众也。

\hypertarget{header-n64}{%
\subsubsection{隐公五年 }\label{header-n64}}

【经】五年春,公矢鱼于棠。夏四月,葬卫桓公。秋,卫师入郕。九月,考仲子之宫。初献六羽。邾人、郑人伐宋。螟。冬十有二月辛巳,公子彄卒。宋人伐郑,围长葛。

【传】五年春,公将如棠观鱼者。臧僖伯谏曰:``凡物不足以讲大事,其材不足以备器用,则君不举焉。君将纳民于轨物者也。故讲事以度轨量谓之轨,取材以章物采谓之物,不轨不物谓之乱政。乱政亟行,所以败也。故春蒐夏苗,秋獮冬狩,皆于农隙以讲事也。三年而治兵,入而振旅,归而饮至,以数军实。昭文章,明贵贱,辨等列,顺少长,习威仪也。鸟兽之肉不登于俎,皮革齿牙、骨角毛羽不登于器,则公不射,古之制也。若夫山林川泽之实,器用之资,皂隶之事,官司之守,非君所及也。」公曰:``吾将略地焉。」遂往,陈鱼而观之。僖伯称疾,不从。书曰``公矢鱼于棠」,非礼也,且言远地也。

曲沃庄伯以郑人、邢人伐翼,王使尹氏、武氏助之。翼侯奔随。

夏,葬卫桓公。卫乱,是以缓。

四月,郑人侵卫牧,以报东门之役。卫人以燕师伐郑。郑祭足、原繁、泄驾以三军军其前,使曼伯与子元潜军军其后。燕人畏郑三军而不虞制人。六月,郑二公子以制人败燕师于北制。君子曰:``不备不虞,不可以师。」

曲沃叛王。秋,王命虢公伐曲沃而立哀侯于翼。

卫之乱也,郕人侵卫,故卫师入郕。

九月,考仲子之宫,将万焉。公问羽数于众仲。对曰:``天子用八,诸侯用六,大夫四,士二。夫舞所以节八音而行八风,故自八以下。」公从之。于是初献六羽,始用六佾也。

宋人取邾田。邾人告于郑曰:``请君释憾于宋,敝邑为道。」郑人以王师会之。伐宋,入其郛,以报东门之役。宋人使来告命。公闻其入郛也,将救之,问于使者曰:``师何及?」对曰:``未及国。」公怒,乃止,辞使者曰:``君命寡人同恤社稷之难,今问诸使者,曰『师未及国』,非寡人之所敢知也。」

冬十二月辛已,臧僖伯卒。公曰:``叔父有憾于寡人,寡人弗敢忘。葬之加一等。

宋人伐郑,围长葛,以报入郛之役也。

\hypertarget{header-n78}{%
\subsubsection{隐公六年 }\label{header-n78}}

【经】六年春,郑人来渝平。,夏五月辛酉,公会齐侯盟于艾。秋七月。冬,宋人取长葛。

【传】六年春,郑人来渝平,更成也。

翼九宗、五正顷父之子嘉父逆晋侯于随,纳诸鄂。晋人谓之鄂侯。

夏,盟于艾,始平于齐也。

五月庚申,郑伯侵陈,大获。

往岁,郑伯请成于陈,陈侯不许。五父谏曰:``亲仁善邻,国之宝也。君其许郑。」陈侯曰:``宋、卫实难,郑何能为?」遂不许。

君子曰:``善不可失,恶不可长,其陈桓公之谓乎!长恶不悛,从自及也。虽欲救之,其将能乎?《商书》曰:『恶之易也,如火之燎于原,不可乡迩,其犹可扑灭?』周任有言曰:『为国家者,见恶如农夫之务去草焉,芟夷蕴崇之,绝其本根,勿使能殖,则善者信矣。』」

秋,宋人取长葛。

冬,京师来告饥。公为之请籴于宋、卫、齐、郑,礼也。

郑伯如周,始朝桓王也。王不礼焉。周桓公言于王曰:``我周之东迁,晋、郑焉依。善郑以劝来者,犹惧不蔇,况不礼焉?郑不来矣!」

\hypertarget{header-n91}{%
\subsubsection{隐公七年 }\label{header-n91}}

【经】七年春王三月,叔姬归于纪。滕侯卒。夏,城中丘。齐侯使其弟年来聘。秋,公伐邾。冬,天王使凡伯来聘。戎伐凡伯于楚丘以归。

【传】七年春,滕侯卒。不书名,未同盟也。凡诸侯同盟,于是称名,故薨则赴以名,告终嗣也,以继好息民,谓之礼经。

夏,城中丘,书,不时也。

齐侯使夷仲年来聘,结艾之盟也。

秋,宋及郑平。七月庚申,盟于宿。公伐邾,为宋讨也。

初,戎朝于周,发币于公卿,凡伯弗宾。冬,王使凡伯来聘。还,戎伐之于楚丘以归。

陈及郑平。十二月,陈五父如郑莅盟。壬申,及郑伯盟,歃如忘泄伯曰:``五父必不免,不赖盟矣。」

郑良佐如陈莅盟,辛巳,及陈侯盟,亦知陈之将乱也。

郑公子忽在王所,故陈侯请妻之。郑伯许之,乃成昏。

\hypertarget{header-n103}{%
\subsubsection{隐公八年 }\label{header-n103}}

【经】八年春,宋公、卫侯遇于垂。三月,郑伯使宛来归祊。庚寅,我入祊。夏六月己亥,蔡侯考父卒。辛亥,宿男卒。秋七月庚午,宋公、齐侯、卫侯盟于瓦屋。八月,葬蔡宣公。九月辛卯,公及莒入盟于浮来。螟。冬十有二月,无骇卒。

【传】八年春,齐侯将平宋、卫,有会期。宋公以币请于卫,请先相见,卫侯许之,故遇于犬丘。

郑伯请释泰山之祀而祀周公,以泰山之祊易许田。三月,郑伯使宛来归祊,不祀泰山也。

夏,虢公忌父始作卿士于周。

四月甲辰,郑公子忽如陈逆妇妫。辛亥,以妫氏归。甲寅,入于郑。陈金咸子送女。先配而后祖。金咸子曰:``是不为夫妇。诬其祖矣,非礼也,何以能育?」

齐人卒平宋、卫于郑。秋,会于温,盟于瓦屋,以释东门之役,礼也。

八月丙戌,郑伯以齐人朝王,礼也。

公及莒人盟于浮来,以成纪好也。

冬,齐侯使来告成三国。公使众仲对曰:``君释三国之图以鸠其民,君之惠也。寡君闻命矣,敢不承受君之明德。」

无骇卒。羽父请谥与族。公问族于众仲。众仲对曰:``天子建德,因生以赐姓,胙之土而命之氏。诸侯以字为谥,因以为族。官有世功,则有官族,邑亦如之。」公命以字为展氏。

\hypertarget{header-n116}{%
\subsubsection{隐公九年}\label{header-n116}}

【经】九年春,天子使南季来聘。三月癸酉,大雨,震电。庚辰,大雨雪。挟卒。夏,城郎。秋七月。冬,公会齐侯于防。

【传】九年春,王三月癸酉,大雨霖以震,书始也。庚辰,大雨雪,亦如之。书,时失也。凡雨,自三日以往为霖。平地尺为大雪。

夏,城郎,书,不时也。

宋公不王。郑伯为王左卿士,以王命讨之,伐宋。宋以入郛之役怨公,不告命。公怒,绝宋使。

秋,郑人以王命来告伐宋。

冬,公会齐侯于防,谋伐宋也。

北戎侵郑,郑伯御之。患戎师,曰;``彼徒我车,惧其侵轶我也。」公子突曰:``使勇而无刚者尝寇,而速去之。君为三覆以待之。戎轻而不整,贪而无亲,胜不相让,败不相救。先者见获必务进,进而遇覆必速奔,后者不救,则无继矣。乃可以逞。」从之。

戎人之前遇覆者奔。祝聃逐之。衷戎师,前后击之,尽殪。戎师大奔。十一月甲寅,郑人大败戎师。

\hypertarget{header-n127}{%
\subsubsection{隐公十年}\label{header-n127}}

【经】十年春王二月,公会齐侯、郑伯于中丘。夏,翬帅师会齐人、郑人伐宋。六月壬戌,公败宋师于菅。辛未,取郜。辛巳,取防。秋,宋人、卫人入郑。宋人、蔡人、卫人伐戴。郑伯伐取之。冬十月壬午,齐人、郑人入郕。

【传】十年春,王正月,公会齐侯,郑伯于中丘。癸丑,盟于邓,为师期。

夏五月羽父先会齐侯、郑伯伐宋。

六月戊申,公会齐侯、郑伯于老桃。壬戌,公败宋师于菅。庚午,郑师入郜。辛未,归于我。庚辰,郑师入防。辛巳,归于我。

君子谓:``郑庄公于是乎可谓正矣。以王命讨不庭,不贪其土以劳王爵,正之体也。」

蔡人、卫人、郕人不会王命。

秋七月庚寅,郑师入郊。犹在郊,宋人、卫人入郑。蔡人从之,伐戴。八月壬戌,郑伯围戴。癸亥,克之,取三师焉。宋、卫既入郑,而以伐戴召蔡人,蔡人怒,故不和而败。

九月戊寅,郑伯入宋。

冬,齐人、郑人入郕,讨违王命也。

\hypertarget{header-n139}{%
\subsubsection{隐公十一年}\label{header-n139}}

【经】十有一年春,滕侯、薛侯来朝。夏,公会郑伯于时来。秋七月壬午,公及齐侯、郑伯入许。冬十有一月壬辰,公薨。

【传】十一年春,滕侯、薛侯来朝,争长。薛侯曰:``我先封。」滕侯曰:``我,周之卜正也。薛,庶姓也,我不可以后之。」

公使羽父请于薛侯曰:``君与滕君辱在寡人。周谚有之曰:『山有木,工则度之;宾有礼,主则择之。』周之宗盟,异姓为后。寡人若朝于薛,不敢与诸任齿。君若辱贶寡人,则愿以滕君为请。」

薛侯许之,乃长滕侯。

夏,公会郑伯于郲,谋伐许也。

郑伯将伐许,五月甲辰,授兵于大宫。公孙阏与颖考叔争车,颖考叔挟輈以走,子都拔棘以逐之,及大逵,弗及,子都怒。

秋七月,公会齐侯、郑伯伐许。庚辰,傅于许,颖考叔取郑伯之旗蝥弧以先登。子都自下射之,颠。瑕叔盈又以蝥弧登,周麾而呼曰:``君登矣!」郑师毕登。壬午,遂入许。许庄公奔卫。

齐侯以许让公。公曰:``君谓许不共,故从君讨之。许既伏其罪矣,虽君有命,寡人弗敢与闻。」乃与郑人。

郑伯使许大夫百里奉许叔以居许东偏,曰:``天祸许国,鬼神实不逞于许君,而假手于我寡人。寡人唯是一二父兄不能共亿,其敢以许自为功乎?寡人有弟,不能和协,而使,糊其口于四方,其况能久有许乎?吾子其奉许叔以抚柔此民也,吾将使获也佐吾子。若寡人得没于地,天其以礼悔祸于许?无宁兹许公复奉其社稷。唯我郑国之有请谒焉,如旧昏媾,其能降以相从也。无滋他族,实逼处此,以与我郑国争此土也。吾子孙其覆亡之不暇,而况能禋祀许乎?寡人之使吾子处此,不唯许国之为,亦聊以固吾圉也。」

乃使公孙获处许西偏,曰:``凡而器用财贿,无置于许。我死,乃亟去之。吾先君新邑于此,王室而既卑矣,周之子孙日失其序。夫许,大岳之胤也,天而既厌周德矣,吾其能与许争乎?」

君子谓:``郑庄公于是乎有礼。礼,经国家,定社稷,序民人,利后嗣者也。许无刑而伐之,服而舍之,度德而处之,量力而行之,相时而动,无累后人,可谓知礼矣。」

郑伯使卒出豭,行出犬鸡,以诅射颖考叔者。君子谓:``郑庄公失政刑矣。政以治民,刑以正邪,既无德政,又无威刑,是以及邪。邪而诅之,将何益矣!」

王取邬、刘、功蒍、邗之田于郑,而与郑人苏忿生之田温、原、丝希、樊、隰郕、欑茅、向、盟、州、陉、隤、怀。君子是以知桓王之失郑也。恕而行之,德之则也,礼之经也。己弗能有而以与人,人之不至,不亦宜乎?

郑、息有违言,息侯伐郑。郑伯与战于竟,息师大败而还。君子是以知息之将亡也。不度德,不量力,不亲亲,不征辞,不察有罪,犯五不韪而以伐人,其丧师也,不亦宜乎!

冬十月,郑伯以虢师伐宋。壬戌,大败宋师,以报其入郑也。宋不告命,故不书。凡诸侯有命,告则书,不然则否。师出臧否,亦如之。虽及灭国,灭不告败,胜不告克,不书于策。羽父请杀桓公,将以求大宰。公曰:``为其少故也,吾将授之矣。使营菟裘,吾将老焉。」羽父惧,反谮公于桓公而请弑之。公之为公子也,与郑人战于狐壤,止焉。郑人囚诸尹氏,赂尹氏而祷于其主钟巫,遂与尹氏归而立其主。十一月,公祭钟巫,齐于社圃,馆于寪氏。壬辰,羽父使贼弑公于寪氏,立桓公而讨寪氏,有死者。不书葬,不成丧也。

\hypertarget{header-n157}{%
\subsection{桓公}\label{header-n157}}

\begin{center}\rule{0.5\linewidth}{\linethickness}\end{center}

\hypertarget{header-n159}{%
\subsubsection{桓公元年}\label{header-n159}}

【经】元年春王正月,公即位。三月,公会郑伯于垂,郑伯以璧假许田。夏四月丁未,公及郑伯盟于越。秋,大水。冬十月。

【传】元年春,公即位,修好于郑。郑人请复祀周公,卒易祊田。公许之。三月,郑伯以璧假许田,为周公、祊故也。

夏四月丁未,公及郑伯盟于越,结祊成也。盟曰:``渝盟无享国。」

秋,大水。凡平原出水为大水。

冬,郑伯拜盟。

宋华父督见孔父之妻于路,目逆而送之,曰:``美而艳。」

\hypertarget{header-n168}{%
\subsubsection{桓公二年}\label{header-n168}}

【经】二年春,王正月戊申,宋督弑其君与夷及其大夫孔父。滕子来朝。三月,公会齐侯、陈侯、郑伯于稷,以成宋乱。夏四月,取郜大鼎于宋。戊申,纳于大庙。秋七月,杞侯来朝。蔡侯、郑伯会于邓。九月,入杞。公及戎盟于唐。冬,公至自唐。

【传】二年春,宋督攻孔氏,杀孔父而取其妻。公怒,督惧,遂弑殇公。

君子以督为有无君之心而后动于恶,故先书弑其君。会于稷以成宋乱,为赂故,立华氏也。

宋殇公立,十年十一战,民不堪命。孔父嘉为司马,督为大宰,故因民之不堪命,先宣言曰:``司马则然。」已杀孔父而弑殇公,召庄公于郑而立之,以亲郑。以郜大鼎赂公,齐、陈、郑皆有赂,故遂相宋公。

夏四月,取郜大鼎于宋。戊申,纳于大庙。非礼也。臧哀伯谏曰:``君人者将昭德塞违,以临照百官,犹惧或失之。故昭令德以示子孙:是以清庙茅屋,大路越席,大羹不致,粢食不凿,昭其俭也。衮、冕、黻、珽,带、裳、幅、舄,衡、紞、紘、綖,昭其度也。藻、率、鞞、革□,鞶、厉、游、缨,昭其数也。火、龙、黼、黻,昭其文也。五色比象,昭其物也。锡、鸾、和、铃,昭其声也。三辰旂旗,昭其明也。夫德,俭而有度,登降有数。文、物以纪之,声、明以发之,以临照百官,百官于是乎戒惧,而不敢易纪律。今灭德立违,而置其赂器于大庙,以明示百官,百官象之,其又何诛焉?国家之败,由官邪也。官之失德,宠赂章也。郜鼎在庙,章孰甚焉?武王克商,迁九鼎于雒邑,义士犹或非之,而况将昭违乱之赂器于大庙,其若之何?」公不听。周内史闻之曰:``臧孙达其有后于鲁乎!君违不忘谏之以德。」

秋七月,杞侯来朝,不敬,杞侯归,乃谋伐之。

蔡侯、郑伯会于邓,始惧楚也。

九月,入杞,讨不敬也。

公及戎盟于唐,修旧好也。

冬,公至自唐,告于庙也。凡公行,告于宗庙;反行,饮至、舍爵,策勋焉,礼也。

特相会,往来称地,让事也。自参以上,则往称地,来称会,成事也。

初,晋穆侯之夫人姜氏以条之役生太子,命之曰仇。其弟以千亩之战生,命之曰成师。师服曰:``异哉,君之名子也!夫名以制义,义以出礼,礼以体政,政以正民。是以政成而民听,易则生乱。嘉耦曰妃。怨耦曰仇,古之命也。今君命大子曰仇,弟曰成师,始兆乱矣,兄其替乎?」

惠之二十四年,晋始乱,故封桓叔于曲沃,靖侯之孙栾宾傅之。师服曰:``吾闻国家之立也,本大而末小,是以能固。故天子建国,诸侯立家,卿置侧室,大夫有贰宗,士有隶子弟,庶人、工、商,各有分亲,皆有等衰。是以民服事其上而下无觊觎。今晋,甸侯也,而建国。本既弱矣,其能久乎?」

惠之三十年,晋潘父弑昭侯而立桓叔,不克。晋人立孝侯。

惠之四十五年,曲沃庄伯伐翼,弑孝侯。翼人立其弟鄂侯。鄂侯生哀侯。哀侯侵陉庭之田。陉庭南鄙启曲沃伐翼。

\hypertarget{header-n186}{%
\subsubsection{桓公三年}\label{header-n186}}

【经】三年春正月,公会齐侯于嬴。夏,齐侯、卫侯胥命于蒲。六月,公会杞侯于郕。秋七月壬辰朔,日有食之,既。公子翬如齐逆女。九月,齐侯送姜氏于欢。公会齐侯于欢。夫人姜氏至自齐。冬,齐侯使其弟年来聘。有年。

【传】三年春,曲沃武公伐翼,次于陉庭,韩万御戎,梁弘为右,逐翼侯于汾隰,骖絓而止。夜获之,及栾共叔。

会于嬴,成昏于齐也。

夏,齐侯、卫侯胥命于蒲,不盟也。

公会杞侯于欢,杞求成也。

秋,公子翬如齐逆女。修先君之好。故曰``公子」。

齐侯送姜氏于欢,非礼也。凡公女嫁于敌国,姊妹则上卿送之,以礼于先君,公子则下卿送之。于大国,虽公子亦上卿送之。于天子,则诸卿皆行,公不自送。于小国,则上大夫送之。

冬,齐仲年来聘,致夫人也。

芮伯万之母芮姜恶芮伯之多宠人也,故逐之,出居于魏。

\hypertarget{header-n198}{%
\subsubsection{桓公四年}\label{header-n198}}

【经】四年春正月,公狩于郎。夏,天王使宰渠伯纠来聘。

【传】四年春正月,公狩于郎。书,时,礼也。

夏,周宰渠伯纠来聘。父在,故名。

秋,秦师侵芮,败焉,小之也。

冬,王师、秦师围魏,执芮伯以归。

\hypertarget{header-n206}{%
\subsubsection{桓公五年}\label{header-n206}}

【经】五年春正月,甲戌、己丑,陈侯鲍卒。夏,齐侯郑伯如纪。天王使仍叔之子来聘。葬陈桓公。城祝丘。秋,蔡人、卫人、陈人从王伐郑。大雩。螽。冬,州公如曹。

【传】五年春正月,甲戌,己丑,陈侯鲍卒,再赴也。于是陈乱,文公子佗杀大子免而代之。公疾病而乱作,国人分散,故再赴。

夏,齐侯、郑伯朝于纪,欲以袭之。纪人知之。

王夺郑伯政,郑伯不朝。

秋,王以诸侯伐郑,郑伯御之。

王为中军;虢公林父将右军,蔡人、卫人属焉;周公黑肩将左军,陈人属焉。

郑子元请为左拒以当蔡人、卫人,为右拒以当陈人,曰:``陈乱,民莫有斗心,若先犯之,必奔。王卒顾之,必乱。蔡、卫不枝,固将先奔,既而萃于王卒,可以集事。」从之。曼伯为右拒,祭仲足为左拒,原繁、高渠弥以中军奉公,为鱼丽之陈,先偏后伍,伍承弥缝。战于繻葛,命二拒曰:``旝动而鼓。」蔡、卫、陈皆奔,王卒乱,郑师合以攻之,王卒大败。祝聃射王中肩,王亦能军。祝聃请从之。公曰:``君子不欲多上人,况敢陵天子乎!苟自救也,社稷无陨,多矣。」

夜,郑伯使祭足劳王,且问左右。

仍叔之子,弱也。

秋,大雩,书,不时也。凡祀,启蛰而郊,龙见而雩,始杀而尝,闭蛰而烝。过则书。

冬,淳于公如曹。度其国危,遂不复。

\hypertarget{header-n220}{%
\subsubsection{桓公六年}\label{header-n220}}

【经】六年春正月,实来。夏四月,公会纪侯于成。秋八月壬午,大阅。蔡人杀陈佗。九月丁卯,子同生。冬,纪侯来朝。

【传】六年春,自曹来朝。书曰``实来」,不复其国也。

楚武王侵随,使薳章求成焉。军于瑕以待之。随人使少师董成。斗伯比言于楚子曰:``吾不得志于汉东也,我则使然。我张吾三军而被吾甲兵,以武临之,彼则惧而协以谋我,故难间也。汉东之国随为大,随张必弃小国,小国离,楚之利也。少师侈,请羸师以张之。」熊率且比曰:``季梁在,何益?」斗伯比曰:``以为后图,少师得其君。」王毁军而纳少师。

少师归,请追楚师,随侯将许之。季梁止之曰:``天方授楚,楚之蠃,其诱我也,君何急焉?臣闻小之能敌大也,小道大淫。所谓道,忠于民而信于神也。上思利民,忠也;祝史正辞,信也。今民馁而君逞欲,祝史矫举以祭,臣不知其可也。」公曰:``吾牲牷肥腯,粢盛丰备,何则不信?」对曰:``夫民,神之主也。是以圣王先成民而后致力于神。故奉牲以告曰『博硕肥腯』,谓民力之普存也,谓其畜之硕大蕃滋也,谓其不疾瘯蠡也,谓其备腯咸有也。奉盛以告曰『洁粢丰盛』,谓其三时不害而民和年丰也。奉酒醴以告曰『嘉栗旨酒』,谓其上下皆有嘉德而无违心也。所谓馨香,无谗慝也。故务其三时,修其五教,亲其九族,以致其禋祀。于是乎民和而神降之福,故动则有成。今民各有心,而鬼神乏主,君虽独丰,其何福之有!君姑修政而亲兄弟之国,庶免于难。」随侯惧而修政,楚不敢伐。

夏,会于成,纪来咨谋齐难也。

北戎伐齐,齐侯使乞师于郑。郑大子忽帅师救齐。六月,大败戎师,获其二帅大良、少良,甲首三百,以献于齐。于是,诸侯之大夫戍齐,齐人馈之饩,使鲁为其班,后郑。郑忽以其有功也,怒,故有郎之师。

公之未昏于齐也,齐侯欲以文姜妻郑大子忽。大子忽辞,人问其故,大子曰:``人各有耦,齐大,非吾耦也。《诗》云:『自求多福。』在我而已,大国何为?」君子曰:``善自为谋。」及其败戎师也,齐侯又请妻之,固辞。人问其故,大子曰:``无事于齐,吾犹不敢。今以君命奔齐之急,而受室以归,是以师昏也。民其谓我何?」遂辞诸郑伯。

秋,大阅,简车马也。

九月丁卯,子同生,以大子生之礼举之,接以大牢,卜士负之,士妻食之。公与文姜、宗妇命之。

公问名于申繻。对曰:``名有五,有信,有义,有象,有假,有类。以名生为信,以德命为义,以类命为象,取于物为假,取于父为类。不以国,不以官,不以山川,不以隐疾,不以畜牲,不以器币。周人以讳事神,名,终将讳之。故以国则废名,以官则废职,以山川则废主,以畜牲则废祀,以器币则废礼。晋以僖侯废司徒,宋以武公废司空,先君献,武废二山,是以大物不可以命。」公曰:``是其生也,与吾同物,命之曰同。」

冬,纪侯来朝,请王命以求成于齐,公告不能。

\hypertarget{header-n234}{%
\subsubsection{桓公七年}\label{header-n234}}

【经】七年春二月己亥,焚咸丘。夏,谷伯绥来朝。邓侯吾离来朝。

【传】七年春,谷伯、邓侯来朝。名,贱之也。

夏,盟、向求成于郑,既而背之。

秋,郑人、齐人、卫人伐盟、向。王迁盟、向之民于郏。

冬,曲沃伯诱晋小子侯,杀之。

\hypertarget{header-n242}{%
\subsubsection{桓公八年}\label{header-n242}}

【经】八年春正月己卯,烝。天王使家父来聘。夏五月丁丑,烝秋,伐邾。冬十月,雨雪。祭公来,遂逆王后于纪。

【传】八年春,灭翼。

随少师有宠。楚斗伯比曰:``可矣。仇有衅,不可失也。」

夏,楚子合诸侯于沈鹿。黄、随不会,使薳章让黄。楚子伐随,军于汉、淮之间。

季梁请下之:``弗许而后战,所以怒我而怠寇也。」少师谓随侯曰:``必速战。不然,将失楚师。」随侯御之,望楚师。季梁曰:``楚人上左,君必左,无与王遇。且攻其右,右无良焉,必败。偏败,众乃携矣。」少师曰:``不当王,非敌也。」弗从。战于速杞,随师败绩。随侯逸,斗丹获其戎车,与其戎右少师。

秋,随及楚平。楚子将不许,斗伯比曰:``天去其疾矣,随未可克也。」乃盟而还。

冬,王命虢仲立晋哀侯之弟缗于晋。

祭公来,遂逆王后于纪,礼也。

\hypertarget{header-n253}{%
\subsubsection{桓公九年}\label{header-n253}}

【经】九年春,纪季姜归于京师。夏四月,秋七月。冬,曹伯使其世子射姑来朝。

【传】九年春,纪季姜归于京师。凡诸侯之女行,唯王后书。

巴子使韩服告于楚,请与邓为好。楚子使道朔将巴客以聘于邓。邓南鄙郁人攻而夺之币,杀道朔及巴行人。楚子使薳章让于邓,邓人弗受。

夏,楚使斗廉帅师及巴师围郁。邓养甥、聃甥帅师郁救。三逐巴师,不克。斗廉衡陈其师于巴师之中,以战,而北。邓人逐之,背巴师而夹攻之。邓师大败,郁人宵溃。

秋,虢仲、芮伯、梁伯、荀侯、贾伯伐曲沃。

冬,曹大子来朝,宾之以上卿,礼也。享曹大子,初献,乐奏而叹。施父曰:``曹大子其有忧乎?非叹所也。」

\hypertarget{header-n262}{%
\subsubsection{桓公十年}\label{header-n262}}

【经】十年春王正月,庚申,曹伯终生卒。夏五月,葬曹桓公。秋,公会卫侯于桃丘,弗遇。冬十有二月丙午,齐侯、卫侯、郑伯来战于郎。

【传】十年春,曹桓公卒。

虢仲谮其大夫詹父于王。詹父有辞,以王师伐虢。夏,虢公出奔虞。

秋,秦人纳芮伯万于芮。

初,虞叔有玉,虞公求旃。弗献。既而悔之。曰:``周谚有之:『匹夫无罪,怀璧其罪。』吾焉用此,其以贾害也?」乃献。又求其宝剑。叔曰:``是无厌也。无厌,将及我。」遂伐虞公,故虞公出奔共池。

冬,齐、卫、郑来战于郎,我有辞也。

初,北戎病齐,诸侯救之。郑公子忽有功焉。齐人饩诸侯,使鲁次之。鲁以周班后郑。郑人怒,请师于齐。齐人以卫师助之。故不称侵伐。先书齐、卫,王爵也。

\hypertarget{header-n272}{%
\subsubsection{桓公十一年}\label{header-n272}}

【经】十有一年春正月,齐人、卫人、郑人盟于恶曹。夏五月癸未,郑伯寤生卒。秋七月,葬郑庄公。九月,宋人执郑祭仲。突归于郑。郑忽出奔卫。柔会宋公、陈侯、蔡叔盟于折。公会宋公于夫钟。冬十月有二月,公会宋公于阚。

【传】十一年春,齐、卫、郑、宋盟于恶曹。

楚屈瑕将盟贰、轸。郧人军于蒲骚,将与随、绞、州、蓼伐楚师。莫敖患之。斗廉曰:``郧人军其郊,必不诫,且日虞四邑之至也。君次于郊郢,以御四邑。我以锐师宵加于郧,郧有虞心而恃其城,莫有斗志。若败郧师,四邑必离。」莫敖曰:``盍请济师于王?」对曰:``师克在和,不在众。商、周之不敌,君之所闻也。成军以出,又何济焉?」莫敖曰:``卜之?」对曰:``卜以决疑,不疑何卜?」遂败郧师于蒲骚,卒盟而还。郑昭公之败北戎也,齐人将妻之,昭公辞。祭仲曰:``必取之。君多内宠,子无大援,将不立。三公子皆君也。」弗从。

夏,郑庄公卒。

初,祭封人仲足有宠于庄公,庄公使为卿。为公娶邓曼,生昭公,故祭仲立之。宋雍氏女于郑庄公,曰雍姞,生厉公。雍氏宗有宠于宋庄公,故诱祭仲而执之,曰:``不立突,将死。」亦执厉公而求赂焉。祭仲与宋人盟,以厉公归而立之。

秋九月丁亥,昭公奔卫。己亥,厉公立。

\hypertarget{header-n281}{%
\subsubsection{桓公十二年}\label{header-n281}}

【经】十有二年春正月。夏六月壬寅,公会杞侯、莒子盟于曲池。秋七月丁亥,公会宋公、燕人盟于谷丘。八月壬辰,陈侯跃卒。公会宋公于虚。冬十有一月,公会宋公于龟。丙戌,公会郑伯,盟于武父。丙戌,卫侯晋卒。十有二月,及郑师伐宋。丁未,战于宋。

【传】十二年夏,盟于曲池,平杞、莒也。

公欲平宋、郑。秋,公及宋公盟于句渎之丘。宋成未可知也,故又会于虚。冬,又会于龟。宋公辞平,故与郑伯盟于武父。遂帅师而伐宋,战焉,宋无信也。

君子曰:``苟信不继,盟无益也。《诗》云:『君子屡盟,乱是用长。』无信也。」

楚伐绞,军其南门。莫敖屈瑕曰:``绞小而轻,轻则寡谋,请无扞采樵者以诱之。」从之。绞人获三十人。明日,绞人争出,驱楚役徒于山中。楚人坐其北门,而覆诸山下,大败之,为城下之盟而还。

伐绞之役,楚师分涉于彭。罗人欲伐之,使伯嘉谍之,三巡数之。

\hypertarget{header-n290}{%
\subsubsection{桓公十三年 }\label{header-n290}}

【经】十有三年春二月,公会纪侯、郑伯。己巳,及齐侯、宋公、卫侯、燕人战。齐师、宋师、卫师、燕师败绩。三月,葬卫宣公。夏,大水。秋七月。冬十月。

【传】十三年春,楚屈瑕伐罗,斗伯比送之。还,谓其御曰:``莫敖必败。举趾高,心不固矣。」遂见楚子曰:``必济师。」楚子辞焉。入告夫人邓曼。邓曼曰:``大夫其非众之谓,其谓君抚小民以信,训诸司以德,而威莫敖以刑也。莫敖狃于蒲骚之役,将自用也,必小罗。君若不镇抚,其不设备乎?夫固谓君训众而好镇抚之,召诸司而劝之以令德,见莫敖而告诸天之不假易也。不然,夫岂不知楚师之尽行也?」楚子使赖人追之,不及。

莫敖使徇于师曰:``谏者有刑。」及鄢,乱次以济。遂无次,且不设备。及罗,罗与卢戎两军之。大败之。莫敖缢于荒谷,群帅囚于冶父以听刑。楚子曰:``孤之罪也。」皆免之。

宋多责赂于郑,郑不堪命。故以纪、鲁及齐与宋、卫、燕战。不书所战,后也。

郑人来请修好。

\hypertarget{header-n298}{%
\subsubsection{桓公十四年 }\label{header-n298}}

【经】十有四年春正月,公会郑伯于曹。无冰。夏五,郑伯使其弟语来盟。秋八月壬申,御廪灾。乙亥,尝。冬十有二月丁巳,齐侯禄父卒。宋人以齐人、蔡人、卫人、陈人伐郑。

【传】十四年春,会于曹。曹人致饩,礼也。

夏,郑子人来寻盟,且修曹之会。

秋八月壬申,御廪灾。乙亥,尝。书,不害也。

冬,宋人以诸侯伐郑,报宋之战也。焚渠门,入,及大逵。伐东郊,取牛首。以大宫之椽归,为卢门之椽。

\hypertarget{header-n306}{%
\subsubsection{桓公十五年}\label{header-n306}}

【经】十有五年春二月,天王使家父来求车。三月乙未,天王崩。夏四月己巳,葬齐僖公。五月,郑伯突出奔蔡。郑世子忽复归于郑。许叔入于许。公会齐侯于艾。邾人、牟人、葛人来朝。秋九月,郑伯突入于栎。冬十有一月,公会宋公、卫侯、陈侯于衰,伐郑。

【传】十五年春,天王使家父来求车,非礼也。诸侯不贡车、服,天子不私求财。

祭仲专,郑伯患之,使其婿雍纠杀之。将享诸郊。雍姬知之,谓其母曰:``父与夫孰亲?」其母曰:``人尽夫也,父一而已,胡可比也?」遂告祭仲曰:``雍氏舍其室而将享子于郊,吾惑之,以告。」祭仲杀雍纠,尸诸周氏之汪。公载以出,曰:``谋及妇人,宜其死也。」夏,厉公出奔蔡。

六月乙亥,昭公入。

许叔入于许。

公会齐侯于艾,谋定许也。

秋,郑伯因栎人杀檀伯,而遂居栎。

冬,会于衰,谋伐郑,将纳厉公也。弗克而还。

\hypertarget{header-n317}{%
\subsubsection{桓公十六年}\label{header-n317}}

【经】十有六年春正月,公会宋公、蔡侯、卫侯于曹。夏四月,公会宋公、卫侯、陈侯、蔡侯伐郑。秋七月,公至自伐郑。冬,城向。十有一月,卫侯朔出奔齐。

【传】十六年春正月,会于曹,谋伐郑也。

夏,伐郑。

秋七月,公至自伐郑,以饮至之礼也。

冬,城向,书,时也。

初,卫宣公烝于夷姜,生急子,属诸右公子。为之娶于齐,而美,公取之,生寿及朔,属寿于左公子。夷姜缢。宣姜与公子朔构急子。公使诸齐,使盗待诸莘,将杀之。寿子告之,使行。不可,曰:``弃父之命,恶用子矣!有无父之国则可也。」及行,饮以酒,寿子载其旌以先,盗杀之。急子至,曰:``我之求也。此何罪?请杀我乎!」又杀之。二公子故怨惠公。

十一月,左公子泄、右公子职立公子黔牟。惠公奔齐。

\hypertarget{header-n327}{%
\subsubsection{桓公十七年}\label{header-n327}}

【经】十有七年春正月丙辰,公会齐侯、纪侯盟于黄。二月丙午,公会邾仪父,盟于趡。夏五月丙午,及齐师战于奚。六月丁丑,蔡侯封人卒。秋八月,蔡季自陈归于蔡。癸巳,葬蔡桓侯。及宋人、卫人伐邾。冬十月朔,日有食之。

【传】十七年春,盟于黄,平齐、纪,且谋卫故也。

乃邾仪父盟于趡,寻蔑之盟也。

夏,及齐师战于奚,疆事也。于是齐人侵鲁疆,疆吏来告,公曰:``疆场之事,慎守其一,而备其不虞。姑尽所备焉。事至而战,又何谒焉?」

蔡桓侯卒。蔡人召蔡季于陈。

秋,蔡季自陈归于蔡,蔡人嘉之也。

伐邾,宋志也。

冬十月朔,日有食之。不书日,官失之也。天子有日官,诸侯有日御。日官居卿以底日,礼也。日御不失日,以授百官于朝。

初,郑伯将以高渠弥为卿,昭公恶之,固谏,不听,昭公立,惧其杀己也。辛卯,弑昭公,而立公子亹。

君子谓昭公知所恶矣。公子达曰:``高伯其为戮乎?复恶已甚矣。」

\hypertarget{header-n340}{%
\subsubsection{桓公十八年}\label{header-n340}}

【经】十有八年春王正月,公会齐侯于泺。公与夫人姜氏遂如齐。夏四月丙子,公薨于齐。丁酉,公之丧至自齐。秋七月,冬十有二月己丑,葬我君桓公。

【传】十八年春,公将有行,遂与姜氏如齐。申繻曰:``女有家,男有室,无相渎也,谓之有礼。易此,必败。」

公会齐侯于泺,遂及文姜如齐。齐侯通焉。公谪之,以告。

夏四月丙子,享公。使公子彭生乘公,公薨于车。

鲁人告于齐曰:``寡君畏君之威,不敢宁居,来修旧好,礼成而不反,无所归咎,恶于诸侯。请以彭生除之。」齐人杀彭生。

秋,齐侯师于首止;子亹会之,高渠弥相。七月戊戌,齐人杀子亹而轘高渠弥,祭仲逆郑子于陈而立之。是行也,祭仲知之,故称疾不往。人曰:``祭仲以知免。」仲曰:``信也。」

周公欲弑庄王而立王子克。辛伯告王,遂与王杀周公黑肩。王子克奔燕。

初,子仪有宠于桓王,桓王属诸周公。辛伯谏曰:``并后、匹嫡、两政、耦国,乱之本也。」周公弗从,故及。

\hypertarget{header-n353}{%
\subsection{庄公}\label{header-n353}}

\begin{center}\rule{0.5\linewidth}{\linethickness}\end{center}

\hypertarget{header-n355}{%
\subsubsection{庄公元年}\label{header-n355}}

【经】元年春王正月。三月,夫人孙于齐。夏,单伯送王姬。秋,筑王姬之馆于外。冬十月乙亥,陈侯林卒。王使荣叔来锡桓公命。王姬归于齐。齐师迁纪、郱、鄑、郚。

【传】元年春,不称即位,文姜出故也。

三月,夫人孙于齐。不称姜氏,绝不为亲,礼也。

秋,筑王姬之馆于外。为外,礼也。

\hypertarget{header-n362}{%
\subsubsection{庄公二年 }\label{header-n362}}

【经】二年春王二月,葬陈庄公。夏,公子庆父帅师伐于余丘。秋七月,齐王姬卒。冬十有二月,夫人姜氏会齐侯于禚。乙酉,宋公冯卒。

【传】二年冬,夫人姜氏会齐侯于禚。书,奸也。

\hypertarget{header-n367}{%
\subsubsection{庄公三年}\label{header-n367}}

【经】三年春王正月,溺会齐师伐卫。夏四月,葬宋庄公。五月,葬桓王。秋,纪季以酅入于齐。冬,公次于滑。

【传】三年春,溺会齐师伐卫,疾之也。

夏五月,葬桓王,缓也。

秋,纪季以酅入于齐,纪于是乎始判。

冬,公次于滑,将会郑伯,谋纪故也。郑伯辞以难。凡师,一宿为舍,再宿为信,过信为次。

\hypertarget{header-n375}{%
\subsubsection{庄公四年}\label{header-n375}}

【经】四年春王二月,夫人姜氏享齐侯于祝丘。三月,纪伯姬卒。夏,齐侯、陈侯、郑伯遇于垂。纪侯大去其国。六月乙丑,齐侯葬纪伯姬。秋七月。冬,公及齐人狩于禚。

【传】四年春,王三月,楚武王荆尸,授师孑焉,以伐随,将齐,入告夫人邓曼曰:``余心荡。」邓曼叹曰:``王禄尽矣。盈而荡,天之道也。先君其知之矣,故临武事,将发大命,而荡王心焉。若师徒无亏,王薨于行,国之福也。」王遂行,卒于樠木之下。令尹斗祁、莫敖屈重除道、梁溠,营军临随。随人惧,行成。莫敖以王命入盟随侯,且请为会于汉汭,而还。济汉而后发丧。

纪侯不能下齐,以与纪季。夏,纪侯大去其国,违齐难也。

\hypertarget{header-n381}{%
\subsubsection{庄公五年}\label{header-n381}}

【经】五年春王正月。夏,夫人姜氏如齐师。秋,郳犁来来朝。冬,公会齐人、宋人、陈人、蔡人伐卫。

【传】五年秋,郳犁来来朝,名,未王命也。

冬,伐卫纳惠公也。

\hypertarget{header-n387}{%
\subsubsection{庄公六年}\label{header-n387}}

【经】六年春王正月,王人子突救卫。夏六月,卫侯朔入于卫。秋,公至自伐卫。螟。冬,齐人来归卫俘。

【传】六年春,王人救卫。

夏,卫侯入,放公子黔牟于周,放宁跪于秦,杀左公子泄、右公子职,乃即位。

君子以二公子之立黔牟为不度矣。夫能固位者,必度于本末而后立衷焉。不知其本,不谋。知本之不枝,弗强。《诗》云:``本枝百世。」

冬,齐人来归卫宝,文姜请之也。

楚文王伐申,过邓。邓祁侯曰:``吾甥也。」止而享之。骓甥、聃甥、养甥请杀楚子,邓侯弗许。三甥曰:``亡邓国者,必此人也。若不早图,后君噬齐。其及图之乎?图之,此为时矣。」邓侯曰:``人将不食吾余。」对曰:``若不从三臣,抑社稷实不血食,而君焉取余?」弗从。还年,楚子伐邓。十六年,楚复伐邓,灭之。

\hypertarget{header-n396}{%
\subsubsection{庄公七年}\label{header-n396}}

【经】七年春,夫人姜氏会齐侯于防。夏四月辛卯,夜,恒星不见。夜中,星陨如雨。秋,大水。无麦、苗。冬,夫人姜氏会齐侯于谷。

【传】七年春,文姜会齐侯于防,齐志也。

夏,恒星不见,夜明也。星陨如雨,与雨偕也。

秋,无麦苗,不害嘉谷也。

\hypertarget{header-n403}{%
\subsubsection{庄公八年}\label{header-n403}}

【经】八年春王正月,师次于郎,以俟陈人,蔡人。甲午,治兵。夏,师及齐师围郕,郕降于齐师。秋,师还。冬十有一月癸未,齐无知弑其君诸儿。

【传】八年春,治兵于庙,礼也。

夏,师及齐师围郕。郕降于齐师。仲庆父请伐齐师。公曰:``不可。我实不德,齐师何罪?罪我之由。《夏书》曰:『皋陶迈种德,德,乃降。』姑务修德以待时乎。」秋,师还。君子是以善鲁庄公。

齐侯使连称、管至父戍葵丘。瓜时而往,曰:``及瓜而代。」期戍,公问不至。请代,弗许。故谋作乱。

僖公之母弟曰夷仲年,生公孙无知,有宠于僖公,衣服礼秩如适。襄公绌之。二人因之以作乱。连称有从妹在公宫,无宠,使间公,曰:``捷,吾以女为夫人。」

冬十二月,齐侯游于姑棼,遂田于贝丘。见大豕,从者曰:``公子彭生也。」公怒曰:``彭生敢见!」射之,豕人立而啼。公惧,坠于车,伤足丧屦。反,诛屦于徒人费。弗得,鞭之,见血。走出,遇贼于门,劫而束之。费曰:``我奚御哉!」袒而示之背,信之。费请先入,伏公而出,斗,死于门中。石之纷如死于阶下。遂入,杀孟阳于床。曰:``非君也,不类。」见公之足于户下,遂弑之,而立无知。

初、襄公立,无常。鲍叔牙曰:``君使民慢,乱将作矣。」奉公子小白出奔莒。乱作,管夷吾、召忽奉公子纠来奔。

初,公孙无知虐于雍廪。

\hypertarget{header-n414}{%
\subsubsection{庄公九年}\label{header-n414}}

【经】九年春,齐人杀无知。公及齐大夫盟于既。夏,公伐齐纳子纠。齐小白入于齐。秋七月丁酉,葬齐襄公。八月庚申,及齐师战于乾时,我师败绩。九月,齐人取子纠杀之。冬,浚洙。

【传】九年春,雍廪杀无知。

公及齐大夫盟于既,齐无君也。

夏,公伐齐,纳子纠。桓公自莒先入。

秋,师及齐师战于乾时,我师败绩,公丧戎路,传乘而归。秦子、梁子以公旗辟于下道,是以皆止。

鲍叔帅师来言曰:``子纠,亲也,请君讨之。管、召、仇也,请受而甘心焉。」乃杀子纠于生窦,召忽死之。管仲请囚,鲍叔受之,乃堂阜而税之。归而以告曰:``管夷吾治于高傒,使相可也。」公从之。

\hypertarget{header-n423}{%
\subsubsection{庄公十年}\label{header-n423}}

【经】十年春王正月,公败齐师于长勺。二月,公侵宋。三月,宋人迁宿。夏六月,齐师、宋师次于郎。公败宋师于乘丘。秋九月,荆败蔡师于莘,以蔡侯献舞归。冬十月,齐师灭谭,谭子奔莒。

【传】十年春,齐师伐我。公将战,曹刿请见。其乡人曰:``肉食者谋之,又何间焉。刿曰:``肉食者鄙,未能远谋。」乃入见。问何以战。公曰:``衣食所安,弗敢专也,必以分人。」对曰:``小惠未遍,民弗从也。」公曰:``牺牲玉帛,弗敢加也,必以信。」对曰:``小信未孚,神弗福也。」公曰:``小大之狱,虽不能察,必以情。」对曰:``忠之属也,可以一战,战则请从。」
公与之乘。战于长勺。公将鼓之。刿曰;``未可。」齐人三鼓,刿曰:``可矣。」齐师败绩。公将驰之。刿曰:``未可。」下,视其辙,登,轼而望之,曰:``可矣。」遂逐齐师。

既克,公问其故。对曰:``夫战,勇气也,一鼓作气,再而衰,三而竭。彼竭我盈,故克之。夫大国难测也,惧有伏焉。吾视其辙乱,望其旗靡,故逐之。」

夏六月,齐师、宋师次于郎。公子偃曰:``宋师不整,可败也。宋败,齐必还,请击之。」公弗许。自雩门窃出,蒙皋比而先犯之。公从之。大败宋师于乘丘。齐师乃还。

蔡哀侯娶于陈,息侯亦娶焉。息妫将归,过蔡。蔡侯曰:``吾姨也。」止而见之,弗宾。息侯闻之,怒,使谓楚文王曰:``伐我,吾求救于蔡而伐之。」楚子从之。秋九月,楚败蔡师于莘,以蔡侯献舞归。

齐侯之出也,过谭,谭不礼焉。及其入也,诸侯皆贺,谭又不至。冬,齐师灭谭,谭无礼也。谭子奔莒,同盟故也。

\hypertarget{header-n432}{%
\subsubsection{庄公十一年}\label{header-n432}}

【经】十有一年春王正月。夏五月,戊寅,公败宋师于鄑。秋,宋大水。冬,王姬归于齐。

【传】十一年夏,宋为乘丘之役故侵我。公御之,宋师未陈而薄之,败诸鄑。

凡师,敌未陈曰败某师,皆陈曰战,大崩曰败绩,得人隽曰克,覆而败之曰取某师,京师败曰王师败绩于某。

秋,宋大水。公使吊焉,曰:``天作淫雨,害于粢盛,若之何不吊?」对曰:``孤实不敬,天降之灾,又以为君忧,拜命之辱。」臧文仲曰:``宋其兴乎。禹、汤罪己,其兴也悖焉、桀、纣罪人,其亡也忽焉。且列国有凶称孤,礼也。言惧而名礼,其庶乎。」既而闻之曰公子御说之辞也。臧孙达曰:``是宜为君,有恤民之心。」

冬,齐侯来逆共姬。

乘丘之役,公之金仆姑射南宫长万,公右遄孙生搏之。宋人请之,宋公靳之,曰:``始吾敬子,今子,鲁囚也。吾弗敬子矣。」病之。

\hypertarget{header-n441}{%
\subsubsection{庄公十二年}\label{header-n441}}

【经】十有二年春王三月,纪叔姬归于酅。夏四月。秋八月甲午,宋万弑其君捷及其大夫仇牧。十月,宋万出奔陈。

【传】十二年秋,宋万弑闵公于蒙泽。遇仇牧于门,批而杀之。遇大宰督于东宫之西,又杀之。立子游。群公子奔萧。公子御说奔亳。南宫牛、猛获帅师围亳。

冬十月,萧叔大心及戴、武、宣、穆、庄之族以曹师伐之。杀南宫牛于师,杀子游于宋,立桓公。猛获奔卫。南宫万奔陈,以乘车辇其母,一日而至。

宋人请猛获于卫,卫人欲勿与,石祁子曰:``不可。天下之恶一也,恶于宋而保于我,保之何补?得一夫而失一国,与恶而弃好,非谋也。」卫人归之。亦请南宫万于陈,以赂。陈人使妇人饮之酒,而以犀革裹之。比及宋手足皆见。宋人皆醢之。

\hypertarget{header-n448}{%
\subsubsection{庄公十三年}\label{header-n448}}

【经】十有三年春,齐侯、宋人、陈人、蔡人、邾人会于北杏。夏六月,齐人灭遂。秋七月。冬,公会齐侯盟于柯。

【传】十三年春,会于北杏,以平宋乱。遂人不至。

夏,齐人灭遂而戍之。

冬,盟于柯,始及齐平也。

宋人背北杏之会。

\hypertarget{header-n456}{%
\subsubsection{庄公十四年}\label{header-n456}}

【经】十有四年春,齐人、陈人、曹人伐宋。夏,单伯会伐宋。秋七月,荆入蔡。冬,单伯会齐侯、宋公、卫侯、郑伯于鄄。

【传】十四年春,诸侯伐宋,齐请师于周。夏,单伯会之,取成于宋而还。

郑厉公自栎侵郑,及大陵,获傅瑕。傅瑕曰:``苟舍我,吾请纳君。」与之盟而赦之。六月甲子,傅瑕杀郑子及其二子而纳厉公。

初,内蛇与外蛇斗于郑南门中,内蛇死。六年而厉公入。公闻之,问于申繻曰:``犹有妖乎?」对曰:``人之所忌,其气焰以取之,妖由人兴也。人无衅焉,妖不自作。人弃常则妖兴,故有妖。」
厉公入,遂杀傅瑕。使谓原繁曰:``傅瑕贰,周有常刑,既伏其罪矣。纳我而无二心者,吾皆许之上大夫之事,吾愿与伯父图之。且寡人出,伯父无里言,入,又不念寡人,寡人憾焉。」对曰:``先君桓公命我先人典司宗祏。社稷有主而外其心,其何贰如之?苟主社稷,国内之民其谁不为臣?臣无二心,天之制也。子仪在位十四年矣,而谋召君者,庸非二乎。庄公之子犹有八人,若皆以官爵行赂劝贰而可以济事,君其若之何?臣闻命矣。」乃缢而死。

蔡哀侯为莘故,绳息妫以语楚子。楚子如息,以食入享,遂灭息。以息妫归,生堵敖及成王焉,未言。楚子问之,对曰:``吾一妇人而事二夫,纵弗能死,其又奚言?」楚子以蔡侯灭息,遂伐蔡。秋七月,楚入蔡。

君子曰:``《商书》所谓『恶之易也,如火之燎于原,不可乡迩,其犹可扑灭』者,其如蔡哀侯乎。」

冬,会于鄄,宋服故也。

\hypertarget{header-n466}{%
\subsubsection{庄公十五年}\label{header-n466}}

【经】十有五年春,齐侯、宋公、陈侯、卫侯、郑伯会于鄄。夏,夫人姜氏如齐。秋,宋人、齐人、邾人伐郳。郑人侵宋。冬十月。

【传】十五年春,复会焉,齐始霸也。

秋,诸侯为宋伐郳。郑人间之而侵宋。

\hypertarget{header-n472}{%
\subsubsection{庄公十六年}\label{header-n472}}

【经】十有六年春王正月。夏,宋人、齐人、卫人伐郑。秋,荆伐郑。冬十有二月,会齐侯、宋公、陈侯、卫侯、郑伯、许男、滑伯、滕子同盟于幽。邾子克卒。

【传】十六年夏,诸侯伐郑,宋故也。

郑伯自栎入,缓告于楚。秋,楚伐郑,及栎,为不礼故也。

郑伯治与于雍纠之乱者。九月,杀公子阏,刖强鉏。公父定叔出奔卫。三年而复之,曰:``不可使共叔无后于郑。」使以十月入,曰:``良月也,就盈数焉。」

君子谓:``强鉏不能卫其足。」

冬,同盟于幽,郑成也。

王使虢公命曲沃伯以一军为晋侯。

初,晋武公伐夷,执夷诡诸。蒍国请而免之。既而弗报。故子国作乱,谓晋人曰:``与我伐夷而取其地。」遂以晋师伐夷,杀夷诡诸。周公忌父出奔虢。惠王立而复之。

\hypertarget{header-n483}{%
\subsubsection{庄公十七年}\label{header-n483}}

【经】十有七年春,齐人执郑詹。夏,齐人歼于遂。秋,郑詹自齐逃来。冬,多麋。

【传】十七年春,齐人执郑詹,郑不朝也。

夏,遂因氏,颌氏、工娄氏、须遂氏飨齐戍,醉而杀之,齐人歼焉。

\hypertarget{header-n489}{%
\subsubsection{庄公十八年}\label{header-n489}}

【经】十有八年春王三月,日有食之。夏,公追戎于济西。秋,有《或虫》。冬十月。

【传】十八年春,虢公、晋侯朝王,王飨醴,命之宥,皆赐玉五珏,马三匹。非礼也。王命诸侯,名位不同,礼亦异数,不以礼假人。

虢公、晋侯、郑伯使原庄公逆王后于陈。陈妫归于京师,实惠后。

夏,公追戎于济西。不言其来,讳之也。

秋,有蜮,为灾也。

初,楚武王克权,使斗缗尹之。以叛,围而杀之。迁权于那处,使阎敖尹之。及文王即位,与巴人伐申而惊其师。巴人叛楚而伐那处,取之,遂门于楚。阎敖游涌而逸。楚子杀之,其族为乱。冬,巴人因之以伐楚。

\hypertarget{header-n498}{%
\subsubsection{庄公十九年 }\label{header-n498}}

【经】十有九年春王正月。夏四月。秋,公子结媵陈人之妇于鄄,遂及齐侯、宋公盟。夫人姜氏如莒。冬,齐人、宋人、陈人伐我西鄙。

【传】十九年春,楚子御之,大败于津。还,鬻拳弗纳。送伐黄,败黄师于碏陵。还,及湫,有疾。夏六月庚申卒,鬻拳葬诸夕室,亦自杀也,而葬于絰

初,鬻拳强谏楚子,楚子弗从,临之以兵,惧而从之。鬻拳曰:``吾惧君以皇。兵,罪莫大焉。」遂自刖也。楚人以为大阍,谓之大伯,使其后掌之。君子``鬻拳可谓爱君矣,谏以自纳于刑,刑犹不忘纳君于善。」

初,王姚嬖于庄王,生子颓。子颓有宠,蒍国为之师。及惠王即位。取蒍国之圃以为囿,边伯之宫近于王宫,王取之。王夺子禽祝跪与詹父田,而收膳夫之秩。故蒍国、边伯、石速、詹父、子禽祝跪作乱,因苏氏。秋,五大夫奉子颓以伐王,不克,出奔温。苏子奉子颓以奔卫。卫师、燕师伐周。冬,立子颓。

\hypertarget{header-n505}{%
\subsubsection{庄公二十年 }\label{header-n505}}

【经】二十年春王二月,夫人姜氏如莒。夏,齐大灾。秋七月。冬,齐人伐戎。

【传】二十年春,郑伯和王室,不克。执燕仲父。夏,郑伯遂以王归,王处于栎。秋,王及郑伯入于邬。遂入成周,取其宝器而还。

冬,王子颓享五大夫,乐及遍舞。郑伯闻之,见虢叔,曰:``寡人闻之,哀乐失时,殃咎必至。今王子颓歌舞不倦,乐祸也。夫司寇行戮,君为之不举,而况敢乐祸乎!奸王之位,祸孰大焉?临祸忘忧,忧必及之。盍纳王乎?」虢公曰:``寡人之愿也。」

\hypertarget{header-n511}{%
\subsubsection{庄公二十一年}\label{header-n511}}

【经】二十有一年春,王正月。夏五月辛酉,郑伯突卒。秋七月戊戌,夫人姜氏薨。冬十有二月,葬郑厉公。

【传】二十一年春,胥命于弭。夏,同伐王城。郑伯将王,自圉门入,虢叔自北门入,杀王子颓及五大夫。郑伯享王于阙西辟,乐备。王与之武公之略,自虎牢以东。原伯曰:``郑伯效尤,其亦将有咎。」五月,郑厉公卒。

王巡虢守。虢公为王宫于玤,王与之酒泉。郑伯之享王也,王以后之鞶鉴予之。虢公请器,王予之爵。郑伯由是始恶于王。

冬,王归自虢。

\hypertarget{header-n518}{%
\subsubsection{庄公二十二年}\label{header-n518}}

【经】二十二年春王正月,肆大眚。癸丑,葬我小君文姜。陈人杀其公子御寇。夏五月。秋七月丙申,及齐高傒盟于防。冬,公如齐纳币。

【传】二十二年春,陈人杀其大子御寇,陈公子完与颛孙奔齐。颛孙自齐来奔。

齐侯使敬仲为卿。辞曰:``羁旅之臣,幸若获宥,及于宽政,赦其不闲于教训而免于罪戾,弛于负担,君之惠也,所获多矣。敢辱高位,以速官谤。请以死告。《诗》云:『翘翘车乘,招我以弓,岂不欲往,畏我友朋。』」使为工正。

饮桓公酒,乐。公曰:``以火继之。」辞曰:``臣卜其昼,未卜其夜,不敢。」君子曰:``酒以成礼,不继以淫,义也。以君成礼,弗纳于淫,仁也。」

初,懿氏卜妻敬仲,其妻占之,曰:``吉,是谓『凤皇于飞,和鸣锵锵,有妫之后,将育于姜。五世其昌,并于正卿。八世之后,莫之与京。』」陈厉公,蔡出也。故蔡人杀五父而立之,生敬仲。其少也。周史有以《周易》见陈侯者,陈侯使筮之,遇《观》之《否》。曰:``是谓『观国之光,利用宾于王。』代陈有国乎。不在此,其在异国;非此其身,在其子孙。光,远而自他有耀者也。《坤》,土也。《巽》,风也。《乾》,天也。风为天于土上,山也。有山之材而照之以天光,于是乎居土上,故曰:『观国之光,利用宾于王。』庭实旅百,奉之以玉帛,天地之美具焉,故曰:『利用宾于王。』犹有观焉,故曰其在后乎。风行而着于土,故曰其在异国乎。若在异国,必姜姓也。姜,大岳之后也。山岳则配天,物莫能两大。陈衰,此其昌乎。」

及陈之初亡也,陈桓子始大于齐。其后亡成,成子得政。

\hypertarget{header-n527}{%
\subsubsection{庄公二十三年 }\label{header-n527}}

【经】二十有三年春,公至自齐。祭叔来聘。夏,公如齐观社。公至自齐。荆人来聘。公及齐侯遇于谷。萧叔朝公。秋,丹桓宫楹。冬十有一月,曹伯射姑卒。十有二月甲寅,公会齐侯盟于扈。

【传】二十三年夏,公如齐观社,非礼也。曹刿谏曰:``不可。夫礼,所以整民也。故会以训上下之则,制财用之节;朝以正班爵之义,帅长幼之序;征伐以讨其不然。诸侯有王,王有巡守,以大习之。非是,君不举矣。君举必书,书而不法,后嗣何观?」

晋桓、庄之族逼,献公患之。士蒍曰:``去富子,则群公子可谋也已。」公曰:``尔试其事。」士蒍与群公子谋,谮富子而去之。

秋,丹桓宫之楹。

\hypertarget{header-n534}{%
\subsubsection{庄公二十四年}\label{header-n534}}

【经】二十有四年春王三月,刻桓宫桷。葬曹庄公。夏,公如齐逆女。秋,公至自齐。八月丁丑,夫人姜氏入。戊寅,大夫宗妇觌,用币。大水。冬,戎侵曹。曹羁出奔陈。赤归于曹。郭公。

【传】二十四年春,刻其桷,皆非礼也。御孙谏曰:``臣闻之:『俭,德之共也;侈,恶之大也。』先君有共德而君纳诸大恶,无乃不可乎!」

秋,哀姜至。公使宗妇觌,用币,非礼也。御孙曰:``男贽大者玉帛,小者禽鸟,以章物也。女贽不过榛栗枣修,以告虔也。今男女同贽,是无别也。男女之别,国之大节也。而由夫人乱之,无乃不可乎!」

晋士蒍又与群公子谋,使杀游氏之二子。士蒍告晋侯曰:``可矣。不过二年,君必无患。」

\hypertarget{header-n541}{%
\subsubsection{庄公二十五年 }\label{header-n541}}

【经】二十有五年春,陈侯使女叔来聘。夏五月癸丑,卫侯朔卒。六月辛未,朔,日有食之,鼓、用牲于社。伯姬归于杞。秋,大水,鼓、用牲于社、于门。冬,公子友如陈。

【传】二十五年春,陈女叔来聘,始结陈好也。嘉之,故不名。

夏六月辛未,朔,日有食之。鼓,用牲于社,非常也。唯正月之朔,慝未作,日有食之,于是乎用币于社,伐鼓于朝。

秋,大水。鼓,用牲于社、于门,亦非常也。凡天灾,有币无牲。非日月之眚,不鼓。

晋士蒍使群公子尽杀游氏之族,乃城聚而处之。

冬,晋侯围聚,尽杀群公子。

\hypertarget{header-n550}{%
\subsubsection{庄公二十六年}\label{header-n550}}

【经】二十有六年春,公伐戎。夏,公至自伐戎。曹杀其大夫。秋,公会宋人、齐人,伐徐。冬十有二月癸亥,朔,日有食之。

【传】二十六年春,晋士蒍为大司空。

夏,士蒍城绛,以深其宫。

秋,虢人侵晋。冬,虢人又侵晋。

\hypertarget{header-n557}{%
\subsubsection{庄公二十七年}\label{header-n557}}

【经】二十有七年春,公会杞伯姬于洮。夏六月,公会齐侯、宋公、陈侯、郑伯同盟于幽。秋,公子友如陈,葬原仲。冬,杞伯姬来。莒庆来逆叔姬。杞伯来朝。公会齐侯于城濮。

【传】二十七年春,公会杞伯姬于洮,非事也。天子非展义不巡守,诸侯非民事不举,卿非君命不越竟。

夏,同盟于幽,陈,郑服也。

秋,公子友如陈,葬原仲,非礼也。原仲,季友之旧也。

冬,杞伯姬来,归宁也。凡诸侯之女,归宁曰来,出曰来归。夫人归宁曰如某,出曰归于某。

晋侯将伐虢,士蒍曰:``不可,虢公骄,若骤得胜于我,必弃其民。无众而后伐之,欲御我谁与?夫礼乐慈爱,战所畜也。夫民让事乐和,爱亲哀丧而后可用也。虢弗畜也,亟战将饥。」

王使召伯廖赐齐侯命,且请伐卫,以其立子颓也。

\hypertarget{header-n567}{%
\subsubsection{庄公二十八年}\label{header-n567}}

【经】二十有八年春,王三月甲寅,齐人伐卫。卫人及齐人战,卫人败绩。夏四月丁未,邾子琐卒。秋,荆伐郑,公会齐人、宋人救郑。冬,筑郿。大无麦、禾,臧孙辰告籴于齐。

【传】二十八年春,齐侯伐卫。战,败卫师。数之以王命,取赂而还。

晋献公娶于贾,无子。烝于齐姜,生秦穆夫人及大子申生。又娶二女于戎,大戎狐姬生重耳,小戎子生夷吾。晋伐骊戎,骊戎男女以骊姬。归生奚齐。其娣生卓子。骊姬嬖,欲立其子,赂外嬖梁五,与东关嬖五,使言于公曰:``曲沃,君之宗也。蒲与二屈,君之疆也。不可以无主。宗邑无主则民不威,疆埸无主则启戎心。戎之生心,民慢其政,国之患也。若使大子主曲沃,而重耳、夷吾主蒲与屈,则可以威民而惧戎,且旌君伐。」使俱曰:``狄之广莫,于晋为都。晋之启土,不亦宜乎?」晋侯说之。夏,使大子居曲沃,重耳居蒲城,夷吾居屈。群公子皆鄙,唯二姬之子在绛。二五卒与骊姬谮群公子而立奚齐,晋人谓之二耦。

楚令尹子元欲蛊文夫人,为馆于其宫侧,而振万焉。夫人闻之,泣曰:``先君以是舞也,习戎备也。今令尹不寻诸仇雠,而于未亡人之侧,不亦异乎!」御人以告子元。子元曰:``妇人不忘袭仇,我反忘之!」

秋,子元以车六百乘伐郑,入于桔柣之门。子元、斗御疆、斗梧、耿之不比为旆,斗班、王孙游、王孙喜殿。众车入自纯门,及逵市。县门不发,楚言而出。子元曰:``郑有人焉。」诸侯救郑,楚师夜遁。郑人将奔桐丘,谍告曰:``楚幕有乌。」乃止。

冬,饥。臧孙辰告籴于齐,礼也。

筑郿,非都也。凡邑有宗庙先君之主曰都,无曰邑。邑曰筑,都曰城。

\hypertarget{header-n577}{%
\subsubsection{庄公二十九年 }\label{header-n577}}

【经】二十有九年春,新延既。夏,郑人侵许。秋,有蜚。冬十有二月,纪叔卒。城诸及防。

【传】二十九年春,新作延。书,不时也。凡马日中而出,日中而入。

夏,郑人侵许。凡师有钟鼓曰伐,无曰侵,轻曰袭。

秋,有蜚,为灾也。凡物不为灾不书。

冬十二月,城诸及防,书,时也。凡土功,龙见而毕务,戒事也。火见而致用,水昏正而栽,日至而毕。

樊皮叛王。

\hypertarget{header-n586}{%
\subsubsection{庄公三十年}\label{header-n586}}

【经】三十年春王正月。夏,次于成。秋七月,齐人降鄣。八月癸亥,葬纪叔。九月庚午朔,日有食之,鼓、用牲于社。冬,公及齐侯遇于鲁济。齐人伐山戎。

【传】三十年春,王命虢公讨樊皮。夏四月丙辰,虢公入樊,执樊仲皮,归于京师。

楚公子元归自伐郑,而处王宫,斗射师谏,则执而梏之。

秋,申公斗班杀子元,斗谷于菟为令尹,自毁其家以纾楚国之难。

冬,遇于鲁济,谋山戎也,以其病燕故也。

\hypertarget{header-n594}{%
\subsubsection{庄公三十一年}\label{header-n594}}

【经】三十有一年春,筑台于郎。夏四月,薛伯卒。筑台于薛。六月,齐侯来献戎捷。秋,筑台于秦。冬,不雨。

【传】三十一年夏六月,齐侯来献戎捷,非礼也。凡诸侯有四夷之功,则献于王,王以警于夷。中国则否。诸侯不相遗俘。

\hypertarget{header-n599}{%
\subsubsection{庄公三十二年}\label{header-n599}}

【经】三十有二年春,城小谷。夏,宋公、齐侯遇于梁丘。秋七月癸巳,公子牙卒。八月癸亥,公薨于路寝。冬十月己未,子般卒。公子庆父如齐。狄伐邢。

【传】三十二年春,城小谷,为管仲也。

齐侯为楚伐郑之故,请会于诸侯。宋公请先见于齐侯。夏,遇于梁丘。

秋七月,有神降于莘。

惠王问诸内史过曰:``是何故也?」对曰:``国之将兴,明神降之,监其德也;将亡,神又降之,观其恶也。故有得神以兴,亦有以亡,虞、夏、商、周皆有之。」王曰:``若之何?」对曰:``以其物享焉,其至之日,亦其物也。」王从之。内史过往,闻虢请命,反曰:``虢必亡矣,虐而听于神。」

神居莘六月。虢公使祝应、宗区、史嚚享焉。神赐之土田。史嚚曰:``虢其亡乎!吾闻之:国将兴,听于民;将亡,听于神。神,聪明正直而一者也,依人而行。虢多凉德,其何土之能得!」

初,公筑台临党氏,见孟任,从之。閟,而以夫人言许之。割臂盟公,生子般焉。雩,讲于梁氏,女公子观之。圉人荦自墙外与之戏。子般怒,使鞭之。公曰:``不如杀之,是不可鞭。荦有力焉,能投盖于稷门。」

公疾,问后于叔牙。对曰:``庆父材。」问于季友,对曰:``臣以死奉般。」公曰:``乡者牙曰庆父材。」成季使以君命命僖叔待于金咸巫氏,使金咸季鸩之,曰:``饮此则有后于鲁国,不然,死且无后。」饮之,归及逵泉而卒,立叔孙氏。

八月癸亥,公薨于路寝。子般即位,次于党氏。冬十月己未,共仲使圉人荦贼子般于党氏。成季奔陈。立闵公。

\hypertarget{header-n610}{%
\subsection{闵公}\label{header-n610}}

\begin{center}\rule{0.5\linewidth}{\linethickness}\end{center}

\hypertarget{header-n612}{%
\subsubsection{闵公元年}\label{header-n612}}

【经】元年春王正月。齐人救邢。夏六月辛酉,葬我君庄公。秋八月,公及齐侯盟于落姑。季子来归。冬,齐仲孙来。

【传】元年春,不书即位,乱故也。

狄人伐邢。管敬仲言于齐侯曰:``戎狄豺狼,不可厌也。诸夏亲昵,不可弃也。宴安鸩毒,不可怀也。《诗》云:『岂不怀归,畏此简书。』简书,同恶相恤之谓也。请救邢以从简书。」齐人救邢。

夏六月,葬庄公,乱故,是以缓。

秋八月,公及齐侯盟于落姑,请复季友也。齐侯许之,使召诸陈,公次于郎以待之。``季子来归」,嘉之也。

冬,齐仲孙湫来省难。书曰``仲孙」,亦嘉之也。

仲孙归曰:``不去庆父,鲁难未已。」公曰:``若之何而去之?」对曰:``难不已,将自毙,君其待之。」公曰:``鲁可取乎?」对曰:``不可,犹秉周礼。周礼,所以本也。臣闻之,国将亡,本必先颠,而后枝叶从之。鲁不弃周礼,未可动也。君其务宁鲁难而亲之。亲有礼,因重固,间携贰,覆□乱,霸王之器也。」

晋侯作二军,公将上军,大子申生将下军。赵夙御戎,毕万为右,以灭耿、灭霍、灭魏。还,为大子城曲沃。赐赵夙耿,赐毕万魏,以为大夫。

士蒍曰:``大子不得立矣,分之都城而位以卿,先为之极,又焉得立。不如逃之,无使罪至。为吴大伯,不亦可乎?犹有令名,与其及也。且谚曰:『心苟无瑕,何恤乎无家。』天若祚大子,其无晋乎。」

卜偃曰:``毕万之后必大。万,盈数也;魏,大名也;以是始赏,天启之矣。天子曰兆民,诸侯曰万民。今名之大,以从盈数,其必有众。」

初,毕万筮仕于晋,遇《屯》之《比》。辛廖占之,曰:``吉。《屯》固《比》入,吉孰大焉?其必蕃昌。《震》为土,车从马,足居之,兄长之,母覆之,众归之,六体不易,合而能固,安而能杀。公侯之卦也。公侯之子孙,必复其始。」

\hypertarget{header-n626}{%
\subsubsection{闵公二年}\label{header-n626}}

【经】二年春王正月,齐人迁阳。夏五月乙酉,吉禘于庄公。秋八月辛丑,公薨。九月,夫人姜氏孙于邾。公子庆父出奔莒。冬,齐高子来盟。十有二月,狄入卫。郑弃其师。

【传】二年春,虢公败犬戎于渭汭。舟之侨曰:``无德而禄,殃也。殃将至矣。」遂奔晋。

夏,吉禘于庄公,速也。

初,公傅夺卜齮田,公不禁。

秋八月辛丑,共仲使卜齮贼公于武闱。成季以僖公适邾。共仲奔莒,乃入,立之。以赂求共仲于莒,莒人归之。及密,使公子鱼请,不许。哭而往,共仲曰:``奚斯之声也。」乃缢。

闵公,哀姜之娣叔姜之子也,故齐人立之。共仲通于哀姜,哀姜欲立之。闵公之死也,哀姜与知之,故孙于邾。齐人取而杀之于夷,以其尸归,僖公请而葬之。

成季之将生也,桓公使卜楚丘之父卜之。曰:``男也。其名曰友,在公之右。间于两社,为公室辅。季氏亡,则鲁不昌。」又筮之,遇《大有》之《乾》,曰:``同复于父,敬如君所。」及生,有文在其手曰``友」,遂以命之。

冬十二月,狄人伐卫。卫懿公好鹤,鹤有乘轩者。将战,国人受甲者皆曰:``使鹤,鹤实有禄位,余焉能战!」公与石祁子玦,与宁庄子矢,使守,曰:``以此赞国,择利而为之。」与夫人绣衣,曰:``听于二子。」渠孔御戎,子伯为右,黄夷前驱,孔婴齐殿。及狄人战于荧泽,卫师败绩,遂灭卫。卫侯不去其旗,是以甚败。狄人囚史华龙滑与礼孔以逐卫人。二人曰:``我,大史也,实掌其祭。不先,国不可得也。」乃先之。至则告守曰:``不可待也。」夜与国人出。狄入卫,遂从之,又败诸河。

初,惠公之即位也少,齐人使昭伯烝于宣姜,不可,强之。生齐子、戴公、文公、宋桓夫人、许穆夫人。文公为卫之多患也,先适齐。及败,宋桓公逆诸河,宵济。卫之遗民男女七百有三十人,益之以共,滕之民为五千人,立戴公以庐于曹。许穆夫人赋《载驰》。齐侯使公子无亏帅车三百乘、甲士三千人以戍曹。归公乘马,祭服五称,牛羊豕鸡狗皆三百,与门材。归夫人鱼轩,重锦三十两。

郑人恶高克,使帅师次于河上,久而弗召。师溃而归,高克奔陈。郑人为之赋《清人》。

晋侯使大子申生伐东山皋落氏。里克谏曰:``大子奉冢祀,社稷之粢盛,以朝夕视君膳者也,故曰冢子。君行则守,有守则从。从曰抚军,守曰监国,古之制也。夫帅师,专行谋,誓车旅,君与国政之所图也,非大子之事也。师在制命而已。禀命则不威,专命则不孝。故君之嗣适不可以帅师。君失其官,帅师不威,将焉用之。且臣闻皋落氏将战,君其舍之。」公曰:``寡人有子,未知其谁立焉。」不对而退。

见大子,大子曰:``吾其废乎?」对曰:``告之以临民,教之以军旅,不共是惧,何故废乎?且子惧不孝,无惧弗得立,修己而不责人,则免于难。」

大子帅师,公衣之偏衣,佩之金玦。狐突御戎,先友为右,梁余子养御罕夷,先丹木为右。羊舌大夫为尉。光友曰:``衣身之偏,握兵之要,在此行也,子其勉之。偏躬无慝,兵要远灾,亲以无灾,又何患焉!」狐突叹曰:``时,事之征也;衣,身之章也;佩,衷之旗也。故敬其事则命以始,服其身则衣之纯,用期衷则佩之度。今命以时卒,閟其事也;衣之龙服,远其躬也;佩以金玦,弃其衷也。服以远之,时以閟之,龙凉冬杀,金寒玦离,胡可恃也?虽欲勉之,狄可尽乎?」梁余子养曰:帅师者受命于庙,受脤于社,有常服矣。不获而龙,命可知也。死而不孝,不如逃之。」罕夷曰:``龙奇无常,金玦不复,虽复何为,君有心矣。」先丹木曰:``是服也。狂夫阻之。曰『尽敌而反』,敌可尽乎!虽尽敌,犹有内谗,不如违之。」狐突欲行。羊舌大夫曰:``不可。违命不孝,弃事不忠。虽知其寒,恶不可取,子其死之。」

大子将战,狐突谏曰:``不可,昔辛伯谂周桓公云:『内宠并后,外宠二政,嬖子配适,大都耦国,乱之本也。』周公弗从,故及于难。今乱本成矣,立可必乎?孝而安民,子其图之,与其危身以速罪也。」

成风闻成季之繇,乃事之,而属僖公焉,故成季立之。

僖之元年,齐桓公迁邢于夷仪。二年,封卫于楚丘。邢迁如归,卫国忘亡。

卫文公大布之衣,大帛之冠,务材训农,通商惠工,敬教劝学,授方任能。元年革车三十乘,季年乃三百乘。

\hypertarget{header-n646}{%
\subsection{僖公}\label{header-n646}}

\begin{center}\rule{0.5\linewidth}{\linethickness}\end{center}

\hypertarget{header-n648}{%
\subsubsection{僖公元年 }\label{header-n648}}

【经】元年春王正月。齐师、宋师、曹伯次于聂北,救邢。夏六月,邢迁于夷仪。齐师、宋师、曹师城邢。秋七月戊辰,夫人姜氏薨于夷,齐人以归。楚人伐郑。八月,公会齐侯、宋公、郑伯、曹伯、邾人于柽。九月,公败邾师于偃。冬十月壬午,公子友帅师败莒于郦。获莒拏。十有二月丁巳,夫人氏之丧至自齐。

【传】元年春,不称即位,公出故也。公出复入,不书,讳之也。讳国恶,礼也。

诸侯救邢。邢人溃,出奔师。师遂逐狄人,具邢器用而迁之,师无私焉。

夏,邢迁夷仪,诸侯城之,救患也。凡侯伯救患分灾讨罪,礼也。

秋,楚人伐郑,郑即齐故也。盟于荦,谋救郑也。

九月,公败邾师于偃,虚丘之戍将归者也。

冬,莒人来求赂。公子友败诸郦,获莒子之弟拏。非卿也,嘉获之也。公赐季友汶阳之田及费。

夫人氏之丧至自齐。君子以齐人杀哀姜也为已甚矣,女子,从人者也。

\hypertarget{header-n659}{%
\subsubsection{僖公二年}\label{header-n659}}

【经】二年春王正月,城楚丘。夏五月辛巳,葬我小君哀姜。虞师、晋师灭下阳。秋九月,齐侯、宋公、江人、黄人盟于贯。冬十月,不雨。楚人侵郑。

【传】二年春,诸侯城楚丘而封卫焉。不书所会,后也。

晋荀息请以屈产之乘与垂棘之璧,假道于虞以伐虢。公曰:``是吾宝也。」对曰:``若得道于虞,犹外府也。」公曰:``宫之奇存焉。」对曰:``宫之奇之为人也,懦而不能强谏,且少长于君,君昵之,虽谏,将不听。」乃使荀息假道于虞,曰:``冀为不道,入自颠軨,伐鄍三门。冀之既病。则亦唯君故。今虢为不道,保于逆旅,以侵敝邑之南鄙。敢请假道以请罪于虢。」虞公许之,且请先伐虢。宫之奇谏,不听,遂起师。夏,晋里克、荀息帅师会虞师伐虢,灭下阳。先书虞,贿故也。

秋,盟于贯,服江、黄也。

齐寺人貂始漏师于多鱼。

虢公败戎于桑田。晋卜偃曰:``虢必亡矣。亡下阳不惧,而又有功,是天夺之鉴,而益其疾也。必易晋而不抚其民矣,不可以五稔。」

冬,楚人伐郑,斗章囚郑聃伯。

\hypertarget{header-n669}{%
\subsubsection{僖公三年}\label{header-n669}}

【经】三年春王正月,不雨。夏四月不雨。徐人取舒。六月雨。秋,齐侯、宋公、江人、黄人会于阳谷。冬,公子友如齐位盟。楚人伐郑。

【传】三年春,不雨。夏六月,雨。自十月不雨至于五月,不曰旱,不为灾也。

秋,会于阳谷,谋伐楚也。

齐侯为阳谷之会,来寻盟。冬,公子友如齐位盟。

楚人伐郑,郑伯欲成。孔叔不可,曰:``齐方勤我,弃德不祥。」

齐侯与蔡姬乘舟于囿,荡公。公惧,变色。禁之,不可。公怒,归之,未绝之也。蔡人嫁之。

\hypertarget{header-n678}{%
\subsubsection{僖公四年}\label{header-n678}}

【经】四年春王正月,公会齐侯、宋公、陈侯、卫侯、郑伯,许男、曹伯侵蔡。蔡溃,遂伐楚,次于陉。夏,许男新臣卒。楚屈完来盟于师,盟于召陵。齐人执陈辕涛涂。秋,及江人、黄人伐陈。八月,公至自伐楚。葬许穆公。冬十有二月,公孙兹帅师会齐人、宋人、卫人、郑人、许人、曹人侵陈。

【传】四年春,齐侯以诸侯之师侵蔡。蔡溃。遂伐楚。楚子使与师言曰:``君处北海,寡人处南海,唯是风马牛不相及也。不虞君之涉吾地也,何故?」管仲对曰:``昔召康公命我先君大公曰:『五侯九伯,女实征之,以夹辅周室。』赐我先君履,东至于海,西至于河,南至于穆陵,北至于无棣。尔贡包茅不入,王祭不共,无以缩酒,寡人是征。昭王南征而不复,寡人是问。」对曰:``贡之不入,寡君之罪也,敢不共给。昭王之不复,君其问诸水滨。」师进,次于陉。

夏,楚子使屈完如师。师退,次于召陵。

齐侯陈诸侯之师,与屈完乘而观之。齐侯曰:``岂不谷是为?先君之好是继。与不谷同好,如何?」对曰:``君惠徼福于敝邑之社稷,辱收寡君,寡君之愿也。」齐侯曰:``以此众战,谁能御之?以此攻城,何城不克?」对曰:``君若以德绥诸侯,谁敢不服?君若以力,楚国方城以为城,汉水以为池,虽众,无所用之。」

屈完及诸侯盟。

陈辕涛涂谓郑申侯曰:``师出于陈、郑之间,国必甚病。若出于东方,观兵于东夷,循海而归,其可也。」申侯曰:``善。」涛涂以告,齐侯许之。申侯见,曰:``师老矣,若出于东方而遇敌,惧不可用也。若出于陈、郑之间,共其资粮悱屦,其可也。」齐侯说,与之虎牢。执辕涛涂。

秋,伐陈,讨不忠也。

许穆公卒于师,葬之以侯,礼也。凡诸侯薨于朝会,加一等;死王事,加二等。于是有以衮敛。

冬,叔孙戴伯帅师,会诸侯之师侵陈。陈成,归辕涛涂。

初,晋献公欲以骊姬为夫人,卜之,不吉;筮之,吉。公曰:``从筮。」卜人曰:``筮短龟长,不如从长。且其繇曰:『专之渝,攘公之羭。一薰一莸,十年尚犹有臭。』必不可。」弗听,立之。生奚齐,其娣生卓子。及将立奚齐,既与中大夫成谋,姬谓大子曰:``君梦齐姜,必速祭之。」大子祭于曲沃,归胙于公。公田,姬置诸宫六日。公至,毒而献之。公祭之地,地坟。与犬,犬毙。与小臣,小臣亦毙。姬泣曰:``贼由大子。」大子奔新城。公杀其傅杜原款。或谓大子:``子辞,君必辩焉。」大子曰:``君非姬氏,居不安,食不饱。我辞,姬必有罪。君老矣,吾又不乐。」曰:``子其行乎!」大子曰:``君实不察其罪,被此名也以出,人谁纳我?」

十二月戊申,缢于新城。姬遂谮二公子曰:``皆知之。」重耳奔蒲。夷吾奔屈。

\hypertarget{header-n692}{%
\subsubsection{僖公五年}\label{header-n692}}

【经】五年春,晋侯杀其世子申生。杞伯姬来朝其子。夏,公孙兹如牟。公及齐侯、宋公、陈侯、卫侯、郑伯、许男、曹伯会王世子于首止。秋八月,诸侯盟于首止。郑伯逃归不盟。楚人灭弦,弦子奔黄。九月戊申朔,日有食之。冬,晋人执虞公。

【传】五年春,王正月辛亥朔,日南至。公既视朔,遂登观台以望。而书,礼也。凡分、至、启、闭,必书云物,为备故也。

晋侯使以杀大子申生之故来告。

初,晋侯使士蒍为二公子筑蒲与屈,不慎,置薪焉。夷吾诉之。公使让之。士蒍□稽首而对曰:``臣闻之,无丧而戚,忧必仇焉。无戎而城,仇必保焉。寇仇之保,又何慎焉!守官废命不敬,固仇之保不忠,失忠与敬,何以事君?《诗》云:『怀德惟宁,宗子惟城。』君其修德而固宗子,何城如之?三年将寻师焉,焉用慎?」退而赋曰:``狐裘尨茸,一国三公,吾谁适从?」及难,公使寺人披伐蒲。重耳曰:``君父之命不校。」乃徇曰:``校者吾仇也。」逾垣而走。披斩其祛,遂出奔翟。

夏,公孙兹如牟,娶焉。

会于首止,会王大子郑,谋宁周也。

陈辕宣仲怨郑申侯之反己于召陵,故劝之城其赐邑,曰:``美城之,大名也,子孙不忘。吾助子请。」乃为之请于诸侯而城之,美。遂谮诸郑伯,曰:``美城其赐邑,将以叛也。」申侯由是得罪。

秋,诸侯盟。王使周公召郑伯,曰:``吾抚女以从楚,辅之以晋,可以少安。」郑伯喜于王命而惧其不朝于齐也,故逃归不盟,孔叔止之曰:``国君不可以轻,轻则失亲。失亲患必至,病而乞盟,所丧多矣,君必悔之。」弗听,逃其师而归。

楚斗谷于菟灭弦,弦子奔黄。

于是江、黄、道、柏方睦于齐,皆弦姻也。弦子恃之而不事楚,又不设备,故亡。

晋侯复假道于虞以伐虢。宫之奇谏曰:``虢,虞之表也。虢亡,虞必从之。晋不可启,寇不可玩,一之谓甚,其可再乎?谚所谓『辅车相依,唇亡齿寒』者,其虞、虢之谓也。」公曰:``晋,吾宗也,岂害我哉?」对曰:大伯、虞仲,大王之昭也。大伯不从,是以不嗣。虢仲、虢叔,王季之穆也,为文王卿士,勋在王室,藏于盟府。将虢是灭,何爱于虞?且虞能亲于桓,庄乎,其爱之也?桓、庄之族何罪,而以为戮,不唯逼乎?亲以宠逼,犹尚害之,况以国乎?」公曰:``吾享祀丰洁,神必据我。」对曰:``臣闻之,鬼神非人实亲,惟德是依。故《周书》曰:『皇天无亲,惟德是辅。』又曰:『黍稷非馨,明德惟馨。』又曰:『民不易物,惟德繄物。』如是,则非德,民不和,神不享矣。神所冯依,将在德矣。若晋取虞而明德以荐馨香,神其吐之乎?」弗听,许晋使。宫之奇以其族行,曰:``虞不腊矣,在此行也,晋不更举矣。」

八月甲午,晋侯围上阳。问于卜偃曰:``吾其济乎」?对曰:``克之。」公曰:``何时?」对曰:``童谣云:『丙之晨,龙尾伏辰,均服振振,取虢之旂。鹑之贲贲,天策焞焞,火中成军,虢公其奔。』其九月、十月之交乎。丙子旦,日在尾,月在策,鹑火中,必是时也。」

冬十二月丙子朔,晋灭虢,虢公丑奔京师。师还,馆于虞,遂袭虞,灭之,执虞公及其大夫井伯,以媵秦穆姬。而修虞祀,且归其职贡于王。

故书曰:``晋人执虞公。」罪虞,且言易也。

\hypertarget{header-n709}{%
\subsubsection{僖公六年}\label{header-n709}}

【经】六年春王正月。夏,公会齐侯、宋公、陈侯、卫侯、曹伯伐郑,围新城。秋,楚人围许,诸侯遂救许。冬,公至自伐郑。

【传】六年春,晋侯使贾华伐屈。夷吾不能守,盟而行。将奔狄郤芮曰:``后出同走,罪也。不如之梁。梁近秦而幸焉。」乃之梁。

夏,诸侯伐郑,以其逃首止之盟故也。围新密,郑所以不时城也。

秋,楚子围许以救郑,诸侯救许,乃还。

冬,蔡穆侯将许僖公以见楚子于武城。许男面缚,衔璧,大夫衰絰,士舆榇。楚子问诸逢伯,对曰:``昔武王克殷,微子启如是。武王亲释其缚,受其璧而祓之。焚其榇,礼而命之,使复其所。」楚子从之。

\hypertarget{header-n717}{%
\subsubsection{僖公七年}\label{header-n717}}

【经】七年春,齐人伐郑。夏,小邾子来朝。郑杀其大夫申侯。秋七月,公会齐侯、宋公、陈世子款、郑世子华盟于宁母。曹伯班卒。公子友如齐。冬葬曹昭公。

【传】七年春,齐人伐郑。孔叔言于郑伯曰:``谚有之曰:『心则不竞,何惮于病。』既不能强,又不能弱,所以毙也。国危矣,请下齐以救国。」公曰:``吾知其所由来矣。姑少待我。」对曰:``朝不及夕,何以待君?」

夏,郑杀申侯以说于齐,且用陈辕涛涂之谮也。

初,申侯,申出也,有宠于楚文王。文王将死,与之璧,使行,曰,``唯我知女,女专利而不厌,予取予求,不女疵瑕也。后之人将求多于女,女必不免。我死,女必速行。无适小国,将不女容焉。」既葬,出奔郑,又有宠于厉公。子文闻其死也,曰:``古人有言曰『知臣莫若君。』弗可改也已。」

秋,盟于宁母,谋郑故也。

管仲言于齐侯曰:``臣闻之,招携以礼,怀远以德,德礼不易,无人不怀。」齐侯修礼于诸侯,诸侯官受方物。

郑伯使大子华听命于会,言于齐侯曰:``泄氏、孔氏、子人氏三族,实违君命。若君去之以为成。我以郑为内臣,君亦无所不利焉。」齐侯将许之。管仲曰:``君以礼与信属诸侯,而以奸终之,无乃不可乎?子父不奸之谓礼,守命共时之谓信。违此二者,奸莫大焉。」公曰:``诸侯有讨于郑,未捷。今苟有衅。从之,不亦可乎?」对曰:``君若绥之以德,加之以训辞,而帅诸侯以讨郑,郑将覆亡之不暇,岂敢不惧?若总其罪人以临之,郑有辞矣,何惧?且夫合诸侯以崇德也,会而列奸,何以示后嗣?夫诸侯之会,其德刑礼义,无国不记。记奸之位,君盟替矣。作而不记,非盛德也。君其勿许,郑必受盟。夫子华既为大子而求介于大国,以弱其国,亦必不免。郑有叔詹、堵叔、师叔三良为政,未可间也。」齐侯辞焉。子华由是得罪于郑。

冬,郑伯请盟于齐。

闰月,惠王崩。襄王恶大叔带之难,惧不立,不发丧而告难于齐。

\hypertarget{header-n729}{%
\subsubsection{僖公八年}\label{header-n729}}

【经】八年春王正月,公会王人、齐侯、宋公、卫侯、许男、曹伯、陈世子款盟于洮。郑伯乞盟。夏,狄伐晋。秋七月,禘于大庙,用致夫人。冬十有二月丁未,天王崩。

【传】八年春,盟于洮,谋王室也。郑伯乞盟,请服也。襄王定位而后发丧。

晋里克帅师,梁由靡御。虢射为右,以败狄于采桑。梁由靡曰:``狄无耻,从之必大克。」里克曰:``拒之而已,无速众狄。」虢射曰:``期年,狄必至,示之弱矣。」

夏,狄伐晋,报采桑之役也。复期月。

秋,禘而致哀姜焉,非礼也。凡夫人不薨于寝,不殡于庙,不赴于同,不祔于姑,则弗致也。

冬,王人来告丧,难故也,是以缓。

宋公疾,大子兹父固请曰:``目夷长,且仁,君其立之。」公命子鱼,子鱼辞,曰:``能以国让,仁孰大焉?臣不及也,且又不顺。」遂走而退。

\hypertarget{header-n739}{%
\subsubsection{僖公九年}\label{header-n739}}

【经】九年春王三月丁丑,宋公御说卒。夏,公会宰周公、齐侯、宋子、卫侯、郑伯、许男、曹伯于葵丘。秋七月乙酉,伯姬卒。九月戊辰,诸侯盟于葵丘。甲子,晋侯佹诸卒。冬,晋里奚克杀其君之子奚齐。

【传】九年春,宋桓公卒,未葬而襄公会诸侯,故曰子。凡在丧,王曰小童,公侯曰子。

夏,会于葵丘,寻盟,且修好,礼也。

王使宰孔赐齐侯胙,曰:``天子有事于文武,使孔赐伯舅胙。」齐侯将下拜。孔曰:``且有后命。天子使孔曰:『以伯舅耋老,加劳,赐一级,无下拜』」。对曰:``天威不违颜咫尺,小白余敢贪天子之命无下拜?恐陨越于下,以遗天子羞。敢不下拜?」下,拜;登,受。

秋,齐侯盟诸侯于葵丘,曰:``凡我同盟之人,既盟之后,言归于好。」宰孔先归,遇晋侯曰:``可无会也。齐侯不务德而勤远略,故北伐山戎,南伐楚,西为此会也。东略之不知,西则否矣。其在乱乎。君务靖乱,无勤于行。」晋侯乃还。

九月,晋献公卒,里克、ぶ郑欲纳文公,故以三公子之徒作乱。

初,献公使荀息傅奚齐,公疾,召之,曰:``以是藐诸孤,辱在大夫,其若之何?」稽首而对曰:``臣竭其股肱之力,加之以忠贞。其济,君之灵也;不济,则以死继之。」公曰:``何谓忠贞?」对曰:``公家之利,知无不为,忠也。送往事居,耦俱无猜。贞也。」及里克将杀奚齐,先告荀息曰:``三怨将作,秦、晋辅之,子将何如?」荀息曰:``将死之。」里克曰:``无益也。」荀叔曰:``吾与先君言矣,不可以贰。能欲复言而爱身乎?虽无益也,将焉辟之?且人之欲善,谁不如我?我欲无贰而能谓人已乎?」

冬十月,里克杀奚齐于次。书曰:``杀其君之子。」未葬也。荀息将死之,人曰:``不如立卓子而辅之。」荀息立公子卓以葬。十一月,里克杀公子卓于朝,荀息死之。君子曰:``诗所谓『白圭之玷,尚可磨也;斯言之玷,不可为也,』荀息有焉。」

齐侯以诸侯之师伐晋,及高梁而还,讨晋乱也。令不及鲁,故不书。

晋郤芮使夷吾重赂秦以求入,曰:``人实有国,我何爱焉。入而能民,土于何有。」从之。齐隰朋帅师会秦师,纳晋惠公。秦伯谓郤芮曰:``公子谁恃?」对曰:``臣闻亡人无党,有党必有仇。夷吾弱不好弄,能斗不过,长亦不改,不识其他。」公谓公孙枝曰:``夷吾其定乎?对曰:``臣闻之,唯则定国。《诗》曰:『不识不知,顺帝之则。』文王之谓也。又曰:『不僭不贼,鲜不为则。』无好无恶,不忌不克之谓也。今其言多忌克,难哉!」公曰:``忌则多怨,又焉能克?是吾利也。」

宋襄公即位,以公子目夷为仁,使为左师以听政,于是宋治。故鱼氏世为左师。

\hypertarget{header-n753}{%
\subsubsection{僖公十年}\label{header-n753}}

【经】十年春王正月,公如齐。狄灭温,温子奔卫。晋里克弑其君卓及其大夫荀息。夏,齐侯、许男伐北戎。晋杀其大夫里克。秋七月。冬,大雨雪。

【传】十年春,狄灭温,苏子无信也。苏子叛王即狄,又不能于狄,狄人伐之,王不救,故灭。苏子奔卫。

夏四月,周公忌父、王子党会齐隰朋立晋侯。晋侯杀里克以说。将杀里克,公使谓之曰:``微子则不及此。虽然,子弑二君与一大夫,为子君者不亦难乎?」对曰:``不有废也,君何以兴?欲加之罪,其无辞乎?臣闻命矣。」伏剑而死。于是ぶ郑聘于秦,且谢缓赂,故不及。

晋侯改葬共大子。

秋,狐突适下国,遇大子,大子使登,仆,而告之曰:``夷吾无礼,余得请于帝矣。将以晋畀秦,秦将祀余。」对曰:``臣闻之,神不歆非类,民不祀非族。君祀无乃殄乎?且民何罪?失刑乏祀,君其图之。」君曰:``诺。吾将复请。七日新城西偏,将有巫者而见我焉。」许之,遂不见。及期而往,告之曰:``帝许我罚有罪矣,敝于韩。」

ぶ郑之如秦也,言于秦伯曰:``吕甥、郤称、冀芮实为不从,若重问以召之,臣出晋君,君纳重耳,蔑不济矣。」

冬,秦伯使冷至报问,且召三子。郤芮曰:``币重而言甘,诱我也。」遂杀ぶ郑、祁举及七舆大夫:左行共华、右行贾华、叔坚、骓颛、累虎、特宫、山祁,皆里、ぶ之党也。ぶ豹奔秦,言于秦伯曰:``晋侯背大主而忌小怨,民弗与也,伐之必出。」公曰:``失众,焉能杀。违祸,谁能出君。」

\hypertarget{header-n763}{%
\subsubsection{僖公十一年}\label{header-n763}}

【经】十有一年春。晋杀其大夫ぶ郑父。夏,公及夫人姜氏会齐侯于阳谷。秋八月,大雩。冬,楚人伐黄。

【传】十一年春,晋侯使以ぶ郑之乱来告。

天王使召武公、内史过赐晋侯命。受玉惰。过归,告王曰:``晋侯其无后乎。王赐之命而惰于受瑞,先自弃也已,其何继之有?礼,国之干也。敬,礼之舆也。不敬则礼不行,礼不行则上下昏,何以长世?」

夏,扬、拒、泉、皋、伊、洛之戎同伐京师,入王城,焚东门,王子带召之也。秦、晋、伐戎以救周。秋,晋侯平戎于王。

黄人不归楚贡。冬,楚人伐黄。

\hypertarget{header-n771}{%
\subsubsection{僖公十二年 }\label{header-n771}}

【经】十有二年春王三月庚午,日有食之。夏,楚人灭黄。秋七月。冬十有二月丁丑,陈侯杵臼卒。

【传】十二年春,诸侯城卫楚丘之郛,惧狄难也。

黄人恃诸侯之睦于齐也,不共楚职,曰:``自郢及我九百里,焉能害我?」夏,楚灭黄。
王以戎难故,讨王子带。秋,王子带奔齐。

冬,齐侯使管夷吾平戎于王,使隰朋平戎于晋。

王以上卿之礼飨管仲,管仲辞曰:``臣,贱有司也,有天子之二守国、高在。若节春秋来承王命,何以礼焉?陪臣敢辞。」王曰:``舅氏,余嘉乃勋,应乃懿德,谓督不忘。往践乃职,无逆朕命。」管仲受下卿之礼而还。君子曰:``管氏之世祀也宜哉!让不忘其上。《诗》曰:『恺悌君子,神所劳矣。』」

\hypertarget{header-n779}{%
\subsubsection{僖公十三年}\label{header-n779}}

【经】十有三年春,狄侵卫。夏四月,葬陈宣公。公会齐侯、宋公、陈侯、郑伯、许男、曹伯于咸。秋九月,大雩。冬,公子友如齐。

【传】十三年春,齐侯使仲孙湫聘于周,且言王子带。事毕,不与王言。归,覆命曰:``未可。王怒未怠,其十年乎。不十年,王弗召也。」

夏,会于咸,淮夷病杞故,且谋王室也。

秋,为戎难故,诸侯戍周,齐仲孙湫致之。

冬,晋荐饥,使乞籴于秦。秦伯谓子桑:``与诸乎?」对曰:``重施而报,君将何求?重施而不报,其民必携,携而讨焉,无众必败。」谓百里:``与诸乎?」对曰:``天灾流行,国家代有,救灾恤邻,道也。行道有福。」

邳郑之子豹在秦,请伐晋。秦伯曰:``其君是恶,其民何罪?」秦于是乎输粟于晋,自雍及绛相继,命之曰泛舟之役。

\hypertarget{header-n788}{%
\subsubsection{僖公十四年}\label{header-n788}}

【经】十有四年春,诸侯城缘陵。夏六月,季姬及鄫子遇于防。使鄫子来朝。秋八月辛卯,沙鹿崩。狄侵郑。冬,蔡侯肝卒。

【传】十四年春,诸侯城缘陵而迁杞焉。不书其人,有阙也。

鄫季姬来宁,公怒,止之,以鄫子之不朝也。夏,遇于防,而使来朝。

秋八月辛卯,沙鹿崩。晋卜偃曰:``期年将有大咎,几亡国。」

冬,秦饥,使乞籴于晋,晋人弗与。庆郑曰:``背施无亲,幸灾不仁,贪爱不祥,怒邻不义。四德皆失,何以守国?」虢射曰:``皮之不存,毛将安傅?」庆郑曰:``弃信背邻,患孰恤之?无信患作,失授必毙,是则然矣。」虢射曰:``无损于怨而厚于寇,不如勿与。」庆郑曰:``背施幸灾,民所弃也。近犹仇之,况怨敌乎?」弗听。退曰:``君其悔是哉!」

\hypertarget{header-n796}{%
\subsubsection{僖公十五年}\label{header-n796}}

【经】十有五年春王正月,公如齐。楚人伐徐。三月,公会齐侯、宋公、陈侯、卫候、郑伯、许男、曹伯盟于牡丘,遂次于匡。公孙敖帅师及诸侯之大夫救徐。夏五月,日有食之。秋七月,齐师、曹师伐厉。八月,螽。九月,公至自会。季姬归于鄫。己卯晦,震夷伯之庙。冬,宋人伐曹。楚人败徐于娄林。十有一月壬戌,晋侯及秦伯战于韩,获晋侯。

【传】十五年春,楚人伐徐,徐即诸夏故也。三月,盟于牡丘,寻蔡丘之盟,且救徐也。孟穆伯帅师及诸侯之师救徐,诸侯次于匡以待之。

夏五月,日有食之。不书朔与日,官失之也。

秋,伐,厉,以救徐也。

晋侯之入也,秦穆姬属贾君焉,且曰:``尽纳群公子。」晋侯烝于贾君,又不纳群公子,是以穆姬怨之。晋侯许赂中大夫,既而皆背之。赂秦伯以河外列城五,东尽虢略,南及华山,内及解梁城,既而不与。晋饥,秦输之粟;秦饥,晋闭之籴,故秦伯伐晋。

卜徒父筮之,吉。涉河,侯车败。诘之,对曰:``乃大吉也,三败必获晋君。其卦遇《蛊》,曰:『千乘三去,三去之馀,获其雄狐。』夫狐蛊,必其君也。《蛊》之贞,风也;其悔,山也。岁云秋矣,我落其实而取其材,所以克也。实落材亡,不败何待?」

三败及韩。晋侯谓庆郑曰:``寇深矣,若之何?」对曰:``君实深之,可若何?」公曰:``不孙。」卜右,庆郑吉,弗使。步扬御戎,家仆徒为右,乘小驷,郑入也。庆郑曰:``古者大事,必乘其产,生其水土而知其人心,安其教训而服习其道,唯所纳之,无不如志。今乘异产,以从戎事,及惧而变,将与人易。乱气狡愤,阴血周作,张脉偾兴,外强中乾。进退不可,周旋不能,君必悔之。」弗听。

九月,晋侯逆秦师,使韩简视师,复曰:``师少于我,斗士倍我。」公曰:``何故?」对曰:``出因其资,入用其宠,饥食其粟,三施而无报,是以来也。今又击之,我怠秦奋,倍犹未也。」公曰:``一夫不可狃,况国乎。」遂使请战,曰:``寡人不佞,能合其众而不能离也,君若不还,无所逃命。」秦伯使公孙枝对曰:``君之未入,寡人惧之,入而未定列,犹吾忧也。苟列定矣,敢不承命。」韩简退曰:``吾幸而得囚。」

壬戌,战于韩原,晋戎马还泞而止。公号庆郑。庆郑曰:``愎谏违卜,固败是求,又何逃焉?」遂去之。梁由靡御韩简,虢射为右,辂秦伯,将止之。郑以救公误之,遂失秦伯。秦获晋侯以归。晋大夫反首拔舍从之。秦伯使辞焉,曰:``二三子何其戚也?寡人之从君而西也,亦晋之妖梦是践,岂敢以至。」晋大夫三拜稽首曰:``君履后土而戴皇天,皇天后土实闻君之言,群臣敢在下风。」

穆姬闻晋侯将至,以大子荦、弘与女简、璧登台而履薪焉,使以免服衰絰逆,且告曰:``上天降灾,使我两君匪以玉帛相见,而以兴戎。若晋君朝以入,则婢子夕以死;夕以入,则朝以死。唯君裁之。」乃舍诸灵台。

大夫请以入。公曰:``获晋侯,以厚归也。既而丧归,焉用之?大夫其何有焉?且晋人戚忧以重我,天地以要我。不图晋忧,重其怒也;我食吾言,背天地也。重怒难任,背天不祥,必归晋君。」公子絷曰:``不如杀之,无聚慝焉。」子桑曰:``归之而质其大子,必得大成。晋未可灭而杀其君,只以成恶。且史佚有言曰:『无始祸,无怙乱,无重怒。』重怒难任,陵人不祥。」乃许晋平。

晋侯使郤乞告瑕吕饴甥,且召之。子金教之言曰:``朝国人而以君命赏,且告之曰:『孤虽归,辱社稷矣。其卜贰圉也。』」众皆哭。晋于是乎作爰田。吕甥曰:``君亡之不恤,而群臣是忧,惠之至也。将若君何?」众曰:``何为而可?」对曰:``征缮以辅孺子,诸侯闻之,丧君有君,群臣辑睦,甲兵益多,好我者劝,恶我者惧,庶有益乎!」众说。晋于是乎作州兵。

初,晋献公筮嫁伯姬于秦,遇《归妹》三之《睽》三。史苏占之曰:``不吉。其繇曰:『士刲羊,亦无亡也。女承筐,亦无贶也。西邻责言,不可偿也。《归妹》之《睽》,犹无相也。』《震》之《离》,亦《离》之《震》,为雷为火。为嬴败姬,车说问其輹,火焚其旗,不利行师,败于宗丘。《归妹》《睽》孤,寇张之弧,侄其从姑,六年其逋,逃归其国,而弃其家,明年其死于高梁之虚。」及惠公在秦,曰:``先君若从史苏之占,吾不及此夫。」韩简侍,曰:``龟,像也;筮,数也。物生而后有象,像而后有滋,滋而后有数。先君之败德,乃可数乎?史苏是占,勿从何益?《诗》曰:『下民之孽,匪降自天,僔沓背憎,职竞由人。』」

震夷伯之庙,罪之也,于是展氏有隐慝焉。

冬,宋人伐曹,讨旧怨也。

楚败徐于娄林,徐恃救也。

十月,晋阴饴甥会秦伯,盟于王城。

秦伯曰:``晋国和乎?」对曰:``不和。小人耻失其君而悼丧其亲,不惮征缮以立圉也,曰:『必报仇,宁事戎狄。』君子爱其君而知其罪,不惮征缮以待秦命,曰:『必报德,有死无二。』以此不和。」秦伯曰:``国谓君何?」对曰:``小人戚,谓之不免。君子恕,以为必归。小人曰:『我毒秦,秦岂归君?』君子曰:『我知罪矣,秦必归君。贰而执之,服而舍之,德莫厚焉,刑莫威焉。服者怀德,贰者畏刑。此一役也,秦可以霸。纳而不定,废而不立,以德为怨,秦不其然。』」秦伯曰:``是吾心也。」改馆晋侯,馈七牢焉。

蛾析谓庆郑曰:``盍行乎?」对曰:``陷君于败,败而不死,又使失刑,非人臣也。臣而不臣,行将焉入?」十一月,晋侯归。丁丑,杀庆郑而后入。
是岁,晋又饥,秦伯又饩之粟,曰:``吾怨其君而矜其民。且吾闻唐叔之封也,箕子曰:『其后必大。』晋其庸可冀乎!姑树德焉以待能者。」于是秦始征晋河东,置官司焉。

\hypertarget{header-n818}{%
\subsubsection{僖公十六年}\label{header-n818}}

【经】十有六年春王正月戊申朔,陨石于宋五。是月,六鷁退飞,过宋都。三月壬申,公子季友卒。夏四月丙申,鄫季姬卒。秋七月甲子,公孙兹卒。冬十有二月,公会齐侯、宋公、陈侯、卫侯、郑伯、许男、邢侯、曹伯于淮。

【传】十六年春,陨石于宋五,陨星也。六鷁退飞过宋都,风也。周内史叔兴聘于宋,宋襄公问焉,曰;``是何祥也?吉凶焉在?」对曰:``今兹鲁多大丧,明年齐有乱,君将得诸侯而不终。」退而告人曰:``君失问。是阴阳之事,非吉凶所生也。吉凶由人,吾不敢逆君故也。」

夏,齐伐厉不克,救徐而还。

秋,狄侵晋,取狐、厨、受铎,涉汾,及昆都,因晋败也。

王以戎难告于齐,齐征诸侯而戍周。

冬,十一月乙卯,郑杀子华。

十二月会于淮,谋郐,且东略也。城鄫,役人病。有夜登丘而呼曰:``齐有乱。」不果城而还。

\hypertarget{header-n828}{%
\subsubsection{僖公十七年}\label{header-n828}}

【经】十有七年春,齐人、徐人伐英氏。夏,灭项。秋,夫人姜氏会齐侯于卞。九月,会至自会。冬十有二月乙亥,齐侯小白卒。

【传】十七年春,齐人为徐伐英氏,以报娄林之役也。

夏,晋大子圉为质于秦,秦归河东而妻之。惠公之在梁也,梁伯妻之。梁赢孕,过期,卜招父与其子卜之。其子曰:``将生一男一女。」招曰:``然。男为人臣,女为人妾。」故名男曰圉,女曰妾。及子圉西质,妾为宦女焉。

师灭项。淮之会,公有诸侯之事未归而取项。齐人以为讨,而止公。

秋,声姜以公故,会齐侯于卞。九月,公至。书曰:``至自会。」犹有诸侯之事焉,且讳之也。

齐侯之夫人三:王姬,徐嬴,蔡姬,皆无子。齐侯好内,多内宠,内嬖如夫人者六人:长卫姬,生武孟;少卫姬,生惠公;郑姬,生孝公;葛嬴,生昭公;密姬,生懿公,宋华子,生公子雍。公与管仲属孝公于宋襄公,以为太子。雍巫有宠于卫共姬,因寺人貂以荐羞于公,亦有宠,公许之立武孟。

管仲卒,五公子皆求立。冬十月乙亥,齐桓公卒。易牙入,与寺人貂因内宠以杀群吏,而立公子无亏。孝公奔宋。十二月乙亥赴。辛巳夜殡。

\hypertarget{header-n838}{%
\subsubsection{僖公十八年}\label{header-n838}}

【经】十有八年春王正月,宋公、曹伯、卫人、邾人伐齐。夏,师救齐。五月戊寅,宋师及齐师战于甗。齐师败绩。狄救齐。秋八月丁亥,葬齐桓公。冬,邢人,狄人伐卫。

【传】十八年春,宋襄公以诸侯伐齐。三月,齐人杀无亏。

郑伯始朝于楚,楚子赐之金,既而悔之,与之盟曰:``无以铸兵。」故以铸三钟。

齐人将立孝公,不胜,四公子之徒遂与宋人战。夏五月,宋败齐师于,立孝公而还。

秋八月,葬齐桓公。

冬,邢人、狄人伐卫,围菟圃。卫侯以国让父兄子弟及朝众曰:``苟能治之,毁请从焉。」众不可,而后师于訾娄。狄师还。

梁伯益其国而不能实也,命曰新里,秦取之。

\hypertarget{header-n848}{%
\subsubsection{僖公十九年}\label{header-n848}}

【经】十有九年春王三月,宋人执滕子婴齐。夏六月,宋公、曹人、邾人盟于曹南。鄫子会盟于邾。己酉,邾人执郐子,用之。秋,宋人围曹。卫人伐邢。冬,会陈人、蔡人、楚人、郑人盟于齐。梁亡。

【传】十九年春,遂城而居之。

宋人执滕宣公。

夏,宋公使邾文公用鄫子于次睢之社,欲以属东夷。司马子鱼曰:``古者六畜不相为用,小事不用大牲,而况敢用人乎?祭祀以为人也。民,神之主也。用人,其谁飨之?齐桓公存三亡国以属诸侯,义士犹曰薄德。今一会而虐二国之君,又用诸淫昏之鬼,将以求霸,不亦难乎?得死为幸!」

秋,卫人伐邢,以报菟圃之役。于是卫大旱,卜有事于山川,不吉。宁庄子曰:``昔周饥,克殷而年丰。今邢方无道,诸侯无伯,天其或者欲使卫讨邢乎?」从之,师兴而雨。

宋人围曹,讨不服也。子鱼言于宋公曰:``文王闻崇德乱而伐之,军三旬而不降,退修教而复伐之,因垒而降。《诗》曰:『刑于寡妻,至于兄弟,以御于家邦。』今君德无乃犹有所阙,而以伐人,若之何?盍姑内省德乎?无阙而后动。」

陈穆公请修好于诸侯,以无忘齐桓之德。冬,盟于齐,修桓公之好也。

梁亡,不书其主,自取之也。初,梁伯好土功,亟城而弗处,民罢而弗堪,则曰:``某寇将至。」乃沟公宫,曰:``秦将袭我。」民惧而溃,秦遂取梁。

\hypertarget{header-n859}{%
\subsubsection{僖公二十年}\label{header-n859}}

【经】二十年春,新作南门。夏,郜子来朝。五月乙巳,西宫灾。郑人入滑。秋,齐人、狄人盟于邢。冬,楚人伐随。

【传】二十年春,新作南门。书,不时也。凡启塞从时。

滑人叛郑而服于卫。夏,郑公子士、泄堵寇帅师入滑。

秋,齐、狄盟于邢,为邢谋卫难也。于是卫方病邢。

随以汉东诸侯叛楚。冬,楚斗谷于菟帅师伐随,取成而还。君子曰:``随之见伐,不量力也。量力而动,其过鲜矣。善败由己,而由人乎哉?《诗》曰:『岂不夙夜,谓行多露。』」

宋襄公欲合诸侯,臧文仲闻之,曰:``以欲从人,则可;以人从欲,鲜济。」

\hypertarget{header-n868}{%
\subsubsection{僖公二十一年}\label{header-n868}}

【经】二十有一年春,狄侵卫。宋人、齐人、楚人盟于鹿上。夏,大旱。秋,宋公、楚子、陈侯、蔡侯、郑伯、许男、曹伯会于盂。执宋公以伐宋。冬,公伐邾。楚人使宜申来献捷。十有二月癸丑,公会诸侯盟于薄。释宋公。

【传】二十一年春,宋人为鹿上之盟,以求诸侯于楚。楚人许之。公子目夷曰:``小国争盟,祸也。宋其亡乎,幸而后败。」

夏,大旱。公欲焚巫兀。臧文仲曰:``非旱备也。修城郭,贬食省用,务穑劝分,此其务也。巫兀何为?天欲杀之,则如勿生;若能为旱,焚之滋甚。」公从之。是岁也,饥而不害。

秋,诸侯会宋公于盂。子鱼曰:``祸其在此乎!君欲已甚,其何以堪之?」于是楚执宋公以伐宋。

冬,会于薄以释之。子鱼曰:``祸犹未也,未足以惩君。」

任、宿、须句、颛臾,风姓也。实司大皞与有济之祀,以服事诸夏。邾人灭须句,须句子来奔,因成风也。成风为之言于公曰:``崇明祀,保小寡,周礼也;蛮夷猾夏,周祸也。若封须句,是崇皞、济而修祀,纾祸也。」

\hypertarget{header-n877}{%
\subsubsection{僖公二十二年}\label{header-n877}}

【经】二十有二年春,公伐邾,取须句。夏,宋公、卫侯、许男、滕子伐郑。秋八月丁未,及邾人战于升陉。冬十有一月己巳朔,宋公及楚人战于泓,宋师败绩。

【传】二十二年春,伐邾,取须句,反其君焉,礼也。

三月,郑伯如楚。

夏,宋公伐郑。子鱼曰:``所谓祸在此矣。」

初,平王之东迁也,辛有适伊川,见被发而祭于野者,曰:``不及百年,此其戎乎!其礼先亡矣。」秋,秦、晋迁陆浑之戎于伊川。

晋大子圉为质于秦,将逃归,谓嬴氏曰:``与子归乎?」对曰:``子,晋大子,而辱于秦,子之欲归,不亦宜乎?寡君之使婢子侍执巾栉,以固子也。从子而归,弃君命也。不敢从,亦不敢言。」遂逃归。

富辰言于王曰:``请召大叔。《诗》曰:『协比其邻,昏姻孔云。』吾兄弟之不协,焉能怨诸侯之不睦?」王说。王子带自齐复归于京师,王召之也。

邾人以须句故出师。公卑邾,不设备而御之。臧文仲曰:``国无小,不可易也。无备,虽众不可恃也。《诗》曰:『战战兢兢,如临深渊,如履薄冰。』又曰:『敬之敬之,天惟显思,命不易哉!』先王之明德,犹无不难也,无不惧也,况我小国乎!君其无谓邾小。蜂虿有毒,而况国乎?」弗听。

八月丁未,公及邾师战于升陉,我师败绩。邾人获公胄,县诸鱼门。

楚人伐宋以救郑。宋公将战,大司马固谏曰:``天之弃商久矣,君将兴之,弗可赦也已。」弗听,

冬十一月己巳朔,宋公及楚人战于泓。宋人既成列,楚人未既济。司马曰:``彼众我寡,及其未既济也请击之。」公曰:``不可。」既济而未成列,又以告。公曰:``未可。」既陈而后击之,宋师败绩。公伤股,门官歼焉。

国人皆咎公。公曰:``君子不重伤,不禽二毛。古之为军也,不以阻隘也。寡人虽亡国之馀,不鼓不成列。」子鱼曰:``君未知战。勍敌之人隘而不列,天赞我也。阻而鼓之,不亦可乎?犹有惧焉。且今之勍者,皆吾敌也。虽及胡《老司》,获则取之,何有于二毛?明耻教战,求杀敌也,伤未及死,如何勿重?若受重伤,则如勿伤;爱其二毛,则如服焉。三军以利用也,金鼓以声气也。利而用之,阻隘可也;声盛致志,鼓儳可也。」

丙子晨,郑文夫人芈氏、姜氏劳楚子于柯泽。楚子使师缙示之俘馘。君子曰:``非礼也。妇人送迎不出门,见兄弟不逾阈,戎事不迩女器。」

丁丑,楚子入飨于郑,九献,庭实旅百,加笾豆六品。飨毕,夜出,文芈送于军,取郑二姬以归。叔詹曰:``楚王其不没乎!为礼卒于无别,无别不可谓礼,将何以没?」诸侯是以知其不遂霸也。

\hypertarget{header-n894}{%
\subsubsection{僖公二十三年 }\label{header-n894}}

【经】二十有三年春,齐侯伐宋,围婚。夏五月庚寅,宋公兹父卒。秋,楚人伐陈。冬十有一月,杞子卒。

【传】二十三年春,齐侯伐宋,围缗,以讨其不与盟于齐也。

夏五月,宋襄公卒,伤于泓故也。

秋,楚成得臣帅师伐陈,讨其贰于宋也。遂取焦、夷,城顿而还。子文以为之功,使为令尹。叔伯曰:``子若国何?」对曰:``吾以靖国也。夫有大功而无贵仕,其人能靖者与有几?」

九月,晋惠公卒。怀公命无从亡人。期,期而不至,无赦。狐突之子毛及偃从重耳在秦,弗召。冬,怀公执狐突曰:``子来则免。」对曰:``子之能仕,父教之忠,古之制也。策名委质,贰乃辟也。今臣之子,名在重耳,有年数矣。若又召之,教之贰也。父教子贰,何以事君?刑之不滥,君之明也,臣之愿也。淫刑以逞,谁则无罪?臣闻命矣。」乃杀之。

卜偃称疾不出,曰:``《周书》有之:『乃大明服。』己则不明而杀人以逞,不亦难乎?民不见德而唯戮是闻,其何后之有?」

十一月,杞成公卒。书曰``子」,杞,夷也。不书名,未同盟也。凡诸侯同盟,死则赴以名,礼也。赴以名,则亦书之,不然则否,辟不敏也。

晋公子重耳之及于难也,晋人伐诸蒲城。蒲城人欲战。重耳不可,曰:``保君父之命而享其生禄,于是乎得人。有人而校,罪莫大焉。吾其奔也。」遂奔狄。从者狐偃、赵衰、颠颉、魏武子、司空季子。狄人伐啬咎如,获其二女:叔隗、季隗,纳诸公子。公子取季隗,生伯儵、叔刘,以叔隗妻赵衰,生盾。将适齐,谓季隗曰:``待我二十五年,不来而后嫁。」对曰:``我二十五年矣,又如是而嫁,则就木焉。请待子。」处狄十二年而行。

过卫。卫文公不礼焉。出于五鹿,乞食于野人,野人与之块,公子怒,欲鞭之。子犯曰:``天赐也。」稽首,受而载之。

及齐,齐桓公妻之,有马二十乘,公子安之。从者以为不可。将行,谋于桑下。蚕妾在其上,以告姜氏。姜氏杀之,而谓公子曰:``子有四方之志,其闻之者吾杀之矣。」公子曰:``无之。」姜曰:』行也。怀与安,实败名。」公子不可。姜与子犯谋,醉而遣之。醒,以戈逐子犯。

及曹,曹共公闻其骈胁。欲观其裸。浴,薄而观之。僖负羁之妻曰:``吾观晋公子之从者,皆足以相国。若以相,夫子必反其国。反其国,必得志于诸侯。得志于诸侯而诛无礼,曹其首也。子盍蚤自贰焉。」乃馈盘飨,置璧焉。公子受飨反璧。

及宋,宋襄公赠之以马二十乘。

及郑,郑文公亦不礼焉。叔詹谏曰:``臣闻天之所启,人弗及也。晋公子有三焉,天其或者将建诸,君其礼焉。男女同姓,其生不蕃。晋公子,姬出也,而至于今,一也。离外之患,而天不靖晋国,殆将启之,二也。有三士足以上人而从之,三也。晋、郑同侪,其过子弟,固将礼焉,况天之所启乎?」弗听。

及楚,楚之飨之,曰:``公子若反晋国,则何以报不谷?」对曰:``子女玉帛则君有之,羽毛齿革则君地生焉。其波及晋国者,君之馀也,其何以报君?」曰:``虽然,何以报我?」对曰:``若以君之灵,得反晋国,晋、楚治兵,遇于中原,其辟君三舍。若不获命,其左执鞭弭、右属櫜健,以与君周旋。」子玉请杀之。楚子曰:``晋公子广而俭,文而有礼。其从者肃而宽,忠而能力。晋侯无亲,外内恶之。吾闻姬姓,唐叔之后,其后衰者也,其将由晋公子乎。天将兴之,谁能废之。违天必有大咎。」乃送诸秦。秦伯纳女五人,怀嬴与焉。奉也活盥,既而挥之。怒曰:``秦、晋匹也,何以卑我!」公子惧,降服而囚。

他日,公享之。子犯曰:``吾不如衰之文也。请使衰从。公子赋《河水》,公赋《六月》。赵衰曰:``重耳拜赐。」公子降,拜,稽首,公降一级而辞焉。衰曰:``君称所以佐天子者命重耳,重耳敢不拜。」

\hypertarget{header-n912}{%
\subsubsection{僖公二十四年}\label{header-n912}}

【经】二十有四年春王正月。夏,狄伐郑。秋七月。冬,天王出居于郑。晋侯夷吾卒。

【传】二十四年春,王正月,秦伯纳之,不书,不告入也。

及河,子犯以璧授公子,曰:``臣负羁绁从君巡于天下,臣之罪甚多矣。臣犹知之,而况君乎?请由此亡。」公子曰:``所不与舅氏同心者,有如白水。」投其璧于河。济河,围令狐,入桑泉,取臼衰。二月甲午,晋师军于庐柳。秦伯使公子絷如晋师,师退,军于郇。辛丑,狐偃及秦、晋之大夫盟于郇。壬寅,公子入于晋师。丙午,入于曲沃。丁未,朝于武宫。戊申,使杀怀公于高梁。不书,亦不告也。吕、郤畏逼,将焚公宫而弑晋侯。寺人披请见,公使让之,且辞焉,曰:``蒲城之役,君命一宿,女即至。其后余从狄君以田渭滨,女为惠公来求杀余,命女三宿,女中宿至。虽有君命,何其速也。夫祛犹在,女其行乎。」对曰:``臣谓君之入也,其知之矣。若犹未也,又将及难。君命无二,古之制也。除君之恶,唯力是视。蒲人、狄人,余何有焉。今君即位,其无蒲、狄乎?齐桓公置射钩而使管仲相,君若易之,何辱命焉?行者甚众,岂唯刑臣。」公见之,以难告。三月,晋侯潜会秦伯于王城。己丑晦,公宫火,瑕甥、郤芮不获公,乃如河上,秦伯诱而杀之。晋侯逆夫人嬴氏以归。秦伯送卫于晋三千人,实纪纲之仆。

初,晋侯之竖头须,守藏者也。其出也,窃藏以逃,尽用以求纳之。及入,求见,公辞焉以沐。谓仆人曰:``沐则心覆,心覆则图反,宜吾不得见也。居者为社稷之守,行者为羁绁之仆,其亦可也,何必罪居者?国君而仇匹夫,惧者甚众矣。」仆人以告,公遽见之。

狄人归季隗于晋而请其二子。文公妻赵衰,生原同、屏括、搂婴。赵姬请逆盾与其母,子余辞。姬曰:``得宠而忘旧,何以使人?必逆之!」固请,许之,来,以盾为才,固请于公以为嫡子,而使其三子下之,以叔隗为内子而己下之。

晋侯赏从亡者,介之推不言禄,禄亦弗及。推曰``献公之子九人,唯君在矣。惠、怀无亲,外内弃之。天未绝晋,必将有主。主晋祀者,非君而谁?天实置之,而二三子以为己力,不亦诬乎?窃人之财,犹谓之盗,况贪天之功以为己力乎?下义其罪,上赏其奸,上下相蒙,难与处矣!」其母曰:``盍亦求之,以死谁怼?」对曰:``尤而效之,罪又甚焉,且出怨言,不食其食。」其母曰:``亦使知之若何?」对曰:``言,身之文也。身将隐,焉用文之?是求显也。」其母曰:``能如是乎?与女偕隐。」遂隐而死。晋侯求之,不获,以绵上为之田,曰:``以志吾过,且旌善人。」

郑之入滑也,滑人听命。师还,又即卫。郑公子士、泄堵俞弥帅师伐滑。王使伯服、游孙伯如郑请滑。郑伯怨惠王之入而不与厉公爵也,又怨襄王之与卫、滑也,故不听王命而执二子。王怒,将以狄伐郑。富辰谏曰:``不可。臣闻之,大上以德抚民,其次亲亲以相及也。昔周公吊二叔之不咸,故封建亲戚以蕃屏周。管蔡郕霍,鲁卫毛聃,郜雍曹滕,毕原酆郇,文之昭也。邗晋应韩,武之穆也。凡蒋刑茅胙祭,周公之胤也。召穆公思周德之不类,故纠合宗族于成周而作诗,曰:『常棣之华,鄂不□韦□韦,凡今之人,莫如兄弟。』其四章曰:『兄弟阋于墙,外御其侮。』如是,则兄弟虽有小忿,不废懿亲。今天子不忍小忿以弃郑亲,其若之何?庸勋亲亲,昵近尊贤,德之大者也。即聋从昧,与顽用嚚,奸也大者也。弃德崇奸,祸之大者也。郑有平、惠之勋,又有厉、宣之亲,弃嬖宠而用三良,于诸姬为近,四德具矣。耳不听五声之和为聋,目不别五色之章为昧,心不则德义之经为顽,口不道忠信之言为嚚,狄皆则之,四奸具矣。周之有懿德也,犹曰『莫如兄弟』,故封建之。其怀柔天下也,犹惧有外侮,扞御侮者莫如亲亲,故以亲屏周。召穆公亦云。今周德既衰,于是乎又渝周、召以从诸奸,无乃不可乎?民未忘祸,王又兴之,其若文、武何?」王弗听,使颓叔、桃子出狄师。夏,狄伐郑,取栎。

王德狄人,将以其女为后。富辰谏曰:``不可。臣闻之曰:『报者倦矣,施者未厌。』狄固贪淋,王又启之,女德无极,妇怨无终,狄必为患。」王又弗听。

初,甘昭公有宠于惠后,惠后将立之,未及而卒。昭公奔齐,王复之,又通于隗氏。王替隗氏。颓叔、桃子曰:``我实使狄,狄其怨我。」遂奉大叔,以狄师攻王。王御士将御之。王曰:``先后其谓我何?宁使诸。侯图之。璲出。及坎□,国人纳之。

秋,颓叔、桃子奉大叔,以狄师伐周,大败周师,获周公忌父、原伯、毛伯、富辰。王出适郑,处于汜。大叔以隗氏居于温。

郑子华之弟子臧出奔宋,好聚鹬冠。郑伯闻而恶之,使盗诱之。八月,盗杀之于陈、宋之间。君子曰:``服之不衷,身之灾也。《诗》曰:『彼己之子,不称其服。』子臧之服,不称也夫。《诗》曰,『自诒伊戚』,其子臧之谓矣。《夏书》曰,『地平天成』,称也。」

宋及楚平。宋成公如楚,还入于郑。郑伯将享之,问礼于皇武子。对曰:``宋,先代之后也,于周为客,天子有事膰焉,有丧拜焉,丰厚可也。」郑伯从之,享宋公有加,礼也。

冬,王使来告难曰:``不谷不德,得罪于母弟之宠子带,鄙在郑地汜,敢告叔父。」臧文仲对曰:``天子蒙尘于外,敢不奔问官守。」王使简师父告于晋,使左鄢父告于秦。天子无出,书曰``天王出居于郑」,辟母弟之难也。天子凶服降名,礼也。郑伯与孔将鉏、石甲父、侯宣多省视官具于汜,而后听其私政,礼也。

卫人将伐邢,礼至曰:``不得其守,国不可得也。我请昆弟仕焉。」乃往,得仕。

\hypertarget{header-n929}{%
\subsubsection{僖公二十五年}\label{header-n929}}

【经】二十有五年春王正月,丙午,卫侯毁灭邢。夏四月癸酉,卫侯毁卒。宋荡伯来逆妇。宋杀其大夫。秋,楚人围陈,纳顿子于顿。葬卫文公。冬十有二月癸亥,公会卫子、莒庆盟于洮。

【传】二十五年春,卫人伐邢,二礼从国子巡城,掖以赴外,杀之。正月丙午,卫侯毁灭邢,同姓也,故名。礼至为铭曰:``余掖杀国子,莫余敢止。」

秦伯师于河上,将纳王。狐偃言于晋侯曰:``求诸侯,莫如勤王。诸侯信之,且大义也。继文之业而信宣于诸侯,今为可矣。」使卜偃卜之,曰:``吉。遇黄帝战于阪泉之兆。」公曰:``吾不堪也。」对曰:``周礼未改。今之王,古之帝也。」公曰:``筮之。」筮之,遇《大有》ⅵⅰ之《睽》ⅵⅷ,曰:``吉。遇『公用享于天子』之卦也。战克而王飨,吉孰大焉,且是卦也,天为泽以当日,天子降心以逆公,不亦可乎?《大有》去《睽》而复,亦其所也。」晋侯辞秦师而下。三月甲辰,次于阳樊。右师围温,左师逆王。夏四月丁巳,王入于王城,取大叔于温,杀之于隰城。

戊午,晋侯朝王,王飨醴,命之宥。请隧,弗许,曰:``王章也。未有代德而有二王,亦叔父之所恶也。」与之阳樊、温、原、欑茅之田。晋于是始启南阳。

阳樊不服,围之。苍葛呼曰:``德以柔中国,邢以威四夷,宜吾不敢服也。此谁非王之亲姻,其俘之也!」乃出其民。

秋,秦、晋伐鄀。楚斗克、屈御寇以申、息之师戍商密。秦人过析隈,入而系舆人以围商密,昏而傅焉。宵,坎血加书,伪与子仪、子边盟者。商密人惧曰:``秦取析矣,戍人反矣。」乃降秦师。囚申公子仪、息公子边以归。楚令尹子玉追秦师,弗及,遂围陈,纳顿子于顿。

冬,晋侯围原,命三日之粮。原不降,命去之。谍出,曰:``原将降矣。」军吏曰:``请待之。」公曰:``信,国之宝也,民之所庇也,得原失信,何以庇之?所亡滋多。」退一舍而原降。迁原伯贯于冀。赵衰为原大夫,狐溱为温大夫。

卫人平莒于我,十二月,盟于洮,修卫文公之好,且及莒平也。

晋侯问原守于寺人勃鞮,对曰:``昔赵衰以壶飧从径,馁而弗食。」故使处原。

\hypertarget{header-n941}{%
\subsubsection{僖公二十六年}\label{header-n941}}

【经】二十有六年春王正月,己未,公会莒子、卫宁速盟于向。齐人侵我西鄙,公追齐师,至酅,不及。夏,齐人伐我北鄙。卫人伐齐。公子遂如楚乞师。秋,楚人灭夔,以夔子归。冬,楚人伐宋,围婚。公以楚师伐齐,取谷。公至自伐齐。

【传】二十六年春,王正月,公会莒兹ぶ宁庄子盟于向,寻洮之盟也。齐师侵我西鄙,讨是二盟也。夏,齐孝公伐我北鄙。卫人伐齐,洮之盟故也。公使展喜犒师,使受命于展禽。

齐侯未入竟,展喜从之,曰:``寡君闻君亲举玉趾,将辱于敝邑,使下臣犒执事。」齐侯曰:``鲁人恐乎?」对曰:``小人恐矣,君子则否。」齐侯曰:``室如县罄,野无青草,何恃而不恐?」对曰:``恃先王之命。昔周公、大公股肱周室,夹辅成王。成王劳之而赐之盟,曰:『世世子孙,无相害也。』载在盟府,大师职之。桓公是以纠合诸侯而谋其不协,弥缝其阙而匡救其灾,昭旧职也。及君即位,诸侯之望曰:『其率桓之功。』我敝邑用不敢保聚,曰:『岂其嗣世九年而弃命废职,其若先君何?』君必不然。恃此以不恐。」齐侯乃还。

东门襄仲、臧文仲如楚乞师,臧孙见子玉而道之伐齐、宋,以其不臣也。

夔子不祀祝融与鬻熊,楚人让之,对曰:``我先王熊挚有疾,鬼神弗赦而自窜于夔。吾是以失楚,又何祀焉?」秋,楚成得臣、斗宜申帅师灭夔,以夔子归。

宋以其善于晋侯也,叛楚即晋。冬,楚令尹子玉、司马子西帅师伐宋,围缗。

公以楚师伐齐,取谷。凡师能左右之曰以。置桓公子雍于谷,易牙奉之以为鲁援。楚申公叔侯戍之。桓公之子七人,为七大夫于楚。

\hypertarget{header-n951}{%
\subsubsection{僖公二十七年}\label{header-n951}}

【经】二十有七年春,杞子来朝。夏六月庚寅,齐侯昭卒。秋八月乙未,葬齐孝公。乙巳,公子遂帅师入杞。冬,楚人、陈侯、蔡侯、郑伯、许男围宋。十有二月甲戌,公会诸侯,盟于宋。

【传】二十七年春,杞桓公来朝,用夷礼,故曰子。公卑杞,杞不共也。

夏,齐孝公卒。有齐怨,不废丧纪,礼也。

秋,入杞,责无礼也。

楚子将围宋,使子文治兵于睽,终朝而毕,不戮一人。子玉复治兵于蒍,终日而毕,鞭七人,贯三人耳。国老皆贺子文,子文饮之酒。蒍贾尚幼,后至,不贺。子文问之,对曰:``不知所贺。子之传政于子玉,曰:『以靖国也。』靖诸内而败诸外,所获几何?子玉之败,子之举也。举以败国,将何贺焉?子玉刚而无礼,不可以治民。过三百乘,其不能以入矣。苟入而贺,何后之有?」

冬,楚子及诸侯围宋,宋公孙固如晋告急。先轸曰:``报施救患,取威定霸,于是乎在矣。」狐偃曰:``楚始得曹而新昏于卫,若伐曹、卫,楚必救之,则齐、宋免矣。」于是乎蒐于被庐,作三军。谋元帅。赵衰曰:``郤縠可。臣亟闻其言矣,说礼乐而敦《诗》《书》。《诗》、《书》,义之府也。礼乐,德之则也。德义,利之本也。《夏书》曰:『赋纳以言,明试以功,车服以庸。』君其试之。」及使郤縠将中军,郤溱佐之;使狐偃将上军,让于狐毛,而佐之;命赵衰为卿,让于栾枝、先轸。使栾枝将下军,先轸佐之。荀林父御戎,魏准为右。

晋侯始入而教其民,二年,欲用之。子犯曰:``民未知义,未安其居。」于是乎出定襄王,入务利民,民怀生矣,将用之。子犯曰:``民未知信,未宣其用。」于是乎伐原以示之信。民易资者不求丰焉,明征其辞。公曰:``可矣乎?」子犯曰:``民未知礼,未生其共。」于是乎大蒐以示之礼,作执秩以正其官,民听不惑而后用之。出谷戍,释宋围,一战而霸,文之教也。

\hypertarget{header-n961}{%
\subsubsection{僖公二十八年}\label{header-n961}}

【经】二十有八年春,晋侯侵曹,晋侯伐卫。公子买戍卫,不卒戍,刺之。楚人救卫。三月丙午,晋侯入曹,执曹伯。畀宋人。夏四月己巳,晋侯、齐师、宋师、秦师及楚人战于城濮,楚师败绩。楚杀其大夫得臣。卫侯出奔楚。五月癸丑,公会晋侯、齐侯、宋公、蔡侯、郑伯、卫子、莒子,盟于践土。陈侯如会。公朝于王所。六月,卫侯郑自楚复归于卫。卫元咺出奔晋。陈侯款卒。秋,杞伯姬来。公子遂如齐。冬,公会晋侯、齐侯、宋公、蔡侯、郑伯、陈子、莒子、邾人、秦人于温。天王狩于河阳。壬申,公朝于王所。晋人执卫侯,归之于京师。卫元咺自晋复归于卫。诸侯遂围许。曹伯襄复归于曹,遂会诸侯围许。

【传】二十八年春,晋侯将伐曹,假道于卫,卫人弗许。还,自南河济。侵曹伐卫。正月戊申,取五鹿。二月,晋郤縠卒。原轸将中军,胥臣佐下军,上德也。晋侯、齐侯盟于敛盂。卫侯请盟,晋人弗许。卫侯欲与楚,国人不欲,故出其君以说于晋。卫侯出居于襄牛。

公子买戍卫,楚人救卫,不克。公惧于晋,杀子丛以说焉。谓楚人曰:``不卒戍也。」

晋侯围曹,门焉,多死,曹人尸诸城上,晋侯患之,听舆人之谋曰称:``舍于墓。」师迁焉,曹人凶惧,为其所得者棺而出之,因其凶也而攻之。三月丙午,入曹。数之,以其不用僖负羁而乘轩者三百人也。且曰:``献状。」令无入僖负羁之宫而免其族,报施也。魏准、颠颉怒曰:``劳之不图,报于何有!」蓺僖负羁氏。魏准伤于胸,公欲杀之而爱其材,使问,且视之。病,将杀之。魏准束胸见使者曰:``以君之灵,不有宁也。」距跃三百,曲踊三百。乃舍之。杀颠颉以徇于师,立舟之侨以为戎右。

宋人使门尹般如晋师告急。公曰:``宋人告急,舍之则绝,告楚不许。我欲战矣,齐、秦未可,若之何?」先轸曰:``使宋舍我而赂齐、秦,藉之告楚。我执曹君而分曹、卫之田以赐宋人。楚爱曹、卫,必不许也。喜赂怒顽,能无战乎?」公说,执曹伯,分曹、卫之田以畀宋人。

楚子入居于申,使申叔去谷,使子玉去宋,曰:``无从晋师。晋侯在外十九年矣,而果得晋国。险阻艰难,备尝之矣;民之情伪,尽知之矣。天假之年,而除其害。天之所置,其可废乎?《军志》曰:『允当则归。』又曰:『知难而退。』又曰:『有德不可敌。』此三志者,晋之谓矣。」子玉使伯棼请战,曰:``非敢必有功也,愿以间执谗慝之口。」王怒,少与之师,唯西广、东宫与若敖之六卒实从之。

子玉使宛春告于晋师曰:``请复卫侯而封曹,臣亦释宋之围。」子犯曰:``子玉无礼哉!君取一,臣取二,不可失矣。」先轸曰:``子与之。定人之谓礼,楚一言而定三国,我一言而亡之。我则无礼,何以战乎?不许楚言,是弃宋也。救而弃之,谓诸侯何?楚有三施,我有三怨,怨仇已多,将何以战?不如私许复曹、卫以携之,执宛春以怒楚,既战而后图之。」公说,乃拘宛春于卫,且私许复曹、卫。曹、卫告绝于楚。

子玉怒,从晋师。晋师退。军吏曰:``以君辟臣,辱也。且楚师老矣,何故退?」子犯曰:``师直为壮,曲为老。岂在久乎?微楚之惠不及此,退三舍辟之,所以报也。背惠食言,以亢其仇,我曲楚直。其众素饱,不可谓老。我退而楚还,我将何求?若其不还,君退臣犯,曲在彼矣。」退三舍。楚众欲止,子玉不可。

夏四月戊辰,晋侯、宋公、齐国归父、崔夭、秦小子憖次于城濮。楚师背酅而舍,晋侯患之,听舆人之诵,曰:``原田每每,舍其旧而新是谋。」公疑焉。子犯曰:``战也。战而捷,必得诸侯。若其不捷,表里山河,必无害也。」公曰:``若楚惠何?」栾贞子曰:``汉阳诸姬,楚实尽之,思小惠而忘大耻,不如战也。」晋侯梦与楚子搏,楚子伏己而监其脑,是以惧。子犯曰:``吉。我得天,楚伏其罪,吾且柔之矣。」

子玉使斗勃请战,曰:``请与君之士戏,君冯轼而观之,得臣与寓目焉。」晋侯使栾枝对曰:``寡君闻命矣。楚君之惠未之敢忘,是以在此。为大夫退,其敢当君乎?既不获命矣,敢烦大夫谓二三子,戒尔车乘,敬尔君事,诘朝将见。」

晋车七百乘,革显、革引、鞅、革半。晋侯登有莘之虚以观师,曰:``少长有礼,其可用也。」遂伐其木以益其兵。鲁巳,晋师陈于莘北,胥臣以下军之佐当陈、蔡。子玉以若敖六卒将中军,曰:``今日必无晋矣。」子西将左,子上将右。胥臣蒙马以虎皮,先犯陈、蔡。陈、蔡奔,楚右师溃。狐毛设二旆而退之。栾枝使舆曳柴而伪遁,楚师驰之。原轸、郤溱以中军公族横击之。狐毛、狐偃以上军夹攻子西,楚左师溃。楚师败绩。子玉收其卒而止,故不败。

晋师三日馆谷,及癸酉而还。甲午,至于衡雍,作王宫于践土。

乡役之三月,郑伯如楚致其师,为楚师既败而惧,使子人九行成于晋。晋栾枝入盟郑伯。五月丙午,晋侯及郑伯盟于衡雍。丁未,献楚俘于王,驷介百乘,徒兵千。郑伯傅王,用平礼也。己酉,王享醴,命晋侯宥。王命尹氏及王子虎、内史叔兴父策命晋侯为侯伯,赐之大辂之服,戎辂之服,彤弓一,彤矢百,玈弓矢千,秬鬯一卣,虎贲三百人。曰:``王谓叔父,敬服王命,以绥四国。纠逖王慝。」晋侯三辞,从命。曰:``重耳敢再拜稽首,奉扬天子之丕显休命。」受策以出,出入三觐。

卫侯闻楚师败,惧,出奔楚,遂适陈,使元咺奉叔武以受盟。癸亥,王子虎盟诸侯于王庭,要言曰:``皆奖王室,无相害也。有渝此盟,明神殛之,俾队其师,无克祚国,及而玄孙,无有老幼。」君子谓是盟也信,谓晋于是役也能以德攻。

初,楚子玉自为琼弁玉缨,未之服也。先战,梦河神谓己曰:``畀余,余赐女孟诸之麋。」弗致也。大心与子西使荣黄谏,弗听。荣季曰:``死而利国。犹或为之,况琼玉乎?是粪土也,而可以济师,将何爱焉?」弗听。出,告二子曰:``非神败令尹,令尹其不勤民,实自败也。」既败,王使谓之曰:``大夫若入,其若申、息之老何?」子西、孙伯曰:``得臣将死,二臣止之曰:『君其将以为戮。』」及连谷而死。晋侯闻之而后喜可知也,曰:``莫馀毒也已!蒍吕臣实为令尹,奉己而已,不在民矣。」

或诉元咺于卫侯曰:``立叔武矣。」其子角从公,公使杀之。咺不废命,奉夷叔以入守。

六月,晋人复卫侯。宁武子与卫人盟于宛濮,曰:``天祸卫国,君君臣不协,以及此忧也。今天诱其衷,使皆降心以相从也。不有居者,谁守社稷?不有行者,谁扞牧圉?不协之故,用昭乞盟于尔大神以诱天衷。自今日以往,既盟之后,行者无保其力,居者无惧其罪。有渝此盟,以相及也。明神先君,是纠是殛。」国人闻此盟也,而后不贰。卫侯先期入,宁子先,长佯守门以为使也,与之乘而入。公子颛犬、华仲前驱。叔孙将沐,闻君至,喜,捉发走出,前驱射而杀之。公知其无罪也,枕之股而哭之。颛犬走出,公使杀之。元咺出奔晋。

城濮之战,晋中军风于泽,亡大旆之左旃。祁瞒奸命,司马杀之,以徇于诸侯,使茅伐代之。师还。壬午,济河。舟之侨先归,士会摄右。秋七月丙申,振旅,恺以入于晋。献俘授馘,饮至大赏,征会讨贰。杀舟之侨以徇于国,民于是大服。

君子谓:``文公其能刑矣,三罪而民服。《诗》云:『惠此中国,以绥四方。』不失赏刑之谓也。」

冬,会于温,讨不服也。

卫侯与元咺讼,宁武子为辅,金咸庄子为坐,士荣为大士。卫侯不胜。杀士荣,刖金咸庄子,谓宁俞忠而免之。执卫侯,归之于京师,置诸深室。宁子职纳橐饘焉。元咺归于卫,立公子瑕。

是会也,晋侯召王,以诸侯见,且使王狩。仲尼曰:``以臣召君,不可以训。」故书曰:``天王狩于河阳。」言非其地也,且明德也。

壬申,公朝于王所。

丁丑,诸侯围许。

晋侯有疾,曹伯之竖侯孺货筮史,使曰:``以曹为解。齐桓公为会而封异姓,今君为会而灭同姓。曹叔振铎,文之昭也。先君唐叔,武之穆也。且合诸侯而灭兄弟,非礼也。与卫偕命,而不与偕复,非信也。同罪异罚,非刑也。礼以行义,信以守礼,刑以正邪,舍此三者,君将若之何?」公说,复曹伯,遂会诸侯于许。

晋侯作三行以御狄,荀林父将中行,屠击将右行,先蔑将左行。

\hypertarget{header-n990}{%
\subsubsection{僖公二十九年}\label{header-n990}}

【经】二十有九年春,介葛卢来。公至自围许。夏六月,会王人、晋人、宋人、齐人、陈人、蔡人、秦人盟于翟泉。秋,大雨雹。冬,介葛卢来。

【传】二十九年春,葛卢来朝,舍于昌衍之上。公在会,馈之刍米,礼也。

夏,公会王子虎、晋狐偃、宋公孙固、齐国归父、陈辕涛涂、秦小子憖,盟于翟泉,寻践土之盟,且谋伐郑也。卿不书,罪之也。在礼,卿不会公、侯,会伯、子、男可也。

秋,大雨雹,为灾也。

冬,介葛卢来,以未见公,故复来朝,礼之,加燕好。

介葛卢闻牛鸣,曰:``是生三牺,皆用之矣,其音云。」问之而信。

\hypertarget{header-n999}{%
\subsubsection{僖公三十年}\label{header-n999}}

【经】三十年春王正月。夏,狄侵齐。秋,卫杀其大夫元咺及公子瑕。卫侯郑归于卫。晋人、秦人围郑。介人侵萧。冬,天王使宰周公来聘。公子遂如京师。遂如晋。

【传】三十年春,晋人侵郑,以观其可攻与否。狄间晋之有郑虞也,夏,狄侵齐。

晋侯使医衍鸩卫侯。宁俞货医,使薄其鸩,不死。公为之请,纳玉于王与晋侯。皆十□王许之。秋,乃释卫侯。卫侯使赂周颛、治廑,曰:``苟能纳我,吾使尔为卿。」周、冶杀元咺及子适、子仪。公入祀先君。周、冶既服将命,周颛先入,及门,遇疾而死。冶廑辞卿。

九月甲午,晋侯、秦伯围郑,以其无礼于晋,且贰于楚也。晋军函陵,秦军汜南。佚之狐言于郑伯曰:``国危矣,若使烛之武见秦君,师必退。」公从之。辞曰:``臣之壮也,犹不如人,今老矣,无能为也已。」公曰:``吾不能早用子,今急而求子,是寡人之过也。然郑亡,子亦有不利焉。」许之,夜缒而出,见秦伯,曰:``秦、晋围郑,郑既知亡矣。若亡郑而有益于君,敢以烦执事。越国以鄙远,君知其难也,焉用亡郑以陪邻。邻之厚,君之薄也。若舍郑以为东道主,行李之往来,共其乏困,君亦无所害。且君尝为晋君赐矣,许君焦、瑕,朝济而夕设版焉,君之所知也。夫晋何厌之有?既东封郑,又欲肆其西封,不阙秦,将焉取之?阙秦以利晋,唯君图之。」秦伯说,与郑人盟,使杞子、逢孙、扬孙戍之,乃还。

子犯请击之,公曰:``不可。微夫人力不及此。因人之力而敝之,不仁。失其所与,不知。以乱易整,不武。吾其还也。」亦去之。

初,郑公子兰出奔晋,从于晋侯。伐郑,请无与围郑。许之,使待命于东。郑石甲父、侯宣多逆以为大子,以求成于晋,晋人许之。

冬,王使周公阅来聘,飨有昌蜀、白、黑、形盐。辞曰:``国君,文足昭也,武可畏也,则有备物之飨以象其德。荐五味,羞嘉谷,盐虎形,以献其功。吾何以堪之?」

东门襄仲将聘于周,遂初聘于晋。

\hypertarget{header-n1010}{%
\subsubsection{僖公三十一年}\label{header-n1010}}

【经】三十有一年春,取济西田。公子遂如晋。夏四月,四卜郊,不从,乃免牲。犹三望。秋七月。冬,杞伯姬来求妇。狄围卫。十有二月,卫迁于帝丘。

【传】三十一年春,取济西田,分曹地也。使臧文仲往,宿于重馆。重馆人告曰:``晋新得诸侯,必亲其共,不速行,将无及也。」从之,分曹地,自洮以南,东傅于济,尽曹地也。

襄仲如晋,拜曹田也。

夏四月,四卜郊,不从,乃免牲,非礼也。犹三望,亦非礼也。礼不卜常祀,而卜其牲、日,牛卜日曰牲。牲成而卜郊,上怠慢也。望,郊之细也。不郊,亦无望可也。

秋,晋搜于清原,作五军御狄。赵衰为卿。

冬,狄围卫,卫迁于帝丘。卜曰三百年。卫成公梦康叔曰:``相夺予享。」公命祀相。宁武子不可,曰:``鬼神非其族类,不歆其祀。杞、鄫何事?相之不享于此。久矣,非卫之罪也,不可以间成王、周公之命祀。请改祀命。」

郑泄驾恶公子瑕,郑伯亦恶之,故公子瑕出奔楚。

\hypertarget{header-n1020}{%
\subsubsection{僖公三十二年}\label{header-n1020}}

【经】三十有二年春王正月。夏四月己丑,郑伯捷卒。卫人侵狄。秋,卫人及狄盟。冬十有二月己卯,晋侯重耳卒。

【传】三十二年春,楚斗章请平于晋,晋阳处父报之。晋、楚始通。

夏,狄有乱。卫人侵狄,狄请平焉。秋,卫人及狄盟。

冬,晋文公卒。庚辰,将殡于曲沃,出绛,柩有声如牛。卜偃使大夫拜。曰:``君命大事。将有西师过轶我,击之,必大捷焉。」杞子自郑使告于秦,曰:``郑人使我掌其北门之管,若潜师以来,国可得也。」穆公访诸蹇叔,蹇叔曰:``劳师以袭远,非所闻也。师劳力竭,远主备之,无乃不可乎!师之所为,郑必知之。勤而无所,必有悖心。且行千里,其谁不知?」公辞焉。召孟明、西乞、白乙,使出师于东门之外。蹇叔哭之,曰:``孟子,吾见师之出而不见其入也。」公使谓之曰:``尔何知?中寿,尔墓之木拱矣。」蹇叔之子与师,哭而送之,曰:``晋人御师必于殽。殽有二陵焉。其南陵,夏后皋之墓也;其北陵,文王之所辟风雨也。必死是间,余收尔骨焉。」秦师遂东。

\hypertarget{header-n1027}{%
\subsubsection{僖公三十三年}\label{header-n1027}}

【经】三十有三年春王二月,秦人入滑。齐侯使国归父来聘。夏四月辛巳,晋人及姜戎败秦师于殽。癸巳,葬晋文公。狄侵齐。公伐邾,取訾娄。秋,公子遂帅师伐邾。晋人败狄于箕。冬十月,公如齐。十有二月,公至自齐。乙巳,公薨于小寝。陨霜不杀草。李梅实。晋人、陈人、郑人伐许。

【传】三十三年春,秦师过周北门,左右免胄而下。超乘者三百乘。王孙满尚幼,观之,言于王曰:``秦师轻而无礼,必败。轻则寡谋,无礼则脱。入险而脱。又不能谋,能无败乎?」及滑,郑商人弦高将市于周,遇之。以乘韦先,牛十二犒师,曰:``寡君闻吾子将步师出于敝邑,敢犒从者,不腆敝邑,为从者之淹,居则具一日之积,行则备一夕之卫。」且使遽告于郑。

郑穆公使视客馆,则束载、厉兵、秣马矣。使皇武子辞焉,曰:``吾子淹久于敝邑,唯是脯资饩牵竭矣。为吾子之将行也,郑之有原圃,犹秦之有具囿也。吾子取其麋鹿以闲敝邑,若何?」杞子奔齐,逢孙、扬孙奔宋。孟明曰:``郑有备矣,不可冀也。攻之不克,围之不继,吾其还也。」灭滑而还。

齐国庄子来聘,自郊劳至于赠贿,礼成而加之以敏。臧文仲言于公曰:``国子为政,齐犹有礼,君其朝焉。臣闻之,服于有礼,社稷之卫也。」

晋原轸曰:``秦违蹇叔,而以贪勤民,天奉我也。奉不可失,敌不可纵。纵敌患生,违天不祥。必伐秦师。」栾枝曰:``未报秦施而伐其师,其为死君乎?」先轸曰:``秦不哀吾丧而伐吾同姓,秦则无礼,何施之为?吾闻之,一日纵敌,数世之患也。谋及子孙,可谓死君乎?」遂发命,遽兴姜戎。子墨衰絰,梁弘御戎,莱驹为右。

夏四月辛巳,败秦师于殽,获百里孟明视、西乞术、白乙丙以归,遂墨以葬文公。晋于是始墨。

文嬴请三帅,曰:``彼实构吾二君,寡君若得而食之,不厌,君何辱讨焉!使归就戮于秦,以逞寡君之志,若何?」公许之,先轸朝。问秦囚。公曰:``夫人请之,吾舍之矣。」先轸怒曰:``武夫力而拘诸原,妇人暂而免诸国。堕军实而长寇仇,亡无日矣。」不顾而唾。公使阳处父追之,及诸河,则在舟中矣。释左骖,以公命赠孟明。孟明稽首曰:``君之惠,不以累臣衅鼓,使归就戮于秦,寡君之以为戮,死且不朽。若从君惠而免之,三年将拜君赐。」

秦伯素服郊次,乡师而哭曰:``孤违蹇叔以辱二三子,孤之罪也。不替孟明,孤之过也。大夫何罪?且吾不以一眚掩大德。」

狄侵齐,因晋丧也。

公伐邾,取訾娄,以报升陉之役。邾人不设备。秋,襄仲复伐邾。

狄伐晋,及箕。八月戊子,晋侯败狄于箕。郤缺获白狄子。先轸曰:``匹夫逞志于君而无讨,敢不自讨乎?」免胄入狄师,死焉。狄人归其元,面如生。

初,臼季使过冀,见冀缺耨,其妻盍之。敬,相待如宾。与之归,言诸文公曰:``敬,德之聚也。能敬必有德,德以治民,君请用之。臣闻之,出门如宾,承事如祭,仁之则也。」公曰:``其父有罪,可乎?」对曰:``舜之罪也殛鲧,其举也兴禹。管敬仲,桓之贼也,实相以济。《康诰》曰:『父不慈,子不祗,兄不友,弟不共,不相及也。』《诗》曰:『采葑采菲,无以下体。』君取节焉可也。」文公以为下军大夫。反自箕,襄公以三命命先且居将中军,以再命命先茅之县赏胥臣曰:``举郤缺,子之功也。」以一命命郤缺为卿,复与之冀,亦未有军行。

冬,公如齐,朝,且吊有狄师也。反,薨于小寝,即安也。

晋、陈、郑伐许,讨其贰于楚也。

楚令尹子上侵陈、蔡。陈、蔡成,遂伐郑,将纳公子瑕,门于桔柣之门。瑕覆于周氏之汪,外仆髡屯禽之以献。文夫人敛而葬之郐城之下。

晋阳处父侵蔡,楚子上救之,与晋师夹泜而军。阳子患之,使谓子上曰:``吾闻之,文不犯顺,武不违敌。子若欲战,则吾退舍,子济而陈,迟速唯命,不然纾我。老师费财,亦无益也。」乃驾以待。子上欲涉,大孙伯曰:``不可。晋人无信,半涉而薄我,悔败何及,不如纾之。」乃退舍。阳子宣言曰:``楚师遁矣。」遂归。楚师亦归。大子商臣谮子上曰:``受晋赂而辟之,楚之耻也,罪莫大焉。」王杀子上。

葬僖公,缓作主,非礼也。凡君薨,卒哭而祔,祔而作主,特祀于主,烝尝禘于庙。

\hypertarget{header-n1046}{%
\subsection{文公}\label{header-n1046}}

\begin{center}\rule{0.5\linewidth}{\linethickness}\end{center}

\hypertarget{header-n1048}{%
\subsubsection{文公元年}\label{header-n1048}}

【经】元年春王正月,公即位。二月癸亥,日有食之。天王使叔服来会葬。夏四月丁巳,葬我君僖公。天王使毛伯来锡公命。晋侯伐卫。叔孙得臣如京师。卫人伐晋。秋,公孙敖会晋侯于戚。冬十月丁未,楚世子商臣弑其君頵。公孙敖如齐。

【传】元年春,王使内史叔服来会葬。公孙敖闻其能相人也,见其二子焉。叔服曰:``谷也食子,难也收子。谷也丰下,必有后于鲁国。」

于是闰三月,非礼也。先王之正时也,履端于始,举正于中,归余于终。履端于始,序则不愆。举正于中,民则不惑。归余于终,事则不悖。

夏四月丁巳,葬僖公。

王使毛伯卫来锡公命。叔孙得臣如周拜。

晋文公之季年,诸侯朝晋。卫成公不朝,使孔达侵郑,伐绵、訾,及匡。晋襄公既祥,使告于诸侯而伐卫,及南阳。先且居曰:``效尤,祸也。请君朝王,臣从师。」晋侯朝王于温,先且居、胥臣伐卫。五月辛酉朔,晋师围戚。六月戊戌,取之,获孙昭子。

卫人使告于陈。陈共公曰:``更伐之,我辞之。」卫孔达帅师伐晋,君子以为古。古者越国而谋。

秋,晋侯疆戚田,故公孙敖会之。

初,楚子将以商臣为大子,访诸令尹子上。子上曰:``君之齿未也。而又多爱,黜乃乱也。楚国之举。恒在少者。且是人也。蜂目而豺声,忍人也,不可立也。」弗听。既又欲立王子职而黜大子商臣。商臣闻之而未察,告其师潘崇曰:``若之何而察之?」潘崇曰:``享江问而勿敬也。」从之。江芈怒曰:``呼,役夫!宜君王之欲杀女而立职也。」告潘崇曰:``信矣。」潘崇曰:``能事诸乎?」曰:``不能。」``能行乎?」曰:``不能。」``能行大事乎?」曰:``能。」

冬十月,以宫甲围成王。王请食熊蹯而死。弗听。丁未,王缢。谥之曰:``灵」,不瞑;曰:``成」,乃瞑。穆王立,以其为大子之室与潘崇,使为大师,且掌环列之尹。

穆伯如齐,始聘焉,礼也。凡君即位,卿出并聘,践修旧好,要结外授,好事邻国,以卫社稷,忠信卑让之道也。忠,德之正也;信,德之固也;卑让,德之基也。

殽之役,晋人既归秦帅,秦大夫及左右皆言于秦伯曰:``是败也,孟明之罪也,必杀之。」秦伯曰:``是孤之罪也。周芮良夫之诗曰;『大风有隧,贪人败类,听言则对,诵言如醉,匪用其良,覆俾我悖。』是贪故也,孤之谓矣。孤实贪以祸夫子,夫子何罪?」复使为政。

\hypertarget{header-n1063}{%
\subsubsection{文公二年}\label{header-n1063}}

【经】二年春王二月甲子,晋侯及秦师战于彭衙,秦师败绩。丁丑,作僖公主。三月乙巳,及晋处父盟。夏六月,公孙敖会宋公、陈侯、郑伯、晋士縠盟于垂陇。自十有二月不雨,至于秋七月。八月丁卯,大事于大庙,跻僖公。冬,晋人、宋人、陈人、郑人伐秦。公子遂如齐纳币。

【传】二年春,秦孟明视帅师伐晋,以报殽之役。二月晋侯御之。先且居将中军,赵衰佐之。王官无地御戎,狐鞫居为右。甲子,及秦师战于彭衙。秦师败绩。晋人谓秦``拜赐之师」。

战于殽也,晋梁弘御戎,莱驹为右。战之明日,晋襄公缚秦囚,使莱驹以戈斩之。囚呼,莱驹失戈,狼瞫取戈以斩囚,禽之以从公乘,遂以为右。箕之役,先轸黜之而立续简伯。狼瞫怒。其友曰:``盍死之?」瞫曰:``吾未获死所。」其友曰:``吾与女为难。」瞫曰;``《周志》有之,『勇则害上,不登于明堂。』死而不义,非勇也。共用之谓勇。吾以勇求右,无勇而黜,亦其所也。谓上不我知,黜而宜,乃知我矣。子姑待之。」及彭衙,既陈,以其属驰秦师,死焉。晋师从之,大败秦师。君子谓:``狼瞫于是乎君子。诗曰:『君子如怒,乱庶遄沮。』又曰:『王赫斯怒,爰整其旅。』怒不作乱而以从师,可谓君子矣。」

秦伯犹用孟明。孟明增修国政,重施于民。赵成子言于诸大夫曰:``秦师又至,将必辟之,惧而增德,不可当也。诗曰:『毋念尔祖,聿修厥德。』孟明念之矣,念德不怠,其可敌乎?」

丁丑,作僖公主,书,不时也。

晋人以公不朝来讨,公如晋。夏四月己巳,晋人使阳处父盟公以耻之。书曰:``及晋处父盟。」以厌之也。适晋不书,讳之也。公未至,六月,穆伯会诸侯及晋司空士縠盟于垂陇,晋讨卫故也。书士縠,堪其事也。

陈侯为卫请成于晋,执孔达以说。

秋八月丁卯,大事于大庙,跻僖公,逆祀也。于是夏父弗忌为宗伯,尊僖公,且明见曰:``吾见新鬼大,故鬼小。先大后小,顺也。跻圣贤,明也。明、顺,礼也。」

君子以为失礼。礼无不顺。祀,国之大事也,而逆之,可谓礼乎?子虽齐圣,不先父食久矣。故禹不先鲧,汤不先契,文、武不先不窋。宋祖帝乙,郑祖厉王,犹上祖也。是以《鲁颂》曰:``春秋匪解,享祀不忒,皇皇后帝,皇祖后稷。」君子曰礼,谓其后稷亲而先帝也。《诗》曰:``问我诸姑,遂及伯姊。」君子曰礼,谓其姊亲而先姑也。仲尼曰:``臧文仲,其不仁者三,不知者三。下展禽,废六关,妾织蒲,三不仁也。作虚器,纵逆祀,祀爰居,三不知也。」

冬,晋先且居、宋公子成、陈辕选、郑公子归生伐秦,取汪,及彭衙而还,以报彭衙之役。卿不书,为穆公故,尊秦也,谓之崇德。

襄仲如齐纳币,礼也。凡君即位,好舅甥,修昏姻,娶元妃以奉粢盛,孝也。孝,礼之始也。

\hypertarget{header-n1077}{%
\subsubsection{文公三年}\label{header-n1077}}

【经】三年春王正月,叔孙得臣会晋人、宋人、陈人、卫人、郑人伐沈。沈溃。夏五月,王子虎卒。秦人伐晋。秋,楚人围江。雨螽于宋。冬,公如晋。十有二月己巳,公及晋侯盟。晋阳处父帅师伐楚以救江。

【传】三年春,庄叔会诸侯之师伐沈,以其服于楚也。沈溃。凡民逃其上曰溃,在上曰逃。

卫侯如陈,拜晋成也。

夏四月乙亥,王叔文公卒,来赴吊如同盟,礼也。

秦伯伐晋,济河焚舟,取王官,及郊。晋人不出,遂自茅津济,封殽尸而还。遂霸西戎,用孟明也。君子是以知``秦穆公之为君也,举人之周也,与人之壹也;孟明之臣也,其不解也,能惧思也;子桑之忠也,其知人也,能举善也。《诗》曰:『于以采蘩,于沼于沚,于以用之公侯之事』,秦穆有焉。『夙夜匪解,以事一人』,孟明有焉。『诒阙孙谋,以燕翼子』,子桑有焉。」

秋,雨螽于宋,队而死也。

楚师围江。晋先仆伐楚以救江。

冬,晋以江故告于周。王叔桓公、晋阳处父伐楚以救江,门于方城,遇息公子朱而还。

晋人惧其无礼于公也,请改盟。公如晋,及晋侯盟。晋侯飨公,赋《菁菁者莪》。庄叔以公降,拜,曰:``小国受命于大国,敢不慎仪。君贶之以大礼,何乐如之。抑小国之乐,大国之惠也。」晋侯降,辞。登,成拜。公赋《嘉乐》。

\hypertarget{header-n1089}{%
\subsubsection{文公四年}\label{header-n1089}}

【经】四年春,公至自晋。夏,逆妇姜于齐。狄侵齐。秋,楚人灭江。晋侯伐秦。卫侯使宁俞来聘。冬十有一月壬寅,夫人风氏薨。

【传】四年春,晋人归孔达于卫,以为卫之良也,故免之。

夏,卫侯如晋拜。曹伯如晋,会正。

逆妇姜于齐,卿不行,非礼也。君子是以知出姜之不允于鲁也。曰:``贵聘而贱逆之,君而卑之,立而废之,弃信而坏其主,在国必乱,在家必亡。不允宜哉?《诗》曰:『畏天之威,于时保之。』敬主之谓也。」

秋,晋侯伐秦,围刓、新城,以报王官之役。

楚人灭江,秦伯为之降服、出次、不举、过数。大夫谏,公曰:``同盟灭,虽不能救,敢不矜乎!吾自惧也。」君子曰:``《诗》云:『惟彼二国,其政不获,惟此四国,爰究爰度。』其秦穆之谓矣。」

卫宁武子来聘,公与之宴,为赋《湛露》及《彤弓》。不辞,又不答赋。使行人私焉。对曰:``臣以为肄业及之也。昔诸侯朝正于王,王宴乐之,于是乎赋《湛露》,则天子当阳,诸侯用命也。诸侯敌王所忾而献其功,王于是乎赐之彤弓一,彤矢百,玈弓矢千,以觉报宴。今陪臣来继旧好,君辱贶之,其敢干大礼以自取戾。」

冬,成风薨。

\hypertarget{header-n1100}{%
\subsubsection{文公五年}\label{header-n1100}}

【经】五年春王正月,王使荣叔归含,且賵。三月辛亥,葬我小君成风。王使召伯来会葬。夏,公孙敖如晋。秦人入鄀。秋,楚人灭六。冬十月甲申,许男业卒。

【传】五年春,王使荣叔来含且賵,召昭公来会葬,礼也。

初,鄀叛楚即秦,又贰于楚。夏,秦人入鄀。

六人叛楚即东夷。秋,楚成大心、仲归帅师灭六。

冬,楚公子燮灭蓼,臧文仲闻六与蓼灭,曰:``皋陶庭坚不祀忽诸。德之不建,民之无援,哀哉!」

晋阳处父聘于卫,反过宁,宁嬴从之,及温而还。其妻问之,嬴曰;``以刚。《商书》曰:『沈渐刚克,高明柔克。』夫子壹之,其不没乎。天为刚德,犹不干时,况在人乎?且华而不实,怨之所聚也,犯而聚怨,不可以定身。余惧不获其利而离其难,是以去之。」

晋赵成子,栾贞子、霍伯、臼季皆卒。

\hypertarget{header-n1110}{%
\subsubsection{文公六年}\label{header-n1110}}

【经】六年春,葬许僖公。夏,季孙行父如陈。秋,季孙行父如晋。八月乙亥,晋侯欢卒。冬十月,公子遂如晋。葬晋襄公。晋杀其大夫阳处父。晋狐射姑出奔狄。闰月不告月,犹朝于庙。

【传】六年春,晋蒐于夷,舍二军。使狐射姑将中军,赵盾佐之。阳处父至自温,改蒐于董,易中军。阳子,成季之属也,故党于赵氏,且谓赵盾能,曰:``使能,国之利也。」是以上之。宣子于是乎始为国政,制事典,正法罪。辟狱刑,董逋逃。由质要,治旧污,本秩礼,续常职,出滞淹。既成,以授大傅阳子与大师贾佗,使行诸晋国,以为常法。

臧文仲以陈、卫之睦也,欲求好于陈。夏,季文子聘于陈,且娶焉。

秦伯任好卒。以子车氏之三子奄息、仲行、金咸虎为殉。皆秦之良也。国人哀之,为之赋《黄鸟》。君子曰:``秦穆之不为盟主也,宜哉。死而弃民。先王违世,犹诒之法,而况夺之善人乎!《诗》曰:『人之云亡,邦国殄瘁。』无善人之谓。若之何夺之?」古之王者知命之不长,是以并建圣哲,树之风声,分之采物,着之话言,为之律度,陈之艺极,引之表仪,予之法制,告之训典,教之防利,委之常秩,道之礼则,使毋失其土宜,众隶赖之,而后即命。圣王同之。今纵无法以遗后嗣,而又收其良以死,难以在上矣。君子是以知秦之不复东征也。

秋,季文子将聘于晋,使求遭丧之礼以行。其人曰:``将焉用之?」文子曰:``备豫不虞,古之善教也。求而无之,实难,过求何害?」

八月乙亥,晋襄公卒。灵公少,晋人以难故,欲立长君。赵孟曰:``立公子雍。好善而长,先君爱之,且近于秦。秦,旧好也。置善则固,事长则顺,立爱则孝,结旧则安。为难故,故欲立长君,有此四德者,难必抒矣。贾季曰:``不如立公子乐。辰嬴嬖于二君,立其子,民必安之。」赵孟曰:``辰嬴贱,班在九人,其子何震之有?且为二嬖,淫也。为先君子,不能求大而出在小国,辟也。母淫子辟,无威。陈小而远,无援。将何安焉?杜祁以君故,让逼姞而上之,以狄故,让季隗而己次之,故班在四。先君是以爱其子而仕诸秦,为亚卿焉。秦大而近,足以为援,母义子爱,足以威民,立之不亦可乎?」使先蔑、士会如秦,逆公子雍。贾季亦使召公子乐于陈。赵孟使杀诸郫。贾季怨阳子之易其班也,而知其无援于晋也。九月,贾季使续鞫居杀阳处父。书曰:``晋杀其大夫。」侵官也。

冬十月,襄仲如晋。葬襄公。

十一月丙寅,晋杀续简伯。贾季奔狄。宣子使臾骈送其帑。夷之蒐,贾季戮臾骈,臾骈之人欲尽杀贾氏以报焉。臾骈曰:``不可。吾闻《前志》有之曰:『敌惠敌怨,不在后嗣』,忠之道也。夫子礼于贾季,我以其宠报私怨,无乃不可乎?介人之宠,非勇也。损怨益仇,非知也。以私害公,非忠也。释此三者,何以事夫子?」尽具其帑,与其器用财贿,亲帅扞之,送致诸竟。

闰月不告朔,非礼也。闰以正时,时以作事,事以厚生,生民之道,于是乎在矣。不告闰朔,弃时政也,何以为民?

\hypertarget{header-n1122}{%
\subsubsection{文公七年}\label{header-n1122}}

【经】七年春,公伐邾。三月甲戌,取须句。遂城郚。夏四月,宋公王臣卒。宋人杀其大夫。戊子,晋人及秦人战于令狐。晋先蔑奔秦。狄侵我西鄙。秋八月,公会诸侯、晋大夫盟于扈。冬,徐伐莒。公孙敖如莒莅盟。

【传】七年春,公伐邾。间晋难也。

三月甲戌,取须句,置文公子焉,非礼也。

夏四月,宋成公卒。于是公子成为右师,公孙友左师,乐豫为司马,鳞矔为司徒,公子荡为司城,华御事为司寇。

昭公将去群公子,乐豫曰:``不可。公族,公室之枝叶也,若去之则本根无所庇荫矣。葛藟犹能庇其本根,故君子以为比,况国君乎?此谚所谓庇焉而纵寻斧焉者也。必不可,君其图之。亲之以德,皆股肱也,谁敢携贰?若之何去之?」不听。穆、襄之族率国人以攻公,杀公孙固、公孙郑于公宫。六卿和公室,乐豫舍司马以让公子卬,昭公即位而葬。书曰:``宋人杀其大夫。」不称名,众也,且言非其罪也。

秦康公送公子雍于晋,曰:``文公之入也无卫,故有吕、郤之难。」乃多与之徒卫。穆赢日抱大子以啼于朝,曰:``先君何罪?其嗣亦何罪?舍适嗣不立而外求君,将焉置此?」出朝,则抱以适赵氏,顿首于宣子曰:``先君奉此子也而属诸子,曰:『此子也才,吾受子之赐;不才,吾唯子之怨。』今君虽终,言犹在耳,而弃之,若何?」宣子与诸大夫皆患穆嬴,且畏逼,乃背先蔑而立灵公,以御秦师。箕郑居守。赵盾将中军,先克佐之。荀林父佐上军。先蔑将下军,先都佐之,步招御戎,戎津为右。及堇阴,宣子曰:``我若受秦,秦则宾也;不受,寇也。既不受矣,而复缓师,秦将生心。先人有夺人之心,军之善谋也。逐寇如追逃,军之善政也。」训卒利兵,秣马蓐食,潜师夜起。戊子,败秦师于令狐,至于刳首。己丑,先蔑奔秦。士会从之。

先蔑之使也,荀林父止之,曰:``夫人、大子犹在,而外求君,此必不行。子以疾辞,若何?不然,将及。摄卿以往可也,何必子?同官为寮,吾尝同寮,敢不尽心乎!」弗听。为赋《板》之三章。又弗听。及亡,荀伯尽送其帑及其器用财贿于秦,曰:``为同寮故也。」

士会在秦三年,不见士伯。其人曰:``能亡人于国,不能见于此,焉用之?」士季曰:``吾与之同罪,非义之也,将何见焉?」及归,遂不见。

狄侵我西鄙,公使告于晋。赵宣子使因贾季问酆舒。且让之。酆舒问于贾季曰:``赵衰、赵盾孰贤?」对曰:``赵衰,冬日之日也。赵盾,夏日之日也。」

秋八月,齐侯、宋公、卫侯、郑伯、许男、曹伯会晋赵盾盟于扈,晋侯立故也。公后至,故不书所会。凡会诸侯,不书所会,后也。后至,不书其国,辟不敏也。

穆伯娶于莒,曰戴己,生文伯,其娣声己生惠叔。戴己卒,又聘于莒,莒人以声己辞,则为襄仲聘焉。

冬,徐伐莒。莒人来请盟。穆伯如莒莅盟,且为仲逆。及鄢陵。登城见之,美,自为娶之。仲请攻之,公将许之。叔仲惠伯谏曰:``臣闻之,兵作于内为乱,于外为寇,寇犹及人,乱自及也。今臣作乱而君不禁,以启寇仇,若之何?」公止之,惠伯成之。使仲舍之,公孙敖反之,复为兄弟如初。从之。

晋郤缺言于赵宣子曰:``日卫不睦,故取其地,今已睦矣,可以归之。叛而不讨,何以示威?服而不柔,何以示怀?非威非怀,何以示德?无德,何以主盟?子为正卿,以主诸侯,而不务德,将若之何?《夏书》曰:『戒之用休,董之用威,劝之以《九歌》,勿使坏。』九功之德皆可歌也,谓之九歌。六府、三事,谓之九功。水、火、金、木、土、谷,谓之六府。正德、利用、厚生,谓之三事。义而行之,谓之德、礼。无礼不乐,所由叛也。若吾子之德莫可歌也,其谁来之?盍使睦者歌吾子乎?」宣子说之。

\hypertarget{header-n1138}{%
\subsubsection{文公八年}\label{header-n1138}}

【经】八年春王正月。夏四月。秋八月戊申,天王崩。冬十月壬午,公子遂会晋赵盾盟于衡雍。乙酉,公子遂会洛戎盟于暴。公孙敖如京师,不至而复。丙戌,奔莒。螽。宋人杀其大夫司马。宋司城来奔。

【传】八年春,晋侯使解扬归匡、戚之田于卫,且复致公婿池之封,自申至于虎牢之竟。

夏,秦人伐晋,取武城,以报令狐之役。

秋,襄王崩。

晋人以扈之盟来讨。冬,襄仲会晋赵孟,盟于衡雍,报扈之盟也,遂会伊洛之戎。书曰``公子遂」,珍之也。

穆伯如周吊丧,不至,以币奔莒,从己氏焉。

宋襄夫人,襄王之姊也,昭公不礼焉。夫人因戴氏之族,以杀襄公之孙孔叔、公孙钟离及大司马公子卬,皆昭公之党也。司马握节以死,故书以官。司城荡意诸来奔,效节于府人而出。公以其官逆之,皆复之,亦书以官,皆贵之也。

夷之蒐,晋侯将登箕郑父、先都,而使士縠、梁益耳将中军。先克曰:``狐、赵之勋,不可废也。」从之。先克夺蒯得田于堇阴。故箕郑父、先都、士縠、梁益耳、蒯得作乱。

\hypertarget{header-n1149}{%
\subsubsection{文公九年}\label{header-n1149}}

【经】九年春,毛伯来求金。夫人姜氏如齐。二月,叔孙得臣如京师。辛丑,葬襄王。晋人杀其大夫先都。三月,夫人姜氏至自齐。晋人杀其大夫士縠及箕郑父。楚人伐郑。公子遂会晋人、宋人、卫人、许人救郑。夏,狄侵齐。秋八月,曹伯襄卒。九月癸西,地震。冬,楚子使椒来聘。秦人来归僖公、成风之襚。葬曹共公。

【传】九年春,王正月己酉,使贼杀先克。乙丑,晋人杀先都,梁益耳。

毛伯卫来求金,非礼也。不书王命,未葬也。

二月庄叔如周。葬襄王。

三月甲戌,晋人杀箕郑父、士縠、蒯得。

范山言于楚子曰:``晋君少,不在诸侯,北方可图也。」楚子师于狼渊以伐郑。囚公子坚、公子龙及乐耳。郑及楚平。公子遂会晋赵盾、宋华耦、卫孔达、许大夫救郑,不及楚师。卿不书,缓也,以惩不恪。

夏,楚侵陈,克壶丘,以其服于晋也。

秋,楚公子朱自东夷伐陈,陈人败之,获公子伐。陈惧,乃及楚平。

冬,楚子越椒来聘,执币傲。叔仲惠伯曰:``是必灭若敖氏之宗。傲其先君,神弗福也。」

秦人来归僖公、成风之襚,礼也。诸侯相吊贺也,虽不当事,苟有礼焉,书也,以无忘旧好。

\hypertarget{header-n1162}{%
\subsubsection{文公十年}\label{header-n1162}}

【经】十年春王三月辛卯,臧孙辰卒。夏,秦伐晋。楚杀其大夫宜申。自正月不雨,至于秋七月。及苏子盟于女栗。冬,狄侵宋。楚子、蔡侯次于厥貉。

【传】十年春,晋人伐秦,取少梁。

夏,秦伯伐晋,取北征。

初,楚范巫矞似谓成王与子玉、子西曰:``三君皆将强死。」城濮之役,王思之,故使止子玉曰:``毋死。」不及。止子西,子西缢而县绝,王使适至,遂止之,使为商公。沿汉溯江,将入郢。王在渚宫,下,见之。惧而辞曰:``臣免于死,又有谗言,谓臣将逃,臣归死于司败也。」王使为工尹,又与子家谋弑穆王。穆王闻之。五月杀斗宜申及仲归。

秋七月,及苏子盟于女栗,顷王立故也。

陈侯、郑伯会楚子于息。冬,遂及蔡侯次于厥貉。将以伐宋。宋华御事曰:``楚欲弱我也。先为之弱乎,何必使诱我?我实不能,民何罪?」乃逆楚子,劳,且听命。遂道以田孟诸。宋公为右盂,郑伯为左盂。期思公复遂为右司马,子朱及文之无畏为左司马。命夙驾载燧,宋公违命,无畏抶其仆以徇。

或谓子舟曰:``国君不可戮也。」子舟曰:``当官而行,何强之有?《诗》曰:『刚亦不吐,柔亦不茹。』『毋从诡随,以谨罔极。』是亦非辟强也,敢爱死以乱官乎!」

厥貉之会,麇子逃归。

\hypertarget{header-n1173}{%
\subsubsection{文公十一年}\label{header-n1173}}

【经】十有一年春,楚子伐麋。夏,叔仲彭生会晋郤缺于承筐。秋,曹伯来朝。公子遂如宋。狄侵齐。冬十月甲午,叔孙得臣败狄于咸。

【传】十一年春,楚子伐麇,成大心败麇师于防渚。潘崇复伐麇,至于锡穴。

夏,叔仲惠伯会晋郤缺于承筐,谋诸侯之从于楚者。

秋,曹文公来朝,即位而来见也。

襄仲聘于宋,且言司城荡意诸而复之,因贺楚师之不害也。

鄋瞒侵齐。遂伐我。公卜使叔孙得臣追之,吉。侯叔夏御庄叔,绵房甥为右,富父终甥驷乘。冬十月甲午,败狄于咸,获长狄侨如。富父终甥舂其喉以戈,杀之,埋其首于子驹之门,以命宣伯。

初,宋武公之世,鄋瞒伐宋,司徒皇父帅师御之,耏班御皇父充石,公子谷甥为右,司寇牛父驷乘,以败狄于长丘,获长狄缘斯,皇父之二子死焉。宋公于是以门赏耏班,使食其征,谓之耏门。晋之灭潞也,获侨如之弟焚如。齐襄公之二年,鄋瞒伐齐,齐王子成父获其弟荣如,埋其首于周首之北门。卫人获其季简如,鄋瞒由是遂亡。

郕大子朱儒自安于夫钟,国人弗徇。

\hypertarget{header-n1184}{%
\subsubsection{文公十二年}\label{header-n1184}}

【经】十有二年春王正月,郕伯来奔。杞伯来朝。二月庚子,子叔姬卒。夏,楚人围巢。秋,滕子来朝。秦伯使术来聘。冬十有二戊午,晋人、秦人战于河曲。季孙行父帅师城诸及郓。

【传】十二年春,郕伯卒,郕人立君。大子以夫钟与郕邽来奔。公以诸侯逆之,非礼也。故书曰:``郕伯来奔。」不书地,尊诸侯也。

杞桓公来朝,始朝公也。且请绝叔姬而无绝昏,公许之。

二月,叔姬卒,不言杞,绝也。书叔姬,言非女也。

楚令尹大孙伯卒,成嘉为令尹。群舒叛楚。夏,子孔执舒子平及宗子,遂围巢。

秋,滕昭公来朝,亦始朝公也。

秦伯使西乞术来聘,且言将伐晋。襄仲辞玉曰:``君不忘先君之好,照临鲁国,镇抚其社稷,重之以大器,寡君敢辞玉。」对曰:``不腆敝器,不足辞也。」主人三辞。宾客曰:``寡君愿徼福于周公、鲁公以事君,不腆先君之敝器,使下臣致诸执事以为瑞节,要结好命,所以藉寡君之命,结二国之好,是以敢致之。」襄仲曰:``不有君子,其能国乎?国无陋矣。」厚贿之。

秦为令狐之役故,冬,秦伯伐晋,取羁马。晋人御之。赵盾将中军,荀林父佐之。郤缺上军,臾骈佐之。栾盾将下军,胥甲佐之。范无恤御戎,以从秦师于河曲。臾骈曰:``秦不能久,请深垒固军以待之。」从之。

秦人欲战,秦伯谓士会曰:``若何而战?」对曰:``赵氏新出其属曰臾骈,必实为此谋,将以老我师也。赵有侧室曰穿,晋君之婿也,有宠而弱,不在军事,好勇而狂,且恶臾骈之佐上军也,若使轻者肆焉,其可。」秦伯以璧祈战于河。

十二月戊午,秦军掩晋上军,赵穿追之,不及。反,怒曰:``裹粮坐甲,固敌是求,敌至不击,将何俟焉?」军吏曰:``将有待也。」穿曰:``我不知谋,将独出。」乃以其属出。宣子曰:``秦获穿也,获一卿矣。秦以胜归,我何以报?」乃皆出战,交绥。秦行人夜戒晋师曰:``两君之士皆未憖也,明日请相见也。」臾骈曰:``使者目动而言肆,惧我也,将遁矣。薄诸河,必败之。」胥甲、赵穿当军门呼曰:``死伤未收而弃之,不惠也;不待期而薄人于险,无勇也。」乃止。秦师夜遁。复侵晋,入瑕。

城诸及郓,书,时也。

\hypertarget{header-n1198}{%
\subsubsection{文公十三年}\label{header-n1198}}

【经】十有三春王正月。夏五月壬午,陈侯朔卒。邾子蘧蒢卒。自正月不雨,至于秋七月。大室屋坏。冬,公如晋。卫侯会公于沓。狄侵卫。十有二月己丑,公及晋侯盟。公还自晋,郑伯会公于棐。

【传】十三年春,晋侯使詹嘉处瑕,以守桃林之塞。

晋人患秦之用士会也,夏,六卿相见于诸浮,赵宣子曰;``随会在秦,贾季在狄,难日至矣,若之何?」中行桓子曰:``请复贾季,能外事,且由旧勋。」郤成子曰:``贾季乱,且罪大,不如随会,能贱而有耻,柔而不犯,其知足使也,且无罪。」

乃使魏寿余伪以魏叛者以诱士会,执其帑于晋,使夜逸。请自归于秦,秦伯许之。履士会之足于朝。秦伯师于河西,魏人在东。寿余曰:``请东人之能与夫二三有司言者,吾与之先。」使士会。士会辞曰:``晋人,虎狼也,若背其言,臣死,妻子为戮,无益于君,不可悔也。」秦伯曰:``若背其言,所不归尔帑者,有如河。」乃行。绕朝赠之以策,曰:``子无谓秦无人,吾谋适不用也。」既济,魏人噪而还。秦人归其帑。其处者为刘氏。

邾文公卜迁于绎。史曰:``利于民而不利于君。」邾子曰:``苟利于民,孤之利也。天生民而树之君,以利之也。民既利矣,孤必与焉。」左右曰:``命可长也,君何弗为?」邾子曰:``命在养民。死之短长,时也。民苟利矣,迁也,吉莫如之!」遂迁于绎。

五月,邾文公卒。君子曰:``知命。」

秋七月,大室之屋坏,书,不共也。

冬,公如晋,朝,且寻盟。卫侯会公于沓,请平于晋。公还,郑伯会公于棐,亦请平于晋。公皆成之。郑伯与公宴于棐。子家赋《鸿雁》。季文子曰:``寡君未免于此。」文子赋《四月》。子家赋《载驰》之四章。文子赋《采薇》之四章。郑伯拜。公答拜。

\hypertarget{header-n1209}{%
\subsubsection{文公十四年}\label{header-n1209}}

【经】十有四年春王正月,公至自晋。邾人伐我南鄙,叔彭生帅师伐邾。夏五月乙亥,齐侯潘卒。六月,公会宋公、陈侯、卫侯、郑伯、许男、曹伯、晋赵盾。癸酉,同盟于新城。秋七月,有星孛入于北斗。公至自会。晋人纳捷菑于邾。弗克纳。九月甲申,公孙敖卒于齐。齐公子商人弑其君舍。宋子哀来奔。冬,单伯如齐。齐人执单伯。齐人执子叔姬。

【传】十四年春,顷王崩。周公阅与王孙苏争政,故不赴。凡崩、薨,不赴,则不书。祸、福,不告亦不书,惩不敬也。

邾文公之卒也,公使吊焉,不敬。邾人来讨,伐我南鄙,故惠伯伐邾。

子叔姬齐昭公,生舍。叔姬无宠,舍无威。公子商人骤施于国,而多聚士,尽其家,贷于公,有司以继之。夏五月,昭公卒,舍即位。

邾文公妃元齐姜生定公,二妃晋姬生捷菑。文公卒,邾人立定公,捷菑奔晋。

六月,同盟于新城,从于楚者服,且谋邾也。

秋七月乙卯夜,齐商人弑舍而让元。元曰:``尔求之久矣。我能事尔,尔不可使多蓄憾。将免我乎?尔为之!」

有星孛入于北斗,周内史叔服曰:``不出七年,宋、齐、晋之君皆将死乱。」

晋赵盾以诸侯之师八百乘纳捷菑于邾。邾人辞曰:``齐出玃且长。」宣子曰:``辞顺而弗从,不祥。」乃还。

周公将与王孙苏讼于晋,王叛王孙苏,而使尹氏与聃启讼周公于晋。赵宣子平王室而复之。

楚庄王立,子孔、潘崇将袭群舒,使公子燮与子仪守而伐舒蓼。二子作乱,城郢而使贼杀子孔,不克而还。八月,二子以楚子出,将如商密。庐戢梨及叔麋诱之,遂杀斗克及公子燮。

初,斗克囚于秦,秦有殽之败,而使归求成,成而不得志。公子燮求令尹而不得。故二子作乱。

穆伯之从己氏也,鲁人立文伯。穆伯生二子于莒而求复,文伯以为请。襄仲使无朝。听命,复而不出,二年而尽室以复适莒。文伯疾而请曰:``谷之子弱,请立难也。」许之。文伯卒,立惠叔。穆伯请重赂以求复,惠叔以为请,许之。将来,九月卒于齐,告丧,请葬,弗许。

宋高哀为萧封人,以为卿,不义宋公而出,遂来奔。书曰:``宋子哀来奔去贵之也。」

齐人定懿公,使来告难,故书以九月。齐公子元不顺懿公之为政也,终不曰``公」,曰``夫己氏」。

襄仲使告于王,请以王宠求昭姬于齐。曰:``杀其子,焉用其母?请受而罪之。」

冬,单伯如齐,请子叔姬,齐人执之。又执子叔姬。

\hypertarget{header-n1229}{%
\subsubsection{文公十五年}\label{header-n1229}}

【经】十有五年春,季孙行父如晋。三月,宋司马华孙来盟。夏,曹伯来朝。齐人归公孙敖之丧。六月辛丑朔,日有食之。鼓、用牲于社。单伯至自齐。晋郤缺帅师伐蔡。戊申,入蔡。齐人侵我西鄙。季孙行父如晋。冬十有一月,诸侯盟于扈。十有二月,齐人来归子叔姬。齐侯侵我西鄙,遂伐曹入期郛。

【传】十五年春,季文子如晋,为单伯与子叔姬故也。

三月,宋华耦来盟,其官皆从之。书曰``宋司马华孙」,贵之也。

公与之宴,辞曰:``君之先臣督,得罪于宋殇公,名在诸侯之策。臣承其祀,其敢辱君,请承命于亚旅。」鲁人以为敏。

夏,曹伯来朝,礼也。诸侯五年再相朝,以修王命,古之制也。

齐人或为孟氏谋,曰:``鲁,尔亲也。饰棺置诸堂阜,鲁必取之。」从之。卞人以告。惠叔犹毁以为请。立于朝以待命。许之,取而殡之。齐人送之。书曰:``齐人归公孙敖之丧。」为孟氏,且国故也。葬视共仲。

声己不视,帷堂而哭。襄仲欲勿哭,惠伯曰:``丧,亲之终也。虽不能始,善终可也。史佚有言曰:『兄弟致美。』救乏、贺善、吊灾、祭敬、丧哀,情虽不同,毋绝其爱,亲之道也。子无失道,何怨于人?」襄仲说,帅兄弟以哭之。他年,其二子来,孟献子爱之,闻于国。或谮之曰:``将杀子。」献子以告季文子。二子曰:``夫子以爱我闻,我以将杀子闻,不亦远于礼乎?远礼不如死。」一人门于句鼆,一人门于戾丘,皆死。

六月辛丑朔,日有食之,鼓、用牲于社,非礼也。日有食之,天子不举,伐鼓于社,诸侯用币于社,伐鼓于朝,以昭事神、训民、事君,示有等威。古之道也。

齐人许单伯请而赦之,使来致命。书曰:``单伯至自齐。」贵之也。

新城之盟,蔡人不与。晋郤缺以上军、下军伐蔡,曰:``君弱,不可以怠。」戊申,入蔡,以城下之盟而还。凡胜国,曰灭之;获大城焉,曰入之。

秋,齐人侵我西鄙,故季文子告于晋。

冬十一月,晋侯、宋公、卫侯、蔡侯、郑伯、许男、曹伯盟于扈,寻新城之盟,且谋伐齐也。齐人赂晋侯,故不克而还。于是有齐难,是以公不会。书曰:``诸侯盟于扈。」无能为故也。凡诸侯会,公不与,不书,讳君恶也。与而不书,后也。

齐人来归子叔姬,王故也。

齐侯侵我西鄙,谓诸侯不能也。遂伐曹,入其郛,讨其来朝也。季文子曰:``齐侯其不免乎。己则无礼,而讨于有礼者,曰:『女何故行礼!』礼以顺天,天之道也,己则反天,而又以讨人,难以免矣。诗曰:『胡不相畏,不畏于天?』君子之不虐幼贱,畏于天也。在周颂曰:『畏天之威,于时保之。』不畏于天,将何能保?以乱取国,奉礼以守,犹惧不终,多行无礼,弗能在矣!」

\hypertarget{header-n1246}{%
\subsubsection{文公十六年}\label{header-n1246}}

【经】十有六年春,季孙行父会齐侯于阳谷,齐侯弗及盟。夏五月,公四不视朔。六月戊辰,公子遂及齐侯盟于郪丘。秋八月辛未,夫人姜氏薨。毁泉台。楚人、秦人、巴人灭庸。冬十有一月,宋人弑其君杵臼。

【传】十六年春,王正月,及齐平。公有疾,使季文子会齐侯于阳谷。请盟,齐侯不肯,曰:``请俟君间。」

夏五月,公四不视朔,疾也。公使襄仲纳赂于齐侯,故盟于郪丘。

有蛇自泉宫出,入于国,如先君之数秋八月辛未,声姜薨,毁泉台。

楚大饥,戎伐其西南,至于阜山,师于大林。又伐其东南,至于阳丘,以侵訾枝。庸人帅群蛮以叛楚。麇人率百濮聚于选,将伐楚。于是申、息之北门不启。

楚人谋徙于阪高。蒍贾曰:``不可。我能往,寇亦能住。不如伐庸。夫麇与百濮,谓我饥不能师,故伐我也。若我出师,必惧而归。百濮离居,将各走其邑,谁暇谋人?」乃出师。旬有五日,百濮乃罢。自庐以往,振廪同食。次于句澨。使庐戢黎侵庸,及庸方城。庸人逐之,囚子扬窗。三宿而逸,曰:``庸师众,群蛮聚焉,不如复大师,且起王卒,合而后进。」师叔曰:``不可。姑又与之遇以骄之。彼骄我怒,而后可克,先君蚡冒所以服陉隰也。」又与之遇,七遇皆北,唯裨、鯈、鱼人实逐之。

庸人曰:``楚不足与战矣。」遂不设备。楚子乘馹,会师于临品,分为二队,子越自石溪,子贝自仞,以伐庸。秦人、巴人从楚师,群蛮从楚子盟。遂灭庸。

宋公子鲍礼于国人,宋饥,竭其粟而贷之。年自七十以上,无不馈诒也,时加羞珍异。无日不数于六卿之门,国之才人,无不事也,亲自桓以下,无不恤也。公子鲍美而艳,襄夫人欲通之,而不可,夫人助之施。昭公无道,国人奉公子鲍以因夫人。

于是华元为右师,公孙友为左师,华耦为司马,鳞鱼雚为司徒,荡意诸为司城,公子朝为司寇。初,司城荡卒,公孙寿辞司城,请使意诸为之。既而告人曰:``君无道,吾官近,惧及焉。弃官则族无所庇。子,身之贰也,姑纾死焉。虽亡子,犹不亡族。」既,夫人将使公田孟诸而杀之。公知之,尽以宝行。荡意诸曰:``盍适诸侯?」公曰:``不能其大夫至于君祖母以及国人,诸侯谁纳我?且既为人君,而又为人臣,不如死。」尽以其宝赐左右以使行。夫人使谓司城去公,对曰:``臣之而逃其难,若后君何?」

冬十一月甲寅,宋昭公将田孟诸,未至,夫人王姬使帅甸攻而杀之。荡意诸死之。书曰:``宋人弑其君杵臼。」君无道也。

文公即位,使母弟须为司城。华耦卒,而使荡虺为司马。

\hypertarget{header-n1260}{%
\subsubsection{文公十七年}\label{header-n1260}}

【经】十有七年春,晋人、卫人、陈人、郑人伐宋。夏四月癸亥,葬我小君声姜。齐侯伐我西鄙。六月癸未,公及齐侯盟于谷。诸侯会于扈。秋,公至自谷。冬,公子遂如齐。

【传】十七年春,晋荀林父、卫孔达、陈公孙宁、郑石楚伐宋。讨曰:``何故弑君!」犹立文公而还,卿不书,失其所也。

夏四月癸亥,葬声姜。有齐难,是以缓。

齐侯伐我北鄙,襄仲请盟。六月,盟于谷。

晋侯蒐于黄父,遂复合诸侯于扈,平宋也。公不与会,齐难故也。书曰``诸侯」,无功也。

于是,晋侯不见郑伯,以为贰于楚也。

郑子家使执讯而与之书,以告赵宣子,曰:``寡君即位三年,召蔡侯而与之事君。九月,蔡侯入于敝邑以行。敝邑以侯宣多之难,寡君是以不得与蔡侯偕。十一月,克灭侯宣多而随蔡侯以朝于执事。十二年六月,归生佐寡君之嫡夷,以请陈侯于楚而朝诸君。十四年七月,寡君又朝,以蒇陈事。十五年五月,陈侯自敝邑往朝于君。往年正月,烛之武往朝夷也。八月,寡君又往朝。以陈、蔡之密迩于楚而不敢贰焉,则敝邑之故也。虽敝邑之事君,何以不免?在位之中,一朝于襄,而再见于君。夷与孤之二三臣相及于绛,虽我小国,则蔑以过之矣。今大国曰:『尔未逞吾志。』敝邑有亡,无以加焉。古人有言曰:『畏首畏尾,身其馀几。』又曰:『鹿死不择音。』小国之事大国也,德,则其人也;不德,则其鹿也,铤而走险,急何能择?命之罔极,亦知亡矣。将悉敝赋以待于鯈,唯执事命之。

文公二年六月壬申,朝于齐。四年二月壬戌,为齐侵蔡,亦获成于楚。居大国之间而从于强令,岂其罪也。大国若弗图,无所逃命。」

晋巩朔行成于郑,赵穿、公婿池为质焉。

秋,周甘蜀败戎于垂,乘其饮酒也。

冬十月,郑大子夷、石楚为质于晋。

襄仲如齐,拜谷之盟。复曰:``臣闻齐人将食鲁之麦。以臣观之,将不能。齐君之语偷。臧文仲有言曰:『民主偷必死』。」

\hypertarget{header-n1275}{%
\subsubsection{文公十八年}\label{header-n1275}}

【经】十有八年春王二月丁丑,公薨于台下。秦伯荦卒。夏五月戊戌,齐人弑其君商人。六月癸酉,葬我君文公。秋,公子遂、叔孙得臣如齐。冬十月,子卒。夫人姜氏归于齐。季孙行父如齐。莒弑其君庶其。

【传】十八年春,齐侯戒师期,而有疾,医曰:``不及秋,将死。」公闻之,卜曰:``尚无及期。」惠伯令龟,卜楚丘占之曰:``齐侯不及期,非疾也。君亦不闻。令龟有咎。」二月丁丑,公薨。

齐懿公之为公子也,与邴蜀之父争田,弗胜。及即位,乃掘而刖之,而使蜀仆。纳阎职之妻,而使职骖乘。

夏五月,公游于申池。二人浴于池,蜀以扑抶职。职怒。曰:``人夺女妻而不怒,一抶女庸何伤!」职曰:``与刖其父而弗能病者何如?」乃谋弑懿公,纳诸竹中。归,舍爵而行。齐人立公子元。

六月,葬文公。

秋,襄仲、庄叔如齐,惠公立故,且拜葬也。

文公二妃敬赢生宣公。敬赢嬖而私事襄仲。宣公长而属诸襄仲,襄仲欲立之,叔仲不可。仲见于齐侯而请之。齐侯新立而欲亲鲁,许之。

冬十月,仲杀恶及视而立宣公。书曰``子卒」,讳之也。仲以君命召惠伯。其宰公冉务人止之,曰:``入必死。」叔仲曰:``死君命可也。」公冉务人曰:``若君命可死,非君命何听?」弗听,乃入,杀而埋之马矢之中。公冉务人奉其帑以奔蔡,既而复叔仲氏。

夫人姜氏归于齐,大归也。将行,哭而过市曰:``天乎,仲为不道,杀适立庶。」市人皆哭,鲁人谓之哀姜。

莒纪公生大子仆,又生季佗,爱季佗而黜仆,且多行无礼于国。仆因国人以弑纪公,以其宝玉来奔,纳诸宣公。公命与之邑,曰:``今日必授。」季文子使司寇出诸竟,曰:``今日必达。」公问其故。季文子使大史克对曰:``先大夫臧文仲教行父事君之礼,行父奉以周旋,弗敢失队。曰:『见有礼于其君者,事之如孝子之养父母也。见无礼于其君者,诛之如鹰鸇之逐鸟雀也。』先君周公制《周礼》曰:『则以观德,德以处事,事以度功,功以食民。』作《誓命》曰:『毁则为贼,掩贼为藏,窃贿为盗,盗器为奸。主藏之名,赖奸之用,为大凶德,有常无赦,在《九刑》不忘。』行父还观莒仆,莫可则也。孝敬忠信为吉德,盗贼藏奸为凶德。夫莒仆,则其孝敬,则弑君父矣;则其忠信,则窃宝玉矣。其人,则盗贼也;其器,则奸兆也,保而利之,则主藏也。以训则昏,民无则焉。不度于善,而皆在于凶德,是以去之。

``昔高阳氏有才子八人,苍舒、隤岂、檮寅、大临、龙降、庭坚、仲容、叔达,齐圣广渊,明允笃诚,天下之民谓之八恺。高辛氏有才子八人,伯奋、仲堪、叔献、季仲、伯虎、仲熊、叔豹、季狸,忠肃共懿,宣慈惠和,天下之民谓之八元。此十六族也,世济其美,不陨其名,以至于尧,尧不能举。舜臣尧,举八恺,使主后土,以揆百事,莫不时序,地平天成。举八元,使布五教于四方,父义、母慈、兄友、弟共、子孝,内平外成。昔帝鸿氏有不才子,掩义隐贼,好行凶德,丑类恶物,顽嚚不友,是与比周,天下之民谓之浑敦。少嗥氏有不才子,毁信废忠,崇饰恶言,靖谮庸回,服谗蒐慝,以诬盛德,天下之民谓之穷奇。颛顼有不才子,不可教训,不知话言,告之则顽,舍之则嚚,傲很明德,以乱天常,天下之民谓之檮杌。此三族也,世济其凶,增其恶名,以至于尧,尧不能去。缙云氏有不才子,贪于饮食,冒于货贿,侵欲崇侈,不可盈厌,聚敛积实,不知纪极,不分孤寡,不恤穷匮,天下之民以比三凶,谓之饕餮。舜臣尧,宾于四门,流四凶族浑敦、穷奇、檮杌、饕餮,投诸四裔,以御魑魅。是以尧崩而天下如一,同心戴舜以为天子,以其举十六相,去四凶也。故《虞书》数舜之功,曰『慎徽五典,五典克从』,无违教也。曰『纳于百揆,百揆时序』,无废事也。曰『宾于四门,四门穆穆』,无凶人也。

舜有大功二十而为天子,今行父虽未获一吉人,去一凶矣,于舜之功,二十之一也,庶几免于戾乎!」

宋武氏之族道昭公子,将奉司城须以作乱。十二月,宋公杀母弟须及昭公子,使戴、庄、桓之族攻武氏于司马子伯之馆。遂出武、穆之族,使公孙师为司城,公子朝卒,使乐吕为司寇,以靖国人。

\hypertarget{header-n1291}{%
\subsection{宣公}\label{header-n1291}}

\begin{center}\rule{0.5\linewidth}{\linethickness}\end{center}

\hypertarget{header-n1293}{%
\subsubsection{宣公元年}\label{header-n1293}}

【经】元年春王正月,公即位。公子遂如齐逆女。三月,遂以夫人妇姜至自齐。夏,季孙行父如齐。晋放其大夫胥甲父于卫。公会齐侯于平州。公子遂如齐。六月,齐人取济西田。秋,邾子来朝。楚子、郑人侵陈,遂侵宋。晋赵盾帅师救陈。宋公、陈侯、卫侯、曹伯会晋师于棐林,伐郑。冬,晋赵穿帅师侵崇。晋人、宋人伐郑。

【传】元年春,王正月,公子遂如齐逆女,尊君命也。三月,遂以夫人妇姜至自齐,尊夫人也。

夏,季文子如齐,纳赂以请会。

晋人讨不用命者,放胥甲父于卫,而立胥克。先辛奔齐。

会于平州,以定公位。东门襄仲如齐拜成。

六月,齐人取济西之田,为立公故,以赂齐也。

宋人之弑昭公也,晋荀林父以诸侯之师伐宋,宋及晋平,宋文公受盟于晋。又会诸侯于扈,将为鲁讨齐,皆取赂而还。郑穆公曰:``晋不足与也。」遂受盟于楚。陈共公之卒,楚人不礼焉。陈灵公受盟于晋。

秋,楚子侵陈,遂侵宋。晋赵盾帅师救陈、宋。会于棐林,以伐郑也。楚蒍贾救郑,遇于北林。囚晋解扬,晋人乃还。

晋欲求成于秦,赵穿曰:``我侵崇,秦急崇,必救之。吾以求成焉。」冬,赵穿侵崇,秦弗与成。

晋人伐郑,以报北林之役。于是,晋侯侈,赵宣子为政,骤谏而不入,故不竞于楚。

\hypertarget{header-n1306}{%
\subsubsection{宣公二年}\label{header-n1306}}

【经】二年春王二月壬子,宋华元帅师及郑公子归生帅师,战于大棘。宋师败绩,获宋华元。秦师伐晋。夏,晋人、宋人、卫人、陈人侵郑。秋九月乙丑,晋赵盾弑其君夷皋。冬十月乙亥,天王崩。

【传】二年春,郑公子归生受命于楚,伐宋。宋华元、乐吕御之。二月壬子,战于大棘,宋师败绩,囚华元,获乐吕,及甲车四百六十乘,俘二百五十人,馘百人。狂狡辂郑人,郑人入于井,倒戟而出之,获狂狡。君子曰:``失礼违命,宜其为禽也。戎,昭果毅以听之之谓礼,杀敌为果,致果为毅。易之,戮也。」

将战,华元杀羊食士,其御羊斟不与。及战,曰:``畴昔之羊,子为政,今日之事,我为政。」与人郑师,故败。君子谓:``羊斟非人也,以其私憾,败国殄民。于是刑孰大焉。《诗》所谓『人之无良』者,其羊斟之谓乎,残民以逞。」

宋人以兵车百乘、文马百驷以赎华元于郑。半入,华元逃归,立于门外,告而入。见叔佯,曰:``子之马然也。」对曰:``非马也,其人也。」既合而来奔。

宋城,华元为植,巡功。城者讴曰:``睅其目,皤其腹,弃甲而复。于思于思,弃甲复来。」使其骖乘谓之曰:``牛则有皮,犀兕尚多,弃甲则那?」役人曰:``从其有皮,丹漆若何?」华元曰:``去之,夫其口众我寡。」

秦师伐晋,以报崇也,遂围焦。夏,晋赵盾救焦,遂自阴地,及诸侯之师侵郑,以报大棘之役。楚斗椒救郑,曰:``能欲诸侯而恶其难乎?」遂次于郑以待晋师。赵盾曰:``彼宗竞于楚,殆将毙矣。姑益其疾。」乃去之。

晋灵公不君:厚敛以雕墙;从台上弹人,而观其辟丸也;宰夫肠熊蹯不熟,杀之,置诸畚,使妇人载以过朝。赵盾、士季见其手,问其故,而患之。将谏,士季曰:``谏而不入,则莫之继也。会请先,不入则子继之。」三进,及溜,而后视之。曰:``吾知所过矣,将改之。」稽首而对曰:``人谁无过?过而能改,善莫大焉。《诗》曰:『靡不有初,鲜克有终。』夫如是,则能补过者鲜矣。君能有终,则社稷之固也,岂唯群臣赖之。又曰:『衮职有阙,惟仲山甫补之。』能补过也。君能补过,兖不废矣。」犹不改。宣子骤谏,公患之,使锄麑贼之。晨往,寝门辟矣,盛服将朝,尚早,坐而假寐。麑退,叹而言曰:``不忘恭敬,民之主也。贼民之主,不忠。弃君之命,不信。有一于此,不如死也。」触槐而死。

秋九月,晋侯饮赵盾酒,伏甲将攻之。其右提弥明知之,趋登曰:``臣侍君宴,过三爵,非礼也。」遂扶以下,公嗾夫獒焉。明搏而杀之。盾曰:``弃人用犬,虽猛何为。」斗且出,提弥明死之。

初,宣子田于首山,舍于翳桑,见灵辄饿,问其病。曰:``不食三日矣。」食之,舍其半。问之,曰:``宦三年矣,未知母之存否,今近焉,请以遗之。」使尽之,而为之箪食与肉,置诸橐以与之。既而与为公介,倒戟以御公徒,而免之。问何故。对曰:``翳桑之饿人也。」问其名居,不告而退,遂自亡也。

乙丑,赵穿攻灵公于桃园。宣子未出山而复。大史书曰:``赵盾弑其君。」以示于朝。宣子曰:``不然。」对曰:``子为正卿,亡不越竟,反不讨贼,非子而谁?」宣子曰:``乌呼,『我之怀矣,自诒伊戚』,其我之谓矣!」孔子曰:``董孤,古之良史也,书法不隐。赵宣子,古之良大夫也,为法受恶。惜也,越竟乃免。」

宣子使赵穿逆公子黑臀于周而立之。壬申,朝于武宫。

初,丽姬之乱,诅无畜群公子,自是晋无公族。及成公即位,乃宦卿之适子而为之田,以为公族,又宦其馀子亦为余子,其庶子为公行。晋于是有公族、余子、公行。赵盾请以括为公族,曰:``君姬氏之爱子也。微君姬氏,则臣狄人也。」公许之。

冬,赵盾为旄车之族。使屏季以其故族为公族大夫。

\hypertarget{header-n1322}{%
\subsubsection{宣公三年}\label{header-n1322}}

【经】三年春王正月,郊牛之口伤,改卜牛。牛死,乃不郊。犹三望。葬匡王。楚子伐陆浑之戎。夏,楚人侵郑。秋,赤狄侵齐。宋师围曹。冬十月丙戌。郑伯兰卒。葬郑穆公。

【传】三年春,不郊而望,皆非礼也。望,郊之属也。不郊亦无望,可也。

晋侯伐郑,及郔。郑及晋平,士会入盟。

楚子伐陆浑之戎,遂至于洛,观兵于周疆。定王使王孙满劳楚子。楚子问鼎之大小轻重焉。对曰:``在德不在鼎。昔夏之方有德也,远方图物,贡金九牧,铸鼎象物,百物而为之备,使民知神、奸。故民入川泽山林,不逢不若。螭魅罔两,莫能逢之,用能协于上下以承天休。桀有昏德,鼎迁于商,载祀六百。商纣暴虐,鼎迁于周。德之休明,虽小,重也。其建回昏乱,虽大,轻也。天祚明德,有所底止。成王定鼎于郏鄏,卜世三十,卜年七百,天所命也。周德虽衰,天命未改,鼎之轻重,未可问也。」

夏,楚人侵郑,郑即晋故也。

宋文公即位三年,杀母弟须及昭公子。武氏之谋也,使戴、桓之族攻武氏于司马子伯之馆。尽逐武、穆之族。武、穆之族以曹师伐宋。秋,宋师围曹,报武氏之乱也。

冬,郑穆公卒。

初,郑文公有贱妾曰燕姞,梦天使与己兰,曰:``余为伯鯈。余,而祖也,以是为而子。以兰有国香,人服媚之如是。」既而文公见之,与之兰而御之。辞曰:``妾不才,幸而有子,将不信,敢征兰乎。」公曰:``诺。」生穆公,名之曰兰。

文公报郑子之妃,曰陈妫,生子华、子臧。子臧得罪而出。诱子华而杀之南里,使盗杀子臧于陈、宋之间。又娶于江,生公子士。朝于楚,楚人鸩之,及叶而死。又娶于苏,生子瑕、子俞弥。俞弥早卒。泄驾恶瑕,文公亦恶之,故不立也。公逐群公子,公子兰奔晋,从晋文公伐郑。石癸曰:``吾闻姬、姞耦,其子孙必蕃。姞,吉人也,后稷之元妃也,今公子兰,姞甥也。天或启之,必将为君,其后必蕃,先纳之可以亢宠。」与孔将锄、侯宣多纳之,盟于大宫而立之。以与晋平。

穆公有疾,曰:``兰死,吾其死乎,吾所以生也。」刈兰而卒。

\hypertarget{header-n1335}{%
\subsubsection{宣公四年}\label{header-n1335}}

【经】四年春王正月,公及齐侯平莒及郯。莒人不肯。公伐莒,取向。秦伯稻卒。夏六月乙酉,郑公子归生弑其君夷。赤狄侵齐。秋,公如齐。公至自齐。冬,楚子伐郑。

【传】四年春,公及齐侯平莒及郯,莒人不肯。公伐莒,取向,非礼也。平国以礼不以乱,伐而不治,乱也。以乱平乱,何治之有?无治,何以行礼?

楚人献鼋于郑灵公。公子宋与子家将见。子公之食指动,以示子家,曰:``他日我如此,必尝异味。」及入,宰夫将解鼋,相视而笑。公问之,子家以告,及食大夫鼋,召子公而弗与也。子公怒,染指于鼎,尝之而出。公怒,欲杀子公。子公与子家谋先。子家曰:``畜老,犹惮杀之,而况君乎?」反谮子家,子家惧而从之。夏,弑灵公。书曰:``郑公子归生弑其君夷。」权不足也。君子曰:``仁而不武,无能达也。」凡弑君,称君,君无道也;称臣,臣之罪也。

郑人立子良,辞曰:``以贤则去疾不足,以顺则公子坚长。」乃立襄公。襄公将去穆氏,而舍子良。子良不可,曰:``穆氏宜存,则固愿也。若将亡之,则亦皆亡,去疾何为?」乃舍之,皆为大夫。

初,楚司马子良生子越椒,子文曰:``必杀之。是子也,熊虎之状,而豺狼之声,弗杀,必灭若敖氏矣。谚曰:『狼子野心。』是乃狼也,其可畜乎?」子良不可。子文以为大戚,及将死,聚其族,曰:``椒也知政,乃速行矣,无及于难。」且泣曰:``鬼犹求食,若敖氏之鬼,不其馁而?」及令尹子文卒,斗般为令尹,子越为司马。蒍贾为工正,谮子扬而杀之,子越为令尹,己为司马。子越又恶之,乃以若敖氏之族圄伯嬴于□□阳而杀之,遂处烝野,将攻王。王以三王之子为质焉,弗受,师于漳澨。秋七月戊戌,楚子与若敖氏战于皋浒。伯棼射王,汰輈,及鼓跗,着于丁宁。又射汰輈,以贯笠毂。师惧,退。王使巡师曰:``吾先君文王克息,获三矢焉。伯棼窃其二,尽于是矣。」鼓而进之,遂灭若敖氏。

初,若敖娶于云阜,生斗伯比。若敖卒,从其母畜于云阜,淫于云阜子之女,生子文焉云阜夫人使弃诸梦中,虎乳之。云阜子田,见之,惧而归,以告,遂使收之。楚人谓乳谷,谓虎于菟,故命之曰斗谷于菟。以其女妻伯比,实为令尹子文。其孙箴尹克黄使于齐,还,及宋,闻乱。其人曰,``不可以入矣。」箴尹曰:``弃君之命,独谁受之?尹,天也,天可逃乎?」遂归,覆命而自拘于司败。王思子文之治楚国也,曰:``子文无后,何以劝善?」使复其所,改命曰生。

冬,楚子伐郑,郑未服也。

\hypertarget{header-n1345}{%
\subsubsection{宣公五年}\label{header-n1345}}

【经】五年春,公如齐。夏,公至自齐。秋九月,齐高固来逆叔姬。叔孙得臣卒。冬,齐高固及子叔姬来。楚人伐郑。

【传】五年春,公如齐,高固使齐侯止公,请叔姬焉。

夏,公至自齐,书,过也。

秋九月,齐高固来逆女,自为也。故书曰:``逆叔姬。」即自逆也。

冬,来,反马也。

楚子伐郑,陈及楚平。晋荀林父救郑,伐陈。

\hypertarget{header-n1354}{%
\subsubsection{宣公六年}\label{header-n1354}}

【经】六年春,晋赵盾、卫孙免侵陈。夏四月。秋八月,螽。冬十月。

【传】六年春,晋、卫侵陈,陈即楚故也。

夏,定王使子服求后于齐。

秋,赤狄伐晋。围怀,及邢丘。晋侯欲伐之。中行桓子曰:``使疾其民,以盈其贯,将可殪也。《周书》曰:『殪戎殷。』此类之谓也。」

冬,召桓公逆王后于齐。

楚人伐郑,取成而还。

郑公子曼满与王子伯廖语,欲为卿。伯廖告人曰:``无德而贪,其在《周易》《丰》三之《离》三,弗过之矣。」间一岁,郑人杀之。

\hypertarget{header-n1364}{%
\subsubsection{宣公七年}\label{header-n1364}}

【经】七年春,卫侯使孙良夫来盟。夏,公会齐侯伐莱。秋,公至自伐莱。大旱。冬,公会晋侯、宋公、卫侯、郑伯、曹伯于黑壤。

【传】七年春,卫孙桓子来盟,始通,且谋会晋也。

夏,公会齐侯伐莱,不与谋也。凡师出,与谋曰及,不与某曰会。

赤狄侵晋,取向阴之禾。

郑及晋平,公子宋之谋也,故相郑伯以会。冬,盟于黑壤,王叔桓公临之,以谋不睦。

晋侯之立也,公不朝焉,又不使大夫聘,晋人止公于会,盟于黄父。公不与盟,以赂免。故黑壤之盟不书,讳之也。

\hypertarget{header-n1373}{%
\subsubsection{宣公八年}\label{header-n1373}}

【经】八年春,公至自会。夏六月,公子遂如齐,至黄乃复。辛巳,有事于大庙,仲遂卒于垂。壬午,犹绎。万入,去籥。戊子,夫人赢氏薨。晋师、白狄伐秦。楚人灭舒蓼。秋七月甲子,日有食之,既。冬十月己丑,葬我小君敬赢。雨,不克葬。庚寅,日中而克葬。城平阳。楚师伐陈。

【传】八年春,白狄及晋平。夏,会晋伐秦。晋人获秦谍,杀诸绛市,六日而苏。

有事于大庙,襄仲卒而绎,非礼也。

楚为众舒叛,故伐舒蓼,灭之。楚子疆之,及滑汭。盟吴、越而还。

晋胥克有蛊疾,郤缺为政。秋,废胥克。使赵朔佐下军。

冬,葬敬赢。旱,无麻,始用葛茀。雨,不克葬,礼也。礼,卜葬,先远日,辟不怀也。

城平阳,书,时也。

陈及晋平。楚师伐陈,取成而还。

\hypertarget{header-n1384}{%
\subsubsection{宣公九年}\label{header-n1384}}

【经】九年春王正月,公如齐。公至自齐。夏,仲孙蔑如京师。齐侯伐莱。秋,取根牟。八月,滕子卒。九月,晋侯、宋公、卫侯、郑伯、曹伯会于扈。晋荀林父帅师伐陈。辛酉,晋侯黑臀卒于扈。冬十月癸酉,卫侯郑卒。宋人围滕。楚子伐郑。晋郤缺帅师救郑。陈杀其大夫泄冶。

【传】九年春,王使来徵聘。夏,孟献于聘于周,王以为有礼,厚贿之。

秋,取根牟,言易也。

滕昭公卒。

会于扈,讨不睦也。陈侯不会。晋荀林父以诸侯之师伐陈。晋侯卒于扈,乃还。

冬,宋人围滕,因其丧也。

陈灵公与孔宁、仪行父通于夏姬,皆衷其示日服以戏于朝。泄冶谏曰:``公卿宣淫,民无效焉,且闻不令,君其纳之。」公曰:``吾能改矣。」公告二子,二子请杀之,公弗禁,遂杀泄冶。孔子曰:``《诗》云:『民之多辟,无自立辟。』其泄冶之谓乎。」

楚子为厉之役故,伐郑。

晋郤缺救郑,郑伯败楚师于柳棼。国人皆喜,唯子良忧曰:``是国之灾也,吾死无日矣。」

\hypertarget{header-n1396}{%
\subsubsection{宣公十年}\label{header-n1396}}

【经】十年春,公如齐。公至自齐。齐人归我济西田。夏四月丙辰,日有食之。己巳,齐侯元卒。齐崔氏出奔卫。公如齐。五月,公至自齐。癸巳,陈夏征舒弑其君平国。六月,宋师伐滕。公孙归父如齐,葬齐惠公。晋人、宋人、卫人、曹人伐郑。秋,天王使王季子来聘。公孙归父帅师伐邾,取绎。大水。季孙行父如齐。冬,公孙归父如齐。齐侯使国佐来聘。饥。楚子伐郑。

【传】十年春,公如齐。齐侯以我服故,归济西之田。

夏,齐惠公卒。崔杼有宠于惠公,高、国畏其逼也,公卒而逐之,奔卫。书曰``崔氏」,非其罪也,且告以族,不以名。凡诸侯之大夫违,告于诸侯曰:``某氏之守臣某,失守宗庙,敢告。」所有玉帛之使者,则告,不然,则否。

公如齐奔丧。

陈灵公与孔宁、仪行父饮酒于夏氏。公谓行父曰:``征舒似女。」对曰:``亦似君。」征舒病之。公出,自其厩射而杀之。二子奔楚。

滕人恃晋而不事宋,六月,宋师伐滕。

郑及楚平。诸侯之师伐郑,取成而还。

秋,刘康公来报聘。

师伐邾,取绎。

季文子初聘于齐。

冬,子家如齐,伐邾故也。

国武子来报聘。

楚子伐郑。晋士会救郑,逐楚师于颖北。诸侯之师戍郑。郑子家卒。郑人讨幽公之乱,斫子家之棺而逐其族。改葬幽公,谥之曰灵。

\hypertarget{header-n1412}{%
\subsubsection{宣公十一年}\label{header-n1412}}

【经】十有一年春王正月。夏,楚子、陈侯、郑伯盟于辰陵。公孙归父会齐人伐莒。秋,晋侯会狄于欑函。冬十月,楚人杀陈夏征舒。丁亥,楚子入陈。纳公孙宁、仪行父于陈。

【传】十一年春,楚子伐郑,及栎。子良曰:``晋、楚不务德而兵争,与其来者可也。晋、楚无信,我焉得有信。」乃从楚。夏,楚盟于辰陵,陈、郑服也。

楚左尹子重侵宋,王待诸郔。令尹蒍艾猎城沂,使封人虑事,以授司徒。量功命日,分财用,平板干,称畚筑,程土物,议远迩,略基趾,具□粮,度有司,事三旬而成,不愆于素。

晋郤成子求成于众狄,众狄疾赤狄之役,遂服于晋。秋,会于欑函,众狄服也。是行也。诸大夫欲召狄。郤成子曰:``吾闻之,非德,莫如勤,非勤,何以求人?能勤有继,其从之也。《诗》曰:『文王既勤止。』文王犹勤,况寡德乎?」

冬,楚子为陈夏氏乱故,伐陈。谓陈人无动,将讨于少西氏。遂入陈,杀夏征舒,轘诸栗门,因县陈。陈侯在晋。

申叔时使于齐,反,覆命而退。王使让之曰:``夏征舒为不道,弑其君,寡人以诸侯讨而戮之,诸侯、县公皆庆寡人,女独不庆寡人,何故」对曰:``犹可辞乎?」王曰:``可哉」曰:夏征舒弑其君,其罪大矣,讨而戮之,君之义也。抑人亦有言曰:『牵牛以蹊人之田,而夺之牛。』牵牛以蹊者,信有罪矣;而夺之牛,罚已重矣。诸侯之从也,曰讨有罪也。今县陈,贪其富也。以讨召诸侯,而以贪归之,无乃不可乎?王曰:``善哉!」吾未之闻也。反之,可乎?对曰:``可哉!吾侪小人所谓取诸其怀而与之也。」乃复封陈,乡取一人焉以归,谓之夏州。故书曰:``楚子入陈,纳公孙宁、仪行父于陈。」书有礼也。

厉之役,郑伯逃归,自是楚未得志焉。郑既受盟于辰陵,又徼事于晋。

\hypertarget{header-n1422}{%
\subsubsection{宣公十二年}\label{header-n1422}}

【经】十有二年春,葬陈灵公。楚子围郑。夏六月乙卯,晋荀林父帅师及楚子战于邲,晋师败绩。秋七月。冬十有二月戊寅,楚子灭萧。晋人、宋人、卫人、曹人同盟于清丘。宋师伐陈。卫人救陈。

【传】十二年春,楚子围郑。旬有七日,郑人卜行成,不吉。卜临于大宫,且巷出车,吉。国人大临,守陴者皆哭。楚子退师,郑人修城,进复围之,三月克之。入自皇门,至于逵路。郑伯肉袒牵羊以逆,曰:``孤不天,不能事君,使君怀怒以及敝邑,孤之罪也。敢不唯命是听。其俘诸江南以实海滨,亦唯命。其翦以赐诸侯,使臣妾之,亦唯命。若惠顾前好,徼福于厉、宣、桓、武,不泯其社稷,使改事君,夷于九县,君之惠也,孤之愿之,非所敢望也。敢布腹心,君实图之。」左右曰:``不可许也,得国无赦。」王曰:``其君能下人,必能信用其民矣,庸可几乎?」退三十里而许之平。潘□入盟,子良出质。

夏六月,晋师救郑。荀林父将中军,先縠佐之。士会将上军,郤克佐之。赵朔将下军,栾书佐之。赵括、赵婴齐为中军大夫。巩朔、韩穿为上军大夫。荀首、赵同为下军大夫。韩厥为司马。及河,闻郑既及楚平,桓子欲还,曰:``无及于郑而剿民,焉用之?楚归而动,不后。」随武子曰:``善。会闻用师,观衅而动。德刑政事典礼不易,不可敌也,不为是征。楚军讨郑,怒其贰而哀其卑,叛而伐之,服而舍之,德刑成矣。伐叛,刑也;柔服,德也。二者立矣。昔岁入陈,今兹入郑,民不罢劳,君无怨讟,政有经矣。荆尸而举,商农工贾不败其业,而卒乘辑睦,事不奸矣。蒍敖为宰,择楚国之令典,军行,右辕,左追蓐,前茅虑无,中权,后劲,百官象物而动,军政不戒而备,能用典矣。其君之举也,内娃选于亲,外姓选于旧;举不失德,赏不失劳;老有加惠,旅有施舍;君子小人,物有服章,贵有常尊,贱有等威;礼不逆矣。德立,刑行,政成,事时,典从,礼顺,若之何敌之?见可而进,知难而退,军之善政也。兼弱攻昧,武之善经也。子姑整军而经武乎,犹有弱而昧者,何必楚?仲虺有言曰:『取乱侮亡。』兼弱也。《汋》曰:『于铄王师,遵养时晦。』耆昧也。《武》曰:『无竞惟烈。』抚弱耆昧以务烈所,可也。」彘子曰:``不可。晋所以霸,师武臣力也。今失诸侯,不可谓力。有敌而不从,不可谓武。由我失霸,不如死。且成师以出,闻敌强而退,非夫也。命为军师,而卒以非夫,唯群子能,我弗为也。」以中军佐济。

知庄子曰:``此师殆哉。《周易》有之,在《师》三之《临》三,曰:『师出以律,否臧凶。』执事顺成为臧,逆为否,众散为弱,川壅为泽,有律以如己也,故曰律。否臧,且律竭也。盈而以竭,夭且不整,所以凶也。不行谓之《临》,有帅而不从,临孰甚焉!此之谓矣。果遇,必败,彘子尸之。虽免而归,必有大咎。」韩献子谓桓子曰:``彘子以偏师陷,子罪大矣。子为元师,师不用命,谁之罪也?失属亡师,为罪已重,不如进也。事之不捷,恶有所分,与其专罪,六人同之,不犹愈乎?」师遂济。

楚子北师次于郔,沈尹将中军,子重将左,子反将右,将饮马于河而归。闻晋师既济,王欲还,嬖人伍参欲战。令尹孙叔敖弗欲,曰:``昔岁入陈,今兹入郑,不无事矣。战而不捷,参之肉其足食乎?」参曰:``若事之捷,孙叔为无谋矣。不捷,参之肉将在晋军,可得食乎?」令尹南辕反旆,伍参言于王曰:``晋之从政者新,未能行令。其佐先縠刚愎不仁,未肯用命。其三帅者专行不获,听而无上,众谁适从?此行也,晋师必败。且君而逃臣,若社稷何?」王病之,告令尹,改乘辕而北之,次于管以待之。

晋师在敖、鄗之间。郑皇戌使如晋师,曰:``郑之从楚,社稷之故也,未有贰心。楚师骤胜而骄,其师老矣,而不设备,子击之,郑师为承,楚师必败。」彘子曰:``败楚服郑,于此在矣,必许之。」栾武子曰:``楚自克庸以来,其君无日不讨国人而训之于民生之不易,祸至之无日,戒惧之不可以怠。在军,无日不讨军实而申儆之于胜之不可保,纣之百克,而卒无后。训以若敖、蚡冒,筚路蓝缕,以启山林。箴之曰:『民生在勤,勤则不匮。』不可谓骄。先大夫子犯有言曰:『师直为壮,曲为老。』我则不德,而徼怨于楚,我曲楚直,不可谓老。其君之戎,分为二广,广有一卒,卒偏之两。右广初驾,数及日中;左则受之,以至于昏。内官序当其夜,以待不虞,不可谓无备。子良,郑之良也。师叔,楚之崇也。师叔入盟,子良在楚,楚、郑亲矣。来劝我战,我克则来,不克遂往,以我卜也,郑不可从。」赵括、赵同曰:``率师以来,唯敌是求。克敌得属,又何矣?必从彘子。」知季曰:``原、屏,咎之徒也。」赵庄子曰:``栾伯善哉,实其言,必长晋国。」

楚少宰如晋师,曰:``寡君少遭闵凶,不能文。闻二先君之出入此行也,将郑是训定,岂敢求罪于晋。二三子无淹久。」随季对曰:``昔平王命我先君文侯曰:『与郑夹辅周室,毋废王命。』今郑不率,寡君使群臣问诸郑,岂敢辱候人?敢拜君命之辱。」彘子以为谄,使赵括从而更之,曰:``行人失辞。寡君使群臣迁大国之迹于郑,曰:『无辟敌。』群臣无所逃命。」

楚子又使求成于晋,晋人许之,盟有日矣。楚许伯御乐伯,摄叔为右,以致晋师,许伯曰:``吾闻致师者,御靡旌摩垒而还。」乐伯曰:``吾闻致师者,左射以菆,代御执辔,御下两马,掉鞅而还。」摄叔曰:``吾闻致师者,右入垒,折馘,执俘而还。」皆行其所闻而复。晋人逐之,左右角之。乐伯左射马而右射人,角不能进,矢一而已。麋兴于前,射麋丽龟。晋鲍癸当其后,使摄叔奉麋献焉,曰:``以岁之非时,献禽之未至,敢膳诸从者。」鲍癸止之,曰:``其左善射,其右有辞,君子也。」既免。

晋魏錡求公族未得,而怒,欲败晋师。请致师,弗许。请使,许之。遂往,请战而还。楚潘党逐之,及荧泽,见六麋,射一麋以顾献曰:``子有军事,兽人无乃不给于鲜,敢献于从者。」叔党命去之。赵旃求卿未得,且怒于失楚之致师者。请挑战,弗许。请召盟。许之。与魏錡皆命而往。郤献子曰:``二憾往矣,弗备必败。」彘子曰:``郑人劝战,弗敢从也。楚人求成,弗能好也。师无成命,多备何为。」士季曰:``备之善。若二子怒楚,楚人乘我,丧师无日矣。不如备之。楚之无恶,除备而盟,何损于好?若以恶来,有备不败。且虽诸侯相见,军卫不彻,警也。」彘子不可。

士季使巩朔、韩穿帅七覆于敖前,故上军不败。赵婴齐使其徒先具舟于河,故败而先济。

潘党既逐魏錡,赵旃夜至于楚军,席于军门之外,使其徒入之。楚子为乘广三十乘,分为左右。右广鸡鸣而驾,日中而说。左则受之,日入而说。许偃御右广,养由基为右。彭名御左广,屈荡为右。乙卯,王乘左广以逐赵旃。赵旃弃车而走林,屈荡搏之,得其甲裳。晋人惧二子之怒楚师也,使軘车逆之。潘党望其尘,使聘而告曰:``晋师至矣。」楚人亦惧王之入晋军也,遂出陈。孙叔曰:``进之。宁我薄人,无人薄我。《诗》云:『元戎十乘,以先启行。』先人也。《军志》曰:『先人有夺人之心』。薄之也。」遂疾进师,车驰卒奔,乘晋军。桓子不知所为,鼓于军中曰:``先济者有赏。」中军、下军争舟,舟中之指可掬也。

晋师右移,上军未动。工尹齐将右拒卒以逐下军。楚子使唐狡与蔡鸠居告唐惠侯曰:``不谷不德而贪,以遇大敌,不谷之罪也。然楚不克,君之羞也,敢藉君灵以济楚师。」使潘党率游阙四十乘,从唐侯以为左拒,以从上军。驹伯曰:``待诸乎?」随季曰:``楚师方壮,若萃于我,吾师必尽,不如收而去之。分谤生民,不亦可乎?」殿其卒而退,不败。

王见右广,将从之乘。屈荡尸之,曰:``君以此始,亦必以终。」自是楚之乘广先左。

晋人或以广队不能进,楚人惎之脱扃,少进,马还,又惎之拔旆投衡,乃出。顾曰:``吾不如大国之数奔也。」

赵旃以其良马二,济其兄与叔父,以他马反,遇敌不能去,弃车而走林。逢大夫与其二子乘,谓其二子无顾。顾曰:``赵叟在后。」怒之,使下,指木曰:``尸女于是。」授赵旃绥,以免。明日以表尸之,皆重获在木下。

楚熊负羁囚知荦。知庄子以其族反之,厨武子御,下军之士多从之。每射,抽矢,菆,纳诸厨子之房。厨子怒曰:``非子之求而蒲之爱,董泽之蒲,可胜既乎?」知季曰:``不以人子,吾子其可得乎?吾不可以苟射故也。」射连尹襄老,获之,遂载其尸。射公子谷臣,囚之。以二者还。

及昏,楚师军于邲,晋之馀师不能军,宵济,亦终夜有声。

丙辰,楚重至于邲,遂次于衡雍。潘党曰:``君盍筑武军,而收晋尸以为京观。臣闻克敌必示子孙,以无忘武功。」楚子曰:``非尔所知也。夫文,止戈为武。武王克商。作《颂》曰:『载戢干戈,载櫜弓矢。我求懿德,肆于时夏,允王保之。』又作《武》,其卒章曰『耆定尔功』。其三曰:『铺时绎思,我徂求定。』其六曰:『绥万邦,屡丰年。』夫武,禁暴、戢兵、保大、定功、安民、和众、丰财者也。故使子孙无忘其章。今我使二国暴骨,暴矣;观兵以威诸侯,兵不戢矣。暴而不戢,安能保大?犹有晋在,焉得定功?所违民欲犹多,民何安焉?无德而强争诸侯,何以和众?利人之几,而安人之乱,以为己荣,何以丰财?武有七德,我无一焉,何以示子孙?其为先君宫,告成事而已。武非吾功也。古者明王伐不敬,取其鲸鲵而封之,以为大戮,于是乎有京观,以惩淫慝。今罪无所,而民皆尽忠以死君命,又可以为京观乎?」祀于河,作先君宫,告成事而还。

是役也,郑石制实入楚师,将以分郑而立公子鱼臣。辛未,郑杀仆叔子服。君子曰:``史佚所谓毋怙乱者,谓是类也。《诗》曰:『乱离瘼矣,爰其适归?』归于怙乱者也夫。」

郑伯、许男如楚。

秋,晋师归,桓子请死,晋侯欲许之。士贞子谏曰:``不可。城濮之役,晋师三日谷,文公犹有忧色。左右曰:『有喜而忧,如有忧而喜乎?』公曰:『得臣犹在,忧未歇也。困兽犹斗,况国相乎!』及楚杀子玉,公喜而后可知也,曰:『莫馀毒也已。』是晋再克而楚再败也。楚是以再世不竞。今天或者大警晋也,而又杀林父以重楚胜,其无乃久不竞乎?林父之事君也,进思尽忠,退思补过,社稷之卫也,若之何杀之?夫其败也,如日月之食焉,何损于明?」晋侯使复其位。

冬,楚子伐萧,宋华椒以蔡人救萧。萧人囚熊相宜僚及公子丙。王曰:``勿杀,吾退。」萧人杀之。王怒,遂围萧。萧溃。申公巫臣曰:``师人多寒。」王巡三军,拊而勉之。三军之士,皆如挟纩。遂傅于萧。还无社与司马卯言,号申叔展。叔展曰:``有麦曲乎?」曰:``无」。``有山鞠穷乎?」曰:``无」。``河鱼腹疾奈何?」曰:``目于眢井而拯之。」``若为茅絰,哭井则己。」明日萧溃,申叔视其井,则茅絰存焉,号而出之。

晋原縠、宋华椒、卫孔达、曹人同盟于清丘。曰:``恤病讨贰。」于是卿不书,不实其言也。宋为盟故,伐陈。卫人救之。孔达曰:``先君有约言焉,若大国讨,我则死之。」

\hypertarget{header-n1448}{%
\subsubsection{宣公十三年}\label{header-n1448}}

【经】十有三年春,齐师伐莒。夏,楚子伐宋。秋,螽。冬,晋杀其大夫先縠。

【传】十三年春,齐师伐莒,莒恃晋而不事齐故也。

夏,楚子伐宋,以其救萧也。君子曰:``清丘之盟,唯宋可以免焉。」

秋,赤狄伐晋,及清,先縠召之也。

冬,晋人讨邲之败,与清之师,归罪于先縠而杀之,尽灭其族。君子曰:``恶之来也,己则取之,其先縠之谓乎。」

清丘之盟,晋以卫之救陈也讨焉。使人弗去,曰:``罪无所归,将加而师。」孔达曰:``苟利社稷,请以我说。罪我之由。我则为政而亢大国之讨,将以谁任?我则死之。」

\hypertarget{header-n1457}{%
\subsubsection{宣公十四年}\label{header-n1457}}

【经】十有四年春,卫杀其大夫孔达。夏五月壬申,曹伯寿卒。晋侯伐郑。秋九月,楚子围宋。葬曹文公。冬,公孙归父会齐侯于谷。

【传】十四年春,孔达缢而死。卫人以说于晋而免。遂告于诸侯曰:``寡君有不令之臣达,构我敝邑于大国,既伏其罪矣,敢告。」卫人以为成劳,复室其子,使复其位。

夏,晋侯伐郑,为邲故也。告于诸侯,搜焉而还。中行桓子之谋也。曰:``示之以整,使谋而来。」郑人惧,使子张代子良于楚。郑伯如楚,谋晋故也。郑以子良为有礼,故召之。

楚子使申舟聘于齐,曰:``无假道于宋。」亦使公子冯聘于晋,不假道于郑。申舟以孟诸之役恶宋,曰:``郑昭宋聋,晋使不害,我则必死。」王曰:``杀女,我伐之。」见犀而行。及宋,宋人止之,华元曰:``过我而不假道,鄙我也。鄙我,亡也。杀其使者必伐我,伐我亦亡也。亡一也。」乃杀之。楚子闻之,投袂而起,屦及于窒皇,剑及于寝门之外,车及于蒲胥之市。秋九月,楚子围宋。

冬,公孙归父会齐侯于谷。见晏桓子,与之言鲁乐。桓子告高宣子曰:``子家其亡乎,怀于鲁矣。怀必贪,贪必谋人。谋人,人亦谋己。一国谋之,何以不亡?」

孟献子言于公曰:``臣闻小国之免于大国也,聘而献物,于是有庭实旅百。朝而献功,于是有容貌采章嘉淑,而有加货。谋其不免也。诛而荐贿,则无及也。今楚在宋,君其图之。」公说。

\hypertarget{header-n1466}{%
\subsubsection{宣公十五年}\label{header-n1466}}

【经】十有五年春,公孙归父会楚子于宋。夏五月,宋人及楚人平。六月癸卯,晋师灭赤狄潞氏,以潞子婴儿归。秦人伐晋。王札子杀召伯、毛伯。秋,螽。仲孙蔑会齐高固于无娄。初,税亩。冬,蝝生。饥。

【传】十五年春,公孙归父会楚子于宋。

宋人使乐婴齐告急于晋。晋侯欲救之。伯宗曰:``不可。古人有言曰:『虽鞭之长,不及马腹。』天方授楚,未可与争。虽晋之强,能违天乎?谚曰:『高下在心。』川泽纳污,山薮藏疾,瑾瑜匿瑕,国君含垢,天之道也,君其待之。」乃止。使解扬如宋,使无降楚,曰:``晋师悉起,将至矣。」郑人囚而献诸楚,楚子厚赂之,使反其言,不许,三而许之。登诸楼车,使呼宋人而告之。遂致其君命。楚子将杀之,使与之言曰:``尔既许不谷而反之,何故?非我无信,女则弃之,速即尔刑。」对曰:``臣闻之,君能制命为义,臣能承命为信,信载义而行之为利。谋不失利,以卫社稷,民之主也。义无二信,信无二命。君之赂臣,不知命也。受命以出,有死无《员雨》,又可赂乎?臣之许君,以成命也。死而成命,臣之禄也。寡君有信臣,下臣获考死,又何求?」楚子舍之以归。

夏五月,楚师将去宋。申犀稽首于王之马前,曰:``毋畏知死而不敢废王命,王弃言焉。」王不能答。申叔时仆,曰:``筑室反耕者,宋必听命。」从之。宋人惧,使华元夜入楚师,登子反之床,起之曰:``寡君使元以病告,曰:『敝邑易子而食,析骸以爨。虽然,城下之盟,有以国毙,不能从也。去我三十里,唯命是听。』」子反惧,与之盟而告王。退三十里。宋及楚平,华元为质。盟曰:``我无尔诈,尔无我虞。」

潞子婴儿之夫人,晋景公之姊也。酆舒为政而杀之,又伤潞子之目。晋侯将伐之,诸大夫皆曰:``不可。酆舒有三俊才,不如待后之人。」伯宗曰:``必伐之。狄有五罪,俊才虽多,何补焉?不祀,一也。耆酒,二也。弃仲章而夺黎氏地,三也。虐我伯姬,四也。伤其君目,五也。怙其俊才,而不以茂德,兹益罪也。后之人或者将敬奉德义以事神人,而申固其命,若之何待之?不讨有罪,曰将待后,后有辞而讨焉,毋乃不可乎?夫恃才与众,亡之道也。商纣由之,故灭。天反时为灾,地反物为妖,民反德为乱,乱则妖灾生。故文反正为乏。尽在狄矣。」晋侯从之。六月癸卯,晋荀林父败赤狄于曲梁。辛亥,灭潞。酆舒奔卫,卫人归诸晋,晋人杀之。

王孙苏与召氏、毛氏争政,使王子捷杀召戴公及毛伯卫。卒立召襄。

秋七月,秦桓公伐晋,次于辅氏。壬午,晋侯治兵于稷以略狄土,立黎侯而还。及洛,魏颗败秦师于辅氏。获杜回,秦之力人也。

初,魏武子有嬖妾,无子。武子疾,命颗曰:``必嫁是。」疾病,则曰:``必以为殉。」及卒,颗嫁之,曰:``疾病则乱,吾从其治也。」及辅氏之役,颗见老人结草以亢杜回,杜回踬而颠,故获之。夜梦之曰:``余,而所嫁妇人之父也。尔用先人之治命,余是以报。」

晋侯赏桓子狄臣千室,亦赏士伯以瓜衍之县。曰:``吾获狄土,子之功也。微子,吾丧伯氏矣。」羊舌职说是赏也,曰:``《周书》所谓『庸庸祗祗』者,谓此物也夫。士伯庸中行伯,君信之,亦庸士伯,此之谓明德矣。文王所以造周,不是过也。故《诗》曰:『陈锡哉周。』能施也。率是道也,其何不济?」

晋侯使赵同献狄俘于周,不敬。刘康公曰:``不及十年,原叔必有大咎,天夺之魄矣。」

初税亩,非礼也。谷出不过藉,以丰财也。

冬,蝝生,饥。幸之也。

\hypertarget{header-n1481}{%
\subsubsection{宣公十六年}\label{header-n1481}}

【经】十有六年春王正月。晋人灭赤狄甲氏及留吁。夏,成周宣榭火。秋,郯伯姬来归。冬,大有年。

【传】十六年春,晋士会帅师灭赤狄甲氏及留吁、铎辰。

三月,献狄俘。晋侯请于王。戊申,以黻冕命士会将中军,且为大傅。于是晋国之盗逃奔于秦。羊舌职曰:``吾闻之,『禹称善人,不善人远』,此之谓也夫。《诗》曰:『战战兢兢,如临深渊,如履薄冰。』善人在上也。善人在上,则国无幸民。谚曰:『民之多幸,国之不幸也。』是无善人之谓也。」

夏,成周宣榭火,人火之也。凡火,人火曰火,天火曰灾。

秋,郯伯姬来归,出也。

为毛、召之难故,王室复乱。王孙苏奔晋,晋人复之。

冬,晋侯使士会平王室,定王享之,原襄公相礼,殽烝。武子私问其故。王闻之,召武子曰:``季氏,而弗闻乎?王享有体荐,宴有折俎。公当享,卿当宴,王室之礼也。」武子归而讲求典礼,以修晋国之法。

\hypertarget{header-n1491}{%
\subsubsection{宣公十七年}\label{header-n1491}}

【经】十有七年春王正月庚子,许男锡我卒。丁未,蔡侯申卒。夏,葬许昭公。葬蔡文公。六月癸卯,日有食之。己未,公会晋侯、卫侯、曹伯、邾子同盟于断道。秋,公至自会。冬十有一月壬午,公弟叔肸卒。

【传】十七年春,晋侯使郤克征会于齐。齐顷公帷妇人,使观之。郤子登,妇人笑于房。献子怒,出而誓曰:``所不此报,无能涉河。」献子先归,使栾京庐待命于齐,曰:``不得齐事,无覆命矣。」郤子至,请伐齐,晋侯弗许。请以其私属,又弗许。

齐侯使高固、晏弱、蔡朝、南郭偃会。及敛孟,高固逃归。夏,会于断道,讨贰也。盟于卷楚,辞齐人。晋人执晏弱于野王,执蔡朝于原,执南郭偃于温。苗贲皇使,见晏桓子,归言于晋侯曰:``夫晏子何罪?昔者诸侯事吾先君,皆如不逮,举言群臣不信,诸侯皆有贰志。齐君恐不得礼,故不出,而使四子来。左右或沮之,曰:『君不出,必执吾使。』故高子及敛盂而逃。夫三子者曰:『若绝君好,宁归死焉。』为是犯难而来,吾若善逆彼以怀来者。吾又执之,以信齐沮,吾不既过矣乎?过而不改,而又久之,以成其悔,何利之有焉?使反者得辞,而害来者,以惧诸侯,将焉用之?」晋人缓之,逸。

秋八月,晋师还。

范武子将老,召文子曰:``燮乎!吾闻之,喜怒以类者鲜,易者实多。《诗》曰:『君子如怒,乱庶遄沮;君子如祉,乱庶遄已。』君子之喜怒,以已乱也。弗已者,必益之。郤子其或者欲已乱于齐乎?不然,余惧其益之也。余将老,使郤子逞其志,庶有豸乎?尔从二三子唯敬。」乃请老,郤献子为政。

冬,公弟叔肸卒。公母弟也。凡大子之母弟,公在曰公子,不在曰弟。凡称弟,皆母弟也。

\hypertarget{header-n1500}{%
\subsubsection{宣公十八年}\label{header-n1500}}

【经】十有八年春,晋侯、卫世子臧伐齐。公伐杞。夏四月。秋七月,邾人伐鄫子于鄫。甲戌,楚子旅卒。公孙归父如晋。冬十月壬戌,公薨于路寝。归父还自晋,至笙。遂奔齐,

【传】十八年春,晋侯、卫大子臧伐齐,至于阳谷。齐侯会晋侯盟于缯,以公子强为质于晋。晋师还,蔡朝、南郭偃逃归。

夏,公使如楚乞师,欲以伐齐。

秋,邾人戕鄫子于鄫。凡自虐其君曰弑,自外曰戕。

楚庄王卒。楚师不出,既而用晋师,楚于是乎有蜀之役。

公孙归父以襄仲之立公也,有宠,欲去三桓以张公室。与公谋而聘于晋,欲以晋人去之。冬,公薨。季文子言于朝曰:``使我杀适立庶以失大援者,仲也夫。」臧宣叔怒曰:``当其时不能治也,后之人何罪?子欲去之,许请去之。」遂逐东门氏。子家还,及笙,坛帷,覆命于介。既覆命,袒、括发,即位哭,三踊而出。遂奔齐。书曰``归父还自晋。」善之也。

\hypertarget{header-n1508}{%
\subsection{成公}\label{header-n1508}}

\begin{center}\rule{0.5\linewidth}{\linethickness}\end{center}

\hypertarget{header-n1510}{%
\subsubsection{成公元年}\label{header-n1510}}

【经】元年春王正月,公即位。二月辛酉,葬我君宣公。无冰。三月,作丘甲。夏,臧孙许及晋侯盟于赤棘。秋,王师败绩于茅戎。冬十月。

【传】元年春,晋侯使瑕嘉平戎于王,单襄公如晋拜成。刘康公徼戎,将遂伐之。叔服曰:``背盟而欺大国,此必败。背盟,不祥;欺大国,不义;神人弗助,将何以胜?」不听,遂伐茅戎。三月癸未,败绩于徐吾氏。

为齐难故,作丘甲。

闻齐将出楚师,夏,盟于赤棘。

秋,王人来告败。

冬,臧宣叔令修赋、缮完、具守备,曰:``齐、楚结好,我新与晋盟,晋、楚争盟,齐师必至。虽晋人伐齐,楚必救之,是齐、楚同我也。知难而有备,乃可以逞。」

\hypertarget{header-n1519}{%
\subsubsection{成公二年 }\label{header-n1519}}

【经】二年春,齐侯伐我北鄙。夏四月丙戌,卫孙良夫帅师及齐师战于新筑,卫师败绩。六月癸酉,季孙行父、臧孙许、叔孙侨如、公孙婴齐帅师会晋郤克、卫孙良夫、曹公子首及齐侯战于鞍,齐师败绩。秋七月,齐侯使国佐如师。己酉,及国佐盟于袁娄。八月壬卒。宋公鲍卒。庚寅,卫侯速卒。取汶阳田。冬,楚师、郑师侵卫。十有一月,公会楚公子婴齐于蜀。丙申,公及楚人、秦人、宋人、陈人、卫人、郑人、齐人、曹人、邾人、薛人、鄫人盟于蜀。

【传】二年春,齐侯伐我北鄙,围龙。顷公之嬖人卢蒲就魁门焉,龙人囚之。齐侯曰:``勿杀!吾与而盟,无入而封。」弗听,杀而膊诸城上。齐侯亲鼓,士陵城,三日,取龙,遂南侵及巢丘。

卫侯使孙良夫、石稷、宁相、向禽将侵齐,与齐师遇。石子欲还,孙子曰:``不可。以师伐人,遇其师而还,将谓君何?若知不能,则如无出。今既遇矣,不如战也。」

夏,有。

石成子曰:``师败矣。子不少须,众惧尽。子丧师徒,何以覆命?」皆不对。又曰:``子,国卿也。陨子,辱矣。子以众退,我此乃止。」且告车来甚众。齐师乃止,次于鞫居。新筑人仲叔于奚救孙桓子,桓子是以免。

既,卫人赏之以邑,辞。请曲县、繁缨以朝,许之。仲尼闻之曰:``惜也,不如多与之邑。唯器与名,不可以假人,君之所司也。名以出信,信以守器,器以藏礼,礼以行义,义以生利,利以平民,政之大节也。若以假人,与人政也。政亡,则国家从之,弗可止也已。」

孙桓子还于新筑,不入,遂如晋乞师。臧宣叔亦如晋乞师。皆主郤献子。晋侯许之七百乘。郤子曰:``此城濮之赋也。有先君之明与先大夫之肃,故捷。克于先大夫,无能为役,请八百乘。」许之。郤克将中军,士燮佐上军,栾书将下军,韩厥为司马,以救鲁、卫。臧宣叔逆晋师,且道之。季文子帅师会之。及卫地,韩献子将斩人,郤献子驰,将救之,至则既斩之矣。郤子使速以徇,告其仆曰:``吾以分谤也。」

师从齐师于莘。六月壬申,师至于靡笄之下。齐侯使请战,曰:``子以君师,辱于敝邑,不腆敝赋,诘朝请见。」对曰:``晋与鲁、卫,兄弟也。来告曰:『大国朝夕释憾于敝邑之地。』寡君不忍,使群臣请于大国,无令舆师淹于君地。能进不能退,君无所辱命。」齐侯曰:``大夫之许,寡人之愿也;若其不许,亦将见也。」齐高固入晋师,桀石以投人,禽之而乘其车,系桑本焉,以徇齐垒,曰:``欲勇者贾余馀勇。」

癸酉,师陈于□安。邴夏御齐侯,逢丑父为右。晋解张御郤克,郑丘缓为右。齐侯曰:``余姑翦灭此而朝食。」不介马而驰之。郤克伤于矢,流血及屦,未绝鼓音,曰:``余病矣!」张侯曰:``自始合,而矢贯余手及肘,余折以御,左轮朱殷,岂敢言病。吾子忍之!」缓曰:``自始合,苟有险,余必下推车,子岂识之?然子病矣!」张侯曰:``师之耳目,在吾旗鼓,进退从之。此车一人殿之,可以集事,若之何其以病败君之大事也?擐甲执兵,固即死也。病未及死,吾子勉之!」左并辔,右援枹而鼓,马逸不能止,师从之。齐师败绩。逐之,三周华不注。

韩厥梦子舆谓己曰:``旦辟左右。」故中御而从齐侯。邴夏曰:``射其御者,君子也。」公曰:``谓之君子而射之,非礼也。」射其左,越于车下。射其右,毙于车中,綦毋张丧车,从韩厥,曰:``请寓乘。」从左右,皆肘之,使立于后。韩厥俛,定其右。逢丑父与公易位。将及华泉,骖絓于木而止。丑父寝于轏中,蛇出于其下,以肱击之,伤而匿之,故不能推车而及。韩厥执絷马前,再拜稽首,奉觞加璧以进,曰:``寡君使群臣为鲁、卫请,曰:『无令舆师陷入君地。』下臣不幸,属当戎行,无所逃隐。且惧奔辟而忝两君,臣辱戎士,敢告不敏,摄官承乏。」丑父使公下,如华泉取饮。郑周父御佐车,宛伐为右,载齐侯以免。韩厥献丑父,郤献子将戮之。呼曰:``自今无有代其君任患者,有一于此,将为戮乎!」郤子曰:``人不难以死免其君。我戮之不祥,赦之以劝事君者。」乃免之。

齐侯免,求丑父,三入三出。每出,齐师以帅退。入于狄卒,狄卒皆抽戈楯冒之。以入于卫师,卫师免之。遂自徐关入。齐侯见保者,曰:``勉之!齐师败矣。」辟女子,女子曰:``君免乎?」曰:``免矣。」曰:``锐司徒免乎?」曰:``免矣。」曰:``苟君与吾父免矣,可若何!」乃奔。齐侯以为有礼,既而问之,辟司徒之妻也。予之石窌。

晋师从齐师,入自丘舆,击马陉。齐侯使宾媚人赂以纪甗、玉磬与地。不可,则听客之所为。宾媚人致赂,晋人不可,曰:``必以萧同叔子为质,而使齐之封内尽东其亩。」对曰:``萧同叔子非他,寡君之母也。若以匹敌,则亦晋君之母也。吾子布大命于诸侯,而曰:『必质其母以为信。』其若王命何?且是以不孝令也。《诗》曰:『孝子不匮,永锡尔类。』若以不孝令于诸侯,其无乃非德类也乎?先王疆理天下物土之宜,而布其利,故《诗》曰:『我疆我理,南东其亩。』今吾子疆理诸侯,而曰『尽东其亩』而已,唯吾子戎车是利,无顾土宜,其无乃非先王之命也乎?反先王则不义,何以为盟主?其晋实有阙。四王之王也,树德而济同欲焉。五伯之霸也,勤而抚之,以役王命。今吾子求合诸侯,以逞无疆之欲。《诗》曰『布政优优,百禄是遒。』子实不优,而弃百禄,诸侯何害焉!不然,寡君之命使臣则有辞矣,曰:『子以君师辱于敝邑,不腆敝赋以,犒从者。畏君之震,师徒□尧败,吾子惠徼齐国之福,不泯其社稷,使继旧好,唯是先君之敝器、土地不敢爱。子又不许,请收合馀烬,背城借一。敝邑之幸,亦云从也。况其不幸,敢不唯命是听。』」鲁、卫谏曰:``齐疾我矣!其死亡者,皆亲昵也。子若不许,仇我必甚。唯子则又何求?子得其国宝,我亦得地,而纾于难,其荣多矣!齐、晋亦唯天所授,岂必晋?」晋人许之,对曰:``群臣帅赋舆以为鲁、卫请,若苟有以藉口而复于寡君,君之惠也。敢不唯命是听。」

禽郑自师逆公。

秋七月,晋师及齐国佐盟于爰娄,使齐人归我汶阳之田。公会晋师于上鄍,赐三帅先路三命之服,司马、司空、舆帅、候正、亚旅,皆受一命之服。

八月,宋文公卒。始厚葬,用蜃炭,益车马,始用殉。重器备,椁有四阿,棺有翰桧。

君子谓:``华元、乐举,于是乎不臣。臣治烦去惑者也,是以伏死而争。今二子者,君生则纵其惑,死又益其侈,是弃君于恶也。何臣之为?」

九月,卫穆公卒,晋二子自役吊焉,哭于大门之外。卫人逆之,妇人哭于门内,送亦如之。遂常以葬。

楚之讨陈夏氏也,庄王欲纳夏姬,申公巫臣曰:``不可。君召诸侯,以讨罪也。今纳夏姬,贪其色也。贪色为淫,淫为大罚。《周书》曰:『明德慎罚。』文王所以造周也。明德,务崇之之谓也;慎罚,务去之之谓也。若兴诸侯,以取大罚,非慎之也。君其图之!」王乃止。子反欲取之,巫臣曰:``是不祥人也!是夭子蛮,杀御叔,弑灵侯,戮夏南,出孔、仪,丧陈国,何不祥如是?人生实难,其有不获死乎?天下多美妇人,何必是?」子反乃止。王以予连尹襄老。襄老死于邲,不获其尸,其子黑要烝焉。巫臣使道焉,曰:``归!吾聘女。」又使自郑召之,曰:``尸可得也,必来逆之。」姬以告王,王问诸屈巫。对曰:``其信!知荦之父,成公之嬖也,而中行伯之季弟也,新佐中军,而善郑皇戌,甚爱此子。其必因郑而归王子与襄老之尸以求之。郑人惧于邲之役而欲求媚于晋,其必许之。」王遣夏姬归。将行,谓送者曰:``不得尸,吾不反矣。」巫臣聘诸郑,郑伯许之。及共王即位,将为阳桥之役,使屈巫聘于齐,且告师期。巫臣尽室以行。申叔跪从其父将适郢,遇之,曰:``异哉!夫子有三军之惧,而又有《桑中之喜,宜将窃妻以逃者也。」及郑,使介反币,而以夏姬行。将奔齐,齐师新败曰:``吾不处不胜之国。」遂奔晋,而因郤至,以臣于晋。晋人使为邢大夫。子反请以重币锢之,王曰:``止!其自为谋也,则过矣。其为吾先君谋也,则忠。忠,社稷之固也,所盖多矣。且彼若能利国家,虽重币,晋将可乎?若无益于晋,晋将弃之,何劳锢焉。」

晋师归,范文子后入。武子曰:``无为吾望尔也乎?」对曰:``师有功,国人喜以逆之,先入,必属耳目焉,是代帅受名也,故不敢。」武子曰:``吾知免矣。」

郤伯见,公曰:``子之力也夫!」对曰:``君之训也,二三子之力也,臣何力之有焉!」范叔见,劳之如郤伯,对曰:``庚所命也,克之制也,燮何力之有焉!栾伯见,公亦如之,对曰:``燮之诏也,士用命也,书何力之有焉!」

宣公使求好于楚。庄王卒,宣公薨,不克作好。公即位,受盟于晋,会晋伐齐。卫人不行使于楚,而亦受盟于晋,从于伐齐。故楚令尹子重为阳桥之役以求齐。将起师,子重曰:``君弱,群臣不如先大夫,师众而后可。《诗》曰:『济济多士,文王以宁。』夫文王犹用众,况吾侪乎?且先君庄王属之曰:『无德以及远方,莫如惠恤其民,而善用之。』」乃大户,已责,逮鳏,救乏,赦罪,悉师,王卒尽行。彭名御戎,蔡景公为左,许灵公为右。二君弱,皆强冠之。

冬,楚师侵卫,遂侵我,师于蜀。使臧孙往,辞曰:``楚远而久,固将退矣。无功而受名,臣不敢。」楚侵及阳桥,孟孙请往,赂之以执斫、执针、织紝,皆百人。公衡为质,以请盟,楚人许平。

十一月,公及楚公子婴齐、蔡侯、许男、秦右大夫说、宋华元、陈公孙宁、卫孙良夫、郑公子去疾及齐国之大夫盟于蜀。卿不书,匮盟也。于是乎畏晋而窃与楚盟,故曰匮盟。蔡侯、许男不书,乘楚车也,谓之失位。君子曰:``位其不可不慎也乎!蔡、许之君,一失其位,不得列于诸侯,况其下乎?《诗》曰:『不解于位,民之攸塈。』其是之谓矣。」

楚师及宋,公衡逃归。臧宣叔曰:``衡父不忍数年之不宴,以弃鲁国,国将若之何?谁居?后之人必有任是夫!国弃矣。」

是行也,晋辟楚,畏其众也。君子曰:``众之不可以已也。大夫为政,犹以众克,况明君而善用其众乎?《大誓》所谓商兆民离,周十人同者众也。」

晋侯使巩朔献齐捷于周,王弗见,使单襄公辞焉,曰:``蛮夷戎狄,不式王命,淫湎毁常,王命伐之,则有献捷,王亲受而劳之,所以惩不敬,劝有功也。兄弟甥舅,侵败王略,王命伐之,告事而已,不献其功,所以敬亲昵,禁淫慝也。今叔父克遂,有功于齐,而不使命卿镇抚王室,所使来抚余一人,而巩伯实来,未有职司于王室,又奸先王之礼,余虽欲于巩伯、其敢废旧典以忝叔父?夫齐,甥舅之国也,而大师之后也,宁不亦淫从其欲以怒叔父,抑岂不可谏诲?」士庄伯不能对。王使委于三吏,礼之如侯伯克敌使大夫告庆之礼,降于卿礼一等。王以巩伯宴,而私贿之。使相告之曰:``非礼也,勿籍。」

\hypertarget{header-n1548}{%
\subsubsection{成公三年 }\label{header-n1548}}

【经】三年春王正月,公会晋侯、宋公、卫侯、曹伯伐郑。辛亥,葬卫穆公。二月,公至自伐郑。甲子,新宫灾。三日哭。乙亥,葬宋文公。夏,公如晋。郑公子去疾帅师伐许。公至自晋。秋,叔孙侨如帅师围棘。大雩。晋郤克、卫孙良夫伐啬咎如。冬十有一月,晋侯使荀庚来聘。卫侯使孙良夫来聘。丙午,及荀庚盟。丁未,及孙良夫盟。郑伐许。

【传】三年春,诸侯伐郑,次于伯牛,讨邲之役也,遂东侵郑。郑公子偃帅师御之,使东鄙覆诸鄤,败诸丘舆。皇戌如楚献捷。

夏,公如晋,拜汶阳之田。

许恃楚而不事郑,郑子良伐许。

晋人归公子谷臣与连尹襄老之尸于楚,以求知荦。于是荀首佐中军矣,故楚人许之。王送知荦,曰:``子其怨我乎?」对曰:``二国治戎,臣不才,不胜其任,以为俘馘。执事不以衅鼓,使归即戮,君之惠也。臣实不才,又谁敢怨?」王曰:``然则德我乎?」对曰:``二国图其社稷,而求纾其民,各惩其忿以相宥也,两释累囚以成其好。二国有好,臣不与及,其谁敢德?」王曰:``子归,何以报我?」对曰:``臣不任受怨,君亦不任受德,无怨无德,不知所报。」王曰:``虽然,必告不谷。」对曰:``以君之灵,累臣得归骨于晋,寡君之以为戮,死且不朽。若从君之惠而免之,以赐君之外臣首;首其请于寡君而以戮于宗,亦死且不朽。若不获命,而使嗣宗职,次及于事,而帅偏师以修封疆,虽遇执事,其弗敢违。其竭力致死,无有二心,以尽臣礼,所以报也。」王曰:``晋未可与争。」重为之礼而归之。

秋,叔孙侨如围棘,取汶阳之田。棘有服,故围之。

晋郤克、卫孙良夫伐啬咎如,讨赤狄之馀焉。啬咎如溃,上失民也。

冬十一月,晋侯使荀庚来聘,且寻盟。卫侯使孙良夫来聘,且寻盟。公问诸臧宣叔曰:``中行伯之于晋也,其位在三。孙子之于卫也,位为上卿,将谁先?」对曰:``次国之上卿当大国之中,中当其下,下当其上大夫。小国之上卿当大国之下卿,中当其上大夫,下当其下大夫。上下如是,古之制也。卫在晋,不得为次国。晋为盟主,其将先之。」丙午,盟晋,丁未,盟卫,礼也。

十二月甲戌,晋作六军。韩厥、赵括、巩朔、韩穿、荀骓、赵旃皆为卿,赏鞍之功也。

齐侯朝于晋,将授玉。郤克趋进曰:``此行也,君为妇人之笑辱也,寡君未之敢任。」晋侯享齐侯。齐侯视韩厥,韩厥曰:``君知厥也乎?」齐侯曰:``服改矣。」韩厥登,举爵曰:``臣之不敢爱死,为两君之在此堂也。」

荀荦之在楚也,郑贾人有将置诸褚中以出。既谋之,未行,而楚人归之。贾人如晋,荀荦善视之,如实出己,贾人曰:``吾无其功,敢有其实乎?吾小人,不可以厚诬君子。」遂适齐。

\hypertarget{header-n1562}{%
\subsubsection{成公四年}\label{header-n1562}}

【经】四年春,宋公使华元来聘。三月壬申,郑伯坚卒。杞伯来朝。夏四月甲寅,臧孙许卒。公如晋。葬郑襄公。秋,公至自晋。冬,城郓。郑伯伐许。

【传】四年春,宋华元来聘,通嗣君也。

杞伯来朝,归叔姬故也。

夏,公如晋,晋侯见公,不敬。季文子曰:``晋侯必不免。《诗》曰:『敬之敬之!天惟显思,命不易哉!』夫晋侯之命在诸侯矣,可不敬乎?」

秋,公至自晋,欲求成于楚而叛晋,季文子曰:``不可。晋虽无道,未可叛也。国大臣睦,而迩于我,诸侯听焉,未可以贰。史佚之《志》有之,曰:『非我族类,其心必异。』楚虽大,非吾族也,其肯字我乎?」公乃止。

冬十一月,郑公孙申帅师疆许田,许人败诸展陂。郑伯伐许,鉏任、泠敦之田。

晋栾书将中军,荀首佐之,士燮佐上军,以救许伐郑,取汜、祭。楚子反救郑,郑伯与许男讼焉。皇戌摄郑伯之辞,子反不能决也,曰:``君若辱在寡君,寡君与其二三臣共听两君之所欲,成其可知也。不然,侧不足以知二国之成。」

晋赵婴通于赵庄姬。

\hypertarget{header-n1573}{%
\subsubsection{成公四年}\label{header-n1573}}

【经】四年春,宋公使华元来聘。三月壬申,郑伯坚卒。杞伯来朝。夏四月甲寅,臧孙许卒。公如晋。葬郑襄公。秋,公至自晋。冬,城郓。郑伯伐许。

【传】四年春,宋华元来聘,通嗣君也。

杞伯来朝,归叔姬故也。

夏,公如晋,晋侯见公,不敬。季文子曰:``晋侯必不免。《诗》曰:『敬之敬之!天惟显思,命不易哉!』夫晋侯之命在诸侯矣,可不敬乎?」

秋,公至自晋,欲求成于楚而叛晋,季文子曰:``不可。晋虽无道,未可叛也。国大臣睦,而迩于我,诸侯听焉,未可以贰。史佚之《志》有之,曰:『非我族类,其心必异。』楚虽大,非吾族也,其肯字我乎?」公乃止。

冬十一月,郑公孙申帅师疆许田,许人败诸展陂。郑伯伐许,鉏任、泠敦之田。

晋栾书将中军,荀首佐之,士燮佐上军,以救许伐郑,取汜、祭。楚子反救郑,郑伯与许男讼焉。皇戌摄郑伯之辞,子反不能决也,曰:``君若辱在寡君,寡君与其二三臣共听两君之所欲,成其可知也。不然,侧不足以知二国之成。」

晋赵婴通于赵庄姬。

\hypertarget{header-n1584}{%
\subsubsection{成公五年}\label{header-n1584}}

【经】五年春王正月,杞叔姬来归。仲孙蔑如宋。夏,叔孙侨如会晋荀首于谷。梁山崩。秋,大水。冬十有一月己酉,天王崩。十有二月己丑,公会晋侯、齐侯、宋公、卫侯、郑伯、曹伯、邾子、杞伯同盟于虫牢。

【传】五年春,原、屏放诸齐。婴曰:``我在,故栾氏不作。我亡,吾二昆其忧哉!且人各有能有不能,舍我何害?」弗听。婴梦天使谓己:``祭余,余福女。」使问诸士贞伯,贞伯曰:``不识也。」既而告其人曰:``神福仁而祸淫,淫而无罚,福也。祭,其得亡乎?」祭之,之明日而亡。、孟献子如宋,报华元也。

孟献子如宋,报华元也。

夏,晋荀首如齐逆女,故宣伯餫诸谷。

梁山崩,晋侯以传召伯宗。伯宗辟重,曰:``辟传!」重人曰:``待我,不如捷之速也。」问其所,曰:``绛人也。」问绛事焉,曰:``梁山崩,将召伯宗谋之。」问:``将若之何?」曰:``山有朽壤而崩,可若何?国主山川。故山崩川竭,君为之不举,降服,乘缦,彻乐,出次,祝币,史辞以礼焉。其如此而已,虽伯宗若之何?」伯宗请见之,不可。遂以告而从之。

许灵公愬郑伯于楚。六月,郑悼公如楚,讼,不胜。楚人执皇戌及子国。故郑伯归,使公子偃请成于晋。秋八月,郑伯及晋赵同盟于垂棘。

宋公子围龟为质于楚而还,华元享之。请鼓噪以出,鼓噪以复入,曰:``习功华氏。」宋公杀之。

冬,同盟于虫牢,郑服也。诸侯谋复会,宋公使向为人辞以子灵之难。

十一月己酉,定王崩。

\hypertarget{header-n1596}{%
\subsubsection{成公六年}\label{header-n1596}}

【经】六年春王正月,公至自会。二月辛巳,立武宫。取鄟卫孙良夫帅师侵宋。夏六月,邾子来朝。公孙婴齐如晋。壬申,郑伯费卒。秋,仲孙蔑、叔孙侨如帅师侵宋。楚公子婴齐帅师伐郑。冬,季孙行父如晋。晋栾书帅师救郑。

【传】六年春,郑伯如晋拜成,子游相,授玉于东楹之东。士贞伯曰:``郑伯其死乎?自弃也已!视流而行速,不安其位,宜不能久。」

二月,季文子以鞍之功立武宫,非礼也。听于人以救其难,不可以立武。立武由己,非由人也。

取鄟,言易也。

三月,晋伯宗、夏阳说,卫孙良夫、宁相,郑人,伊、洛之戎,陆浑,蛮氏侵宋,以其辞会也。师于金咸,卫人不保。说欲袭卫,曰:``虽不可入,多俘而归,有罪不及死。」伯宗曰:``不可。卫唯信晋,故师在其郊而不设备。若袭之,是弃信也。虽多卫俘,而晋无信,何以求诸侯?」乃止,师还,卫人登陴。

晋人谋去故绛。诸大夫皆曰:``必居郇瑕氏之地,沃饶而近盬,国利君乐,不可失也。」韩献子将新中军,且为仆大夫。公揖而入。献子从。公立于寝庭,谓献子曰:``何如?」对曰:``不可。郇瑕氏土薄水浅,其恶易觏。易觏则民愁,民愁则垫隘,于是乎有沉溺重膇之疾。不如新田,土厚水深,居之不疾,有汾、浍以流其恶,且民从教,十世之利也。夫山、泽、林、盬,国之宝也。国饶,则民骄佚。近宝,公室乃贫,不可谓乐。」公说,从之。夏四月丁丑,晋迁于新田。

六月,郑悼公卒。

子叔声伯如晋。命伐宋。

秋,孟献子、叔孙宣伯侵宋,晋命也。

楚子重伐郑,郑从晋故也。

冬,季文子如晋,贺迁也。

晋栾书救郑,与楚师遇于绕角。楚师还,晋师遂侵蔡。楚公子申、公子成以申、息之师救蔡,御诸桑隧。赵同、赵括欲战,请于武子,武子将许之。知庄子、范文子、韩献子谏曰:``不可。吾来救郑,楚师去我,吾遂至于此,是迁戮也。戮而不已,又怒楚师,战必不克。虽克,不令。成师以出,而败楚之二县,何荣之有焉?若不能败,为辱已甚,不如还也。」乃遂还。

于是,军帅之欲战者众,或谓栾武子曰:``圣人与众同欲,是以济事。子盍从众?子为大政,将酌于民者也。子之佐十一人,其不欲战者,三人而已。欲战者可谓众矣。《商书》曰:『三人占,从二人。』众故也。」武子曰:``善钧,从众。夫善,众之主也。三卿为主,可谓众矣。从之,不亦可乎?」

\hypertarget{header-n1612}{%
\subsubsection{成公七年 }\label{header-n1612}}

【经】七年春王正月,鼷鼠食郊牛角,改卜牛。鼷鼠又食其角,乃免牛。吴伐郯。夏五月,曹伯来朝。不郊,犹三望。秋,楚公子婴齐帅师伐郑。公会晋侯、齐侯、宋公、卫侯、曹伯、莒子、邾子、杞伯救郑。八月戊辰,同盟于马陵。公至自会。吴入州来。冬,大雩。卫孙林父出奔晋。

【传】七年春,吴伐郯,郯成。季文子曰:``中国不振旅,蛮夷入伐,而莫之或恤,无吊者也夫!《诗》曰:『不吊昊天,乱靡有定。』其此之谓乎!有上不吊,其谁不受乱?吾亡无日矣!」君子曰:``如惧如是,斯不亡矣。」

郑子良相成公以如晋,见,且拜师。

夏,曹宣公来朝。

秋,楚子重伐郑,师于汜。诸侯救郑。郑共仲、侯羽军楚师,囚郧公钟仪,献诸晋。

八月,同盟于马陵,寻虫牢之盟,且莒服故也。

晋人以钟仪归,囚诸军府。

楚围宋之役,师还,子重请取于申、吕以为赏田,王许之。申公巫臣曰:``不可。此申、吕所以邑也,是以为赋,以御北方。若取之,是无申、吕也。晋、郑必至于汉。」王乃止。子重是以怨巫臣。子反欲取夏姬,巫臣止之,遂取以行,子反亦怨之。及共王即位,子重、子反杀巫臣之族子阎、子荡及清尹弗忌及襄老之子黑要,而分其室。子重取子阎之室,使沈尹与王子罢分子荡之室,子反取黑要与清尹之室。巫臣自晋遗二子书,曰:``尔以谗慝贪婪事君,而多杀不辜。余必使尔罢于奔命以死。」

巫臣请使于吴,晋侯许之。吴子寿梦说之。乃通吴于晋。以两之一卒适吴,舍偏两之一焉。与其射御,教吴乘车,教之战陈,教之叛楚。置其子狐庸焉,使为行人于吴。吴始伐楚,伐巢、伐徐。子重奔命。马陵之会,吴入州来。子重自郑奔命。子重、子反于是乎一岁七奔命。蛮夷属于楚者,吴尽取之,是以始大,通吴于上国。

卫定公恶孙林父。冬,孙林父出奔晋。卫侯如晋,晋反戚焉。

\hypertarget{header-n1625}{%
\subsubsection{成公八年}\label{header-n1625}}

【经】八年春,晋侯使韩穿来言汶阳之田,归之于齐。晋栾书帅师侵蔡。公孙婴齐如莒。宋公使华元来聘。夏,宋公使公孙寿来纳币。晋杀其大夫赵同、赵括。秋七月,天子使召伯来赐公命。冬十月癸卯,杞叔姬卒。晋侯使士燮来聘。叔孙侨如会晋士燮、齐人、邾人代郯。卫人来媵。

【传】八年春,晋侯使韩穿来言汶阳之田,归之于齐。季文子饯之,私焉,曰:``大国制义以为盟主,是以诸侯怀德畏讨,无有贰心。谓汶阳之田,敝邑之旧也,而用师于齐,使归诸敝邑。今有二命曰:『归诸齐。』信以行义,义以成命,小国所望而怀也。信不可知,义无所立,四方诸侯,其谁不解体?《诗》曰:『女也不爽,士贰其行。士也罔极,二三其德。』七年之中,一与一夺,二三孰甚焉!士之二三,犹丧妃耦,而况霸主?霸主将德是以,而二三之,其何以长有诸侯乎?《诗》曰:『犹之未远,是用大简。』行父惧晋之不远犹而失诸侯也,是以敢私言之。」

晋栾书侵蔡,遂侵楚获申骊。楚师之还也,晋侵沈,获沈子揖初,从知、范、韩也。君子曰:``从善如流,宜哉!《诗》曰:『恺悌君子,遐不作人。』求善也夫!作人,斯有功绩矣。」是行也,郑伯将会晋师,门于许东门,大获焉。

声伯如莒,逆也。

宋华元来聘,聘共姬也。

夏,宋公使公孙寿来纳币,礼也。

晋赵庄姬为赵婴之亡故,谮之于晋侯,曰:``原、屏将为乱。」栾、郤为征。六月,晋讨赵同、赵括。武从姬氏畜于公宫。以其田与祁奚。韩厥言于晋侯曰:``成季之勋,宣孟之忠,而无后,为善者其惧矣。三代之令王,皆数百年保天之禄。夫岂无辟王,赖前哲以免也。《周书》曰:『不敢侮鳏寡。』所以明德也。」乃立武,而反其田焉。

秋,召桓公来赐公命。

晋侯使申公巫臣如吴,假道于莒。与渠丘公立于池上,曰:``城已恶!」莒子曰:``辟陋在夷,其孰以我为虞?」对曰:``夫狡焉思启封疆以利社稷者,何国蔑有?唯然,故多大国矣,唯或思或纵也。勇夫重闭,况国乎?」

冬,杞叔姬卒。来归自杞,故书。

晋士燮来聘,言伐郯也,以其事吴故。公赂之,请缓师,文子不可,曰:``君命无贰,失信不立。礼无加货,事无二成。君后诸侯,是寡君不得事君也。燮将复之。」季孙惧,使宣伯帅师会伐郯。

卫人来媵共姬,礼也。凡诸侯嫁女,同姓媵之,异姓则否。

\hypertarget{header-n1640}{%
\subsubsection{成公九年 }\label{header-n1640}}

【经】九年春王正月,杞伯来逆叔姬之丧以归。公会晋侯、齐侯、宋公、卫侯、郑伯、曹伯、莒子、杞伯,同盟于蒲。公至自会。二月伯姬归于宋。夏,季孙行父如宋致女。晋人来媵。秋七月丙子,齐侯无野卒。晋人执郑伯。晋栾书帅师伐郑。冬十有一月,葬齐顷公。楚公子婴齐帅师伐莒。庚申,莒溃。楚人入郓。秦人、白狄伐晋。郑人围许。城中城。

【传】九年春,杞桓公来逆叔姬之丧,请之也。杞叔姬卒,为杞故也。逆叔姬,为我也。

为归汶阳之田故,诸侯贰于晋。晋人惧,会于蒲,以寻马陵之盟。季文子谓范文子曰:``德则不竞,寻盟何为?」范文子曰:``勤以抚之,宽以待之,坚强以御之,明神以要之,柔服而伐贰,德之次也。」是行也,将始会吴,吴人不至。

二月,伯姬归于宋。

楚人以重赂求郑,郑伯会楚公子成于邓。

夏,季文子如宋致女,覆命,公享之。赋《韩奕》之五章,穆姜出于房,再拜,曰:``大夫勤辱,不忘先君以及嗣君,施及未亡人。先君犹有望也!敢拜大夫之重勤。」又赋《绿衣》之卒章而入。

晋人来媵,礼也。

秋,郑伯如晋。晋人讨其贰于楚也,执诸铜鞮。

栾书伐郑,郑人使伯蠲行成,晋人杀之,非礼也。兵交,使在其间可也。楚子重侵陈以救郑。

晋侯观于军府,见钟仪,问之曰:``南冠而絷者,谁也?」有司对曰:``郑人所献楚囚也。」使税之,召而吊之。再拜稽首。问其族,对曰:``泠人也。」公曰:``能乐乎?」对曰:``先父之职官也,敢有二事?」使与之琴,操南音。公曰:``君王何如?」对曰:``非小人之所得知也。」固问之,对曰:``其为大子也,师保奉之,以朝于婴齐而夕于侧也。不知其他。」公语范文子,文子曰:``楚囚,君子也。言称先职,不背本也。乐操土风,不忘旧也。称大子,抑无私也。名其二卿,尊君也。不背本,仁也。不忘旧,信也。无私,忠也。尊君。敏也。仁以接事,信以守之,忠以成之,敏以行之。事虽大,必济。君盍归之,使合晋、楚之成。」公从之,重为之礼,使归求成。

冬十一月,楚子重自陈伐莒,围渠丘。渠丘城恶,众溃,奔莒。戊申,楚入渠丘。莒人囚楚公子平,楚人曰:``勿杀!吾归而俘。」莒人杀之。楚师围莒。莒城亦恶,庚申,莒溃。楚遂入郓,莒无备故也。

君子曰:``恃陋而不备,罪之大者也;备豫不虞,善之大者也。莒恃其陋,而不修城郭,浃辰之间,而楚克其三都,无备也夫!《诗》曰:『虽有丝、麻,无弃菅、蒯;虽有姬、姜,无弃蕉萃。凡百君子,莫不代匮。』言备之不可以已也。」

秦人、白狄伐晋,诸侯贰故也。

郑人围许,示晋不急君也。是则公孙申谋之,曰:``我出师以围许,为将改立君者,而纾晋使,晋必归君。」

城中城,书,时也。

十二月,楚子使公子辰如晋,报钟仪之使,请修好结成。

\hypertarget{header-n1659}{%
\subsubsection{成公十年}\label{header-n1659}}

【经】十年春,卫侯之弟黑背帅师侵郑。夏四月,五卜郊,不从,乃不郊。五月,公会晋侯、齐侯、宋公、卫侯、曹伯伐郑。齐人来媵。丙午,晋侯獳卒。秋七月,公如晋。冬十月。

【传】十年春,晋侯使籴伐如楚,报大宰子商之使也。

卫子叔黑背侵郑,晋命也。

郑公子班闻叔申之谋。三月,子如立公子繻。夏四月,郑人杀繻,立髡顽。子如奔许。栾武子曰:``郑人立君,我执一人焉,何益?不如伐郑而归其君,以求成焉。」晋侯有疾。五月,晋立大子州蒲以为君,而会诸侯伐郑。郑子罕赂以襄钟,子然盟于修泽,子驷为质。辛巳,郑伯归。

晋侯梦大厉,被发及地,搏膺而踊,曰:``杀余孙,不义。余得请于帝矣!」坏大门及寝门而入。公惧,入于室。又坏户。公觉,召桑田巫。巫言如梦。公曰:``何如?曰:``不食新矣。」公疾病,求医于秦。秦伯使医缓为之。未至,公梦疾为二竖子,曰:``彼,良医也。惧伤我,焉逃之?」其一曰:``居肓之上,膏之下,若我何?」医至,曰:``疾不可为也。在肓之上,膏之下,攻之不可,达之不及,药不至焉,不可为也。」公曰:``良医也。」厚为之礼而归之。六月丙午,晋侯欲麦,使甸人献麦,馈人为之。召桑田巫,示而杀之。将食,张,如厕,陷而卒。小臣有晨梦负公以登天,及日中,负晋侯出诸厕,遂以为殉。

郑伯讨立君者,戊申,杀叔申、叔禽。君子曰:``忠为令德,非其人犹不可,况不令乎?」

秋,公如晋。晋人止公,使送葬。于是籴伐未反。

冬,葬晋景公。公送葬,诸侯莫在。鲁人辱之,故不书,讳之也。

\hypertarget{header-n1670}{%
\subsubsection{成公十一年}\label{header-n1670}}

【经】十有一年春王三月,公至自晋。晋侯使郤犨来聘,己丑,及郤犨盟。夏,季孙行父如晋。秋,叔孙侨如如齐。冬十月。

【传】十一年春,王三月,公至自晋。晋人以公为贰于楚,故止公。公请受盟,而后使归。

郤犨来聘,且莅盟。

声伯之母不聘,穆姜曰:``吾不以妾为姒。」生声伯而出之,嫁于齐管于奚。生二子而寡,以归声伯。声伯以其外弟为大夫,而嫁其外妹于施孝叔。郤犨来聘,求妇于声伯。声伯夺施氏妇以与之。妇人曰:``鸟兽犹不失俪,子将若何?」曰:``吾不能死亡。」妇人遂行,生二子于郤氏。郤氏亡,晋人归之施氏,施氏逆诸河,沉其二子。妇人怒曰:``己不能庇其伉俪而亡之,又不能字人之孤而杀之,将何以终?」遂誓施氏。

夏,季文子如晋报聘,且莅盟也。

周公楚恶惠、襄之逼也,且与伯与争政,不胜,怒而出。及阳樊,王使刘子复之,盟于鄄而入。三日,复出奔晋。

秋,宣伯聘于齐,以修前好。

晋郤至与周争鄇田,王命刘康公、单襄公讼诸晋。郤至曰:``温,吾故也,故不敢失。」刘子、单子曰:``昔周克商,使诸侯抚封,苏忿生以温为司寇,与檀伯达封于河。苏氏即狄,又不能于狄而奔卫。襄王劳文公而赐之温,狐氏、阳氏先处之,而后及子。若治其故,则王官之邑也,子安得之?」晋侯使郤至勿敢争。

宋华元善于令尹子重,又善于栾武子。闻楚人既许晋籴伐成,而使归覆命矣。冬,华元如楚,遂如晋,合晋、楚之成。

秦、晋为成,将会于令狐。晋侯先至焉,秦伯不肯涉河,次于王城,使史颗盟晋侯于河东。晋郤犨盟秦伯于河西。范文子曰:``是盟也何益?齐盟,所以质信也。会所,信之始也。始之不从,其何质乎?」秦伯归而背晋成。

\hypertarget{header-n1683}{%
\subsubsection{成公十二年}\label{header-n1683}}

【经】十有二年春,周公出奔晋。夏,公会晋侯、卫侯于琐泽。秋,晋人败狄于交刚。冬十月。

【传】十二年春,王使以周公之难来告。书曰:``周公出奔晋。」凡自周无出,周公自出故也。

宋华元克合晋、楚之成。夏五月,晋士燮会楚公子罢、许偃。癸亥,盟于宋西门之外,曰:``凡晋、楚无相加戎,好恶同之,同恤菑危,备救凶患。若有害楚,则晋伐之。在晋,楚亦如之。交贽往来,道路无壅,谋其不协,而讨不庭有渝此盟,明神殛之,俾队其师,无克胙国。」郑伯如晋听成,会于琐泽,成故也。

狄人间宋之盟以侵晋,而不设备。秋,晋人败狄于交刚。

晋郤至如楚聘,且莅盟。楚子享之,子反相,为地室而县焉。郤至将登,金奏作于下,惊而走出。子反曰:``日云莫矣,寡君须矣,吾子其入也!」宾曰:``君不忘先君之好,施及下臣,贶之以大礼,重之以备乐。如天之福,两君相见,何以代此。下臣不敢。」子反曰:``如天之福,两君相见,无亦唯是一矢以相加遗,焉用乐?寡君须矣,吾子其入也!」宾曰:``若让之以一矢,祸之大者,其何福之为?世之治也,诸侯间于天子之事,则相朝也,于是乎有享宴之礼。享以训共俭,宴以示慈惠。共俭以行礼,而慈惠以布政。政以礼成,民是以息。百官承事,朝而不夕,此公侯之所以扞城其民也。故《诗》曰:『赳赳武夫,公侯干城。』及其乱也,诸侯贪冒,侵欲不忌,争寻常以尽其民,略其武夫,以为己腹心股肱爪牙。故《诗》曰:『赳赳武夫,公侯腹心。』天下有道,则公侯能为民干城,而制其腹心。乱则反之。今吾子之言,乱之道也,不可以为法。然吾子,主也,至敢不从?」遂入,卒事。归,以语范文子。文子曰:``无礼必食言,吾死无日矣夫!」

冬,楚公子罢如晋聘,且莅盟。十二月,晋侯及楚公子罢盟于赤棘。

\hypertarget{header-n1692}{%
\subsubsection{成公十三年 }\label{header-n1692}}

【经】十有三年春,晋侯使郤錡来乞师。三月,公如京师。夏五月,公自京师,遂会晋侯、齐侯、宋公、卫侯、郑伯、曹伯、邾人、滕人伐秦。曹伯卢卒于师。秋七月,公至自伐秦。冬,葬曹宣公。

【传】十三年春,晋侯使郤錡来乞师,将事不敬。孟献子曰:``郤氏其亡乎!礼,身之干也。敬,身之基也。郤子无基。且先君之嗣卿也,受命以求师,将社稷是卫,而惰,弃君命也。不亡何为?」

三月,公如京师。宣伯欲赐,请先使,王以行人之礼礼焉。孟献子从。王以为介,而重贿之。

公及诸侯朝王,遂从刘康公、成肃公会晋侯伐秦。成子受脤于社,不敬。刘子曰:``吾闻之,民受天地之中以生,所谓命也。是以有动作礼义威仪之则,以定命也。能者养以之福,不能者败以取祸。是故君子勤礼,小人尽力,勤礼莫如致敬,尽力莫如敦笃。敬在养神,笃在守业。国之大事,在祀与戎,祀有执膰,戎有受脤,神之大节也。今成子惰,弃其命矣,其不反乎?」

夏四月戊午,晋侯使吕相绝秦,曰:``昔逮我献公,及穆公相好,戮力同心,申之以盟誓,重之以昏姻。天祸晋国,文公如齐,惠公如秦。无禄,献公即世,穆公不忘旧德,俾我惠公用能奉祀于晋。又不能成大勋,而为韩之师。亦悔于厥心,用集我文公,是穆之成也。文公躬擐甲胄,跋履山川,逾越险阻,征东之诸侯,虞、夏、商、周之胤,而朝诸秦,则亦既报旧德矣。郑人怒君之疆埸,我文公帅诸侯及秦围郑。秦大夫不询于我寡君,擅及郑盟。诸侯疾之,将致命于秦。文公恐惧,绥静诸侯,秦师克还无害,则是我有大造于西也。无禄,文公即世,穆为不吊,蔑死我君,寡我襄公,迭我淆地,奸绝我好,伐我保城,殄灭我费滑,散离我兄弟,挠乱我同盟,倾覆我国家。我襄公未忘君之旧勋,而惧社稷之陨,是以有淆之师。犹愿赦罪于穆公,穆公弗听,而即楚谋我。天诱其衷,成王殒命,穆公是以不克逞志于我。穆、襄即世,康、灵即位。康公,我之自出,又欲阙翦我公室,倾覆我社稷,帅我蝥贼,以来荡摇我边疆。我是以有令狐之役。康犹不悛,入我河曲,伐我涷川,俘我王官,翦我羁马,我是以有河曲之战。东道之不通,则是康公绝我好也。

及君之嗣也,我君景公引领西望曰:『庶抚我乎!』君亦不惠称盟,利吾有狄难,入我河县,焚我箕、郜,芟夷我农功,虔刘我边陲。我是以有辅氏之聚。``君亦悔祸之延,而欲徼福于先君献、穆,使伯车来,命我景公曰:『吾与女同好弃恶,复修旧德,以追念前勋,』言誓未就,景公即世,我寡君是以有令狐之会。君又不祥,背弃盟誓。白狄及君同州,君之仇仇,而我之昏姻也。君来赐命曰:『吾与女伐狄。』寡君不敢顾昏姻,畏君之威,而受命于吏。君有二心于狄,曰:『晋将伐女。』狄应且憎,是用告我。楚人恶君之二三其德也,亦来告我曰:『秦背令狐之盟,而来求盟于我:``昭告昊天上帝、秦三公、楚三王曰:『余虽与晋出入,余唯利是视。』不谷恶其无成德,是用宣之,以惩不壹。」诸侯备闻此言,斯是用痛心疾首,昵就寡人。寡人帅以听命,唯好是求。君若惠顾诸侯,矜哀寡人,而赐之盟,则寡人之愿也。其承宁诸侯以退,岂敢徼乱。君若不施大惠,寡人不佞,其不能以诸侯退矣。敢尽布之执事,俾执事实图利之!」

秦桓公既与晋厉公为令狐之盟,而又召狄与楚,欲道以伐晋,诸侯是以睦于晋。晋栾书将中军,荀庚佐之。士燮将上军,郤錡佐之。韩厥将下军,荀罃佐之。赵旃将新军,郤至佐之。郤毅御戎,栾金咸为右。孟献子曰:``晋帅乘和,师必有大功。」五月丁亥,晋师以诸侯之师及秦师战于麻隧。秦师败绩,获秦成差及不更女父。曹宣公卒于师。师遂济泾,及侯丽而还。迓晋侯于新楚。

成肃公卒于瑕。

六月丁卯夜,郑公子班自訾求入于大宫,不能,杀子印、子羽。反军于市,己巳,予驷帅国人盟于大宫,遂从而尽焚之,杀子如、子□龙、孙叔、孙知。

曹人使公子负刍守,使公子欣时逆曹伯之丧。秋,负刍杀其大子而自立也。诸侯乃请讨之,晋人以其役之劳,请俟他年。冬,葬曹宣公。既葬,子臧将亡,国人皆将从之。成公乃惧,告罪,且请焉,乃反,而致其邑。

\hypertarget{header-n1705}{%
\subsubsection{成公十四年}\label{header-n1705}}

【经】十有四年春王正月,莒子朱卒。夏,卫孙林父自晋归于卫。秋,叔孙侨如如齐逆女。郑公子喜帅师伐许。九月,侨如以夫人妇姜氏至自齐。冬十月庚寅,卫侯臧卒。秦伯卒。

【传】十四年春,卫侯如晋,晋侯强见孙林父焉,定公不可。夏,卫侯既归,晋侯使郤犨送孙林父而见之。卫侯欲辞,定姜曰:``不可。是先君宗卿之嗣也,大国又以为请,不许,将亡。虽恶之,不犹愈于亡乎?君其忍之!安民而宥宗卿,不亦可乎?」卫侯见而复之。

卫侯飨苦成叔,宁惠子相。苦成叔傲。宁子曰:``苦成家其亡乎!古之为享食也,以观威仪、省祸福也。故《诗》曰:『兕觥其觩,旨酒思柔,彼交匪傲,万福来求。』今夫子傲,取祸之道也。」

秋,宣伯如齐逆女。称族,尊君命也。

八月,郑子罕伐许,败焉。戊戌,郑伯复伐许。庚子,入其郛。许人平以叔申之封。

九月,侨如以夫人妇姜氏至自齐。舍族,尊夫人也。故君子曰:``《春秋》之称,微而显,志而晦,婉而成章,尽而不污,惩恶而劝善。非圣人谁能修之?」

卫侯有疾,使孔成子、宁惠子立敬姒之子衎以为大子。冬十月,卫定公卒。夫人姜氏既哭而息,见大子之不哀也,不内酌饮。叹曰:``是夫也,将不唯卫国之败,其必始于未亡人!乌呼!天祸卫国也夫!吾不获鱄也使主社稷。」大夫闻之,无不耸惧。孙文子自是不敢舍其重器于卫,尽置诸戚,而甚善晋大夫。

\hypertarget{header-n1715}{%
\subsubsection{成公十五年}\label{header-n1715}}

【经】十有五年春王二月,葬卫定公。三月乙巳,仲婴齐卒。癸丑,公会晋侯、卫侯、郑伯、曹伯、宋世子成、齐国佐,邾人同盟于戚。晋侯执曹伯归于京师。公至自会。夏六月,宁公固卒。楚子伐郑。秋八月庚辰,葬宋共公。宋华元出奔晋。宋华元自晋归于宋。宋杀其大夫山。宋鱼石出奔楚。冬十有一月,叔孙侨如会晋士燮、齐高无咎、宋华元、卫孙林父、郑公子酉、邾人会吴于钟离。许迁于叶。

【传】十五年春,会于戚,讨曹成公也。执而归诸京师。书曰:``晋侯执曹伯。」不及其民也。凡君不道于其民,诸侯讨而执之,则曰某人执某侯。不然,则否。

诸侯将见子臧于王而立之,子臧辞曰:``《前志》有之,曰:『圣达节,次守节,下失节。』为君,非吾节也。虽不能圣,敢失守乎?」遂逃,奔宋。

夏六月,宋共公卒。

楚将北师。子囊曰:``新与晋盟而背之,无乃不可乎?」子反曰:``敌利则进,何盟之有?」申叔时老矣,在申,闻之,曰:``子反必不免。信以守礼,礼以庇身,信礼之亡,欲免得乎?」楚子侵郑,及暴隧,遂侵卫,及首止。郑子罕侵楚,取新石。栾武子欲报楚,韩献子曰:``无庸,使重其罪,民将叛之。无民,孰战?」

秋八月,葬宋共公。于是华元为右师,鱼石为左师,荡泽为司马,华喜为司徒,公孙师为司城,向为人为大司寇,鳞朱为少司寇,向带为大宰,鱼府为少宰。荡泽弱公室,杀公子肥。华元曰:``我为右师,君臣之训,师所司也。今公室卑而不能正,吾罪大矣。不能治官,敢赖宠乎?」乃出奔晋。

二华,戴族也;司城,庄族也;六官者,皆桓族也。鱼石将止华元,鱼府曰:``右师反,必讨,是无桓氏也。」鱼石曰:``右师苟获反,虽许之讨,必不敢。且多大功,国人与之,不反,惧桓氏之无祀于宋也。右师讨,犹有戌在,桓氏虽亡,必偏。」鱼石自止华元于河上。请讨,许之,乃反。使华喜、公孙师帅国人攻荡氏,杀子山。书曰:``宋杀大夫山。」言背其族也。

鱼石、向为人、鳞朱、向带、鱼府出舍于睢上。华元使止之,不可。冬十月,华元自止之,不可。乃反。鱼府曰:``今不从,不得入矣。右师视速而言疾,有异志焉。若不我纳,今将驰矣。」登丘而望之,则驰。聘而从之,则决睢澨,闭门登陴矣。左师、二司寇、二宰遂出奔楚。华元使向戌为左师,老佐为司马,乐裔为司寇,以靖国人。

晋三郤害伯宗,谮而杀之,及栾弗忌。伯州犁奔楚。韩献子曰:``郤氏其不免乎!善人,天地之纪也,而骤绝之,不亡何待?」

初,伯宗每朝,其妻必戒之曰:``『盗憎主人,民恶其上。』子好直言,必及于难。」

十一月,会吴于钟离,始通吴也。

许灵公畏逼于郑,请迁于楚。辛丑,楚公子申迁许于叶。

\hypertarget{header-n1730}{%
\subsubsection{成公十六年}\label{header-n1730}}

【经】十有六年春王正月,雨,木冰。夏四月辛未,滕子卒。郑公子喜帅师侵宋。六月丙寅朔,日有食之。晋侯使栾□来乞师。甲午晦,晋侯及楚子、郑伯战于鄢陵。楚子、郑师败绩。楚杀其大夫公子侧。秋,公会晋侯、齐侯、卫侯、宋华元、邾人于沙随,不见公。公至自会。公会尹子,晋侯、齐国佐、邾人伐郑。曹伯归自京师。九月,晋人执季孙行父,舍之于苕丘。冬十月乙亥,叔孙侨如出奔齐。十有二月乙丑,季孙行父及晋郤犨盟于扈。公至自会。乙酉,刺公子偃。

【传】十六年春,楚子自武城使公子成以汝阴之田求成于郑。郑叛晋,子驷从楚子盟于武城。

夏四月,滕文公卒。

郑子罕伐宋,宋将鉏、乐惧败诸汋陂。退,舍于夫渠,不儆,郑人覆之,败诸汋陵,获将鉏、乐惧。宋恃胜也。

卫侯伐郑,至于鸣雁,为晋故也。

晋侯将伐郑,范文子曰:``若逞吾愿,诸侯皆叛,晋可以逞。若唯郑叛,晋国之忧,可立俟也。」栾武子曰:``不可以当吾世而失诸侯,必伐郑。」乃兴师。栾书将中军,士燮佐之。郤錡将上军,荀偃佐之。韩厥将下军,郤至佐新军,荀罃居守。郤犨如卫,遂如齐,皆乞师焉。栾□来乞师,孟献子曰:``有胜矣。」戊寅,晋师起。

郑人闻有晋师,使告于楚,姚句耳与往。楚子救郑,司马将中军,令尹将左,右尹子辛将右。过申,子反入见申叔时,曰:``师其何如?」对曰:``德、刑、详、义、礼、信,战之器也。德以施惠,刑以正邪,详以事神,义以建利,礼以顺时,信以守物。民生厚而德正,用利而事节,时顺而物成。上下和睦,周旋不逆,求无不具,各知其极。故《诗》曰:『立我烝民,莫匪尔极。』是以神降之福,时无灾害,民生敦庞,和同以听,莫不尽力以从上命,致死以补其阙。此战之所由克也。今楚内弃其民,而外绝其好,渎齐盟,而食话言,奸时以动,而疲民以逞。民不知信,进退罪也。人恤所底,其谁致死?子其勉之!吾不复见子矣。」姚句耳先归,子驷问焉,对曰:``其行速,过险而不整。速则失志,不整丧列。志失列丧,将何以战?楚惧不可用也。」

五月,晋师济河。闻楚师将至,范文子欲反,曰:``我伪逃楚,可以纾忧。夫合诸侯,非吾所能也,以遗能者。我若群臣辑睦以事君,多矣。」武子曰:``不可。」

六月,晋、楚遇于鄢陵。范文子不欲战,郤至曰:``韩之战,惠公不振旅。箕之役,先轸不反命,邲之师,荀伯不复从。皆晋之耻也。子亦见先君之事矣。今我辟楚,又益耻也。」文子曰:``吾先君之亟战也,有故。秦、狄、齐、楚皆强,不尽力,子孙将弱。今三强服矣,敌楚而已。唯圣人能外内无患,自非圣人,外宁必有内忧。盍释楚以为外惧乎?」

甲午晦,楚晨压晋军而陈。军吏患之。范□趋进,曰:``塞井夷灶,陈于军中,而疏行首。晋、楚唯天所授,何患焉?」文子执戈逐之,曰:``国之存亡,天也。童子何知焉?」栾书曰:``楚师轻窕,固垒而待之,三日必退。退而击之,必获胜焉。」郤至曰:``楚有六间,不可失也。其二卿相恶。王卒以旧。郑陈而不整。蛮军而不陈。陈不违晦,在陈而嚣,合而加嚣,各顾其后,莫有斗心。旧不必良,以犯天忌。我必克之。」

楚子登巢车以望晋军,子重使大宰伯州犁侍于王后。王曰:``骋而左右,何也?」曰:``召军吏也。」``皆聚于军中矣!」曰:``合谋也。」``张幕矣。」曰:``虔卜于先君也。」``彻幕矣!」曰:``将发命也。」``甚嚣,且尘上矣!」曰:``将塞井夷灶而为行也。」``皆乘矣,左右执兵而下矣!」曰:``听誓也。」``战乎?」曰:``未可知也。」``乘而左右皆下矣!」曰:``战祷也。」伯州犁以公卒告王。苗贲皇在晋侯之侧,亦以王卒告。皆曰:``国士在,且厚,不可当也。」苗贲皇言于晋侯曰:``楚之良,在其中军王族而已。请分良以击其左右,而三军萃于王卒,必大败之。」公筮之,史曰:``吉。其卦遇《复》三,曰:『南国戚,射其元王中厥目。』国戚王伤,不败何待?」公从之。有淖于前,乃皆左右相违于淖。步毅御晋厉公,栾金咸为右。彭名御楚共王,潘党为右。石首御郑成公,唐苟为右。栾、范以其族夹公行,陷于淖。栾书将载晋侯,金咸曰:``书退!国有大任,焉得专之?且侵官,冒也;失官,慢也;离局,奸也。有三不罪焉,可犯也。」乃掀公以出于淖。

癸巳,潘□之党与养由基蹲甲而射之,彻七札焉。以示王,曰:``君有二臣如此,何忧于战?」王怒曰:``大辱国。诘朝,尔射,死艺。」吕錡梦射月,中之,退入于泥。占之,曰:``姬姓,日也。异姓,月也,必楚王也。射而中之,退入于泥,亦必死矣。」及战,射共王,中目。王召养由基,与之两矢,使射吕錡,中项,伏弢。以一矢覆命。

郤至三遇楚子之卒,见楚子,必下,免胄而趋风。楚子使工尹襄问之以弓,曰:``方事之殷也,有韎韦之跗注,君子也。识见不谷而趋,无乃伤乎?」郤至见客,免胄承命,曰:``君之外臣至,从寡君之戎事,以君之灵,间蒙甲胄,不敢拜命,敢告不宁君命之辱,为事之故,敢肃使者。」三肃使者而退。

晋韩厥从郑伯,其御杜溷罗曰:``速从之!其御屡顾,不在马,可及也。」韩厥曰:``不可以再辱国君。」乃止。郤至从郑伯,其右茀翰胡曰:``谍辂之,余从之乘而俘以下。」郤至曰:``伤国君有刑。」亦止。石首曰:``卫懿公唯不去其旗,是以败于荧。」乃旌于弢中。唐苟谓石首曰:``子在君侧,败者壹大。我不如子,子以君免,我请止。」乃死。

楚师薄于险,叔山冉谓养由基曰:``虽君有命,为国故,子必射!」乃射。再发,尽殪。叔山冉搏人以投,中车,折轼。晋师乃止。囚楚公子伐。

栾金咸见子重之旌,请曰:``楚人谓夫旌,子重之麾也。彼其子重也。日臣之使于楚也,子重问晋国之勇。臣对曰:『好以众整。』曰:『又何如?』臣对曰:『好以暇。』今两国治戎,行人不使,不可谓整。临事而食言,不可谓暇。请摄饮焉。」公许之。使行人执榼承饮,造于子重,曰:``寡君乏使,使金咸御持矛。是以不得犒从者,使某摄饮。」子重曰:``夫子尝与吾言于楚,必是故也,不亦识乎!」受而饮之。免使者而复鼓。

旦而战,见星未已。子反命军吏察夷伤,补卒乘,缮甲兵,展车马,鸡鸣而食,唯命是听。晋人患之。苗贲皇徇曰:``搜乘补卒,秣马利兵,修陈固列,蓐食申祷,明日复战。」乃逸楚囚。王闻之,召子反谋。谷阳竖献饮于子反,子反醉而不能见。王曰:``天败楚也夫!余不可以待。」乃宵遁。晋入楚军,三日谷。范文子立于戎马之前,曰:``君幼,诸臣不佞,何以及此?君其戒之!《周书》曰『唯命不于常』,有德之谓。」

楚师还,及瑕,王使谓子反曰:``先大夫之覆师徒者,君不在。子无以为过,不谷之罪也。」子反再拜稽首曰:``君赐臣死,死且不朽。臣之卒实奔,臣之罪也。」子重复谓子反曰:``初陨师徒者,而亦闻之矣!盍图之?」对曰:``虽微先大夫有之,大夫命侧,侧敢不义?侧亡君师,敢忘其死。」王使止之,弗及而卒。

战之日,齐国佐、高无咎至于师。卫侯出于卫,公出于坏隤。宣伯通于穆姜,欲去季、孟,而取其室。将行,穆姜送公,而使逐二子。公以晋难告,曰:``请反而听命。」姜怒,公子偃、公子鉏趋过,指之曰:``女不可,是皆君也。」公待于坏隤,申宫儆备,设守而后行,是以后。使孟献子守于公宫。

秋,会于沙随,谋伐郑也。宣伯使告郤犨曰:``鲁侯待于坏隤以待胜者。」郤犨将新军,且为公族大夫,以主东诸侯。取货于宣伯而诉公于晋侯,晋侯不见公。

曹人请于晋曰:``自我先君宣公即位,国人曰:『若之何忧犹未弭?』而又讨我寡君,以亡曹国社稷之镇公子,是大泯曹也。先君无乃有罪乎?若有罪,则君列诸会矣。君唯不遗德刑,以伯诸侯。岂独遗诸敝邑?取私布之。」

七月,公会尹武公及诸侯伐郑。将行,姜又命公如初。公又申守而行。诸侯之师次于郑西。我师次于督扬,不敢过郑。子叔声伯使叔孙豹请逆于晋师。为食于郑郊。师逆以至。声伯四日不食以待之,食使者而后食。

诸侯迁于制田。知武子佐下军,以诸侯之师侵陈,至于鸣鹿。遂侵蔡。未反,诸侯迁于颖上。戊午,郑子罕宵军之,宋、齐、卫皆失军。

曹人复请于晋,晋侯谓子臧:``反,吾归而君。」子臧反,曹伯归。子臧尽致其邑与卿而不出。

宣伯使告郤犨曰:``鲁之有季、孟,犹晋之有栾、范也,政令于是乎成。今其谋曰:『晋政多门,不可从也。宁事齐、楚,有亡而已,蔑从晋矣。』若欲得志于鲁,请止行父而杀之,我毙蔑也而事晋,蔑有贰矣。鲁不贰,小国必睦。不然,归必叛矣。」

九月,晋人执季文子于苕丘。公还,待于郓。使子叔声伯请季孙于晋,郤犨曰:``苟去仲孙蔑而止季孙行父,吾与子国,亲于公室。」对曰:``侨如之情,子必闻之矣。若去蔑与行父,是大弃鲁国而罪寡君也。若犹不弃,而惠徼周公之福,使寡君得事晋君。则夫二人者,鲁国社稷之臣也。若朝亡之,鲁必夕亡。以鲁之密迩仇雠,亡而为仇,治之何及?」郤犨曰:``吾为子请邑。」对曰:``婴齐,鲁之常隶也,敢介大国以求厚焉!承寡君之命以请,若得所请,吾子之赐多矣。又何求?」范文子谓栾武子曰:``季孙于鲁,相二君矣。妾不衣帛,马不食粟,可不谓忠乎?信谗慝而弃忠良,若诸侯何?子叔婴齐奉君命无私,谋国家不贰,图其身不忘其君。若虚其请,是弃善人也。子其图之!」乃许鲁平,赦季孙。

冬十月,出叔孙侨如而盟之,侨如奔齐。

十二月,季孙及郤犨盟于扈。归,刺公子偃,召叔孙豹于齐而立之。

齐声孟子通侨如,使立于高、国之间。侨如曰:``不可以再罪。」奔卫,亦间于卿。

晋侯使郤至献楚捷于周,与单襄公语,骤称其伐。单子语诸大夫曰:``温季其亡乎!位于七人之下,而求掩其上。怨之所聚,乱之本也。多怨而阶乱,何以在位?《夏书》曰:『怨岂在明?不见是图。』将慎其细也。今而明之,其可乎?」

\hypertarget{header-n1763}{%
\subsubsection{成公十七年}\label{header-n1763}}

【经】十有七年春,卫北宫括帅师侵郑。夏,公会尹子、单子、晋侯、齐侯、宋公、卫侯、曹伯、邾人伐郑。六月乙酋,同盟于柯陵。秋,公至自会。齐高无咎出奔莒。九月辛丑,用郊。晋侯使荀罃来乞师。冬,公会单子、晋侯、宋公、卫侯、曹伯、齐人、邾人伐郑。十有一月,公至自伐郑。壬申,公孙婴卒于貍脤。十有二月丁巳朔,日有食之。邾子玃且卒。晋杀其大夫郤錡、郤犨、郤至。楚人灭舒庸。

【传】十七年春,王正月,郑子驷侵晋虚、滑。卫北宫括救晋,侵郑,至于高氏。

夏五月,郑大子髡顽、侯孺为质于楚,楚公子成、公子寅戍郑。公会尹武公、单襄公及诸侯伐郑,自戏童至于曲洧。

晋范文子反自鄢陵,使其祝宗祈死,曰:``君骄侈而克敌,是天益其疾也。难将作矣!爱我者惟祝我,使我速死,无及于难,范氏之福也。」六月戊辰,士燮卒。

乙酉同盟于柯陵,寻戚之盟也。

楚子重救郑,师于首止。诸侯还。

齐庆克通于声孟子,与妇人蒙衣乘辇而入于闳。鲍牵见之,以告国武子,武子召庆克而谓之。庆克久不出,而告夫人曰:``国子谪我!」夫人怒。国子相灵公以会,高、鲍处守。及还,将至,闭门而索客。孟子诉之曰:``高、鲍将不纳君,而立公子角。国子知之。」秋七月壬寅,刖鲍牵而逐高无咎。无咎奔莒,高弱以卢叛。齐人来召鲍国而立之。

初,鲍国去鲍氏而来为施孝叔臣。施氏卜宰,匡句须吉。施氏之宰,有百室之邑。与匡句须邑,使为宰。以让鲍国,而致邑焉。施孝叔曰:``子实吉。」对曰:``能与忠良,吉孰大焉!」鲍国相施氏忠,故齐人取以为鲍氏后。仲尼曰:``鲍庄子之知不如葵,葵犹能卫其足。」

冬,诸侯伐郑。十月庚午,围郑。楚公子申救郑,师于汝上。十一月,诸侯还。

初,声伯梦涉洹,或与己琼瑰,食之,泣而为琼瑰,盈其怀。从而歌之曰:``济洹之水,赠我以琼瑰。归乎!归乎!琼瑰盈吾怀乎!」惧不敢占也。还自郑,壬申,至于狸脤而占之,曰:``余恐死,故不敢占也。今众繁而从余三年矣,无伤也。」言之,之莫而卒。

齐侯使崔杼为大夫,使庆克佐之,帅师围卢。国佐从诸侯围郑,以难请而归。遂如卢师,杀庆克,以谷叛。齐侯与之盟于徐关而复之。十二月,卢降。使国胜告难于晋,待命于清。

晋厉公侈,多外嬖。反自鄢陵,欲尽去群大夫,而立其左右。胥童以胥克之废也,怨郤氏,而嬖于厉公。郤錡夺夷阳五田,五亦嬖于厉公。郤犨与长鱼矫争田,执而梏之,与其父母妻子同一辕。既,矫亦嬖于厉公。栾书怨郤至,以其不从己而败楚师也,欲废之。使楚公子伐告公曰:``此战也,郤至实召寡君。以东师之未至也,与军帅之不具也,曰:『此必败!吾因奉孙周以事君。』」公告栾书,书曰:``其有焉!不然,岂其死之不恤,而受敌使乎?君盍尝使诸周而察之?」郤至聘于周,栾书使孙周见之。公使觇之,信。遂怨郤至。

厉公田,与妇人先杀而饮酒,后使大夫杀。郤至奉豕,寺人孟张夺之,郤至射而杀之。公曰:``季子欺余。」

厉公将作难,胥童曰:``必先三郤,族大多怨。去大族不逼,敌多怨有庸。」公曰:``然。」郤氏闻之,郤錡欲攻公,曰:``虽死,君必危。」郤至曰:``人所以立,信、知、勇也。信不叛君,知不害民,勇不作乱。失兹三者,其谁与我?死而多怨,将安用之?君实有臣而杀之,其谓君何?我之有罪,吾死后矣!若杀不辜,将失其民,欲安,得乎?待命而已!受君之禄是以聚党。有党而争命,罪孰大焉!」

壬午,胥童、夷羊五帅甲八百,将攻郤氏。长鱼矫请无用众,公使清沸魋助之,抽戈结衽,而伪讼者。三郤将谋于榭。矫以戈杀驹伯、苦成叔于其位。温季曰:``逃威也!」遂趋。矫及诸其车,以戈杀之,皆尸诸朝。

胥童以甲劫栾书、中行偃于朝。矫曰:``不杀二子,忧必及君。」公曰:``一朝而尸三卿,余不忍益也。」对曰:``人将忍君。臣闻乱在外为奸,在内为轨。御奸以德,御轨以刑。不施而杀,不可谓德。臣逼而不讨,不可谓刑。德刑不立,奸轨并至。臣请行。」遂出奔狄。公使辞于二子,曰:``寡人有讨于郤氏,既伏其辜矣。大夫无辱,其复职位。」皆再拜稽首曰:``君讨有罪,而免臣于死,君之惠也。二臣虽死,敢忘君德。」乃皆归。公使胥童为卿。

公游于匠丽氏,栾书、中行偃遂执公焉。召士□,士□辞。召韩厥,韩厥辞,曰:``昔吾畜于赵氏,孟姬之谗,吾能违兵。古人有言曰:『杀老牛莫之敢尸。』而况君乎?二三子不能事君,焉用厥也!」

舒庸人以楚师之败也,道吴人围巢,伐驾,围厘、虺,遂恃吴而不设备。楚公子櫜师袭舒庸,灭之。

闰月乙卯晦,栾书、中行偃杀胥童。民不与郤氏,胥童道君为乱,故皆书曰:``晋杀其大夫。」

\hypertarget{header-n1785}{%
\subsubsection{成公十八年}\label{header-n1785}}

【经】十有八年春王正月,晋杀其大夫胥童。庚申,晋弑其君州蒲。齐杀其大夫国佐。公如晋。夏,楚子、郑伯伐宋。宋鱼石复入于彭城。公至自晋。晋侯使士□来聘。秋,杞伯来朝。八月,邾子来朝,筑鹿囿。己丑,公薨于路寝。冬,楚人、郑人侵宋。晋侯使士鲂来乞师。十有二月,仲孙蔑会晋侯、宋公、卫侯、邾子、齐崔杼同盟于虚朾。丁未,葬我君成公。

【传】十八年春,王正月庚申,晋栾书、中行偃使程滑弑厉公,葬之于翼东门之外,以车一乘。使荀罃、士鲂逆周子于京师而立之,生十四年矣。大夫逆于清原,周子曰:``孤始愿不及此。虽及此,岂非天乎!抑人之求君,使出命也,立而不从,将安用君?二三子用我今日,否亦今日,共而从君,神之所福也。」对曰:``群臣之愿也,敢不唯命是听。」庚午,盟而入,馆于伯子同氏。辛巳,朝于武宫,逐不臣者七人。周子有兄而无慧,不能辨菽麦,故不可立。

齐为庆氏之难故,甲申晦,齐侯使士华免以戈杀国佐于内宫之朝。师逃于夫人之宫。书曰:``齐杀其大夫国佐。」弃命,专杀,以谷叛故也。使清人杀国胜。国弱来奔,王湫奔莱。庆封为大夫,庆佐为司寇。既,齐侯反国弱,使嗣国氏,礼也。

二月乙酉朔,晋侯悼公即位于朝。始命百官,施舍、己责,逮鳏寡,振废滞,匡乏困,救灾患,禁淫慝,薄赋敛,宥罪戾,节器用,时用民,欲无犯时。使魏相、士鲂、魏颉、赵武为卿。荀家、荀会、栾□、韩无忌为公族大夫,使训卿之子弟共俭孝弟。使士渥浊为大傅,使修范武子之法。右行辛为司空,使修士蒍之法。弁纠御戎,校正属焉,使训诸御知义。荀宾为右,司士属焉,使训勇力之士时使。卿无共御,立军尉以摄之。祁奚为中军尉,羊舌职佐之,魏绛为司马,张老为候奄。铎遏寇为上军尉,籍偃为之司马,使训卒乘亲以听命。程郑为乘马御,六驺属焉,使训群驺知礼。凡六官之长,皆民誉也。举不失职,官不易方,爵不逾德,师不陵正,旅不逼师,民无谤言,所以复霸也。

公如晋,朝嗣君也。

夏六月,郑伯侵宋,及曹门外。遂会楚子伐宋,取朝郏。楚子辛、郑皇辰侵城郜,取幽丘,同伐彭城,纳宋鱼石、向为人、鳞朱、向带、鱼府焉,以三百乘戍之而还。书曰``复入」,凡去其国,国逆而立之,曰``入」;复其位,曰``复归」;诸侯纳之,曰``归」。以恶曰复入。宋人患之。西鉏吾曰:``何也?若楚人与吾同恶,以德于我,吾固事之也,不敢贰矣。大国无厌,鄙我犹憾。不然,而收吾憎,使赞其政,以间吾衅,亦吾患也。今将崇诸侯之奸,而披其地,以塞夷庚。逞奸而携服,毒诸侯而惧吴、晋。吾庸多矣,非吾忧也。且事晋何为?晋必恤之。」

公至自晋。晋范宣子来聘,且拜朝也。君子谓:``晋于是乎有礼。」

秋,杞桓公来朝,劳公,且问晋故。公以晋君语之。杞伯于是骤朝于晋而请为昏。

七月,宋老佐、华喜围彭城,老佐卒焉。

八月,邾宣公来朝,即位而来见也。

筑鹿囿,书,不时也。

己丑,公薨于路寝,言道也。

冬十一月,楚子重救彭城,伐宋,宋华元如晋告急。韩献子为政,曰:``欲求得人,必先勤之,成霸安强,自宋始矣。」晋侯师于台谷以救宋,遇楚师于靡角之谷。楚师还。

晋士鲂来乞师。季文子问师数于臧武仲,对曰:``伐郑之役,知伯实来,下军之佐也。今彘季亦佐下军,如伐郑可也。事大国,无失班爵而加敬焉,礼也。」从之。

十二月,孟献子会于虚朾,谋救宋也。宋人辞诸侯而请师以围彭城。孟献子请于诸侯,而先归会葬。

丁未,葬我君成公,书,顺也。

\hypertarget{header-n1807}{%
\subsection{襄公}\label{header-n1807}}

\begin{center}\rule{0.5\linewidth}{\linethickness}\end{center}

\hypertarget{header-n1809}{%
\subsubsection{襄公元年 }\label{header-n1809}}

【经】元年春王正月,公即位。仲孙蔑会晋栾□、宋华元、卫宁殖、曹人、莒人、邾人、滕人、薛人围宋彭城。夏,晋韩厥帅师伐郑,仲孙蔑会齐崔杼、曹人、邾人、杞人次于鄫。秋,楚公子壬夫帅师侵宋。九月辛酉,天王崩。邾子来朝。冬,卫侯使公孙剽来聘。晋侯使荀罃来聘。

【传】元年春己亥,围宋彭城。非宋地,追书也。于是为宋讨鱼石,故称宋,且不登叛人也,谓之宋志。彭城降晋,晋人以宋五大夫在彭城者归,置诸瓠丘。齐人不会彭城,晋人以为讨。二月,齐大子光为质于晋。

夏五月,晋韩厥、荀偃帅诸侯之师伐郑,入其郛,败其徒兵于洧上。于是东诸侯之师次于鄫,以待晋师。晋师自郑以鄫之师侵楚焦夷及陈,晋侯、卫侯次于戚,以为之援。

秋,楚子辛救郑,侵宋吕、留。郑子然侵宋,取犬丘。

九月,邾子来朝,礼也。

冬,卫子叔、晋知武子来聘,礼也。凡诸侯即位,小国朝之,大国聘焉,以继好结信,谋事补阙,礼之大者也。

\hypertarget{header-n1818}{%
\subsubsection{襄公二年}\label{header-n1818}}

【经】二年春王正月,葬简王。郑师伐宋。夏五月庚寅,夫人姜氏薨。六月庚辰,郑伯仑卒。晋师、宋师、卫宁殖侵郑。秋七月,仲孙蔑会晋荀罃、宋华元、卫孙林父、曹人、邾人于戚。己丑,葬我小君齐姜。叔孙豹如宋。冬,仲孙蔑会晋荀罃、齐崔杼、宋华元、卫孙林父、曹人、邾人、滕人、薛人、小邾人于戚,遂城虎牢。楚杀其大夫公子申。

【传】二年春,郑师侵宋,楚令也。

齐侯伐莱,莱人使正舆子赂夙沙卫以索马牛,皆百匹,齐师乃还。君子是以知齐灵公之为``灵」也。

夏,齐姜薨。初,穆姜使择美檟,以自为榇与颂琴。季文子取以葬。君子曰:``非礼也。礼无所逆,妇,养姑者也,亏姑以成妇,逆莫大焉。《诗》曰:『其惟哲人,告之话言,顺德之行。』季孙于是为不哲矣。且姜氏,君之妣也。《诗》曰:『为酒为醴,烝畀祖妣,以洽百礼,降福孔偕。』」

齐侯使诸姜宗妇来送葬。召莱子,莱子不会,故晏弱城东阳以逼之。

郑成公疾,子驷请息肩于晋。公曰:``楚君以郑故,亲集矢于其目,非异人任,寡人也。若背之,是弃力与言,其谁昵我?免寡人,唯二三子!」

秋七月庚辰,郑伯仑卒。于是子罕当国,子驷为政,子国为司马。晋师侵郑,诸大夫欲从晋。子驷曰:``官命未改。」
会于戚,谋郑故也。孟献子曰:``请城虎牢以逼郑。」知武子曰:``善。鄫之会,吾子闻崔子之言,今不来矣。滕、薛、小邾之不至,皆齐故也。寡君之忧不唯郑。罃将复于寡君,而请于齐。得请而告,吾子之功也。若不得请,事将在齐。君子之请,诸侯之福也,岂唯寡君赖之。」

穆叔聘于宋,通嗣君也。

冬,复会于戚,齐崔武子及滕、薛、小邾之大夫皆会,知武子之言故也。遂城虎牢,郑人乃成。

楚公子申为右司马,多受小国之赂,以逼子重、子辛,楚人杀之。故书曰:``楚杀其大夫公子申。」

\hypertarget{header-n1831}{%
\subsubsection{襄公三年}\label{header-n1831}}

【经】三年春,楚公子婴齐帅师伐吴。公如晋。夏四月壬戌,公及晋侯盟于长樗。公至自晋。六月,公会单子、晋侯、宋公、卫侯、郑伯、莒子、邾子、齐世子光。己未,同盟于鸡泽。陈侯使袁侨如会。戊寅,叔孙豹及诸侯之大夫及陈袁侨盟。秋,公至自会。冬,晋荀罃帅师伐许。

【传】三年春,楚子重伐吴,为简之师,克鸠兹,至于衡山。使邓廖帅组甲三百、被练三千以侵吴。吴人要而击之,获邓廖。其能免者,组甲八十、被练三百而已。子重归,既饮至,三日,吴人伐楚,取驾。驾,良邑也。邓廖,亦楚之良也。君子谓:``子重于是役也,所获不如所亡。」楚人以是咎子重。子重病之,遂遇心病而卒。

公如晋,始朝也。夏,盟于长樗。孟献子相,公稽首。知武子曰:``天子在,而君辱稽首,寡君惧矣。」孟献子曰:``以敝邑介在东表,密迩仇雠,寡君将君是望,敢不稽首?」

晋为郑服故,且欲修吴好,将合诸侯。使士□告于齐曰:``寡君使□,以岁之不易,不虞之不戒,寡君愿与一二兄弟相见,以谋不协,请君临之,使□乞盟。」齐侯欲勿许,而难为不协,乃盟于耏外。

祁奚请老,晋侯问嗣焉。称解狐,其仇也,将立之而卒。又问焉,对曰:``午也可。」于是羊舌职死矣,晋侯曰:``孰可以代之?」对曰:``赤也可。」于是使祁午为中军尉,羊舌赤佐之。君子谓:``祁奚于是能举善矣。称其仇,不为谄。立其子,不为比。举其偏,不为党。《商书》曰:『无偏无党,王道荡荡。』其祁奚之谓矣!解狐得举,祁午得位,伯华得官,建一官而三物成,能举善也夫!唯善,故能举其类。《诗》云:『惟其有之,是以似之。』祁奚有焉。」

六月,公会单顷公及诸侯。己未,同盟于鸡泽。

晋侯使荀会逆吴子于淮上,吴子不至。

楚子辛为令尹,侵欲于小国。陈成公使袁侨如会求成,晋侯使和组父告于诸侯。秋,叔孙豹及诸侯之大夫及陈袁侨盟,陈请服也。

晋侯之弟扬干乱行于曲梁,魏绛戮其仆。晋侯怒,谓羊舌赤曰:``合诸侯以为荣也,扬干为戮,何辱如之?必杀魏绛,无失也!」对曰:``绛无贰志,事君不辟难,有罪不逃刑,其将来辞,何辱命焉?」言终,魏绛至,授仆人书,将伏剑。士鲂、张老止之。公读其书曰:``日君乏使,使臣斯司马。臣闻师众以顺为武,军事有死无犯为敬。君合诸侯,臣敢不敬?君师不武,执事不敬,罪莫大焉。臣惧其死,以及扬干,无所逃罪。不能致训,至于用金戊。臣之罪重,敢有不从,以怒君心,请归死于司寇。」公跣而出,曰:``寡人之言,亲爱也。吾子之讨,军礼也。寡人有弟,弗能教训,使干大命,寡人之过也。子无重寡人之过,敢以为请。」

晋侯以魏绛为能以刑佐民矣,反役,与之礼食,使佐新军。张老为中军司马,士富为候奄。

楚司马公子何忌侵陈,陈叛故也。

许灵公事楚,不会于鸡泽。冬,晋知武子帅师伐许。

\hypertarget{header-n1846}{%
\subsubsection{襄公四年}\label{header-n1846}}

【经】四年春王三月己酉,陈侯午卒。夏,叔孙豹如晋。秋七月戊子,夫人姒氏薨。葬陈成公。八月辛亥,葬我小君定姒。冬,公如晋。陈人围顿。

【传】四年春,楚师为陈叛故,犹在繁阳。韩献子患之,言于朝曰:``文王帅殷之叛国以事纣,唯知时也。今我易之,难哉!」

三月,陈成公卒。楚人将伐陈,闻丧乃止。陈人不听命。臧武仲闻之,曰:``陈不服于楚,必亡。大国行礼焉而不服,在大犹有咎,而况小乎?」夏,楚彭名侵陈,陈无礼故也。

穆叔如晋,报知武子之聘也,晋侯享之。金奏《肆夏》之三,不拜。工歌《文王》之三,又不拜。歌《鹿鸣》之三,三拜。韩献子使行人子员问之,曰:``子以君命,辱于敝邑。先君之礼,藉之以乐,以辱吾子。吾子舍其大,而重拜其细,敢问何礼也?」对曰:``三《夏》,天子所以享元侯也,使臣弗敢与闻。《文王》,两君相见之乐也,使臣不敢及。《鹿鸣》,君所以嘉寡君也,敢不拜嘉。?《四牡》,君所以劳使臣也,敢不重拜?《皇皇者华》,君教使臣曰:『必咨于周。』臣闻之:『访问于善为咨,咨亲为询,咨礼为度,咨事为诹,咨难为谋。』臣获五善,敢不重拜?」

秋,定姒薨。不殡于庙,无榇,不虞。匠庆谓季文子曰:``子为正卿,而小君之丧不成,不终君也。君长,谁受其咎?」

初,季孙为己树六檟于蒲圃东门之外。匠庆请木,季孙曰:``略。」匠庆用蒲圃之檟,季孙不御。君子曰:``《志》所谓『多行无礼,必自及也』,其是之谓乎!」

冬,公如晋听政,晋侯享公。公请属鄫,晋侯不许。孟献子曰:``以寡君之密迩于仇雠,而愿固事君,无失官命。鄫无赋于司马,为执事朝夕之命敝邑,敝邑褊小,阙而为罪,寡君是以愿借助焉!」晋侯许之。

楚人使顿间陈而侵伐之,故陈人围顿。

无终子嘉父使孟乐如晋,因魏庄子纳虎豹之皮,以请和诸戎。晋侯曰:``戎狄无亲而贪,不如伐之。」魏绛曰:``诸侯新服,陈新来和,将观于我,我德则睦,否则携贰。劳师于戎,而楚伐陈,必弗能救,是弃陈也,诸华必叛。戎,禽兽也,获戎失华,无乃不可乎?《夏训》有之曰:『有穷后羿。』」公曰:``后羿何如?」对曰:``昔有夏之方衰也,后羿自鉏迁于穷石,因夏民以代夏政。恃其射也,不修民事而淫于原兽。弃武罗、伯困、熊髡、龙圉而用寒浞。寒浞,伯明氏之谗子弟也。伯明后寒弃之,夷羿收之,信而使之,以为己相。浞行媚于内而施赂于外,愚弄其民而虞羿于田,树之诈慝以取其国家,外内咸服。羿犹不悛,将归自田,家众杀而亨之,以食其子。其子不忍食诸,死于穷门。靡奔有鬲氏。浞因羿室,生浇及豷,恃其谗慝诈伪而不德于民。使浇用师,灭斟灌及斟寻氏。处浇于过,处豷于戈。靡自有鬲氏,收二国之烬,以灭浞而立少康。少康灭浇于过,后杼灭豷于戈。有穷由是遂亡,失人故也。昔周辛甲之为大史也,命百官,官箴王阙。于《虞人之箴》曰:『芒芒禹迹,尽为九州,经启九道。民有寝庙,兽有茂草,各有攸处,德用不扰。在帝夷羿,冒于原兽,忘其国恤,而思其麀牡。武不可重,用不恢于夏家。兽臣司原,敢告仆夫。』《虞箴》如是,可不惩乎?」于是晋侯好田,故魏绛及之。

公曰:``然则莫如和戎乎?」对曰:``和戎有五利焉:戎狄荐居,贵货易土,土可贾焉,一也。边鄙不耸,民狎其野,穑人成功,二也。戎狄事晋,四邻振动,诸侯威怀,三也。以德绥戎,师徒不勤,甲兵不顿,四也。鉴于后羿,而用德度,远至迩安,五也。君其图之!」公说,使魏绛盟诸戎,修民事,田以时。

冬十月,邾人、莒人伐鄫。臧纥救鄫,侵邾,败于狐骀。国人逆丧者皆髽。鲁于是乎始髽,国人诵之曰:``臧之狐裘,败我于狐骀。我君小子,朱儒是使。朱儒!朱儒!使我败于邾。」

\hypertarget{header-n1860}{%
\subsubsection{襄公五年}\label{header-n1860}}

【经】五年春,公至自晋。夏,郑伯使公子发来聘。叔孙豹、鄫世子巫如晋。仲孙蔑、卫孙林父子会吴于善道。秋,大雩。楚杀其大夫公子壬夫。公会晋侯、宋公、陈侯、卫侯、郑伯、曹伯、莒子、邾子、滕子、薛伯、齐世子光、吴人、鄫人于戚。公至自会。冬,戍陈。楚公子贞帅师伐陈。公会晋侯、宋公、卫侯、郑伯、曹伯、齐世子光救陈。十有二月,公至自救陈。辛未,季孙行父卒。

【传】五年春,公至自晋。

王使王叔陈生愬戎于晋,晋人执之。士鲂如京师,言王叔之贰于戎也。

夏,郑子国来聘,通嗣君也。

穆叔觌鄫大子于晋,以成属鄫。书曰:``叔孙豹、鄫大子巫如晋。」言比诸鲁大夫也。

吴子使寿越如晋,辞不会于鸡泽之故,且请听诸侯之好。晋人将为之合诸侯,使鲁、卫先会吴,且告会期。故孟献子、孙文子会吴于善道。

秋,大雩,旱也。

楚人讨陈叛故,曰:``由令尹子辛实侵欲焉。」乃杀之。书曰:``楚杀其大夫公子壬夫。」贪也。君子谓:``楚共王于是不刑。《诗》曰:『周道挺挺,我心扃扃,讲事不令,集人来定。』己则无信,而杀人以逞,不亦难乎?《夏书》曰:『成允成功。』」

九月丙午,盟于戚,会吴,且命戍陈也。穆叔以属鄫为不利,使鄫大夫听命于会。

楚子囊为令尹。范宣子曰:``我丧陈矣!楚人讨贰而立子囊,必改行而疾讨陈。陈近于楚,民朝夕急,能无往乎?有陈,非吾事也,无之而后可。」

冬,诸侯戍陈。子囊伐陈。十一月甲午,会于城棣以救之。

季文子卒。大夫入敛,公在位。宰庀家器为葬备,无衣帛之妾,无食粟之马,无藏金玉,无重器备。君子是以知季文子之忠于公室也。相三君矣,而无私积,可不谓忠乎?

\hypertarget{header-n1875}{%
\subsubsection{襄公六年}\label{header-n1875}}

【经】六年春王三月,壬午,杞伯姑容卒。夏,宋华弱来奔。秋,杞葬桓公。滕子来朝。莒人灭鄫。冬,叔孙豹如邾,季孙宿如晋。十有二月,齐侯灭莱。

【传】六年春,杞桓公卒,始赴以名,同盟故也。

宋华弱与乐辔少相狎,长相优,又相谤也。子荡怒,以弓梏华弱于朝。平公见之,曰:``司武而梏于朝,难以胜矣!」遂逐之。夏,宋华弱来奔。司城子罕曰:``同罪异罚,非刑也。专戮于朝,罪孰大焉!」亦逐子荡。子荡射子罕之门,曰:``几日而不我从!」子罕善之如初。

秋,滕成公来朝,始朝公也。

莒人灭鄫,鄫恃赂也。

冬,穆叔如邾,聘,且修平。

晋人以鄫故来讨,曰:``何故亡鄫?」季武子如晋见,且听命。

十一月,齐侯灭莱,莱恃谋也。于郑子国之来聘也,四月,晏弱城东阳,而遂围莱。甲寅,堙之环城,傅于堞。及杞桓公卒之月,乙未,王湫帅师及正舆子、棠人军齐师,齐师大败之。丁未,入莱。莱共公浮柔奔棠。正舆子、王湫奔莒,莒人杀之。四月,陈无宇献莱宗器于襄宫。晏弱围棠,十一月丙辰,而灭之。迁莱于郳。高厚、崔杼定其田。

\hypertarget{header-n1886}{%
\subsubsection{襄公七年}\label{header-n1886}}

【经】七年春,郯子来朝。夏四月,三卜郊,不从,乃免牲。小邾子来朝。城费。秋,季孙宿如卫。八月,螽。冬十月,卫侯使孙林父来聘。壬戌,及孙林父盟。楚公子贞帅师围陈。十有二月,公会晋侯、宋公、陈侯、卫侯、曹伯、莒子、邾子于鄬。郑伯髡顽如会,未见诸侯,丙戌,卒于鄵。陈侯逃归。

【传】七年春,郯子来朝,始朝公也。

夏四月,三卜郊,不从,乃免牲。孟献子曰:``吾乃今而后知有卜筮。夫郊,祀后稷以祈农事也。是故启蛰而郊,郊而后耕。今既耕而卜郊,宜其不从也。」

南遗为费宰。叔仲昭伯为隧正,欲善季氏而求媚于南遗,谓遗:``请城费,吾多与而役。」故季氏城费。

小邾穆公来朝,亦始朝公也。

秋,季武子如卫,报子叔之聘,且辞缓报,非贰也。

冬十月,晋韩献子告老。公族穆子有废疾,将立之。辞曰:``《诗》曰:『岂不夙夜,谓行多露。』又曰:『弗躬弗亲,庶民弗信。』无忌不才,让,其可乎?请立起也!与田苏游,而曰好仁。《诗》曰:『靖共尔位,好是正直。神之听之,介尔景福。』恤民为德,正直为正,正曲为直,参和为仁。如是,则神听之,介福降之。立之,不亦可乎?」庚戌,使宣子朝,遂老。晋侯谓韩无忌仁,使掌公族大夫。

卫孙文子来聘,且拜武子之言,而寻孙桓子之盟。公登亦登。叔孙穆子相,趋进曰:``诸侯之会,寡君未尝后卫君。今吾子不后寡君,寡君未知所过。吾子其少安!」孙子无辞,亦无悛容。

穆叔曰:``孙子必亡。为臣而君,过而不悛,亡之本也。《诗》曰:『退食自公,委蛇委蛇。』谓从者也。衡而委蛇必折。」

楚子囊围陈,会于鄬以救之。

郑僖公之为大子也,于成之十六年,与子罕适晋,不礼焉。又与子丰适楚,亦不礼焉。及其元年,朝于晋。子丰欲愬诸晋而废之,子罕止之。及将会于鄬,子驷相,又不礼焉。侍者谏,不听,又谏,杀之。及鄵,子驷使贼夜弑僖公,而以疟疾赴于诸侯。简公生五年,奉而立之。

陈人患楚。庆虎、庆寅谓楚人曰:``吾使公子黄往而执之。」楚人从之。二庆使告陈侯于会,曰:``楚人执公子黄矣!君若不来,群臣不忍社稷宗庙,惧有二图。」陈侯逃归。

\hypertarget{header-n1901}{%
\subsubsection{襄公八年}\label{header-n1901}}

【经】八年春王正月,公如晋。夏,葬郑僖公。郑人侵蔡,获蔡公子燮。季孙宿会晋侯、郑伯、齐人、宋人、卫人、邾人于邢丘。公至自晋。莒人伐我东鄙。秋九月,大雩。冬,楚公子贞帅师伐郑。晋侯使士□来聘。

【传】八年春,公如晋,朝,且听朝聘之数。

郑群公子以僖公之死也,谋子驷。子驷先之。夏四月庚辰,辟杀子狐、子熙、子侯、子丁。孙击、孙恶出奔卫。

庚寅,郑子国、子耳侵蔡,获蔡司马公子燮。郑人皆喜,唯子产不顺,曰:``小国无文德,而有武功,祸莫大焉。楚人来讨,能勿从乎?从之,晋师必至。晋、楚伐郑,自今郑国不四五年,弗得宁矣。」子国怒之曰:``尔何知?国有大命,而有正卿。童子言焉,将为戮矣。」

五月甲辰,会于邢丘,以命朝聘之数,使诸侯之大夫听命。季孙宿、齐高厚、宋向戌、卫宁殖、邾大夫会之。郑伯献捷于会,故亲听命。大夫不书,尊晋侯也。

莒人伐我东鄙,以疆鄫田。

秋九月,大雩,旱也。

冬,楚子囊伐郑,讨其侵蔡也。

子驷、子国、子耳欲从楚,子孔、子蟜、子展欲待晋。子驷曰:``《周诗》有之曰:『俟河之清,人寿几何?兆云询多,职竞作罗。』谋之多族,民之多违,事滋无成。民急矣,姑从楚以纾吾民。晋师至,吾又从之。敬共币帛,以待来者,小国之道也。牺牲玉帛,待于二竞,以待强者而庇民焉。寇不为害,民不罢病,不亦可乎?」子展曰:``小所以事大,信也。小国无信,兵乱日至,亡无日矣。五会之信,今将背之,虽楚救我,将安用之?亲我无成,鄙我是欲,不可从也。不如待晋。晋君方明,四军无阙,八卿和睦,必不弃郑。楚师辽远,粮食将尽,必将速归,何患焉?舍之闻之:『杖莫如信。』完守以老楚,杖信以待晋,不亦可乎?」子驷曰:``《诗》云:『谋夫孔多,是用不集。发言盈庭,谁敢执其咎?如匪行迈谋,是用不得于道。』请从楚,□非也受其咎。」乃及楚平。

使王子伯骈告于晋,曰:``君命敝邑:『修而车赋,儆而师徒,以讨乱略。』蔡人不从,敝邑之人,不敢宁处,悉索敝赋,以讨于蔡,获司马燮,献于邢丘。今楚来讨曰:『女何故称兵于蔡?』焚我郊保,冯陵我城郭。敝邑之众,夫妇男女,不皇启处,以相救也。翦焉倾覆,无所控告。民死亡者,非其父兄,即其子弟,夫人愁痛,不知所庇。民知穷困,而受盟于楚,狐也与其二三臣不能禁止。不敢不告。」知武子使行人子员对之曰:``君有楚命,亦不使一介行李告于寡君,而即安于楚。君之所欲也,谁敢违君?寡君将帅诸侯以见于城下,唯君图之!」

晋范宣子来聘,且拜公之辱,告将用师于郑。公享之,宣子赋《摽有梅》。季武子曰:``谁敢哉!今譬于草木,寡君在君,君之臭味也。欢以承命,何时之有?」武子赋《角弓》。宾将出,武子赋《彤弓》。宣子曰:``城濮之役,我先君文公献功于衡雍,受彤弓于襄王,以为子孙藏。□也,先君守官之嗣也,敢不承命?」君子以为知礼。

\hypertarget{header-n1915}{%
\subsubsection{襄公九年}\label{header-n1915}}

【经】九年春,宋灾。夏,季孙宿如晋。五月辛酉,夫人姜氏薨。秋八月癸未,葬我小君穆姜。冬,公会晋侯、宋公、卫侯、曹伯、莒子、邾子、滕子、薛伯、杞伯,小邾子、齐世子光伐郑。十有二月己亥,同盟于戏。楚子伐郑。

【传】九年春,宋灾。乐喜为司城以为政。使伯氏司里,火所未至,彻小屋,涂大屋;陈畚挶具绠缶,备水器;量轻重,蓄水潦,积土涂;巡丈城,缮守备,表火道。使华臣具正徒,令隧正纳郊保,奔火所。使华阅讨右官,官庀其司。向戌讨左,亦如之。使乐遄庀刑器,亦如之。使皇郧命校正出马,工正出车,备甲兵,庀武守使西鉏吾庀府守,令司宫、巷伯儆宫。二师令四乡正敬享,祝宗用马于四墉,祀盘庚于西门之外。

晋侯问于士弱曰:``吾闻之,宋灾,于是乎知有天道。何故?」对曰:``古之火正,或食于心,或食于咮,以出内火。是故咮为鹑火,心为大火。陶唐氏之火正阏伯居商丘,祀大火,而火纪时焉。相土因之,故商主大火。商人阅其祸败之衅,必始于火,是以日知其有天道也。」公曰:``可必乎?」对曰:``在道。国乱无象,不可知也。」

夏,季武子如晋,报宣子之聘也。

穆姜薨于东宫。始往而筮之,遇《艮》之八三。史曰:``是谓《艮》之《随》三。《随》其出也。君必速也。」姜曰:``亡。是于《周易》曰:『《随》,元亨利贞,无咎。』元,体之长也;享,嘉之会也;利,义之和也;贞,事之干也。体仁足以长人,嘉德足以合礼,利物足以和义,贞固足以干事,然,故不可诬也,是以虽《随》无咎。今我妇人而与于乱。固在下位而有不仁,不可谓元。不靖国家,不可谓亨。作而害身,不可谓利。弃位而姣,不可谓贞。有四德者,《随》而无咎。我皆无之,岂《随》也哉?我则取恶,能无咎乎?必死于此,弗得出矣。」

秦景公使士雅乞师于楚,将以伐晋,楚子许之。子囊曰:``不可。当今吾不能与晋争。晋君类能而使之,举不失选,官不易方。其卿让于善,其大夫不失守,其士竞于教,其庶人力于农穑。商工皂隶,不知迁业。韩厥老矣,知罃禀焉以为政。范□少于中行偃而上之,使佐中军。韩起少于栾□,而栾□、士鲂上之,使佐上军。魏绛多功,以赵武为贤而为之佐。君明臣忠,上让下竞。当是时也,晋不可敌,事之而后可。君其图之!」王曰:``吾既许之矣。虽不及晋,必将出师。」秋,楚子师于武城以为秦援。秦人侵晋,晋饥,弗能报也。

冬十月,诸侯伐郑。庚午,季武子、齐崔杼、宋皇郧从荀罃、士□门于鄟门。卫北宫括、曹人、邾人从荀偃、韩起门于师之梁。滕人、薛人从栾□、士鲂门于北门。杞人、郳人从赵武、魏绛斩行栗。甲戌,师于汜,令于诸侯曰:``修器备,盛□粮,归老幼,居疾于虎牢,肆眚,围郑。」郑人恐,乃行成。中行献子曰:``遂围之,以待楚人之救也而与之战。不然,无成。」知武子曰:``许之盟而还师,以敝楚人。吾三分四军,与诸侯之锐以逆来者,于我未病,楚不能矣,犹愈于战。暴骨以逞,不可以争。大劳未艾。君子劳心,小人劳力,先王之制也」诸侯皆不欲战,乃许郑成。十一月己亥,同盟于戏,郑服也。

将盟,郑六卿公子□非、公子发、公子嘉、公孙辄、公孙虿、公孙舍之及其大夫、门子皆从郑伯。晋士庄子为载书,曰:``自今日既盟之后,郑国而不唯晋命是听,而或有异志者,有如此盟。」公子□非趋进曰:``天祸郑国,使介居二大国之间。大国不加德音而乱以要之,使其鬼神不获歆其禋祀,其民人不获享其土利,夫妇辛苦垫隘,无所底告。自今日既盟之后,郑国而不唯有礼与强可以庇民者是从,而敢有异志者,亦如之。」荀偃曰:``改载书。」公孙舍之曰:``昭大神,要言焉。若可改也,大国亦可叛也。」知武子谓献子曰:``我实不德,而要人以盟,岂礼也哉!非礼,何以主盟?姑盟而退,修德息师而来,终必获郑,何必今日?我之不德,民将弃我,岂唯郑?若能休和,远人将至,何恃于郑?」乃盟而还。

晋人不得志于郑,以诸侯复伐之。十二月癸亥,门其三门。闰月,戊寅,济于阴阪,侵郑。次于阴口而还。子孔曰:``晋师可击也,师老而劳,且有归志,必大克之。」子展曰:``不可。」

公送晋侯。晋侯以公晏于河上,问公年,季武子对曰:``会于沙随之岁,寡君以生。」晋侯曰:``十二年矣!是谓一终,一星终也。国君十五而生子。冠而生子,礼也,君可以冠矣!大夫盍为冠具?」武子对曰:``君冠,必以祼享之礼行之,以金石之乐节之,以先君之祧处之。今寡君在行,未可具也。请及兄弟之国而假备焉。」晋侯曰:``诺。」公还,及卫,冠于成公之庙,假钟磬焉,礼也。

楚子伐郑,子驷将及楚平。子孔、子蟜曰:``与大国盟,口血未干而背之,可乎?」子驷、子展曰:``吾盟固云:『唯强是从。』今楚师至,晋不我救,则楚强矣。盟誓之言,岂敢背之?且要盟无质,神弗临也,所临唯信。信者,言之瑞也,善之主也,是故临之。明神不蠲要盟,背之可也。」乃及楚平。公子罢戎入盟,同盟于中分。

楚庄夫人卒,王未能定郑而归。

晋侯归,谋所以息民。魏绛请施舍,输积聚以贷。自公以下,苟有积者,尽出之。国无滞积,亦无困人。公无禁利,亦无贪民。祈以币更,宾以特性,器用不作,车服从给。行之期年,国乃有节。三驾而楚不能与争。

\hypertarget{header-n1931}{%
\subsubsection{襄公十年}\label{header-n1931}}

【经】十年春,公会晋侯、宋公、卫侯、曹伯、莒子、邾子、滕子、薛伯、杞伯、小邾子、齐世子光会吴于柤。夏,五月甲午,遂灭逼阳。公至自会。楚公子贞、郑公孙辄帅师伐宋。晋师伐秦。秋,莒人伐我东鄙。公会晋侯、宋公、卫侯、曹伯、莒子、邾子、齐世子光、滕子、薛伯、杞伯、小邾子伐郑。冬,盗杀郑公子□非、公子发、公孙辄。戍郑虎牢。楚公子贞帅师救郑。公至自伐郑。

【传】十年春,会于柤,会吴子寿梦也。三月癸丑,齐高厚相大子光以先会诸侯于钟离,不敬。士庄子曰:``高子相大子以会诸侯,将社稷是卫,而皆不敬,弃社稷也,其将不免乎!」

夏四月戊午,会于柤。

晋荀偃、士□请伐逼阳,而封宋向戌焉。荀罃曰:``城小而固,胜之不武,弗胜为笑。」固请。丙寅,围之,弗克。孟氏之臣秦堇父辇重如役。逼阳人启门,诸侯之士门焉。县门发,郰人纥抉之以出门者。狄虒弥建大车之轮而蒙之以甲以为橹,左执之,右拔戟,以成一队。孟献子曰:``《诗》所谓『有力如虎』者也。」主人县布,堇父登之,及堞而绝之。队则又县之,苏而复上者三。主人辞焉乃退,带其断以徇于军三日。

诸侯之师久于逼阳,荀偃、士□请于荀罃曰:``水潦将降,惧不能归,请班师。」知伯怒,投之以机,出于其间,曰:``女成二事而后告余。余恐乱命,以不女违。女既勤君而兴诸侯,牵帅老夫以至于此,既无武守,而又欲易余罪,曰:『是实班师,不然克矣』。余赢老也,可重任乎?七日不克,必尔乎取之!」五月庚寅,荀偃、士□帅卒攻逼阳,亲受矢石。甲午,灭之。书曰``遂灭逼阳」,言自会也。以与向戌,向戌辞曰:``君若犹辱镇抚宋国,而以逼阳光启寡君,群臣安矣,其何贶如之?若专赐臣,是臣兴诸侯以自封也,其何罪大焉?敢以死请。」乃予宋公。

宋公享晋侯于楚丘,请以《桑林》。荀罃辞。荀偃、士□曰:``诸侯宋、鲁,于是观礼。鲁有禘乐,宾祭用之。宋以《桑林》享君,不亦可乎?」舞,师题以旌夏,晋侯惧而退入于房。去旌,卒享而还。及着雍,疾。卜,桑林见。荀偃、士□欲奔请祷焉。荀罃不可,曰:``我辞礼矣,彼则以之。犹有鬼神,于彼加之。」晋侯有间,以逼阳子归,献于武宫,谓之夷俘。逼阳妘姓也。使周内史选其族嗣,纳诸霍人,礼也。

师归,孟献子以秦堇父为右。生秦丕兹,事仲尼。

六月,楚子囊、郑子耳伐宋,师于訾毋。庚午,围宋,门于桐门。

晋荀罃伐秦,报其侵也。

卫侯救宋,师于襄牛。郑子展曰:``必伐卫,不然,是不与楚也。得罪于晋,又得罪于楚,国将若之何?」子驷曰:``国病矣!」子展曰:``得罪于二大国,必亡。病不犹愈于亡乎?」诸大夫皆以为然。故郑皇耳帅师侵卫,楚令也。孙文子卜追之,献兆于定姜。姜氏问繇。曰:``兆如山陵,有夫出征,而丧其雄。」姜氏曰:``征者丧雄,御寇之利也。大夫图之!」卫人追之,孙蒯获郑皇耳于犬丘。

秋七月,楚子囊、郑子耳伐我西鄙。还,围萧,八月丙寅,克之。九月,子耳侵宋北鄙。孟献子曰:``郑其有灾乎!师竞已甚。周犹不堪竞,况郑乎?有灾,其执政之三士乎!」

莒人间诸侯之有事也,故伐我东鄙。

诸侯伐郑。齐崔杼使大子光先至于师,故长于滕。己酉,师于牛首。

初,子驷与尉止有争,将御诸侯之师而黜其车。尉止获,又与之争。子驷抑尉止曰:``尔车,非礼也。」遂弗使献。初,子驷为田洫,司氏、堵氏、侯氏、子师氏皆丧田焉,故五族聚群不逞之人,因公子之徒以作乱。于是子驷当国,子国为司马,子耳为司空,子孔为司徒。冬十月戊辰,尉止、司臣、侯晋、堵女父、子师仆帅贼以入,晨攻执政于西宫之朝,杀子驷、子国、子耳,劫郑伯以如北宫。子孔知之,故不死。书曰``盗」,言无大夫焉。

子西闻盗,不儆而出,尸而追盗,盗入于北宫,乃归授甲。臣妾多逃,器用多丧。子产闻盗,为门者,庀群司,闭府库,慎闭藏,完守备,成列而后出,兵车十七乘,尸而攻盗于北宫。子蟜帅国人助之,杀尉止,子师仆,盗众尽死。侯晋奔晋。堵女父、司臣、尉翩、司齐奔宋。

子孔当国,为载书,以位序,听政辟。大夫、诸司、门子弗顺,将诛之。子产止之,请为之焚书。子孔不可,曰:``为书以定国,众怒而焚之,是众为政也,国不亦难乎?」子产曰:``众怒难犯,专欲难成,合二难以安国,危之道也。不如焚书以安众,子得所欲,众亦得安,不亦可乎?专欲无成,犯众兴祸,子必从之。」乃焚书于仓门之外,众而后定。

诸侯之师城虎牢而戍之。晋师城梧及制,士鲂、魏绛戍之。书曰``戍郑虎牢」,非郑地也,言将归焉。郑及晋平。楚子囊救郑。十一月,诸侯之师还郑而南,至于阳陵,楚师不退。知武子欲退,曰:``今我逃楚,楚必骄,骄则可与战矣。」栾□曰:``逃楚,晋之耻也。合诸侯以益耻,不如死!我将独进。」师遂进。己亥,与楚师夹颖而军。子矫曰:``诸侯既有成行,必不战矣。从之将退,不从亦退。退,楚必围我。犹将退也。不如从楚,亦以退之。」宵涉颖,与楚人盟。栾□欲伐郑师,荀罃不可,曰:``我实不能御楚,又不能庇郑,郑何罪?不如致怨焉而还。今伐其师,楚必救之,战而不克,为诸侯笑。克不可命,不如还也!」丁未,诸侯之师还,侵郑北鄙而归。楚人亦还。

王叔陈生与伯舆争政。王右伯舆,王叔陈生怒而出奔。及河,王复之,杀史狡以说焉。不入,遂处之。晋侯使士□平王室,王叔与伯舆讼焉。王叔之宰与伯舆之大夫瑕禽坐狱于王庭,士□听之。王叔之宰曰:``筚门闺窦之人而皆陵其上,其难为上矣!」瑕禽曰:``昔平王东迁,吾七姓从王,牲用备具。王赖之,而赐之騂旄之盟,曰:『世世无失职。』若筚门闺窦,其能来东底乎?且王何赖焉?今自王叔之相也,政以贿成,而刑放于宠。官之师旅,不胜其富,吾能无筚门闺窦乎?唯大国图之!下而无直,则何谓正矣?」范宣子曰:``天子所右,寡君亦右之。所在,亦左之。」使王叔氏与伯舆合要,王叔氏不能举其契。王叔奔晋。不书,不告也。单靖公为卿士,以相王室。

\hypertarget{header-n1952}{%
\subsubsection{襄公十一年}\label{header-n1952}}

【经】十有一年春王正月,作三军。夏四月,四卜郊,不从,乃不郊。郑公孙舍之帅师侵宋。公会晋侯、宋公、卫侯、曹伯、齐世子光、莒子、邾子、滕子、薛伯、杞伯、小邾子伐郑。秋七月己未,同盟于亳城北。公至自伐郑。楚子、郑伯伐宋。公会晋侯、宋公、卫侯、曹伯、齐世子光、莒子、邾子、滕子、薛伯、杞伯、小邾子伐郑,会于萧鱼。公至自会。楚执郑行人良霄。冬,秦人伐晋。

【传】十一年春,季武子将作三军,告叔孙穆子曰:``请为三军,各征其军。」穆子曰:``政将及子,子必不能。」武子固请之,穆子曰:``然则盟诸?」乃盟诸僖闳,诅诸五父之衢。

正月,作三军,三分公室而各有其一。三子各毁其乘。季氏使其乘之人,以其役邑入者,无征;不入者,倍征。孟氏使半为臣,若子若弟。叔孙氏使尽为臣,不然,不舍。

郑人患晋、楚之故,诸大夫曰:``不从晋,国几亡。楚弱于晋,晋不吾疾也。晋疾,楚将辟之。何为而使晋师致死于我,楚弗敢敌,而后可固与也。」子展曰:``与宋为恶,诸侯必至,吾从之盟。楚师至,吾又从之,则晋怒甚矣。晋能骤来,楚将不能,吾乃固与晋。」大夫说之,使疆埸之司恶于宋。宋向戌侵郑,大获。子展曰:``师而伐宋可矣。若我伐宋,诸侯之伐我必疾,吾乃听命焉,且告于楚。楚师至,吾又与之盟,而重赂晋师,乃免矣。」夏,郑子展侵宋。

四月,诸侯伐郑。己亥,齐大子光、宋向戌先至于郑,门于东门。其莫,晋荀罃至于西郊,东侵旧许。卫孙林父侵其北鄙。六月,诸侯会于北林,师于向,右还,次于琐,围郑。观兵于南门,西济于济隧。郑人惧,乃行成。

秋七月,同盟于亳。范宣子曰:``不慎,必失诸侯。诸侯道敝而无成,能无贰乎?」乃盟,载书曰:``凡我同盟,毋蕴年,毋壅利,毋保奸,毋留慝,救灾患,恤祸乱,同好恶,奖王室。或间兹命,司慎司盟,名山名川,群神群祀,先王先公,七姓十二国之祖,明神殛之,俾失其民,队命亡氏,踣其国家。」

楚子囊乞旅于秦,秦右大夫詹帅师从楚子,将以伐郑。郑伯逆之。丙子,伐宋。

九月,诸侯悉师以复伐郑。郑人使良霄、大宰石□如楚,告将服于晋,曰:``孤以社稷之故,不能怀君。君若能以玉帛绥晋,不然则武震以摄威之,孤之愿也。」楚人执之,书曰``行人」,言使人也。诸侯之师观兵于郑东门,郑人使王子伯骈行成。甲戌,晋赵武入盟郑伯。冬十月丁亥,郑子展出盟晋侯。十二月戊寅,会于萧鱼。庚辰,赦郑囚,皆礼而归之。纳斥候,禁侵掠。晋侯使叔肸告于诸侯。公使臧孙纥对曰:``凡我同盟,小国有罪,大国致讨,苟有以藉手,鲜不赦宥。寡君闻命矣。」郑人赂晋侯以师悝、师触、师蠲,广车、軘车淳十五乘,甲兵备,凡兵车百乘,歌钟二肆,及其鏄磐,女乐二八。

晋侯以乐之半赐魏绛,曰:``子教寡人和诸戎狄,以正诸华。八年之中,九合诸侯,如乐之和,无所不谐。请与子乐之。」辞曰:``夫和戎狄,国之福也;八年之中,九合诸侯,诸侯无慝,君之灵也,二三子之劳也,臣何力之有焉?抑臣愿君安其乐而思其终也!《诗》曰:『乐只君子,殿天子之邦。乐只君子,福禄攸同。便蕃左右,亦是帅从。』夫乐以安德,义以处之,礼以行之,信以守之,仁以厉之,而后可以殿邦国,同福禄,来远人,所谓乐也。《书》曰:『居安思危。』思则有备,有备无患,敢以此规。」公曰:``子之教,敢不承命。抑微子,寡人无以待戎,不能济河。夫赏,国之典也,藏在盟府,不可废也,子其受之!」魏绛于是乎始有金石之乐,礼也。

秦庶长鲍、庶长武帅师伐晋以救郑。鲍先入晋地,士鲂御之,少秦师而弗设备。壬午,武济自辅氏,与鲍交伐晋师。己丑,秦、晋战于栎,晋师败绩,易秦故也。

\hypertarget{header-n1965}{%
\subsubsection{襄公十二年}\label{header-n1965}}

【经】十有二年春王二月,莒人伐我东鄙,围台。季孙宿帅师救台,遂入郓。夏,晋侯使士鲂来聘。秋九月,吴子乘卒。冬,楚公子贞帅师侵宋。公如晋。

【传】十二年春,莒人伐我东鄙,围台。季武子救台,遂入郓,取其钟以为公盘。

夏,晋士鲂来聘,且拜师。

秋,吴子寿梦卒。临于周庙,礼也。凡诸侯之丧,异姓临于外,同姓于宗庙,同宗于祖庙,同族于祢庙。是故鲁为诸姬,临于周庙。为邢、凡、蒋、茅、胙、祭临于周公之庙。

冬,楚子囊、秦庶长无地伐宋,师于扬梁,以报晋之取郑也。

灵王求后于齐。齐侯问对于晏桓子,桓子对曰:``先王之礼辞有之,天子求后于诸侯,诸侯对曰:『夫妇所生若而人。妾妇之子若而人。』无女而有姊妹及姑姊妹,则曰:『先守某公之遗女若而人。』」齐侯许昏,王使阴里逆之。

公如晋,朝,且拜士鲂之辱,礼也。

秦嬴归于楚。楚司马子庚聘于秦,为夫人宁,礼也。

\hypertarget{header-n1976}{%
\subsubsection{襄公十三年}\label{header-n1976}}

【经】十有三年春,公至自晋。夏,取邿。秋九月庚辰,楚子审卒。冬,城防。

【传】十三年春,公至自晋,孟献子书劳于庙,礼也。

夏,邿乱,分为三。师救邿,遂取之。凡书``取」,言易也。用大师焉曰``灭」。弗地曰``入」。

荀罃、士鲂卒。晋侯搜于上以治兵,使士□将中军,辞曰:``伯游长。昔臣习于知伯,是以佐之,非能贤也。请从伯游。」荀偃将中军,士□佐之。使韩起将上军,辞以赵武。又使栾□,辞曰:``臣不如韩起。韩起愿上赵武,君其听之!」使赵武将上军,韩起佐之。栾□将下军,魏绛佐之。新军无帅,晋侯难其人,使其什吏,率其卒乘官属,以从于下军,礼也。晋国之民,是以大和,诸侯遂睦。君子曰:``让,礼之主也。范宣子让,其下皆让。栾□为汰,弗敢违也。晋国以平,数世赖之。刑善也夫!一人刑善,百姓休和,可不务乎?《书》曰:『一人有庆,兆民赖之,其宁惟永。』其是之谓乎?周之兴也,其《诗》曰:『仪刑文王,万邦作孚。』言刑善也。及其衰也,其《诗》曰:『大夫不均,我从事独贤。』言不让也。世之治也,君子尚能而让其下,小人农力以事其上,是以上下有礼,而谗慝黜远,由不争也,谓之懿德。及其乱也,君子称其功以加小人,小人伐其技以冯君子,是以上下无礼,乱虐并生,由争善也,谓之昏德。国家之敝,恒必由之。」

楚子疾,告大夫曰:``不谷不德,少主社稷,生十年而丧先君,未及习师保之教训,而应受多福。是以不德,而亡师于鄢,以辱社稷,为大夫忧,其弘多矣。若以大夫之灵,获保首领以殁于地,唯是春秋窀穸之事,所以从先君于祢庙者,请为『灵』若『厉』。大夫择焉!」莫对。及五命乃许。

秋,楚共王卒。子囊谋谥。大夫曰:``君有命矣。」子囊曰:``君命以共,若之何毁之?赫赫楚国,而君临之,抚有蛮夷,奄征南海,以属诸夏,而知其过,可不谓共乎?请谥之『共』。」大夫从之。

吴侵楚,养由基奔命,子庚以师继之。养叔曰:``吴乘我丧,谓我不能师也,必易我而不戒。子为三覆以待我,我请诱之。」子庚从之。战于庸浦,大败吴师,获公子党。君子以吴为不吊。《诗》曰:``不吊昊天,乱靡有定。」

冬,城防,书事,时也。于是将早城,臧武仲请俟毕农事,礼也。

郑良霄、大宰石□犹在楚。石□言于子囊曰:``先王卜征五年,而岁习其祥,祥习则行,不习则增修德而改卜。今楚实不竞,行人何罪?止郑一卿,以除其逼,使睦而疾楚,以固于晋,焉用之?使归而废其使,怨其君以疾其大夫,而相牵引也,不犹愈乎?」楚人归之。

\hypertarget{header-n1988}{%
\subsubsection{襄公十四年}\label{header-n1988}}

【经】十有四年春王正月,季孙宿、叔老会晋士□、齐人、宋人、卫人、郑公孙虿、曹人、莒人、邾人、滕人、薛人、杞人、小邾人会吴于向。二月乙朔,日有食之。夏四月,叔孙豹会晋荀偃、齐人、宋人、卫北宫括、郑公孙虿、曹人、莒人、邾人、滕人、薛人、杞人、小邾人伐秦。己未,卫侯出奔齐。莒人侵我东鄙。秋,楚公子贞帅师伐吴。冬,季孙宿会晋士□、宋华阅、卫孙林父、郑公孙虿、莒人、邾人于戚。

【传】十四年春,吴告败于晋。会于向,为吴谋楚故也。范宣子数吴之不德也,以退吴人。

执莒公子务娄,以其通楚使也。

将执戎子驹支。范宣子亲数诸朝,曰:``来!姜戎氏!昔秦人迫逐乃祖吾离于瓜州,乃祖吾离被苫盖,蒙荆棘,以来归我先君。我先君惠公有不腆之田,与女剖分而食之。今诸侯之事我寡君不知昔者,盖言语漏泄,则职女之由。诘朝之事,尔无与焉!与将执女!」对曰:``昔秦人负恃其众,贪于土地,逐我诸戎。惠公蠲其大德,谓我诸戎,是四岳之裔胄也,毋是翦弃。赐我南鄙之田,狐狸所居,豺狼所嗥。我诸戎除翦其荆棘,驱其狐狸豺狼,以为先君不侵不叛之臣,至于今不贰。昔文公与秦伐郑,秦人窃与郑盟而舍戍焉,于是乎有殽之师。晋御其上,戎亢其下,秦师不复,我诸戎实然。譬如捕鹿,晋人角之,诸戎掎之,与晋踣之,戎何以不免?自是以来,晋之百役,与我诸戎相继于时,以从执政,犹殽志也。岂敢离逖?今官之师旅,无乃实有所阙,以携诸侯,而罪我诸戎!我诸戎饮食衣服,不与华同,贽币不通,言语不达,何恶之能为?不与于会,亦无瞢焉!」赋《青蝇》而退。宣子辞焉,使即事于会,成恺悌也。于是,子叔齐子为季武子介以会,自是晋人轻鲁币,而益敬其使。

吴子诸樊既除丧,将立季札。季札辞曰:``曹宣公之卒也,诸侯与曹人不义曹君,将立子臧。子臧去之,遂弗为也,以成曹君。君子曰:『能守节。』君,义嗣也。谁敢奸君?有国,非吾节也。札虽不才,愿附于子臧,以无失节。」固立之。弃其室而耕。乃舍之。

夏,诸侯之大夫从晋侯伐秦,以报栎之役也。晋侯待于竟,使六卿帅诸侯之师以进。及泾,不济。叔向见叔孙穆子。穆子赋《匏有苦叶》。叔向退而具舟,鲁人、莒人先济。郑子蟜见卫北宫懿子曰:``与人而不固,取恶莫甚焉!若社稷何?」懿子说。二子见诸侯之师而劝之济,济泾而次。秦人毒泾上流,师人多死。郑司马子蟜帅郑师以进,师皆从之,至于棫林,不获成焉。荀偃令曰:``鸡鸣而驾,塞井夷灶,唯余马首是瞻!」栾□曰:``晋国之命,未是有也。余马首欲东。」乃归。下军从之。左史谓魏庄子曰:``不待中行伯乎?」庄子曰:``夫子命从帅。栾伯,吾帅也,吾将从之。从帅,所以待夫子也。」伯游曰:``吾令实过,悔之何及,多遗秦禽。」乃命大还。晋人谓之迁延之役。

栾金咸曰:``此役也,报栎之败也。役又无功,晋之耻也。吾有二位于戎路,敢不耻乎?」与士鞅驰秦师,死焉。士鞅反,栾□谓士□曰:``余弟不欲住,而子召之。余弟死,而子来,是而子杀余之弟也。弗逐,余亦将杀之。」士鞅奔秦。

于是,齐崔杼、宋华阅、仲江会伐秦,不书,惰也。向之会亦如之。卫北宫括不书于向,书于伐秦,摄也。

秦伯问于士鞅曰:``晋大夫其谁先亡?」对曰:``其栾氏乎!」秦伯曰:``以其汰乎?」对曰:``然。栾□汰虐已甚,犹可以免。其在盈乎!」秦伯曰:``何故?」对曰:``武子之德在民,如周人之思召公焉,爱其甘棠,况其子乎?栾□死,盈之善未能及人,武子所施没矣,而□之怨实章,将于是乎在。」秦伯以为知言,为之请于晋而复之。

卫献公戒孙文子、宁惠子食,皆服而朝。日旰不召,而射鸿于囿。二子从之,不释皮冠而与之言。二子怒。孙文子如戚,孙蒯入使。公饮之酒,使大师歌《巧言》之卒章。大师辞,师曹请为之。初,公有嬖妾,使师曹诲之琴,师曹鞭之。公怒,鞭师曹三百。故师曹欲歌之,以怒孙子以报公。公使歌之,遂诵之。

蒯惧,告文子。文子曰:``君忌我矣,弗先。必死。」并帑于戚而入,见蘧伯玉曰:``君之暴虐,子所知也。大惧社稷之倾覆,将若之何?」对曰:``君制其国,臣敢奸之?虽奸之,庸如愈乎?」遂行,从近关出。公使子蟜、子伯、子皮与孙子盟于丘宫,孙子皆杀之。四月己未,子展奔齐。公如鄄,使子行于孙子,孙子又杀之。公出奔齐,孙氏追之,败公徒于河泽。鄄人执之。

初,尹公佗学射于庚公差,庚公差学射于公孙丁。二子追公,公孙丁御公。子鱼曰:``射为背师,不射为戮,射为礼乎。」射两軥而还。尹公佗曰:``子为师,我则远矣。」乃反之。公孙丁授公辔而射之,贯臂。

子鲜从公,及竟,公使祝宗告亡,且告无罪。定姜曰:``无神何告?若有,不可诬也。有罪,若何告无?舍大臣而与小臣谋,一罪也。先君有冢卿以为师保,而蔑之,二罪也。余以巾栉事先君,而暴妾使余,三罪也。告亡而已,无告无罪。」

公使厚成叔吊于卫,曰:``寡君使瘠,闻君不抚社稷,而越在他竟,若之何不吊?以同盟之故,使瘠敢私于执事曰:『有君不吊,有臣不敏,君不赦宥,臣亦不帅职,增淫发泄,其若之何?』」卫人使大叔仪对曰:``群臣不佞,得罪于寡君。寡君不以即刑而悼弃之,以为君忧。君不忘先君之好,辱吊群臣,又重恤之。敢拜君命之辱,重拜大贶。」厚孙归,覆命,语臧武仲曰:``卫君其必归乎!有大叔仪以守,有母弟鱄以出,或抚其内,或营其外,能无归乎?」

齐人以郲寄卫侯。及其复也,以郲粮归。右宰谷从而逃归,卫人将杀之。辞曰:``余不说初矣,余狐裘而羔袖。」乃赦之。卫人立公孙剽,孙林父、宁殖相之,以听命于诸侯。

卫侯在郲,臧纥如齐,唁卫侯。与之言,虐。退而告其人曰:``卫侯其不得入矣!其言粪土也,亡而不变,何以复国?」子展、子鲜闻之,见臧纥,与之言,道。臧孙说,谓其人曰:``卫君必入。夫二子者,或挽之,或推之,欲无入,得乎?」

师归自伐秦,晋侯舍新军,礼也。成国不过半天子之军,周为六军,诸侯之大者,三军可也。于是知朔生盈而死,盈生六年而武子卒,彘裘亦幼,皆未可立也。新军无帅,故舍之。

师旷侍于晋侯。晋侯曰:``卫人出其君,不亦甚乎?」对曰:``或者其君实甚。良君将赏善而刑淫,养民如子,盖之如天,容之如地。民奉其君,爱之如父母,仰之如日月,敬之如神明,畏之如雷霆,其可出乎?夫君,神之主而民之望也。若困民之主,匮神乏祀,百姓绝望,社稷无主,将安用之?弗去何为?天生民而立之君,使司牧之,勿使失性。有君而为之贰,使师保之,勿使过度。是故天子有公,诸侯有卿,卿置侧室,大夫有贰宗,士有朋友,庶人、工、商、皂、隶、牧、圉皆有亲昵,以相辅佐也。善则赏之,过则匡之,患则救之,失则革之。自王以下,各有父兄子弟,以补察其政。史为书,瞽为诗,工诵箴谏,大夫规诲,士传言,庶人谤,商旅于市,百工献艺。故《夏书》曰:『遒人以木铎徇于路。官师相规,工执艺事以谏。』正月孟春,于是乎有之,谏失常也。天之爱民甚矣。岂其使一人肆于民上,以从其淫,而弃天地之性?必不然矣。」

秋,楚子为庸浦之役故,子囊师于棠以伐吴,吴不出而还。子囊殿,以吴为不能而弗儆。吴人自皋舟之隘要而击之,楚人不能相救。吴人败之,获楚公子宜谷。

王使刘定公赐齐侯命,曰:``昔伯舅大公,右我先王,股肱周室,师保万民,世胙大师,以表东海。王室之不坏,繄伯舅是赖。今余命女环!兹率舅氏之典,纂乃祖考,无忝乃旧。敬之哉,无废朕命!」

晋侯问卫故于中行献子,对曰:``不如因而定之。卫有君矣,伐之,未可以得志而勤诸侯。史佚有言曰:『因重而抚之。』仲虺有言曰:『亡者侮之,乱者取之,推亡固存,国之道也。』君其定卫以待时乎!」

冬,会于戚,谋定卫也。

范宣子假羽毛于齐而弗归,齐人始贰。

楚子囊还自伐吴,卒。将死,遗言谓子庚:``必城郢。」君子谓:``子囊忠。君薨不忘增其名,将死不忘卫社稷,可不谓忠乎?忠,民之望也。《诗》曰:『行归于周,万民所望。』忠也。」

\hypertarget{header-n2015}{%
\subsubsection{襄公十五年}\label{header-n2015}}

【经】十有五年春,宋公使向戌来聘。二月己亥,及向戌盟于刘。刘夏逆王后于齐。夏,齐侯伐我北鄙,围成。公救成,至遇。季孙宿、叔孙豹帅师城成郛。秋八月丁巳,日有食之。邾人伐我南鄙。冬十有一月癸亥,晋侯周卒。

【传】十五年春,宋向戌来聘,且寻盟。见孟献子,尤其室,曰:``子有令闻,而美其室,非所望也!」对曰:``我在晋,吾兄为之,毁之重劳,且不敢间。」

官师从单靖公逆王后于齐。卿不行,非礼也。

楚公子午为令尹,公子罢戎为右尹,蒍子冯为大司马,公子櫜师为右司马,公子成为左司马,屈到为莫敖,公子追舒为箴尹,屈荡为连尹,养由基为宫厩尹,以靖国人。君子谓:``楚于是乎能官人。官人,国之急也。能官人,则民无觎心。《诗》云:``嗟我怀人,置彼周行。』能官人也。王及公、侯、伯、子、男,甸、采、卫大夫,各居其列,所谓周行也。」

郑尉氏、司氏之乱,其馀盗在宋。郑人以子西、伯有、子产之故,纳贿于宋,以马四十乘与师伐、师慧。三月,公孙黑为质焉。司城子罕以堵女父、尉翩、司齐与之。良司臣而逸之,托诸季武子,武子置诸卞。郑人醢之,三人也。

师慧过宋朝,将私焉。其相曰:``朝也。」慧曰:``无人焉。」相曰:``朝也,何故无人?」慧曰:``必无人焉。若犹有人,岂其以千乘之相易淫乐之?必无人焉故也。」子罕闻之,固请而归之。

夏,齐侯围成,贰于晋故也。于是乎城成郛。

秋,邾人伐我南鄙。使告于晋,晋将为会以讨邾、莒晋侯有疾,乃止。冬,晋悼公卒,遂不克会。

郑公孙夏如晋奔丧,子蟜送葬。

宋人或得玉,献诸子罕。子罕弗受。献玉者曰:``以示玉人,玉人以为宝也,故敢献之。」子罕曰:``我以不贪为宝,尔以玉为宝,若以与我,皆丧宝也。不若人有其宝。」稽首而告曰:``小人怀璧,不可以越乡。纳此以请死也。」子罕置堵其里,使玉人为之攻之,富而后使复其所。

十二月,郑人夺堵狗之妻,而归诸范氏。

\hypertarget{header-n2029}{%
\subsubsection{襄公十六年}\label{header-n2029}}

【经】十有六年春王正月,葬晋悼公。三月,公会晋侯、宋公、卫侯、郑伯、曹伯、莒子、邾子、薛伯、杞伯、小邾子,于湨梁。戊寅,大夫盟。晋人执莒子、邾子以归。齐侯伐我北鄙。夏,公至自会。五月甲子,地震。叔老会郑伯、晋荀偃、卫宁殖、宋人伐许。秋,齐侯伐我北鄙,围郕。大雩。冬,叔孙豹如晋。

【传】十六年春,葬晋悼公。平公即位,羊舌肸为傅,张君臣为中军司马,祁奚、韩襄、栾盈、士鞅为公族大夫,虞丘书为乘马御。改服修官,烝于曲沃。警守而下,会于湨梁。命归侵田。以我故,执邾宣公、莒犁比公,且曰:``通齐、楚之使。」

晋侯与诸侯宴于温,使诸大夫舞,曰:``歌诗必类!」齐高厚之诗不类。荀偃怒,且曰:``诸侯有异志矣!」使诸大夫盟高厚,高厚逃归。于是,叔孙豹、晋荀偃、宋向戌、卫宁殖、郑公孙虿、小邾之大夫盟曰:``同讨不庭。」

许男请迁于晋。诸侯遂迁许,许大夫不可。晋人归诸侯。

郑子蟜闻将伐许,遂相郑伯以从诸侯之师。穆叔从公。齐子帅师会晋荀偃。书曰:``会郑伯。」为夷故也。

夏六月,次于棫林。庚寅,伐许,次于函氏。

晋荀偃、栾□帅师伐楚,以报宋扬梁之役。楚公子格帅师及晋师战于湛阪,楚师败绩。晋师遂侵方城之外,复伐许而还。

秋,齐侯围郕,孟孺子速缴之。齐侯曰:``是好勇,去之以为之名。」速遂塞海陉而还。

冬,穆叔如晋聘,且言齐故。晋人曰:``以寡君之未禘祀,与民之未息。不然,不敢忘。」穆叔曰:``以齐人之朝夕释憾于敝邑之地,是以大请!敝邑之急,朝不及夕,引领西望曰:『庶几乎!』比执事之间,恐无及也!」见中行献子,赋《圻父》。献子曰:``偃知罪矣!敢不从执事以同恤社稷,而使鲁及此。」见范宣子,赋《鸿雁》之卒章。宣子曰:``□在此,敢使鲁无鸠乎?」

\hypertarget{header-n2041}{%
\subsubsection{襄公十七年}\label{header-n2041}}

【经】十有七年春王二月庚午,邾子卒。宋人伐陈。夏,卫石买帅师伐曹。秋,齐侯伐我北鄙,围桃。高厚帅师伐我北鄙,围防。九月,大雩。宋华臣出奔陈。冬,邾人伐我南鄙。

【传】十七年春,宋庄朝伐陈,获司徒卬,卑宋也。

卫孙蒯田于曹隧,饮马于重丘,毁其瓶。重丘人闭门而呴之,曰:``亲逐而君,尔父为厉。是之不忧,而何以田为?」

夏,卫石买、孙蒯伐曹,取重丘。曹人愬于晋。

齐人以其未得志于我故,秋,齐侯伐我北鄙,围桃。高厚围臧纥于防。师自阳关逆臧孙,至于旅松。郰叔纥、臧畴、臧贾帅甲三百,宵犯齐师,送之而复。齐师去之。

齐人获臧坚。齐侯使夙沙卫唁之,且曰:``无死!」坚稽首曰:``拜命之辱!抑君赐不终,姑又使其刑臣礼于士。」以杙抉其伤而死。

冬,邾人伐我南鄙,为齐故也。

宋华阅卒。华臣弱皋比之室,使贼杀其宰华吴。贼六人以铍杀诸卢门合左师之后。左师惧曰:``老夫无罪。」贼曰:``皋比私有讨于吴。」遂幽其妻,曰:``畀余而大璧!」宋公闻之,曰:``臣也,不唯其宗室是暴,大乱宋国之政,必逐之!」左师曰:``臣也,亦卿也。大臣不顺,国之耻也。不如盖之。」乃舍之。左师为己短策,苟过华臣之门,必聘。

十一月甲午,国人逐□狗,□狗入于华臣氏,国人从之。华臣惧,遂奔陈。

宋皇国父为大宰,为平公筑台,妨于农功。子罕请俟农功之毕,公弗许。筑者讴曰:``泽门之皙,实兴我役。邑中之黔,实尉我心。」子罕闻之,亲执扑,以行筑者,而抶其不勉者,曰:``吾侪小人,皆有阖庐以辟燥湿寒暑。今君为一台而不速成,何以为役?」讴者乃止。或问其故,子罕曰:``宋国区区,而且诅有祝,祸之本也。」

齐晏桓子卒。晏婴粗縗斩,苴絰、带、杖,菅屦,食鬻,居倚庐,寝苫,枕草。其老曰:``非大夫之礼也。」曰:``唯卿为大夫。」

\hypertarget{header-n2055}{%
\subsubsection{襄公十八年}\label{header-n2055}}

【经】十有八年春,白狄来。夏,晋人执卫行人石买。秋,齐师伐我北鄙。冬十月,公会晋侯、宋公、卫侯、郑伯、曹伯、莒子、邾子、滕子、薛伯、杞伯、小邾子同围齐。曹伯负刍卒于师。楚公子午帅师伐郑。

【传】十八年春,白狄始来。

夏,晋人执卫行人石买于长子,执孙蒯于纯留,为曹故也。

秋,齐侯伐我北鄙。中行献子将伐齐,梦与厉公讼,弗胜,公以戈击之,首队于前,跪而戴之,奉之以走,见梗阳之巫皋。他日,见诸道,与之言,同。巫曰:``今兹主必死,若有事于东方,则可以逞。」献子许诺。

晋侯伐齐,将济河。献子以朱丝系玉二□,而祷曰:``齐环怙恃其险,负其众庶,弃好背盟,陵虐神主。曾臣彪将率诸侯以讨焉,其官臣偃实先后之。苟捷有功,无作神羞,官臣偃无敢复济。唯尔有神裁之!」沉玉而济。

冬十月,会于鲁济,寻湨梁之言,同伐齐。齐侯御诸平阴,堑防门而守之,广里。夙沙卫曰:``不能战,莫如守险。」弗听。诸侯之士门焉,齐人多死。范宣子告析文子曰:``吾知子,敢匿情乎?鲁人、莒人皆请以车千乘自其乡入,既许之矣。若入,君必失国。子盍图之?」子家以告公,公恐。晏婴闻之曰:``君固无勇,而又闻是,弗能久矣。」齐侯登巫山以望晋师。晋人使司马斥山泽之险,虽所不至,必旗而疏陈之。使乘车者左实右伪,以旗先,舆曳柴而从之。齐侯见之,畏其众也,乃脱归。丙寅晦,齐师夜遁。师旷告晋侯曰:``鸟乌之声乐,齐师其遁。」邢伯告中行伯曰:``有班马之声,齐师其遁。」叔向告晋侯曰:``城上有乌,齐师其遁。」

十一月丁卯朔,入平阴,遂从齐师。夙沙卫连大车以塞隧而殿。殖绰、郭最曰:``子殿国师,齐之辱也。子姑先乎!」乃代之殿。卫杀马于隘以塞道。晋州绰及之,射殖绰,中肩,两矢夹脰,曰:``止,将为三军获。不止,将取其衷。」顾曰:``为私誓。」州绰曰:``有如日!」乃弛弓而自后缚之。其右具丙亦舍兵而缚郭最,皆衿甲面缚,坐于中军之鼓下。

晋人欲逐归者,鲁、卫请攻险。己卯,荀偃、士□以中军克京兹。乙酉,魏绛、栾盈以下军克邿。赵武、韩起以上军围卢,弗克。十二月戊戌,及秦周,伐雍门之萩。范鞅门于雍门,其御追喜以戈杀犬于门中。孟庄子斩其以为公琴。己亥,焚雍门及西郭、南郭。刘难、士弱率诸侯之师焚申池之竹木。壬寅,焚东郭、北郭。范鞅门于扬门。州绰门于东闾,左骖迫,还于门中,以枚数阖。

齐侯驾,将走邮棠。大子与郭荣扣马,曰:``师速而疾,略也。将退矣,君何惧焉!且社稷之主,不可以轻,轻则失众。君必待之。」将犯之,大子抽剑断鞅,乃止。甲辰,东侵及潍,南及沂。

郑子孔欲去诸大夫,将叛晋而起楚师以去之。使告子庚,子庚弗许。楚子闻之,使杨豚尹宜告子庚曰:``国人谓不谷主社稷,而不出师,死不从礼。不谷即位,于今五年,师徒不出,人其以不谷为自逸,而忘先君之业矣。大夫图之!其若之何?」子庚叹曰:``君王其谓午怀安乎!吾以利社稷也。」见使者,稽首而对曰:``诸侯方睦于晋,臣请尝之。若可,君而继之。不可,收师而退,可以无害,君亦无辱。」子庚帅师治兵于汾。于是子蟜、伯有、子张从郑伯伐齐,子孔、子展、子西守。二子知子孔之谋,完守入保。子孔不敢会楚师。

楚师伐郑,次于鱼陵。右师城上棘,遂涉颖,次于旃然。蒍子冯、公子格率锐师侵费滑、胥靡、献于、雍梁,右回梅山,侵郑东北,至于虫牢而反。子庚门于纯门,信于城下而还。涉于鱼齿之下,甚雨及之,楚师多冻,役徒几尽。

晋人闻有楚师,师旷曰:``不害。吾骤歌北风,又歌南风。南风不竞,多死声。楚必无功。」董叔曰:``天道多在西北,南师不时,必无功。」叔向曰:``在其君之德也。」

\hypertarget{header-n2070}{%
\subsubsection{襄公十九年}\label{header-n2070}}

【经】十有九年春王正月,诸侯盟于祝柯。晋人执邾子,公至自伐齐。取邾田,自漷水。季孙宿如晋。葬曹成公。夏,卫孙林父帅师伐齐。秋七月辛卯,齐侯环卒。晋士□帅师侵齐,至谷,闻齐侯卒,乃还。八月丙辰,仲孙蔑卒。齐杀其大夫高厚。郑杀其大夫公子嘉。冬,葬齐灵公。城西郛。叔孙豹会晋士□于柯。城武城。

【传】十九年春,诸侯还自沂上,盟于督扬,曰:``大毋侵小。」

执邾悼公,以其伐我故。遂次于泗上,疆我田。取邾田,自漷水归之于我。晋侯先归。公享晋六卿于蒲圃,赐之三命之服。军尉、司马、司空、舆尉、候奄,皆受一命之服。贿荀偃束锦,加璧,乘马,先吴寿梦之鼎。

荀偃瘅疽,生疡于头。济河,及着雍,病,目出。大夫先归者皆反。士□请见,弗内。请后,曰:``郑甥可。」二月甲寅,卒,而视,不可含。宣子盥而抚之,曰:``事吴,敢不如事主!」犹视。栾怀子曰:``其为未卒事于齐故也乎?」乃复抚之曰:``主苟终,所不嗣事于齐者,有如河!」乃暝,受含。宣子出,曰:``吾浅之为丈夫也。」

晋栾鲂帅师从卫孙文子伐齐。季武子如晋拜师,晋侯享之。范宣子为政,赋《黍苗》。季武子兴,再拜稽首曰:``小国之仰大国也,如百谷之仰膏雨焉!若常膏之,其天下辑睦,岂唯敝邑?」赋《六月》。

季武子以所得于齐之兵,作林钟而铭鲁功焉。臧武仲谓季孙曰:``非礼也。夫铭,天子令德,诸侯言时计功,大夫称伐。今称伐则下等也,计功则借人也,言时则妨民多矣,何以为铭?且夫大伐小,取其所得以作彝器,铭其功烈以示子孙,昭明德而惩无礼也。今将借人之力以救其死,若之何铭之?小国幸于大国,而昭所获焉以怒之,亡之道也。」

齐侯娶于鲁,曰颜懿姬,无子。其侄鬲声姬,生光,以为大子。诸子仲子、戎子,戎子嬖。仲子生牙,属诸戎子。戎子请以为大子,许之。仲子曰:``不可。废常,不祥;间诸侯,难。光之立也,列于诸侯矣。今无故而废之,是专黜诸侯,而以难犯不祥也。君必悔之。」公曰:``在我而已。」遂东大子光。使高厚傅牙,以为大子,夙沙卫为少傅。

齐侯疾,崔杼微逆光。疾病,而立之。光杀戎子,尸诸朝,非礼也。妇人无刑。虽有刑,不在朝市。

夏五月壬辰晦,齐灵公卒。庄公即位,执公子牙于句渎之丘。以夙沙卫易己,卫奔高唐以叛。

晋士□侵齐,及谷,闻丧而还,礼也。

于四月丁未,郑公孙虿卒,赴于晋大夫。范宣子言于晋侯,以其善于伐秦也。六月,晋侯请于王,王追赐之大路,使以行,礼也。

秋八月,齐崔杼杀高厚于洒蓝而兼其室。书曰:``齐杀其大夫。」从君于昏也。

郑子孔之为政也专。国人患之,乃讨西宫之难,与纯门之师。子孔当罪,以其甲及子革、子良氏之甲守。甲辰,子展、子西率国人伐之,杀子孔而分其室。书曰:``郑杀其大夫。」专也。子然、子孔,宋子之子也;士子孔,圭妫之子也。圭妫之班亚宋子,而相亲也;二子孔亦相亲也。僖之四年,子然卒,简之元年,士子孔卒。司徒孔实相子革、子良之室,三室如一,故及于难。子革、子良出奔楚,子革为右尹。郑人使子展当国,子西听政,立子产为卿。

齐庆封围高唐,弗克。冬十一月,齐侯围之,见卫在城上,号之,乃下。问守备焉,以无备告。揖之,乃登。闻师将傅,食高唐人。殖绰、工偻会夜缒纳师,醢卫于军。

城西郛,惧齐也。

齐及晋平,盟于大隧。故穆叔会范宣子于柯。穆叔见叔向,赋《载驰》之四章。叔向曰:``肸敢不承命。」穆叔曰:``齐犹未也,不可以不惧。」乃城武城。

卫石共子卒,悼子不哀。孔成子曰:``是谓蹶其本,必不有其宗。」

\hypertarget{header-n2090}{%
\subsubsection{襄公二十年}\label{header-n2090}}

【经】二十年春王正月辛亥,仲孙速会莒人盟于向。夏六月庚申,公会晋侯、齐侯、宋公、卫侯、郑伯、曹伯、莒子、邾子、滕子、薛伯、杞伯,小邾子盟于澶渊。秋,公至自会。仲孙速帅师伐邾。蔡杀其大夫公子燮。蔡公子履出奔楚。陈侯之弟黄出奔楚。叔老如齐。冬十月丙辰朔,日有食之。季孙宿如宋。

【传】二十年春,及莒平。孟庄子会莒人,盟于向,督扬之盟故也。

夏,盟于澶渊,齐成故也。

邾人骤至,以诸侯之事,弗能报也。秋,孟庄子伐邾以报之。

蔡公子燮欲以蔡之晋,蔡人杀之。公子履,其母弟也,故出奔楚。

陈庆虎、庆寅畏公子黄之逼,愬诸楚曰:``与蔡司马同谋。」楚人以为讨。公子黄出奔楚。

初,蔡文侯欲事晋,曰:``先君与于践士之盟,晋不可弃,且兄弟也。」畏楚,不能行而卒。楚人使蔡无常,公子燮求从先君以利蔡,不能而死。书曰:``蔡杀其大夫公子燮」,言不与民同欲也;``陈侯之弟黄出奔楚」,言非其罪也。公子黄将出奔,呼于国曰:``庆氏无道,求专陈国,暴蔑其君,而去其亲,五年不灭,是无天也。」

齐子初聘于齐,礼也。

冬,季武子如宋,报向戌之聘也。褚师段逆之以受享,赋《常棣》之七章以卒。宋人重贿之。归,覆命,公享之。赋《鱼丽》之卒章。公赋《南山有台》。武子去所,曰:``臣不堪也。」

卫宁惠子疾,召悼子曰:``吾得罪于君,悔而无及也。名藏在诸侯之策,曰:『孙林父、宁殖出其君。』君入则掩之。若能掩之,则吾子也。若不能,犹有鬼神,吾有馁而已,不来食矣。」悼子许诺,惠子遂卒。

\hypertarget{header-n2103}{%
\subsubsection{襄公二十一年}\label{header-n2103}}

【经】二十有一年春王正月,公如晋。邾庶其以漆、闾丘来奔。夏,公至自晋。秋,晋栾出奔楚。九月庚戌朔,日有食之。冬十月庚辰朔,日有食之。曹伯来朝。公会晋侯、齐侯、宋公、卫侯、郑伯、曹伯、莒子、邾子于商任。

【传】二十一年春,公如晋,拜师及取邾田也。

邾庶其以漆、闾丘来奔。季武子以公姑姊妻之,皆有赐于其从者。

于是鲁多盗。季孙谓臧武仲曰:``子盍诘盗?」武仲曰:``不可诘也,纥又不能。」季孙曰:``我有四封,而诘其盗,何故不可?子为司寇,将盗是务去,若之何不能?」武仲曰:``子召外盗而大礼焉,何以止吾盗?子为正卿,而来外盗;使纥去之,将何以能?庶其窃邑于邾以来,子以姬氏妻之,而与之邑,其从者皆有赐焉。若大盗礼焉以君之姑姊与其大邑,其次皋牧舆马,其小者衣裳剑带,是赏盗也。赏而去之,其或难焉。纥也闻之,在上位者,洒濯其心,壹以待人,轨度其信,可明征也,而后可以治人。夫上之所为,民之归也。上所不为而民或为之,是以加刑罚焉,而莫敢不惩。若上之所为而民亦为之,乃其所也,又可禁乎?《夏书》曰:『念兹在兹,释兹在兹,名言兹在兹,允出兹在兹,惟帝念功。』将谓由己壹也。信由己壹,而后功可念也。」

庶其非卿也,以地来,虽贱必书,重地也。

齐侯使庆佐为大夫,复讨公子牙之党,执公子买于句渎之丘。公子鉏来奔。叔孙还奔燕。

夏,楚子庚卒,楚子使薳子冯为令尹。访于申叔豫,叔豫曰:``国多宠而王弱,国不可为也。」遂以疾辞。方署,阙地,下冰而床焉。重茧衣裘,鲜食而寝。楚子使医视之,复曰:``瘠则甚矣,而血气未动。」乃使子南为令尹。

栾桓子娶于范宣子,生怀子。范鞅以其亡也,怨栾氏,故与栾盈为公族大夫而不相能。桓子卒,栾祁与其老州宾通,几亡室矣。怀子患之。祁惧其讨也,愬诸宣子曰:``盈将为乱,以范氏为死桓主而专政矣,曰:『吾父逐鞅也,不怒而以宠报之,又与吾同官而专之,吾父死而益富。死吾父而专于国,有死而已,吾蔑从之矣!』其谋如是,惧害于主,吾不敢不言。」范鞅为之征。怀子好施,士多归之。宣子畏其多士也,信之。怀子为下卿,宣子使城着而遂逐之。

秋,栾盈出奔楚。宣子杀箕遗、黄渊、嘉父、司空靖、邴豫、董叔、邴师、申书、羊舌虎、叔罴。囚伯华、叔向、籍偃。人谓叔向曰:``子离于罪,其为不知乎?」叔向曰:``与其死亡若何?《诗》曰:『优哉游哉,聊以卒岁。』知也。」乐王鲋见叔向曰:``吾为子请!」叔向弗应。出,不拜。其人皆咎叔向。叔向曰:``必祁大夫。。」室老闻之,曰:``乐王鲋言于君无不行,求赦吾子,吾子不许。祁大夫所不能也,而曰『必由之』,何也?」叔向曰:``乐王鲋,从君者也,何能行?祁大夫外举不弃仇,内举不失亲,其独遗我乎?《诗》曰:『有觉德行,四国顺之。』夫子,觉者也。」

晋侯问叔向之罪于乐王鲋,对曰:``不弃其亲,其有焉。」于是祁奚老矣,闻之,乘馹而见宣子,曰:``《诗》曰:『惠我无疆,子孙保之。』《书》曰:『圣有谟勋,明征定保。』夫谋而鲜过,惠训不倦者,叔向有焉,社稷之固也。犹将十世宥之,以劝能者。今壹不免其身,以弃社稷,不亦惑乎?鲧殛而禹兴。伊尹放大甲而相之,卒无怨色。管、蔡为戮,周公右王。若之何其以虎也弃社稷?子为善,谁敢不勉?多杀何为?」宣子说,与之乘,以言诸公而免之。不见叔向而归。叔向亦不告免焉而朝。

初,叔向之母□石叔虎之母美而不使,其子皆谏其母。其母曰:``深山大泽,实生龙蛇。彼美,余惧其生龙蛇以祸女。女,敝族也。国多大宠,不仁人间之,不亦难乎?余何爱焉!」使往视寝,生叔虎。美而有勇力,栾怀子嬖之,故羊舌氏之族及于难。

栾盈过于周,周西鄙掠之。辞于行人,曰:``天子陪臣盈,得罪于王之守臣,将逃罪。罪重于郊甸,无所伏窜,敢布其死。昔陪臣书能输力于王室,王施惠焉。其子□,不能保任其父之劳。大君若不弃书之力,亡臣犹有所逃。若弃书之力,而思□之罪,臣,戮余也,将归死于尉氏,不敢还矣。敢布四体,唯大君命焉!」王曰:``尤而效之,其又甚焉!」使司徒禁掠栾氏者,归所取焉。使候出诸轘辕。

冬,曹武公来朝,始见也。

会于商任,锢栾氏也。齐侯、卫侯不敬。叔向曰:``二君者必不免。会朝,礼之经也;礼,政之舆也;政,身之守也;怠礼失政,失政不立,是以乱也。」

知起、中行喜、州绰、邢蒯出奔齐,皆栾氏之党也。乐王鲋谓范宣子曰:``盍反州绰、邢蒯,勇士也。」宣子曰:``彼栾氏之勇也,余何获焉?」王鲋曰:``子为彼栾氏,乃亦子之勇也。」

齐庄公朝,指殖绰、郭最曰:``是寡人之雄也。」州绰曰:``君以为雄,谁敢不雄?然臣不敏,平阴之役,先二子鸣。」庄公为勇爵。殖绰、郭最欲与焉。州绰曰:``东闾之役,臣左骖迫,还于门中,识其枚数。其可以与于此乎?」公曰:``子为晋君也。」对曰:``臣为隶新。然二子者,譬于禽兽,臣食其肉而寝处其皮矣。」

\hypertarget{header-n2122}{%
\subsubsection{襄公二十二年}\label{header-n2122}}

【经】二十有二年春王正月,公至自会。夏四月。秋七月辛酉,叔老卒。冬,公会晋侯、齐侯、宋公、卫侯、郑伯、曹伯、莒子、邾子、薛伯、杞伯、小邾子于沙随。公至自会。楚杀其大夫公子追舒。

【传】二十二年春,臧武仲如晋,雨,过御叔。御叔在其邑,将饮酒,曰:``焉用圣人!我将饮酒而己,雨行,何以圣为?」穆叔闻之曰:``不可使也,而傲使人,国之蠹也。」令倍其赋。

夏,晋人征朝于郑。郑人使少正公孙侨对曰:``在晋先君悼公九年,我寡君于是即位。即位八月,而我先大夫子驷从寡君以朝于执事。执事不礼于寡君。寡君惧,因是行也,我二年六月朝于楚,晋是以有戏之役。楚人犹竞,而申礼于敝邑。敝邑欲从执事而惧为大尤,曰晋其谓我不共有礼,是以不敢携贰于楚。我四年三月,先大夫子蟜又从寡君以观衅于楚,晋于是乎有萧鱼之役。谓我敝邑,迩在晋国,譬诸草木,吾臭味也,而何敢差池?楚亦不竞,寡君尽其土实,重之以宗器,以受齐盟。遂帅群臣随于执事以会岁终。贰于楚者,子侯、石盂,归而讨之。湨梁之明年,子蟜老矣,公孙夏从寡君以朝于君,见于尝酎,与执燔焉。间二年,闻君将靖东夏,四月又朝,以听事期。不朝之间,无岁不聘,无役不从。以大国政令之无常,国家罢病,不虞荐至,无日不惕,岂敢忘职?大国若安定之,其朝夕在庭,何辱命焉?若不恤其患,而以为口实,其无乃不堪任命,而翦为仇雠,敝邑是惧。其敢忘君命?委诸执事,执事实重图之。」

秋,栾盈自楚适齐。晏平仲言于齐侯曰:``商任之会,受命于晋。今纳栾氏,将安用之?小所以事大,信也。失信不立,君其图之。」弗听。退告陈文子曰:``君人执信,臣人执共,忠信笃敬,上下同之,天之道也。君自弃也,弗能久矣!」

九月,郑公孙黑肱有疾,归邑于公。召室老、宗人立段,而使黜官、薄祭。祭以特羊,殷以少牢。足以共祀,尽归其馀邑。曰:``吾闻之,生于乱世,贵而能贫,民无求焉,可以后亡。敬共事君,与二三子。生在敬戒,不在富也。」己巳,伯张卒。君子曰:``善戒。《诗》曰:『慎尔侯度,用戒不虞。』郑子张其有焉。」

冬,会于沙随,复锢栾氏也。

栾盈犹在齐,晏子曰:``祸将作矣!齐将伐晋,不可以不惧。」

楚观起有宠于令尹子南,未益禄,而有马数十乘。楚人患之,王将讨焉。子南之子弃疾为王御士,王每见之,必泣。弃疾曰:``君三泣臣矣,敢问谁之罪也?」王曰:``令尹之不能,尔所知也。国将讨焉,尔其居乎?」对曰:``父戮子居,君焉用之?泄命重刑,臣亦不为。」王遂杀子南于朝,轘观起于四竟。子南之臣谓弃疾,请徙子尸于朝,曰:``君臣有礼,唯二三子。」三日,弃疾请尸,王许之。既葬,其徒曰:``行乎?」曰:``吾与杀吾父,行将焉入?」曰:``然则臣王乎?」曰:``弃父事仇,吾弗忍也。」遂缢而死。

复使薳子冯为令尹,公子齮为司马。屈建为莫敖。有宠于薳子者八人,皆无禄而多马。他日朝,与申叔豫言。弗应而退。从之,入于人中。又从之,遂归。退朝,见之,曰:``子三困我于朝,吾惧,不敢不见。吾过,子姑告我。何疾我也?」对曰:``吾不免是惧,何敢告子?」曰:``何故?」对曰:``昔观起有宠于子南,子南得罪,观起车裂。何故不惧?」自御而归,不能当道。至,谓八人者曰:``吾见申叔,夫子所谓生死而肉骨也。知我者,如夫子则可。不然,请止。」辞八人者,而后王安之。

十二月,郑游贩将归晋,未出竟,遭逆妻者,夺之,以馆于邑。丁巳,其夫攻子明,杀之,以其妻行。子展废良而立大叔,曰:``国卿,君之贰也,民之主也,不可以苟。请舍子明之类。」求亡妻者,使复其所。使游氏勿怨,曰:``无昭恶也。」

\hypertarget{header-n2135}{%
\subsubsection{襄公二十三年}\label{header-n2135}}

【经】二十有三年春王二月癸酉朔,日有食之。三月己巳,杞伯□卒。夏,邾畀我来奔。葬杞孝公。陈杀其大夫庆虎及庆寅。陈侯之弟黄自楚归于陈。晋栾盈复入于晋,入于曲沃。秋,齐侯伐卫,遂伐晋。八月,叔孙豹帅师救晋次于雍榆。己卯,仲孙速卒。冬十月乙亥,臧孙纥出奔邾。晋人杀栾盈。齐侯袭莒。

【传】二十三年春,杞孝公卒,晋悼夫人丧之。平公不彻乐,非礼也。礼,为邻国阙。

陈侯如楚。公子黄愬二庆于楚,楚人召之。使庆乐往,杀之。庆氏以陈叛。夏,屈建从陈侯围陈。陈人城,板队而杀人。役人相命,各杀其长。遂杀庆虎、庆寅。楚人纳公子黄。君子谓:``庆氏不义,不可肆也。故《书》曰:『惟命不于常。』」

晋将嫁女于吴,齐侯使析归父媵之,以藩载栾盈及其士,纳诸曲沃。栾盈夜见胥午而告之。对曰:``不可。天之所废,谁能兴之?子必不免。吾非爱死也,知不集也。」盈曰:``虽然,因子而死,吾无悔矣。我实不天,子无咎焉。」许诺。伏之,而觞曲沃人。乐作。午言曰:``今也得栾孺子,何如?」对曰:``得主而为之死,犹不死也。」皆叹,有泣者。爵行,又言。皆曰:``得主,何贰之有?」盈出,遍拜之。

四月,栾盈帅曲沃之甲,因魏献子,以昼入绛。初,栾盈佐魏庄子于下军,献子私焉,故因之。赵氏以原、屏之难怨栾氏,韩、赵方睦。中行氏以伐秦之役怨栾氏,而固与范氏和亲。知悼子少,而听于中行氏。程郑嬖于公。唯魏氏及七舆大夫与之。

乐王鲋待坐于范宣子。或告曰:``栾氏至矣!」宣子惧。桓子曰:``奉君以走固宫,必无害也。且栾氏多怨,子为政,栾氏自外,子在位,其利多矣。既有利权,又执民柄,将何惧焉?栾氏所得,其唯魏氏乎!而可强取也。夫克乱在权,子无懈矣。」公有姻丧,王鲋使宣子墨縗冒絰,二妇人辇以如公,奉公以如固宫。

范鞅逆魏舒,则成列既乘,将逆栾氏矣。趋进,曰:``栾氏帅贼以入,鞅之父与二三子在君所矣。使鞅逆吾子。鞅请骖乘。」持带,遂超乘,右抚剑,左援带,命驱之出。仆请,鞅曰:``之公。」宣子逆诸阶,执其手,赂之以曲沃。

初,斐豹隶也,着于丹书。栾氏之力臣曰督戎,国人惧之。斐豹谓宣子曰:``苟焚丹书,我杀督戎。」宣子喜,曰:``而杀之,所不请于君焚丹书者,有如日!」乃出豹而闭之,督戎从之。逾隐而待之,督戎逾入,豹自后击而杀之。范氏之徒在台后,栾氏乘公门。宣子谓鞅曰:``矢及君屋,死之!」鞅用剑以帅卒,栾氏退。摄车从之,遇栾氏,曰:``乐免之,死将讼女于天。」乐射之,不中;又注,则乘槐本而覆。或以戟钩之,断肘而死。栾鲂伤。栾盈奔曲沃,晋人围之。

秋,齐侯伐卫。先驱,谷荣御王孙挥,召扬为右。申驱,成秩御莒恒,申鲜虞之傅挚为右。曹开御戎,晏父戎为右。贰广,上之登御邢公,卢蒲癸为右。启,牢成御襄罢师,狼蘧疏为右。胠,商子车御侯朝,桓跳为右。大殿,商子游御夏之御寇,崔如为右,烛庸之越驷乘。

自卫将遂伐晋。晏平仲曰:``君恃勇力以伐盟主,若不济,国之福也。不德而有功,忧必及君。」崔杼谏曰:``不可。臣闻之,小国间大国之败而毁焉,必受其咎。君其图之!」弗听。陈文子见崔武子,曰:``将如君何?」武子曰:``吾言于君,君弗听也。以为盟主,而利其难。群臣若急,君于何有?子姑止之。」文子退,告其人曰:``崔子将死乎!谓君甚,而又过之,不得其死。过君以义,犹自抑也,况以恶乎?」

齐侯遂伐晋,取朝歌,为二队,入孟门,登大行,张武军于荧庭,戍郫邵,封少水,以报平阴之役,乃还。赵胜帅东阳之师以追之,获晏□。八月,叔孙豹帅师救晋,次于雍榆,礼也。

季武子无适子,公弥长,而爱悼子,欲立之。访于申丰,曰:``弥与纥,吾皆爱之,欲择才焉而立之。」申丰趋退,归,尽室将行。他日,又访焉,对曰:``其然,将具敝车而行。」乃止。访于臧纥,臧纥曰:``饮我酒,吾为子立之。」季氏饮大夫酒,臧纥为客。既献,臧孙命北面重席,新尊絜之。召悼之,降,逆之。大夫皆起。及旅,而召公鉏,使与之齿,季孙失色。

季氏以公鉏为马正,愠而不出。闵子马见之,曰:``子无然!祸福无门,唯人所召。为人子者,患不孝,不患无所。敬共父命,何常之有?若能孝敬,富倍季氏可也。奸回不轨,祸倍下民可也。」公鉏然之。敬共朝夕,恪居官次。季孙喜,使饮己酒,而以具往,尽舍旃。故公鉏氏富,又出为公左宰。

孟孙恶臧孙,季孙爱之。孟氏之御驺丰点好羯也,曰:``从余言,必为孟孙。」再三云,羯从之。孟庄子疾,丰点谓公鉏:``苟立羯,请仇臧氏。」公鉏谓季孙曰:``孺子秩,固其所也。若羯立,则季氏信有力于臧氏矣。」弗应。己卯,孟孙卒,公鉏奉羯立于户侧。季孙至,入,哭,而出,曰:``秩焉在?」公鉏曰:``羯在此矣!」季孙曰:``孺子长。」公鉏曰:``何长之有?唯其才也。且夫子之命也。」遂立羯。秩奔邾。

臧孙入,哭甚哀,多涕。出,其御曰:``孟孙之恶子也,而哀如是。季孙若死,其若之何?」臧孙曰:``季孙之爱我,疾疢也。孟孙之恶我,药石也。美疢不如恶石。夫石犹生我,疢之美,其毒滋多。孟孙死,吾亡无日矣。」

孟氏闭门,告于季秋曰:``臧氏将为乱,不使我葬。」季孙不信。臧孙闻之,戒。冬十月,孟氏将辟,藉除于臧氏。臧孙使正夫助之,除于东门,甲从己而视之。孟氏又告季孙。季孙怒,命攻臧氏。乙亥,臧纥斩鹿门之关以出,奔邾。

初,臧宣叔娶于铸,生贾及为而死。继室以其侄,穆姜之姨子也。生纥,长于公宫。姜氏爱之,故立之。臧贾、臧为出在铸。臧武仲自邾使告臧贾,且致大蔡焉,曰:``纥不佞,失守宗祧,敢告不吊。纥之罪,不及不祀。子以大蔡纳请,其可。」贾曰:``是家之祸也,非子之过也。贾闻命矣。」再拜受龟。使为以纳请,遂自为也。臧孙如防,使来告曰:``纥非能害也,知不足也。非敢私请!苟守先祀,无废二勋,敢不辟邑。」乃立臧为。臧纥致防而奔齐。其人曰:``其盟我乎?」臧孙曰:``无辞。」将盟臧氏,季孙召外史掌恶臣,而问盟首焉,对曰:``盟东门氏也,曰:『毋或如东门遂,不听公命,杀适立庶。』盟叔孙氏也,曰:『毋或如叔孙侨如,欲废国常,荡覆公室。』」季孙曰:``臧孙之罪,皆不及此。」孟椒曰:``盍以其犯门斩关?」季孙用之。乃盟臧氏曰:``无或如臧孙纥,干国之纪,犯门斩关。」臧孙闻之,曰:``国有人焉!谁居?其孟椒乎!」

晋人克栾盈于曲沃,尽杀栾氏之族党。栾鲂出奔宋。书曰:``晋人杀栾盈。」不言大夫,言自外也。

齐侯还自晋,不入。遂袭莒,门于且于,伤股而退。明日,将复战,期于寿舒。杞殖、华还载甲,夜入且于之隧,宿于莒郊。明日,先遇莒子于蒲侯氏。莒子重赂之,使无死,曰:``请有盟。」华周对曰:``贪货弃命,亦君所恶也。昏而受命,日未中而弃之,何以事君?」莒子亲鼓之,从而伐之,获杞梁。莒人行成。

齐侯归,遇杞梁之妻于郊,使吊之。辞曰:``殖之有罪,何辱命焉?若免于罪,犹有先人之敝庐在,下妾不得与郊吊。」齐侯吊诸其室。

齐侯将为臧纥田。臧孙闻之,见齐侯,与之言伐晋,对曰:``多则多矣!抑君似鼠。夫鼠昼伏夜动,不穴于寝庙,畏人故也。今君闻晋之乱而后作焉。宁将事之,非鼠如何?」乃弗与田。

仲尼曰:``知之难也。有臧武仲之知,而不容于鲁国,抑有由也。作不顺而施不恕也。《夏书》曰:『念兹在兹。』顺事、恕施也。」

\hypertarget{header-n2160}{%
\subsubsection{襄公二十四年}\label{header-n2160}}

【经】二十有四年春,叔孙豹如晋。仲孙羯帅师侵齐。夏,楚子伐吴。秋七月甲子朔,日有食之,既。齐崔杼帅师伐莒。大水。八月癸巳朔,日有食之。公会晋侯、宋公、卫侯、郑伯、曹伯、莒子、邾子、滕子、薛伯、杞伯、小邾子于夷仪。冬,楚子、蔡侯、陈侯、许男伐郑。公至自会。陈金咸宜咎出奔楚。叔孙豹如京师。大饥。

【传】二十四年春,穆叔如晋。范宣子逆之,问焉,曰:``古人有言曰,『死而不朽』,何谓也?」穆叔未对。宣子曰:``昔□之祖,自虞以上,为陶唐氏,在夏为御龙氏,在商为豕韦氏,在周为唐杜氏,晋主夏盟为范氏,其是之谓乎?」穆叔曰:``以豹所闻,此之谓世禄,非不朽也。鲁有先大夫曰臧文仲,既没,其言立。其是之谓乎!豹闻之,大上有立德,其次有立功,其次有立言,虽久不废,此之谓不朽。若夫保姓受氏,以守宗祊,世不绝祀,无国无之,禄之大者,不可谓不朽。」

范宣子为政,诸侯之币重。郑人病之。二月,郑伯如晋。子产寓书于子西以告宣子,曰:``子为晋国,四邻诸侯,不闻令德,而闻重币,侨也惑之。侨闻君子长国家者,非无贿之患,而无令名之难。夫诸侯之贿聚于公室,则诸侯贰。若吾子赖之,则晋国贰。诸侯贰,则晋国坏。晋国贰,则子之家坏。何没没也!将焉用贿?夫令名,德之舆也。德,国家之基也。有基无坏,无亦是务乎!有德则乐,乐则能久。《诗》云:『乐只君子,邦家之基。』有令德也夫!『上帝临女,无贰尔心。』有令名也夫!恕思以明德,则令名载而行之,是以远至迩安。毋宁使人谓子『子实生我』,而谓『子濬我以生』乎?像有齿以焚其身,贿也。」宣子说,乃轻币。是行也,郑伯朝晋,为重币故,且请伐陈也。郑伯稽首,宣子辞。子西相,曰:``以陈国之介恃大国而陵虐于敝邑,寡君是以请罪焉。敢不稽首。」

孟孝伯侵齐,晋故也。

夏,楚子为舟师以伐吴,不为军政,无功而还。

齐侯既伐晋而惧,将欲见楚子。楚子使薳启强如齐聘,且请期。齐社,搜军实,使客观之。陈文子曰:``齐将有寇。吾闻之,兵不戢,必取其族。」

秋,齐侯闻将有晋师,使陈无宇从薳启强如楚,辞,且乞师。崔杼帅师送之,遂伐莒,侵介根。

会于夷仪,将以伐齐,水,不克。

冬,楚子伐郑以救齐,门于东门,次于棘泽。诸侯还救郑。晋侯使张骼、辅跞致楚师,求御于郑。郑人卜宛射犬,吉。子大叔戒之曰:``大国之人,不可与也。」对曰:``无有众寡,其上一也。」大叔曰:``不然,部娄无松柏。」二子在幄,坐射犬于外,既食而后食之。使御广车而行,己皆乘乘车。将及楚师,而后从之乘,皆踞转而鼓琴。近,不告而驰之。皆取胄于櫜而胄,入垒,皆下,搏人以投,收禽挟囚。弗待而出。皆超乘,抽弓而射。既免,复踞转而鼓琴,曰:``公孙!同乘,兄弟也。胡再不谋?」对曰:``曩者志入而已,今则怯也。」皆笑,曰:``公孙之亟也。」

楚子自棘泽还,使薳启强帅师送陈无宇。

吴人为楚舟师之役故,召舒鸠人,舒鸠人叛楚。楚子师于荒浦,使沈尹寿与师祁犁让之。舒鸠子敬逆二子,而告无之,且请受盟。二子覆命,王欲伐之。薳子曰:``不可。彼告不叛,且请受盟,而又伐之,伐无罪也。姑归息民,以待其卒。卒而不贰,吾又何求?若犹叛我,无辞有庸。」乃还。

陈人复讨庆氏之党,金咸宜咎出奔楚。

齐人城郏。穆叔如周聘,且贺城。王嘉其有礼也,赐之大路。

晋侯嬖程郑,使佐下军。郑行人公孙挥如晋聘。程郑问焉,曰:``敢问降阶何由?」子羽不能对。归以语然明,然明曰:``是将死矣。不然将亡。贵而知惧,惧而思降,乃得其阶,下人而已,又何问焉?且夫既登而求降阶者,知人也,不在程郑。其有亡衅乎?不然,其有惑疾,将死而忧也。」

\hypertarget{header-n2177}{%
\subsubsection{襄公二十五年}\label{header-n2177}}

【经】二十有五年春,齐崔杼帅师伐我北鄙。夏五月乙亥,齐崔杼弑其君光。公会晋侯、宋公、卫侯、郑伯、曹伯、莒子、邾子、滕子、薛伯、杞伯、小邾子于夷仪。六月壬子,郑公孙舍之帅师入陈。秋八月己巳,诸侯同盟于重丘。公至自会。卫侯入于夷仪。楚屈建帅师灭舒鸠。冬,郑公孙夏帅师伐陈。十有二月,吴子遏伐楚,门于巢,卒。

【传】二十五年春,齐崔杼帅师伐我北鄙,以报孝伯之师也。公患之,使告于晋。孟公绰曰:``崔子将有大志,不在病我,必速归,何患焉!其来也不寇,使民不严,异于他日。」齐师徒归。

齐棠公之妻,东郭偃之姊也。东郭偃臣崔武子。棠公死,偃御武子以吊焉。见棠姜而美之,使偃取之。偃曰:``男女辨姓,今君出自丁,臣出自桓,不可。」武子筮之,遇《困》三之《大过》三。史皆曰:``吉。」示陈文子,文子曰:``夫从风,风陨,妻不可娶也。且其《繇》曰:『困于石,据于蒺藜,入于其宫,不见其妻,凶。』困于石,往不济也。据于蒺藜,所恃伤也。入于其宫,不见其妻,凶,无所归也。」崔子曰:``嫠也何害?先夫当之矣。」遂取之。庄公通焉,骤如崔氏。以崔子之冠赐人,侍者曰:``不可。」公曰:``不为崔子,其无冠乎?」崔子因是,又以其间伐晋也,曰:``晋必将报。」欲弑公以说于晋,而不获间。公鞭侍人贾举而又近之,乃为崔子间公。

夏五月,莒为且于之役故,莒子朝于齐。甲戌,飨诸北郭。崔子称疾,不视事。乙亥,公问崔子,遂从姜氏。姜入于室,与崔子自侧户出。公拊楹而歌。侍人贾举止众从者,而入闭门。甲兴,公登台而请,弗许;请盟,弗许;请自刃于庙,勿许。皆曰:``君之臣杼疾病,不能听命。近于公宫,陪臣干掫有淫者,不知二命。」公逾墙。又射之,中股,反队,遂弑之。贾举,州绰、邴师、公孙敖、封具、铎父、襄伊、偻堙皆死。祝佗父祭于高唐,至,覆命。不说弁而死于崔氏。申蒯侍渔者,退,谓其宰曰:``尔以帑免,我将死。」其宰曰:``免,是反子之义也。」与之皆死。崔氏杀融蔑于平阴。

晏子立于崔氏之门外,其人曰:``死乎?」曰:``独吾君也乎哉?吾死也。」曰:``行乎?」曰:``吾罪也乎哉?吾亡也。」``归乎?」曰:``君死,安归?君民者,岂以陵民?社稷是主。臣君者,岂为其口实,社稷是养。故君为社稷死,则死之;为社稷亡,则亡之。若为己死而为己亡,非其私昵,谁敢任之?且人有君而弑之,吾焉得死之,而焉得亡之?将庸何归?」门启而入,枕尸股而哭。兴,三踊而出。人谓崔子:``必杀之!」崔子曰:``民之望也!舍之,得民。」卢蒲癸奔晋,王何奔莒。

叔孙宣伯之在齐也,叔孙还纳其女于灵公。嬖,生景公。丁丑,崔杼立而相之。庆封为左相。盟国人于大宫,曰:``所不与崔、庆者。」晏子仰天叹曰:``婴所不唯忠于君利社稷者是与,有如上帝。」乃歃。辛巳,公与大夫及莒子盟。

大史书曰:``崔杼弑其君。」崔子杀之。其弟嗣书而死者,二人。其弟又书,乃舍之。南史氏闻大史尽死,执简以往。闻既书矣,乃还。

闾丘婴以帷缚其妻而栽之,与申鲜虞乘而出,鲜虞推而下之,曰:``君昏不能匡,危不能救,死不能死,而知匿其昵,其谁纳之?」行及弇中,将舍。婴曰:``崔、庆其追我!」鲜虞曰:``一与一,谁能惧我?」遂舍,枕辔而寝,食马而食。驾而行,出弇中,谓婴曰:``速驱这!崔、庆之众,不可当也。」遂来奔。

崔氏侧庄公于北郭。丁亥,葬诸士孙之里,四翣,不跸,下车七乘,不以兵甲。

晋侯济自泮,会于夷仪,伐齐,以报朝歌之役。齐人以庄公说,使隰鉏请成。庆封如师,男女以班。赂晋侯以宗器、乐器。自六正、五吏、三十帅、三军之大夫、百官之正长、师旅及处守者,皆有赂。晋侯许之。使叔向告于诸侯。公使子服惠伯对曰:``君舍有罪,以靖小国,君之惠也。寡君闻命矣!」

晋侯使魏舒、宛没逆卫侯,将使卫与之夷仪。崔子止其帑,以求五鹿。

初,陈侯会楚子伐郑,当陈隧者,井堙木刊。郑人怨之,六月,郑子展、子产帅车七百乘伐陈,宵突陈城,遂入之。陈侯扶其大子偃师奔墓,遇司马桓子,曰:``载余!」曰:``将巡城。」遇贾获,载其母妻,下之,而授公车。公曰:``舍而母!」辞曰:``不祥。」与其妻扶其母以奔墓,亦免。子展命师无入公宫,与子产亲御诸门。陈侯使司马桓子赂以宗器。陈侯免,拥社。使其众,男女别而累,以待于朝。子展执絷而见,再拜稽首,承饮而进献。子美入,数俘而出。祝祓社,司徒致民,司马致节,司空致地,乃还。

秋七月己巳,同盟于重丘,齐成故也。

赵文子为政,令薄诸侯之币而重其礼。穆叔见之,谓穆叔曰:``自今以往,兵其少弭矣!齐崔、庆新得政,将求善于诸侯。武也知楚令尹。若敬行其礼,道之以文辞,以靖诸侯,兵可以弭。」

楚薳子冯卒,屈建为令尹。屈荡为莫敖。舒鸠人卒叛楚。令尹子木伐之,及离城。吴人救之,子木遽以右师先,子强、息桓、子捷、子骈、子盂帅左师以退。吴人居其间七日。子强曰:``久将垫隘,隘乃禽也。不如速战!请以其私卒诱之,简师陈以待我。我克则进,奔则亦视之,乃可以免。不然,必为吴禽。」从之。五人以其私卒先击吴师。吴师奔,登山以望,见楚师不继,复逐之,傅诸其军。简师会之,吴师大败。遂围舒鸠,舒鸠溃。八月,楚灭舒鸠。

卫献公入于夷仪。

郑子产献捷于晋,戎服将事。晋人问陈之罪,对曰:``昔虞阏父为周陶正,以服事我先王。我先王赖其利器用也,与其神明之后也,庸以元女大姬配胡公,而封诸陈,以备三恪。则我周之自出,至于今是赖。桓公之乱,蔡人欲立其出。我先君庄公奉五父而立之,蔡人杀之。我又与蔡人奉戴厉公,至于庄、宣,皆我之自立。夏氏之乱,成公播荡,又我之自入,君所知也。今陈忘周之大德,蔑我大惠,弃我姻亲,介恃楚众,以凭陵我敝邑,不可亿逞。我是以有往年之告。未获成命,则有我东门之役。当陈隧者,井堙木刊。敝邑大惧不竟,而耻大姬。天诱其衷,启敝邑之心。陈知其罪,授手于我。用敢献功!」晋人曰:``何故侵小?」对曰:``先王之命,唯罪所在,各致其辟。且昔天子之地一圻,列国一同,自是以衰。今大国多数圻矣!若无侵小,何以至焉?」晋人曰:``何故戎服?」对曰:``我先君武、庄,为平、桓卿士。城濮之役,文公布命,曰:『各复旧职!』命我文公戎服辅王,以授楚捷,不敢废王命故也。」士庄伯不能诘,复于赵文子。文子曰:``其辞顺,犯顺不祥。」乃受之。

冬十月,子展相郑伯如晋,拜陈之功。子西复伐陈,陈及郑平。仲尼曰:``《志》有之:『言以足志,文以足言。』不言,谁知其志?言之无文,行而不远。晋为伯,郑入陈,非文辞不为功。慎辞也!」

楚蒍掩为司马,子木使庀赋,数甲兵。甲午,蒍掩书土田,度山林,鸠薮泽,辨京陵,表淳卤,数疆潦,规偃猪,町原防,牧隰皋,井衍沃,量入修赋。赋车籍马,赋车兵、徒卒、甲楯之数。既成,以授子木,礼也。

十二月,吴子诸樊伐楚,以报舟师之役。门于巢。巢牛臣曰:``吴王勇而轻,若启之,将亲门。我获射之,必殪。是君也死,强其少安!」从之。吴子门焉,牛臣隐于短墙以射之,卒。

楚子以灭舒鸠赏子木。辞曰:``先大夫蒍子之功也。」以与蒍掩。

晋程郑卒。子产始知然明,问为政焉。对曰:``视民如子。见不仁者诛之,如鹰鸇之逐鸟雀也。」子产喜,以语子大叔,且曰:``他日吾见蔑之面而已,今吾见其心矣。」子大叔问政于子产。子产曰:``政如农功,日夜思之,思其始而成其终。朝夕而行之,行无越思,如农之有畔。其过鲜矣。」

卫献公自夷仪使与宁喜言,宁喜许之。大叔文子闻之,曰:``乌乎!《诗》所谓『我躬不说,皇恤我后』者,宁子可谓不恤其后矣。将可乎哉?殆必不可。君子之行,思其终也,思其复也。《书》曰:『慎始而敬终,终以不困。』《诗》曰:『夙夜匪解,以事一人。』今宁子视君不如弈棋,其何以免乎?弈者举棋不定,不胜其耦。而况置君而弗定乎?必不免矣。九世之卿族,一举而灭之。可哀也哉!」

会于夷仪之岁,齐人城郏。其五月,秦、晋为成。晋韩起如秦莅盟,秦伯车如晋莅盟,成而不结。

\hypertarget{header-n2204}{%
\subsubsection{襄公二十六年}\label{header-n2204}}

【经】二十有六年春王二月辛卯,卫宁喜弑其君剽。卫孙林父入于戚以叛。甲午,卫侯衎复归于卫。夏,晋侯使荀吴来聘。公会晋人、郑良霄、宋人、曹人于澶渊。秋,宋公弑其世子痤。晋人执卫宁喜。八月壬午,许男宁卒于楚。冬,楚子、蔡侯、陈侯伐郑。葬许灵公。

【传】二十六年春,秦伯之弟金咸如晋修成,叔向命召行人子员。行人子朱曰:``朱也当御。」三云,叔向不应。子朱怒,曰:``班爵同,何以黜朱于朝?」抚剑从之。叔向曰:``秦、晋不和久矣!今日之事,幸而集,晋国赖之。不集,三军暴骨。子员道二国之言无私,子常易之。奸以事君者,吾所能御也。」拂衣从之。人救之。平公曰:``晋其庶乎!吾臣之所争者大。」师旷曰:``公室惧卑。臣不心竞而力争,不务德而争善,私欲已侈,能无卑乎?」

卫献公使子鲜为复,辞。敬姒强命之。对曰:``君无信,臣惧不免。」敬姒曰:``虽然,以吾故也。」许诺。初,献公使与宁喜言,宁喜曰:``必子鲜在,不然必败。」故公使子鲜。子鲜不获命于敬姒,以公命与宁喜言,曰:``苟反,政由宁氏,祭则寡人。」宁喜告蘧伯玉,伯玉曰:``瑗不得闻君之出,敢闻其入?」遂行,从近关出。告右宰谷,右宰谷曰:``不可。获罪于两君,天下谁畜之?」悼子曰:``吾受命于先人,不可以贰。」谷曰:``我请使焉而观之。」遂见公于夷仪。反曰:``君淹恤在外十二年矣,而无忧色,亦无宽言,犹夫人也。若不已,死无日矣。」悼子曰:``子鲜在。」右宰谷曰:``子鲜在,何益?多而能亡,于我何为?」悼子曰:``虽然,不可以已。」孙文子在戚,孙嘉聘于齐,孙襄居守。

二月庚寅,宁喜、右宰谷伐孙氏,不克。伯国伤。宁子出舍于郊。伯国死,孙氏夜哭。国人召宁子,宁子复攻孙氏,克之。辛卯,杀子叔及大子角。书曰:``宁喜弑其君剽。」言罪之在宁氏也。孙林父以戚如晋。书曰:``入于戚以叛。」罪孙氏也。臣之禄,君实有之。义则进,否则奉身而退,专禄以周旋,戮也。

甲午,卫侯入。书曰:``复归。」国纳之也。大夫逆于竟者,执其手而与之言。道逆者,自车揖之。逆于门者,颔之而已。公至,使让大叔文子曰:``寡人淹恤在外,二三子皆使寡人朝夕闻卫国之言,吾子独不在寡人。古人有言曰:『非所怨勿怨。』寡人怨矣。」对曰:``臣知罪矣!臣不佞不能负羁泄,以从手干牧圉,臣之罪一也。有出者,有居者。臣不能贰,通外内之言以事君,臣之罪二也。有二罪,敢忘其死?」乃行,从近关出。公使止之。

卫人侵戚东鄙,孙氏愬于晋,晋戍茅氏。殖绰伐茅氏,杀晋戍三百人。孙蒯追之,弗敢击。文子曰:``厉之不如!」遂从卫师,败之圉。雍鉏获殖绰。复愬于晋。

郑伯赏入陈之功。三月甲寅朔,享子展,赐之先路,三命之服,先八邑。赐子产次路,再命之服,先六邑。子产辞邑,曰:``自上以下,隆杀以两,礼也。臣之位在四,且子展之功也。臣不敢及及赏礼,请辞邑。」公固予之,乃受三邑。公孙挥曰:``子产其将知政矣!让不失礼。」

晋人为孙氏故,召诸侯,将以讨卫也。夏,中行穆子来聘,召公也。

楚子、秦人侵吴,及雩娄,闻吴有备而还。遂侵郑,五月,至于城麇。郑皇颉戍之,出,与楚师战,败。穿封戌囚皇颉,公子围与之争之。正于伯州犁,伯州犁曰:``请问于囚。」乃立囚。伯州犁曰:``所争,君子也,其何不知?」上其手,曰:``夫子为王子围,寡君之贵介弟也。」下其手,曰:``此子为穿封戌,方城外之县尹也。谁获子?」囚曰:``颉遇王子,弱焉。」戌怒,抽戈逐王子围,弗及。楚人以皇颉归。

印堇父与皇颉戍城麇,楚人囚之,以献于秦。郑人取货于印氏以请之,子大叔为令正,以为请。子产曰:``不获。受楚之功而取货于郑,不可谓国,秦不其然。若曰:『拜君之勤郑国,微君之惠,楚师其犹在敝邑之城下。』其可。」弗从,遂行。秦人不予。更币,从子产而后获之。

六月,公会晋赵武、宋向戌、郑良霄、曹人于澶渊以讨卫,疆戚田。取卫西鄙懿氏六十以与孙氏。赵武不书,尊公也。向戌不书,后也。郑先宋,不失所也。于是卫侯会之。晋人执宁喜、北宫遗,使女齐以先归。卫侯如晋,晋人执而囚之于士弱氏。

秋七月,齐侯、郑伯为卫侯故,如晋,晋侯兼享之。晋侯赋《嘉乐》。国景子相齐侯,赋《蓼萧》。子展相郑伯,赋《缁衣》。叔向命晋侯拜二君曰:``寡君敢拜齐君之安我先君之宗祧也,敢拜郑君之不贰也。」国子使晏平仲私于叔向,曰:``晋君宣其明德于诸侯,恤其患而补其阙,正其违而治其烦,所以为盟主也。今为臣执君,若之何?」叔向告赵文子,文子以告晋侯。晋侯言卫侯之罪,使叔向告二君。国子赋《辔之柔矣》,子展赋《将仲子兮》,晋侯乃许归卫侯。叔向曰:``郑七穆,罕氏其后亡者也。子展俭而壹。」

初,宋芮司徒生女子,赤而毛,弃诸堤下,共姬之妾取以入,名之曰弃。长而美。平公入夕,共姬与之食。公见弃也,而视之,尤。姬纳诸御,嬖,生佐。恶而婉。大子痤美而很,合左师畏而恶之。寺人惠墙伊戾为大子内师而无宠。

秋,楚客聘于晋,过宋。大子知之,请野享之。公使往,伊戾请从之。公曰:``夫不恶女乎?」对曰:``小人之事君子也,恶之不敢远,好之不敢近。敬以待命,敢有贰心乎?纵有共其外,莫共其内,臣请往也。」遣之。至,则□欠,用牲,加书,征之,而聘告公曰:``大子将为乱,既与楚客盟矣。」公曰:``为我子,又何求?」对曰:``欲速。」公使视之,则信有焉。问诸夫人与左师,则皆曰:``固闻之。」公囚大子。大子曰:``唯佐也能免我。」召而使请,曰:``日中不来,吾知死矣。」左师闻之,聒而与之语。过期,乃缢而死。佐为大子。公徐闻其无罪也,乃亨伊戾。

左师见夫人之步马者,问之,对曰:``君夫人氏也。」左师曰:``谁为君夫人?余胡弗知?」圉人归,以告夫人。夫人使馈之锦与马,先之以玉,曰:``君之妾弃使某献。」左师改命曰:``君夫人。」而后再拜稽首受之。

郑伯归自晋,使子西如晋聘,辞曰:``寡君来烦执事,惧不免于戾,使夏谢不敏。」君子曰:``善事大国。」

初,楚伍参与蔡太师子朝友,其子伍举与声子相善也。伍举娶于王子牟,王子牟为申公而亡,楚人曰:``伍举实送之。」伍举奔郑,将遂奔晋。声子将如晋,遇之于郑郊,班荆相与食,而言复故。声子曰:``子行也!吾必复子。」及宋向戌将平晋、楚,声子通使于晋。还如楚,令尹子木与之语,问晋故焉,且曰:``晋大夫与楚孰贤?」对曰:``晋卿不如楚,其大夫则贤,皆卿材也。如杞、梓、皮革,自楚往也。虽楚有材,晋实用之。」子木曰:``夫独无族姻乎?」对曰:``虽有,而用楚材实多。归生闻之:『善为国者,赏不僭而刑不滥。』赏僭,则惧及淫人;刑滥,则惧及善人。若不幸而过,宁僭无滥。与其失善,宁其利淫。无善人,则国从之。《诗》曰:『人之云亡,邦国殄瘁。』无善人之谓也。故《夏书》曰:『与其杀不幸,宁失不经。』惧失善也。《商颂》有之曰:『不僭不滥,不敢怠皇,命于下国,封建厥福。』此汤所以获天福也。古之治民者,劝赏而畏刑,恤民不倦。赏以春夏,刑以秋冬。是以将赏,为之加膳,加膳则饫赐,此以知其劝赏也。将刑,为之不举,不举则彻乐,此以知其畏刑也。夙兴夜寐,朝夕临政,此以知其恤民也。三者,礼之大节也。有礼无败。今楚多淫刑,其大夫逃死于四方,而为之谋主,以害楚国,不可救疗,所谓不能也。子仪之乱,析公奔晋。晋人置诸戎车之殿,以为谋主。绕角之役,晋将遁矣,析公曰:『楚师轻窕,易震荡也。若多鼓钧声,以夜军之,楚师必遁。』晋人从之,楚师宵溃。晋遂侵蔡,袭沈,获其君;败申、息之师于桑隧,获申丽而还。郑于是不敢南面。楚失华夏,则析公之为也。雍子之父兄谮雍子,君与大夫不善是也。雍子奔晋。晋人与之鄐,以为谋主。彭城之役,晋、楚遇于靡角之谷。晋将遁矣。雍子发命于军曰:『归老幼,反孤疾,二人役,归一人,简兵搜乘,秣马蓐食,师陈焚次,明日将战。』行归者而逸楚囚,楚师宵溃。晋绛彭城而归诸宋,以鱼石归。楚失东夷,子辛死之,则雍子之为也。子反与子灵争夏姬,而雍害其事,子灵奔晋。晋人与之邢,以为谋主。扞御北狄,通吴于晋,教吴判楚,教之乘车、射御、驱侵,使其子孤庸为吴行人焉。吴于是伐巢、取驾、克棘、入州来,楚罢于奔命,至今为患,则子灵之为也。若敖之乱,伯贲之子贲皇奔晋。晋人与之苗,以为谋主。鄢陵之役,楚晨压晋军而陈,晋将遁矣。苗贲皇曰:『楚师之良,在其中军王族而已。若塞井夷灶,成陈以当之,栾、范易行以诱之,中行、二郤必克二穆。吾乃四萃于其王族,必大败之。』晋人从之,楚师大败,王夷师熠,子反死之。郑叛吴兴,楚失诸侯,则苗贲皇之为也。」子木曰:``是皆然矣。」声子曰:``今又有甚于此。椒举娶于申公子牟,子牟得戾而亡,君大夫谓椒举:『女实遣之!』惧而奔郑,引领南望曰:『庶几赦余!』亦弗图也。今在晋矣。晋人将与之县,以比叔向。彼若谋害楚国,岂不为患?」子木惧,言诸王,益其禄爵而复之。声子使椒鸣逆之。

许灵公如楚,请伐郑,曰:``师不兴,孤不归矣!」八月,卒于楚。楚子曰:``不伐郑,何以求诸侯?」冬十月,楚子伐郑。郑人将御之,子产曰:``晋、楚将平,诸侯将和,楚王是故昧于一来。不如使逞而归,乃易成也。夫小人之性,衅于勇,啬于祸,以足其性而求名焉者,非国家之利也。若何从之?」子展说,不御寇。十二月乙酉,入南里,堕其城。涉于乐氏,门于师之梁。县门发,获九人焉。涉入汜而归,而后葬许灵公。

卫人归卫姬于晋,乃释卫侯。君子是以知平公之失政也。

晋韩宣子聘于周。王使请事。对曰:``晋士起将归时事于宰旅,无他事矣。」王闻之曰:``韩氏其昌阜于晋乎!辞不失旧。」

齐人城郏之岁,其夏,齐乌余以廪丘奔晋,袭卫羊角,取之;遂袭我高鱼。有大雨,自其窦入,介于其库,以登其城,克而取之。又取邑于宋。于是范宣子卒,诸侯弗能治也,及赵文子为政,乃卒治之。文子言于晋侯曰:``晋为盟主。诸侯或相侵也,则讨而使归其地。今乌余之邑,皆讨类也,而贪之,是无以为盟主也。请归之!」公曰:``诺。孰可使也?」对曰:``胥梁带能无用师。」晋侯使往。

\hypertarget{header-n2228}{%
\subsubsection{襄公二十七年}\label{header-n2228}}

【经】二十有七春,齐侯使庆封聘。夏,叔孙豹会晋赵武、楚屈建、蔡公孙归生、卫石恶、陈孔奂、郑良霄、许人、曹人于宋。卫杀其大夫宁喜。卫侯之弟鱄出奔晋。秋七月辛巳,豹及诸侯之大夫盟于宋。冬十有二月乙卯朔,日有食之。

【传】二十七年春,胥梁带使诸丧邑者具车徒以受地,必周。使乌余车徒以受封,乌余以众出。使诸侯伪效乌余之封者,而遂执之,尽获之。皆取其邑而归诸侯,诸侯是以睦于晋。

齐庆封来聘,其车美。孟孙谓叔孙曰:``庆季之车,不亦美乎?」叔孙曰:``豹闻之:『服美不称,必以恶终。』美车何为?」叔孙与庆封食,不敬。为赋《相鼠》,亦不知也。卫宁喜专,公患之。公孙免余请杀之。公曰:``微宁子不及此,吾与之言矣。事未可知,只成恶名,止也。」对曰:``臣杀之,君勿与知。」乃与公孙无地、公孙臣谋,使攻宁氏。弗克,皆死。公曰:``臣也无罪,父子死余矣!」夏,免余复攻宁氏,杀宁喜及右宰谷,尸诸朝。石恶将会宋之盟,受命而出。衣其尸,枕之股而哭之。欲敛以亡,惧不免,且曰:``受命矣。」乃行。

子鲜曰:``逐我者出,纳我者死,赏罚无章,何以沮劝?君失其信,而国无刑。不亦难乎!且鱄实使之。」遂出奔晋。公使止之,不可。及河,又使止之。止使者而盟于河,托于木门,不乡卫国而坐。木门大夫劝之仕,不可,曰:``仕而废其事,罪也。从之,昭吾所以出也。将准愬乎?吾不可以立于人之朝矣。」终身不仕。公丧之,如税服,终身。

公与免余邑六十,辞曰:``唯卿备百邑,臣六十矣。下有上禄,乱也,臣弗敢闻。且宁子唯多邑,故死。臣惧死之速及也。」公固与之,受其半。以为少师。公使为卿,辞曰:``大叔仪不贰,能赞大事。君其命之!」乃使文子为卿。

宋向戌善于赵文子,又善于令尹子木,欲弭诸侯之兵以为名。如晋,告赵孟。赵孟谋于诸大夫,韩宣子曰:``兵,民之残也,财用之蠹,小国之大灾也。将或弭之,虽曰不可,必将许之。弗许,楚将许之,以召诸侯,则我失为盟主矣。」晋人许之。如楚,楚亦许之。如齐,齐人难之。陈文子曰:``晋、楚许之,我焉得已。且人曰弭兵,而我弗许,则固携吾民矣!将焉用之?」齐人许之。告于秦,秦亦许之。皆告于小国,为会于宋。

五月甲辰,晋赵武至于宋。丙午,郑良霄至。六月丁未朔,宋人享赵文子,叔向为介。司马置折俎,礼也。仲尼使举是礼也,以为多文辞。戊申,叔孙豹、齐庆封、陈须无、卫石恶至。甲寅,晋荀盈从赵武至。丙辰,邾悼公至。壬戌,楚公子黑肱先至,成言于晋。丁卯,宋戌如陈,从子木成言于楚。戊辰,滕成公至。子木谓向戌:``请晋、楚之从交相见也。」庚午,向戌复于赵孟。赵孟曰:``晋、楚、齐、秦,匹也。晋之不能于齐,犹楚之不能于秦也。楚君若能使秦君辱于敝邑,寡君敢不固请于齐?」壬申,左师复言于子木。子木使馹谒诸王,王曰:``释齐、秦,他国请相见也。」秋七月戊寅,左师至。是夜也,赵孟及子皙盟,以齐言。庚辰,子木至自陈。陈孔奂、蔡公孙归生至。曹、许之大夫皆至。以藩为军,晋、楚各处其偏。伯夙谓赵孟曰:``楚氛甚恶,惧难。」赵孟曰:``吾左还,入于宋,若我何?」

辛巳,将盟于宋西门之外,楚人衷甲。伯州犁曰:``合诸侯之师,以为不信,无乃不可乎?夫诸侯望信于楚,是以来服。若不信,是弃其所以服诸侯也。」固请释甲。子木曰:``晋、楚无信久矣,事利而已。苟得志焉,焉用有信?」大宰退,告人曰:``令尹将死矣,不及三年。求逞志而弃信,志将逞乎?志以发言,言以出信,信以立志,参以定之。信亡,何以及三?」赵孟患楚衷甲,以告叔向。叔向曰:``何害也?匹夫一为不信,犹不可,单毙其死。若合诸侯之卿,以为不信,必不捷矣。食言者不病,非子之患也。夫以信召人,而以僭济之。必莫之与也,安能害我?且吾因宋以守病,则夫能致死,与宋致死,虽倍楚可也。子何惧焉?又不及是。曰弭兵以召诸侯,而称兵以害我,吾庸多矣,非所患也。」

季武子使谓叔孙以公命,曰:``视邾、滕。」既而齐人请邾,宋人请滕,皆不与盟。叔孙曰:``邾、滕,人之私也;我,列国也,何故视之?宋、卫,吾匹也。」乃盟。故不书其族,言违命也。

晋、楚争先。晋人曰:``晋固为诸侯盟主,未有先晋者也。」楚人曰:``子言晋、楚匹也,若晋常先,是楚弱也。且晋、楚狎主诸侯之盟也久矣!岂专在晋?」叔向谓赵孟曰:``诸侯归晋之德只,非归其尸盟也。子务德,无争先!且诸侯盟,小国固必有尸盟者。楚为晋细,不亦可乎?」乃先楚人。书先晋,晋有信也。

壬午,宋公兼享晋、楚之大夫,赵孟为客。子木与之言,弗能对。使叔向侍言焉,子木亦不能对也。

乙酉,宋公及诸侯之大夫盟于蒙门之外。子木问于赵孟曰:``范武子之德何如?」对曰:``夫人之家事治,言于晋国无隐情。其祝史陈信于鬼神,无愧辞。」子木归,以语王。王曰:``尚矣哉!能歆神人,宜其光辅五君以为盟主也。」子木又语王曰:``宜晋之伯也!有叔向以佐其卿,楚无以当之,不可与争。」晋荀寅遂如楚莅盟。

郑伯享赵孟于垂陇,子展、伯有、子西、子产、子大叔、二子石从。赵孟曰:``七子从君,以宠武也。请皆赋以卒君贶,武亦以观七子之志。」子展赋《草虫》,赵孟曰:``善哉!民之主也。抑武也不足以当之。」伯有赋《鹑之贲贲》,赵孟曰:``床第之言不逾阈,况在野乎?非使人之所得闻也。」子西赋《黍苗》之四章,赵孟曰:``寡君在,武何能焉?」子产赋《隰桑》,赵孟曰:``武请受其卒章。」子大叔赋《野有蔓草》,赵孟曰:``吾子之惠也。」印段赋《蟋蟀》,赵孟曰:``善哉!保家之主也,吾有望矣!」公孙段赋《桑扈》,赵孟曰:``『匪交匪敖』,福将焉往?若保是言也,欲辞福禄,得乎?」卒享。文子告叔向曰:``伯有将为戮矣!诗以言志,志诬其上,而公怨之,以为宾荣,其能久乎?幸而后亡。」叔向曰:``然。已侈!所谓不及五稔者,夫子之谓矣。」文子曰:``其馀皆数世之主也。子展其后亡者也,在上不忘降。印氏其次也,乐而不荒。乐以安民,不淫以使之,后亡,不亦可乎?」

宋左师请赏,曰:``请免死之邑。」公与之邑六十。以示子罕,子罕曰:``凡诸侯小国,晋、楚所以兵威之。畏而后上下慈和,慈和而后能安靖其国家,以事大国,所以存也。无威则骄,骄则乱生,乱生必灭,所以亡也。天生五材,民并用之,废一不可,谁能去兵?兵之设久矣,所以威不轨而昭文德也。圣人以兴,乱人以废,废兴存亡昏明之术,皆兵之由也。而子求去之,不亦诬乎?以诬道蔽诸侯,罪莫大焉。纵无大讨,而又求赏,无厌之甚也!」削而投之。左师辞邑。向氏欲攻司城,左师曰:``我将亡,夫子存我,德莫大焉,又可攻乎?」君子曰:``『彼己之子,邦之司直。』乐喜之谓乎?『何以恤我,我其收之。』向戌之谓乎?」

齐崔杼生成及强而寡。娶东郭姜,生明。东郭姜以孤入,曰棠无咎,与东郭偃相崔氏。崔成有病,而废之,而立明。成请老于崔,崔子许之。偃与无咎弗予,曰:``崔,宗邑也,必在宗主。」成与强怒,将杀之。告庆封曰:``夫子之身亦子所知也,唯无咎与偃是从,父兄莫得进矣。大恐害夫子,敢以告。」庆封曰:``子姑退,吾图之。」告卢蒲弊。卢蒲弊曰:``彼,君之仇也。天或者将弃彼矣。彼实家乱,子何病焉!崔之薄,庆之厚也。」他日又告。庆封曰:``苟利夫子,必去之!难,吾助女。」

九月庚辰,崔成、崔强杀东郭偃、棠无咎于崔氏之朝。崔子怒而出,其众皆逃,求人使驾,不得。使圉人驾,寺人御而出。且曰:``崔氏有福,止余犹可。」遂见庆封。庆封曰:``崔、庆一也。是何敢然?请为子讨之。」使卢蒲弊帅甲以攻崔氏。崔氏堞其宫而守之,弗克。使国人助之,遂灭崔氏,杀成与强,而尽俘其家。其妻缢。弊覆命于崔子,且御而归之。至,则无归矣,乃缢。崔明夜辟诸大墓。辛巳,崔明来奔,庆封当国。

楚薳罢如晋莅盟,晋将享之。将出,赋《既醉》。叔向曰:``薳氏之有后于楚国也,宜哉!承君命,不忘敏。子荡将知政矣。敏以事君,必能养民。政其焉往?」

崔氏之乱,申鲜虞来奔,仆赁于野,以丧庄公。冬,楚人召之,遂如楚为右尹。

十一月乙亥朔,日有食之。辰在申,司历过也,再失闰矣。

\hypertarget{header-n2250}{%
\subsubsection{襄公二十八年}\label{header-n2250}}

【经】二十有八年春,无冰。夏,卫石恶出奔晋。邾子来朝。秋八月,大雩。仲孙羯如晋。冬,齐庆封来奔。十有一月,公如楚。十有二月甲寅,天王崩。乙未,楚子昭卒。

【传】二十八年春,无冰。梓慎曰:``今兹宋、郑其饥乎?岁在星纪,而淫于玄枵,以有时灾,阴不堪阳。蛇乘龙。龙,宋、郑之星也,宋、郑必饥。玄枵,虚中也。枵,秏名也。土虚而民秏,不饥何为?」

夏。齐侯、陈侯、蔡侯、北燕伯、杞伯、胡子、沈子、白狄朝于晋,宋之盟故也。齐侯将行,庆封曰:``我不与盟,何为于晋?」陈文子曰:``先事后贿,礼也。小事大,未获事焉,从之如志,礼也。虽不与盟,敢叛晋乎?重丘之盟,未可忘也。子其劝行!」

卫人讨宁氏之党,故石恶出奔晋。卫人立其从子圃以守石氏之祀,礼也。

邾悼公来朝,时事也。

秋八月,大雩,旱也。

蔡侯归自晋,入于郑。郑伯享之,不敬。子产曰:``蔡侯其不免乎?日其过此也,君使子展廷劳于东门之外,而傲。吾曰:『犹将更之。』今还,受享而惰,乃其心也。君小国事大国,而惰傲以为己心,将得死乎?若不免,必由其子。其为君也,淫而不父。侨闻之,如是者,恒有子祸。」

孟孝伯如晋,告将为宋之盟故如楚也。

蔡侯之如晋也,郑伯使游吉如楚。及汉,楚人还之,曰:``宋之盟,君实亲辱。今吾子来,寡君谓吾子姑还!吾将使馹奔问诸晋而以告。」子大叔曰:``宋之盟,君命将利小国,而亦使安定其社稷,镇抚其民人,以礼承天之休,此君之宪令,而小国之望也。寡君是故使吉奉其皮币,以岁之不易,聘于下执事。今执事有命曰:『女何与政令之有?必使而君弃而封守,跋涉山川,蒙犯霜露,以逞君心。』小国将君是望,敢不唯命是听。无乃非盟载之言,以阙君德,而执事有不利焉,小国是惧。不然,其何劳之敢惮?」子大叔归,覆命,告子展曰:``楚子将死矣!不修其政德,而贪昧于诸侯,以逞其愿,欲久,得乎?《周易》有之,在《复》三之《颐》三,曰:『迷复,凶。』其楚子之谓乎?欲复其愿,而弃其本,复归无所,是谓迷复。能无凶乎?君其往也!送葬而归,以快楚心。楚不几十年,未能恤诸侯也。吾乃休吾民矣。」裨灶曰:``今兹周王及楚子皆将死。岁弃其次,而旅于明年之次,以害鸟帑。周、楚恶之。」

九月,郑游吉如晋,告将朝于楚,以从宋之盟。子产相郑伯以如楚,舍不为坛。外仆言曰:``昔先大夫相先君,适四国,未尝不为坛。自是至今,亦皆循之。今子草舍,无乃不可乎?」子产曰:``大适小,则为坛。小适大,苟舍而已,焉用坛?侨闻之,大适小有五美:宥其罪戾,赦其过失,救其灾患,赏其德刑,教其不及。小国不困,怀服如归。是故作坛以昭其功,宣告后人,无怠于德。小适大有五恶:说其罪戾,请其不足,行其政事,共某职贡,从其时命。不然,则重其币帛,以贺其福而吊其凶,皆小国之祸也。焉用作坛以昭其祸?所以告子孙,无昭祸焉可也。」

齐庄封好田而耆酒,与庆舍政。则以其内实迁于卢蒲弊氏,易内而饮酒。数日,国迁朝焉。使诸亡人得贼者,以告而反之,故反卢蒲癸。癸臣子之,有宠,妻之。庆舍之士谓卢蒲癸曰:``男女辨姓。子不辟宗,何也?」曰:``宗不馀辟,余独焉辟之?赋诗断章,余取所求焉,恶识宗?」癸言王何而反之,二人皆嬖,使执寝戈,而先后之。

公膳,日双鸡。饔人窃更之以鹜。御者知之,则去其肉而以其洎馈。子雅、子尾怒。庆封告卢蒲弊。卢蒲弊曰;``譬之如禽兽,吾寝处之矣。」使析归父告晏平仲。平仲曰:``婴之众不足用也,知无能谋也。言弗敢出,有盟可也。」子家曰:``子之言云,又焉用盟?」告北郭子车。子车曰:``人各有以事君,非佐之所能也。」陈文子谓桓子曰:``祸将作矣!吾其何得?」对曰:``得庆氏之木百车于庄。」文子曰:``可慎守也已!」

卢蒲癸、王何卜攻庆氏,示子之兆,曰:``或卜攻仇,敢献其兆。」子之曰:``克,见血。」冬十月,庆封田于莱,陈无宇从。丙辰,文子使召之。请曰:``无宇之母疾病,请归。」庆季卜之,示之兆,曰:``死。」奉龟而泣。乃使归。庆嗣闻之,曰:``祸将作矣!谓子家:``速归!祸作必于尝,归犹可及也。」子家弗听,亦无悛志。子息曰:``亡矣!幸而获在吴、越。」陈无宇济水而戕舟发梁。卢蒲姜谓癸曰:``有事而不告我,必不捷矣。」癸告之。姜曰:``夫子愎,莫之止,将不出,我请止之。」癸曰:``诺。」十一月乙亥,尝于大公之庙,庆舍莅事。卢蒲姜告之,且止之。弗听,曰:``谁敢者!」遂如公。麻婴为尸,庆圭为上献。卢蒲癸、王何执寝戈。庆氏以其甲环公宫。陈氏、鲍氏之圉人为优。庆氏之马善惊,士皆释甲束马而饮酒,且观优,至于鱼里。栾、高、陈、鲍之徒介庆氏之甲。子尾抽桷击扉三,卢蒲癸自后刺子之,王何以戈击之,解其左肩。犹援庙桷,动于甍,以俎壶投,杀人而后死。遂杀庆绳、麻婴。公惧,鲍国曰:``群臣为君故也。」陈须无以公归,税服而如内宫。

庆封归,遇告乱者,丁亥,伐西门,弗克。还伐北门,克之。入,伐内宫,弗克。反,陈于岳,请战,弗许。遂来奔。献车于季武子,美泽可以鉴。展庄叔见之,曰:``车甚泽,人必瘁,宜其亡也。」叔孙穆子食庆封,庆封汜祭。穆子不说,使工为之诵《茅鸱》,亦不知。既而齐人来让,奔吴。吴句余予之朱方,聚其族焉而居之,富于其旧。子服惠伯谓叔孙曰:``天殆富淫人,庆封又富矣。」穆子曰:``善人富谓之赏,淫人富谓之殃。天其殃之也,其将聚而歼旃?」

癸巳,天王崩。未来赴,亦未书,礼也。

崔氏之乱,丧群公子。故鉏在鲁,叔孙还在燕,贾在句渎之丘。及庆氏亡,皆召之,具其器用而反其邑焉。与晏子邶殿,其鄙六十,弗受。子尾曰:``富,人之所欲也,何独弗欲?」对曰:``庆氏之邑足欲,故亡。吾邑不足欲也。益之以邶殿,乃足欲。足欲,亡无日矣。在外,不得宰吾一邑。不受邶殿,非恶富也,恐失富也。且夫富如布帛之有幅焉,为之制度,使无迁也。夫民生厚而用利,于是乎正德以幅之,使无黜嫚,谓之幅利。利过则为败。吾不敢贪多,所谓幅也。」与北郭佐邑六十,受之。与子雅邑,辞多受少。与子尾邑,受而稍致之。公以为忠,故有宠。

释卢蒲弊于北竟。求崔杼之尸,将戮之,不得。叔孙穆子曰:``必得之。武王有乱臣十人,崔杼其有乎?不十人,不足以葬。」既,崔氏之臣曰:``与我其拱璧,吾献其柩。」于是得之。十二月乙亥朔,齐人迁庄公,殡于大寝。以其棺尸崔杼于市,国人犹知之,皆曰:``崔子也。」

为宋之盟故,公及宋公、陈侯、郑伯、许男如楚。公过郑,郑伯不在。伯有廷劳于黄崖,不敬。穆叔曰:``伯有无戾于郑,郑必有大咎。敬,民之主也,而弃之,何以承守?郑人不讨,必受其辜,济泽之阿,行潦之苹藻,置诸宗室,季兰尸之,敬也。敬可弃乎?」

及汉,楚康王卒。公欲反,叔仲昭伯曰:``我楚国之为,岂为一人?行也!」子服惠伯曰:``君子有远虑,小人从迩。饥寒之不恤,谁遑其后?不如姑归也。」叔孙穆子曰:``叔仲子专之矣,子服子始学者也。」荣成伯曰:``远图者,忠也。」公遂行。宋向戌曰:``我一人之为,非为楚也。饥寒之不恤,谁能恤楚?姑归而息民,待其立君而为之备。」宋公遂反。

楚屈建卒。赵文子丧之如同盟,礼也。

王人来告丧,问崩日,以甲寅告,故书之,以征过也。

\hypertarget{header-n2274}{%
\subsubsection{襄公二十九年}\label{header-n2274}}

【经】二十有九年春王正月,公在楚。夏五月,公至自楚。庚午,卫侯衎卒,阍弑吴子余祭。仲孙羯会晋荀盈、齐高止、宋华定、卫世叔仪、郑公孙段、曹人、莒人、滕子、薛人、小邾人城杞。晋侯使士鞅来聘。杞子来盟。吴子使札来聘。秋九月,葬卫献公。齐高止出奔北燕。冬,仲孙羯如晋。

【传】二十九年春,王正月,公在楚,释不朝正于庙也。楚人使公亲襚,公患之。穆叔曰:``祓殡而襚,则布币也。」乃使巫以桃列先祓殡。楚人弗禁,既而悔之。

二月癸卯,齐人葬庄公于北郭。

夏四月,葬楚康王。公及陈侯、郑伯、许男送葬,至于西门之外。诸侯之大夫皆至于墓。楚郏敖即位。王子围为令尹。郑行人子羽曰:``是谓不宜,必代之昌。松柏之下,其草不殖。」

公还,及方城。季武子取卞,使公冶问,玺书追而与之,曰:``闻守卞者将叛,臣帅徒以讨之,既得之矣,敢告。」公冶致使而退,及舍而后闻取卞。公曰:``欲之而言叛,只见疏也。」公谓公冶曰:``吾可以入乎?」对曰:``君实有国,谁敢违君!」公与公冶冕服。固辞,强之而后受。公欲无入,荣成伯赋《式微》,乃归。五月,公至自楚。公冶致其邑于季氏,而终不入焉。曰:``欺其君,何必使余?」季孙见之,则言季氏如他日。不见,则终不言季氏。及疾,聚其臣,曰:``我死,必以在冕服敛,非德赏也。且无使季氏葬我。」

葬灵王,郑上卿有事,子展使印段往。伯有曰:``弱,不可。」子展曰:``与其莫往,弱不犹愈乎?《诗》云:『王事靡盬,不遑启处,东西南北,谁敢宁处?坚事晋、楚,以蕃王室也。王事无旷,何常之有?」遂使印段如周。

吴人伐越,获俘焉,以为阍,使守舟。吴子余祭观舟,阍以刀弑之。

郑子展卒,子皮即位。于是郑饥而未及麦,民病。子皮以子展之命,饩国人粟,户一钟,是以得郑国之民。故罕氏常掌国政,以为上卿。宋司城子罕闻之,曰:``邻于善,民之望也。」宋亦饥,请于平公,出公粟以贷。使大夫皆贷。司城氏贷而不书,为大夫之无者贷。宋无饥人。叔向闻之,曰:``郑之罕,宋之乐,其后亡者也!二者其皆得国乎!民之归也。施而不德,乐氏加焉,其以宋升降乎!」

晋平公,杞出也,故治杞。六月,知悼子合诸侯之大夫以城杞,孟孝伯会之。郑子大叔与伯石往。子大叔见大叔文子,与之语。文子曰:``甚乎!其城杞也。」子大叔曰:``若之何哉?晋国不恤周宗之阙,而夏肄是屏。其弃诸姬,亦可知也已。诸姬是弃,其谁归之?吉也闻之,弃同即异,是谓离德。《诗》曰:『协比其邻,昏姻孔云。』晋不邻矣,其谁云之?」

齐高子容与宋司徒见知伯,女齐相礼。宾出,司马侯言于知伯曰:``二子皆将不免。子容专,司徒移,皆亡家之主也。」知伯曰:``何如?」对曰:``专则速及,侈将以其力毙,专则人实毙之,将及矣。」

范献子来聘,拜城杞也。公享之,展庄叔执币。射者三耦,公臣不足,取于家臣,家臣:展瑕、展玉父为一耦。公臣,公巫召伯、仲颜庄叔为一耦,鄫鼓父、党叔为一耦。

晋侯使司马女叔侯来治杞田,弗尽归也。晋悼夫人愠曰:``齐也取货。先君若有知也,不尚取之!」公告叔侯,叔侯曰:``虞、虢、焦、滑、霍、扬、韩、魏,皆姬姓也,晋是以大。若非侵小,将何所取?武、献以下,兼国多矣,谁得治之?杞,夏余也,而即东夷。鲁,周公之后也,而睦于晋。以杞封鲁犹可,而何有焉?鲁之于晋也,职贡不乏,玩好时至,公卿大夫相继于朝,史不绝书,府无虚月。如是可矣,何必瘠鲁以肥杞?且先君而有知也,毋宁夫人,而焉用老臣?」

杞文公来盟。书曰``子」,贱之也。

吴公子札来聘,见叔孙穆子,说之。谓穆子曰:``子其不得死乎?好善而不能择人。吾闻『君子务在择人』。吾子为鲁宗卿,而任其大政,不慎举,何以堪之?祸必及子!」

请观于周乐。使工为之歌《周南》、《召南》,曰:``美哉!始基之矣,犹未也。然勤而不怨矣。」为之歌《邶》、《鄘》、《卫》,曰:``美哉,渊乎!忧而不困者也。吾闻卫康叔、武公之德如是,是其《卫风》乎?」为之歌《王》,曰:``美哉!思而不惧,其周之东乎?」为之歌《郑》,曰:``美哉!其细已甚,民弗堪也,是其先亡乎!」为之歌《齐》,曰:``美哉!泱泱乎!大风也哉!表东海者,其大公乎!国未可量也。」为之歌《豳》,曰:``美哉!荡乎!乐而不淫,其周公之东乎?」为之歌《秦》,曰:``此之谓夏声。夫能夏则大,大之至也,其周之旧乎?」为之歌《魏》,曰:``美哉!渢渢乎!大而婉,险而易行,以德辅此,则明主也。」为之歌《唐》,曰:``思深哉!其有陶唐氏之遗民乎?不然,何忧之远也?非令德之后,谁能若是?」为之歌《陈》,曰:``国无主,其能久乎?」自《郐》以下无讥焉。为之歌《小雅》,曰:``美哉!思而不贰,怨而不言,其周德之衰乎?犹有先王之遗民焉。」为之歌《大雅》,曰:``广哉!熙熙乎!曲而有直体,其文王之德乎?」为之歌《颂》,曰:``至矣哉!直而不倨,曲而不屈,迩而不逼,远而不携,迁而不淫,复而不厌,哀而不愁,乐而不荒,用而不匮,广而不宣,施而不费,取而不贪,处而不底,行而不流,五声和,八风平,节有度,守有序,盛德之所同也。」

见舞《象箾》《南籥》者,曰:``美哉!犹有憾。」见舞《大武》者,曰:``美哉!周之盛也,其若此乎!」见舞《韶濩》者,曰:``圣人之弘也,而犹有惭德,圣人之难也。」见舞《大夏》者,曰:``美哉!勤而不德,非禹其谁能修之?」见舞《韶箾》者,曰:``德至矣哉!大矣!如天之无不帱也,如地之无不载也,虽甚盛德,其蔑以加于此矣。观止矣!若有他乐,吾不敢请已!」

其出聘也,通嗣君也。故遂聘于齐,说晏平仲,谓之曰:``子速纳邑与政!无邑无政,乃免于难。齐国之政,将有所归,未获所归,难未歇也。」故晏子因陈桓子以纳政与邑,是以免于栾、高之难。

聘于郑,见子产,如旧相识,与之缟带,子产献丝宁衣焉。谓子产曰:``郑之执政侈,难将至矣!政必及子。子为政,慎之以礼。不然,郑国将败。」

适卫,说蘧瑗、史狗、史鳅,公子荆、公叔发、公子朝,曰:``卫多君子,未有患也。」

自卫如晋,将宿于戚。闻钟声焉,曰:``异哉!吾闻之也:『辩而不德,必加于戮。』夫子获罪于君以在此,惧犹不足,而又何乐?夫子之在此也,犹燕之巢于幕上。君又在殡,而可以乐乎?」遂去之。文子闻之,终身不听琴瑟。

适晋,说赵文子、韩宣子、魏献子,曰:``晋国其萃于三族乎!」说叔向,将行,谓叔向曰:``吾子勉之!君侈而多良,大夫皆富,政将在家。吾子好直,必思自免于难。」

秋九月,齐公孙虿、公孙灶放其大夫高止于北燕。乙未,出。书曰:``出奔。」罪高止也。高止好以事自为功,且专,故难及之。

冬,孟孝伯如晋,报范叔也。

为高氏之难故,高竖以卢叛。十月庚寅,闾丘婴帅师围卢。高竖曰:``苟请高氏有后,请致邑。」齐人立敬仲之曾孙宴,良敬仲也。十一月乙卯,高竖致卢而出奔晋,晋人城绵而置旃。

郑伯有使公孙黑如楚,辞曰:``楚、郑方恶,而使余往,是杀余也。」伯有曰:``世行也。」子皙曰:``可则往,难则已,何世之有?」伯有将强使之。子皙怒,将伐伯有氏,大夫和之。十二月己巳,郑大夫盟于伯有氏。裨谌曰:``是盟也,其与几何?《诗》曰:『君子屡盟,乱是用长。』今是长乱之道也。祸未歇也,必三年而后能纾。」然明曰:``政将焉往?」裨谌曰:``善之代不善,天命也,其焉辟子产?举不逾等,则位班也。择善而举,则世隆也。天又除之,夺伯有魄,子西即世,将焉辟之?天祸郑久矣,其必使子产息之,乃犹可以戾。不然,将亡矣。」

\hypertarget{header-n2302}{%
\subsubsection{襄公三十年}\label{header-n2302}}

【经】三十年春王正月,楚子使薳罢来聘。夏四月,蔡世子般弑其君固。五月甲午。宋灾。宋伯姬卒。天王杀其弟佞夫。王子瑕奔晋。秋七月,叔弓如宋,葬宋共姬。郑良霄出奔许,自许入于郑,郑人杀良霄。冬十月,葬蔡景公。晋人、齐人、宋人、卫人、郑人、曹人、莒人、邾人、滕子、薛人、杞人、小邾人会于澶渊,宋灾故。

【传】三十年春,王正月,楚子使薳罢来聘,通嗣君也。穆叔问:``王子之为政何如?」对曰:``吾侪小人,食而听事,犹惧不给命而不免于戾,焉与知政?」固问焉,不告。穆叔告大夫曰:``楚令尹将有大事,子荡将与焉,助之匿其情矣。」

子产相郑伯以如晋,叔向问郑国之政焉。对曰:``吾得见与否,在此岁也。驷、良方争,未知所成。若有所成,吾得见,乃可知也。」叔向曰:``不既和矣乎?」对曰:``伯有侈而愎,子皙好在人上,莫能相下也。虽其和也,犹相积恶也,恶至无日矣。」

三月癸未,晋悼夫人食舆人之城杞者。绛县人或年长矣,无子,而往与于食。有与疑年,使之年。曰:``臣小人也,不知纪年。臣生之岁,正月甲子朔,四百有四十五甲子矣,其季于今三之一也。」吏走问诸朝,师旷曰:``鲁叔仲惠伯会郤成子于承匡之岁也。是岁也,狄伐鲁。叔孙庄叔于是乎败狄于咸,获长狄侨如及虺也豹也,而皆以名其子。七十三年矣。」史赵曰:``亥有二首六身,下二如身,是其日数也。」士文伯曰:``然则二万六千六百有六旬也。」

赵孟问其县大夫,则其属也。召之,而谢过焉,曰:``武不才,任君之大事,以晋国之多虞,不能由吾子,使吾子辱在泥涂久矣,武之罪也。敢谢不才。」遂仕之,使助为政。辞以老。与之田,使为君复陶,以为绛县师,而废其舆尉。于是,鲁使者在晋,归以语诸大夫。季武子曰:``晋未可媮也。有赵孟以为大夫,有伯瑕以为佐,有史赵、师旷而咨度焉,有叔向、女齐以师保其君。其朝多君子,其庸可媮乎?勉事之而后可。」

夏四月己亥,郑伯及其大夫盟。君子是以知郑难之不已也。

蔡景侯为大子般娶于楚,通焉。大子弑景侯。

初,王儋季卒,其子括将见王,而叹。单公子愆期为灵王御士,过诸廷,闻其叹而言曰:``乌乎!必有此夫!」入以告王,且曰:``必杀之!不戚而愿大,视躁而足高,心在他矣。不杀,必害。」王曰:``童子何知?」及灵王崩,儋括欲立王子佞夫,佞夫弗知。戊子,儋括围蒍,逐成愆。成愆奔平畦。五月癸巳,尹言多、刘毅、单蔑、甘过、巩成杀佞夫。括、瑕、廖奔晋。书曰``天王杀其弟佞夫。」罪在王也。

或叫于宋大庙,曰:``譆,譆!出出!」鸟鸣于亳社,如曰:``譆譆。」甲午,宋大灾。宋伯姬卒,待姆也。君子谓:``宋共姬,女而不妇。女待人,妇义事也。」

六月,郑子产如陈莅盟。归,覆命。告大夫曰:``陈,亡国也,不可与也。聚禾粟,缮城郭,恃此二者,而不抚其民。其君弱植,公子侈,大子卑,大夫敖,政多门,以介于大国,能无亡乎?不过十年矣。」

秋七月,叔弓如宋,葬共姬也。

郑伯有耆酒,为窟室,而夜饮酒击钟焉,朝至未已。朝者曰:``公焉在?」其人曰:``吾公在壑谷。」皆自朝布路而罢。既而朝,则又将使子皙如楚,归而饮酒。庚子,子皙以驷氏之甲伐而焚之。伯有奔雍梁,醒而后知之,遂奔许。大夫聚谋,子皮曰:``《仲虺之志》云:『乱者取之,亡者侮之。推亡固存,国之利也。』罕、驷、丰同生。伯有汰侈,故不免。」

人谓子产:``就直助强!」子产曰:``岂为我徒?国之祸难,谁知所儆?或主强直,难乃不生。姑成吾所。」辛丑,子产敛伯有氏之死者而殡之,不乃谋而遂行。印段从之。子皮止之,众曰:``人不我顺,何止焉?」子皮曰:``夫人礼于死者,况生者乎?」遂自止之。壬寅,子产入。癸卯,子石入。皆受盟于子皙氏。乙巳,郑伯及其大夫盟于大宫。盟国人于师之梁之外。

伯有闻郑人之盟己也,怒。闻子皮之甲不与攻己也,喜。曰:``子皮与我矣。」癸丑,晨,自墓门之渎入,因马师颉介于襄库,以伐旧北门。驷带率国人以伐之。皆召子产。子产曰:``兄弟而及此,吾从天所与。」伯有死于羊肆,子产襚之,枕之股而哭之,敛而殡诸伯有之臣在市侧者。既而葬诸斗城。子驷氏欲攻子产,子皮怒之曰:``礼,国之干也,杀有礼,祸莫大焉。」乃止。

于是游吉如晋还,闻难不入,覆命于介。八月甲子,奔晋。驷带追之,及酸枣。与子上盟,用两珪质于河。使公孙肸入盟大夫。己巳,复归。书曰``郑人杀良霄。」不称大夫,言自外入也。

于子蟜之卒也,将葬,公孙挥与裨灶晨会事焉。过伯有氏,其门上生莠。子羽曰:``其莠犹在乎?」于是岁在降娄,降娄中而旦。裨灶指之曰:``犹可以终岁,岁不及此次也已。」及其亡也,岁在娵訾之口。其明年,乃及降娄。

仆展从伯有,与之皆死。羽颉出奔晋,为任大夫。鸡泽之会,郑乐成奔楚,遂适晋。羽颉因之,与之比,而事赵文子,言伐郑之说焉。以宋之盟故,不可。子皮以公孙鉏为马师。

楚公子围杀大司马蒍掩而取其室。申无宇曰:``王子必不免。善人,国之主也。王子相楚国,将善是封殖,而虐之,是祸国也。且司马,令尹之偏,而王之四体也。绝民之主,去身之偏,艾王之体,以祸其国,无不祥大焉!何以得免?」

为宋灾故,诸侯之大夫会,以谋归宋财。冬十月,叔孙豹会晋赵武、齐公孙虿、宋向戌、卫北宫佗、郑罕虎及小邾之大夫,会于澶渊。既而无归于宋,故不书其人。

君子曰:``信其不可不慎乎!澶渊之会,卿不书,不信也夫!诸侯之上卿,会而不信,宠名皆弃,不信之不可也如是!《诗》曰:『文王陟降,在帝左右。』信之谓也。又曰:『淑慎尔止,无载尔伪。』不信之谓也。」书曰``某人某人会于澶渊,宋灾故。」尤之也。不书鲁大夫,讳之也。

郑子皮授子产政,辞曰:``国小而逼,族大宠多,不可为也。」子皮曰:``虎帅以听,谁敢犯子?子善相之,国无小,小能事大,国乃宽。」

子产为政,有事伯石,赂与之邑。子大叔曰:``国,皆其国也。奚独赂焉?」子产曰:``无欲实难。皆得其欲,以从其事,而要其成,非我有成,其在人乎?何爱于邑?邑将焉往?」子大叔曰:``若四国何?」子产曰:``非相违也,而相从也,四国何尤焉?《郑书》有之曰:『安定国家,必大焉先。』姑先安大,以待其所归。」既,伯石惧而归邑,卒与之。伯有既死,使大史命伯石为卿,辞。大史退,则请命焉。覆命之,又辞。如是三,乃受策入拜。子产是以恶其为人也,使次己位。

子产使都鄙有章,上下有服,田有封洫,庐井有伍。大人之忠俭者,从而与之。泰侈者,因而毙之。

丰卷将祭,请田焉。弗许,曰:``唯君用鲜,众给而已。」子张怒,退而征役。子产奔晋,子皮止之而逐丰卷。丰卷奔晋。子产请其田里,三年而复之,反其田里及其入焉。

从政一年,舆人诵之,曰:``取我衣冠而褚之,取我田畴而伍之。孰杀子产,吾其与之!」及三年,又诵之,曰;``我有子弟,子产诲之。我有田畴,子产殖之。子产而死,谁其嗣之?」

\hypertarget{header-n2330}{%
\subsubsection{襄公三十一年}\label{header-n2330}}

【经】三十有一年春王正月。夏六月辛巳,公薨于楚宫。秋九月癸巳,子野卒。己亥,仲孙羯卒。冬十月,滕子来会葬。癸酉,葬我君襄公。十有一月,莒人杀其君密州。

【传】三十一年春,王正月,穆叔至自会,见孟孝伯,语之曰:``赵孟将死矣。其语偷,不似民主。且年未盈五十,而谆谆焉如八九十者,弗能久矣。若赵孟死,为政者其韩子乎!吾子盍与季孙言之,可以树善,君子也。晋君将失政矣,若不树焉,使早备鲁,既而政在大夫,韩子懦弱,大夫多贪,求欲无厌,齐、楚未足与也,鲁其惧哉!」孝伯曰:``人生几何?谁能无偷?朝不及夕,将安用树?」穆叔出而告人曰:``孟孙将死矣。吾语诸赵孟之偷也,而又甚焉。」又与季孙语晋故,季孙不从。

及赵文子卒,晋公室卑,政在侈家。韩宣子为政,为能图诸侯。鲁不堪晋求,谗慝弘多,是以有平丘之会。

齐子尾害闾丘婴,欲杀之,使帅师以伐阳州。我问师故。夏五月,子尾杀闾丘婴以说于我师。工偻洒、渻灶、孔虺、贾寅出奔莒。出群公子。

公作楚宫。穆叔曰:``《大誓》云:『民之所欲,天必从之。』君欲楚也夫!故作其宫。若不复适楚,必死是宫也。」六月辛巳,公薨于楚宫。叔仲带窃其拱璧,以与御人,纳诸其怀而从取之,由是得罪。

立胡女敬归之子子野,次于季氏。秋九月癸巳,卒,毁也。

己亥,孟孝伯卒。

立敬归之娣齐归之子公子裯,穆叔不欲,曰:``大子死,有母弟则立之,无则长立。年钧择贤,义钧则卜,古之道也。非适嗣,何必娣之子?且是人也,居丧而不哀,在戚而有嘉容,是谓不度。不度之人,鲜不为患。若果立之,必为季氏忧。」武子不听,卒立之。比及葬,三易衰,衰衽如故衰。于是昭公十九年矣,犹有童心,君子是以知其不能终也。

冬十月,滕成公来会葬,惰而多涕。子服惠伯曰:``滕君将死矣!怠于其位,而哀已甚,兆于死所矣。能无从乎?」癸酉,葬襄公。

公薨之月,子产相郑伯以如晋,晋侯以我丧故,未之见也。子产使尽坏其馆之垣而纳车马焉。士文伯让之,曰:``敝邑以政刑之不修,寇盗充斥,无若诸侯之属辱在寡君者何?是以令吏人完客所馆,高其□闳,厚其墙垣,以无忧客使。今吾子坏之,虽从者能戒,其若异客何?以敝邑之为盟主,缮完葺墙,以待宾客,若皆毁之,其何以共命?寡君使□请命。」对曰:``以敝邑褊小,介于大国,诛求无时,是以不敢宁居,悉索敝赋,以来会时事。逢执之不间,而未得见,又不获闻命,未知见时,不敢输币,亦不敢暴露。其输之,则君之府实也,非荐陈之,不敢输也。其暴露之,则恐燥湿之不时而朽蠹,以重敝邑之罪。侨闻文公之为盟主也,宫室卑庳,无观台榭,以崇大诸侯之馆。馆如公寝,库厩缮修,司空以时平易道路,圬人以时塓馆宫室。诸侯宾至,甸设庭燎,仆人巡宫,车马有所,宾从有代,巾车脂辖,隶人牧圉,各瞻其事,百官之属,各展其物。公不留宾,而亦无废事,忧乐同之,事则巡之,教其不知,而恤其不足。宾至如归,无宁灾患?不畏寇盗,而亦不患燥湿。今铜鞮之宫数里,而诸侯舍于隶人。门不容车,而不可逾越。盗贼公行,而天厉不戒。宾见无时,命不可知。若又勿坏,是无所藏币,以重罪也。敢请执事,将何以命之?虽君之有鲁丧,亦敝邑之忧也。若获荐币,修垣而行,君之惠也,敢惮勤劳?」文伯覆命,赵文子曰:``信!我实不德,而以隶人之垣以赢诸侯,是吾罪也。」使士文伯谢不敏焉。晋侯见郑伯,有加礼,厚其宴好而归之。乃筑诸侯之馆。

叔向曰:``辞之不可以已也如是夫!子产有辞,诸侯赖之,若之何其释辞也?《诗》曰:『辞之辑矣,民之协矣。辞之绎矣,民之莫矣。』其知之矣。」

郑子皮使印段如楚,以适晋告,礼也。

莒犁比公生去疾及展舆,既立展舆,又废之。犁比公虐,国人患之。十一月,展舆因国人以攻莒子,弑之,乃立。去疾奔齐,齐出也。展舆,吴出也。书曰``莒人弑其君买朱鉏。」言罪之在也。

吴子使屈狐庸聘于晋,通路也。赵文子问焉,曰:``延州来季子其果立乎?巢陨诸樊,阍戕戴吴,天似启之,何如?」对曰:``不立。是二王之命也,非启季子也。若天所启,其在今嗣君乎!甚德而度,德不失民,度不失事,民亲而事有序,其天所启也。有吴国者,必此君之子孙实终之。季子,守节者也。虽有国,不立。」

十二月,北宫文子相卫襄公以如楚,宋之盟故也。过郑,印段廷劳于棐林,如聘礼而以劳辞。文子入聘。子羽为行人,冯简子与子大叔逆客。事毕而出,言于卫侯曰:``郑有礼,其数世之福也,其无大国之讨乎!《诗》曰:『谁能执热,逝不以濯。』礼之于政,如热之有濯也。濯以救热,何患之有?」

子产之从政也,择能而使之。冯简子能断大事,子大叔美秀而文,公孙挥能知四国之为,而辨于其大夫之族姓、班位、贵贱、能否,而又善为辞令,裨谌能谋,谋于野则获,谋于邑则否。郑国将有诸侯之事,子产乃问四国之为于子羽,且使多为辞令。与裨谌乘以适野,使谋可否。而告冯简子,使断之。事成,乃授子大叔使行之,以应对宾客。是以鲜有败事。北宫文子所谓有礼也。

郑人游于乡校,以论执政。然明谓子产曰:``毁乡校,何如?」子产曰:``何为?夫人朝夕退而游焉,以议执政之善否。其所善者,吾则行之。其所恶者,吾则改之。是吾师也,若之何毁之?我闻忠善以损怨,不闻作威以防怨。岂不遽止,然犹防川,大决所犯,伤人必多,吾不克救也。不如小决使道。不如吾闻而药之也。」然明曰:``蔑也今而后知吾子之信可事也。小人实不才,若果行此,其郑国实赖之,岂唯二三臣?」

仲尼闻是语也,曰:``以是观之,人谓子产不仁,吾不信也。」

子皮欲使尹何为邑。子产曰:``少,未知可否?」子皮曰:``愿,吾爱之,不吾叛也。使夫往而学焉,夫亦愈知治矣。」子产曰:``不可。人之爱人,求利之也。今吾子爱人则以政,犹未能操刀而使割也,其伤实多。子之爱人,伤之而已,其谁敢求爱于子?子于郑国,栋也,栋折榱崩,侨将厌焉,敢不尽言?子有美锦,不使人学制焉。大官、大邑,身之所庇也,而使学者制焉,其为美锦,不亦多乎?侨闻学而后入政,未闻以政学者也。若果行此,必有所害。譬如田猎,射御贯则能获禽,若未尝登车射御,则败绩厌覆是惧,何暇思获?」子皮曰:``善哉!虎不敏。吾闻君子务知大者、远者,小人务知小者、近者。我,小人也。衣服附在吾身,我知而慎之。大官、大邑所以庇身也,我远而慢之。微子之言,吾不知也。他日我曰:『子为郑国,我为吾家,以庇焉,其可也。』今而后知不足。自今,请虽吾家,听子而行。」子产曰:``人心之不同,如其面焉。吾岂敢谓子面如吾面乎?抑心所谓危,亦以告也。」子皮以为忠,故委政焉。子产是以能为郑国。

卫侯在楚,北宫文子见令尹围之威仪,言于卫侯曰:``令尹似君矣!将有他志,虽获其志,不能终也。《诗》云:『靡不有初,鲜克有终。』终之实难,令尹其将不免?」公曰:``子何以知之?」对曰:``《诗》云:『敬慎威仪,惟民之则。』令尹无威仪,民无则焉。民所不则,以在民上,不可以终。」公曰:``善哉!何谓威仪?」对曰:``有威而可畏谓之威,有仪而可像谓之仪。君有君之威仪,其臣畏而爱之,则而象之,故能有其国家,令闻长世。臣有臣之威仪,其下畏而爱之,故能守其官职,保族宜家。顺是以下皆如是,是以上下能相固也。《卫诗》曰:『威仪棣棣,不可选也。』言君臣、上下、父子、兄弟、内外、大小皆有威仪也。《周诗》曰:『朋友攸摄,摄以威仪。』言朋友之道,必相教训以威仪也。《周书》数文王之德,曰:『大国畏其力,小国怀其德。』言畏而爱之也。《诗》云:『不识不知,顺帝之则。』言则而象之也。纣囚文王七年,诸侯皆从之囚。纣于是乎惧而归之,可谓爱之。文王伐崇,再驾而降为臣,蛮夷帅服,可谓畏之。文王之功,天下诵而歌舞之,可谓则之,文王之行,至今为法,可谓象之。有威仪也。故君子在位可畏,施舍可爱,进退可度,周旋可则,容止可观,作事可法,德行可像,声气可乐,动作有文,言语有章,以临其下,谓之有威仪也。」

\hypertarget{header-n2352}{%
\subsection{昭公}\label{header-n2352}}

\begin{center}\rule{0.5\linewidth}{\linethickness}\end{center}

\hypertarget{header-n2354}{%
\subsubsection{昭公元年}\label{header-n2354}}

【经】元年春王正月,公即位。叔孙豹会晋赵武、楚公子围、齐国弱、宋向戌、卫齐恶、陈公子招、蔡公孙归生、郑罕虎、许人、曹人于虢。三月,取郓。夏,秦伯之弟金咸出奔晋。六月丁巳,邾子华卒。晋荀吴帅师败狄于大卤。秋,莒去疾自齐入于莒。莒展舆出奔吴。叔弓帅师疆郓田。葬邾悼公。冬十有一月己酉,楚子麇卒。公子比出奔晋。

【传】元年春,楚公子围聘于郑,且娶于公孙段氏,伍举为介。将入馆,郑人恶之,使行人子羽与之言,乃馆于外。既聘,将以众逆。子产患之,使子羽辞,曰:``以敝邑褊小,不足以容从者,请墠听命!」令尹命大宰伯州犁对曰:``君辱贶寡大夫围,谓围:『将使丰氏抚有而室。围布几筵,告于庄、共之庙而来。若野赐之,是委君贶于草莽也!是寡大夫不得列于诸卿也!不宁唯是,又使围蒙其先君,将不得为寡君老,其蔑以复矣。唯大夫图之!」子羽曰:``小国无罪,恃实其罪。将恃大国之安靖己,而无乃包藏祸心以图之。小国失恃而惩诸侯,使莫不憾者,距违君命,而有所壅塞不行是惧!不然,敝邑,馆人之属也,其敢爱丰氏之祧?」伍举知其有备也,请垂橐而入。许之。

正月乙未,入,逆而出。遂会于虢,寻宋之盟也。祁午谓赵文子曰:``宋之盟,楚人得志于晋。今令尹之不信,诸侯之所闻也。子弗戒,惧又如宋。子木之信称于诸侯,犹诈晋而驾焉,况不信之尤者乎?楚重得志于晋,晋之耻也。子相晋国以为盟主,于今七年矣!再合诸侯,三合大夫,服齐、狄,宁东夏,平秦乱,城淳于,师徒不顿,国家不罢,民无谤讟,诸侯无怨,天无大灾,子之力也。有令名矣,而终之以耻,午也是惧。吾子其不可以不戒!」文子曰:``武受赐矣!然宋之盟,子木有祸人之心,武有仁人之心,是楚所以驾于晋也。今武犹是心也,楚又行僭,非所害也。武将信以为本,循而行之。譬如农夫,是□麃是衮,虽有饥馑,必有丰年。且吾闻之:『能信不为人下。』吾未能也。《诗》曰:『不僭不贼,鲜不为则。』信也。能为人则者,不为人下矣。吾不能是难,楚不为患。」

楚令尹围请用牲,读旧书,加于牲上而已。晋人许之。

三月甲辰,盟。楚公子围设服离卫。叔孙穆子曰:``楚公子美矣,君哉!」郑子皮曰:``二执戈者前矣!」蔡子家曰:``蒲宫有前,不亦可乎?」楚伯州犁曰:``此行也,辞而假之寡君。」郑行人挥曰:``假不反矣!」伯州犁曰:``子姑忧子皙之欲背诞也。」子羽曰:``当璧犹在,假而不反,子其无忧乎?」齐国子曰:``吾代二子愍矣!」陈公子招曰:``不忧何成,二子乐矣。」卫齐子曰:``苟或知之,虽忧何害?」宋合左师曰:``大国令,小国共。吾知共而已。」晋乐王鲋曰:``《小旻》之卒章善矣,吾从之。」

退会,子羽谓子皮曰:``叔孙绞而婉,宋左师简而礼,乐王鲋字而敬,子与子家持之,皆保世之主也。齐、卫、陈大夫其不免乎?国子代人忧,子招乐忧,齐子虽忧弗害。夫弗及而忧,与可优而乐,与忧而弗害,皆取忧之道也,忧必及之。《大誓》曰:『民之所欲,天必从之。』三大夫兆忧,能无至乎?言以知物,其是之谓矣。」

季武子伐莒,取郓,莒人告于会。楚告于晋曰:``寻盟未退,而鲁伐莒,渎齐盟,请戮其使。」乐桓子相赵文子,欲求货于叔孙而为之请,使请带焉,弗与。梁其跁曰:``货以藩身,子何爱焉?」叔孙曰:``诸侯之会,卫社稷也。我以货免,鲁必受师。是祸之也,何卫之为?人之有墙,以蔽恶也。墙之隙坏,谁之咎也?卫而恶之,吾又甚焉。虽怨季孙,鲁国何罪?叔出季处,有自来矣,吾又谁怨?然鲋也贿,弗与,不已。」召使者,裂裳帛而与之,曰:``带其褊矣。」赵孟闻之,曰:``临患不忘国,忠也。思难不越官,信也;图国忘死,贞也;谋主三者,义也。有是四者,又可戮乎?」乃请诸楚曰:``鲁虽有罪,其执事不辟难,畏威而敬命矣。子若免之,以劝左右可也。若子之群吏处不辟污,出不逃难,其何患之有?患之所生,污而不治,难而不守,所由来也。能是二者,又何患焉?不靖其能,其谁从之?鲁叔孙豹可谓能矣,请免之以靖能者。子会而赦有罪,又赏其贤,诸侯其谁不欣焉望楚而归之,视远如迩?疆埸之邑,一彼一此,何常之有?王伯之令也,引其封疆,而树之官。举之表旗,而着之制令。过则有刑,犹不可壹。于是乎虞有三苗,夏有观、扈,商有姺、邳,周有徐、奄。自无令王,诸侯逐进,狎主齐盟,其又可壹乎?恤大舍小,足以为盟主,又焉用之?封疆之削,何国蔑有?主齐盟者,谁能辩焉?吴、濮有衅,楚之执事岂其顾盟?莒之疆事,楚勿与知,诸侯无烦,不亦可乎?莒、鲁争郓,为日久矣,苟无大害于其社稷,可无亢也。去烦宥善,莫不竞劝。子其图之!」固请诸楚,楚人许之,乃免叔孙。

令尹享赵孟,赋《大明》之首章。赵孟赋《小宛》之二章。事毕,赵孟谓叔向曰:``令尹自以为王矣,何如?」对曰:``王弱,令尹强,其可哉!虽可,不终。」赵孟曰:``何故?」对曰:``强以克弱而安之,强不义也。不义而强,其毙必速。《诗》曰:『赫赫宗周,褒姒灭之。』强不义也。令尹为王,必求诸侯。晋少懦矣,诸侯将往。若获诸侯,其虐滋甚。民弗堪也,将何以终?夫以强取,不义而克,必以为道。道以淫虐,弗可久已矣!」

夏四月,赵孟、叔孙豹、曹大夫入于郑,郑伯兼享之。子皮戒赵孟,礼终,赵孟赋《瓠叶》。子皮遂戒穆叔,且告之。穆叔曰:``赵孟欲一献,子其从之!」子皮曰:``敢乎?」穆叔曰:``夫人之所欲也,又何不敢?」及享,具五献之笾豆于幕下。赵孟辞,私于子产曰:``武请于冢宰矣。」乃用一献。赵孟为客,礼终乃宴。穆叔赋《鹊巢》。赵孟曰:``武不堪也。」又赋《采蘩》,曰:``小国为蘩,大国省穑而用之,其何实非命?」子皮赋《野有死麇》之卒章。赵孟赋《常棣》,且曰:``吾兄弟比以安,龙也可使无吠。」穆叔、子皮及曹大夫兴,拜,举兕爵,曰:``小国赖子,知免于戾矣。」饮酒乐。赵孟出,曰:``吾不复此矣。」

天王使刘定公劳赵孟于颖,馆于洛汭。刘子曰:``美哉禹功,明德远矣!微禹,吾其鱼乎!吾与子弁冕端委,以治民临诸侯,禹之力也。子盍亦远绩禹功,而大庇民乎?」对曰:``老夫罪戾是惧,焉能恤远?吾侪偷食,朝不谋夕,何其长也?」刘子归,以语王曰:``谚所为老将知而耄及之者,其赵孟之谓乎!为晋正卿,以主诸侯,而侪于隶人,朝不谋夕,弃神人矣。神怒民叛,何以能久?赵孟不复年矣。神怒,不歆其祀;民叛,不即其事。祀事不从,又何以年?」

叔孙归,曾夭御季孙以劳之。旦及日中不出。曾夭谓曾阜曰:``旦及日中,吾知罪矣。鲁以相忍为国也,忍其外不忍其内,焉用之?」阜曰:``数月于外,一旦于是,庸何伤?贾而欲赢,而恶嚣乎?」阜谓叔孙曰:``可以出矣!」叔孙指楹曰:``虽恶是,其可去乎?」乃出见之。

郑徐吾犯之妹美,公孙楚聘之矣,公孙黑又使强委禽焉。犯惧,告子产。子产曰:``是国无政,非子之患也。唯所欲与。」犯请于二子,请使女择焉。皆许之,子皙盛饰入,布币而出。子南戎服入。左右射,超乘而出。女自房观之,曰:``子皙信美矣,抑子南夫也。夫夫妇妇,所谓顺也。」适子南氏。子皙怒,既而櫜甲以见子南,欲杀之而取其妻。子南知之,执戈逐之。及冲,击之以戈。子皙伤而归,告大夫曰:``我好见之,不知其有异志也,故伤。」

大夫皆谋之。子产曰:``直钧,幼贱有罪。罪在楚也。」乃执子南而数之,曰:``国之大节有五,女皆奸之:畏君之威,听其政,尊其贵,事其长,养其亲。五者所以为国也。今君在国,女用兵焉,不畏威也。奸国之纪,不听政也。子皙,上大夫,女,嬖大夫,而弗下之,不尊贵也。幼而不忌,不事长也。兵其从兄,不养亲也。君曰:『余不女忍杀,宥女以远。』勉,速行乎,无重而罪!」

五月庚辰,郑放游楚于吴,将行子南,子产咨于大叔。大叔曰:``吉不能亢身,焉能亢宗?彼,国政也,非私难也。子图郑国,利则行之,又何疑焉?周公杀管叔而蔡蔡叔,夫岂不爱?王室故也。吉若获戾,子将行之,何有于诸游?」

秦后子有宠于桓,如二君于景。其母曰:``弗去,惧选。」癸卯,金咸适晋,其车千乘。书曰:``秦伯之弟金咸出奔晋。」罪秦伯也。后子享晋侯,造舟于河,十里舍车,自雍及绛。归取酬币,终事八反。司马侯问焉,曰:``子之车,尽于此而已乎?」对曰:``此之谓多矣!若能少此,吾何以得见?」女叔齐以告公,且曰:``秦公子必归。臣闻君子能知其过,必有令图。令图,天所赞也。」

后子见赵孟。赵孟曰:``吾子其曷归?」对曰:``金咸惧选于寡君,是以在此,将待嗣君。」赵孟曰:``秦君何如?」对曰:``无道。」赵孟曰:``亡乎?」对曰:``何为?一世无道,国未艾也。国于天地,有与立焉。不数世淫,弗能毙也。」赵孟曰:``天乎?」对曰:``有焉。」赵孟曰:``其几何?」对曰:``金咸闻之,国无道而年谷和熟,天赞之也。鲜不五稔。」赵孟视荫,曰:``朝夕不相及,谁能待五?」后子出,而告人曰:``赵孟将死矣。主民,玩岁而愒日,其与几何?」

郑为游楚乱故,六月丁巳,郑伯及其大夫盟于公孙段氏,罕虎、公孙侨、公孙段、印段、游吉、驷带私盟于闺门之外,实薰隧。公孙黑强与于盟,使大史书其名,且曰七子。子产弗讨。

晋中行穆子败无终及群狄于大原,崇卒也。将战,魏舒曰:``彼徒我车,所遇又厄,以什共车必克。困诸厄,又克。请皆卒,自我始。」乃毁车以为行,五乘为三伍。荀吴之嬖人不肯即卒,斩以徇。为五陈以相离,两于前,伍于后,专为左角,参为左角,偏为前拒,以诱之。翟人笑之。未陈而薄之,大败之。

莒展舆立,而夺群公子秩。公子召去疾于齐。秋,齐公子鉏纳去疾,展舆奔吴。

叔弓帅师疆郓田,因莒乱也。于是莒务娄、瞀胡及公子灭明以大厖与常仪靡奔齐。君子曰:``莒展之不立,弃人也夫!人可弃乎?《诗》曰:『无竞维人。』善矣。」

晋侯有疾,郑伯使公孙侨如晋聘,且问疾。叔向问焉,曰:``寡君之疾病,卜人曰:『实沈、台骀为祟。』史莫之知,敢问此何神也?」子产曰:``昔高辛氏有二子,伯曰阏伯,季曰实沈,居于旷林,不相能也。日寻干戈,以相征讨。后帝不臧,迁阏伯于商丘,主辰。商人是因,故辰为商星。迁实沈于大夏,主参。唐人是因,以服事夏、商。其季世曰唐叔虞。当武王邑姜方震大叔,梦帝谓己:『余命而子曰虞,将与之唐,属诸参,其蕃育其子孙。』及生,有文在其手曰:『虞』,遂以命之。及成王灭唐而封大叔焉,故参为晋星。由是观之,则实沈,参神也。昔金天氏有裔子曰昧,为玄冥师,生允格、台骀。台骀能业其官,宣汾、洮,障大泽,以处大原。帝用嘉之,封诸汾川。沈、姒、蓐、黄,实守其祀。今晋主汾而灭之矣。由是观之,则台骀,汾神也。抑此二者,不及君身。山川之神,则水旱疠疫之灾,于是乎禜之。日月星辰之神,则雪霜风雨之不时,于是乎禜之。若君身,则亦出入饮食哀乐之事也,山川星辰之神,又何为焉」?侨闻之,君子有四时:朝以听政,昼以访问,夕以修令,夜以安身。于是乎节宣其气,勿使有所壅闭湫底,以露其体。兹心不爽,而昏乱百度。今无乃壹之,则生疾矣。侨又闻之,内官不及同姓,其生不殖,美先尽矣,则相生疾,君子是以恶之。故《志》曰:『买妾不知其姓,则卜之。』违此二者,古之所慎也。男女辨姓,礼之大司也。今君内实有四姬焉,其无乃是也乎?若由是二者,弗可为也已。四姬有省犹可,无则必生疾矣。」叔向曰:``善哉!肸未之闻也。此皆然矣。」

叔向出,行人挥送之。叔向问郑故焉,且问子皙。对曰:``其与几何?无礼而好陵人,怙富而卑其上,弗能久矣。」

晋侯闻子产之言,曰:``博物君子也。」重贿之。

晋侯求医于秦。秦伯使医和视之,曰:``疾不可为也。是谓:『近女室,疾如蛊。非鬼非食,惑以丧志。良巨将死,天命不佑』」公曰:``女不可近乎?」对曰:``节之。先王之乐,所以节百事也。故有五节,迟速本末以相及,中声以降,五降之后,不容弹矣。于是有烦手淫声,慆堙心耳,乃忘平和,君子弗德也。物亦如之,至于烦,乃舍也已,无以生疾。君子之近琴瑟,以仪节也,非以慆心也。天有六气,降生五味,发为五色,征为五声,淫生六疾。六气曰阴、阳、风、雨、晦、明也。分为四时,序为五节,过则为灾。阴淫寒疾,阳淫热疾,风淫末疾,雨淫腹疾,晦淫惑疾,明淫心疾。女,阳物而晦时,淫则生内热惑蛊之疾。今君不节不时,能无及此乎?」出,告赵孟。赵孟曰:``谁当良臣?」对曰:``主是谓矣!主相晋国,于今八年,晋国无乱,诸侯无阙,可谓良矣。和闻之,国之大臣,荣其宠禄,任其宠节,有灾祸兴而无改焉,必受其咎。今君至于淫以生疾,将不能图恤社稷,祸孰大焉!主不能御,吾是以云也。」赵孟曰:``何谓蛊」对曰:``淫溺惑乱之所生也。于文,皿虫为蛊。谷之飞亦为蛊。在《周易》,女惑男,风落山,谓之《蛊》三。皆同物也。」赵孟曰:``良医也。」厚其礼归之。

楚公子围使公子黑肱、伯州犁城雠、栎、郏,郑人惧。子产曰:``不害。令尹将行大事,而先除二子也。祸不及郑,何患焉?」

冬,楚公子围将聘于郑,伍举为介。未出竟,闻王有疾而还。伍举遂聘。十一月己酉,公子围至,入问王疾,缢而弑之。遂杀其二子幕及平夏。右尹子干出奔晋。宫厩尹子皙出奔郑。杀大宰伯州犁于郏。葬王于郏,谓之郏敖。使赴于郑,伍举问应为后之辞焉。对曰:``寡大夫围。」伍举更之曰:``共王之子围为长。」

子干奔晋,从车五乘。叔向使与秦公子同食,皆百人之饩。赵文子曰:``秦公子富。」叔向曰:``底禄以德,德钧以年,年同以尊。公子以国,不闻以富。且夫以千乘去其国,强御已甚。《诗》曰:『不侮鳏寡,不畏强御。』秦、楚,匹也。」使后子与子干齿。辞曰:``金咸惧选,楚公子不获,是以皆来,亦唯命。且臣与羁齿,无乃不可乎?史佚有言曰:『非羁何忌?』」

楚灵王即位,薳罢为令尹,薳启强为大宰。郑游吉如楚,葬郏敖,且聘立君。归,谓子产曰:``具行器矣!楚王汰侈而自说其事,必合诸侯。吾往无日矣。」子产曰:``不数年,未能也。」

十二月,晋既烝,赵孟适南阳,将会孟子余。甲辰朔,烝于温。庚戌,卒。郑伯如晋吊,及雍乃复。

\hypertarget{header-n2386}{%
\subsubsection{昭公二年}\label{header-n2386}}

【经】二年春,晋侯使韩起来聘。夏,叔弓如晋。秋,郑杀其大夫公孙黑。冬,公如晋,至河乃复。季孙宿如晋。

【传】二年春,晋侯使韩宣子来聘,且告为政而来见,礼也。观书于大史氏,见《易》《象》与《鲁春秋》,曰:``周礼尽在鲁矣。吾乃今知周公之德,与周之所以王也。」公享之。季武子赋《绵》之卒章。韩子赋《角弓》。季武子拜,曰:``敢拜子之弥缝敝邑,寡君有望矣。」武子赋《节》之卒章。既享,宴于季氏,有嘉树焉,宣子誉之。武子曰:``宿敢不封殖此树,以无忘《角弓》。」遂赋《甘棠》。宣子曰:``起不堪也,无以及召公。」

宣子遂如齐纳币。见子雅。子雅召子旗,使见宣子。宣子曰:``非保家之主也,不臣。」见子尾。子尾见强,宣子谓之如子旗。大夫多笑之,唯晏子信之,曰:``夫子,君子也。君子有信,其有以知之矣。」自齐聘于卫。卫侯享之,北宫文子赋《淇澳》。宣子赋《木瓜》。

夏四月,韩须如齐逆女。齐陈无宇送女,致少姜。少姜有宠于晋侯,晋侯谓之少齐。谓陈无宇非卿,执诸中都。少姜为之请曰:``送从逆班,畏大国也,犹有所易,是以乱作。」

叔弓聘于晋,报宣子也。晋侯使郊劳。辞曰:``寡君使弓来继旧好,固曰:『女无敢为宾!』彻命于执事,敝邑弘矣。敢辱郊使?请辞。」致馆。辞曰:``寡君命下臣来继旧好,好合使成,臣之禄也。敢辱大馆?」叔向曰:``子叔子知礼哉!吾闻之曰:『忠信,礼之器也。卑让,礼之宗也。』辞不忘国,忠信也。先国后己,卑让也。《诗》曰:『敬慎威仪,以近有德。』夫子近德矣。」

秋,郑公孙黑将作乱,欲去游氏而代其位,伤疾作而不果。驷氏与诸大夫欲杀之。子产在鄙,闻之,惧弗及,乘遽而至。使吏数之,曰:``伯有之乱,以大国之事,而未尔讨也。尔有乱心,无厌,国不女堪。专伐伯有,而罪一也。昆弟争室,而罪二也。薰隧之盟,女矫君位,而罪三也。有死罪三,何以堪之?不速死,大刑将至。」再拜稽首,辞曰:``死在朝夕,无助天为虐。」子产曰:``人谁不死?凶人不终,命也。作凶事,为凶人。不助天,其助凶人乎?」请以印为褚师。子产曰:``印也若才,君将任之。不才,将朝夕从女。女罪之不恤,而又何请焉?不速死,司寇将至。」七月壬寅,缢。尸诸周氏之衢,加木焉。

晋少姜卒。公如晋,及河。晋侯使士文伯来辞,曰:``非伉俪也。请君无辱!」公还,季孙宿遂致服焉。叔向言陈无宇于晋侯曰:``彼何罪?君使公族逆之,齐使上大夫送之。犹曰不共,君求以贪。国则不共,而执其使。君刑已颇,何以为盟主?且少姜有辞。」冬十月,陈无宇归。

十一月,郑印段如晋吊。

\hypertarget{header-n2397}{%
\subsubsection{昭公三年}\label{header-n2397}}

【经】三年春王正月丁未,滕子原卒。夏,叔弓如滕。五月,葬滕成公。秋,小邾子来朝。八月,大雩。冬,大雨雹。北燕伯款出奔齐。

【传】三年春,王正月,郑游吉如晋,送少姜之葬。梁丙与张趯见之。梁丙曰:``甚矣哉!子之为此来也。」子大叔曰:``将得已乎?昔文、襄之霸也,其务不烦诸侯。令诸侯三岁而聘,五岁而朝,有事而会,不协而盟。君薨,大夫吊,卿共葬事。夫人,士吊,大夫送葬。足以昭礼命事谋阙而已,无加命矣。今嬖宠之丧,不敢择位,而数于守适,唯惧获戾,岂敢惮烦?少姜有宠而死,齐必继室。今兹吾又将来贺,不唯此行也。」张趯曰:``善哉!吾得闻此数也。然自今,子其无事矣。譬如火焉,火中,寒暑乃退。此其极也,能无退乎?晋将失诸侯,诸侯求烦不获。」二大夫退。子大叔告人曰:``张趯有知,其犹在君子之后乎!」

丁未,滕子原卒。同盟,故书名。

齐侯使晏婴请继室于晋,曰:``寡君使婴曰:『寡人愿事君,朝夕不倦,将奉质币,以无失时,则国家多难,是以不获。不腆先君之适,以备内官,焜耀寡人之望,则又无禄,早世殒命,寡人失望。君若不忘先君之好,惠顾齐国,辱收寡人,徼福于大公、丁公,照临敝邑,镇抚其社稷,则犹有先君之适及遗姑姊妹若而人。君若不弃敝邑,而辱使董振择之,以备嫔嫱,寡人之望也。』」韩宣子使叔向对曰:``寡君之愿也。寡君不能独任其社稷之事,未有伉俪。在縗絰之中,是以未敢请。君有辱命,惠莫大焉。若惠顾敝邑,抚有晋国,赐之内主,岂唯寡君,举群臣实受其贶。其自唐叔以下,实宠嘉之。」

既成昏,晏子受礼。叔向从之宴,相与语。叔向曰:``齐其何如?」晏子曰:``此季世也,吾弗知。齐其为陈氏矣!公弃其民,而归于陈氏。齐旧四量,豆、区、釜、钟。四升为豆,各自其四,以登于釜。釜十则钟。陈氏三量,皆登一焉,钟乃大矣。以家量贷,而以公量收之。山木如市,弗加于山。鱼盐蜃蛤,弗加于海。民参其力,二入于公,而衣食其一。公聚朽蠹,而三老冻馁。国之诸市,屦贱踊贵。民人痛疾,而或燠休之,其爱之如父母,而归之如流水,欲无获民,将焉辟之?箕伯、直柄、虞遂、伯戏,其相胡公、大姬,已在齐矣。」

叔向曰:``然。虽吾公室,今亦季世也。戎马不驾,卿无军行,公乘无人,卒列无长。庶民罢敝,而宫室滋侈。道堇相望,而女富溢尤。民闻公命,如逃寇仇。栾、郤、胥、原、狐、续、庆、伯,降在皂隶。政在家门,民无所依,君日不悛,以乐慆忧。公室之卑,其何日之有?《谗鼎之铭》曰:『昧旦丕显,后世犹怠。』况日不悛,其能久乎?」

宴子曰:``子将若何?」叔向曰:``晋之公族尽矣。肸闻之,公室将卑,其宗族枝叶先落,则公从之。肸之宗十一族,唯羊舌氏在而已。肸又无子。公室无度,幸而得死,岂其获祀?」

初,景公欲更晏子之宅,曰:``子之宅近市,湫隘嚣尘,不可以居,请更诸爽垲者。」辞曰:``君之先臣容焉,臣不足以嗣之,于臣侈矣。且小人近市,朝夕得所求,小人之利也。敢烦里旅?」公笑曰:``子近市,识贵贱乎?」对曰:``既利之,敢不识乎?」公曰:``何贵何贱?」于是景公繁于刑,有鬻踊者。故对曰:``踊贵屦贱。」既已告于君,故与叔向语而称之。景公为是省于刑。君子曰:``仁人之言,其利博哉。晏子一言而齐侯省刑。《诗》曰:『君子如祉,乱庶遄已。』其是之谓乎!」

及宴子如晋,公更其宅,反,则成矣。既拜,乃毁之,而为里室,皆如其旧。则使宅人反之,曰:``谚曰:『非宅是卜,唯邻是卜。』二三子先卜邻矣,违卜不祥。君子不犯非礼,小人不犯不祥,古之制也。吾敢违诸乎?」卒复其旧宅。公弗许,因陈桓子以请,乃许之。

夏四月,郑伯如晋,公孙段相,甚敬而卑,礼无违者。晋侯嘉焉,授之以策,曰:``子丰有劳于晋国,余闻而弗忘。赐女州田,以胙乃旧勋。」伯石再拜稽首,受策以出。君子曰:``礼,其人之急也乎!伯石之汰也,一为礼于晋,犹荷其禄,况以礼终始乎?《诗》曰:『人而无礼,胡不遄死?』其是之谓乎!」

初,州县,栾豹之邑也。及栾氏亡,范宣子、赵文子、韩宣子皆欲之。文子曰:``温,吾县也。」二宣子曰:``自郤称以别,三传矣。晋之别县不唯州,谁获治之?」文子病之,乃舍之。二子曰:``吾不可以正议而自与也。」皆舍之。及文子为政,赵获曰:``可以取州矣。」文子曰:``退!二子之言,义也。违义,祸也。余不能治余县,又焉用州?其以徼祸也?君子曰:『弗知实难。』知而弗从,祸莫大焉。有言州必死。」

丰氏故主韩氏,伯石之获州也,韩宣子为请之,为其复取之之故。

五月,叔弓如滕,葬滕成公,子服椒为介。及郊,遇懿伯之忌,敬子不入。惠伯曰:``公事有公利,无私忌,椒请先入。」乃先受馆。敬子从之。

晋韩起如齐逆女。公孙虿为少姜之有宠也,以其子更公女而嫁公子。人谓宣子:``子尾欺晋,晋胡受之?」宣子曰:``我欲得齐而远其宠,宠将来乎?」

秋七月,郑罕虎如晋,贺夫人,且告曰:``楚人日征敝邑,以不朝立王之故。敝邑之往,则畏执事其谓寡君『而固有外心。』其不往,则宋之盟云。进退罪也。寡君使虎布之。」宣子使叔向对曰:``君若辱有寡君,在楚何害?修宋盟也。君苟思盟,寡君乃知免于戾矣。君若不有寡君,虽朝夕辱于敝邑,寡君猜焉。君实有心,何辱命焉?君其往也!苟有寡君,在楚犹在晋也。」

张趯使谓大叔曰:``自子之归也,小人粪除先人之敝庐,曰子其将来。今子皮实来,小人失望。」大叔曰:``吉贱,不获来,畏大国,尊夫人也。且孟曰:『而将无事。』吉庶几焉。」

小邾穆公来朝。季武子欲卑之,穆叔曰:``不可。曹、滕、二邾,实不忘我好,敬以逆之,犹惧其贰。又卑一睦,焉逆群好也?其如旧而加敬焉!《志》曰:『能敬无灾。』又曰:『敬逆来者,天所福也。』」季孙从之。

八月,大雩,旱也。

齐侯田于莒,卢蒲弊见,泣且请曰:``余发如此种种,余奚能为?」公曰:``诺,吾告二子。」归而告之。子尾欲复之,子雅不可,曰:``彼其发短而心甚长,其或寝处我矣。」九月,子雅放卢蒲弊于北燕。

燕简公多嬖宠,欲去诸大夫而立其宠人。冬,燕大夫比以杀公之外嬖。公惧,奔齐。书曰:``北燕伯款出奔齐。」罪之也。

十月,郑伯如楚,子产相。楚子享之,赋《吉日》。既享,子产乃具田备,王以田江南之梦。

齐公孙灶卒。司马灶见晏子,曰:``又丧子雅矣。」晏子曰:``惜也!子旗不免,殆哉!姜族弱矣,而妫将始昌。二惠竞爽,犹可,又弱一个焉,姜其危哉!」

\hypertarget{header-n2422}{%
\subsubsection{昭公四年}\label{header-n2422}}

【经】四年春王正月,大雨雹。夏,楚子、蔡侯、陈侯、郑伯、许男、徐子、滕子、顿子、胡子、沈子、小邾子、宋世子佐、淮夷会于申。楚子执徐子。秋七月,楚子、蔡侯、陈侯、许男、顿子、胡子、沈子、淮夷伐吴,执齐庆封,杀之。遂灭赖。九月,取鄫。冬十有二月乙卯,叔孙豹卒。

【传】四年春,王正月,许男如楚,楚子止之,遂止郑伯,复田江南,许男与焉。使椒举如晋求诸侯,二君待之。椒举致命曰:``寡君使举曰:『日君有惠,赐盟于宋,曰:晋、楚之从,交相见也。以岁之不易,寡人愿结欢于二三君。』使举请间。君若苟无四方之虞,则愿假宠以请于诸侯。」

晋侯欲勿许。司马侯曰:``不可。楚王方侈,天或者欲逞其心,以厚其毒而降之罚,未可知也。其使能终,亦未可知也。晋、楚唯天所相,不可与争。君其许之,而修德以待其归。若归于德,吾犹将事之,况诸侯乎?若适淫虐,楚将弃之,吾又谁与争?」曰:``晋有三不殆,其何敌之有?国险而多马,齐、楚多难。有是三者,何乡而不济?」对曰:``恃险与马,而虞邻国之难,是三殆也。四岳、三涂、阳城、大室、荆山、中南,九州之险也,是不一姓。冀之北土,马之所生,无兴国焉。恃险与马,不可以为固也,从古以然。是以先王务修德音以亨神人,不闻其务险与马也。邻国之难,不可虞也。或多难以固其国,启其疆土;或无难以丧其国,失其守宇。若何虞难?齐有仲孙之难而获桓公,至今赖之。晋有里、丕之难而获文公,是以为盟主。卫、邢无难,敌亦丧之。故人之难,不可虞也。恃此三者,而不修政德,亡于不暇,又何能济?君其许之!纣作淫虐,文王惠和,殷是以陨,周是以兴,夫岂争诸侯?」乃许楚使。使叔向对曰:``寡君有社稷之事,是以不获春秋时见。诸侯,君实有之,何辱命焉?」椒举遂请昏,晋侯许之。

楚子问于子产曰:``晋其许我诸侯乎?」对曰:``许君。晋君少安,不在诸侯。其大夫多求,莫匡其君。在宋之盟,又曰如一,若不许君,将焉用之?」王曰:``诸侯其来乎?」对曰:``必来。从宋之盟,承君之欢,不畏大国,何故不来?不来者,其鲁、卫、曹、邾乎?曹畏宋,邾畏鲁,鲁、卫逼于齐而亲于晋,唯是不来。其馀,君之所及也,谁敢不至?」王曰:``然则吾所求者,无不可乎?」对曰:``求逞于人,不可;与人同欲,尽济。」

大雨雹。季武子问于申丰曰:``雹可御乎?」对曰:``圣人在上,无雹,虽有,不为灾。古者,日在北陆而藏冰;西陆,朝觌而出之。其藏冰也,深山穷谷,固阴冱寒,于是乎取之。其出之也,朝之禄位,宾食丧祭,于是乎用之。其藏之也,黑牲、秬黍,以享司寒。其出之也,桃弧、棘矢,以除其灾。其出入也时。食肉之禄,冰皆与焉。大夫命妇,丧浴用冰。祭寒而藏之,献羔而启之,公始用之。火出而毕赋。自命夫、命妇,至于老疾,无不受冰。山人取之,县人传之,舆人纳之,隶人藏之。夫冰以风壮,而以风出。其藏之也周,其用之也遍,则冬无愆阳,夏无伏阴,春无凄风,秋无苦雨,雷不出震,无灾霜雹,疠疾不降,民不夭札。今藏川池之冰,弃而不用。风不越而杀,雷不发而震。雹之为灾,谁能御之?《七月》之卒章,藏冰之道也。」

夏,诸侯如楚,鲁、卫、曹、邾不会。曹、邾辞以难,公辞以时祭,卫侯辞以疾。郑伯先待于申。六月丙午,楚子合诸侯于申。椒举言于楚子曰:``臣闻诸侯无归,礼以为归。今君始得诸侯,其慎礼矣。霸之济否,在此会也。夏启有钧台之享,商汤有景亳之命,周武有孟津之誓,成有岐阳之搜,康有酆宫之朝,穆有涂山之会,齐桓有召陵之师,晋文有践土之盟。君其何用?宋向戌、郑公孙侨在,诸侯之良也,君其选焉。」王曰:``吾用齐桓。」王使问礼于左师与子产。左师曰:``小国习之,大国用之,敢不荐闻?」献公合诸侯之礼六。子产曰:``小国共职,敢不荐守?」献伯、子、男会公之礼六。君子谓合左师善守先代,子产善相小国。王使椒举侍于后,以规过。卒事,不规。王问其故,对曰:``礼,吾所未见者有六焉,又何以规?」宋大子佐后至,王田于武城,久而弗见。椒举请辞焉。王使往,曰:``属有宗祧之事于武城,寡君将堕币焉,敢谢后见。」

徐子,吴出也,以为贰焉,故执诸申。

楚子示诸侯侈,椒举曰:``夫六王二公之事,皆所以示诸侯礼也,诸侯所由用命也。夏桀为仍之会,有婚叛之。商纣为黎之搜,东夷叛之。周幽为大室之盟,戎狄叛之。皆所以示诸侯汰也,诸侯所由弃命也。今君以汰,无乃不济乎?」王弗听。

子产见左师曰:``吾不患楚矣,汰而愎谏,不过十年。」左师曰:``然。不十年侈,其恶不远,远恶而后弃。善亦如之,德远而后兴。」

秋七月,楚子以诸侯伐吴。宋大子、郑伯先归。宋华费遂、郑大夫从。使屈申围朱方,八月甲申,克之。执齐庆封而尽灭其族。将戮庆封。椒举曰:``臣闻无瑕者可以戮人。庆封唯逆命,是以在此,其肯从于戮乎?播于诸侯,焉用之?」王弗听,负之斧钺,以徇于诸侯,使言曰:``无或如齐庆封,弑其君,弱其孤,以盟其大夫。」庆封曰:``无或如楚共王之庶子围,弑其君、兄之子麇而代之,以盟诸侯。」王使速杀之。

遂以诸侯灭赖。赖子面缚衔璧,士袒,舆榇从之,造于中军。王问诸椒举,对曰:``成王克许,许僖公如是,王亲释其缚,受其璧,焚其榇。」王从之。迁赖于鄢。楚子欲迁许于赖,使斗韦龟与公子弃疾城之而还。申无宇曰:``楚祸之首,将在此矣。召诸侯而来,伐国而克,城竟莫校。王心不违,民其居乎?民之不处,其谁堪之?不堪王命,乃祸乱也。」

九月,取鄫,言易也。莒乱,着丘公立而不抚鄫,鄫叛而来,故曰取。凡克邑不用师徒曰取。

郑子产作丘赋。国人谤之,曰:``其父死于路,己为虿尾。以令于国,国将若之何?」子宽以告。子产曰:``何害?苟利社稷,死生以之。且吾闻为善者不改其度,故能有济也。民不可逞,度不可改。《诗》曰:『礼义不愆,何恤于人言。』吾不迁矣。浑罕曰:``国氏其先亡乎!君子作法于凉,其敝犹贪。作法于贪,敝将若之何?姬在列者,蔡及曹、滕其先亡乎!逼而无礼。郑先卫亡,逼而无法。政不率法,而制于心。民各有心,何上之有?」

冬,吴伐楚,入棘、栎、麻,以报朱方之役。楚沈尹射奔命于夏汭,咸尹宜咎城钟离,薳启强城巢,然丹城州来。东国水,不可以城。彭生罢赖之师。

初,穆子去叔孙氏,及庚宗,遇妇人,使私为食而宿焉。问其行,告之故,哭而送之。适齐,娶于国氏,生孟丙、仲壬。梦天压己,弗胜。顾而见人,黑而上偻,深目而豭喙。号之曰:``牛!助余!」乃胜之。旦而皆召其徒,无之。且曰:``志之。」及宣伯奔齐,馈之。宣伯曰:``鲁以先子之故,将存吾宗,必召女。召女,何如?」对曰:``愿之久矣。」鲁人召之,不告而归。既立,所宿庚宗之妇人,献以雉。问其姓,对曰:``余子长矣,能奉雉而从我矣。」召而见之,则所梦也。未问其名,号之曰:``牛!」曰:``唯」。皆召其徒,使视之,遂使为竖。有宠,长使为政。公孙明知叔孙于齐,归,未逆国姜,子明取之。故怒,其子长而后使逆之。田于丘莸,遂遇疾焉。竖牛欲乱其室而有之,强与孟盟,不可。叔孙为孟钟,曰:``尔未际,飨大夫以落之。」既具,使竖牛请日。入,弗谒。出,命之日。及宾至,闻钟声。牛曰:``孟有北妇人之客。」怒,将往,牛止之。宾出,使拘而杀诸外,牛又强与仲盟,不可。仲与公御莱书观于公,公与之环。使牛入示之。入,不示。出,命佩之。牛谓叔孙:``见仲而何?」叔孙曰:``何为?」曰:``不见,既自见矣。公与之环而佩之矣。」遂逐之,奔齐。疾急,命召仲,牛许而不召。

杜泄见,告之饥渴,授之戈。对曰:``求之而至,又何去焉?」竖牛曰:``夫子疾病,不欲见人。」使置馈于个而退。牛弗进,则置虚,命彻。十二月癸丑,叔孙不食。乙卯,卒。牛立昭子而相之。

公使杜泄葬叔孙。竖牛赂叔仲昭子与南遗,使恶杜泄于季孙而去之。杜泄将以路葬,且尽卿礼。南遗谓季孙曰:``叔孙未乘路,葬焉用之?且冢卿无路,介卿以葬,不亦左乎?」季孙曰:``然。」使杜泄舍路。不可,曰:``夫子受命于朝,而聘于王。王思旧勋而赐之路。覆命而致之君,君不敢逆王命而复赐之,使三官书之。吾子为司徒,实书名。夫子为司马,与工正书服。孟孙为司空,以书勋。今死而弗以,同弃君命也。书在公府而弗以,是废三官也。若命服,生弗敢服,死又不以,将焉用之?」乃使以葬。

季孙谋去中军。竖牛曰:``夫子固欲去之。」

\hypertarget{header-n2443}{%
\subsubsection{昭公五年}\label{header-n2443}}

【经】五年春王正月,舍中军。楚杀其大夫屈申。公如晋。夏,莒牟夷以牟娄及防、兹来奔。秋七月,公至自晋。戊辰,叔弓帅师败莒师于蚡泉。秦伯卒。冬,楚子、蔡侯、陈侯、许男、顿子、沈子、徐人、越人伐吴。

【传】五年春,王正月,舍中军,卑公室也。毁中军于施氏,成诸臧氏。初作中军,三分公室而各有其一。季氏尽征之,叔孙氏臣其子弟,孟氏取其半焉。及其舍之也,四分公室,季氏择二,二子各一。皆尽征之,而贡于公。以书。使杜泄告于殡,曰:``子固欲毁中军,既毁之矣,故告。」杜泄曰:``夫子唯不欲毁也,故盟诸僖闳,诅诸五父之衢。」受其书而投之,帅士而哭之。叔仲子谓季孙曰:``带受命于子叔孙曰:『葬鲜者自西门。』」季孙命杜泄。杜泄曰:``卿丧自朝,鲁礼也。吾子为国政,未改礼,而又迁之。群臣惧死,不敢自也。」既葬而行。

仲至自齐,季孙欲立之。南遗曰:``叔孙氏厚则季氏薄。彼实家乱,子勿与知,不亦可乎?」南遗使国人助竖牛以攻诸大库之庭。司宫射之,中目而死。竖牛取东鄙三十邑,以与南遗。

昭子即位,朝其家众,曰:``竖牛祸叔孙氏,使乱大从,杀适立庶,又披其邑,将以赦罪,罪莫大焉。必速杀之。」竖牛惧,奔齐。孟、仲之子杀诸塞关之外,投其首于宁风之棘上。

仲尼曰:``叔孙昭子之不劳,不可能也。周任有言曰:『为政者不赏私劳,不罚私怨。』《诗》云:『有觉德行,四国顺之。』」

初,穆子之生也,庄叔以《周易》筮之,遇《明夷》三之《谦》三,以示卜楚丘。曰:``是将行,而归为子祀。以谗人入,其名曰牛,卒以馁死。《明夷》,日也。日之数十,故有十时,亦当十位。自王已下,其二为公,其三为卿。日上其中,食日为二,旦日为三。《明夷》之《谦》,明而未融,其当旦乎,故曰:『为子祀』。日之《谦》,当鸟,故曰『明夷于飞』。明之未融,故曰『垂其翼』。象日之动,故曰『君子于行』。当三在旦,故曰『三日不食』。《离》,火也。《艮》,山也。《离》为火,火焚山,山败。于人为言,败言为谗,故曰『有攸往,主人有言』,言必谗也。纯《离》为牛,世乱谗胜,胜将适《离》,故曰『其名曰牛』。谦不足,飞不翔,垂不峻,翼不广,故曰『其为子后乎』。吾子,亚卿也,抑少不终。」

楚子以屈申为贰于吴,乃杀之。以屈生为莫敖,使与令尹子荡如晋逆女。过郑,郑伯劳子荡于汜,劳屈生于菟氏。晋侯送女于邢丘。子产相郑伯,会晋侯于邢丘。

公如晋,自郊劳至于赠贿,无失礼。晋侯谓女叔齐曰:``鲁侯不亦善于礼乎?」对曰:``鲁侯焉知礼?」公曰:``何为?自郊劳至于赠贿,礼无违者,何故不知?」对曰:``是仪也,不可谓礼。礼所以守其国,行其政令,无失其民者也。今政令在家,不能取也。有子家羁,弗能用也。奸大国之盟,陵虐小国。利人之难,不知其私。公室四分,民食于他。思莫在公,不图其终。为国君,难将及身,不恤其所。礼这本末,将于此乎在,而屑屑焉习仪以亟。言善于礼,不亦远乎?君子谓:``叔侯于是乎知礼。」

晋韩宣子如楚送女,叔向为介。郑子皮、子大叔劳诸索氏。大叔谓叔向曰:``楚王汰侈已甚,子其戒之。」叔向曰:``汰侈已甚,身之灾也,焉能及人?若奉吾币帛,慎吾威仪,守之以信,行之以礼,敬始而思终,终无不复,从而不失仪,敬而不失威,道之以训辞,奉之以旧法,考之以先王,度之以二国,虽汰侈,若我何?」

及楚,楚子朝其大夫,曰:``晋,吾仇敌也。苟得志焉,无恤其他。今其来者,上卿、上大夫也。若吾以韩起为阍,以羊舌肸为司宫,足以辱晋,吾亦得志矣。可乎?」大夫莫对。薳启强曰:``可。苟有其备,何故不可?耻匹夫不可以无备,况耻国乎?是以圣王务行礼,不求耻人,朝聘有珪,享《兆见》有璋。小有述职,大有巡功。设机而不倚,爵盈而不饮;宴有好货,飧有陪鼎,入有郊劳,出有赠贿,礼之至也。国家之败,失之道也,则祸乱兴。城濮之役,晋无楚备,以败于邲。邲之役,楚无晋备,以败于鄢。自鄢以来,晋不失备,而加之以礼,重之以睦,是以楚弗能报而求亲焉。既获姻亲,又欲耻之,以召寇仇,备之若何?谁其重此?若有其人,耻之可也。若其未有,君亦图之。晋之事君,臣曰可矣:求诸侯而麇至;求昏而荐女,君亲送之,上卿及上大夫致之。犹欲耻之,君其亦有备矣。不然,奈何?韩起之下,赵成、中行吴、魏舒、范鞅、知盈;羊舌肸之下,祁午、张趯、籍谈、女齐、梁丙、张骼、辅跞、苗贲皇,皆诸侯之选也。韩襄为公族大夫,韩须受命而使矣。箕襄、邢带、叔禽、叔椒、子羽,皆大家也。韩赋七邑,皆成县也。羊舌四族,皆强家也。晋人若丧韩起、杨肸,五卿八大夫辅韩须、杨石,因其十家九县,长毂九百,其馀四十县,遗守四千,奋其武怒,以报其大耻,伯华谋之,中行伯、魏舒帅之,其蔑不济矣。君将以亲易怨,实无礼以速寇,而未有其备,使群臣往遗之禽,以逞君心,何不可之有?」王曰:``不谷之过也,大夫无辱。」厚为韩子礼。王欲敖叔向以其所不知,而不能,亦厚其礼。

韩起反,郑伯劳诸圉。辞不敢见,礼也。

郑罕虎如齐,娶于子尾氏。晏子骤见之,陈桓子问其故,对曰:``能用善人,民之主也。」

夏,莒牟夷以牟娄及防兹来奔。牟夷非卿而书,尊地也。莒人愬于晋。晋侯欲止公,范献子曰:``不可。人朝而执之,诱也。讨不以师,而诱以成之,惰也。为盟主而犯此二者,无乃不可乎?请归之,间而以师讨焉。」乃归公。秋七月,公至自晋。

莒人来讨,不设备。戊辰,叔弓败诸□分泉,莒未陈也。

冬十月,楚子以诸侯及东夷伐吴,以报棘、栎、麻之役。薳射以繁扬之师,会于夏汭。越大夫常寿过帅师会楚子于琐。闻吴师出,薳启强帅师从之,遽不设备,吴人败诸鹊岸。

楚子以馹至于罗汭。吴子使其弟蹶由犒师,楚人执之,将以衅鼓。王使问焉,曰:``女卜来吉乎?」对曰:``吉。寡君闻君将治兵于敝邑,卜之以守龟,曰:『余亟使人犒师,请行以观王怒之疾徐,而为之备,尚克知之。』龟兆告吉,曰:『克可知也。』君若欢焉,好逆使臣,滋邑休殆,而忘其死,亡无日矣。今君奋焉,震电冯怒,虐执使臣,将以衅鼓,则吴知所备矣。敝邑虽羸,若早修完,其可以息师。难易有备,可谓吉矣。且吴社稷是卜,岂为一人?使臣获衅军鼓,而敝邑知备,以御不虞,其为吉孰大焉?国之守龟,其何事不卜?一臧一否,其谁能常之?城濮之兆,其报在邲。今此行也,其庸有报志?」乃弗杀。

楚师济于罗汭,沈尹赤会楚子,次于莱山。薳射帅繁扬之师,先入南怀,楚师从之。及汝清,吴不可入。楚子遂观兵于坻箕之山。是行也,吴早设备,楚无功而还,以蹶由归。楚子惧吴,使沈尹射待命于巢。薳启强待命于雩娄。礼也。

秦后子复归于秦,景公卒故也。

\hypertarget{header-n2464}{%
\subsubsection{昭公六年}\label{header-n2464}}

【经】六年春王正月,杞伯益姑卒。葬秦景公。夏,季孙宿如晋。葬杞文公。宋华合比出奔卫。秋九月,大雩。楚薳罢帅师伐吴。冬,叔弓如楚。齐侯伐北燕。

【传】六年春,王正月,杞文公卒,吊如同盟,礼也。大夫如秦,葬景公,礼也。

三月,郑人铸刑书。叔向使诒子产书,曰:``始吾有虞于子,今则已矣。昔先王议事以制,不为刑辟,惧民之有争心也。犹不可禁御,是故闲之以义,纠之以政,行之以礼,守之以信,奉之以仁,制为禄位以劝其从,严断刑罚以威其淫。惧其未也,故诲之以忠,耸之以行,教之以务,使之以和,临之以敬,莅之以强,断之以刚。犹求圣哲之上,明察之官,忠信之长,慈惠之师,民于是乎可任使也,而不生祸乱。民知有辟,则不忌于上,并有争心,以征于书,而徼幸以成之,弗可为矣。夏有乱政而作《禹刑》,商有乱政而作《汤刑》,周有乱政而作《九刑》,三辟之兴,皆叔世也。今吾子相郑国,作封洫,立谤政,制参辟,铸刑书,将以靖民,不亦难乎?《诗》曰:『仪式刑文王之德,日靖四方。』又曰:『仪刑文王,万邦作孚。』如是,何辟之有?民知争端矣,将弃礼而征于书。锥刀之末,将尽争之。乱狱滋丰,贿赂并行,终子之世,郑其败乎!肸闻之,国将亡,必多制,其此之谓乎!」复书曰:``若吾子之言,侨不才,不能及子孙,吾以救世也。既不承命,敢忘大惠?」

士文伯曰:``火见,郑其火乎?火未出而作火以铸刑器,藏争辟焉。火如象之,不火何为?」

夏,季孙宿如晋,拜莒田也。晋侯享之,有加笾。武子退,使行人告曰:``小国之事大国也,苟免于讨,不敢求贶。得贶不过三献。今豆有加,下臣弗堪,无乃戾也。」韩宣子曰:``寡君以为欢也。」对曰:``寡君犹未敢,况下臣,君之隶也,敢闻加贶?」固请彻加而后卒事。晋人以为知礼,重其好货。

宋寺人柳有宠,大子佐恶之。华合比曰:``我杀之。」柳闻之,乃坎、用牲、埋书,而告公曰:``合比将纳亡人之族,既盟于北郭矣。」公使视之,有焉,遂逐华合比,合比奔卫。于是华亥欲代右师,乃与寺人柳比,从为之征,曰``闻之久矣。」公使代之,见于左师,左师曰:``女夫也。必亡!女丧而宗室,于人何有?人亦于女何有?《诗》曰:『宗子维城,毋俾城坏,毋独斯畏。』女其畏哉!」

六月丙戌,郑灾。

楚公子弃疾如晋,报韩子也。过郑,郑罕虎、公孙侨、游吉从郑伯以劳诸柤。辞不敢见,固请见之,见,如见王,以其乘马八匹私面。见子皮如上卿,以马六匹。见子产,以马四匹。见子大叔,以马二匹。禁刍牧采樵,不入田,不樵树,不采刈,不抽屋,不强丐。誓曰:``有犯命者,君子废,小人降。」舍不为暴,主不慁宾。往来如是。郑三卿皆知其将为王也。

韩宣子之适楚也,楚人弗逆。公子弃疾及晋竟,晋侯将亦弗逆。叔向曰:``楚辟我衷,若何效辟?《诗》曰:『尔之教矣,民胥效矣。』从我而已,焉用效人之辟?《书》曰:『圣作则。』无宁以善人为则,而则人之辟乎?匹夫为善,民犹则之,况国君乎?」晋侯说,乃逆之。

秋九月,大雩,旱也。

徐仪楚聘于楚。楚子执之,逃归。惧其叛也,使薳泄伐徐。吴人救之。令尹子荡帅师伐吴,师于豫章,而次于乾溪。吴人败其师于房钟,获宫厩尹弃疾。子荡归罪于薳泄而杀之。

冬,叔弓如楚聘,且吊败也。

十一月,齐侯如晋,请伐北燕也。士□相士鞅,逆诸河,礼也。晋侯许之。十二月,齐侯遂伐北燕,将纳简公。晏子曰:``不入。燕有君矣,民不贰。吾君贿,左右谄谀,作大事不以信,未尝可也。」

\hypertarget{header-n2480}{%
\subsubsection{昭公七年}\label{header-n2480}}

【经】七年春王正月,暨齐平。三月,公如楚。叔孙婼如齐莅盟。夏四月甲辰朔,日有食之。秋八月戊辰,卫侯恶卒。九月,公至自楚。冬十有一月癸未,季孙宿卒。十有二月癸亥,葬卫襄公。

【传】七年春,王正月,暨齐平,齐求之也。癸巳,齐侯次于虢。燕人行成,曰:``敝邑知罪,敢不听命?先君之敝器,请以谢罪。」公孙皙曰:``受服而退,俟衅而动,可也。」二月戊午,盟于濡上。燕人归燕姬,赂以瑶瓮、玉椟、斗耳,不克而还。

楚子之为令尹也,为王旌以田。芋尹无宇断之,曰:``一国两君,其谁堪之?」及即位,为章华之宫,纳亡人以实之。无宇之阍入焉。无宇执之,有司弗与,曰:``执人于王宫,其罪大矣。」执而谒诸王。王将饮酒,无宇辞曰:``天子经略,诸侯正封,古之制也。封略之内,何非君土?食土之毛,谁非君臣?故《诗》曰:『普天之下,莫非王土。率土之滨,莫非王臣。』天有十日,人有十等,下所以事上,上所以共神也。故王臣公,公臣大夫,大夫臣士,士臣皂,皂臣舆,舆臣隶,隶臣僚,僚臣仆,仆臣台。马有圉,牛有牧,以待百事。今有司曰:『女胡执人于王宫?』将焉执之?周文王之法曰:『有亡,荒阅』,所以得天下也。吾先君文王,作仆区之法,曰:『盗所隐器,与盗同罪』,所以封汝也。若从有司,是无所执逃臣也。逃而舍之,是无陪台也。王事无乃阙乎?昔武王数纣之罪,以告诸侯曰:『纣为天下逋逃主,萃渊薮』,故夫致死焉。君王始求诸侯而则纣,无乃不可乎?若以二文之法取之,盗有所在矣。」王曰:``取而臣以往,盗有宠,未可得也。」遂赦之。

楚子成章华之台,愿与诸侯落之。大宰薳启强曰:``臣能得鲁侯。」薳启强来召公,辞曰:``昔先君成公,命我先大夫婴齐曰:『吾不忘先君之好,将使衡父照临楚国,镇抚其社稷,以辑宁尔民』。婴齐受命于蜀,奉承以来,弗敢失陨,而致诸宗祧。日我先君共王,引领北望,日月以冀。传序相授,于今四王矣。嘉惠未至,唯襄公之辱临我丧。孤与其二三臣,悼心失图,社稷之不皇,况能怀思君德!今君若步玉趾,辱见寡君,宠灵楚国,以信蜀之役,致君之嘉惠,是寡君既受贶矣,何蜀之敢望?其先君鬼神,实嘉赖之,岂唯寡君?君若不来,使臣请问行期,寡君将承质币而见于蜀,以请先君之贶。」

公将往,梦襄公祖。梓慎曰:``君不果行。襄公之适楚也,梦周公祖而行。今襄公实祖,君其不行。」子服惠伯曰:``行。先君未尝适楚,故周公祖以道之。襄公适楚矣,而祖以道君,不行,何之?」

三月,公如楚,郑伯劳于师之梁。孟僖子为介,不能相仪。及楚,不能答郊劳。

夏四月甲辰朔,日有食之。晋侯问于士文伯曰:``谁将当日食?」对曰:``鲁、卫恶之,卫大鲁小。」公曰:``何故?」对曰:``去卫地,如鲁地。于是有灾,鲁实受之。其大咎,其卫君乎?鲁将上卿。」公曰:``《诗》所谓『彼日而食,于何不臧』者,何也?」对曰:``不善政之谓也。国无政,不用善,则自取谪于日月之灾,故政不可不慎也。务三而已,一曰择人,二曰因民,三曰从时。」

晋人来治杞田,季孙将以成与之。谢息为孟孙守,不可。曰:``人有言曰:『虽有挈瓶之知,守不假器,礼也』。夫子从君,而守臣丧邑,虽吾子亦有猜焉。」季孙曰:``君之在楚,于晋罪也。又不听晋,鲁罪重矣。晋师必至,吾无以待之,不如与之,间晋而取诸杞。吾与子桃,成反,谁敢有之?是得二成也。鲁无忧而孟孙益邑,子何病焉?」辞以无山,与之莱、柞,乃迁于桃。晋人为杞取成。

楚子享公于新台,使长鬣者相,好以大屈。既而悔之。薳启强闻之,见公。公语之,拜贺。公曰:``何贺?对曰:``齐与晋、越欲此久矣。寡君无适与也,而传诸君,君其备御三邻。慎守宝矣,敢不贺乎?」公惧,乃反之。

郑子产聘于晋。晋侯疾,韩宣子逆客,私焉,曰:``寡君寝疾,于今三月矣,并走群望,有加而无瘳。今梦黄熊入于寝门,其何厉鬼也?」对曰:``以君之明,子为大政,其何厉之有?昔尧殛鲧于羽山,其神化为黄熊,以入于羽渊,实为夏郊,三代祀之。晋为盟主,其或者未之祀也乎?」韩子祀夏郊,晋侯有间,赐子产莒之二方鼎。

子产为丰施归州田于韩宣子,曰:``日君以夫公孙段为能任其事,而赐之州田,今无禄早世,不获久享君德。其子弗敢有,不敢以闻于君,私致诸子。」宣子辞。子产曰:``古人有言曰:『其父析薪,其子弗克负荷』。施将惧不能任其先人之禄,其况能任大国之赐?纵吾子为政而可,后之人若属有疆场之言,敝邑获戾,而丰氏受其大讨。吾子取州,是免敝邑于戾,而建置丰氏也。敢以为请。」宣子受之,以告晋侯。晋侯以与宣子。宣子为初言,病有之,以易原县于乐大心。

郑人相惊以伯有,曰``伯有至矣」,则皆走,不知所往。铸刑书之岁二月,或梦伯有介而行,曰:``壬子,余将杀带也。明年壬寅,余又将杀段也。」及壬子,驷带卒,国人益惧。齐、燕平之月壬寅,公孙段卒。国人愈惧。其明月,子产立公孙泄及良止以抚之,乃止。子大叔问其故,子产曰:``鬼有所归,乃不为厉,吾为之归也。」大叔曰:``公孙泄何为?」子产曰:``说也。为身无义而图说,从政有所反之,以取媚也。不媚,不信。不信,民不从也。」

及子产适晋,赵景子问焉,曰:``伯有犹能为鬼乎?」子产曰:``能。人生始化曰魄,既生魄,阳曰魂。用物精多,则魂魄强。是以有精爽,至于神明。匹夫匹妇强死,其魂魄犹能冯依于人,以为淫厉,况良霄,我先君穆公之胄,子良之孙,子耳之子,敝邑之卿,从政三世矣。郑虽无腆,抑谚曰『蕞尔国』,而三世执其政柄,其用物也弘矣,其取精也多矣。其族又大,所冯厚矣。而强死,能为鬼,不亦宜乎?」

子皮之族饮酒无度,故马师氏与子皮氏有恶。齐师还自燕之月,罕朔杀罕魋。罕朔奔晋。韩宣子问其位于子产。子产曰:``君之羁臣,苟得容以逃死,何位之敢择?卿违,从大夫之位,罪人以其罪降,古之制也。朔于敝邑,亚大夫也,其官,马师也。获戾而逃,唯执政所置之。得免其死,为惠大矣,又敢求位?」宣子为子产之敏也,使从嬖大夫。

秋八月,卫襄公卒。晋大夫言于范献子曰:``卫事晋为睦,晋不礼焉,庇其贼人而取其地,故诸侯贰。《诗》曰:『即□鴒在原,兄弟急难。』又曰:『死丧之威,兄弟孔怀。』兄弟之不睦,于是乎不吊,况远人,谁敢归之?今又不礼于卫之嗣,卫必叛我,是绝诸侯也。」献子以告韩宣子。宣子说,使献子如卫吊,且反戚田。

卫齐恶告丧于周,且请命。王使臣简公如卫吊,且追命襄公曰:``叔父陟恪,在我先王之左右,以佐事上帝。余敢高圉、亚圉?」

九月,公至自楚。孟僖子病不能相礼,乃讲学之,苟能礼者从之。及其将死也,召其大夫曰:``礼,人之干也。无礼,无以立。吾闻将有达者曰孔丘,圣人之后也,而灭于宋。其祖弗父何,以有宋而授厉公。及正考父,佐戴、武、宣,三命兹益共。故其鼎铭云:『一命而偻,再命而伛,三命而俯。循墙而走,亦莫余敢侮。饘是,鬻于是,以糊余口。』其共也如是。臧孙纥有言曰:『圣人有明德者,若不当世,其后必有达人。』今其将在孔丘乎?我若获没,必属说与何忌于夫子,使事之,而学礼焉,以定其位。」故孟懿子与南宫敬叔师事仲尼。仲尼曰:``能补过者,君子也。《诗》曰:『君子是则是效。』孟僖子可则效已矣。」

单献公弃亲用羁。冬十月辛酉,襄、顷之族杀献公而立成公。

十一月,季武子卒。晋侯谓伯瑕曰:``吾所问日食,从矣,可常乎?」对曰:``不可。六物不同,民心不一,事序不类,官职不则,同始异终,胡可常也?《诗》曰:『或燕燕居息,或憔悴事国。』其异终也如是。」公曰:``何谓六物?」对曰:``岁、时、日、月、星、辰,是谓也。」公曰:``多语寡人辰,而莫同。何谓辰?」对曰:``日月之会,是谓辰,故以配日。」

卫襄公夫人姜氏无子,嬖人婤姶生孟絷。孔成子梦康叔谓己:``立元,余使羁之孙圉与史苟相之。」史朝亦梦康叔谓己:``余将命而子苟与孔烝锄之曾孙圉相元。」史朝见成子,告之梦,梦协。晋韩宣子为政聘于诸侯之岁,婤姶生子,名之曰元。孟絷之足不良,能行。孔成子以《周易》筮之,曰:``元尚享卫国主其社稷。」遇《屯》三。又曰:``余尚立絷,尚克嘉之。」遇《屯》三之《比三。以示史朝。史朝曰:『元亨』,又何疑焉?」成子曰:``非长之谓乎?」对曰:``康叔名之,可谓长矣。孟非人也,将不列于宗,不可谓长。且其繇曰『利建侯』。嗣吉,何建?建非嗣也。二卦皆云,子其建之。康叔命之,二筮袭于梦,武王所用也,弗从何为?弱足者居,侯主社稷,临祭祀,奉民人,事民人,鬼神,从会朝,又焉得居?各以所利,不亦可乎?」故孔成子立灵公。十二月癸亥,葬卫襄公。

\hypertarget{header-n2503}{%
\subsubsection{昭公八年}\label{header-n2503}}

【经】八年春,陈侯之弟招杀陈世子偃师,夏四月辛丑,陈侯溺卒。叔弓如晋。楚人执陈行人干征师杀之。陈公子留出奔郑。秋,蒐于红。陈人杀其大夫公子过。大雩,冬十月壬午,楚师灭陈。执陈公子招,放之于越。杀陈孔奂。葬陈哀公。

【传】八年春,石言于晋魏榆。晋侯问于师旷曰:``石何故言?」对曰:``石不能言,或冯焉。不然,民听滥也。抑臣又闻之曰:『作事不时,怨讟动于民,则有非言之物而言。』今宫室崇侈,民力凋尽,怨讟并作,莫保其性。石言,不亦宜乎?」于是晋侯方筑虒祁之宫。叔向曰:``子野之言,君子哉!君子之言,信而有徵,故怨远于其身。小人之言,僭而无征,故怨咎及之。《诗》曰:『哀哉不能言,匪舌是出,唯躬是瘁。哿矣能言,巧言如流,俾躬处休。』其是之谓乎?是宫也成,诸侯必叛,君必有咎,夫子知之矣。」

陈哀公元妃郑姬,生悼大子偃师,二妃生公子留,下妃生公子胜。二妃嬖,留有宠,属诸徒招与公子过。哀公有废疾。三月甲申,公子招、公子过杀悼大子偃师,而立公子留。

夏四月辛亥,哀公缢。干征师赴于楚,且告有立君。公子胜愬之于楚,楚人执而杀之。公子留奔郑。书曰``陈侯之弟招杀陈世子偃师」,罪在招也;``楚人执陈行人干征师杀之」,罪不在行人也。

叔弓如晋,贺虒祁也。游吉相郑伯以如晋,亦贺虒祁也。史赵见子大叔,曰:``甚哉,其相蒙也!可吊也,而又贺之?」子大叔曰:``若何吊也?其非唯我贺,将天下实贺。」

秋,大蒐于红,自根牟至于商、卫,革车千乘。

七月甲戌,齐子尾卒,子旗欲治其室。丁丑,杀梁婴。八月庚戌,逐子成、子工、子车,皆来奔,而立子良氏之宰。其臣曰:``孺子长矣,而相吾室,欲兼我也。」授甲,将攻之。陈桓子善于子尾,亦授甲,将助之。或告子旗,子旗不信。则数人告。将往,又数人告于道,遂如陈氏。桓子将出矣,闻之而还,游服而逆之。请命,对曰:``闻强氏授甲将攻子,子闻诸?」曰:``弗闻。」``子盍亦授甲?无宇请从。」子旗曰:``子胡然?彼孺子也,吾诲之犹惧其不济,吾又宠秩之。其若先人何?子盍谓之?《周书》曰:『惠不惠,茂不茂。』康叔所以服弘大也。」桓子稽颡曰:``顷、灵福子,吾犹有望。」遂和之如初。

陈公子招归罪于公子过而杀之。九月,楚公子弃疾帅师奉孙吴围陈,宋戴恶会之。冬十一月壬午,灭陈。舆嬖袁克,杀马毁玉以葬。楚人将杀之,请置之。既又请私,私于幄,加絰于颡而逃。使穿封戌为陈公,曰:``城麇之役,不谄。」侍饮酒于王,王曰:``城麇之役,女知寡人之及此,女其辟寡人乎?」对曰:``若知君之及此,臣必致死礼,以息楚。」晋侯问于史赵,曰:``陈其遂亡乎?」对曰:``未也。」公曰:``何故?」对曰:``陈,颛顼之族也。岁在鹑火,是以卒灭,陈将如之。今在析木之津,犹将复由。且陈氏得政于齐而后陈卒亡。自幕至于瞽瞍,无违命。舜重之以明德,置德于遂,遂世守之。及胡公不淫,胡周赐之姓,使祀虞帝。臣闻盛德必百世祀,虞之世数未也。继守将在齐,其兆既存矣。」

\hypertarget{header-n2514}{%
\subsubsection{昭公九年}\label{header-n2514}}

【经】九年春,叔弓会楚子于陈。许迁于夷。夏四月,陈灾。秋,仲孙玃如齐。冬,筑郎囿。

【传】九年春,叔弓、宋华亥、郑游吉、卫赵□会楚子于陈。

二月庚申,楚公子弃疾迁许于夷,实城父,取州来淮北之田以益之。伍举授许男田。然丹迁城父人于陈,以夷濮西田益之。迁方城外人于许。

周甘人与晋阎嘉争阎田。晋梁丙、张趯率阴戎伐颖。王使詹桓伯辞于晋曰:``我自夏以后稷,魏、骀、芮、岐、毕,吾西土也。及武王克商,蒲姑、商奄,吾东土也;巴、濮、楚、邓,吾南土也;肃慎、燕、亳,吾北土也。吾何迩封之有?文、武、成、康之建母弟,以蕃屏周,亦其废队是为,岂如弁髦而因以敝之?先王居檮杌于四裔,以御螭魅,故允姓之奸,居于瓜州,伯父惠公归自秦,而诱以来,使逼我诸姬,入我郊甸,则戎焉取之。戎有中国,谁之咎也?后稷封殖天下,今戎制之,不亦难乎?伯父图之。我在伯父,犹衣服之有冠冕,木水之有本原,民人之有谋主也。伯父若裂冠毁冕,拔本塞原,专弃谋主,虽戎狄其何有馀一人?」叔向谓宣子曰:``文之伯也,岂能改物?翼戴天子而加之以共。自文以来,世有衰德而暴灭宗周,以宣示其侈,诸侯之贰,不亦宜乎?且王辞直,子其图之。」宣子说。

王有姻丧,使赵成如周吊,且致阎田与襚,反颖俘。王亦使宾滑执甘大夫襄以说于晋,晋人礼而归之。

夏四月,陈灾。郑裨灶曰:``五年,陈将复封。封五十二年而遂亡。」子产问其故,对曰:``陈,水属也,火,水妃也,而楚所相也。今火出而火陈,逐楚而建陈也。妃以五成,故曰五年。岁五及鹑火,而后陈卒亡,楚克有之,天之道也,故曰五十二年。」

晋荀盈如齐逆女,还,六月,卒于戏阳。殡于绛,未葬。晋侯饮酒,乐。膳宰屠蒯趋入,请佐公使尊,许之。而遂酌以饮工,曰:``女为君耳,将司聪也。辰在子卯,谓之疾日。君彻宴乐,学人舍业,为疾故也。君之卿佐,是谓股肱。股肱或亏,何痛如之?女弗闻而乐,是不聪也。」又饮外嬖嬖叔曰:``女为君目,将司明也。服以旌礼,礼以行事,事有其物,物有其容。今君之容,非其物也,而女不见。是不明也。」亦自饮也,曰:``味以行气,气以实志,志以定言,言以出令。臣实司味,二御失官,而君弗命,臣之罪也。」公说,彻酒。

初,公欲废知氏而立其外嬖,为是悛而止。秋八月,使荀跞佐下军以说焉。

孟僖子如齐殷聘,礼也。

冬,筑郎囿,书,时也。季平子欲其速成也,叔孙昭子曰:``《诗》曰:『经始勿亟,庶民子来。』焉用速成?其以剿民也?无囿犹可,无民其可乎?」

\hypertarget{header-n2527}{%
\subsubsection{昭公十年}\label{header-n2527}}

【经】十年春王正月。夏,齐栾施来奔。秋七月,季孙意如、叔弓、仲孙玃帅师伐莒。戊子,晋侯彪卒。九月,叔孙婼如晋,葬晋平公。十有二月甲子,宋公成卒。

【传】十年春,王正月,有星出于婺女。郑裨灶言于子产曰:``七月戊子,晋君将死。今兹岁在颛顼之虚,姜氏、任氏实守其地。居其维首,而有妖星焉,告邑姜也。邑姜,晋之妣也。天以七纪。戊子,逢公以登,星斯于是乎出。吾是以讥之。」

齐惠栾、高氏皆耆酒,信内多怨,强于陈、鲍氏而恶之。

夏,有告陈桓子曰:``子旗、子良将攻陈、鲍。」亦告鲍氏。桓子授甲而如鲍氏,遭子良醉而骋,遂见文子,则亦授甲矣。使视二子,则皆从饮酒。桓子曰:``彼虽不信,闻我授甲,则必逐我。及其饮酒也,先伐诸?」陈、鲍方睦,遂伐栾、高氏。子良曰:``先得公,陈、鲍焉往?」遂伐虎门。

晏平仲端委立于虎门之外,四族召之,无所往。其徒曰:``助陈、鲍乎?」曰:``何善焉?」``助栾、高乎?」曰:``庸愈乎?」``然则归乎?」曰:``君伐,焉归?」公召之而后入。公卜使王黑以灵姑金ぶ率,吉,请断三尺焉而用之。五月庚辰,战于稷,栾、高败,又败诸庄。国人追之,又败诸鹿门。栾施、高强来奔。陈、鲍分其室。

晏子谓桓子:``必致诸公。让,德之主也,谓懿德。凡有血气,皆有争心,故利不可强,思义为愈。义,利之本也,蕴利生孽。姑使无蕴乎!可以滋长。」桓子尽致诸公,而请老于莒。

桓子召子山,私具幄幕、器用、从者之衣屦,而反棘焉。子商亦如之,而反其邑。子周亦如之,而与之夫于。反子城、子公、公孙捷,而皆益其禄。凡公子、公孙之无禄者,私分之邑。国之贫约孤寡者,私与之粟。曰:``《诗》云:『陈锡载周』,能施也,桓公是以霸。」

公与桓子莒之旁邑,辞。穆孟姬为之请高唐,陈氏始大。秋七月,平子伐莒,取郠,献俘,始用人于亳社。臧武仲在齐,闻之,曰:``周公其不飨鲁祭乎!周公飨义,鲁无义。《诗》曰:『德音孔昭,视民不佻。』佻之谓甚矣,而壹用之,将谁福哉?」

戊子,晋平公卒。郑伯如晋,及河,晋人辞之。游吉遂如晋。九月,叔孙婼、齐国弱、宋华定、卫北宫喜、郑罕虎、许人、曹人、莒人、邾人、薛人、杞人、小邾人如晋,葬平公也。郑子皮将以币行。子产曰:``丧焉用币?用币必百两,百两必千人,千人至,将不行。不行,必尽用之。几千人而国不亡?」子皮固请以行。既葬,诸侯之大夫欲因见新君。叔孙昭子曰:``非礼也。」弗听。叔向辞之,曰:``大夫之事毕矣。而又命孤,孤斩焉在衰絰之中。其以嘉服见,则丧礼未毕。其以丧服见,是重受吊也。大夫将若之何?」皆无辞以见。子皮尽用其币,归,谓子羽曰:``非知之实难,将在行之。夫子知之矣,我则不足。《书》曰:『欲败度,纵败礼。』我之谓矣。夫子知度与礼矣,我实纵欲而不能自克也。」

昭子至自晋,大夫皆见。高强见而退。昭子语诸大夫曰:``为人子,不可不慎也哉!昔庆封亡,子尾多受邑而稍致诸君,君以为忠而甚宠之。将死,疾于公宫,辇而归,君亲推之。其子不能任,是以在此。忠为令德,其子弗能任,罪犹及之,难不慎也?丧夫人之力,弃德旷宗,以及其身,不亦害乎?《诗》曰:『不自我先,不自我后。』其是之谓乎!」

冬十二月,宋平公卒。初,元公恶寺人柳。欲杀之。及丧,柳炽炭于位,将至,则去之。比葬,又有宠。

\hypertarget{header-n2541}{%
\subsubsection{昭公十一年}\label{header-n2541}}

【经】十有一年春王二月,叔弓如宋。葬宋平公。夏四月丁巳,楚子虔诱蔡侯般杀之于申。楚公子弃疾帅师围蔡。五月甲申,夫人归氏薨。大蒐于比蒲。仲孙玃会邾子,盟于祲祥。秋,季孙意如会晋韩起、齐国弱、宋华亥、卫北宫佗、郑罕虎、曹人、杞人于厥憖。九月己亥,葬我小君齐归。冬十有一月丁酉,楚师灭蔡,执蔡世子有以归,用之。

【传】十一年春,王二月,叔弓如宋,葬平公也。

景王问于苌弘曰:``今兹诸侯,何实吉?何实凶?」对曰:``蔡凶。此蔡侯般弑其君之岁也,岁在豕韦,弗过此矣。楚将有之,然壅也。岁及大梁,蔡复,楚凶,天之道也。」

楚子在申,召蔡灵侯。灵侯将往,蔡大夫曰:``王贪而无信,唯蔡于感,今币重而言甘,诱我也,不如无往。」蔡侯不可。五月丙申,楚子伏甲而飨蔡侯于申,醉而执之。夏四月丁巳,杀之,刑其士七十人。公子弃疾帅师围蔡。

韩宣子问于叔向曰:``楚其克乎?」对曰:``克哉!蔡侯获罪于其君,而不能其民,天将假手于楚以毙之,何故不克?然肸闻之,不信以幸,不可再也。楚王奉孙吴以讨于陈,曰:『将定而国。』陈人听命,而遂县之。今又诱蔡而杀其君,以围其国,虽幸而克,必受其咎,弗能久矣。桀克有婚以丧其国,纣克东夷而陨其身。楚小位下,而亟暴于二王,能无咎乎?天之假助不善,非祚之也,厚其凶恶而降之罚也。且譬之如天,其有五材而将用之,力尽而敝之,是以无拯,大可没振。」

五月,齐归薨,大蒐于比蒲,非礼也。

孟僖子会邾庄公,盟于祲祥,修好,礼也。泉丘人有女梦以其帷幕孟氏之庙,遂奔僖子,其僚从之。盟于清丘之社,曰:``有子,无相弃也。」僖子使助薳氏之簉。反自祲祥,宿于薳氏,生懿子及南宫敬叔于泉丘人。其僚无子,使字敬叔。

楚师在蔡,晋荀吴谓韩宣子曰:``不能救陈,又不能救蔡,物以无亲,晋之不能,亦可知也已!为盟主而不恤亡国,将焉用之?」

秋,会于厥憖,谋救蔡也。郑子皮将行,子产曰:``行不远。不能救蔡也。蔡小而不顺,楚大而不德,天将弃蔡以壅楚,盈而罚之。蔡必亡矣,且丧君而能守者,鲜矣。三年,王其有咎乎!美恶周必复,王恶周矣。」晋人使狐父请蔡于楚,弗许。

单子会韩宣子于戚,视下言徐。叔向曰:``单子其将死乎!朝有着定,会有表,衣有禬带有结。会朝之言,必闻于表着之位,所以昭事序也。视不过结、禬之中,所以道容貌也。言以命之,容貌以明之,失则有阙。今单子为王官伯,而命事于会,视不登带,言不过步,貌不道容,而言不昭矣。不道,不共;不昭,不从。无守气矣。」

九月,葬齐归,公不戚。晋士之送葬者,归以语史赵。史赵曰:``必为鲁郊。」侍者曰:``何故?」曰:``归姓也,不思亲,祖不归也。」叔向曰:``鲁公室其卑乎?君有大丧,国不废蒐。有三年之丧,而无一日之戚。国不恤丧,不忌君也。君无戚容,不顾亲也。国不忌君,君不顾亲,能无卑乎?殆其失国。」

冬十一月,楚子灭蔡,用隐大子于冈山。申无宇曰:``不祥。五牲不相为用,况用诸侯乎?王必悔之。」

十二月,单成公卒。

楚子城陈、蔡、不羹。使弃疾为蔡公。王问于申无宇曰:``弃疾在蔡,何如?」对曰:``择子莫如父,择臣莫如君。郑庄公城栎而置子元焉,使昭公不立。齐桓公城谷而置管仲焉,至于今赖之。臣闻五大不在边,五细不在庭。亲不在外,羁不在内,今弃疾在外,郑丹在内。君其少戒。」王曰:``国有大城,何如?」对曰:``郑京、栎实杀曼伯,宋萧、亳实杀子游,齐渠丘实杀无知,卫蒲、戚实出献公,若由是观之,则害于国。末大必折,尾大不掉,君所知也。」

\hypertarget{header-n2558}{%
\subsubsection{昭公十二年}\label{header-n2558}}

【经】十有二年春,齐高偃帅师纳北燕伯于阳。三月壬申,郑伯嘉卒。夏,宋公使华定来聘。公如晋,至河乃复。五月,葬郑简公。楚杀其大夫成熊。秋七月。冬十月,公子憖出奔齐。楚子伐徐。晋伐鲜虞。

【传】十二年春,齐高偃纳北燕伯款于唐,因其众也。

三月,郑简公卒,将为葬除。及游氏之庙,将毁焉。子大叔使其除徒执用以立,而无庸毁,曰:``子产过女,而问何故不毁,乃曰:『不忍庙也!诺,将毁矣!』」既如是,子产乃使辟之。司墓之室有当道者,毁之,则朝而塴;弗毁,则日中而塴。子大叔请毁之,曰:``无若诸侯之宾何!」子产曰:``诸侯之宾,能来会吾丧,岂惮日中?无损于宾,而民不害,何故不为?」遂弗毁,日中而葬。君子谓:``子产于是乎知礼。礼,无毁人以自成也。」

夏,宋华定来聘,通嗣君也。享之,为赋《蓼萧》,弗知,又不答赋。昭子曰:``必亡。宴语之不怀,宠光之不宣,令德之不知,同福之不受,将何以在?」

齐侯、卫侯、郑伯如晋,朝嗣君也。公如晋,至河乃复。取郠之役,莒人诉于晋,晋有平公之丧,未之治也,故辞公。公子憖遂如晋。晋侯享诸侯,子产相郑伯,辞于享,请免丧而后听命。晋人许之,礼也。晋侯以齐侯宴,中行穆子相。投壶,晋侯先。穆子曰:``有酒如淮,有肉如坻。寡君中此,为诸侯师。」中之。齐侯举矢,曰:``有酒如渑,有肉如陵。寡人中此,与君代兴。」亦中之。伯瑕谓穆子曰:``子失辞。吾固师诸侯矣,壶何为焉,其以中俊也?齐君弱吾君,归弗来矣!」穆子曰:``吾军帅强御,卒乘竞劝,今犹古也,齐将何事?」公孙叟趋进曰:``日旰君勤,可以出矣!」以齐侯出。

楚子谓成虎若敖之馀也,遂杀之。或谮成虎于楚子,成虎知之而不能行。书曰:``楚杀其大夫成虎。」怀宠也。

六月,葬郑简公。

晋荀吴伪会齐师者,假道于鲜虞,遂入昔阳。秋八月壬午,灭肥,以肥子绵皋归。

周原伯绞虐其舆臣,使曹逃。冬十月壬申朔,原舆人逐绞而立公子跪寻,绞奔郊。

甘简公无子,立其弟过。过将去成、景之族,成、景之族赂刘献公。丙申,杀甘悼公,而立成公之孙鳅。丁酉,杀献太子之傅庾皮之子过,杀瑕辛于市,及宫嬖绰、王孙没、刘州鸠、阴忌、老阳子。

季平子立,而不礼于南蒯。南蒯谓子仲:``吾出季氏,而归其室于公。子更其位。我以费为公臣。」子仲许之。南蒯语叔仲穆子,且告之故。

季悼子之卒也,叔孙昭子以再命为卿。及平子伐莒,克之,更受三命。叔仲子欲构二家,谓平子曰:``三命逾父兄,非礼也。」平子曰:``然。」故使昭子。昭子曰:``叔孙氏有家祸,杀适立庶,故婼也及此。若因祸以毙之,则闻命矣。若不废君命,则固有着矣。」昭子朝,而命吏曰:``婼将与季氏讼,书辞无颇。」季孙惧,而归罪于叔仲子。故叔仲小、南蒯、公子憖谋季氏。憖告公,而遂从公如晋。南蒯惧不克,以费叛如齐。子仲还,及卫,闻乱,逃介而先。及郊,闻费叛,遂奔齐。

南蒯之将叛也,其乡人或知之,过之而叹,且言曰:``恤恤乎,湫乎,攸乎!深思而浅谋,迩身而远志,家臣而君图,有人矣哉」南蒯枚筮之,遇《坤》三之《比》三,曰:``黄裳元吉。」以为大吉也,示子服惠伯,曰:``即欲有事,何如?」惠伯曰:``吾尝学此矣,忠信之事则可,不然必败。外强内温,忠也。和以率贞,信也。故曰『黄裳元吉』。黄,中之色也。裳,下之饰也。元,善之长也。中不忠,不得其色。下不共,不得其饰。事不善,不得其极。外内倡和为忠,率事以信为共,供养三德为善,非此三者弗当。且夫《易》,不可以占险,将何事也?且可饰乎?中美能黄,上美为元,下美则裳,参成可筮。犹有阙也,筮虽吉,未也。」

将适费,饮乡人酒。乡人或歌之曰:``我有圃,生之杞乎!从我者子乎,去我者鄙乎,倍其邻者耻乎!已乎已乎,非吾党之士乎!」

平子欲使昭子逐叔仲小。小闻之,不敢朝。昭子命吏谓小待政于朝,曰:``吾不为怨府。」楚子狩于州来,次于颖尾,使荡侯、潘子、司马督、嚣尹午、陵尹喜帅师围徐以惧吴。楚子次于乾溪,以为之援。雨雪,王皮冠,秦复陶,翠被,豹舄,执鞭以出,仆析父从。右尹子革夕,王见之,去冠、被,舍鞭,与之语曰:``昔我先王熊绎,与吕级、王孙牟、燮父、禽父,并事康王,四国皆有分,我独无有。今吾使人于周,求鼎以为分,王其与我乎?」对曰:``与君王哉!昔我先王熊绎,辟在荆山,筚路蓝缕,以处草莽。跋涉山林,以事天子。唯是桃弧、棘矢,以共御王事。齐,王舅也。晋及鲁、卫,王母弟也。楚是以无分,而彼皆有。今周与四国服事君王,将唯命是从,岂其爱鼎?」王曰:``昔我皇祖伯父昆吾,旧许是宅。今郑人贪赖其田,而不我与。我若求之,其与我乎?」对曰:``与君王哉!周不爱鼎,郑敢爱田?」王曰:``昔诸侯远我而畏晋,今我大城陈、蔡、不羹,赋皆千乘,子与有劳焉。诸侯其畏我乎?」对曰:``畏君王哉!是四国者,专足畏也,又加之以楚,敢不畏君王哉!」

工尹路请曰:``君王命剥圭以为金戚铋,敢请命。」王入视之。析父谓子革:``吾子,楚国之望也!今与王言如响,国其若之何?」子革曰:``摩厉以须,王出,吾刃将斩矣。」王出,复语。左史倚相趋过。王曰:``是良史也,子善视之。是能读《三坟》、《五典》、《八索》、《九丘》。」对曰:``臣尝问焉。昔穆王欲肆其心,周行天下,将皆必有车辙马迹焉。祭公谋父作《祈招》之诗,以止王心,王是以获没于祗宫。臣问其诗而不知也。若问远焉,其焉能知之?」王曰:``子能乎?」对曰:``能。其诗曰:『祈招之愔愔,式昭德音。思我王度,式如玉,式如金。形民之力,而无醉饱之心。』」王揖而入,馈不食,寝不寐,数日,不能自克,以及于难。

仲尼曰:``古也有志:『克己复礼,仁也』。信善哉!楚灵王若能如是,岂其辱于乾溪?」

晋伐鲜虞,因肥之役也。

\hypertarget{header-n2579}{%
\subsubsection{昭公十三年}\label{header-n2579}}

【经】十有三年春,叔弓帅师围费。夏四月,楚公子比自晋归于楚,杀其君虔于乾溪。楚公子弃疾杀公子比。秋,公会刘子、晋侯、宋公、卫侯、郑伯、曹伯、莒子、邾子、滕子、薛伯、杞伯、小邾子于平丘。八月甲戌,同盟于平丘。公不与盟。晋人执季孙意如以归。公至自会。蔡侯庐归于蔡。陈侯吴归于陈。冬十月,葬蔡灵公。公如晋,至河乃复。吴灭州来。

【传】十三年春,叔弓围费,弗克,败焉。平子怒,令见费人执之以为囚俘。冶区夫曰:``非也。若见费人,寒者衣之,饥者食之,为之令主,而共其乏困。费来如归,南氏亡矣,民将叛之,谁与居邑?若惮之以威,惧之以怒,民疾而叛,为之聚也。若诸侯皆然,费人无归,不亲南氏,将焉入矣?」平子从之,费人叛南氏。

楚子之为令尹也,杀大司马薳掩而取其室。及即位,夺薳居田;迁许而质许围。蔡洧有宠于王,王之灭蔡也,其父死焉,王使与于守而行。申之会,越大夫戮焉。王夺斗韦龟中犨,又夺成然邑而使为郊尹。蔓成然故事蔡公,故薳氏之族及薳居、许围、蔡洧、蔓成然,皆王所不礼也。因群丧职之族,启越大夫常寿过作乱,围固城,克息舟,城而居之。

观起之死也,其子从在蔡,事朝吴,曰:``今不封蔡,蔡不封矣。我请试之。」以蔡公之命召子干、子皙,及郊,而告之情,强与之盟,入袭蔡。蔡公将食,见之而逃。观从使子干食,坎,用牲,加书,而速行。己徇于蔡曰:``蔡公召二子,将纳之,与之盟而遣之矣,将师而从之。」蔡人聚,将执之。辞曰:``失贼成军,而杀余,何益?」乃释之。朝吴曰:``二三子若能死亡,则如违之,以待所济。若求安定,则如与之,以济所欲。且违上,何适而可?」众曰:``与之。」乃奉蔡公,召二子而盟于邓,依陈、蔡人以国。楚公子比、公子黑肱、公子弃疾、蔓成然、蔡朝吴帅陈、蔡、不羹、许、叶之师,因四族之徒,以入楚。及郊,陈、蔡欲为名,故请为武军。蔡公知之曰:``欲速。且役病矣,请藩而已。」乃藩为军。蔡公使须务牟与史卑先入,因正仆人杀大子禄及公子罢敌。公子比为王,公子黑肱为令尹,次于鱼陂。公子弃疾为司马,先除王宫。使观从从师于乾溪,而遂告之,且曰:``先归复所,后者劓。」师及訾梁而溃。

王闻群公子之死也,自投于车下,曰:``人之爱其子也,亦如余乎?」侍者曰:``甚焉。小人老而无子,知挤于沟壑矣。」王曰:``余杀人子多矣,能无及此乎?」右尹子革曰:``请待于郊,以听国人。」王曰:``众怒不可犯也。」曰:``若入于大都而乞师于诸侯。」王曰:``皆叛矣。」曰:``若亡于诸侯,以听大国之图君也。」王曰:``大福不再,只取辱焉。」然丹乃归于楚。王沿夏,将欲入鄢。芋尹无宇之子申亥曰:``吾父再奸王命,王弗诛,惠孰大焉?君不可忍,惠不可弃,吾其从王。」乃求王,遇诸棘围以归。夏五月癸亥,王缢于芋尹申亥氏。申亥以其二女殉而葬之。

观从谓子干曰:``不杀弃疾,虽得国,犹受祸也。」子干曰:``余不忍也。」子玉曰:``人将忍子,吾不忍俟也。」乃行。国每夜骇曰:``王入矣!」乙卯夜,弃疾使周走而呼曰:``王至矣!」国人大惊。使蔓成然走告子干、子皙曰:``王至矣!国人杀君司马,将来矣!君若早自图也,可以无辱。众怒如水火焉,不可为谋。」又有呼而走至者曰:``众至矣!」二子皆自杀。丙辰,弃疾即位,名曰熊居。葬子干于訾,实訾敖。杀囚,衣之王服而流诸汉,乃取而葬之,以靖国人。使子旗为令尹。

楚师还自徐,吴人败诸豫章,获其五帅。

平王封陈、蔡,复迁邑,致群赂,施舍宽民,宥罪举职。召观从,王曰:``唯尔所欲。」对曰:``臣之先,佐开卜。」乃使为卜尹。使枝如子躬聘于郑,且致犨、栎之田。事毕,弗致。郑人请曰:``闻诸道路,将命寡君以犨、栎,敢请命。」对曰:``臣未闻命。」既复,王问犨、栎。降服而对,曰:``臣过失命,未之致也。」王执其手,曰:``子毋勤。姑归,不谷有事,其告子也。」他年芋尹申亥以王柩告,乃改葬之。

初,灵王卜,曰:``余尚得天下。」不吉,投龟,诟天而呼曰:``是区区者而不馀畀,余必自取之。」民患王之无厌也,故从乱如归。

初,共王无冢适,有宠子五人,无适立焉。乃大有事于群望,而祈曰:``请神择于五人者,使主社稷。」乃遍以璧见于群望,曰:``当璧而拜者,神所立也,谁敢违之?」既,乃与巴姬密埋璧于大室之庭,使五人齐,而长入拜。康王跨之,灵王肘加焉,子干、子皙皆远之。平王弱,抱而入,再拜,皆厌纽。斗韦龟属成然焉,且曰:``弃礼违命,楚其危哉!」

子干归,韩宣子问于叔向曰:``子干其济乎?」对曰:``难。」宣子曰:``同恶相求,如市贾焉,何难?」对曰:``无与同好,谁与同恶?取国有五难:有宠而无人,一也;有人而无主,二也;有主而无谋,三也;有谋而无民,四也;有民而无德,五也。子干在晋十三年矣,晋、楚之从,不闻达者,可谓无人。族尽亲叛,可谓无主。无衅而动,可谓无谋。为羁终世,可谓无民。亡无爱征,可谓无德。王虐而不忌,楚君子干,涉五难以弑旧君,谁能济之?有楚国者,其弃疾乎!君陈、蔡,城外属焉。苛慝不作,盗贼伏隐,私欲不违,民无怨心。先神命之。国民信之,芈姓有乱,必季实立,楚之常也。获神,一也;有民,二也;令德,三也;宠贵,四也;居常,五也。有五利以去五难,谁能害之?子干之官,则右尹也。数其贵宠,则庶子也。以神所命,则又远之。其贵亡矣,其宠弃矣,民无怀焉,国无与焉,将何以立?」宣子曰:``齐桓、晋文,不亦是乎?」对曰:``齐桓,卫姬之子也,有宠于僖。有鲍叔牙、宾须无、隰朋以为辅佐,有莒、卫以为外主,有国、高以为内主。从善如流,下善齐肃,不藏贿,不从欲,施舍不倦,求善不厌,是以有国,不亦宜乎?我先君文公,狐季姬之子也,有宠于献。好学而不贰,生十七年,有士五人。有先大夫子余、子犯以为腹心,有魏犨、贾佗以为股肱,有齐、宋、秦、楚以为外主,有栾、郤、狐、先以为内主。亡十九年,守志弥笃。惠、怀弃民,民从而与之。献无异亲,民无异望,天方相晋,将何以代文?此二君者,异于子干。共有宠子,国有奥主。无施于民,无援于外,去晋而不送,归楚而不逆,何以冀国?」

晋成虒祁,诸侯朝而归者皆有贰心。为取郠故,晋将以诸侯来讨。叔向曰:``诸侯不可以不示威。」乃并征会,告于吴。秋,晋侯会吴子于良。水道不可,吴子辞,乃还。

七月丙寅,治兵于邾南,甲车四千乘,羊舌鲋摄司马,遂合诸侯于平丘。子产、子大叔相郑伯以会。子产以幄幕九张行。子大叔以四十,既而悔之,每舍,损焉。及会,亦如之。

次于卫地,叔鲋求货于卫,淫刍荛者。卫人使屠伯馈叔向羹,与一箧锦,曰:``诸侯事晋,未敢携贰,况卫在君之宇下,而敢有异志?刍荛者异于他日,敢请之。」叔向受羹反锦,曰:``晋有羊舌鲋者,渎货无厌,亦将及矣。为此役也,子若以君命赐之,其已。」客从之,未退,而禁之。

晋人将寻盟,齐人不可。晋侯使叔向告刘献公曰:``抑齐人不盟,若之何?」对曰:``盟以厎信。君苟有信,诸侯不贰,何患焉?告之以文辞,董之以武师,虽齐不许,君庸多矣。天子之老,请帅王赋,『元戎十乘,以先启行』,迟速唯君。」叔向告于齐,曰:``诸侯求盟,已在此矣。今君弗利,寡君以为请。」对曰:``诸侯讨贰,则有寻盟。若皆用命,何盟之寻?」叔向曰:``国家之败,有事而无业,事则不经。有业而无礼,经则不序。有礼而无威,序则不共。有威而不昭,共则不明。不明弃共,百事不终,所由倾覆也。是故明王之制,使诸侯岁聘以志业,间朝以讲礼,再朝而会以示威,再会而盟以显昭明。志业于好,讲礼于等。示威于众,昭明于神。自古以来,未之或失也。存亡之道,恒由是兴。晋礼主盟,惧有不治。奉承齐牺,而布诸君,求终事也。君曰:『余必废之,何齐之有?』唯君图之,寡君闻命矣!」齐人惧,对曰:``小国言之,大国制之,敢不听从?既闻命矣,敬共以往,迟速唯君。」叔向曰:``诸侯有间矣,不可以不示众。」八月辛未,治兵,建而不旆。壬申,复旆之。诸侯畏之。

邾人、莒人言斥于晋曰:``鲁朝夕伐我,几亡矣。我之不共,鲁故之以。」晋侯不见公,使叔向来辞曰:``诸侯将以甲戌盟,寡君知不得事君矣,请君无勤。」子服惠伯对曰:``君信蛮夷之诉,以绝兄弟之国,弃周公之后,亦唯君。寡君闻命矣。」叔向曰:``寡君有甲车四千乘在,虽以无道行之,必可畏也,况其率道,其何敌之有?牛虽瘠,偾于豚上,其畏不死?南蒯、子仲之忧,其庸可弃乎?若奉晋之众,用诸侯之师,因邾、莒、杞、鄫之怒,以讨鲁罪,间其二忧,何求而弗克?」鲁人惧,听命。

甲戌,同盟于平丘,齐服也。令诸侯日中造于除。癸酉,退朝。子产命外仆速张于除,子大叔止之,使待明日。及夕,子产闻其未张也,使速往,乃无所张矣。

及盟,子产争承,曰:``昔天子班贡,轻重以列,列尊贡重,周之制也。卑而贡重者,甸服也。郑伯,男也,而使从公侯之贡,惧弗给也,敢以为请。诸侯靖兵,好以为事。行理之命,无月不至,贡之无艺,小国有阙,所以得罪也。诸侯修盟,存小国也。贡献无及,亡可待也。存亡之制,将在今矣。」自日中以争,至于昏,晋人许之。既盟,子大叔咎之曰:``诸侯若讨,其可渎乎?」子产曰:``晋政多门,贰偷之不暇,何暇讨?国不竞亦陵,何国之为?」

公不与盟。晋人执季孙意如,以幕蒙之,使狄人守之。司铎射怀锦,奉壶饮冰,以蒲伏焉。守者御之,乃与之锦而入。晋人以平子归,子服湫从。

子产归,未至,闻子皮卒,哭,且曰:``吾已,无为为善矣,唯夫子知我。」仲尼谓:``子产于是行也,足以为国基矣。《诗》曰:『乐只君子,邦家之基。』子产,君子之求乐者也。」且曰:``合诸侯,艺贡事,礼也。」

鲜虞人闻晋师之悉起也,而不警边,且不修备。晋荀吴自着雍以上军侵鲜虞,及中人,驱冲竞,大获而归。

楚之灭蔡也,灵王迁许、胡、沈、道、房、申于荆焉。平王即位,既封陈、蔡,而皆复之,礼也。隐大子之子庐归于蔡,礼也。悼大子之子吴归于陈,礼也。

冬十月,葬蔡灵公,礼也。

公如晋。荀吴谓韩宣子曰:``诸侯相朝,讲旧好也,执其卿而朝其君,有不好焉,不如辞之。」乃使士景伯辞公于河。

吴灭州来。令尹子期请伐吴,王弗许,曰:``吾未抚民人,未事鬼神,未修守备,未定国家,而用民力,败不可悔。州来在吴,犹在楚也。子姑待之。」

季孙犹在晋,子服惠伯私于中行穆子曰:``鲁事晋,何以不如夷之小国?鲁,兄弟也,土地犹大,所命能具。若为夷弃之,使事齐、楚,其何瘳于晋?亲亲,与大,赏共、罚否,所以为盟主也。子其图之。谚曰:『臣一主二。』吾岂无大国?」穆子告韩,且曰:``楚灭陈、蔡,不能救,而为夷执亲,将焉用之?」乃归季孙。惠伯曰:``寡君未知其罪,合诸侯而执其老。若犹有罪,死命可也。若曰无罪而惠免之,诸侯不闻,是逃命也,何免之?为请从君惠于会。」宣子患之,谓叔向曰:``子能归季孙乎?」对曰:``不能。鲋也能。」乃使叔鱼。叔鱼见季孙曰:``昔鲋也得罪于晋君,自归于鲁君。微武子之赐,不至于今。虽获归骨于晋,犹子则肉之,敢不尽情?归子而不归,鲋也闻诸吏,将为子除馆于西河,其若之何?」且泣。平子惧,先归。惠伯待礼。

\hypertarget{header-n2608}{%
\subsubsection{昭公十四年}\label{header-n2608}}

【经】十有四年春,意如至自晋。三月,曹伯滕卒。夏四月。秋,葬曹武公。八月,莒子去疾卒。冬,莒杀其公子意恢。

【传】十四年春,意如至自晋,尊晋罪己也。尊晋、罪己,礼也。

南蒯之将叛也,盟费人。司徒老祁、虑癸伪废疾,使请于南蒯曰:``臣愿受盟而疾兴,若以君灵不死,请待间而盟。」许之。二子因民之欲叛也,请朝众而盟。遂劫南蒯曰:``群臣不忘其君,畏子以及今,三年听命矣。子若弗图,费人不忍其君,将不能畏子矣。子何所不逞欲?请送子。」请期五日。遂奔齐。侍饮酒于景公。公曰:``叛夫?」对曰:``臣欲张公室也。」子韩皙曰:``家臣而欲张公室,罪莫大焉。」司徒老祁、虑癸来归费,齐侯使鲍文子致之。

夏,楚子使然丹简上国之兵于宗丘,且抚其民。分贫,振穷;长孤幼,养老疾,收介特,救灾患,宥孤寡,赦罪戾;诘奸慝,举淹滞;礼新,叙旧;禄勋,合亲;任良,物官。使屈罢简东国之兵于召陵,亦如之。好于边疆,息民五年,而后用师,礼也。

秋八月,莒着丘公卒,郊公不戚。国人弗顺,欲立着丘公之弟庚舆。蒲余侯恶公子意恢而善于庚舆,郊公恶公子铎而善于意恢。公子铎因蒲余侯而与之谋曰:``尔杀意恢,我出君而纳庚舆。」许之。

楚令尹子旗有德于王,不知度。与养氏比,而求无厌。王患之。九月甲午,楚子杀斗成然,而灭养氏之族。使斗辛居郧,以无忘旧勋。

冬十二月,蒲余侯兹夫杀莒公子意恢,郊公奔齐。公子铎逆庚舆于齐。齐隰党、公子锄送之,有赂田。

晋邢侯与雍子争赂田,久而无成。士景伯如楚,叔鱼摄理,韩宣子命断旧狱,罪在雍子。雍子纳其女于叔鱼,叔鱼蔽罪邢侯。邢侯怒,杀叔鱼与雍子于朝。宣子问其罪于叔向。叔向曰:``三人同罪,施生戮死可也。雍子自知其罪而赂以买直,鲋也鬻狱,刑侯专杀,其罪一也。己恶而掠美为昏,贪以败官为墨,杀人不忌为贼。《夏书》曰:『昏、墨、贼,杀。』皋陶之刑也。请从之。」乃施邢侯而尸雍子与叔鱼于市。

仲尼曰:``叔向,古之遗直也。治国制刑,不隐于亲,三数叔鱼之恶,不为末减。曰义也夫,可谓直矣。平丘之会,数其贿也,以宽卫国,晋不为暴。归鲁季孙,称其诈也,以宽鲁国,晋不为虐。邢侯之狱,言其贪也,以正刑书,晋不为颇。三言而除三恶,加三利,杀亲益荣,犹义也夫!」

\hypertarget{header-n2620}{%
\subsubsection{昭公十五年}\label{header-n2620}}

【经】十有五年春王正月,吴子夷末卒。二月癸酉,有事于武宫。籥入,叔弓卒。去乐,卒事。夏,蔡朝吴出奔郑。六月丁巳朔,日有食之。秋,晋荀吴帅师伐鲜虞。冬,公如晋。

【传】十五年春,将禘于武公,戒百官。梓慎曰:``禘之日,其有咎乎!吾见赤黑之祲,非祭祥也,丧氛也。其在莅事乎?」二月癸酉,禘,叔弓莅事,籥入而卒。去乐,卒事,礼也。

楚费无极害朝吴之在蔡也,欲去之。乃谓之曰:``王唯信子,故处子于蔡。子亦长矣,而在下位,辱。必求之,吾助子请。」又谓其上之人曰:``王唯信吴,故处诸蔡,二三子莫之如也。而在其上,不亦难乎?弗图,必及于难。」夏,蔡人遂朝吴。朝吴出奔郑。王怒,曰:``余唯信吴,故置诸蔡。且微吴,吾不及此。女何故去之?」无极对曰:``臣岂不欲吴?然而前知其为人之异也。吴在蔡,蔡必速飞。去吴,所以翦其翼也。」

六月乙丑,王大子寿卒。

秋八月戊寅,王穆后崩。

晋荀吴帅师伐鲜虞,围鼓。鼓人或请以城叛,穆子弗许。左右曰:``师徒不勤,而可以获城,何故不为?」穆子曰:``吾闻诸叔向曰:『好恶不愆,民知所适,事无不济。』或以吾城叛,吾所甚恶也。人以城来,吾独何好焉?赏所甚恶,若所好何?若其弗赏,是失信也,何以庇民?力能则进,否则退,量力而行。吾不可以欲城而迩奸,所丧滋多。」使鼓人杀叛人而缮守备。围鼓三月,鼓人或请降,使其民见,曰:``犹有食色,姑修而城。」军吏曰:``获城而弗取,勤民而顿兵,何以事君?」穆子曰:``吾以事君也。获一邑而教民怠,将焉用邑?邑以贾怠,不如完旧,贾怠无卒,弃旧不祥。鼓人能事其君,我亦能事吾君。率义不爽,好恶不愆,城可获而民知义所,有死命而无二心,不亦可乎!」鼓人告食竭力尽,而后取之。克鼓而反,不戮一人,以鼓子鸢鞮归。

冬,公如晋,平丘之会故也。

十二月,晋荀跞如周,葬穆后,籍谈为介。既葬,除丧,以文伯宴,樽以鲁壶。王曰:``伯氏,诸侯皆有以镇抚室,晋独无有,何也?」文伯揖籍谈,对曰:``诸侯之封也,皆受明器于王室,以镇抚其社稷,故能荐彝器于王。晋居深山,戎狄之与邻,而远于王室。王灵不及,拜戎不暇,其何以献器?」王曰:``叔氏,而忘诸乎?叔父唐叔,成王之母弟也,其反无分乎?密须之鼓,与其大路,文所以大蒐也。阙巩之甲,武所以克商也。唐叔受之以处参虚,匡有戎狄。其后襄之二路,金戚钺,秬鬯,彤弓,虎贲,文公受之,以有南阳之田,抚征东夏,非分而何?夫有勋而不废,有绩而载,奉之以土田,抚之以彝器,旌之以车服,明之以文章,子孙不忘,所谓福也。福祚之不登,叔父焉在?且昔而高祖孙伯□,司晋之典籍,以为大政,故曰籍氏。及辛有之二子董之晋,于是乎有董史。女,司典之后也,何故忘之?」籍谈不能对。宾出,王曰:``籍父其无后乎!数典而忘其祖。」

籍谈归,以告叔向。叔向曰:``王其不终乎!吾闻之:『所乐必卒焉。』今王乐忧,若卒以忧,不可谓终。王一岁而有三年之丧二焉,于是乎以丧宾宴,又求彝器,乐忧甚矣,且非礼也。彝器之来,嘉功之由,非由丧也。三年之丧,虽贵遂服,礼也。王虽弗遂,宴乐以早,亦非礼也。礼,王之大经也。一动而失二礼,无大经矣。言以考典,典以志经,忘经而多言举典,将焉用之?」

\hypertarget{header-n2632}{%
\subsubsection{昭公十六年}\label{header-n2632}}

【经】十有六年春,齐侯伐徐。楚子诱戎蛮子杀之。夏,公至自晋。秋八月己亥,晋侯夷卒。九月,大雩。季孙意如如晋。冬十月,葬晋昭公。

【传】十六年春,王正月,公在晋,晋人止公。不书,讳之也。

齐侯伐徐。

楚子闻蛮氏之乱也,与蛮子之无质也,使然丹诱戎蛮子嘉杀之,遂取蛮氏。既而复立其子焉,礼也。

二月丙申,齐师至于蒲隧。徐人行成。徐子及郯人、莒人会齐侯,盟于蒲隧,赂以甲父之鼎。叔孙昭子曰:``诸侯之无伯,害哉!齐君之无道也,兴师而伐远方,会之,有成而还,莫之亢也,无伯也夫!《诗》曰:『宗周既灭,靡所止戾。正大夫离居,莫知我肄。』其是之谓乎!」

二月,晋韩起聘于郑,郑伯享之。子产戒曰:``苟有位于朝,无有不共恪。」孔张后至,立于客间。执政御之,适客后。又御之,适县间。客从而笑之。事毕,富子谏曰:``夫大国之人,不可不慎也,几为之笑而不陵我?我皆有礼,夫犹鄙我。国而无礼,何以求荣?孔张失位,吾子之耻也。」子产怒曰:``发命之不衷,出令之不信,刑之颇类,狱之放纷,会朝之不敬,使命之不听,取陵于大国,罢民而无功,罪及而弗知,侨之耻也。孔张,君之昆孙子孔之后也,执政之嗣也,为嗣大夫,承命以使,周于诸侯,国人所尊,诸侯所知。立于朝而祀于家,有禄于国,有赋于军,丧祭有职,受脤、归脤,其祭在庙,已有着位,在位数世,世守其业,而忘其所,侨焉得耻之?辟邪之人而皆及执政,是先王无刑罚也。子宁以他规我。」

宣子有环,有一在郑商。宣子谒诸郑伯,子产弗与,曰:``非官府之守器也,寡君不知。」子大叔、子羽谓子产曰:``韩子亦无几求,晋国亦未可以贰。晋国、韩子,不可偷也。若属有谗人交斗其间,鬼神而助之,以兴其凶怒,悔之何及?吾子何爱于一环,其以取憎于大国也,盍求而与之?」子产曰:``吾非偷晋而有二心,将终事之,是以弗与,忠信故也。侨闻君子非无贿之难,立而无令名之患。侨闻为国非不能事大字小之难,无礼以定其位之患。夫大国之人,令于小国,而皆获其求,将何以给之?一共一否,为罪滋大。大国之求,无礼以斥之,何餍之有?吾且为鄙邑,则失位矣。若韩子奉命以使,而求玉焉,贪淫甚矣,独非罪乎?出一玉以起二罪,吾又失位,韩子成贪,将焉用之?且吾以玉贾罪,不亦锐乎?」

韩子买诸贾人,既成贾矣,商人曰:``必告君大夫。」韩子请诸子产曰:``日起请夫环,执政弗义,弗敢复也。今买诸商人,商人曰,必以闻,敢以为请。」子产对曰:``昔我先君桓公,与商人皆出自周,庸次比耦,以艾杀此地,斩之蓬蒿藜藿,而共处之。世有盟誓,以相信也,曰:『尔无我叛,我无强贾,毋或丐夺。尔有利市宝贿,我勿与知。』恃此质誓,故能相保,以至于今。今吾子以好来辱,而谓敝邑强夺商人,是教弊邑背盟誓也,毋乃不可乎!吾子得玉而失诸侯,必不为也。若大国令,而共无艺,郑,鄙邑也,亦弗为也。侨若献玉,不知所成,敢私布之。」韩子辞玉,曰:``起不敏,敢求玉以徼二罪?敢辞之。」

夏四月,郑六卿饯宣子于郊。宣子曰:``二三君子请皆赋,起亦以知郑志。」子\\
赋《野有蔓草》。宣子曰:``孺子善哉!吾有望矣。」子产赋《郑之羔裘》。宣子曰:``起不堪也。」子大叔赋《褰裳》。宣子曰:``起在此,敢勤子至于他人乎?」子大叔拜。宣子曰:``善哉,子之言是!不有是事,其能终乎?」子游赋《风雨》,子旗赋《有女同车》,子柳赋《蘀兮》。宣子喜曰:``郑其庶乎!二三君子以君命贶起,赋不出郑志,皆昵燕好也。二三君子数世之主也,可以无惧矣。」宣子皆献马焉,而赋《我将》。子产拜,使五卿皆拜,曰:``吾子靖乱,敢不拜德?」宣子私觐于子产以玉与马,曰:``子命起舍夫玉,是赐我玉而免吾死也,敢不藉手以拜?」

公至自晋。子服昭伯语季平子曰:``晋之公室,其将遂卑矣。君幼弱,六卿强而奢傲,将因是以习,习实为常,能无卑乎?」

平子曰:``尔幼,恶识国?」

秋八月,晋昭公卒。

九月,大雩,旱也。郑大旱,使屠击、祝款、竖柎有事于桑山。斩其木,不雨。子产曰:``有事于山,蓺山林也,而斩其木,其罪大矣。」夺之官邑。

冬十月,季平子如晋葬昭公。平子曰:``子服回之言犹信,子服氏有子哉!」

\hypertarget{header-n2649}{%
\subsubsection{昭公十七年}\label{header-n2649}}

【经】十有七年春,小邾子来朝。夏六月甲戌朔,日有食之。秋,郯子来朝。八月,晋荀吴帅师灭陆浑之戎。冬,有星孛于大辰。楚人及吴战于长岸。

【传】十七年春,小邾穆公来朝,公与之燕。季平子赋《采叔》,穆公赋《菁菁者莪》。昭子曰:``不有以国,其能久乎?」

夏六月甲戌朔,日有食之。祝史请所用币。昭子曰:``日有食之,天子不举,伐鼓于社;诸侯用币于社,伐鼓于朝。礼也。」平子御之,曰:``止也。唯正月朔,慝未作,日有食之,于是乎有伐鼓用币,礼也。其馀则否。」大史曰:``在此月也。日过分而未至,三辰有灾。于是乎百官降物,君不举,辟移时,乐奏鼓,祝用币,史用辞。故《夏书》曰:『辰不集于房,瞽奏鼓,啬夫驰,庶人走。』此月朔之谓也。当夏四月,是谓孟夏。」平子弗从。昭子退曰:``夫子将有异志,不君君矣。」

秋,郯子来朝,公与之宴。昭子问焉,曰:``少皞氏鸟名官,何故也?」郯子曰:``吾祖也,我知之。昔者黄帝氏以云纪,故为云师而云名;炎帝氏以火纪,故为火师而火名;共工氏以水纪,故为水师而水名;大皞氏以龙纪,故为龙师而龙名。我高祖少皞挚之立也,凤鸟适至,故纪于鸟,为鸟师而鸟名。凤鸟氏,历正也。玄鸟氏,司分者也;伯赵氏,司至者也;青鸟氏,司启者也;丹鸟氏,司闭者也。祝鸠氏,司徒也;□鸠氏,司马也;鳲鸠氏,司空也;爽鸠氏,司寇也;鹘鸠氏,司事也。五鸠,鸠民者也。五雉,为五工正,利器用、正度量,夷民者也。九扈为九农正,扈民无淫者也。自颛顼以来,不能纪远,乃纪于近,为民师而命以民事,则不能故也。」仲尼闻之,见于郯子而学之。既而告人曰:``吾闻之:『天子失官,学在四夷』,犹信。」

晋侯使屠蒯如周,请有事于雒与三涂。苌弘谓刘子曰:``客容猛,非祭也,其伐戎乎?陆浑氏甚睦于楚,必是故也。君其备之!」乃警戎备。九月丁卯,晋荀吴帅师涉自棘津,使祭史先用牲于洛。陆浑人弗知,师从之。庚午,遂灭陆浑,数之以其贰于楚也。陆浑子奔楚,其众奔甘鹿。周大获。宣子梦文公携荀吴而授之陆浑,故使穆子帅师,献俘于文宫。

冬,有星孛于大辰,西及汉。申须曰:``彗所以除旧布新也。天事恒象,今除于火,火出必布焉。诸侯其有火灾乎?」梓慎曰:``往年吾见之,是其征也,火出而见。今兹火出而章,必火入而伏。其居火也久矣,其与不然乎?火出,于夏为三月,于商为四月,于周为五月。夏数得天。若火作,其四国当之,在宋、卫、陈、郑乎?宋,大辰之虚也;陈,大皞之虚也;郑,祝融之虚也,皆火房也。星孛天汉,汉,水祥也。卫,颛顼之虚也,故为帝丘,其星为大水,水,火之牡也。其以丙子若壬午作乎?水火所以合也。若火入而伏,必以壬午,不过其见之月。」郑裨灶言于子产曰:``宋、卫、陈、郑将同日火,若我用瓘斝玉瓒,郑必不火。」子产弗与。

吴伐楚。阳丐为令尹,卜战,不吉。司马子鱼曰:``我得上流,何故不吉。且楚故,司马令龟,我请改卜。」令曰:``鲂也,以其属死之,楚师继之,尚大克之」。吉。战于长岸,子鱼先死,楚师继之,大败吴师,获其乘舟余皇。使随人与后至者守之,环而堑之,及泉,盈其隧炭,陈以待命。吴公子光请于其众,曰:``丧先王之乘舟,岂唯光之罪,众亦有焉。请藉取之,以救死。」众许之。使长鬣者三人,潜伏于舟侧,曰:``我呼皇,则对,师夜从之。」三呼,皆迭对。楚人从而杀之,楚师乱,吴人大败之,取余皇以归。

\hypertarget{header-n2659}{%
\subsubsection{昭公十八年}\label{header-n2659}}

【经】十有八年春王三月,曹伯须卒。夏五月壬午,宋、卫、陈、郑灾。六月,邾人入鄅。秋,葬曹平公。冬,许迁于白羽。

【传】十八年春,王二月乙卯,周毛得杀毛伯过而代之。苌弘曰:``毛得必亡,是昆吾稔之日也,侈故之以。而毛得以济侈于王都,不亡何待!」

三月,曹平公卒。

夏五月,火始昏见。丙子,风。梓慎曰:``是谓融风,火之始也。七日,其火作乎!」戊寅,风甚。壬午,大甚。宋、卫、陈、郑皆火。梓慎登大庭氏之库以望之,曰:``宋、卫、陈、郑也。」数日,皆来告火。裨灶曰:``不用吾言,郑又将火。」郑人请用之,子产不可。子大叔曰:``宝,以保民也。若有火,国几亡。可以救亡,子何爱焉?」子产曰:``天道远,人道迩,非所及也,何以知之?灶焉知天道?是亦多言矣,岂不或信?」遂不与,亦不复火。

郑之未灾也,里析告子产曰:``将有大祥,民震动,国几亡。吾身泯焉,弗良及也。国迁其可乎?」子产曰:``虽可,吾不足以定迁矣。」及火,里析死矣,未葬,子产使舆三十人,迁其柩。火作,子产辞晋公子、公孙于东门。使司寇出新客,禁旧客勿出于宫。使子宽、子上巡群屏摄,至于大宫。使公孙登徙大龟。使祝史徙主祏于周庙,告于先君。使府人、库人各儆其事。商成公儆司宫,出旧宫人,置诸火所不及。司马、司寇列居火道,行火所□欣。城下之人,伍列登城。明日,使野司寇各保其征。郊人助祝史除于国北,禳火于玄冥、回禄,祈于四鄘。书焚室而宽其征,与之材。三日哭,国不市。使行人告于诸侯。宋、卫皆如是。陈不救火,许不吊灾,君子是以知陈、许之先亡也。

六月,鄅人藉稻。邾人袭鄅,鄅人将闭门。邾人羊罗摄其首焉,遂入之,尽俘以归。鄅子曰:``余无归矣。」从帑于邾,邾庄公反鄅夫人,而舍其女。秋,葬曹平公。往者见周原伯鲁焉,与之语,不说学。归以语闵子马。闵子马曰:``周其乱乎?夫必多有是说,而后及其大人。大人患失而惑,又曰:『可以无学,无学不害。』不害而不学,则苟而可。于是乎下陵上替,能无乱乎?夫学,殖也,不学将落,原氏其亡乎?」

七月,郑子产为火故,大为社祓禳于四方,振除火灾,礼也。乃简兵大蒐,将为蒐除。子大叔之庙在道南,其寝在道北,其庭小。过期三日,使除徒陈于道南庙北,曰:``子产过女而命速除,乃毁于而乡。」子产朝,过而怒之,除者南毁。子产及冲,使从者止之曰:``毁于北方。」

火之作也,子产授兵登陴。子大叔曰:``晋无乃讨乎?」子产曰:``吾闻之,小国忘守则危,况有灾乎?国之不可小,有备故也。」既,晋之边吏让郑曰:``郑国有灾,晋君、大夫不敢宁居,卜筮走望,不爱牲玉。郑之有灾,寡君之忧也。今执事手间然授兵登陴,将以谁罪?边人恐惧不敢不告。子产对曰:``若吾子之言,敝邑之灾,君之忧也。敝邑失政,天降之灾,又惧谗慝之间谋之,以启贪人,荐为弊邑不利,以重君之忧。幸而不亡,犹可说也。不幸而亡,君虽忧之,亦无及也。郑有他竟,望走在晋。既事晋矣,其敢有二心?」

楚左尹王子胜言于楚子曰:``许于郑,仇敌也,而居楚地,以不礼于郑。晋、郑方睦,郑若伐许,而晋助之,楚丧地矣。君盍迁许?许不专于楚。郑方有令政。许曰:『余旧国也。』郑曰:『余俘邑也。』叶在楚国,方城外之蔽也。土不可易,国不可小,许不可俘,仇不可启,君其图之。」楚子说。冬,楚子使王子胜迁许于析,实白羽。

\hypertarget{header-n2671}{%
\subsubsection{昭公十九年}\label{header-n2671}}

【经】十有九年春,宋公伐邾。夏五月戊辰,许世子止弑其君买。己卯,地震。秋,齐高发帅师伐莒。冬,葬许悼公。

【传】十九年春,楚工尹赤迁阴于下阴,令尹子瑕城郏。叔孙昭子曰:``楚不在诸侯矣!其仅自完也,以持其世而已。」

楚子之在蔡也,狊阜阳封人之女奔之,生大子建。及即位,使伍奢为之师。费无极为少师,无宠焉,欲谮诸王,曰:``建可室矣。」王为之聘于秦,无极与逆,劝王取之,正月,楚夫人嬴氏至自秦。

鄅夫人,宋向戌之女也,故向宁请师。二月,宋公伐邾,围虫。三月,取之。乃尽归鄅俘。

夏,许悼公疟。五月戊辰,饮大子止之药卒。大子奔晋。书曰:``弑其君。」君子曰:``尽心力以事君,舍药物可也。」

邾人、郳人、徐人会宋公。乙亥,同盟于虫。

楚子为舟师以伐濮。费无极言于楚子曰:``晋之伯也,迩于诸夏,而楚辟陋,故弗能与争。若大城城父而置大子焉,以通北方,王收南方,是得天下也。」王说,从之。故太子建居于城父。

令尹子瑕聘于秦,拜夫人也。

秋,齐高发帅师伐莒。莒子奔纪鄣。使孙书伐之。初,莒有妇人,莒子杀其夫,已为嫠妇。及老,托于纪鄣,纺焉以度而去之。及师至,则投诸外。或献诸子占,子占使师夜缒而登。登者六十人。缒绝。师鼓噪,城上之人亦噪。莒共公惧,启西门而出。七月丙子,齐师入纪。

是岁也,郑驷偃卒。子游娶于晋大夫,生丝,弱。其父兄立子瑕。子产憎其为人也,且以为不顺,弗许,亦弗止。驷氏耸。他日,丝以告其舅。冬,晋人使以币如郑,问驷乞之立故。驷氏惧,驷乞欲逃。子产弗遣。请龟以卜,亦弗予。大夫谋对,子产不待而对客曰:``郑国不天,寡君之二三臣,札瘥夭昏,今又丧我先大夫偃。其子幼弱,其一二父兄惧队宗主,私族于谋而立长亲。寡君与其二三老曰:『抑天实剥乱是,吾何知焉?』谚曰:『无过乱门。』民有兵乱,犹惮过之,而况敢知天之所乱?今大夫将问其故,抑寡君实不敢知,其谁实知之?平丘之会,君寻旧盟曰:『无或失职。』若寡君之二三臣,其即世者,晋大夫而专制其位,是晋之县鄙也,何国之为?」辞客币而报其使。晋人舍之。

楚人城州来。沈尹戌曰:``楚人必败。昔吴灭州来,子旗请伐之。王曰:『吾未抚吾民。』今亦如之,而城州来以挑吴,能无败乎?」侍者曰:``王施舍不倦,息民五年,可谓抚之矣。」戌曰:``吾闻抚民者,节用于内,而树德于外,民乐其性,而无寇仇。今宫室无量,民人日骇,劳罢死转,忘寝与食,非抚之也。」

郑大水,龙斗于时门之外洧渊。国人请为\\
焉,子产弗许,曰:``我斗,龙不我觌也。龙斗,我独何觌焉?禳之,则彼其室也。吾无求于龙,龙亦无求于我。」乃止也。

令尹子瑕言蹶由于楚子曰:``彼何罪?谚所谓『室于怒,市于色』者,楚之谓矣。舍前之忿可也。」乃归蹶由。

\hypertarget{header-n2687}{%
\subsubsection{昭公二十年}\label{header-n2687}}

【经】二十年春王正月。夏,曹公孙会自鄸出奔宋。秋,盗杀卫侯之兄絷。冬十月,宋华亥、向宁、华定出奔陈。十有一月辛卯,蔡侯卢卒。

【传】二十年春,王二月己丑,日南至。梓慎望氛曰:``今兹宋有乱,国几亡,三年而后弭。蔡有大丧。」叔孙昭子曰:``然则戴、桓也!汏侈无礼已甚,乱所在也。」

费无极言于楚子曰:``建与伍奢将以方城之外叛。自以为犹宋、郑也,齐、晋又交辅之,将以害楚。其事集矣。」王信之,问伍奢。伍奢对曰:``君一过多矣,何言于谗?」王执伍奢。使城父司马奋扬杀大子,未至,而使遣之。三月,大子建奔宋。王召奋扬,奋扬使城父人执己以至。王曰:``言出于余口,入于尔耳,谁告建也?」对曰:``臣告之。君王命臣曰:『事建如事余。』臣不佞,不能苟贰。奉初以还,不忍后命,故遣之。既而悔之,亦无及已。」王曰:``而敢来,何也?」对曰:``使而失命,召而不来,是再奸也。逃无所入。」王曰:``归。」从政如他日。

无极曰:``奢之子材,若在吴,必忧楚国,盍以免其父召之。彼仁,必来。不然,将为患。」王使召之,曰:``来,吾免而父。」棠君尚谓其弟员曰:``尔适吴,我将归死。吾知不逮,我能死,尔能报。闻免父之命,不可以莫之奔也;亲戚为戮,不可以莫之报也。奔死免父,孝也;度功而行,仁也;择任而往,知也;知死不辟,勇也。父不可弃,名不可废,尔其勉之,相从为愈。」伍尚归。奢闻员不来,曰:``楚君、大夫其旰食乎!」楚人皆杀之。

员如吴,言伐楚之利于州于。公子光曰:``是宗为戮而欲反其仇,不可从也。」员曰:``彼将有他志。余姑为之求士,而鄙以待之。」乃见鱄设诸焉,而耕于鄙。

宋元公无信多私,而恶华、向。华定、华亥与向宁谋曰:``亡愈于死,先诸?」华亥伪有疾,以诱群公子。公子问之,则执之。夏六月丙申,杀公子寅、公子御戎、公子朱、公子固、公孙援、公孙丁、拘向胜、向行于其廪。公如华氏请焉,弗许,遂劫之。癸卯,取大子栾与母弟辰、公子地以为质。公亦取华亥之子无戚、向宁之子罗、华定之子启,与华氏盟,以为质。

卫公孟絷狎齐豹,夺之司寇与鄄,有役则反之,无则取之。公孟恶北宫喜、褚师圃,欲去之。公子朝通于襄夫人宣姜,惧,而欲以作乱。故齐豹、北宫喜、褚师圃、公子朝作乱。

初,齐豹见宗鲁于公孟,为骖乘焉。将作乱,而谓之曰:``公孟之不善,子所知也。勿与乘,吾将杀之。」对曰:``吾由子事公孟,子假吾名焉,故不吾远也。虽其不善,吾亦知之。抑以利故,不能去,是吾过也。今闻难而逃,是僭子也。子行事乎,吾将死之,以周事子,而归死于公孟,其可也。」

丙辰,卫侯在平寿,公孟有事于盖获之门外,齐子氏帷于门外而伏甲焉。使祝蛙置戈于车薪以当门,使一乘从公孟以出。使华齐御公孟,宗鲁骖乘。及闳中,齐氏用戈击公孟,宗鲁以背蔽之,断肱,以中公孟之肩,皆杀之。

公闻乱,乘,驱自阅门入,庆比御公,公南楚骖乘,使华寅乘贰车。及公宫,鸿\\
魋驷乘于公,公载宝以出。褚师子申遇公于马路之衢,遂从。过齐氏,使华寅肉袒,执盖以当其阙。齐氏射公,中南楚之背,公遂出。寅闭郭门,逾而从公。公如死鸟,析朱锄宵从窦出,徒行从公。

齐侯使公孙青聘于卫。既出,闻卫乱,使请所聘。公曰:``犹在竟内,则卫君也。」乃将事焉。遂从诸死鸟,请将事。辞曰:``亡人不佞,失守社稷,越在草莽,吾子无所辱君命。」宾曰:``寡君命下臣于朝,曰:『阿下执事。』臣不敢贰。」主人曰:``君若惠顾先君之好,昭临敝邑,镇抚其社稷,则有宗祧在。」乃止。卫侯固请见之,不获命,以其良马见,为未致使故也。卫侯以为乘马。宾将掫,主人辞曰:``亡人之忧,不可以及吾子。草莽之中,不足以辱从者。敢辞。」宾曰:``寡君之下臣,君之牧圉也。若不获扞外役,是不有寡君也。臣惧不免于戾,请以除死。」亲执铎,终夕与于燎。

齐氏之宰渠子召北宫子。北宫氏之宰不与闻谋,杀渠子,遂伐齐氏,灭之。丁巳晦,公入,与北宫喜盟于彭水之上。秋七月戊午朔,遂盟国人。八月辛亥,公子朝、褚师圃、子玉霄、子高鲂出奔晋。闰月戊辰,杀宣姜。卫侯赐北宫喜谥曰贞子,赐析朱锄谥曰成子,而以齐氏之墓予之。

卫侯告宁于齐,且言子石。齐侯将饮酒,遍赐大夫曰:``二三子之教也。」苑何忌辞,曰:``与于青之赏,必及于其罚。在《康诰》曰:『父子兄弟,罪不相及。』况在群臣?臣敢贪君赐以干先王?」

琴张闻宗鲁死,将往吊之。仲尼曰:``齐豹之盗,而孟絷之贼,女何吊焉?君子不食奸,不受乱,不为利疚于回,不以回待人,不盖不义,不犯非礼。」

宋华、向之乱,公子城、公孙忌、乐舍、司马强、向宜、向郑、楚建、郳甲出奔郑。其徒与华氏战于鬼阎,败子城。子城适晋。华亥与其妻必盥而食所质公子者而后食。公与夫人每日必适华氏,食公子而后归。华亥患之,欲归公子。向宁曰:``唯不信,故质其子。若又归之,死无日矣。」公请于华费遂,将攻华氏。对曰:``臣不敢爱死,无乃求去忧而滋长乎!臣是以惧,敢不听命?」公曰:``子死亡有命,余不忍其呴。」冬十月,公杀华、向之质而攻之。戊辰,华、向奔陈,华登奔吴。向宁欲杀大子,华亥曰:``干君而出,又杀其子,其谁纳我?且归之有庸。」使少司寇牼以归,曰:``子之齿长矣,不能事人,以三公子为质,必免。」公子既入,华牼将自门行。公遽见之,执其手曰:``余知而无罪也,入,复而所。」

齐侯疥,遂痁,期而不瘳,诸侯之宾问疾者多在。梁丘据与裔款言于公曰:``吾事鬼神丰,于先君有加矣。今君疾病,为诸侯忧,是祝史之罪也。诸侯不知,其谓我不敬。君盍诛于祝固、史嚣以辞宾?」公说,告晏子。晏子曰:``日宋之盟,屈建问范会之德于赵武。赵武曰:『夫子之家事治,言于晋国,竭情无私。其祝史祭祀,陈信不愧。其家事无猜,其祝史不祈。』建以语康王,康王曰:『神人无怨,宜夫子之光辅五君,以为诸侯主也。』」公曰:``据与款谓寡人能事鬼神,故欲诛于祝史。子称是语,何故?」对曰:``若有德之君,外内不废,上下无怨,动无违事,其祝史荐信,无愧心矣。是以鬼神用飨,国受其福,祝史与焉。其所以蕃祉老寿者,为信君使也,其言忠信于鬼神。其适遇淫君,外内颇邪,上下怨疾,动作辟违,从欲厌私。高台深池,撞钟舞女,斩刈民力,输掠其聚,以成其违,不恤后人。暴虐淫从,肆行非度,无所还忌,不思谤讟不惮鬼神,神怒民痛,无悛于心。其祝史荐信,是言罪也。其盖失数美,是矫诬也。进退无辞,则虚以求媚。是以鬼神不飨其国以祸之,祝史与焉。所以夭昏孤疾者,为暴君使也。溲再辕稼鬼神。」公曰:``然则若之何?」对曰:``不可为也:山林之木,衡鹿守之;泽之萑蒲,舟鲛守之;薮之薪蒸,虞候守之。海之盐蜃,祈望守之。县鄙之人,入从其政。逼介之关,暴征其私。承嗣大夫,强易其贿。布常无艺,征敛无度;宫室日更,淫乐不违。内宠之妾,肆夺于市;外宠之臣,僭令于鄙。私欲养求,不给则应。民人苦病,夫妇皆诅。祝有益也,诅亦有损。聊、摄以东,姑、尤以西,其为人也多矣。虽其善祝,岂能胜亿兆人之诅?君若欲诛于祝史,修德而后可。」公说,使有司宽政,毁关,去禁,薄敛,已责。

十二月,齐侯田于沛,招虞人以弓,不进。公使执之,辞曰:``昔我先君之田也,旃以招大夫,弓以招士,皮冠以招虞人。臣不见皮冠,故不敢进。」乃舍之。仲尼曰:``守道不如守官,君子韪之。」

齐侯至自田,晏子侍于遄台,子犹驰而造焉。公曰:``唯据与我和夫!」晏子对曰:``据亦同也,焉得为和?」公曰:``和与同异乎?」对曰:``异。和如羹焉,水火醯醢盐梅以烹鱼肉,燀之以薪。宰夫和之,齐之以味,济其不及,以泄其过。君子食之,以平其心。君臣亦然。君所谓可而有否焉,臣献其否以成其可。君所谓否而有可焉,臣献其可以去其否。是以政平而不干,民无争心。故《诗》曰:『亦有和羹,既戒既平。鬷嘏无言,时靡有争。』先王之济五味,和五声也,以平其心,成其政也。声亦如味,一气,二体,三类,四物,五声,六律,七音,八风,九歌,以相成也。清浊,小大,短长,疾徐,哀乐,刚柔,迟速,高下,出入,周疏,以相济也。君子听之,以平其心。心平,德和。故《诗》曰:『德音不瑕。』今据不然。君所谓可,据亦曰可;君所谓否,据亦曰否。若以水济水,谁能食之?若琴瑟之专一,谁能听之?同之不可也如是。」

饮酒乐。公曰:``古而无死,其乐若何?」晏子对曰:``古而无死,则古之乐也,君何得焉?昔爽鸠氏始居此地,季萴因之,有逢伯陵因之,蒲姑氏因之,而后大公因之。古者无死,爽鸠氏之乐,非君所愿也。」

郑子产有疾,谓子大叔曰:``我死,子必为政。唯有德者能以宽服民,其次莫如猛。夫火烈,民望而畏之,故鲜死焉。水懦弱,民狎而玩之,则多死焉。故宽难。」疾数月而卒。大叔为政,不忍猛而宽。郑国多盗,取人于萑苻之泽。大叔悔之,曰:``吾早从夫子,不及此。」兴徒兵以攻萑苻之盗,尽杀之,盗少止。

仲尼曰:``善哉!政宽则民慢,慢则纠之以猛。猛则民残,残则施之以宽。宽以济猛,猛以济宽,政是以和。《诗》曰:『民亦劳止,汔可小康。惠此中国,以绥四方。』施之以宽也。『毋从诡随,以谨无良。式遏寇虐,惨不畏明。』纠之以猛也。『柔远能迩,以定我王。』平之以和也。又曰:『不竞不絿,不刚不柔。布政优优,百禄是遒。』和之至也。」

及子产卒,仲尼闻之,出涕曰:``古之遗爱也。」

\hypertarget{header-n2712}{%
\subsubsection{昭公二十一年}\label{header-n2712}}

【经】二十有一年春王三月,葬蔡平公。夏,晋侯使士鞅来聘。宋华亥、向宁、华定自陈入于宋南里以叛。秋七月壬午朔,日有食之。八月乙亥,叔辄卒。冬,蔡侯朱出奔楚。公如晋,至河乃复。

【传】二十一年春,天王将铸无射。泠州鸠曰:``王其以心疾死乎?夫乐,天子之职也。夫音,乐之舆也。而钟,音之器也。天子省风以作乐,器以钟之,舆以行之。小者不窕,大者不□瓠,则和于物,物和则嘉成。故和声入于耳而藏于心,心亿则乐。窕则不咸,总则不容,心是以感,感实生疾。今钟□瓠矣,王心弗堪,其能久乎?」

三月,葬蔡平公。蔡大子朱失位,位在卑。大夫送葬者归,见昭子。昭子问蔡故,以告。昭子叹曰:``蔡其亡乎!若不亡,是君也必不终。《诗》曰:『不解于位,民之攸塈。』今蔡侯始即位,而适卑,身将从之。」

夏,晋士鞅来聘,叔孙为政。季孙欲恶诸晋,使有司以齐鲍国归费之礼为士鞅。士鞅怒,曰:``鲍国之位下,其国小,而使鞅从其牢礼,是卑敝邑也。将复诸寡君。」鲁人恐,加四牢焉,为十一牢。

宋华费遂生华貙、华多僚、华登。貙为少司马,多僚为御士,与貙相恶,乃谮诸公曰:``貙将纳亡人。」亟言之。公曰:``司马以吾故,亡其良子。死亡有命,吾不可以再亡之。」对曰:``君若爱司马,则如亡。死如可逃,何远之有?」公惧,使侍人召司马之侍人宜僚,饮之酒而使告司马。司马叹曰:``必多僚也。吾有谗子而弗能杀,吾又不死,抑君有命,可若何?」乃与公谋逐华貙,将使田孟诸而遣之。公饮之酒,厚酬之,赐及从者。司马亦如之。张丐尤之,曰:``必有故。」使子皮承宜僚以剑而讯之。宜僚尽以告。张丐欲杀多僚,子皮曰:``司马老矣,登之谓甚,吾又重之,不如亡也。」五月丙申,子皮将见司马而行,则遇多僚御司马而朝。张丐不胜其怒,遂与子皮、臼任、郑翩杀多僚,劫司马以叛,而召亡人。壬寅,华、向入。乐大心、丰愆、华牼御诸横。华氏居卢门,以南里叛。六月庚午,宋城旧鄘及桑林之门而守之。

秋七月壬午朔,日有食之。公问于梓慎曰:``是何物也,祸福何为?」对曰:``二至、二分,日有食之,不为灾。日月之行也,分,同道也;至,相过也。其他月则为灾,阳不克也,故常为水。」

于是叔辄哭日食。昭子曰:``子叔将死,非所哭也。」八月,叔辄卒。

冬十月,华登以吴师救华氏。齐乌枝鸣戍宋。厨人濮曰:``《军志》有之:『先人有夺人之心,后人有待其衰。』盍及其劳且未定也伐诸?若入而固,则华氏众矣,悔无及也。」从之。丙寅,齐师、宋师败吴师于鸿口,获其二帅公子苦雂、偃州员。华登帅其馀以败宋师。公欲出,厨人濮曰:``吾小人,可藉死而不能送亡,君请待之。」乃徇曰:``杨徽者,公徒也。」众从之。公自杨门见之,下而巡之,曰:``国亡君死,二三子之耻也,岂专孤之罪也?」齐乌枝鸣曰:``用少莫如齐致死,齐致死莫如去备。彼多兵矣,请皆用剑。」从之。华氏北,复即之。厨人濮以裳裹首而荷以走,曰:``得华登矣!」遂败华氏于新里。翟偻新居于新里,既战,说甲于公而归。华妵居于公里,亦如之。

十一月癸未,公子城以晋师至。曹翰胡会晋荀吴、齐苑何忌、卫公子朝救宋。丙戌,与华氏战于赭丘。郑翩愿为鹳,其御愿为鹅。子禄御公子城,庄堇为右。干犨御吕封人华豹,张丐为右。相遇,城还。华豹曰:``城也!」城怒而反之,将注,豹则关矣。曰:``平公之灵,尚辅相余。」豹射,出其间。将注,则又关矣。曰:``不狎,鄙!」押矢。城射之,殪。张丐抽殳而下,射之,折股。扶伏而击之,折轸。又射之,死。干丐请一矢,城曰:``余言汝于君。」对曰:``不死伍乘,军之大刑也。干刑而从子,君焉用之?子速诸。」乃射之,殪。大败华氏,围诸南里。华亥搏膺而呼,见华貙,曰:``吾为栾氏矣。」貙曰:``子无我迋。不幸而后亡。」使华登如楚乞师。华貙以车十五乘,徒七十人,犯师而出,食于睢上,哭而送之,乃复入。楚薳越帅师将逆华氏。大宰犯谏曰:``诸侯唯宋事其君,今又争国,释君而臣是助,无乃不可乎?」王曰:``而告我也后,既许之矣。」

蔡侯朱出奔楚。费无极取货于东国,而谓蔡人曰:``朱不用命于楚,君王将立东国。若不先从王欲,楚必围蔡。」蔡人惧,出朱而立东国。朱诉于楚,楚子将讨蔡。无极曰:``平侯与楚有盟,故封。其子有二心,故废之。灵王杀隐大子,其子与君同恶,德君必甚。又使立之,不亦可乎?且废置在君,蔡无他矣。」公如晋,及河,鼓叛晋。晋将伐鲜虞,故辞公。

\hypertarget{header-n2725}{%
\subsubsection{昭公二十二年}\label{header-n2725}}

【经】二十有二年春,齐侯伐莒。宋华亥、向宁、华定自宋南里出奔楚。大蒐于昌间。夏四月乙丑,天王崩。六月,叔鞅如京师,葬景王,王室乱。刘子、单子以王猛居于皇。秋,刘子、单子以王猛入于王城。冬十月,王子猛卒。十有二月癸酉朔,日有食之。

【传】二十二年春,王二月甲子,齐北郭启帅师伐莒。莒子将战,苑羊牧之谏曰:``齐帅贱,其求不多,不如下之。大国不可怒也。」弗听,败齐师于寿余。齐侯伐莒,莒子行成。司马灶如莒莅盟,莒子如齐莅盟,盟子稷门之外。莒于是乎大恶其君。

楚薳越使告于宋曰:``寡君闻君有不令之臣为君忧,无宁以为宗羞?寡君请受而戮之。」对曰:``孤不佞,不能媚于父兄,以为君忧,拜命之辱。抑君臣日战,君曰『余必臣是助』,亦唯命。人有言曰:『唯乱门之无过』。君若惠保敝邑,无亢不衷,以奖乱人,孤之望也。唯君图之!」楚人患之。诸侯之戍谋曰:``若华氏知困而致死,楚耻无功而疾战,非吾利也。不如出之,以为楚功,其亦能无为也已。救宋而除其害,又何求?」乃固请出之。宋人从之。己巳,宋华亥、向宁、华定、华貙、华登、皇奄伤、省臧,士平出奔楚。宋公使公孙忌为大司马,边卬为大司徒,乐祁为司马,仲几为左师,乐大心为右师,乐挽为大司寇,以靖国人。

王子朝、宾起有宠于景王,王与宾孟说之,欲立之。刘献公之庶子伯蚡事单穆公,恶宾孟之为人也,愿杀之。又恶王子朝之言,以为乱,愿去之。宾孟适郊,见雄鸡自断其尾。问之,侍者曰:``自惮其牺也。」遽归告王,且曰:``鸡其惮为人用乎?人异于是。牺者,实用人,人牺实难,己牺何害?」王弗应。

夏四月,王田北山,使公卿皆从,将杀单子、刘子。王有心疾,乙丑,崩于荣錡氏。戊辰,刘子挚卒,无子,单子立刘。五月庚辰,见王,遂攻宾起,杀之,盟群王子于单氏。

晋之取鼓也,既献,而反鼓子焉,又叛于鲜虞。

六月,荀吴略东阳,使师伪籴负甲以息于昔阳之门外,遂袭鼓,灭之。以鼓子鸢鞮归,使涉佗守之。

丁巳,葬景王。王子朝因旧官、百工之丧职秩者,与灵、景之族以作乱。帅郊、要、饯之甲,以逐刘子。壬戌、刘子奔扬。单子逆悼王于庄宫以归。王子还夜取王以如庄宫。癸亥,单子出。王子还与召庄公谋,曰:``不杀单旗,不捷。与之重盟,必来。背盟而克者多矣。」从之。樊顷子曰:``非言也,必不克。」遂奉王以追单子。及领,大盟而复,杀挚荒以说。刘子如刘,单子亡。乙丑,奔于平畤,群王子追之。单子杀还、姑、发、弱、鬷延、定、稠,子朝奔京。丙寅,伐之,京人奔山。刘子入于王城。辛未,巩简公败绩于京。乙亥,甘平公亦败焉。叔鞅至自京师,言王室之乱也。闵马父曰:``子朝必不克,其所与者,天所废也。」单子欲告急于晋,秋七月戊寅,以王如平畤,遂如圃车,次于皇。刘子如刘。单子使王子处守于王城,盟百工于平宫。辛卯,鄩肸伐皇,大败,获鄩肸。壬辰,焚诸王城之市。八月辛酉,司徒丑以王师败绩于前城,百工叛。己巳,伐单氏之宫,败焉。庚午,反伐之。辛未,伐东圉。冬十月丁巳,晋籍谈、荀跞帅九州之戎及焦、瑕、温、原之师,以纳王于王城。庚申,单子、刘蚡以王师败绩于郊,前城人败陆浑于社。十一月乙酉,王子猛卒,不成丧也。已丑,敬王即位,馆于子族氏。

十二月庚戌,晋籍谈、荀跞、贾辛、司马督帅师军于阴,于侯氏,于溪泉,次于社。王师军于泛,于解,次于任人。闰月,晋箕遗、乐征,右行诡济师,取前城,军其东南。王师军于京楚。辛丑,伐京,毁其西南。

\hypertarget{header-n2737}{%
\subsubsection{昭公二十三年}\label{header-n2737}}

【经】二十有三年春王正月,叔孙□若如晋。癸丑,叔鞅卒。晋人执我行人叔孙□若。晋人围郊。夏六月,蔡侯东国卒于楚。秋七月,莒子庚舆来奔。戊辰,吴败顿、胡沈、蔡、陈、许之师于鸡父,胡子髡、沈子逞灭,获陈夏啮。天王居于狄泉。尹氏立王子朝。八月乙未,地震。冬,公如晋,至河,有疾,乃复。

【传】二十三年春,王正月壬寅朔,二师围郊。癸卯,郊、鄩溃。丁未,晋师在平阴,王师在泽邑。王使告间,庚戌,还。

邾人城翼,还,将自离姑。公孙锄曰:``鲁将御我。」欲自武城还,循山而南。徐锄、丘弱、茅地曰:``道下,遇雨,将不出,是不归也。」遂自离姑。武城人塞其前,断其后之木而弗殊。邾师过之,乃推而蹶之。遂取邾师,获锄、弱、地。

邾人诉于晋,晋人来讨。叔孙蹶如晋,晋人执之。书曰:``晋人执我行人叔孙□若。」言使人也。晋人使与邾大夫坐。叔孙曰:``列国之卿,当小国之君,固周制也。邾又夷也。寡君之命介子服回在,请使当之,不敢废周制故也。」乃不果坐。

韩宣子使邾人取其众,将以叔孙与之。叔孙闻之,去众与兵而朝。士弥牟谓韩宣子曰:``子弗良图,而以叔孙与其仇,叔孙必死之。鲁亡叔孙,必亡邾。邾君亡国,将焉归?子虽悔之,何及?所谓盟主,讨违命也。若皆相执,焉用盟主?」乃弗与,使各居一馆。士伯听其辞而诉诸宣子,乃皆执之。士伯御叔孙,从者四人,过邾馆以如吏。先归邾子。士伯曰:``以刍荛之难,从者之病,将馆子于都。」叔孙旦而立,期焉。乃馆诸箕。舍子服昭伯于他邑。

范献子求货于叔孙,使请冠焉。取其冠法,而与之两冠,曰:``尽矣。」为叔孙故,申丰以货如晋。叔孙曰:``见我,吾告女所行货。」见,而不出。吏人之与叔孙居于箕者,请其吠狗,弗与。及将归,杀而与之食之。叔孙所馆者,虽一日必葺其墙屋,去之如始至。

夏四月乙酉,单子取訾,刘子取墙人、直人。六月壬午,王子朝入于尹。癸未,尹圉诱刘佗杀之。丙戌,单子从阪道,刘子从尹道伐尹。单子先至而败,刘子还。己丑,召伯奂、南宫极以成周人戍尹。庚寅,单子、刘子、樊齐以王如刘。甲午,王子朝入于王城,次于左巷。秋七月戊申,鄩罗纳诸庄宫。尹辛败刘师于唐。丙辰,又败诸鄩。甲子,尹辛取西闱。丙寅,攻蒯,蒯溃。

莒子庚舆虐而好剑,苟铸剑,必试诸人。国人患之。又将叛齐。乌存帅国人以逐之。庚舆将出,闻乌存执殳而立于道左,惧将止死。苑羊牧之曰:``君过之!乌存以力闻可矣,何必以弑君成名?」遂来奔。齐人纳郊公。

吴人伐州来,楚薳越帅师及诸侯之师奔命救州来。吴人御诸钟离。子瑕卒,楚师熸薳。吴公子光曰:``诸侯从于楚者众,而皆小国也。畏楚而不获己,是以来。吾闻之曰:『作事威克其爱,虽小必济』。胡、沈之君幼而狂,陈大夫啮壮而顽,顿与许、蔡疾楚政。楚令尹死,其师熸。帅贱、多宠,政令不壹。而七国同役不同心,帅贱而不能整,无大威命,楚可败也,若分师先以犯胡、沈与陈,必先奔。三国败,诸侯之师乃摇心矣。诸侯乖乱,楚必大奔。请先者去备薄威,后者敦陈整旅。」吴子从之。戊辰晦,战于鸡父。吴子以罪人三千,先犯胡、沈与陈,三国争之。吴为三军以击于后,中军从王,光帅右,掩余帅左。吴之罪人或奔或止,三国乱。吴师击之,三国败,获胡、沈之君及陈大夫。舍胡、沈之囚,使奔许与蔡、顿,曰:``吾君死矣!」师噪而从之,三国奔,楚师大奔。书曰:``胡子髡、沈子逞灭,获陈夏啮。」君臣之辞也。不言战,楚未陈也。

八月丁酉,南宫极震。苌弘谓刘文公曰:``君其勉之!先君之力可济也。周之亡也,其三川震。今西王之大臣亦震,天弃之矣!东王必大克。」

楚大子建之母在狊阜,召吴人而启之。冬十月甲申,吴大子诸樊入狊阜,取楚夫人与其宝器以归。楚司马薳越追之,不及。将死,众曰:``请遂伐吴以徼之。」薳越曰:``再败君师,死且有罪。亡君夫人,不可以莫之死也。」乃缢于薳澨。

公为叔孙故如晋,及河,有疾而复。

楚囊瓦为令尹,城郢。沈尹戌曰:``子常必亡郢!苟不能卫,城无益也。古者,天子守在四夷;天子卑,守在诸侯。诸侯守在四邻;诸侯卑,守在四竟。慎其四竟,结其四援,民狎其野,三务成功,民无内忧,而又无外惧,国焉用城?今吴是惧而城于郢,守己小矣。卑之不获,能无亡乎?昔梁伯沟其公宫而民溃。民弃其上,不亡何待?夫正其疆场,修其土田,险其走集,亲其民人,明其伍候,信其邻国,慎其官守,守其交礼,不僭不贪,不懦不耆,完其守备,以待不虞,又何畏矣?《诗》曰:『无念尔祖,聿修厥德。』无亦监乎若敖、蚡冒至于武、文?土不过同,慎其四竟,犹不城郢。今土数圻,而郢是城,不亦难乎?」

\hypertarget{header-n2753}{%
\subsubsection{昭公二十四年}\label{header-n2753}}

【经】二十四年春王三月丙戌,仲孙玃卒。□若至自晋。夏五月乙未朔,日有食之。秋八月,大雩。丁酉,杞伯郁厘卒。冬,吴灭巢。葬杞平公。

【传】二十四年春,王正月辛丑,召简公、南宫嚚以甘桓公见王子朝。刘子谓苌弘曰:``甘氏又往矣。」对曰:``何害?同德度义。《大誓》曰:『纣有亿兆夷人,亦有离德。余有乱臣十人,同心同德。』此周所以兴也。君其务德,无患无人。」戊午,王子朝入于邬。

晋士弥牟逆叔孙于箕。叔孙使梁其迳待于门内,曰:``余左顾而欬,乃杀之。右顾而笑,乃止。」叔孙见士伯,士伯曰:``寡君以为盟主之故,是以久子。不腆敝邑之礼,将致诸从者。使弥牟逆吾子。」叔孙受礼而归。二月,□若至自晋,尊晋也。

三月庚戌,晋侯使士景伯莅问周故,士伯立于乾祭而问于介众。晋人乃辞王子朝,不纳其使。

夏五月乙未朔,日有食之。梓慎曰:``将水。」昭子曰:``旱也。日过分而阳犹不克,克必甚,能无旱乎?阳不克莫,将积聚也。」

六月壬申,王子朝之师攻瑕及杏,皆溃。

郑伯如晋,子大叔相,见范献子。献子曰:``若王室何?」对曰:``老夫其国家不能恤,敢及王室。抑人亦有言曰:『嫠不恤其纬,而忧宗周之陨,为将及焉。』今王室实蠢蠢焉,吾小国惧矣。然大国之忧也,吾侪何知焉?吾子其早图之!《诗》曰:瓶之罄矣,惟罍之耻。』王室之不宁,晋之耻也。」献子惧,而与宣子图之。乃征会于诸侯,期以明年。

秋八月,大雩,旱也。

冬十月癸酉,王子朝用成周之宝珪于河。甲戌,津人得诸河上。阴不佞以温人南侵,拘得玉者,取其玉,将卖之,则为石。王定而献之,与之东訾。

楚子为舟师以略吴疆。沈尹戌曰:``此行也,楚必亡邑。不抚民而劳之,吴不动而速之,吴踵楚,而疆埸无备,邑能无亡乎?」

越大夫胥犴劳王于豫章之汭。越公子仓归王乘舟,仓及寿梦帅师从王,王及圉阳而还。吴人踵楚,而边人不备,遂灭巢及钟离而还。沈尹戌曰:``亡郢之始,于此在矣。王一动而亡二姓之帅,几如是而不及郢?《诗》曰:『谁生厉阶,至今为梗?』其王之谓乎?」

\hypertarget{header-n2767}{%
\subsubsection{昭公二十五年}\label{header-n2767}}

【经】二十五年春,叔孙□若如宋。夏,叔诣会晋赵鞅、宋乐大心,卫北宫喜、郑游吉、曹人、邾人、滕人、薛人、小邾人于黄父。有鸲鹆来巢。秋七月上辛,大雩;季辛,又雩。九月己亥,公孙于齐,次于阳州。齐侯唁公于野井。冬十月戊辰,叔孙□若卒。十有一月己亥,宋公佐卒于曲棘。十有二月,齐侯取郓。

【传】二十五年春,叔孙□若聘于宋,桐门右师见之。语,卑宋大夫,而贱司城氏。昭子告其人曰:``右师其亡乎!君子贵其身而后能及人,是以有礼。今夫子卑其大夫而贱其宗,是贱其身也,能有礼乎?无礼必亡。」

宋公享昭子,赋《新宫》。昭子赋《车辖》。明日宴,饮酒,乐,宋公使昭子右坐,语相泣也。乐祁佐,退而告人曰:``今兹君与叔孙,其皆死乎?吾闻之:『哀乐而乐哀,皆丧心也。』心之精爽,是谓魂魄。魂魄去之,何以能久?」

季公若之姊为小邾夫人,生宋元夫人,生子以妻季平子。昭子如宋聘,且逆之。公若从,谓曹氏勿与,鲁将逐之。曹氏告公,公告乐祁。乐祁曰:``与之。如是,鲁君必出。政在季氏三世矣,鲁君丧政四公矣。无民而能逞其志者,未之有也。国君是以镇抚其民。《诗》曰:『人之云亡,心之忧矣。』鲁君失民矣,焉得逞其志?靖以待命犹可,动必忧。」

夏,会于黄父,谋王室也。赵简子令诸侯之大夫输王粟,具戍人,曰:``明年将纳王。」子大叔见赵简子,简子问揖让周旋之礼焉。对曰:``是仪也,非礼也。」简子曰:``敢问何谓礼?」对曰:``吉也闻诸先大夫子产曰:『夫礼,天之经也。地之义也,民之行也。』天地之经,而民实则之。则天之明,因地之性,生其六气,用其五行。气为五味,发为五色,章为五声,淫则昏乱,民失其性。是故为礼以奉之:为六畜、五牲、三牺,以奉五味;为九文、六采、五章,以奉五色;为九歌、八风、七音、六律,以奉五声;为君臣、上下,以则地义;为夫妇、外内,以经二物;为父子、兄弟、姑姊、甥舅、昏媾、姻亚,以象天明,为政事、庸力、行务,以从四时;为刑罚、威狱,使民畏忌,以类其震曜杀戮;为温慈、惠和,以效天之生殖长育。民有好、恶、喜、怒、哀、乐,生于六气。是故审则宜类,以制六志。哀有哭泣,乐有歌舞,喜有施舍,怒有战斗;喜生于好,怒生于恶。是故审行信令,祸福赏罚,以制死生。生,好物也;死,恶物也;好物,乐也;恶物,哀也。哀乐不失,乃能协于天地之性,是以长久。」简子曰:``甚哉,礼之大也!」对曰:``礼,上下之纪,天地之经纬也,民之所以生也,是以先王尚之。故人之能自曲直以赴礼者,谓之成人。大,不亦宜乎?」简子曰:``鞅也请终身守此言也。」宋乐大心曰:``我不输粟。我于周为客?」若之何使客?」晋士伯曰:``自践土以来,宋何役之不会,而何盟之不同?曰『同恤王室』,子焉得辟之?子奉君命,以会大事,而宋背盟,无乃不可乎?」右师不敢对,受牒而退。士伯告简子曰:``宋右师必亡。奉君命以使,而欲背盟以干盟主,无不祥大焉。」

『有鸲鹆来巢』,书所无也。师己曰:``异哉!吾闻文、武之世,童谣有之,曰:『鸲之鹆之,公出辱之。鸲鹆之羽,公在外野,往馈之马。鸲鹆跦跦,公在乾侯,征褰与襦。鸲鹆之巢,远哉遥遥。稠父丧劳,宋父以骄。鸲鹆鸲鹆,往歌来哭。』童谣有是,今鸲鹆来巢,其将及乎?」

秋,书再雩,旱甚也。

初,季公鸟娶妻于齐鲍文子,生甲。公鸟死,季公亥与公思展与公鸟之臣申夜姑相其室。及季姒与饔人檀通,而惧,乃使其妾抶己,以示秦遄之妻,曰:``公若欲使余,余不可而抶余。」又诉于公甫,曰:``展与夜姑将要余。」秦姬以告公之,公之与公甫告平子。平子拘展于卞而执夜姑,将杀之。公若泣而哀之,曰:``杀是,是杀余也。」将为之请。平子使竖勿内,日中不得请。有司逆命,公之使速杀之。故公若怨平子。

季、郤之鸡斗。季氏介其鸡,郤氏为之金距。平子怒,益宫于郤氏,且让之。故郤昭伯亦怨平子。臧昭伯之从弟会,为谗于臧氏,而逃于季氏,臧氏执旃。平子怒,拘臧氏老。将褅于襄公,万者二人,其众万于季氏。臧孙曰:``此之谓不能庸先君之庙。」大夫遂怨平子。公若献弓于公为,且与之出射于外,而谋去季氏。公为告公果、公贲。公果、公贲使侍人僚柤告公。公寝,将以戈击之,乃走。公曰:``执之。」亦无命也。惧而不出,数月不见,公不怒。又使言,公执戈惧之,乃走。又使言,公曰:``非小人之所及也。」公果自言,公以告臧孙,臧孙以难。告郤孙,郤孙以可,劝。告子家懿伯,懿伯曰:``谗人以君侥幸,事若不克,君受其名,不可为也。舍民数世,以求克事,不可必也。且政在焉,其难图也。」公退之。辞曰:``臣与闻命矣,言若泄,臣不获死。」乃馆于公。

叔孙昭子如阚,公居于长府。九月戊戌,伐季氏,杀公之于门,遂入之。平子登台而请曰:``君不察臣之罪,使有司讨臣以干戈,臣请待于沂上以察罪。」弗许。请囚于费,弗许。请以五乘亡,弗许。子家子曰:``君其许之!政自之出久矣,隐民多取食焉。为之徒者众矣,日入慝作,弗可知也。众怒不可蓄也,蓄而弗治,将温。温畜,民将生心。生心,同求将合。君必悔之。」弗听。郤孙曰:``必杀之。」公使郤孙逆孟懿子。叔孙氏之司马鬷戾言于其众曰:``若之何?」莫对。又曰:``我,家臣也,不敢知国。凡有季氏与无,于我孰利?」皆曰:``无季氏,是无叔孙氏也。」鬷戾曰:``然则救诸!」帅徒以往,陷西北隅以入。公徒释甲,执冰而踞。遂逐之。孟氏使登西北隅,以望季氏。见叔孙氏之旌,以告。孟氏执郈昭伯,杀之于南门之西,遂伐公徒。子家子曰:``诸臣伪劫君者,而负罪以出,君止。意如之事君也,不敢不改。」公曰:``余不忍也。」与臧孙如墓谋,遂行。

己亥,公孙于齐,次于阳州。齐侯将唁公于平阴,公先于野井。齐侯曰:``寡人之罪也。使有司待于平阴,为近故也。」书曰:``公孙于齐,次于阳州,齐侯唁公于野井。」礼也。将求于人,则先下之,礼之善物也。齐侯曰:``自莒疆以西,请致千社,以待君命。寡人将帅敝赋以从执事,唯命是听,君之忧,寡人之忧也。」公喜。子家子曰:``天禄不再,天若胙君,不过周公,以鲁足矣。失鲁,而以千社为臣,谁与之立」且齐君无信,不如早之晋。」弗从。臧昭伯率从者将盟,载书曰:``戮力壹心,好恶同之。信罪之有无,缱绻从公,无通外内。」以公命示子家子。子家子曰:``如此,吾不可以盟,羁也不佞,不能与二三子同心,而以为皆有罪。或欲通外内,且欲去君。二三子好亡而恶定,焉可同也?陷君于难,罪孰大焉?通外内而去君,君将速入,弗通何为?而何守焉?」乃不与盟。

昭子自阚归,见平子。平子稽颡,曰:``子若我何?」昭子曰:``人谁不死?子以逐君成名,子孙不忘,不亦伤乎!将若子何?」平子曰:``苟使意如得改事君,所谓生死而肉骨也。」昭子从公于齐,与公言。子家子命适公馆者执之。公与昭子言于幄内,曰将安众而纳公。公徒将杀昭子,伏诸道。左师展告公,公使昭子自铸归。平子有异志。冬十月辛酉,昭子齐于其寝,使祝宗祈死。戊辰,卒。左师展将以公乘马而归,公徒执之。

壬申,尹文公涉于巩,焚东訾,弗克。

十一月,宋元公将为公故如晋。梦大子栾即位于庙,己与平公服而相之。旦,召六卿。公曰:``寡人不佞,不能事父兄,以为二三子忧,寡人之罪也。若以群子之灵,获保首领以没,唯是匾柎所以藉干者,请无及先君。」仲几对曰:``君若以社稷之故,私降昵宴,群臣弗敢知。若夫宋国之法,死生之度,先君有命矣。群臣以死守之,弗敢失队。臣之失职,常刑不赦。臣不忍其死,君命只辱。」宋公遂行。己亥,卒于曲棘。

十二月庚辰,齐侯围郓。

初,臧昭伯如晋,臧会窃其宝龟偻句,以卜为信与僭,僭吉。臧氏老将如晋问,会请往。昭伯问家故,尽对。及内子与母弟叔孙,则不对。再三问,不对。归,及郊,会逆,问,又如初。至,次于外而察之,皆无之。执而戮之,逸,奔郤。郤鲂假使为贾正焉。计于季氏。臧氏使五人以戈盾伏诸桐汝之闾。会出,逐之,反奔,执诸季氏中门之外。平子怒,曰:``何故以兵入吾门?」拘臧氏老。季、臧有恶。及昭伯从公,平子立臧会。会曰:``偻句不馀欺也。」

楚子使薳射城州屈,复茄人焉。城丘皇,迁訾人焉。使熊相衣某郭巢,季然郭卷。子大叔闻之,曰:``楚王将死矣。使民不安其土,民必忧,忧将及王,弗能久矣。」

\hypertarget{header-n2787}{%
\subsubsection{昭公二十六年}\label{header-n2787}}

【经】二十有六年春王正月,葬宋元公。三月,公至自齐,居于郓。夏,公围成。秋,公会齐侯、莒子、邾子、杞伯,盟于鄟陵。公至自会,居于郓。九月庚申,楚子居卒。冬十月,天王入于成周。尹氏、召伯、毛伯以王子朝奔楚。

【传】二十六年春,王正月庚申,齐侯取郓。

葬宋元公,如先君,礼也。

三月,公至自齐,处于郓,言鲁地也。

夏,齐侯将纳公,命无受鲁货。申丰从女贾,以币锦二两,缚一如瑱,适齐师。谓子犹之人高齮:``能货子犹,为高氏后,粟五千庾。」高齮以锦示子犹,子犹欲之。能货子犹,为高氏后,粟五千庚。高齮以锦示子犹,子犹欲之。齮曰:``鲁人买之,百两一布,以道之不通,先入币财。」子犹受之,言于齐侯曰:``群臣不尽力于鲁君者,非不能事君也。然据有异焉。宋元公为鲁君如晋,卒于曲棘。叔孙昭子求纳其君,无疾而死。不知天之弃鲁耶,抑鲁君有罪于鬼神,故及此也?君若待于曲棘,使群臣从鲁君以卜焉。若可,师有济也。君而继之,兹无敌矣。若其无成,君无辱焉。」齐侯从之,使公子锄帅师从公。成大夫公孙朝谓平子曰:``有都以卫国也,请我受师。」许之。请纳质,弗许,曰:``信女,足矣。」告于齐师曰:``孟氏,鲁之敝室也。用成已甚,弗能忍也,请息肩于齐。」齐师围成。成人伐齐师之饮马于淄者,曰:``将以厌众。」鲁成备而后告曰:``不胜众。」师及齐师战于炊鼻。齐子渊捷从泄声子,射之,中楯瓦。繇朐汰輈,匕入者三寸。声子射其马,斩鞅,殪。改驾,人以为鬷戾也而助之。子车曰:``齐人也。」将击子车,子车射之,殪。其御曰:``又之。」子车曰:``众可惧也,而不可怒也。」子囊带从野泄,叱之。泄曰:``军无私怒,报乃私也,将亢子。」又叱之,亦叱之。冉竖射陈武子,中手,失弓而骂。以告平子,曰:``有君子白皙,鬒须眉,甚口。」平子曰:``必子强也,无乃亢诸?」对曰:``谓之君子,何敢亢之?」林雍羞为颜鸣右,下。苑何忌取其耳,颜鸣去之。苑子之御曰:``视下顾。」苑子刜林雍,断其足。\{轻金\}而乘于他车以归,颜鸣三入齐师,呼曰:``林雍乘!」

四月,单子如晋告急。五月戊午,刘人败王城之师于尸氏。戊辰,王城人、刘人战于施谷,刘师败绩。

秋,盟于鄟陵,谋纳公也。

七月己巳,刘子以王出。庚午,次于渠。王城人焚刘。丙子,王宿于褚氏。丁丑,王次于萑谷。庚辰,王入于胥靡。辛巳,王次于滑。晋知跞、赵鞅帅师纳王,使汝宽守关塞。

九月,楚平王卒。令尹子常欲立子西,曰:``大子壬弱,其母非适也,王子建实聘之。子西长而好善。立长则顺,建善则治。王顺国治,可不务乎?」子西怒曰:``是乱国而恶君王也。国有外援,不可渎也。王有适嗣,不可乱也。败亲、速仇、乱嗣,不祥,我受其名。赂吾以天下,吾滋不从也。楚国何为?必杀令尹!」令尹惧,乃立昭王。

冬十月丙申,王起师于滑。辛丑,在郊,遂次于尸。十一月辛酉,晋师克巩。召伯盈逐王子朝,王子朝及召氏之族、毛伯得、尹氏固、南宫嚚奉周之典籍以奔楚。阴忌奔莒以叛。召伯逆王于尸,及刘子、单子盟。遂军圉泽,次于堤上。癸酉,王入于成周。甲戌,盟于襄宫。晋师使成公般戍周而还。十二月癸未,王入于庄宫。

王子朝使告于诸侯曰:``昔武王克殷,成王靖四方,康王息民,并建母弟,以蕃屏周。亦曰:『吾无专享文、武之功,且为后人之迷败倾覆,而溺入于难,则振救之。』至于夷王,王愆于厥身,诸侯莫不并走其望,以祈王身。至于厉王,王心戾虐,万民弗忍,居王于彘。诸侯释位,以间王政。宣王有志,而后效官。至于幽王,天不吊周,王昏不若,用愆厥位。携王奸命,诸侯替之,而建王嗣,用迁郏鄏。则是兄弟之能用力于王室也。至于惠王,天不靖周,生颓祸心,施于叔带,惠、襄辟难,越去王都。则有晋、郑,咸黜不端,以绥定王家。则是兄弟之能率先王之命也。在定王六年,秦人降妖,曰:『周其有王,亦克能修其职。诸侯服享,二世共职。王室其有间王位,诸侯不图,而受其乱灾。』至于灵王,生而有。王甚神圣,无恶于诸侯。灵王、景王,克终其世。今王室乱,单旗、刘狄,剥乱天下,壹行不若。谓:『先王何常之有?唯余心所命,其谁敢请之?』帅群不吊之人,以行乱于王室。侵欲无厌,规求无度,贯渎鬼神,慢弃刑法,倍奸齐盟,傲很威仪,矫诬先王。晋为不道,是摄是赞,思肆其罔极。兹不谷震荡播越,窜在荆蛮,未有攸厎。若我一二兄弟甥舅,奖顺天法,无助狡猾,以从先王之命,毋速天罚,赦图不谷,则所愿也。敢尽布其腹心,及先王之经,实深图之。昔先王之命曰:『王后无适,则择立长。年钧以德,德钧以卜。』王不立爱,公卿无私,古之制也。穆后及大子寿早夭即世,单、刘赞私立少,以间先王,亦唯伯仲叔季图之!」

闵马父闻子朝之辞,曰:``文辞以行礼也。子朝干景之命,远晋之大,以专其志,无礼甚矣,文辞何为?」

齐有彗星,齐侯使禳之。晏子曰:``无益也,只取诬焉。天道不谄,不贰其命,若之何禳之?且天之有彗也,以除秽也。君无秽德,又何禳焉?若德之秽,禳之何损?《诗》曰:『惟此文王,小心翼翼,昭事上帝,聿怀多福。厥德不回,以受方国。』君无违德,方国将至,何患于彗?《诗》曰:『我无所监,夏后及商。用乱之故,民卒流亡。』若德回乱,民将流亡,祝史之为,无能补也。」公说,乃止。

齐侯与晏子坐于路寝,公叹曰:``美哉室!其谁有此乎?」晏子曰:``敢问何谓也?」公曰:``吾以为在德。」对曰:``如君之言,其陈氏乎!陈氏虽无大德,而有施于民。豆区釜钟之数,其取之公也簿,其施之民也厚。公厚敛焉,陈氏厚施焉,民归之矣。《诗》曰:『虽无德与女,式歌且舞。』陈氏之施,民歌舞之矣。后世若少惰,陈氏而不亡,则国其国也已。」公曰:``善哉!是可若何?」对曰:``唯礼可以已之。在礼,家施不及国,民不迁,农不移,工贾不变,士不滥,官不滔,大夫不收公利。」公曰:``善哉!我不能矣。吾今而后知礼之可以为国也。」对曰:``礼之可以为国也久矣。与天地并。君令臣共,父慈子孝,兄爱弟敬,夫和妻柔,姑慈妇听,礼也。君令而不违,臣共而不贰,父慈而教,子孝而箴;兄爱而友,弟敬而顺;夫和而义,妻柔而正;姑慈而从,妇听而婉:礼之善物也。」公曰:``善哉!寡人今而后闻此礼之上也。」对曰:``先王所禀于天地,以为其民也,是以先王上之。」

\hypertarget{header-n2804}{%
\subsubsection{昭公二十七年}\label{header-n2804}}

【经】二十有七年春,公如齐。公至自齐,居于郓。夏四月,吴弑其君僚。楚杀其大夫郤宛。秋,晋士鞅、宋乐祁犁、卫北宫喜、曹人、邾人、滕人会于扈。冬十月,曹伯午卒。邾快来奔。公如齐。公至自齐,居于郓。

【传】二十七年春,公如齐。公至自齐,处于郓,言在外也。

吴子欲因楚丧而伐之,使公子掩余、公子烛庸帅师围潜。使延州来季子聘于上国,遂聘于晋,以观诸侯。楚莠尹然,工尹麇帅师救潜。左司马沈尹戌帅都君子与王马之属以济师,与吴师遇于穷。令尹子常以舟师及沙汭而还。左尹郤宛、工尹寿帅师至于潜,吴师不能退。

吴公子光曰:``此时也,弗可失也。」告鱄设诸曰:``上国有言曰:『不索何获?』我,王嗣也,吾欲求之。事若克,季子虽至,不吾废也。」鱄设诸曰:``王可弑也。母老子弱,是无若我何。」光曰:``我,尔身也。」

夏四月,光伏甲于堀室而享王。王使甲坐于道,及其门。门阶户席,皆王亲也,夹之以铍。羞者献体改服于门外,执羞者坐行而入,执铍者夹承之,及体以相授也。光伪足疾,入于堀室。鱄设诸置剑于鱼中以进,抽剑剌王,铍交于胸,遂弑王。阖庐以其子为卿。

季子至,曰:``苟先君废无祀,民人无废主,社稷有奉,国家无倾,乃吾君也。吾谁敢怨?哀死事生,以待天命。非我生乱,立者从之,先人之道也。」覆命哭墓,复位而待。吴公子掩余奔徐,公子烛庸奔钟吾。楚师闻吴乱而还。

郤宛直而和,国人说之。鄢将师为右领,与费无极比而恶之。令尹子常贿而信谗,无极谮郤宛焉,谓子常曰:``子恶欲饮子酒。」又谓子恶:``令尹欲饮酒于子氏。」子恶曰:``我,贱人也,不足以辱令尹。令尹将必来辱,为惠已甚。吾无以酬之,若何?」无极曰:``令尹好甲兵,子出之,吾择焉。」取五甲五兵,曰:``置诸门,令尹至,必观之,而从以酬之。」及飨日,帷诸门左。无极谓令尹曰:``吾几祸子。子恶将为子不利,甲在门矣,子必无往。且此役也,吴可以得志,子恶取赂焉而还,又误群帅,使退其师,曰:『乘乱不祥。』吴乘我丧,我乘其乱,不亦可乎?」令尹使视郤氏,则有甲焉。不往,召鄢将师而告之。将师退,遂令攻郤氏,且爇之。子恶闻之,遂自杀也。国人弗爇,令曰:``爇郤氏,与之同罪。」或取一编菅焉,或取一秉秆焉,国人投之,遂弗也。令尹炮之,尽灭郤氏之族党,杀阳令终与其弟完及佗与晋陈及其子弟。晋陈之族呼于国曰:``鄢氏、费氏自以为王,专祸楚国,弱寡王室,蒙王与令尹以自利也。令尹尽信之矣,国将如何?」令尹病之。

秋,会于扈,令戍周,且谋纳公也。宋、卫皆利纳公,固请之。范献子取货于季孙,谓司城子梁与北宫贞子曰:``季孙未知其罪,而君伐之,请囚,请亡,于是乎不获。君又弗克,而自出也。夫\{山乙\}无备而能出君乎?季氏之复,天救之也。休公徒之怒,而启叔孙氏之心。不然,岂其伐人而说甲执冰以游?叔孙氏惧祸之滥,而自同于季氏,天之道也。鲁君守齐,三年而无成。季氏甚得其民,淮夷与之,有十年之备,有齐、楚之援,有天之赞,有民之助,有坚守之心,有列国之权,而弗敢宣也,事君如在国。故鞅以为难。二子皆图国者也,而欲纳鲁君,鞅之愿也,请从二子以围鲁。无成,死之。」二子惧,皆辞。乃辞小国,而以难复。

孟懿子、阳虎伐郓。郓人将战,子家子曰:``天命不慆久矣。使君亡者,必此众也。天既祸之,而自福也,不亦难乎?犹有鬼神,此必败也。乌呼!为无望也夫,其死于此乎!」公使子家子如晋,公徒败于且知。

楚郤宛之难,国言未已,进胙者莫不谤令尹。沈尹戌言于子常曰:``夫左尹与中厩尹莫知其罪,而子杀之,以兴谤讟,至于今不已。戌也惑之。仁者杀人以掩谤,犹弗为也。今吾子杀人以兴谤,而弗图,不亦异乎?夫无极,楚之谗人也,民莫不知。去朝吴,出蔡侯朱,丧太子建,杀连尹奢,屏王之耳目,使不聪明。不然,平王之温惠共俭,有过成、庄,无不及焉。所以不获诸侯,迩无极也。今又杀三不辜,以兴大谤,几及子矣。子而不图,将焉用之?夫鄢将师矫子之命,以灭三族,国之良也,而不愆位。吴新有君,疆埸日骇,楚国若有大事,子其危哉!知者除谗以自安也,今子爱谗以自危也,甚矣其惑也!」子常曰:``是瓦之罪,敢不良图。」九月己未,子常杀费无极与鄢将师,尽灭其族,以说于国。谤言乃止。

冬,公如齐,齐侯请飨之。子常子曰:``朝夕立于其朝,又何飨焉?其饮酒也。」乃饮酒,使宰献,而请安。子仲之子曰重,为齐侯夫人,曰:``请使重见。」子家子乃以君出。

十二月,晋籍秦致诸侯之戍于周,鲁人辞以难。

\hypertarget{header-n2819}{%
\subsubsection{昭公二十八年}\label{header-n2819}}

【经】二十有八年春王三月,葬曹悼公。公如晋,次于乾侯。夏四月丙戌,郑伯宁卒。六月,葬郑定公。秋七月癸巳,滕子宁卒。冬,葬滕悼公。

【传】二十八年春,公如晋,将如乾侯。子家子曰:``有求于人,而即其安,人孰矜之?其造于竟。」弗听。使请逆于晋。晋人曰:``天祸鲁国,君淹恤在外。君亦不使一个辱在寡人,而即安于甥舅,其亦使逆君?」使公复于竟而后逆之。

晋祁胜与邬臧通室,祁盈将执之,访于司马叔游。叔游曰:``《郑书》有之:『恶直丑正,实蕃有徒。』无道立矣,子惧不免。《诗》曰:『民之多辟,无自立辟。』姑已,若何?」盈曰:``祁氏私有讨,国何有焉?」遂执之。祁胜赂荀跞,荀跞为之言于晋侯,晋侯执祁盈。祁盈之臣曰:``钧将皆死,憖使吾君闻胜与臧之死以为快。」乃杀之。夏六月,晋杀祁盈及杨食我。食我,祁盈之党也,而助乱,故杀之。遂灭祁氏、羊舌氏。

初,叔向欲娶于申公巫臣氏,其母欲娶其党。叔向曰:``吾母多而庶鲜,吾惩舅氏矣。」其母曰:``子灵之妻杀三夫,一君,一子,而亡一国、两卿矣。可无惩乎?吾闻之:『甚美必有甚恶,』是郑穆少妃姚子之子,子貉之妹也。子貉早死,无后,而天钟美于是,将必以是大有败也。昔有仍氏生女,鬒黑而甚美,光可以鉴,名曰玄妻。乐正后夔取之,生伯封,实有豕心,贪婪无餍,忿类无期,谓之封豕。有穷后羿灭之,夔是以不祀。且三代之亡,共子之废,皆是物也。女何以为哉?夫有尤物,足以移人,苟非德义,则必有祸。」叔向惧,不敢取。平公强使取之,生伯石。伯石始生,子容之母走谒诸姑,曰:``长叔姒生男。」姑视之,及堂,闻其声而还,曰:``是豺狼之声也。狼子野心,非是,莫丧羊舌氏矣。」遂弗视。

秋,晋韩宣子卒,魏献子为政。分祁氏之田以为七县,分羊舌氏之田以为三县。司马弥牟为邬大夫,贾辛为祁大夫,司马乌为平陵大夫,魏戊为梗阳大夫,知徐吾为涂水大夫,韩固为马首大夫,孟丙为盂大夫,乐霄为铜鞮大夫,赵朝为平阳大夫,僚安为杨氏大夫。谓贾辛、司马乌为有力于王室,故举之。谓知徐吾、赵朝、韩固、魏戊,余子之不失职,能守业者也。其四人者,皆受县而后见于魏子,以贤举也。

魏子谓成鱄:``吾与戊也县,人其以我为党乎?」对曰:``何也?戊之为人也,远不忘君,近不逼同,居利思义,在约思纯,有守心而无淫行。虽与之县,不亦可乎?昔武王克商,光有天下。」其兄弟之国者十有五人,姬姓之国者四十人,皆举亲也。夫举无他,唯善所在,亲疏一也。《诗》曰:『唯此文王,帝度其心。莫其德音,其德克明。克明克类,克长克君。王此大国,克顺克比。比于文王,其德靡悔。既受帝祉,施于孙子。』心能制义曰度,德正应和曰莫,照临四方曰明,勤施无私曰类,教诲不倦曰长,赏庆刑威曰君,慈和遍服曰顺,择善而从之曰比,经纬天地曰文。九德不愆,作事无悔,故袭天禄,子孙赖之。主之举也,近文德矣,所及其远哉!」

贾辛将适其县,见于魏子。魏子曰:``辛来!昔叔向适郑,鬲□蔑恶,欲观叔向,从使之收器者而往,立于堂下。一言而善。叔向将饮酒,闻之,曰:『必鬷明也。』下,执其手以上,曰『昔贾大夫恶,娶妻而美,三年不言不笑,御以如皋,射雉,获之。其妻始笑而言。贾大夫曰:``才之不可以已,我不能射,女遂不言不笑夫!」今子少不扬,子若无言,吾几失子矣。言不可以已也如是。』遂知故在。今女有力于王室,吾是以举女。行乎!敬之哉!毋堕乃力!」

仲尼闻魏子之举也,以为义,曰:``近不失亲,远不失举,可谓义矣。」又闻其命贾辛也,以为忠:``《诗》曰:『永言配命,自求多福』,忠也。魏子之举也义,其命也忠,其长有后于晋国乎!」

冬,梗阳人有狱,魏戊不能断,以狱上。其大宗赂以女乐,魏子将受之。魏戊谓阎没、女宽曰:``主以不贿闻于诸侯,若受梗阳人,贿莫甚焉。吾子必谏。」皆许诺。退朝,待于庭。馈入,召之。比置,三叹。既食,使坐。魏子曰:」吾闻诸伯叔,谚曰:『唯食忘忧。』吾子置食之间三叹,何也?」同辞而对曰:``或赐二小人酒,不夕食。馈之始至,恐其不足,是以叹。中置,自咎曰:『岂将军食之,而有不足?』是以再叹。及馈之毕,愿以小人之腹为君子之心,属厌而已。」献子辞梗阳人。

\hypertarget{header-n2831}{%
\subsubsection{昭公二十九年}\label{header-n2831}}

【经】二十有九年春,公至自乾侯,居于郓,齐侯使高张来唁公。公如晋,次于乾侯。夏四月庚子,叔诣卒。秋七月。冬十月,郓溃。

【传】二十九年春,公至自乾侯,处于郓。齐侯使高张来唁公,称主君。子家子曰:``齐卑君矣,君只辱焉。」公如乾侯。

三月己卯,京师杀召伯盈、尹氏固及原伯鲁之子。尹固之复也,有妇人遇之周郊,尤之,曰:``处则劝人为祸,行则数日而反,是夫也,其过三岁乎?」

夏五月庚寅,王子赵车入于鄻以叛,阴不佞败之。

平子每岁贾马,具从者之衣屦,而归之于乾侯。公执归马者,卖之,乃不归马。卫侯来献其乘马曰启服,堑而死,公将为之椟。子家子曰:``从者病矣,请以食之。」乃以帏裹之。

公赐公衍羔裘,使献龙辅于齐侯,遂入羔裘。齐侯喜,与之阳谷。公衍、公为之生也,其母偕出。公衍先生,公为之母曰:``相与偕出,请相与偕告。」三日,公为生,其母先以告,公为为兄。公私喜于阳谷而思于鲁,曰:``务人为此祸也。且后生而为兄,其诬也久矣。」乃黜之,而以公衍为大子。

秋,龙见于绛郊。魏献子问于蔡墨曰:``吾闻之,虫莫知于龙,以其不生得也。谓之知,信乎?」对曰:``人实不知,非龙实知。古者畜龙,故国有豢龙氏,有御龙氏。」献子曰:``是二氏者,吾亦闻之,而知其故,是何谓也?」对曰:``昔有飂叔安,有裔子曰董父,实甚好龙,能求其耆欲以饮食之,龙多归之。乃扰畜龙,以服事帝舜。帝赐之姓曰董,氏曰豢龙。封诸鬷川,鬷夷氏其后也。故帝舜氏世有畜龙。及有夏孔甲,扰于有帝,帝赐之乘龙,河、汉各二,各有雌雄,孔甲不能食,而未获豢龙氏。有陶唐氏既衰,其后有刘累,学扰龙于豢龙氏,以事孔甲,能饮食之。夏后嘉之,赐氏曰御龙,以更豕韦之后。龙一雌死,潜醢以食夏后。夏后飨之,既而使求之。惧而迁于鲁县,范氏其后也。」献子曰:``今何故无之?」对曰:``夫物,物有其官,官修其方,朝夕思之。一日失职,则死及之。失官不食。官宿其业,其物乃至。若泯弃之,物乃坻伏,郁湮不育。故有五行之官,是谓五官。实列受氏姓,封为上公,祀为贵神。社稷五祀,是尊是奉。木正曰句芒,火正曰祝融,金正曰蓐收,水正曰玄冥,土正曰后土。龙,水物也。水官弃矣,故龙不生得。不然,《周易》有之,在《乾》ⅰⅰ之《姤》ⅰⅳ,曰:『潜龙勿用。』其《同人》ⅰⅵ曰:『见龙在田。』其《大有》ⅵⅰ曰:『飞龙在天。』其《夬》ⅷⅰ曰:『亢龙有悔。』其《坤》ⅱⅱ曰:『见群龙无首,吉。』《坤》之《剥》ⅶⅱ曰:『龙战于野。』若不朝夕见,谁能物之?」献子曰:``社稷五祀,谁氏之五官也?」对曰:``少皞氏有四叔,曰重、曰该、曰修、曰熙,实能金、木及水。使重为句芒,该为蓐收,修及熙为玄冥,世不失职,遂济穷桑,此其三祀也。颛顼氏有子曰犁,为祝融;共工氏有子曰句龙,为后土,此其二祀也。后土为社;稷,田正也。有烈山氏之子曰柱为稷,自夏以上祀之。周弃亦为稷,自商以来祀之。」

冬,晋赵鞅、荀寅帅师城汝滨,遂赋晋国一鼓铁,以铸刑鼎,着范宣子所为刑书焉。仲尼曰:``晋其亡乎!失其度矣。夫晋国将守唐叔之所受法度,以经纬其民,卿大夫以序守之。民是以能尊其贵,贵是以能守其业。贵贱不愆,所谓度也。文公是以作执秩之官,为被庐之法,以为盟主。今弃是度也,而为刑鼎,民在鼎矣,何以尊贵?贵何业之守?贵贱无序,何以为国?且夫宣子之刑,夷之蒐也,晋国之乱制也,若之何以为法?蔡史墨曰:``范氏、中行氏其亡乎!中行寅为下卿,而干上令,擅作刑器,以为国法,是法奸也。又加范氏焉,易之,亡也。其及赵氏,赵孟与焉。然不得已,若德,可以免。」

\hypertarget{header-n2842}{%
\subsubsection{昭公三十年}\label{header-n2842}}

【经】三十年春王正月,公在乾侯。夏六月庚辰,晋侯去疾卒。秋八月,葬晋顷公。冬十有二月,吴灭徐,徐子章羽奔楚。

【传】三十年春,王正月,公在乾侯。不先书郓与乾侯,非公,且征过也。

夏六月,晋顷公卒。秋八月,葬。郑游吉吊,且送葬,魏献子使士景伯诘之,曰:``悼公之丧,子西吊,子蟜送葬。今吾子无贰,何故?」对曰:``诸侯所以归晋君,礼也。礼也者,小事大,大字小之谓。事大在共其时命,字小在恤其所无。以敝邑居大国之间,共其职贡,与其备御不虞之患,岂忘共命?先王之制:诸侯之丧,士吊,大夫送葬;唯嘉好、聘享、三军之事,于是乎使卿。晋之丧事,敝邑之间,先君有所助执绋矣。若其不间,虽士大夫有所不获数矣。大国之惠,亦庆其加,而不讨其乏,明厎其情,取备而已,以为礼也。灵王之丧,我先君简公在楚,我先大夫印段实往,敝邑之少卿也。王吏不讨,恤所无也。今大夫曰:『女盍从旧?』旧有丰有省,不知所从。从其丰,则寡君幼弱,是以不共。从其省,则吉在此矣。唯大夫图之。」晋人不能诘。

吴子使徐人执掩余,使钟吾人执烛庸二公子奔楚,楚子大封,而定其徙。使监马尹大心逆吴公子,使居养莠尹然、左司马沈尹戌城之,取于城父与胡田以与之。将以害吴也。子西谏曰:``吴光新得国,而亲其民,视民如子,辛苦同之,将用之也。若好吴边疆,使柔服焉,犹惧其至。吾又疆其仇以重怒之,无乃不可乎!吴,周之胄裔也,而弃在海滨,不与姬通。今而始大,比于诸华。光又甚文,将自同于先王。不知天将以为虐乎,使翦丧吴国而封大异姓乎?其抑亦将卒以祚吴乎?其终不远矣。我盍姑亿吾鬼神,而宁吾族姓,以待其归。将焉用自播扬焉?」王弗听。吴子怒。冬十二月,吴子执钟吴子,遂伐徐,防山以水之。己卯,灭徐。徐子章禹断其发,携其夫人,以逆吴子。吴子唁而送之,使其迩臣从之,遂奔楚。楚沈尹戌帅师救徐,弗及,遂城夷,使徐子处之。

吴子问于伍员曰:``初而言伐楚,余知其可也,而恐其使余往也,又恶人之有馀之功也。今余将自有之矣,伐楚何如?」对曰:``楚执政众而乖,莫适任患。若为三师以肄焉,一师至,彼必皆出。彼出则归,彼归则出,楚必道敝。亟肄以罢之,多方以误之。既罢而后以三军继之,必大克之。」阖庐从之,楚于是乎始病。

\hypertarget{header-n2850}{%
\subsubsection{昭公三十一年}\label{header-n2850}}

【经】三十有一年春王正月,公在乾侯。季孙意如晋荀跞于适历。夏四月丁巳,薛伯谷卒。晋侯使荀跞唁公于乾侯。秋,葬薛献公。冬,黑肱以滥来奔。十有二月辛亥朔,日有食之。

【传】三十一年春,王正月,公在乾侯,言不能外内也。

晋侯将以师纳公。范献子曰:``若召季孙而不来,则信不臣矣。然后伐之,若何?」晋人召季孙,献子使私焉,曰:``子必来,我受其无咎。」季孙意如会晋荀跞于适历。荀跞曰:``寡君使跞谓吾子:『何故出君?有君不事,周有常刑,子其图之!』」季孙练冠麻衣跣行,伏而对曰:``事君,臣之所不得也,敢逃刑命?君若以臣为有罪,请囚于费,以待君之察也,亦唯君。若以先臣之故,不绝季氏,而赐之死。若弗杀弗亡,君之惠也,死且不朽。若得从君而归,则固臣之愿也。敢有异心?」

夏四月,季孙从知伯如乾侯。子家子曰:``君与之归。一惭之不忍,而终身惭乎?」公曰:``诺。」众曰:``在一言矣,君必逐之。」荀跞以晋侯之命唁公,且曰:``寡君使跞以君命讨于意如,意如不敢逃死,君其入也!」公曰:``君惠顾先君之好,施及亡人将使归粪除宗祧以事君,则不能夫人。己所能见夫人者,有如河!」荀跞掩耳而走,曰:``寡君其罪之恐,敢与知鲁国之难?臣请复于寡君。」退而谓季孙:``君怒未怠,子姑归祭。」子家子曰:``君以一乘入于鲁师,季孙必与君归。」公欲从之,众从者胁公,不得归。

薛伯谷卒,同盟,故书。

秋,吴人侵楚,伐夷,侵潜、六。楚沈尹戌帅师救潜,吴师还。楚师迁潜于南冈而还。吴师围弦。左司马戌、右司马稽帅师救弦,及豫章。吴师还。始用子胥之谋也。

冬,邾黑肱以滥来奔,贱而书名,重地故也。君子曰:``名之不可不慎也如是。夫有所名,而不如其已。以地叛,虽贱,必书地,以名其人。终为不义,弗可灭已。是故君子动则思礼,行则思义,不为利回,不为义疚。或求名而不得,或欲盖而名章,惩不义也。齐豹为卫司寇,守嗣大夫,作而不义,其书为『盗』。邾庶其、莒牟夷、邾黑肱以土地出,求食而已,不求其名,贱而必书。此二物者,所以惩肆而去贪也。若艰难其身,以险危大人,而有名章彻,攻难之士将奔走之。若窃邑叛君,以徼大利而无名,贪冒之民将置力焉。是以《春秋》书齐豹曰『盗』,三叛人名,以惩不义,数恶无礼,其善志也。故曰:《春秋》之称微而显,婉而辨。上之人能使昭明,善人劝焉,淫人惧焉,是以君子贵之。」

十二月辛亥朔,日有食之。是夜也,赵简子梦童子羸而转以歌。旦占诸史墨,曰:``吾梦如是,今而日食,何也?」对曰:``六年及此月也,吴其入郢乎!终亦弗克。入郢,必以庚辰,日月在辰尾。庚午之日,日始有谪。火胜金,故弗克。」

\hypertarget{header-n2861}{%
\subsubsection{昭公三十二年}\label{header-n2861}}

【经】三十有二年春王正月,公在乾侯。取阚。夏,吴伐越。秋七月。冬,仲孙何忌会晋韩不信、齐高张、宋仲几、卫世叔申、郑国参、曹人、莒人、薛人、杞人、小邾人城成周。十有二月己未,公薨于乾侯。

【传】三十二年春,王正月,公在乾侯。言不能外内,又不能用其人也。

夏,吴伐越,始用师于越也。史墨曰:``不及四十年,越其有吴乎!越得岁而吴伐之,必受其凶。」

秋八月,王使富辛与石张如晋,请城成周。天子曰:``天降祸于周,俾我兄弟并有乱心,以为伯父忧。我一二亲昵甥舅,不遑启处,于今十年,勤戍五年。余一人无日忘之,闵闵焉如农夫之望岁,惧以待时。伯父若肆大惠,复二文之业,驰周室之忧,徼文、武之福,以固盟主,宣昭令名,则余一人有大愿矣。昔成王合诸侯,城成周,以为东都,崇文德焉。今我欲徼福假灵于成王,修成周之城,俾戍人无勤,诸侯用宁,蝥贼远屏,晋之力也。其委诸伯父,使伯父实重图之。俾我一人无征怨于百姓,而伯父有荣施,先王庸之。」范献子谓魏献子曰:``与其戍周,不如城之。天子实云,虽有后事,晋勿与知可也。从王命以纾诸侯,晋国无忧。是之不务,而又焉从事?」魏献子曰:``善!」使伯音对曰:``天子有命,敢不奉承,以奔告于诸侯。迟速衰序,于是焉在。」

冬十一月,晋魏舒、韩不信如京师,合诸侯之大夫于狄泉,寻盟,且令城成周。魏子南面。卫彪徯曰:``魏子必有大咎。干位以令大事,非其任也。《诗》曰:『敬天之怒,不敢戏豫。敬天之渝,不敢驰驱。』况敢干位以作大事乎?」

己丑,士弥牟营成周,计丈数,揣高卑,度厚薄,仞沟恤,物土方,议远迩,量事期,计徒庸,虑材用,书餱粮,以令役于诸侯,属役赋丈,书以授帅,而效诸刘子。韩简子临之,以为成命。

十二月,公疾,遍赐大夫,大夫不受。赐子家子双琥,一环,一璧,轻服,受之。大夫皆受其赐。己未,公薨。子家子反赐于府人,曰:``吾不敢逆君命也。」大夫皆反其赐。书曰:``公薨于乾侯。」言失其所也。

赵简子问于史墨曰:``季氏出其君,而民服焉,诸侯与之,君死于外,而莫之或罪也。」对曰:``物生有两,有三,有五,有陪贰。故天有三辰,地有五行,体有左右,各有妃耦。王有公,诸侯有卿,皆有贰也。天生季氏,以贰鲁侯,为日久矣。民之服焉,不亦宜乎?鲁君世从其失,季氏世修其勤,民忘君矣。虽死于外,其谁矜之?社稷无常奉,君臣无常位,自古以然。故《诗》曰:『高岸为谷,深谷为陵。』三后之姓,于今为庶,王所知也。在《易》卦,雷乘《乾》曰《大壮》,天之道也。昔成季友,桓之季也,文姜之爱子也,始震而卜。卜人谒之,曰:『生有嘉闻,其名曰友,为公室辅。』及生,如卜人之言,有文在其手曰『友』,遂以名之。既而有大功于鲁,受费以为上卿。至于文子、武子,世增其业,不废旧绩。鲁文公薨,而东门遂杀适立庶,鲁君于是乎失国,政在季氏,于此君也,四公矣。民不知君,何以得国?是以为君,慎器与名,不可以假人。」

\hypertarget{header-n2872}{%
\subsection{定公}\label{header-n2872}}

\begin{center}\rule{0.5\linewidth}{\linethickness}\end{center}

\hypertarget{header-n2874}{%
\subsubsection{定公元年}\label{header-n2874}}

【经】元年春王三月。晋人执宋仲几于京师。夏六月癸亥,公之丧至自乾侯。戊辰,公即位。秋七月癸巳,葬我君昭公。九月,大雩。立炀宫。冬十月,陨霜杀菽。

【传】元年春,王正月辛巳,晋魏舒合诸侯之大夫于狄泉,将以城成周。魏子莅政。卫彪傒曰:``将建天子,而易位以令,非义也。大事奸义,必有大咎。晋不失诸侯,魏子其不免乎!」是行也,魏献子属役于韩简子及原寿过,而田于大陆,焚焉,还,卒于宁。范献子去其柏椁,以其未覆命而田也。

孟懿子会城成周,庚寅,栽。宋仲几不受功,曰:``滕、薛、郳,吾役也。」薛宰曰:``宋为无道,绝我小国于周,以我适楚,故我常从宋。晋文公为践土之盟,曰:『凡我同盟,各复旧职。』若从践土,若从宋,亦唯命。」仲几曰:``践土固然。」薛宰曰:``薛之皇祖奚仲,居薛以为夏车正。奚仲迁于邳,仲虺居薛,以为汤左相。若复旧职,将承王官,何故以役诸侯?」仲几曰:``三代各异物,薛焉得有旧?为宋役,亦其职也。」士弥牟曰:``晋之从政者新,子姑受功。归,吾视诸故府。」仲几曰:``纵子忘之,山川鬼神其忘诸乎?」士伯怒,谓韩简子曰:``薛征于人,宋征于鬼,宋罪大矣。且己无辞而抑我以神,诬我也。启宠纳侮,其此之谓矣。必以仲几为戮。」乃执仲几以归。三月,归诸京师。

城三旬而毕,乃归诸侯之戌。

齐高张后,不从诸侯。晋女叔宽曰:``周苌弘、齐高张皆将不免。苌叔违天,高子违人。天之所坏,不可支也。众之所为,不可奸也。」

夏,叔孙成子逆公之丧于乾侯。季孙曰:``子家子亟言于我,未尝不中吾志也。吾欲与之从政,子必止之,且听命焉。」子家子不见叔孙,易几而哭。叔孙请见子家子,子家子辞,曰:``羁未得见,而从君以出。君不命而薨,羁不敢见。」叔孙使告之曰:``公衍、公为实使群臣不得事君。若公子宋主社稷,则群臣之愿也。凡从君出而可以入者,将唯子是听。子家氏未有后,季孙愿与子从政,此皆季孙之愿也,使不敢以告。」对曰:``若立君,则有卿士、大夫与守龟在,羁弗敢知。若从君者,则貌而出者,入可也;寇而出者,行可也。若羁也,则君知其出也,而未知其入也,羁将逃也。」

丧及坏隤,公子宋先入,从公者皆自坏隤反。

六月癸亥,公之丧至自乾侯。戊辰,公即位。季孙使役如阚公氏,将沟焉。荣驾鹅曰:``生不能事,死又离之,以自旌也。纵子忍之,后必或耻之。」乃止。季孙问于荣驾鹅曰:``吾欲为君谥,使子孙知之。」对曰:``生弗能事,死又恶之,以自信也。将焉用之?」乃止。

秋七月癸巳,葬昭公于墓道南。孔子之为司寇也,沟而合诸墓。

昭公出,故季平子祷于炀公。九月,立炀宫。

周巩简公弃其子弟,而好用远人。

\hypertarget{header-n2887}{%
\subsubsection{定公二年}\label{header-n2887}}

【经】二年春王正月。夏五月壬辰,雉门及两观灭。秋,楚人伐吴。冬十月,新作雉门及两观。

【传】二年夏四月辛酉,巩氏之群子弟贼简公。

桐叛楚。吴子使舒鸠氏诱楚人,曰:``以师临我,我伐桐,为我使之无忌。」

秋,楚囊瓦伐吴,师于豫章。吴人见舟于豫章,而潜师于巢。冬十月,吴军楚师于豫章,败之。遂围巢,克之,获楚公子繁。

邾庄公与夷射姑饮酒,私出。阍乞肉焉。夺之杖以敲之。

\hypertarget{header-n2895}{%
\subsubsection{定公三年}\label{header-n2895}}

【经】三年春王正月,公如晋,至河,乃复。二月辛卯,邾子穿卒。夏四月。秋,葬邾庄公。冬,仲孙何忌及邾子盟于拔。

【传】三年春二月辛卯,邾子在门台,临廷。阍以瓶水沃廷。邾子望见之,怒。阍曰:``夷射姑旋焉。」命执之,弗得,滋怒。自投于床,废于炉炭,烂,遂卒。先葬以车五乘,殉五人。庄公卞急而好洁,故及是。

秋九月,鲜虞人败晋师于平中,获晋观虎,恃其勇也。

冬,盟于郯,修邾好也。

蔡昭侯为两佩与两裘,以如楚,献一佩一裘于昭王。昭王服之,以享蔡侯。蔡侯亦服其一。子常欲之,弗与,三年止之。唐成公如楚,有两肃爽马,子常欲之,弗与,亦三年止之。唐人或相与谋,请代先从者,许之。饮先从者酒,醉之,窃马而献之子常。子常归唐侯。自拘于司败,曰:``君以弄马之故,隐君身,弃国家,群臣请相夫人以偿马,必如之。」唐侯曰:``寡人之过也,二三子无辱。」皆赏之。蔡人闻之,固请而献佩于子常。子常朝,见蔡侯之徒,命有司曰:``蔡君之久也,官不共也。明日,礼不毕,将死。」蔡侯归,及汉,执玉而沈,曰``余所有济汉而南者,有若大川。」蔡侯如晋,以其子元与其大夫之子为质焉,而请伐楚。

\hypertarget{header-n2903}{%
\subsubsection{定公四年 }\label{header-n2903}}

【经】四年春王二月癸巳,陈侯吴卒。三月,公会刘子、晋侯、宋公、蔡侯、卫侯、陈子、郑伯、许男、曹伯、莒子、邾子、顿子、胡子、滕子、薛伯、杞伯、小邾子、齐国夏于召陵,侵楚。夏四月庚辰,蔡公孙姓帅师灭沈,以沈子嘉归,杀之。五月,公及诸侯盟于皋鼬。杞伯成卒于会。六月,葬陈惠公。许迁于容城。秋七月,至自会。刘卷卒。葬杞悼公。楚人围蔡。晋士鞅、卫孔围帅师伐鲜虞。葬刘文公。冬十有一月庚午,蔡侯以吴子及楚人战于柏举,楚师败绩。楚囊瓦出奔郑。庚辰,吴入郢。

【传】四年春三月,刘文公合诸侯于召陵,谋伐楚也。

晋荀寅求货于蔡侯,弗得。言于范献子曰:``国家方危,诸侯方贰,将以袭敌,不亦难乎!水潦方降,疾疟方起,中山不服,弃盟取怨,无损于楚,而失中山,不如辞蔡侯。吾自方城以来,楚未可以得志,只取勤焉。」乃辞蔡侯。

晋人假羽旄于郑,郑人与之。明日,或旆以会。晋于是乎失诸侯。将会,卫子行敬子言于灵公曰:``会同难,啧有烦言,莫之治也。其使祝佗从!」公曰:``善。」乃使子鱼。子鱼辞,曰:``臣展四体,以率旧职,犹惧不给而烦刑书,若又共二,徼大罪也。且夫祝,社稷之常隶也。社稷不动,祝不出竟,官之制也。君以军行,祓社衅鼓,祝奉以从,于是乎出竟。若嘉好之事,君行师从,卿行旅从,臣无事焉。」公曰:``行也。」及皋鼬,将长蔡于卫。卫侯使祝佗私于苌弘曰:``闻诸道路,不知信否。若闻蔡将先卫,信乎?」苌弘曰:``信。蔡叔,康叔之兄也,先卫,不亦可乎?」子鱼曰:``以先王观之,则尚德也。昔武王克商,成王定之,选建明德,以蕃屏周。故周公相王室,以尹天下,于周为睦。分鲁公以大路,大旂,夏后氏之璜,封父之繁弱,殷民六族,条氏、徐氏、萧氏、索氏、长勺氏、尾勺氏。使帅其宗氏,辑其分族,将其类丑,以法则周公,用即命于周。是使之职事于鲁,以昭周公之明德。分之土田倍敦,祝、宗、卜、史,备物、典策,官司、彝器。因商奄之民,命以《伯禽》,而封于少皞之虚。分康叔以大路、少帛、綪茷、旃旌、大吕,殷民七族,陶氏、施氏、繁氏、錡氏、樊氏、饥氏、终葵氏;封畛土略,自武父以南,及圃田之北竟,取于有阎之土,以共王职。取于相土之东都,以会王之东蒐。聃季授土,陶叔授民,命以《康诰》,而封于殷虚。皆启以商政,疆以周索。分唐叔以大路,密须之鼓,阙巩,沽洗,怀姓九宗,职官五正。命以《唐诰》,而封于夏虚,启以夏政,疆以戎索。三者皆叔也,而有令德,故昭之以分物。不然,文、武、成康、之伯犹多,而不获是分也,唯不尚年也。管蔡启商,惎间王室。王于是乎杀管叔而蔡蔡叔,以车七乘,徒七十人。其子蔡仲,改行帅德,周公举之,以为己卿士。见诸王而命之以蔡,其命书云:『王曰:胡!无若尔考之违王命也。』若之何其使蔡先卫也?武王之母弟八人,周公为大宰,康叔为司寇,聃季为司空,五叔无官,岂尚年哉!曹,文之昭也;晋,武之穆也。曹为伯甸,非尚年也。今将尚之,是反先王也。晋文公为践土之盟,卫成公不在,夷叔,其母弟也,犹先蔡。其载书云:『王若曰,晋重、鲁申、卫武、蔡甲午、郑捷、齐潘、宋王臣、莒期。』藏在周府,可覆视也。吾子欲覆文、武之略,而不正其德,将如之何?」苌弘说,告刘子,与范献子谋之,乃长卫侯于盟。

反自召陵,郑子大叔未至而卒。晋赵简子为之临,甚哀,曰:``黄父之会,夫子语我九言,曰:『无始乱,无怙富,无恃宠,无违同,无敖礼,无骄能,无复怒,无谋非德,无犯非义。』」

沈人不会于召陵,晋人使蔡伐之。夏,蔡灭沈。

秋,楚为沈故,围蔡。伍员为吴行人以谋楚。楚之杀郤宛也,伯氏之族出。伯州犁之孙嚭为吴大宰以谋楚。楚自昭王即位,无岁不有吴师。蔡侯因之,以其子乾与其大夫之子为质于吴。

冬,蔡侯、吴子、唐侯伐楚。舍舟于淮汭,自豫章与楚夹汉。左司马戌谓子常曰:``子水公汉而与之上下,我悉方城外以毁其舟,还塞大隧、直辕、冥厄,子济汉而伐之,我自后击之,必大败之。」既谋而行。武城黑谓子常曰:``吴用木也,我用革也,不可久也。不如速战。」史皇谓子常:``楚人恶而好司马,若司马毁吴舟于淮,塞城口而入,是独克吴也。子必速战,不然不免。」乃济汉而陈,自小别至于大别。三战,子常知不可,欲奔。史皇曰:``安求其事,难而逃之,将何所入?子必死之,初罪必尽说。」

十一月庚午,二师陈于柏举。阖庐之弟夫概王,晨请于阖庐曰:``楚瓦不仁,其臣莫有死志,先伐之,其卒必奔。而后大师继之,必克。」弗许。夫概王曰:``所谓『臣义而行,不待命』者,其此之谓也。今日我死,楚可入也。」以其属五千,先击子常之卒。子常之卒奔,楚师乱,吴师大败之。子常奔郑。史皇以其乘广死。吴从楚师,及清发,将击之。夫□王曰:``困兽犹斗,况人乎?若知不免而致死,必败我。若使先济者知免,后者慕之,蔑有斗心矣。半济而后可击也。」从之。又败之。楚人为食,吴人及之,奔。食而从之,败诸雍澨五战及郢。

己卯,楚子取其妹季芈畀我以出,涉睢。针尹固与王同舟,王使执燧象以奔吴师。

庚辰,吴入郢,以班处宫。子山处令尹之宫,夫概王欲攻之,惧而去之,夫□王入之。

左司马戌及息而还,败吴师于雍澨,伤。初,司马臣阖庐,故耻为禽焉。谓其臣曰:``谁能免吾首?」吴句卑曰:``臣贱可乎?」司马曰:``我实失子,可哉!」三战皆伤,曰:``吾不用也已。」句卑布裳,刭而裹之,藏其身而以其首免。楚子涉雎,济江,入于云中。王寝,盗攻之,以戈击王。王孙由于以背受之。中肩。王奔郧,钟建负季芈以从,由于徐苏而从。郧公辛之弟怀将弑王,曰:``平王杀吾父,我杀其子,不亦可乎?」辛曰:``君讨臣,谁敢仇之?君命,天也,若死天命,将谁仇?《诗》曰:『柔亦不茹,刚亦不吐,不侮矜寡,不畏强御。』唯仁者能之。违强陵弱,非勇也。乘人之约,非仁也。灭宗废祀,非孝也。动无令名,非知也。必犯是,余将杀女。」斗辛与其弟巢以王奔随。吴人从之,谓随人曰:``周之子孙在汉川者,楚实尽之。天诱其衷,致罚于楚,而君又窜之。周室何罪?君若顾报周室,施及寡人,以奖天衷,君之惠也。汉阳之田,君实有之。」楚子在公宫之北,吴人在其南。子期似王,逃王,而己为王,曰:``以我与之,王必免。」随人卜与之,不吉。乃辞吴曰:``以随之辟小而密迩于楚,楚实存之,世有盟誓,至于今未改。若难而弃之,何以事君?执事之患,不唯一人。若鸠楚竟,敢不听命。」吴人乃退。鑢金初官于子期氏,实与随人要言。王使见,辞,曰:``不敢以约为利。」王割子期之心,以与随人盟。

初,伍员与申包胥友。其亡也,谓申包胥曰:``我必复楚国。」申包胥曰:``勉之!子能复之,我必能兴之。」及昭王在随,申包胥如秦乞师,曰:``吴为封豕、长蛇,以荐食上国,虐始于楚。寡君失守社稷,越在草莽。使下臣告急,曰:『夷德无厌,若邻于君,疆埸之患也。逮吴之未定,君其取分焉。若楚之遂亡,君之土也。若以君灵抚之,世以事君。』」秦伯使辞焉,曰:``寡人闻命矣。子姑就馆,将图而告。」对曰:``寡君越在草莽,未获所伏。下臣何敢即安?」立,依于庭墙而哭,日夜不绝声,勺饮不入口七日。秦哀公为之赋《无衣》,九顿首而坐,秦师乃出。

\hypertarget{header-n2919}{%
\subsubsection{定公五年}\label{header-n2919}}

【经】五年春王三月辛亥朔,日有食之。夏,归粟于蔡。于越入吴。六月丙申,季孙意如卒。秋七月壬子,叔孙不敢卒。冬,晋士鞅帅师围鲜虞。

【传】五年春,王人杀子朝于楚。

夏,归粟于蔡,以周亟,矜无资。

越入吴,吴在楚也。

六月,季平子行东野,还,未至,丙申,卒于房。阳虎将以与璠敛,仲梁怀弗与,曰:``改步改玉。」阳虎欲逐之,告公山不狃。不狃曰:``彼为君也,子何怨焉?」既葬,桓子行东野,及费。子泄为费宰,逆劳于郊,桓子敬之。劳仲梁怀,仲梁怀弗敬。子泄怒,谓阳虎:``子行之乎?」

申包胥以秦师至,秦子蒲、子虎帅车五百乘以救楚。子蒲曰:``吾未知吴道。」使楚人先与吴人战,而自稷会之,大败夫□王于沂。吴人获薳射于柏举,其子帅奔徒以从子西,败吴师于军祥。秋七月,子期、子蒲灭唐。

九月,夫□王归,自立也。以与王战而败,奔楚,为堂溪氏。吴师败楚师于雍澨,秦师又败吴师。吴师居麇,子期将焚之,子西曰:``父兄亲暴骨焉,不能收,又焚之,不可。」子期曰:``国亡矣!死者若有知也,可以歆旧祀,岂惮焚之?」焚之,而又战,吴师败。又战于公婿之溪,吴师大败,吴子乃归。囚闉舆罢,闉舆罢请先,遂逃归。叶公诸梁之弟后臧从其母于吴,不待而归。叶公终不正视。

乙亥,阳虎囚季桓子及公父文伯,而逐仲梁怀。冬十月丁亥,杀公何藐。己丑,盟桓子于稷门之内。庚寅,大诅,逐公父歜及秦遄,皆奔齐。

楚子入于郢。初,斗辛闻吴人之争宫也,曰:``吾闻之:『不让则不和,不和不可以远征。』吴争于楚,必有乱。有乱则必归,焉能定楚?」王之奔随也,将涉于成臼,蓝尹亹涉其帑,不与王舟。及宁,王欲杀之。子西曰:``子常唯思旧怨以败,君何效焉?」王曰:``善。使复其所,吾以志前恶。」王赏斗辛、王孙由于、王孙圉、钟建、斗巢、申包胥、王孙贾、宋木、斗怀。子西曰:``请舍怀也。」王曰:``大德灭小怨,道也。」申包胥曰:``吾为君也,非为身也。君既定矣,又何求?且吾尤子旗,其又为诸?」遂逃赏。王将嫁季芈,季芈辞曰:``所以为女子,远丈夫也。钟建负我矣。」以妻钟建,以为乐尹。

王之在随也,子西为王舆服以保路,国于脾泄。闻王所在,而后从王。王使由于城麇,覆命,子西问高厚焉,弗知。子西曰:``不能,如辞。城不知高厚,小大何知?」对曰:``固辞不能,子使余也。人各有能有不能。王遇盗于云中,余受其戈,其所犹在。」袒而示之背,曰:``此余所能也。脾泄之事,余亦弗能也。」

晋士鞅围鲜虞,报观虎之役也。

\hypertarget{header-n2933}{%
\subsubsection{定公六年}\label{header-n2933}}

【经】六年春王正月癸亥,郑游速帅师灭许,以许男斯归。二月,公侵郑。公至自侵郑。夏,季孙斯、仲孙何忌如晋。秋,晋人执宋行人乐祁犁。冬,城中城。季孙斯、仲孙忌帅师围郓。

【传】六年春,郑灭许,因楚败也。

二月,公侵郑,取匡,为晋讨郑之伐胥靡也。往不假道于卫;及还,阳虎使季、孟自南门入,出自东门,舍于豚泽。卫侯怒,使弥子瑕追之。公叔文子老矣,辇而如公,曰:``尤人而效之,非礼也。昭公之难,君将以文之舒鼎,成之昭兆,定之鞶鉴,苟可以纳之,择用一焉。公子与二三臣之子,诸侯苟忧之,将以为之质。此群臣之所闻也。今将以小忿蒙旧德,无乃不可乎!大姒之子,唯周公、康叔为相睦也。而效小人以弃之,不亦诬乎!天将多阳虎之罪以毙之,君姑待之,若何?」乃止。

夏,季桓子如晋,献郑俘也。阳虎强使孟懿子往报夫人之币。晋人兼享之。孟孙立于房外,谓范献子曰:``阳虎若不能居鲁,而息肩于晋,所不以为中军司马者,有如先君!」献子曰:``寡君有官,将使其人。鞅何知焉?」献子谓简子曰:``鲁人患阳虎矣,孟孙知其衅,以为必适晋,故强为之请,以取入焉。」

四月己丑,吴大子终累败楚舟师,获潘子臣、小惟子及大夫七人。楚国大惕,惧亡。子期又以陵师败于繁扬。令尹子西喜曰:``乃今可为矣。」于是乎迁郢于郤,而改纪其政,以定楚国。

周儋翩率王子朝之徒,因郑人将以作乱于周。郑于是乎伐冯、滑、胥靡、负黍、狐人、阙外。六月,晋阎没戍周,且城胥靡。

秋八月,宋乐祁言于景公曰:``诸侯唯我事晋,今使不往,晋其憾矣。」乐祁告其宰陈寅。陈寅曰:``必使子往。」他日,公谓乐祁曰:``唯寡人说子之言,子必往。」陈寅曰:``子立后而行,吾室亦不亡,唯君亦以我为知难而行也。」见溷而行。赵简子逆,而饮之酒于绵上,献杨楯六十于简子。陈寅曰:``昔吾主范氏,今子主赵氏,又有纳焉。以杨楯贾祸,弗可为也已。然子死晋国,子孙必得志于宋。」范献子言于晋侯曰:``以君命越疆而使,未致使而私饮酒,不敬二君,不可不讨也。」乃执乐祁。

阳虎又盟公及三桓于周社,盟国人于亳社,诅于五父之衢。

冬,十二月,天王处于姑莸,辟儋翩之乱也。

\hypertarget{header-n2945}{%
\subsubsection{·定公七年}\label{header-n2945}}

【经】七年春王正月。夏四月。秋,齐侯、郑伯盟于咸。齐人执卫行人北宫结以侵卫。齐侯、卫侯盟于沙。大雩。齐国夏帅师伐我西鄙。九月,大雩。冬十月。

【传】七年春二月,周儋翩入于仪栗以叛。

齐人归郓、阳关,阳虎居之以为政。

夏四月,单武公、刘桓公败尹氏于穷谷。

秋,齐侯、郑伯盟于咸,征会于卫。卫侯欲叛晋,诸大夫不可。使北宫结如齐,而私于齐侯曰:``执结以侵我。」齐侯从之,乃盟于琐。

齐国夏伐我。阳虎御季桓子,公敛处父御孟懿子,将宵军齐师。齐师闻之,堕,伏而待之。处父曰:``虎不图祸,而必死。」苫夷曰:``虎陷二子于难,不待有司,余必杀女。」虎惧,乃还,不败。

冬十一月戊午,单子、刘子逆王于庆氏。晋籍秦送王。己巳,王入于王城,馆于公族党氏,而后朝于庄宫。

\hypertarget{header-n2955}{%
\subsubsection{定公八年}\label{header-n2955}}

【经】八年春王正月,公侵齐。公至自侵齐。二月,公侵齐。三月,公至自侵齐。曹伯露卒。夏,齐国夏帅师伐我西鄙。公会晋师于瓦。公至自瓦。秋七月戊辰,陈侯柳卒。晋士鞅帅师侵郑,遂侵卫。葬曹靖公。九月,葬陈怀公。季孙斯、仲孙何忌帅师侵卫。冬,卫侯、郑伯盟于曲濮。从祀先公。盗窃宝玉、大弓。

【传】八年春,王正月,公侵齐,门于阳州。士皆坐列,曰:``颜高之弓六钧。」皆取而传观之。阳州人出,颜高夺人弱弓,籍丘子锄击之,与一人俱毙。偃,且射子锄,中颊,殪。颜息射人中眉,退曰:``我无勇,吾志其目也。」师退,冉猛伪伤足而先。其兄会乃呼曰:``猛也殿!」

二月己丑,单子伐谷城,刘子伐仪栗。辛卯,单子伐简城,刘子伐盂,以定王室。

赵鞅言于晋侯曰:``诸侯唯宋事晋,好逆其使,犹惧不至。今又执之,是绝诸侯也。」将归乐祁。士鞅曰:``三年止之,无故而归之,宋必,叛晋。``献子私谓子梁曰:``寡君惧不得事宋君,是以止子。子姑使溷代子。」子梁以告陈寅,陈寅曰:``宋将叛晋是弃溷也,不如侍之。」乐祁归,卒于大行。士鞅曰:``宋必叛,不如止其尸以求成焉。」乃止诸州。

公侵齐,攻廪丘之郛。主人焚冲,或濡马褐以救之,遂毁之。主人出,师奔。阳虎伪不见冉猛者,曰:``猛在此,必败。」猛逐之,顾而无继,伪颠。虎曰:``尽客气也。」苫越生子,将待事而名之。阳州之役获焉,名之曰阳州。

夏,齐国夏、高张伐我西鄙。晋士鞅、赵鞅、荀寅救我。公会晋师于瓦。范献子执羔,赵简子、中行文子皆执雁。鲁于是始尚羔。

晋师将盟卫侯于鄟泽。赵简子曰:``群臣谁敢盟卫君者?」涉佗、成何曰:``我能盟之。」卫人请执牛耳。成何曰:``卫,吾温、原也,焉得视诸侯?」将歃,涉佗捘卫侯之手,及捥。卫侯怒,王孙贾趋进,曰:``盟以信礼也。有如卫君,其敢不唯礼是事,而受此盟也。」

卫侯欲叛晋,而患诸大夫。王孙贾使次于郊,大夫问故。公以晋诟语之,且曰:``寡人辱社稷,其改卜嗣,寡人从焉。」大夫曰:``是卫之祸,岂君之过也?」公曰:``又有患焉。谓寡人『必以而子与大夫之子为质。』」大夫曰:``苟有益也,公子则往。群臣之子,敢不皆负羁绁以从?」将行。王孙贾曰:``苟卫国有难,工商未尝不为患,使皆行而后可。」公以告大夫,乃皆将行之。行有日,公朝国人,使贾问焉,曰:``若卫叛晋,晋五伐我,病何如矣?」皆曰:``五伐我,犹可以能战。」贾曰:``然则如叛之,病而后质焉,何迟之有?」乃叛晋。晋人请改盟,弗许。

秋,晋士鞅会成桓公,侵郑,围虫牢,报伊阙也。遂侵卫。

九月,师侵卫,晋故也。

季寤、公锄极、公山不狃皆不得志于季氏,叔孙辄无宠于叔孙氏,叔仲志不得志于鲁。故五人因阳虎。阳虎欲去三桓,以季寤更季氏,以叔孙辄更叔孙氏,己更孟氏。冬十月,顺祀先公而祈焉。辛卯,禘于僖公。壬辰,将享季氏于蒲圃而杀之,戒都车曰:``癸巳至。」成宰公敛处父告孟孙,曰:``季氏戒都车,何故?」孟孙曰:``吾弗闻。」处父曰:``然则乱也,必及于子,先备诸?」与孟孙以壬辰为期。

阳虎前驱,林楚御桓子,虞人以铍盾夹之,阳越殿,将如蒲圃。桓子咋谓林楚曰:``而先皆季氏之良也,尔以是继之。」对曰:``臣闻命后。阳虎为政,鲁国服焉。违之,征死。死无益于主。」桓子曰:``何后之有?而能以我适孟氏乎?」对曰:``不敢爱死,惧不免主。」桓子曰:``往也。」孟氏选圉人之壮者三百人,以为公期筑室于门外。林楚怒马及衢而骋,阳越射之,不中,筑者阖门。有自门间射阳越,杀之。阳虎劫公与武叔,以伐孟氏。公敛处父帅成人,自上东门入,与阳氏战于南门之内,弗胜。又战于棘下,阳氏败。阳虎说甲如公宫,取宝玉、大弓以出,舍于五父之衢,寝而为食。其徒曰:``追其将至。」虎曰:``鲁人闻余出,喜于征死,何暇追余?」从者曰:」嘻!速驾!公敛阳在。」公敛阳请追之,孟孙弗许。阳欲杀桓子,孟孙惧而归之。子言辨舍爵于季氏之庙而出。阳虎入于欢、阳关以叛。

郑驷歂嗣子大叔为政。

\hypertarget{header-n2971}{%
\subsubsection{定公九年}\label{header-n2971}}

【经】九年春王正月。夏四月戊申,郑伯虿卒。得宝玉、大弓。六月,葬郑献公。秋,齐侯、卫侯次于五氏。秦伯卒。冬,葬秦哀公。

【传】九年春,宋公使乐大心盟于晋,且逆乐祁之尸。辞,伪有疾。乃使向巢如晋盟,且逆子梁之尸。子明谓桐门右师出,曰:``吾犹衰絰,而子击钟,何也?」右师曰:``丧不在此故也。」既而告人曰:``己衰絰而生子,余何故舍钟?」子明闻之,怒,言于公曰:``右师将不利戴氏,不肯适晋,将作乱也。不然无疾。」乃逐桐门右师。

郑驷歂杀邓析,而用其《竹刑》。君子谓子然:``于是不忠。苟有可以加于国家者,弃其邪可也。《静女》之三章,取彤管焉。《竿旄》『何以告之』,取其忠也。故用其道,不弃其人。《诗》云:『蔽芾甘棠,勿翦勿伐、召伯所茇。』思其人犹爱其树,况用其道而不恤其人乎?子然无以劝能矣。」

夏,阳虎归宝玉、大弓。书曰``得」,器用也。凡获器用曰得,得用焉曰获。

六月,伐阳关。阳虎使焚莱门。师惊,犯之而出,奔齐,请师以伐鲁,曰:``三加必取之。」齐侯将许之。鲍文子谏曰:``臣尝为隶于施氏矣,鲁未可取也。上下犹和,众庶犹睦,能事大国,而无天灾,若之何取之?阳虎欲勤齐师也,齐师罢,大臣必多死亡,己于是乎奋其诈谋。夫阳虎有宠于季氏,而将杀季孙,以不利鲁国,而求容焉。亲富不亲仁,君焉用之?君富于季氏,而大于鲁国,兹阳虎所欲倾覆也。鲁免其疾,而君又收之,无乃害乎!」齐侯执阳虎,将东之。阳虎愿东,乃囚诸西鄙。尽借邑人之车,锲其轴,麻约而归之。载葱灵,寝于其中而逃。追而得之,囚于齐。又以葱灵逃,奔晋,适赵氏。仲尼曰:``赵氏其世有乱乎!」

秋,齐侯伐晋夷仪。敝无存之父将室之,辞,以与其弟,曰:``此役也不死,反,必娶于高、国。」先登,求自门出,死于溜下。东郭书让登,犁弥从之,曰:``子让而左,我让而右,使登者绝而后下。」书左,弥先下。书与王猛息。猛曰:``我先登。」书敛甲,曰:``曩者之难,今又难焉!」猛笑曰:``吾从子如骖之靳。」

晋车千乘在中牟。卫侯将如五氏,卜过之,龟焦。卫侯曰:``可也。卫车当其半,寡人当其半,敌矣。」乃过中牟。中牟人欲伐之,卫褚师圃亡在中牟,曰:``卫虽小,其君在焉,未可胜也。齐师克城而骄,其帅又贱,遇,必败之。不如从齐。」乃伐齐师,败之。齐侯致禚、媚、杏于卫。齐侯赏犁弥,犁弥辞,曰:``有先登者,臣从之,皙帻而衣狸制。」公使视东郭书,曰:``乃夫子也,吾贶子。」公赏东郭书,辞,曰:``彼,宾旅也。」乃赏犁弥。

齐师之在夷仪也,齐侯谓夷仪人曰:``得敝无存者,以五家免。」乃得其尸。公三襚之。与之犀轩与直盖,而先归之。坐引者,以师哭之,亲推之三。

\hypertarget{header-n2982}{%
\subsubsection{定公十年}\label{header-n2982}}

【经】十年春王三月,乃齐平。夏,公会齐侯于夹谷。公至自夹谷。晋赵鞅帅师围卫。齐人来归郓、欢、龟阴田。叔孙州仇、仲孙何忌帅师围郈。秋,叔孙州仇、仲孙何忌帅师围郈。宋乐大心出奔曹。宋公子地出奔陈。冬,齐侯、卫侯、郑游速会于安甫。叔孙州仇如齐。宋公之弟辰暨仲佗、石彄出奔陈。

【传】十年春,及齐平。

夏,公会齐侯于祝其,实夹谷。孔丘相。犁弥言于齐侯曰:``孔丘知礼而无勇,若使莱人以兵劫鲁侯,必得志焉。」齐侯从之。孔丘以公退,曰:``士,兵之!两君合好,而裔夷之俘以兵乱之,非齐君所以命诸侯也。裔不谋夏,夷不乱华,俘不干盟,兵不逼好。于神为不祥,于德为愆义,于人为失礼,君必不然。」齐侯闻之,遽辟之。

将盟,齐人加于载书曰:``齐师出竟,而不以甲车三百乘从我者,有如此盟。」孔丘使兹无还揖对曰:``而不反我汶阳之田,吾以共命者,亦如之。」齐侯将享公,孔丘谓梁丘据曰:``齐、鲁之故,吾子何不闻焉?事既成矣,而又享之,是勤执事也。且牺象不出门,嘉乐不野合。飨而既具,是弃礼也。若其不具,用秕稗也。用秕稗,君辱,弃礼,名恶,子盍图之?夫享,所以昭德也。不昭,不如其已也。」乃不果享。

齐人来归郓、欢、龟阴之田。

晋赵鞅围卫,报夷仪也。

初,卫侯伐邯郸午于寒氏,城其西北而守之,宵熸。及晋围卫,午以徒七十人门于卫西门,杀人于门中,曰:``请报寒氏之役。」涉佗曰:``夫子则勇矣,然我往,必不敢启门。」亦以徒七十人,旦门焉,步左右,皆至而立,如植。日中不启门,乃退。反役,晋人讨卫之叛故,曰:``由涉佗、成何。」于是执涉佗以求成于卫。卫人不许,晋人遂杀涉佗。成何奔燕。君子曰:``此之谓弃礼,必不钧。《诗》曰:『人而无礼,胡不遄死。』涉佗亦遄矣哉!」

初,叔孙成子欲立武叔,公若藐固谏曰:``不可。」成子立之而卒。公南使贼射之,不能杀。公南为马正,使公若为郈宰。武叔既定,使郈马正侯犯杀公若,不能。其圉人曰:``吾以剑过朝,公若必曰:『谁也剑也?』吾称子以告,必观之。吾伪固,而授之末,则可杀也。」使如之,公若曰:``尔欲吴王我乎?」遂杀公若。侯犯以郈叛,武叔懿子围郈,弗克。

秋,二子及齐师复围郈,弗克。叔孙谓郈工师驷赤曰:``郈非唯叔孙氏之忧,社稷之患也。将若之何?」对曰:``臣之业,在《扬水》卒章之四言矣。」叔孙稽首。驷赤谓侯犯曰:``居齐、鲁之际,而无事,必不可矣。子盍求事于齐以临民?不然,将叛。」侯犯从之。齐使至,驷赤与郈人为之宣言于郈中曰:``侯犯将以郈易于齐,齐人将迁郈民。」众凶惧。驷赤谓侯犯曰:``众言异矣。子不如易于齐,与其死也。犹是郈也,而得纾焉,何必此?齐人欲以此逼鲁,必倍与子地。且盍多舍甲于子之门,以备不虞?」侯犯曰:``诺。」乃多舍甲焉。侯犯请易于齐,齐有司观郈,将至。驷赤使周走呼曰:``齐师至矣!」郈人大骇,介侯犯之门甲,以围侯犯。驷赤将射之。侯犯止之,曰:``谋免我。」侯犯请行,许之。驷赤先如宿,侯犯殿。每出一门,郈人闭之。及郭门,止之,曰:``子以叔孙氏之甲出,有司若诛之,群臣惧死。」驷赤曰:``叔孙氏之甲有物,吾未敢以出。」犯谓驷赤曰:``子止而与之数。」驷赤止,而纳鲁人。侯犯奔齐,齐人乃致郈。

宋公子地嬖蘧富猎,十一分其室,而以其五与之。公子地有白马四。公嬖向魋,魋欲之,公取而朱其尾鬣以与之。地怒,使其徒扶魋而夺之。魋惧,将走。公闭门而泣之,目尽肿。母弟辰曰:``子分室以与猎也,而独卑魋,亦有颇焉。子为君礼,不过出竟,君必止子。」公子地奔陈,公弗止。辰为之请,弗听。辰曰:``是我迋吾兄也。吾以国人出,君谁与处?」冬,母弟辰暨仲佗、石彄出奔陈。

武叔聘于齐,齐侯享之,曰:``子叔孙!若使郈在君之他竟,寡人何知焉?属与敝邑际,故敢助君忧之。」对曰:``非寡君之望也。所以事君,封疆社稷是以。敢以家隶勤君之执事?夫不令之臣,天下之所恶也。君岂以为寡君赐?」

\hypertarget{header-n2996}{%
\subsubsection{定公十一年}\label{header-n2996}}

【经】十有一年春,宋公之弟辰及仲佗、石彄、公子地自陈入于萧以叛。夏四月。秋,宋乐大心自曹入于萧。冬,及郑平。叔还如郑莅盟。

【传】十一年春,宋公母弟辰暨仲佗、石彄、公子地入于萧以叛。秋,乐大心从之,大为宋患,宠向魋故也。

冬,及郑平,始叛晋也。

\hypertarget{header-n3002}{%
\subsubsection{定公十二年}\label{header-n3002}}

【经】十有二年春,薛伯定卒。夏,葬薛襄公。叔孙州仇帅师堕郈。卫公孟彄帅师伐曹。季孙斯、仲孙何忌帅师堕费。秋,大雩。冬十月癸亥,公会齐侯盟于黄。十有一月丙寅朔,日有食之。公至自黄。十有二月,公围成。公至自侯成。

【传】十二年夏,卫公孟彄伐曹,克郊。还,滑罗殿。未出,不退于列。其御曰:``殿而在列,其为无勇乎?」罗曰:``与其素厉,宁为无勇。」

仲由为季氏宰,将堕三都,于是叔孙氏堕郈。季氏将堕费,公山不狃、叔孙辄帅费人以袭鲁。公与三子入于季氏之宫,登武子之台。费人攻之,弗克。入及公侧。仲尼命申句须、乐颀下,伐之,费人北。国人追之,败诸姑蔑。二子奔齐,遂堕费。将堕成,公敛处父谓孟孙:``堕成,齐人必至于北门。且成,孟氏之保障也,无成,是无孟氏也。子伪不知,我将不堕。」

冬十二月,公围成,弗克。

\hypertarget{header-n3009}{%
\subsubsection{定公十三年 }\label{header-n3009}}

【经】十有三年春,齐侯、卫侯次于垂葭。夏,筑蛇渊囿。大蒐于比蒲。卫公孟彄帅师伐曹。晋赵鞅入于晋阳以叛。冬,晋荀寅、士吉射入于朝歌以叛。晋赵鞅归于晋。薛弑其君比。

【传】十三年春,齐侯、卫侯次于垂葭,实狊阜氏。使师伐晋,将济河。诸大夫皆曰:``不可。」邴意兹曰:``可。锐师伐河内,传必数日而后及绛。绛不三月,不能出河,则我既济水矣。」乃伐河内。齐侯皆敛诸大夫之轩,唯邴意兹乘轩。齐侯欲与卫侯乘,与之宴,而驾乘广,载甲焉。使告曰:``晋师至矣!」齐侯曰:``比君之驾也,寡人请摄。」乃介而与之乘,驱之。或告曰:``无晋师。」乃止。

晋赵鞅谓邯郸午曰:``归我卫贡五百家,吾舍诸晋阳。」午许诺。归,告其父兄,父兄皆曰:``不可。卫是以为邯郸,而置诸晋阳,绝卫之道也。不如侵齐而谋之。」乃如之,而归之于晋阳。赵孟怒,召午,而囚诸晋阳。使其从者说剑而入,涉宾不可。乃使告邯郸人曰:``吾私有讨于午也,二三子唯所欲立。」遂杀午。赵稷、涉宾以邯郸叛。夏六月,上军司马籍秦围邯郸。邯郸午,荀寅之甥也;荀寅,范吉射之姻也,而相与睦。故不与围邯郸,将作乱。董安于闻之,告赵孟,曰:``先备诸?」赵孟曰:``晋国有命,始祸者死,为后可也。」安于曰:``与其害于民,宁我独死,请以我说。」赵孟不可。秋七月,范氏、中行氏伐赵氏之宫,赵鞅奔晋阳。晋人围之。范皋夷无宠于范吉射,而欲为乱于范氏。梁婴父嬖于知文子,文子欲以为卿。韩简子与中行文子相恶,魏襄子亦与范昭子相恶。故五子谋,将逐荀寅而以梁婴父代之,逐范吉射而以范皋夷代之。荀跞言于晋侯曰:``君命大臣,始祸者死,载书在河。今三臣始祸,而独逐鞅,刑已不钧矣。请皆逐之。」

冬十一月,荀跞、韩不信、魏曼多奉公以伐范氏、中行氏,弗克。二子将伐公,齐高强曰:``三折肱知为良医。唯伐君为不可,民弗与也。我以伐君在此矣。三家未睦,可尽克也。克之,君将谁与?若先伐君,是使睦也。」弗听,遂伐公。国人助公,二子败,从而伐之。丁未,荀寅、士吉射奔朝歌。

韩、魏以赵氏为请。十二月辛未,赵鞅入于绛,盟于公宫。

初,卫公叔文子朝而请享灵公。退,见史鳅而告之。史鳅曰:``子必祸矣。子富而君贪,其及子乎!」文子曰:``然。吾不先告子,是吾罪也。君既许我矣,其若之何?」史鳅曰:``无害。子臣,可以免。富而能臣,必免于难,上下同之。戍也骄,其亡乎。富而不骄者鲜,吾唯子之见。骄而不亡者,未之有也。戍必与焉。」及文子卒,卫侯始恶于公叔戍,以其富也。公叔戍又将去夫人之党,夫人诉之曰:``戍将为乱。」

\hypertarget{header-n3018}{%
\subsubsection{定公十四年}\label{header-n3018}}

【经】十有四年春,卫公叔戍来奔。卫赵阳出奔宋。二月辛巳,楚公子结、陈公孙佗人帅师灭顿,以顿子牂归。夏,卫北宫结来奔。五月,于越败吴于檇李。吴子光卒。公会齐侯、卫侯于牵。公至自会。秋,齐侯、宋公会于洮。天王使石尚来归脤。卫世子蒯瞶出奔宋。卫公孟彄出奔郑。宋公之弟辰自萧来奔。大蒐于比蒲。邾子来会公。城莒父及霄。

【传】十四年春,卫侯逐公叔戍与其党,故赵阳奔宋,戍来奔。

梁婴父恶董安于,谓知文子曰:``不杀安于,使终为政于赵氏,赵氏必得晋国。盍以其先发难也,讨于赵氏?」文子使告于赵孟曰:``范、中行氏虽信为乱,安于则发之,是安于与谋乱也。晋国有命,始祸者死。二子既伏其罪矣,敢以告。」赵孟患之。安于曰:``我死而晋国宁,赵氏定,将焉用生?人谁不死,吾死莫矣。」乃缢而死。赵孟尸诸市,而告于知氏曰:``主命戮罪人,安于既伏其罪矣,敢以告。」知伯从赵孟盟,而后赵氏定,祀安于于庙。

顿子牂欲事晋,背楚而绝陈好。二月,楚灭顿。

夏,卫北宫结来奔,公叔戍之故也。

吴伐越。越子句践御之,陈于檇李。句践患吴之整也,使死士再禽焉,不动。使罪人三行,属剑于颈,而辞曰:``二君有治,臣奸旗鼓,不敏于君之行前,不敢逃刑,敢归死。」遂自刭也。师属之目,越子因而伐之,大败之。灵姑浮以戈击阖庐,阖庐伤将指,取其一屦。还,卒于陉,去檇李七里。夫差使人立于庭,苟出入,必谓己曰:``夫差!而忘越王之杀而父乎?」则对曰:``唯,不敢忘!」三年,乃报越。

晋人围朝歌,公会齐侯、卫侯于脾、上梁之间,谋救范、中行氏。析成鲋、小王桃甲率狄师以袭晋,战于绛中,不克而还。士鲋奔周,小王桃甲入于朝歌。秋,齐侯、宋公会于洮,范氏故也。

卫侯为夫人南子召宋朝,会于洮。大子蒯聩献盂于齐,过宋野。野人歌之曰:``既定尔娄猪,盍归吾艾豭。」大子羞之,谓戏阳速曰:``从我而朝少君,少君见我,我顾,乃杀之。」速曰:``诺。」乃朝夫人。夫人见大子,大子三顾,速不进。夫人见其色,啼而走,曰:``蒯聩将杀余。」公执其手以登台。大子奔宋,尽逐其党。故公孟彄出奔郑,自郑奔齐。

大子告人曰:``戏阳速祸余。」戏阳速告人曰:``大子则祸余。大子无道,使余杀其母。余不许,将戕于余;若杀夫人,将以余说。余是故许而弗为,以纾余死。谚曰:『民保于信。』吾以信义也。」

冬十二月,晋人败范、中行氏之师于潞,获籍秦、高强。又败郑师及范氏之师于百泉。

\hypertarget{header-n3031}{%
\subsubsection{定公十五年}\label{header-n3031}}

【经】十有五年春王正月,邾子来朝。鼷鼠食郊牛,牛死,改卜牛。二月辛丑,楚子灭胡,以胡子豹归。夏五辛亥,郊。壬申,公薨于高寝。郑罕达帅师伐宋。齐侯、卫侯次于渠蒢。邾子来奔丧。秋七月壬申,姒氏卒。八月庚辰朔,日有食之。九月,滕子来会葬。丁巳,葬我君定公,雨,不克葬。戊午,日下昊,乃克葬。辛巳,葬定姒。冬,城漆。

【传】十五年春,邾隐公来朝。子贡观焉。邾子执玉高,其容仰。公受玉卑,其容俯。子贡曰:``以礼观之,二君者,皆有死亡焉。夫礼,死生存亡之体也。将左右周旋,进退俯仰,于是乎取之;朝祀丧戎,于是乎观之。今正月相朝,而皆不度,心已亡矣。嘉事不体,何以能久?高仰,骄也,卑俯,替也。骄近乱,替近疾。君为主,其先亡乎!」

吴之入楚也,胡子尽俘楚邑之近胡者。楚既定,胡子豹又不事楚,曰:``存亡有命,事楚何为?多取费焉。」二月,楚灭胡。

夏五月壬申,公薨。仲尼曰:``赐不幸言而中,是使赐多言者也。」

郑罕达败宋师于老丘。

齐侯、卫侯次于蘧挐,谋救宋也。

秋七月壬申,姒氏卒。不称夫人,不赴,且不祔也。

葬定公。雨,不克襄事,礼也。

葬定姒。不称小君,不成丧也。

冬,城漆。书,不时告也。

\hypertarget{header-n3043}{%
\subsection{哀公}\label{header-n3043}}

\begin{center}\rule{0.5\linewidth}{\linethickness}\end{center}

\hypertarget{header-n3045}{%
\subsubsection{哀公元年}\label{header-n3045}}

【经】元年春王正月,公即位。楚子、陈侯、随侯、许男围蔡。鼷鼠食郊牛,改卜牛。夏四月辛巳,郊。秋,齐侯,卫侯伐晋。冬,仲孙何忌帅师伐邾。

【传】元年春,楚子围蔡,报柏举也。里而栽,广丈,高倍。夫屯昼夜九日,如子西之素。蔡人男女以辨,使疆于江、汝之间而还。蔡于是乎请迁于吴。

吴王夫差败越于夫椒,报檇李也。遂入越。越子以甲楯五千,保于会稽。使大夫种因吴大宰嚭以行成,吴子将许之。伍员曰:``不可。臣闻之树德莫如滋,去疾莫如尽。昔有过浇杀斟灌以伐斟鄩,灭夏后相。后婚方娠,逃出自窦,归于有仍,生少康焉,为仍牧正。惎浇,能戒之。浇使椒求之,逃奔有虞,为之庖正,以除其害。虞思于是妻之以二姚,而邑诸纶。有田一成,有众一旅,能布其德,而兆其谋,以收夏众,抚其官职。使女艾谍浇,使季杼诱豷,遂灭过、戈,复禹之绩。祀夏配天,不失旧物。今吴不如过,而越大于少康,或将丰之,不亦难乎?句践能亲而务施,施不失人,亲不弃劳。与我同壤而世为仇雠,于是乎克而弗取,将又存之,违天而长寇仇,后虽悔之,不可食已。姬之衰也,日可俟也。介在蛮夷,而长寇仇,以是求伯,必不行矣。」弗听。退而告人曰:``越十年生聚,而十年教训,二十年之外,吴其为沼乎!」三月,越及吴平。吴入越,不书,吴不告庆,越不告败也。

夏四月,齐侯、卫侯救邯郸,围五鹿。

吴之入楚也,使召陈怀公。怀公朝国人而问焉,曰:``欲与楚者右,欲与吴者左。陈人从田,无田从党。」逢滑当公而进,曰:``臣闻国之兴也以福,其亡也以祸。今吴未有福,楚未有祸。楚未可弃,吴未可从。而晋,盟主也,若以晋辞吴,若何?」公曰:``国胜君亡,非祸而何?」对曰:``国之有是多矣,何必不复。小国犹复,况大国乎?臣闻国之兴也,视民如伤,是其福也。其亡也,以民为土芥,是其祸也。楚虽无德,亦不艾杀其民。吴日敝于兵,暴骨如莽,而未见德焉。天其或者正训楚也!祸之适吴,其何日之有?」陈侯从之。及夫差克越,乃修先君之怨。秋八月,吴侵陈,修旧怨也。

齐侯、卫侯会于乾侯,救范氏也,师及齐师、卫孔圉、鲜虞人伐晋,取棘蒲。

吴师在陈,楚大夫皆惧,曰:``阖庐惟能用其民,以败我于柏举。今闻其嗣又甚焉,将若之何?」子西曰:``二三子恤不相睦,无患吴矣。昔阖庐食不二味,居不重席,室不崇坛,器不彤镂,宫室不观,舟车不饰,衣服财用,择不取费。在国,天有灾疠,亲巡孤寡,而共其乏困。在军,熟食者分,而后敢食。其所尝者,卒乘与焉。勤恤其民而与之劳逸,是以民不罢劳,死知不旷。吾先大夫子常易之,所以败我也。今闻夫差次有台榭陂池焉,宿有妃嫱嫔御焉。一日之行,所欲必成,玩好必从。珍异是聚,观乐是务,视民如仇,而用之日新。夫先自败也已。安能败我?」

冬十一月,晋赵鞅伐朝歌。

\hypertarget{header-n3056}{%
\subsubsection{哀公二年}\label{header-n3056}}

【经】二年春王二月,季孙斯、叔孙州仇、仲孙何忌帅师伐邾,取漷东田及沂西田。癸巳,叔孙州仇、仲孙何忌及邾子盟于句绎。夏四月丙子,卫侯元卒。滕子来朝。晋赵鞅帅师纳卫世子蒯聩于戚。秋八月甲戌,晋赵鞅帅师及郑罕达帅师战于铁,郑师败绩。冬十月,葬卫灵公。十有一月,蔡迁于州来。蔡杀其大夫公子驷。

【传】二年春,伐邾,将伐绞。邾人爱其土,故赂以淳阜、沂之田而受盟。

初,卫侯游于郊,子南仆。公曰:``余无子,将立女。」不对。他日,又谓之。对曰:``郢不足以辱社稷,君其改图。君夫人在堂,三揖在下。君命只辱。」

夏,卫灵公卒。夫人曰:``命公子郢为大子,君命也。」对曰:``郢异于他子。且君没于吾手,若有之,郢必闻之。且亡人之子辄在。」乃立辄。

六月乙酉,晋赵鞅纳卫大子于戚。宵迷,阳虎曰:``右河而南,必至焉。」使大子絻,八人衰絰,伪自卫逆者。告于门,哭而入,遂居之。

秋八月,齐人输范氏粟,郑子姚、子般送之。士吉射逆之,赵鞅御之,遇于戚。阳虎曰:``吾车少,以兵车之旆,与罕、驷兵车先陈。罕、驷自后随而从之,彼见吾貌,必有惧心。于是乎会之,必大败之。」从之。卜战,龟焦。乐丁曰:``《诗》曰:『爰始爰谋,爰契我龟。』谋协,以故兆询可也。」简子誓曰:``范氏、中行氏,反易天明,斩艾百姓,欲擅晋国而灭其君。寡君恃郑而保焉。今郑为不道,弃君助臣,二三子顺天明,从君命,经德义,除诟耻,在此行也。克敌者,上大夫受县,下大夫受郡,士田十万,庶人工商遂,人臣隶圉免。志父无罪,君实图之。若其有罪,绞缢以戮,桐棺三寸,不设属辟,素车朴马,无入于兆,下卿之罚也。」甲戌,将战,邮无恤御简子,卫太子为右。登铁上,望见郑师众,大子惧,自投于车下。子良授大子绥而乘之,曰:``妇人也。」简子巡列,曰:``毕万,匹夫也。七战皆获,有马百乘,死于牖下。群子勉之,死不在寇。」繁羽御赵罗,宋勇为右。罗无勇,麇之。吏诘之,御对曰:``痁作而伏。」卫大子祷曰:``会孙蒯聩敢昭告皇祖文王、烈祖康叔、文祖襄公:郑胜乱从,晋午在难,不能治乱,使鞅讨之。蒯聩不敢自佚,备持矛焉。敢告无绝筋,无折骨,无面伤,以集大事,无作三祖羞。大命不敢请,佩玉不敢爱。」

郑人击简子中肩,毙于车中,获其峰旗。大子救之以戈,郑师北,获温大夫赵罗。大子复伐之,郑师大败,获齐粟千车。赵孟喜曰:``可矣。」傅叟曰:``虽克郑,犹有知在,忧未艾也。」

初,周人与范氏田,公孙尨税焉。赵氏得而献之,吏请杀之。赵孟曰:``为其主也,何罪?」止而与之田。及铁之战,以徒五百人宵攻郑师,取峰旗于子姚之幕下,献曰:``请报主德。」

追郑师。姚、般、公孙林殿而射,前列多死。赵孟曰:``国无小。」既战,简子曰:``吾伏弢呕血,鼓音不衰,今日我上也。」大子曰:``吾救主于车,退敌于下,我,右之上也。」邮良曰:``我两靷将绝,吾能止之,我,御之上也。」驾而乘材,两靷皆绝。

吴泄庸如蔡纳聘,而稍纳师。师毕入,众知之。蔡侯告大夫,杀公子驷以说,哭而迁墓。冬,蔡迁于州来。

\hypertarget{header-n3069}{%
\subsubsection{哀公三年}\label{header-n3069}}

【经】三年春,齐国夏、卫石曼姑帅师围戚。夏四月甲午,地震。五月辛卯,桓宫、僖宫灾。季孙斯、叔孙州仇帅师城启阳。宋乐髡帅师伐曹。秋七月丙子,季孙斯卒。蔡人放其大夫公孙猎于吴。冬十月癸卯,秦伯卒。叔孙州仇、仲孙何忌帅师围邾。

【传】三年春,齐、卫围戚,救援于中山。

夏五月辛卯,司铎火。火逾公宫,桓、僖灾。救火者皆曰:``顾府。」南宫敬叔至,命周人出御书,俟于宫,曰:``庀女而不在,死。」子服景伯至,命宰人出礼书,以待命:``命不共,有常刑。」校人乘马,巾车脂辖。百官官备,府库慎守,官人肃给。济濡帷幕,郁攸从之,蒙葺公屋。自大庙始,外内以悛,助所不给。有不用命,则有常刑,无赦。公父文伯至,命校人驾乘车。季桓子至,御公立于象魏之外,命救火者伤人则止,财可为也。命藏《象魏》,曰:``旧章不可亡也。」富父槐至,曰:``无备而官办者,犹拾也。」于是乎去表之蒿,道还公宫。孔子在陈,闻火,曰:``其桓、僖乎!」

刘氏、范氏世为婚姻,苌弘事刘文公,故周与范氏。赵鞅以为讨。六月癸卯,周人杀苌弘。

秋,季孙有疾,命正常曰:``无死。南孺子之子,男也,则以告而立之。女也,则肥也可。」季孙卒,康子即位。既葬,康子在朝。南氏生男,正常载以如朝,告曰:``夫子有遗言,命其圉臣曰:『南氏生男,则以告于君与大夫而立之。』今生矣,男也,敢告。」遂奔卫。康子请退。公使共刘视之,则或杀之矣,乃讨之。召正常,正常不反。

冬十月,晋赵鞅围朝歌,师于其南。荀寅伐其郛,使其徒自北门入,己犯师而出。癸丑,奔邯郸。十一月,赵鞅杀士皋夷,恶范氏也。

\hypertarget{header-n3078}{%
\subsubsection{哀公四年}\label{header-n3078}}

【经】四年春王二月庚戌,盗杀蔡侯申。蔡公孙辰出奔吴。葬秦惠公。宋人执小邾子。夏,蔡杀其大夫公孙姓、公孙霍。晋人执戎蛮子赤归于楚。城西郛。六月辛丑,亳社灾。秋八月甲寅,滕子结卒。冬十有二月,葬蔡昭公。葬滕顷公。

【传】四年春,蔡昭侯将如吴,诸大夫恐其又迁也,承,公孙翩逐而射之,入于家人而卒。以两矢门之。众莫敢进。文之锴后至,曰:``如墙而进,多而杀二人。」锴执弓而先,翩射之,中肘。锴遂杀之。故逐公孙辰,而杀公孙姓、公孙盱。

夏,楚人既克夷虎,乃谋北方。左司马眅、申公寿余、叶公诸梁致蔡于负函,致方城之外于缯关,曰:``吴将水斥江入郢,将奔命焉。」为一昔之期,袭梁及霍。单浮余围蛮氏,蛮氏溃。蛮子赤奔晋阴地。司马起丰、析与狄戎,以临上雒。左师军于菟和,右师军于仓野,使谓阴地之命大夫士蔑曰:``晋、楚有盟,好恶同之。若将不废,寡君之愿也。不然,将通于少习以听命。」士蔑请诸赵孟。赵孟曰:``晋国未宁,安能恶于楚,必速与之。」士蔑乃致九州之戎。将裂田以与蛮子而城之,且将为之卜。蛮子听卜,遂执之,与其五大夫,以畀楚师于三户。司马致邑,立宗焉,以诱其遗民,而尽俘以归。

秋七月,齐陈乞、弦施、卫宁跪救范氏。庚午,围五鹿。九月,赵鞅围邯郸。冬十一月,邯郸降。荀寅奔鲜虞,赵稷奔临。十二月,弦施逆之,遂堕临。国夏伐晋,取邢、任、栾、鄗、逆畤、阴人、盂、壶口。会鲜虞,纳荀寅于柏人。

\hypertarget{header-n3086}{%
\subsubsection{哀公五年}\label{header-n3086}}

【经】五年春,城毗。夏,齐侯伐宋。晋赵鞅帅师伐卫。秋九月癸酉,齐侯杵臼卒。冬,叔还如齐。闰月,葬齐景公。

【传】五年春,晋围柏人,荀寅、士吉射奔齐。初,范氏之臣王生恶张柳朔,言诸昭子,使为柏人。昭子曰:``夫非而仇乎?」对曰:``私仇不及公,好不废过,恶不去善,义之经也。臣敢违之?」及范氏出,张柳朔谓其子:``尔从主,勉之!我将止死,王生授我矣。吾不可以僭之。」遂死于柏人。

夏,赵鞅伐卫,范氏之故也,遂围中牟。

齐燕姬生子,不成而死,诸子鬻姒之子荼嬖。诸大夫恐其为大子也,言于公曰:``君之齿长矣,未有大子,若之何?」公曰:``二三子间于忧虞,则有疾疢。亦姑谋乐,何忧于无君?」公疾,使国惠子、高昭子立荼,置群公子于莱。秋,齐景公卒。冬十月,公子嘉、公子驹、公子黔奔卫,公子锄、公子阳生来奔。莱人歌之曰:``景公死乎不与埋,三军之事乎不与谋。师乎师乎,何党之乎?」

郑驷秦富而侈,嬖大夫也,而常陈卿之车服于其庭。郑人恶而杀之。子思曰:``《诗》曰:『不解于位,民之攸塈。』不守其位,而能久者鲜矣。《商颂》曰:『不僭不滥,不敢怠皇,命以多福。』」

\hypertarget{header-n3094}{%
\subsubsection{哀公六年}\label{header-n3094}}

【经】六年春,城邾瑕。晋赵鞅帅师伐鲜虞。吴伐陈。夏,齐国夏及高张来奔。叔还公吴于柤。秋七月庚寅,楚子轸卒。齐阳生入齐。齐陈乞弑其君荼。冬,仲孙何忌帅师伐邾。宋向巢帅师伐曹。

【传】六年春,晋伐鲜虞,治范氏之乱也。

吴伐陈,复修旧怨也。楚子曰:``吾先君与陈有盟,不可以不救。」乃救陈,师于城父。

齐陈乞伪事高、国者,每朝必骖乘焉。所从必言诸大夫,曰:``彼皆偃蹇,将弃子之命。皆曰:『高、国得君,必逼我,盍去诸?』固将谋子,子早图之。图之,莫如尽灭之。需,事之下也。」及朝,则曰:``彼虎狼也,见我在子之侧,杀我无日矣。请就之位。」又谓诸大夫曰:``二子者祸矣!恃得君而欲谋二三子,曰:『国之多难,贵宠之由,尽去之而后君定。』既成谋矣,盍及其未作也,先诸?作而后悔,亦无及也。」大夫从之。

夏六月戊辰,陈乞、鲍牧及诸大夫,以甲入于公宫。昭子闻之,与惠子乘如公,战于庄,败。国人追之,国夏奔莒,遂及高张、晏圉、弦施来奔。

秋七月,楚子在城父,将救陈。卜战,不吉;卜退,不吉。王曰:``然则死也!再败楚师,不如死。弃盟逃仇,亦不如死。死一也,其死仇乎!」命公子申为王,不可;则命公子结,亦不可;则命公子启,五辞而后许。将战,王有疾。庚寅,昭王攻大冥,卒于城父。子闾退,曰:``君王舍其子而让,群臣敢忘君乎?从君之命,顺也。立君之子,亦顺也。二顺不可失也。」与子西、子期谋,潜师闭涂,逆越女之子章,立之而后还。

是岁也,有云如众赤鸟,夹日以飞,三日。楚子使问诸周大史。周大史曰:``其当王身乎!若禜之,可移于令尹、司马。」王曰:``除腹心之疾,而置诸股肱,何益?不谷不有大过,天其夭诸?有罪受罚,又焉移之?」遂弗禜。

初,昭王有疾。卜曰:``河为祟。」王弗祭。大夫请祭诸郊,王曰:``三代命祀,祭不越望。江、汉、雎、章,楚之望也。祸福之至,不是过也。不谷虽不德,河非所获罪也。」遂弗祭。孔子曰:``楚昭王知大道矣!其不失国也,宜哉!《夏书》曰:『惟彼陶唐,帅彼天常,有此冀方。今失其行,乱其纪纲,乃灭而亡。』又曰:『允出兹在兹。』由己率常可矣。」

八月,齐邴意兹来奔。

陈僖子使召公子阳生。阳生驾而见南郭且于,曰:``尝献马于季孙,不入于上乘,故又献此,请与子乘之。」出莱门而告之故。阚止知之,先待诸外。公子曰:``事未可知,反,与壬也处。」戒之,遂行。逮夜,至于齐,国人知之。僖子使子士之母养之,与馈者皆入。

冬十月丁卯,立之。将盟,鲍子醉而往。其臣差车鲍点曰:``此谁之命也?」陈子曰:``受命于鲍子。」遂诬鲍子曰:``子之命也。」鲍子曰:``女忘君之为孺子牛而折其齿乎?而背之也!」悼公稽首,曰:``吾子奉义而行者也。若我可,不必亡一大夫。若我不可,不必亡一公子。义则进,否则退,敢不唯子是从?废兴无以乱,则所愿也。」鲍子曰:``谁非君之子?」乃受盟。使胡姬以安孺子如赖。去鬻姒,杀王甲,拘江说,囚王豹于句窦之丘。

公使朱毛告于陈子,曰:``微子则不及此。然君异于器,不可以二。器二不匮,君二多难,敢布诸大夫。」僖子不对而泣,曰:``君举不信群臣乎?以齐国之困,困又有忧。少君不可以访,是以求长君,庶亦能容群臣乎!不然,夫孺子何罪?」毛覆命,公悔之。毛曰:``君大访于陈子,而图其小可也。」使毛迁孺子于骀,不至,杀诸野幕之下,葬诸殳冒淳。

\hypertarget{header-n3109}{%
\subsubsection{哀公七年}\label{header-n3109}}

【经】七年春,宋皇瑗帅师侵郑。晋魏曼多帅师侵卫。夏,公会吴于鄫。秋,公伐邾。八月己酉,入邾,以邾子益来。宋人围曹。冬,郑驷弘帅师救曹。

【传】七年春,宋师侵郑,郑叛晋故也。

晋师侵卫,卫不服也。

夏,公会吴于鄫。吴来征百牢,子服景伯对曰:``先王未之有也。」吴人曰:``宋百牢我,鲁不可以后宋。且鲁牢晋大夫过十,吴王百牢,不亦可乎?」景伯曰:``晋范鞅贪而弃礼,以大国惧敝邑,故敝邑十一牢之。君若以礼命于诸侯,则有数矣。若亦弃礼,则有淫者矣。周之王也,制礼,上物不过十二,以为天之大数也。今弃周礼,而曰必百牢,亦唯执事。」吴人弗听。景伯曰:``吴将亡矣!弃天而背本不与,必弃疾于我。」乃与之。

大宰嚭召季康子,康子使子贡辞。大宰嚭曰:``国君道长,而大夫不出门,此何礼也?」对曰:``岂以为礼?畏大国也。大国不以礼命于诸侯,苟不以礼,岂可量也?寡君既共命焉,其老岂敢弃其国?大伯端委以治周礼,仲雍嗣之,断发文身,赢以为饰,岂礼也哉?有由然也。」反自鄫,以吴为无能为也。

季康子欲伐邾,乃飨大夫以谋之。子服景伯曰:``小所以事大,信也。大所以保小,仁也。背大国,不信。伐小国,不仁。民保于城,城保于德,失二德者,危,将焉保?」孟孙曰:``二三子以为何如?恶贤而逆之?」对曰:``禹合诸侯于涂山,执玉帛者万国。今其存者,无数十焉。唯大不字小,小不事大也。知必危,何故不言?鲁德如邾,而以众加之,可乎?」不乐而出。

秋,伐邾,及范门,犹闻钟声。大夫谏,不听,茅成子请告于吴,不许,曰:``鲁击柝闻于邾,吴二千里,不三月不至,何及于我?且国内岂不足?」成子以茅叛,师遂入邾,处其公宫,众师昼掠,邾众保于绎。师宵掠,以邾子益来,献于亳社,囚诸负瑕。负瑕故有绎。邾茅夷鸿以束帛乘韦,自请救于吴,曰:``鲁弱晋而远吴,冯恃其众,而背君之盟,辟君之执事,以陵我小国。邾非敢自爱也,惧君威之不立。君威之不立,小国之忧也。若夏盟于鄫衍,秋而背之,成求而不违,四方诸侯,其何以事君?且鲁赋八百乘,君之贰也。邾赋六百乘,君之私也。以私奉贰,唯君图之。」吴子从之。

宋人围曹。郑桓子思曰:``宋人有曹,郑之患也。不可以不救。」冬,郑师救曹,侵宋。

初,曹人或梦众君子立于社宫,而谋亡曹,曹叔振铎请待公孙强,许之。旦而求之曹,无之。戒其子曰:``我死,尔闻公孙强为政,必去之。」及曹伯阳即位,好田弋。曹鄙人公孙强好弋,获白雁,献之,且言田弋之说,说之。因访政事,大说之。有宠,使为司城以听政。梦者之子乃行。强言霸说于曹伯,曹伯从之,乃背晋而奸宋。宋人伐之,晋人不救。筑五邑于其郊,曰黍丘、揖丘、大城、钟、邗。

\hypertarget{header-n3121}{%
\subsubsection{哀公八年}\label{header-n3121}}

【经】八年春王正月,宋公入曹,以曹伯阳归。吴伐我。夏,齐人取讙及阐。归邾子益子邾。秋七月。冬十有二月癸亥,杞伯过卒。齐人归讙及阐。

【传】八年春,宋公伐曹,将还,褚师子肥殿。曹人诟之,不行,师待之。公闻之,怒,命反之,遂灭曹。执曹伯及司城强以归,杀之。

吴为邾故,将伐鲁,问于叔孙辄。叔孙辄对曰:``鲁有名而无情,伐之,必得志焉。」退而告公山不狃。公山不狃曰:``非礼也。君子违,不适仇国。未臣而有伐之,奔命焉,死之可也。所托也则隐。且夫人之行也,不以所恶废乡。今子以小恶而欲覆宗国,不亦难乎?若使子率,子必辞,王将使我。」子张疾之。王问于子泄,对曰:``鲁虽无与立,必有与毙;诸侯将救之,未可以得志焉。晋与齐、楚辅之,是四仇也。夫鲁、齐、晋之唇,唇亡齿寒,君所知也。不救何为?」

三月,吴伐我,子泄率,故道险,从武城。初,武城人或有因于吴竟田焉,拘鄫谒之沤菅者,曰:``何故使吾水滋?」及吴师至,拘者道之,以伐武城,克之。王犯尝为之宰,澹枱子羽之父好焉。国人惧,懿子谓景伯:``若之何?」对曰:``吴师来,斯与之战,何患焉?且召之而至,又何求焉?」吴师克东阳而进,舍于五梧,明日,舍于蚕室。公宾庚、公甲叔子与战于夷,获叔子与析朱锄。献于王,王曰:``此同车,必使能,国未可望也。」明日,舍于庚宗,遂次于泗上。微虎欲宵攻王舍,私属徒七百人,三踊于幕庭,卒三百人,有若与焉,及稷门之内。或谓季孙曰:``不足以害吴,而多杀国士,不如已也。」乃止之。吴子闻之,一夕三迁。吴人行成,将盟。景伯曰:``楚人围宋,易子而食,析骸而爨,犹无城下之盟。我未及亏,而有城下之盟,是弃国也。吴轻而远,不能久,将归V请少待之。」弗从。景伯负载,造于莱门,乃请释子服何于吴,吴人许之。以王子姑曹当之,而后止。吴人盟而还。

齐悼公之来也,季康子以其妹妻之,即位而逆之。季鲂侯通焉,女言其情,弗敢与也。齐侯怒,夏五月,齐鲍牧帅师伐我,取讙及阐。

或谮胡姬于齐侯,曰:``安孺子之党也。」六月,齐侯杀胡姬。

齐侯使如吴请师,将以伐我,乃归邾子。邾子又无道,吴子使大宰子余讨之,囚诸楼台,栫之以棘。使诸大夫奉大子革以为政。

秋,及齐平。九月,臧宾如如齐莅盟,齐闾丘明来莅盟,且逆季姬以归,嬖。

鲍牧又谓群公子曰:``使女有马千乘乎?」公子愬之。公谓鲍子:``或谮子,子姑居于潞以察之。若有之,则分室以行。若无之,则反子之所。」出门,使以三分之一行。半道,使以二乘。及潞,麇之以入,遂杀之。

冬十二月,齐人归讙及阐,季姬嬖故也。

\hypertarget{header-n3134}{%
\subsubsection{哀公九年}\label{header-n3134}}

【经】九年春王二月,葬杞僖公。宋皇瑗帅师取郑师于雍丘。夏,楚人伐陈。秋,宋公伐郑。冬十月。

【传】九年春,齐侯使公孟绰辞师于吴。吴子曰:``昔岁寡人闻命。今又革之,不知所从,将进受命于君。」

郑武子剩之嬖许瑕求邑,无以与之。请外取,许之。故围宋雍丘。宋皇瑗围郑师,每日迁舍,垒合,郑师哭。子姚救之,大败。二月甲戌,宋取郑师于雍丘,使有能者无死,以郏张与郑罗归。

夏,楚人伐陈,陈即吴故也。

宋公伐郑。

秋,吴城邗,沟通江、淮。

晋赵鞅卜救郑,遇水适火,占诸史赵、史墨、史龟。史龟曰:``是谓渖阳,可以兴兵。利以伐姜,不利子商。伐齐则可,敌宋不吉。」史墨曰:``盈,水名也。子,水位也。名位敌,不可干也。炎帝为火师,姜姓其后也。水胜火,伐姜则可。」史赵曰:``是谓如川之满,不可游也。郑方有罪,不可救也。救郑则不吉,不知其他。」阳虎以《周易》筮之,遇《泰》ⅱⅰ之《需》ⅴⅰ,曰:``宋方吉,不可与也。微子启,帝乙之元子也。宋、郑,甥舅也。祉,禄也。若帝乙之元子归妹,而有吉禄,我安得吉焉?」乃止。

冬,吴子使来人敬师伐齐。

\hypertarget{header-n3145}{%
\subsubsection{哀公十年}\label{header-n3145}}

【经】十年春王二月,邾子益来奔。公会吴伐齐。三月戊戌,齐侯阳生卒。夏,宋人伐郑。晋赵鞅帅师侵齐。五月,公至自伐齐。葬齐悼公。卫公孟彄自齐归于卫。薛伯夷卒。秋,葬薛惠公。冬,楚公子结帅师伐陈。吴救陈。

【传】十年春,邾隐公来奔。齐甥也,故遂奔齐。

公会吴子、邾子、郯子伐齐南鄙,师于鄎。齐人弑悼公,赴于师。吴子三日哭于军门之外。徐承帅舟师,将自海入齐,齐人败之,吴师乃还。

夏,赵鞅帅师伐齐,大夫请卜之。赵孟曰:``吾卜于此起兵,事不再令,卜不袭吉,行也。」于是乎取犁及辕,毁高唐之郭,侵及赖而还。

秋,吴子使来复人】敬师。

冬,楚子期伐陈。吴延州来季子救陈,谓子期曰:``二君不务德,而力争诸侯,民何罪焉?我请退,以为子名,务德而安民。」乃还。

\hypertarget{header-n3154}{%
\subsubsection{哀公十一年}\label{header-n3154}}

【经】十有一年春,齐国书帅师伐我。夏,陈辕颇出奔郑。五月,公会吴伐齐。甲戌,齐国书帅师及吴战于艾陵,齐师败绩,获齐国书。秋七月辛酉,滕子虞母卒。冬十有一月,葬滕隐公。卫世叔齐出奔宋。

【传】十一年春,齐为鄎故,国书、高无丕帅师伐我,及清。季孙谓其宰冉求曰:``齐师在清,必鲁故也。若之何?」求曰:``一子守,二子从公御诸竟。」季孙曰:``不能。」求曰:``居封疆之间。」季孙告二子,二子不可。求曰:``若不可,则君无出。一子帅师,背城而战。不属者,非鲁人也。鲁之群室,众于齐之兵车。一室敌车,优矣。子何患焉?二子之不欲战也宜,政在季氏。当子之身,齐人伐鲁而不能战,子之耻也。大不列于诸侯矣。」季孙使从于朝,俟于党氏之沟。武叔呼而问战焉,对曰:``君子有远虑,小人何知?」懿子强问之,对曰:``小人虑材而言,量力而共者也。」武叔曰:``是谓我不成丈夫也。」退而蒐乘,孟孺子泄帅右师,颜羽御,邴泄为右。冉求帅左师,管周父御,樊迟为右。季孙曰:``须也弱。」有子曰:``就用命焉。」季氏之甲七千,冉有以武城人三百为己徒卒。老幼守宫,次于雩门之外。五日,右师从之。公叔务人见保者而泣,曰:``事充政重,上不能谋,士不能死,何以治民?吾既言之矣,敢不勉乎!」

师及齐师战于郊,齐师自稷曲,师不逾沟。樊迟曰:``非不能也,不信子也。请三刻而逾之。」如之,众从之。师入齐军,右师奔,齐人从之,陈瓘、陈庄涉泗。孟之侧后入以为殿,抽矢策其马,曰:``马不进也。」林不狃之伍曰:``走乎?」不狃曰:``谁不如?」曰:``然则止乎?」不狃曰:``恶贤?」徐步而死。师获甲首八十,齐人不能师。宵,谍曰:``齐人遁。」冉有请从之三,季孙弗许。孟孺子语人曰:``我不如颜羽,而贤于邴泄。子羽锐敏,我不欲战而能默。泄曰:『驱之。』」公为与其嬖僮汪錡乘,皆死,皆殡。孔子曰:``能执干戈以卫社稷,可无殇也。」冉有用矛于齐师,故能入其军。孔子曰:``义也。」

夏,陈辕颇出奔郑。初,辕颇为司徒,赋封田以嫁公女。有馀,以为己大器。国人逐之,故出。道渴,其族辕咺进稻醴、梁糗、糗脯焉。喜曰:``何其给也?」对曰:``器成而具。」曰:``何不吾谏?」对曰:``惧先行。」

为郊战故,公会吴子伐齐。五月,克博,壬申,至于羸。中军从王,胥门巢将上军,王子姑曹将下军,展如将右军。齐国书将中军,高无丕将上军,宗楼将下军。陈僖子谓其弟书:``尔死,我必得志。」宗子阳与闾丘明相厉也。桑掩胥御国子,公孙夏曰:``二子必死。」将战,公孙夏命其徒歌《虞殡》。陈子行命其徒具含玉。公孙挥命其徒曰:``人寻约,吴发短。」东郭书曰:``三战必死,于此三矣。」使问弦多以琴,曰:``吾不复见子矣。」陈书曰:``此行也,吾闻鼓而已,不闻金矣。」

甲戌,战于艾陵,展如败高子,国子败胥门巢。王卒助之,大败齐师。获国书、公孙夏、闾丘明、陈书、东郭书,革车八百乘,甲首三千,以献于公。将战,吴子呼叔孙,曰:``而事何也?」对曰:``从司马。」王赐之甲、剑、铍,曰:``奉尔君事,敬无废命。」叔孙未能对,卫赐进,曰:``州仇奉甲从君。」而拜。公使大史固归国子之元,置之新箧,褽之以玄纁,加组带焉。置书于其上,曰:``天若不识不衷,何以使下国?」

吴将伐齐,越子率其众以朝焉,王及列士,皆有馈赂。吴人皆喜,惟子胥惧,曰:``是豢吴也夫!」谏曰:``越在我,心腹之疾也。壤地同,而有欲于我。夫其柔服,求济其欲也,不如早从事焉。得志于齐,犹获石田也,无所用之。越不为沼,吴其泯矣,使医除疾,而曰:『必遗类焉』者,未之有也。《盘庚之诰》曰:『其有颠越不共,则劓殄无遗育,无俾易种于兹邑。』是商所以兴也。今君易之,将以求大,不亦难乎?」弗听,使于齐,属其子于鲍氏,为王孙氏。反役,王闻之,使赐之属镂以死,将死,曰:``树吾墓梵檟檟可材也。吴其亡乎!三年,其始弱矣。盈必毁,天之道也。」

秋,季孙命修守备,曰:``小胜大,祸也。齐至无日矣。」

冬,卫大叔疾出奔宋。初,疾娶于宋子朝,其娣嬖。子朝出。孔文子使疾出其妻而妻之。疾使侍人诱其初妻之娣,置于犁,而为之一宫,如二妻。文子怒,欲攻之。仲尼止之。遂夺其妻。或淫于外州,外州人夺之轩以献。耻是二者,故出。卫人立遗,使室孔姞。疾臣向魋纳美珠焉,与之城锄。宋公求珠,魋不与,由是得罪。及桓氏出,城锄人攻大叔疾,卫庄公复之。使处巢,死焉。殡于郧,葬于少禘。

初,晋悼公子憖亡在卫,使其女仆而田。大叔懿子止而饮之酒,遂聘之,生悼子。悼子即位,故夏戊为大夫。悼子亡,卫人翦夏戊。孔文子之将攻大叔也,访于仲尼。仲尼曰:``胡簋之事,则尝学之矣。甲兵之事,未之闻也。」退,命驾而行,曰:``鸟则择木,木岂能择鸟?」文子遽止之,曰:``圉岂敢度其私,访卫国之难也。」将止。鲁人以币召之,乃归。

季孙欲以田赋,使冉有访诸仲尼。仲尼曰:``丘不识也。」三发,卒曰:``子为国老,待子而行,若之何子之不言也?」仲尼不对。而私于冉有曰:``君子之行也,度于礼,施取其厚,事举其中,敛从其薄。如是则以丘亦足矣。若不度于礼,而贪冒无厌,则虽以田赋,将又不足。且子季孙若欲行而法,则周公之典在。若欲苟而行,又何访焉?」弗听。

\hypertarget{header-n3168}{%
\subsubsection{哀公十二年}\label{header-n3168}}

【经】十有二年春,用田赋。夏五月甲辰,孟子卒。公会吴于皋阜。秋,公会卫侯、宋皇瑗于郧。宋向巢帅师伐郑。冬十有二月,螽。

【传】十二年春,王正月,用田赋。

夏五月,昭夫人孟子卒。昭公娶于吴,故不书姓。死不赴,故不称夫人。不反哭,故言不葬小君。孔子与吊,适季氏。季氏不絻,放絰而拜。

公会吴于橐皋。吴子使大宰嚭请寻盟。公不欲,使子贡对曰:``盟所以周信也,故心以制之,玉帛以奉之,言以结之,明神以要之。寡君以为苟有盟焉,弗可改也已。若犹可改,日盟何益?今吾子曰:『必寻盟。』若可寻也,亦可寒也。」乃不寻盟。

吴征会于卫。初,卫人杀吴行人且姚而惧,谋于行人子羽。子羽曰:``吴方无道,无乃辱吾君,不如止也。」子木曰:``吴方无道,国无道,必弃疾于人。吴虽无道,犹足以患卫。往也。长木之毙,无不噬也。国狗之□,无不噬也。而况大国乎?」

秋,卫侯会吴于郧。公及卫侯、宋皇瑗盟,而卒辞吴盟。吴人藩卫侯之舍。子服景伯谓子贡曰:``夫诸侯之会,事既毕矣,侯伯致礼,地主归饩,以相辞也。今吴不行礼于卫,而藩其君舍以难之,子盍见大宰?」乃请束锦以行。语及卫故,大宰嚭曰:``寡君愿事卫君,卫君之来也缓,寡君惧,故将止之。」子贡曰:``卫君之来,必谋于其众。其众或欲或否,是以缓来。其欲来者,子之党也。其不欲来者,子之仇也。若执卫君,是堕党而崇仇也。夫堕子者得其志矣!且合诸侯而执卫君,谁敢不惧?堕党崇仇,而惧诸侯,或者难以霸乎!」大宰嚭说,乃舍卫侯。卫侯归,效夷言。子之尚幼,曰:``君必不免,其死于夷乎!执焉,而又说其言,从之固矣。」

冬十二月,螽。季孙问诸仲尼,仲尼曰:``丘闻之,火伏而后蜇者毕。今火犹西流,司历过也。」

宋郑之间有隙地焉,曰弥作、顷丘、玉畅、岩、戈、锡。子产与宋人为成,曰:``勿有是。」及宋平、元之族自萧奔郑,郑人为之城岩、戈、锡。九月,宋向巢伐郑,取锡,杀元公之孙,遂围岩。十二月,郑罕达救岩。丙申,围宋师。

\hypertarget{header-n3179}{%
\subsubsection{哀公十三年}\label{header-n3179}}

【经】十有三年春,郑罕达帅师取宋师于岩。夏,许男成卒。公会晋侯及吴子于黄池。楚公子申帅师伐陈。于越入吴。秋,公至自会。晋魏曼多帅师侵卫。葬许元公。九月,螽。冬十有一月,有星孛于东方。盗杀陈夏区夫。十有二月,螽。

【传】十三年春,宋向魋救其师。郑子剩使徇曰:``得桓魋者有赏。」魋也逃归,遂取宋师于岩,获成讙、郜延。以六邑为虚。

夏,公会单平公、晋定公、吴夫差于黄池。

六月丙子,越子伐吴,为二隧。畴无馀、讴阳自南方,先及郊。吴大子友、王子地、王孙弥庸、寿于姚自泓上观之。弥庸见姑蔑之旗,曰:``吾父之旗也。不可以见仇而弗杀也。」大子曰:``战而不克,将亡国。请待之。」弥庸不可,属徒五千,王子地助之。乙酉,战,弥庸获畴无馀,地获讴阳。越子至,王子地守。丙戌,复战,大败吴师。获大子友、王孙弥庸、寿于姚。丁亥,入吴。吴人告败于王,王恶其闻也,自刭七人于幕下。

秋七月辛丑,盟,吴、晋争先。吴人曰:``于周室,我为长。」晋人曰:``于姬姓,我为伯。」赵鞅呼司马寅曰:``日旰矣,大事未成,二臣之罪也。建鼓整列,二臣死之,长幼必可知也。」对曰:``请姑视之。」反,曰:``肉食者无墨。今吴王有墨,国胜乎?大子死乎?且夷德轻,不忍久,请少待之。」乃先晋人。吴人将以公见晋侯,子服景伯对使者曰:``王合诸侯,则伯帅侯牧以见于王。伯合诸侯,则侯帅子男以见于伯。自王以下,朝聘玉帛不同。故敝邑之职贡于吴,有丰于晋,无不及焉,以为伯也。今诸侯会,而君将以寡君见晋君,则晋成为伯矣,敝邑将改职贡。鲁赋于吴八百乘,若为子男,则将半邾以属于吴,而如邾以事晋。且执事以伯召诸侯,而以侯终之,何利之有焉?」吴人乃止。既而悔之,将囚景伯,景伯曰:``何也立后于鲁矣。将以二乘与六人从,迟速唯命。」遂囚以还。及户牖,谓大宰曰:``鲁将以十月上辛,有事于上帝先王,季辛而毕。何世有职焉,自襄以来,未之改也。若不会,祝宗将曰:『吴实然。』且谓鲁不共,而执其贱者七人,何损焉?」大宰嚭言于王曰:``无损于鲁,而只为名,不如归之。」乃归景伯。

吴申叔仪乞粮于公孙有山氏,曰:``佩玉、忌兮,余无所系之。旨酒一盛兮,余与褐之父睨之。」对曰:``梁则无矣,粗则有之。若登首山以呼曰:『庚癸乎!』则诺。」

王欲伐宋,杀其丈夫而囚其妇人。大宰嚭曰:``可胜也,而弗能居也。」乃归。

冬,吴及越平。

\hypertarget{header-n3190}{%
\subsubsection{哀公十四年}\label{header-n3190}}

【经】十有四年春,西狩获麟。小邾射以句绎来奔。夏四月,齐陈心互执其君,置于舒州。庚戌,叔还卒。五月庚申朔,日有食之。陈宗竖出奔楚。宋向魋入于曹以叛。莒子狂卒。六月,宋向魋自曹出奔卫。宋向巢来奔。齐人弑其君壬于舒州。秋,晋赵鞅帅师伐卫。八月辛丑,仲孙何忌卒。冬,陈宗竖自楚复入于陈,陈人杀之。陈辕买出奔楚。有星孛。饥。

【传】十四年春,西狩于大野,叔孙氏之车子锄商获麟,以为不祥,以赐虞人。仲尼观之,曰:``麟也。」然后取之。

小邾射以句绎来奔,曰:``使季路要我,吾无盟矣。」使子路,子路辞。季康子使冉有谓之曰:``千乘之国,不信其盟,而信子之言,子何辱焉?」对曰:``鲁有事于小邾,不敢问故,死其城下可也。彼不臣而济其言,是义之也。由弗能。」

齐简公之在鲁也,阚止有宠焉。及即位,使为政。陈成子惮之,骤顾诸朝。诸御鞅言于公曰:``陈、阚不可并也,君其择焉。」弗听。子我夕,陈逆杀人,逢之,遂执以入。陈氏方睦,使疾,而遗之潘沐,备酒肉焉,飨守囚者,醉而杀之,而逃。子我盟诸陈于陈宗。

初,陈豹欲为子我臣,使公孙言己,已有丧而止。既,而言之,曰:``有陈豹者,长而上偻,望视,事君子必得志,欲为子臣。吾惮其为人也,故缓以告。」子我曰:``何害?是其在我也。」使为臣。他日,与之言政,说,遂有宠,谓之曰:``我尽逐陈氏,而立女,若何?」对曰:``我远于陈氏矣。且其违者,不过数人,何尽逐焉?」遂告陈氏。子行曰:``彼得君,弗先,必祸子。」子行舍于公宫。

夏五月壬申,成子兄弟四乘如公。子我在幄,出,逆之。遂入,闭门。侍人御之,子行杀侍人。公与妇人饮酒于檀台,成子迁诸寝。公执戈,将击之。大史子余曰:``非不利也,将除害也。」成子出舍于库,闻公犹怒,将出,曰:``何所无君?」子行抽剑,曰:``需,事之贼也。谁非陈宗?所不杀子者,有如陈宗!」乃止。子我归,属徒,攻闱与大门,皆不胜,乃出。陈氏追之,失道于弇中,适丰丘。丰丘人执之,以告,杀诸郭关。成子将杀大陆子方,陈逆请而免之。以公命取车于道,及耏,众知而东之。出雍门,陈豹与之车,弗受,曰:``逆为余请,豹与余车,余有私焉。事子我而有私于其仇,何以见鲁、卫之士?」东郭贾奔卫。

庚辰,陈恒执公于舒州。公曰:``吾早从鞅之言,不及此。」

宋桓魋之宠害于公,公使夫人骤请享焉,而将讨之。未及,魋先谋公,请以鞍易薄,公曰:``不可。薄,宗邑也。」乃益鞍七邑,而请享公焉。以日中为期,家备尽往。公知之,告皇野曰:``余长魋也,今将祸余,请即救。」司马子仲曰:``有臣不顺,神之所恶也,而况人乎?敢不承命。不得左师不可,请以君命召之。」左师每食击钟。闻钟声,公曰:``夫子将食。」既食,又奏。公曰:``可矣。」以乘车往,曰:``迹人来告曰:『逢泽有介麇焉。』公曰:『虽魋未来,得左师,吾与之田,若何?』君惮告子。野曰:『尝私焉。』君欲速,故以乘车逆子。」与之乘,至,公告之故,拜,不能起。司马曰:``君与之言。」公曰:``所难子者,上有天,下有先君。」对曰:``魋之不共,宋之祸也,敢不唯命是听。」司马请瑞焉,以命其徒攻桓氏。其父兄故臣曰:``不可。」其新臣曰:``从吾君之命。」遂攻之。子颀骋而告桓司马。司马欲入,子车止之,曰:``不能事君,而又伐国,民不与也,只取死焉。」向魋遂入于曹以叛。六月,使左师巢伐之。欲质大夫以入焉,不能。亦入于曹,取质。魋曰:``不可。既不能事君,又得罪于民,将若之何?」乃舍之。民遂叛之。向魋奔卫。向巢来奔,宋公使止之,曰:``寡人与子有言矣,不可以绝向氏之祀。」辞曰:``臣之罪大,尽灭桓氏可也。若以先臣之故,而使有后,君之惠也。若臣,则不可以入矣。」

司马牛致其邑与珪焉,而适齐。向魋出于卫地,公文氏攻之,求夏后氏之璜焉。与之他玉,而奔齐,陈成子使为次卿。司马牛又致其邑焉,而适吴。吴人恶之,而反。赵简子召之,陈成子亦召之。卒于鲁郭门之外,阬氏葬诸丘舆。

甲午,齐陈恒弑其君壬于舒州。孔丘三日齐,而请伐齐三。公曰:``鲁为齐弱久矣,子之伐之,将若之何?」对曰:``陈恒弑其君,民之不与者半。以鲁之众,加齐之半,可克也。」公曰:``子告季孙。」孔子辞。退而告人曰:``吾以从大夫之后也,故不敢不言。」

初,孟孺子泄将圉马于成。成宰公孙宿不受,曰:``孟孙为成之病,不圉马焉。」孺子怒,袭成。从者不得入,乃反。成有司使,孺子鞭之。秋八月辛丑,孟懿子卒。成人奔丧,弗内。袒免哭于衢,听共,弗许。惧,不归。

\hypertarget{header-n3204}{%
\subsubsection{哀公十五年}\label{header-n3204}}

【经】十有五年春王正月,成叛。夏五月,齐高无ぶ出奔北燕。郑伯伐宋。秋八月,大雩。晋赵鞅帅师伐卫。冬,晋侯伐郑。及齐平。卫公孟彄出奔齐。

【传】十五年春,成叛于齐。武伯伐成,不克,遂城输。

夏,楚子西、子期伐吴,乃桐汭。陈侯使公孙贞子吊焉,及良而卒,将以尸入。吴子使大宰嚭劳,且辞曰:``以水潦之不时,无乃廪然陨大夫之尸,以重寡君之忧。寡君敢辞。」上介芋尹盖对曰:``寡君闻楚为不道,荐伐吴国,灭厥民人。寡君使盖备使,吊君之下吏。无禄,使人逢天之戚,大命陨队,绝世于良,废日共积,一日迁次。今君命逆使人曰:『无以尸造于门。』是我寡君之命委于草莽也。且臣闻之曰:『事死如事生,礼也。』于是乎有朝聘而终,以尸将事之礼。又有朝聘而遭丧之礼。若不以尸将命,是遭丧而还也,无乃不可乎!以礼防民,犹或逾之。今大夫曰:『死而弃之』,是弃礼也。其何以为诸侯主?先民有言曰:『无秽虐士。』备使奉尸将命,苟我寡君之命达于君所,虽陨于深渊,则天命也,非君与涉人之过也。」吴人内之。

秋,齐陈瓘如楚。过卫,仲田见之,曰:``天或者以陈氏为斧斤,既斫丧公室,而他人有之,不可知也。其使终飨之,亦不可知也。若善鲁以待时,不亦可乎?何必恶焉?」子玉曰:``然,吾受命矣,子使告我弟。」

冬,及齐平。子服景伯如齐,子赣为介,见公孙成,曰:``人皆臣人,而有背人之心。况齐人虽为子役,其有不贰乎?子,周公之孙也,多飨大利,犹思不义。利不可得,而丧宗国,将焉用之?」成曰:``善哉!吾不早闻命。」

陈成子馆客,曰:``寡君使恒告曰:『寡君愿事君如事卫君。』」景伯揖子赣而进之。对曰:``寡君之愿也。昔晋人伐卫,齐为卫故,伐晋冠氏,丧车五百,因与卫地,自济以西,禚、媚、杏以南,书社五百。吴人加敝邑以乱,齐因其病,取讙与阐。寡君是以寒心。若得视卫君之事君也,则固所愿也。」成子病之,乃归成。公孙宿以其兵甲入于嬴。

卫孔圉取大子蒯聩之姊,生悝。孔氏之竖浑良夫长而美,孔文子卒,通于内。大子在戚,孔姬使之焉。大子与之言曰:``苟使我入获国,服冕乘轩,三死无与。」与之盟,为请于伯姬。

闰月,良夫与大子入,舍于孔氏之外圃。昏,二人蒙衣而乘,寺人罗御,如孔氏。孔氏之老栾宁问之,称姻妾以告。遂入,适伯姬氏。既食,孔伯姬杖戈而先,大子与五人介,舆豭从之。迫孔悝于厕,强盟之,遂劫以登台。栾宁将饮酒,炙未熟,闻乱,使告季子。召获驾乘车,行爵食炙,奉卫侯辄来奔。季子将入,遇子羔将出,曰:``门已闭矣。」季子曰:``吾姑至焉。」子羔曰:``弗及,不践其难。」季子曰:``食焉,不辟其难。」子羔遂出。子路入,及门,公孙敢门焉,曰:``无入为也。」季子曰:``是公孙,求利焉而逃其难。由不然,利其禄,必救其患。」有使者出,乃入。曰:``大子焉用孔悝?虽杀之,必或继之。」且曰:``大子无勇,若燔台,半,必舍孔叔。」大子闻之,惧,下石乞、盂□敌子路。以戈击之,断缨。子路曰:``君子死,冠不免。」结缨而死。孔子闻卫乱,曰:``柴也其来,由也死矣。」孔悝立庄公。庄公害故政,欲尽去之,先谓司徒瞒成曰:``寡人离病于外久矣,子请亦尝之。」归告褚师比,欲与之伐公,不果。

\hypertarget{header-n3215}{%
\subsubsection{哀公十六年}\label{header-n3215}}

【经】十有六年春王正月己卯,卫世子蒯聩自戚入于卫,卫侯辄来奔。二月,卫子还成出奔宋。夏四月己丑,孔丘卒。

【传】十六年春,瞒成、褚师比出奔宋。

卫侯使鄢武子告于周曰:``蒯聩得罪于君父君母,逋窜于晋。晋以王室之故,不弃兄弟,置诸河上。天诱其衷,获嗣守封焉。使下臣肸敢告执事。」王使单平公对曰:``肸以嘉命来告余一人。往谓叔父,余嘉乃成世,复尔禄次。敬之哉!方天之休,弗敬弗休,悔其可追?」

夏四月己丑,孔丘卒。公诔之曰:``旻天不吊,不憖遗一老。俾屏余一人以在位,茕茕余在疚。呜呼哀哉!尼父。无自律。」子赣曰:``君其不没于鲁乎!夫子之言曰:『礼失则昏,名失则愆。』失志为昏,失所为愆。生不能用,死而诔之,非礼也。称一人,非名也。君两失之。」

六月,卫侯饮孔悝酒于平阳,重酬之,大夫皆有纳焉。醉而送之,夜半而遣之。载伯姬于平阳而行,及西门,使贰车反祏于西圃。子伯季子初为孔氏臣,新登于公,请追之,遇载祏者,杀而乘其车。许公为反祏,遇之,曰:``与不仁人争明,无不胜。」必使先射,射三发,皆远许为。许为射之,殪。或以其车从,得祏于囊中。孔悝出奔宋。

楚大子建之遇谗也,自城父奔宋。又辟华氏之乱于郑,郑人甚善之。又适晋,与晋人谋袭郑,乃求复焉。郑人复之如初。晋人使谍于子木,请行而期焉。子木暴虐于其私邑,邑人诉之。郑人省之,得晋谍焉。遂杀子木。其子曰胜,在吴。子西欲召之,叶公曰:``吾闻胜也诈而乱,无乃害乎?」子西曰:``吾闻胜也信而勇,不为不利,舍诸边竟,使卫藩焉。」叶公曰:``周仁之谓信,率义之谓勇。吾闻胜也好复言,而求死士,殆有私乎?复言,非信也。期死,非勇也。子必悔之。」弗从。召之使处吴竟,为白公。请伐郑,子西曰:``楚未节也。不然,吾不忘也。」他日,又请,许之。未起师,晋人伐郑,楚救之,与之盟。胜怒,曰:``郑人在此,仇不远矣。」

胜自厉剑,子期之子平见之,曰:``王孙何自厉也?」曰:``胜以直闻,不告女,庸为直乎?将以杀尔父。」平以告子西。子西曰:``胜如卵,余翼而长之。楚国第,我死,令尹、司马,非胜而谁?」胜闻之,曰:``令尹之狂也!得死,乃非我。」子西不悛。胜谓石乞曰:``王与二卿士,皆五百人当之,则可矣。」乞曰:``不可得也。」曰:``市南有熊宜僚者,若得之,可以当五百人矣。」乃从白公而见之,与之言,说。告之故,辞。承之以剑,不动。胜曰:``不为利谄,不为威惕,不泄人言以求媚者,去之。」

吴人伐慎,白公败之。请以战备献,许之。遂作乱。秋七月,杀子西、子期于朝,而劫惠王。子西以袂掩面而死。子期曰:``昔者吾以力事君,不可以弗终。」抉豫章以杀人而后死。石乞曰:``焚库弑王,不然不济。」白公曰:``不可。弑王,不祥,焚库,无聚,将何以守矣?」乞曰:``有楚国而治其民,以敬事神,可以得祥,且有聚矣,何患?」弗从。叶公在蔡,方城之外皆曰:``可以入矣。」子高曰:``吾闻之,以险侥幸者,其求无餍,偏重必离。」闻其杀齐管修也而后入。

白公欲以子闾为王,子闾不可,遂劫以兵。子闾曰:``王孙若安靖楚国,匡正王室,而后庇焉,启之愿也,敢不听从。若将专利以倾王室,不顾楚国,有死不能。」遂杀之,而以王如高府,石乞尹门,圉公阳穴宫,负王以如昭夫人之宫。叶公亦至,及北门,或遇之,曰:``君胡不胄?国人望君如望慈父母焉。盗贼之矢若伤君,是绝民望也。若之何不胄?」乃胄而进。又遇一人曰:``君胡胄?国人望君如望岁焉,日日以几。若见君面,是得艾也。民知不死,其亦夫有奋心,犹将旌君以徇于国,而反掩面以绝民望,不亦甚乎?」乃免胄而进。遇箴尹固,帅其属将与白公。子高曰:``微二子者,楚不国矣。弃德从贼,其可保乎?」乃从叶公。使与国人以攻白公。白公奔山而缢,其徒微之。生拘石乞而问白公之死焉,对曰:``余知其死所,而长者使余勿言。」曰:``不言将烹。」乞曰:``此事克则为卿,不克则烹,固其所也,何害?」乃烹石乞。王孙燕奔黄氏。诸梁兼二事,国宁,乃使宁为令尹,使宽为司马,而老于叶。

卫侯占梦,嬖人求酒于大叔僖子,不得,与卜人比而告公曰:``君有大臣在西南隅,弗去,惧害。」乃逐大叔遗。遗奔晋。卫侯谓浑良夫曰:``吾继先君而不得其器,若之何?良夫代执火者而言,曰:``疾与亡君,皆君之子也。召之而择材焉可也,若不材,器可得也。」竖告大子。大子使五人舆豭从己,劫公而强盟之,且请杀良夫。公曰:``其盟免三死。」曰:``请三之后,有罪杀之。」公曰:``诺哉!」

\hypertarget{header-n3228}{%
\subsubsection{哀公十七年}\label{header-n3228}}

【传】十七年春,卫侯为虎幄于藉圃,成,求令名者,而与之始食焉。大子请使良夫。良夫乘衷甸两牡,紫衣狐裘,至,袒袭,不释剑而食。大子使牵以退,数之以三罪而杀之。

三月,越子伐吴。吴子御之笠泽,夹水而陈。越子为左右句卒,使夜或左或右,鼓噪而进。吴师分以御之。越子以三军潜涉,当吴中军而鼓之,吴师大乱,遂败之。

晋赵鞅使告于卫曰:``君之在晋也,志父为主。请君若大子来,以免志父。不然,寡君其曰,志父之为也。」卫侯辞以难。大子又使椓之。

夏六月,赵鞅围卫。齐国观、陈瓘救卫,得晋人之致师者。子玉使服而见之,曰:``国子实执齐柄,而命瓘曰:『无辟晋师。』岂敢废命?子又何辱?」简子曰:``我卜伐卫,未卜与齐战。」乃还。

楚白公之乱,陈人恃其聚而侵楚。楚既宁,将取陈麦。楚子问帅于大师子谷与叶公诸梁,子谷曰:``右领差车与左史老,皆相令尹、司马以伐陈,其可使也。」子高曰:``率贱,民慢之,惧不用命焉。」子谷曰:``观丁父,鄀俘也,武王以为军率,是以克州、蓼,服随、唐,大启群蛮。彭仲爽,申俘也,文王以为令尹,实县申、息,朝陈、蔡,封畛于汝。唯其任也,何贱之有?」子高曰:``天命不谄。令尹有憾于陈,天若亡之,其必令尹之子是与,君盍舍焉?臣惧右领与左史有二俘之贱,而无其令德也。」王卜之,武城尹吉。使帅师取陈麦。陈人御之,败,遂围陈。秋七月己卯,楚公孙朝帅师灭陈。

王与叶公枚卜子良以为令尹。沈尹朱曰:``吉,过于其志。」叶公曰:``王子而相国,过将何为?」他日,改卜子国而使为令尹。

卫侯梦于北宫,见人登昆吾之观,被发北面而噪曰:``登此昆吾之虚,绵绵生之瓜。余为浑良夫,叫天无辜。」公亲筮之,胥弥赦占之,曰:``不害。」与之邑,置之,而逃奔宋。卫侯贞卜,其繇曰:``如鱼赬尾,衡流而方羊。裔焉大国,灭之将亡。阖门塞窦,乃自后逾。」

冬十月,晋复伐卫,入其郛。将入城,简子曰:``止。叔向有言曰:『怙乱灭国者无后。』」卫人出庄公而晋平,晋立襄公之孙般师而还。十一月,卫侯自鄄入,般师出。

初,公登城以望,见戎州。问之,以告。公曰:``我姬姓也,何戎之有焉?」翦之。公使匠久。公欲逐石圃,未及而难作。辛已,石圃因匠氏攻公,公阖门而请,弗许。逾于北方而队,折股。戎州人攻之,大子疾、公子青逾从公,戎州人杀之。公入于戎州己氏。初,公自城上见己氏之妻发美,使髡之,以为吕姜□。既入焉,而示之璧,曰:``活我,吾与女璧。」己氏曰:``杀女,璧其焉往?」遂杀之而取其璧。卫人复公孙般师而立之。十二月,齐人伐卫,卫人请平。立公子起,执般师以归,舍诸潞。

公会齐侯,盟于蒙,孟武伯相。齐侯稽首,公拜。齐人怒,武伯曰:``非天子,寡君无所稽首。」武伯问于高柴曰:``诸侯盟,谁执牛耳?」季羔曰:``鄫衍之役,吴公子姑曹。发阳之役,卫石魋。」武伯曰:``然则彘也。」

宋皇瑗之子麇,有友曰田丙,而夺其兄劖般邑以与之。劖般愠而行,告桓司马之臣子仪克。子仪克适宋,告夫人曰:``麇将纳桓氏。」公问诸子仲。初,仲将以杞姒之子非我为子。曰:``必立伯也,是良材。」子仲怒,弗从,故对曰:``右师则老矣,不识麇也。」公执之。皇瑗奔晋,召之。

\hypertarget{header-n3242}{%
\subsubsection{哀公十八年}\label{header-n3242}}

【传】十八年春,宋杀皇瑗。公闻其情,复皇氏之族,使皇缓为右师。

巴人伐楚,围。初,右司马子国之卜也,观瞻曰:``如志。」故命之。及巴师至,将卜帅。王曰:``宁如志,何卜焉?」使帅师而行。请承,王曰:``寝尹、工尹,勤先君者也。」三月,楚公孙宁、吴由于、薳固败巴师于,故封子国于析。君子曰:``惠王知志。《夏书》曰『官占,唯能蔽志,昆命于元龟。』其是之谓乎!《志》曰:『圣人不烦卜筮。』惠王其有焉!」

夏,卫石圃逐其君起,起奔齐。卫侯辄自齐复归,逐石圃,而复石魋与大叔遗。

\hypertarget{header-n3248}{%
\subsubsection{哀公十九年}\label{header-n3248}}

【传】十九年春,越人侵楚,以误吴也。夏,楚公子庆、公孙宽追越师,至冥,不及,乃还。

秋,楚沈诸梁伐东夷,三夷男女及楚师盟于敖。

冬,叔青如京师,敬王崩故也。

\hypertarget{header-n3254}{%
\subsubsection{哀公二十年}\label{header-n3254}}

【传】二十年春,齐人来征会。夏,会于廪丘。为郑故,谋伐晋。郑人辞诸子侯,秋,师还。

吴公子庆忌骤谏吴子,曰:``不改,必亡。」弗听。出居于艾,遂适楚。闻越将伐吴,冬,请归平越,遂归。欲除不忠者以说于越,吴人杀之。

十一月,越围吴。赵孟降于丧食。楚隆曰:``三年之丧,亲昵之极也。主又降之,无乃有故乎!」赵孟曰:``黄池之役,先主与吴王有质,曰:『好恶同之。』今越围吴,嗣子不废旧业而敌之,非晋之所能及也,吾是以为降。」楚隆曰:``若使吴王知之,若何?」赵孟曰:``可乎?」隆曰:``请尝之。」乃往。先造于越军,曰:``吴犯间上国多矣,闻君亲讨焉,诸夏之人莫不欣喜,唯恐君志之不从。请入视之。」许之。告于吴王曰:``寡君之老无恤,使陪臣隆敢展谢其不共。黄池之役,君之先臣志父得承齐盟,曰:『好恶同之。』今君在难,无恤不敢惮劳。非晋国之所能及也,使陪臣敢展布之。」王拜稽首曰:``寡人不佞,不能事越,以为大夫忧,拜命之辱。」与之一箪珠,使问赵孟,曰:``句践将生忧寡人,寡人死之不得矣。」王曰:``溺人必笑,吾将有问也,史黯何以得为君子?」对曰:``黯也进不见恶,退无谤言。」王曰:``宜哉。」

\hypertarget{header-n3260}{%
\subsubsection{哀公二十一年}\label{header-n3260}}

【传】二十一年夏五月,越人始来。

秋八月,公及齐侯、邾子盟于顾。齐有责稽首,因歌之曰:``鲁人之皋,数年不觉,使我高蹈。唯其儒书。以为二国忧。」

是行也,公先至于阳谷。齐闾丘息曰:``君辱举玉趾,以在寡君之军。群臣将传遽以告寡君,比其复也,君无乃勤。为仆人之未次,请除馆于舟道。」辞曰:``敢勤仆人?」

\hypertarget{header-n3266}{%
\subsubsection{哀公二十二年}\label{header-n3266}}

【传】二十二年夏四月,邾隐公自齐奔越,曰:``吴为无道,执父立子。」越人归之,大子革奔越。

冬十一月丁卯,越灭吴。请使吴王居甬东,辞曰:``孤老矣,焉能事君?」乃缢。越人以归。

\hypertarget{header-n3271}{%
\subsubsection{哀公二十三年}\label{header-n3271}}

【传】二十三年春,宋景曹卒。季康子使冉有吊,且送葬,曰:``敝邑有社稷之事,使肥与有职竞焉,是以不得助执绋,使求从舆人。曰:『以肥人得备弥甥也,有不腆先人之产马,使求荐诸夫人之宰,其可以称旌繁乎?』」

夏六月,晋荀瑶伐齐。高无丕帅师御之。知伯视齐师,马骇,遂驱之,曰:``齐人知余旗,其谓余畏而反也。」乃垒而还。将战,长武子请卜。知伯曰:``君告于天子,而卜之以守龟于宗祧,吉矣,吾又何卜焉?且齐人取我英丘,君命瑶,非敢耀武也,治英丘也。以辞伐罪足矣,何必卜?」

壬辰,战于犁丘。齐师败绩,知伯亲禽颜庚。

秋八月,叔青如越,始使越也。越诸鞅来聘,报叔青也。

\hypertarget{header-n3278}{%
\subsubsection{哀公二十四年}\label{header-n3278}}

【传】二十四年夏四月,晋侯将伐齐,使来乞师,曰:``昔臧文仲以楚师伐齐,取谷。宣叔以晋师伐齐,取汶阳。寡君欲徼福于周公,愿乞灵于臧氏。」臧石帅师会之,取廪丘。军吏令缮,将进。莱章曰:``君卑政暴,往岁克敌,今又胜都。天奉多矣,又焉能进?是躗言也。役将班矣!」晋师乃还。饩臧石牛,大史谢之,曰:``以寡君之在行,牢礼不度,敢展谢之。」

邾子又无道,越人执之以归,而立公子何。何亦无道。

公子荆之母嬖,将以为夫人,使宗人衅夏献其礼。对曰:``无之。」公怒曰:``女为宗司,立夫人,国之大礼也,何故无之?」对曰:``周公及武公娶于薛,孝、惠娶于商,自桓以下娶于齐,此礼也则有。若以妾为夫人,则固无其礼也。」公卒立之,而以荆为大子。国人始恶之。

闰月,公如越,得大子适郢,将妻公,而多与之地。公孙有山使告于季孙,季孙惧,使因大宰嚭而纳赂焉,乃止。

\hypertarget{header-n3285}{%
\subsubsection{哀公二十五年}\label{header-n3285}}

【传】二十五年夏五月庚辰,卫侯出奔宋。卫侯为灵台于藉圃,与诸大夫饮酒焉。褚师声子袜而登席,公怒,辞曰:``臣有疾,异于人。若见之,君将之,是以不敢。」公愈怒,大夫辞之,不可。褚师出,公戟其手,曰:``必断而足。」闻之,褚师与司寇亥乘,曰:``今日幸而后亡。」公之入也,夺南氏邑,而夺司寇亥政。公使侍人纳公文懿子之车于池。

初,卫人翦夏丁氏,以其帑赐彭封弥子。弥子饮公酒,纳夏戊之女,嬖,以为夫人。其弟期,大叔疾之从孙甥也,少畜于公,以为司徒。夫人宠衰,期得罪。公使三匠久。公使优狡盟拳弥,而甚近信之。故褚师比、公孙弥牟、公文要、司寇亥、司徒期因三匠与拳弥以作乱,皆执利兵,无者执斤。使拳弥入于公宫,而自大子疾之宫噪以攻公。鄄子士请御之。弥援其手,曰:``子则勇矣,将若君何?不见先君乎?君何所不逞欲?且君尝在外矣,岂必不反?当今不可,众怒难犯,休而易间也。」乃出。将适蒲,弥曰:``晋无信,不可。」将适鄄,弥曰:``齐、晋争我,不可。」将适泠,弥曰:``鲁不足与,请适城锄以钩越,越有君。」乃适城锄。弥曰:``卫盗不可知也,请速,自我始。」乃载宝以归。

公为支离之卒,因祝史挥以侵卫。卫人病之。懿子知之,见子之,请逐挥。文子曰:``无罪。」懿子曰:``彼好专利而妄。夫见君之入也,将先道焉。若逐之,必出于南门而适君所。夫越新得诸侯,将必请师焉。」挥在朝,使吏遣诸其室。挥出,信,弗内。五日,乃馆诸外里,遂有宠,使如越请师。

六月,公至自越。季康子、孟武伯逆于五梧。郭重仆,见二子,曰:``恶言多矣,君请尽之。」公宴于五梧,武伯为祝,恶郭重,曰:``何肥也!」季孙曰:``请饮彘也。以鲁国之密迩仇雠,臣是以不获从君,克免于大行,又谓重也肥。」公曰:``是食言多矣,能无肥乎?」饮酒不乐,公与大夫始有恶。

\hypertarget{header-n3292}{%
\subsubsection{哀公二十六年}\label{header-n3292}}

【传】二十六年夏五月,叔孙舒帅师会越皋如、后庸、宋乐茷,纳卫侯。文子欲纳之,懿子曰:``君愎而虐,少待之,必毒于民,乃睦于子矣。」师侵外州,大获。出御之,大败。掘褚师定子之墓,焚之于平庄之上。文子使王孙齐私于皋如,曰:``子将大灭卫乎,抑纳君而已乎?」皋如曰:``寡君之命无他,纳卫君而已。」文子致众而问焉,曰:``君以蛮夷伐国,国几亡矣。请纳之。」众曰:``勿纳。」曰:``弥牟亡而有益,请自北门出。」众曰:``勿出。」重赂越人,申开守陴而纳公,公不敢入。师还,立悼公,南氏相之,以城锄与越人。公曰:``期则为此。」令苟有怨于夫人者,报之。司徒期聘于越。公攻而夺之币。期告王,王命取之。期以众取之。公怒,杀期之甥之为大子者。遂卒于越。

宋景公无子,取公孙周之子得与启,畜诸公宫,未有立焉。于是皇缓为右师,皇非我为大司马,皇怀为司徒,灵不缓为左师,乐茷为司城,乐朱锄为大司寇。六卿三族降听政,因大尹以达。大尹常不告,而以其欲称君命以令。国人恶之。司城欲去大尹,左师曰:``纵之,使盈其罪。重而无基,能无敝乎?」

冬十月,公游于空泽。辛巳,卒于连中。大尹兴空泽之士千甲,奉公自空桐入,如沃宫。使召六子,曰:``闻下有师,君请六子画。」六子至,以甲劫之,曰:``君有疾病,请二三子盟。」乃盟于少寝之庭,曰:``无为公室不利。」大尹立启,奉丧殡于大宫。三日,而后国人知之。司城茷使宣言于国曰:``大尹惑蛊其君而专其利,令君无疾而死,死又匿之,是无他矣,大尹之罪也。」得梦启北首而寝于卢门之外,己为鸟而集于其上,咮加于南门,尾加于桐门。曰:``余梦美,必立。」大尹谋曰:``我不在盟,无乃逐我,复盟之乎?」使祝为载书,六子在唐盂。将盟之。祝襄以载书告皇非我,皇非我因子潞、门尹得、左师谋曰:``民与我,逐之乎?」皆归授甲,使徇于国曰:``大尹惑蛊其君,以陵虐公室。与我者,救君者也。」众曰:``与之。」大尹徇曰:``戴氏、皇氏将不利公室,与我者,无忧不富。」众曰:``无别。」戴氏、皇氏欲伐公,乐得曰:``不可。彼以陵公有罪,我伐公,则甚焉。」使国人施于大尹,大尹奉启以奔楚,乃立得。司城为上卿,盟曰:``三族共政,无相害也。」

卫出公自城锄使以弓问子赣,且曰:``吾其入乎?」子赣稽首受弓,对曰:``臣不识也。」私于使者曰:``昔成公孙于陈,宁武子、孙庄子为宛濮之盟而君入。献公孙于卫齐,子鲜、子展为夷仪之盟而君入。今君再在孙矣,不闻献之亲,外不闻成之卿,则赐不识所由入也。《诗》曰:『无竞惟人,四方其顺之。』若得其人,四方以为主,而国于何有?」

\hypertarget{header-n3299}{%
\subsubsection{哀公二十七年}\label{header-n3299}}

【传】二十七年春,越子使后庸来聘,且言邾田,封于骀上。

二月,盟于平阳,三子皆从。康子病之,言及子赣,曰:``若在此,吾不及此夫!」武伯曰:``然。何不召?」曰:``固将召之。」文子曰:``他日请念。」

夏四月己亥,季康子卒。公吊焉,降礼。

晋荀瑶帅师伐郑,次于桐丘。郑驷弘请救于齐。齐师将兴,陈成子属孤子三日朝。设乘车两马,系五色焉。召颜涿聚之子晋,曰:``隰之役,而父死焉。以国之多难,未女恤也。今君命女以是邑也,服车而朝,毋废前劳。」乃救郑。及留舒,违谷七里,谷人不知。乃濮,雨,不涉。子思曰:``大国在敝邑之宇下,是以告急。今师不行,恐无及也。」成子衣制,杖戈,立于阪上,马不出者,助之鞭之。知伯闻之,乃还,曰:``我卜伐郑,不卜敌齐。」使谓成子曰:``大夫陈子,陈之自出。陈之不祀,郑之罪也。故寡君使瑶察陈衷焉。谓大夫其恤陈乎?若利本之颠,瑶何有焉?」成子怒曰:``多陵人者皆不在,知伯其能久乎?」中行文子告成子曰:``有自晋师告寅者,将为轻车千乘,以厌齐师之门,则可尽也。」成子曰:``寡君命恒曰:『无及寡,无畏众。』虽过千乘,敢辟之乎?将以子之命告寡君。」文子曰:``吾乃今知所以亡。君子之谋也,始衷终皆举之,而后入焉。今我三不知而入之,不亦难乎?」

公患三桓之侈也,欲以诸侯去之。三桓亦患公之妄也,故君臣多间。公游于陵阪,遇孟武伯于孟氏之衢,曰:``请有问于子,余及死乎?」对曰:``臣无由知之。」三问,卒辞不对。公欲以越伐鲁,而去三桓。秋八月甲戌,公如公孙有陉氏,因孙于邾,乃遂如越。国人施公孙有山氏。

悼之四年,晋荀瑶帅师围郑。未至,郑驷弘曰:``知伯愎而好胜,早下之,则可行也。」乃先保南里以待之。知伯入南里,门于桔柣之门。郑人俘酅魁垒,赂之以知政,闭其口而死。将门,知伯谓赵孟:``入之。」对曰:``主在此。」知伯曰:``恶而无勇,何以为子?」对曰:``以能忍耻,庶无害赵宗乎!」知怕不悛,赵襄子由是惎知伯,遂丧之。知伯贪而愎,故韩、魏反而丧之。

\end{document}
